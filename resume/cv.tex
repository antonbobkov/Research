%%%%%%%%%%%%%%%%%%%%%%%%%%%%%%%%%%%%%%%%%
% Medium Length Graduate Curriculum Vitae
% LaTeX Template
% Version 1.1 (9/12/12)
%
% This template has been downloaded from:
% http://www.LaTeXTemplates.com
%
% Original author:
% Rensselaer Polytechnic Institute (http://www.rpi.edu/dept/arc/training/latex/resumes/)
%
% Important note:
% This template requires the res.cls file to be in the same directory as the
% .tex file. The res.cls file provides the resume style used for structuring the
% document.
%
%%%%%%%%%%%%%%%%%%%%%%%%%%%%%%%%%%%%%%%%%

%----------------------------------------------------------------------------------------
%	PACKAGES AND OTHER DOCUMENT CONFIGURATIONS
%----------------------------------------------------------------------------------------

\documentclass[margin, 10pt]{res} % Use the res.cls style, the font size can be changed to 11pt or 12pt here
\usepackage[T1]{fontenc}
\usepackage{hyperref}
\usepackage{verbatim}
\usepackage{enumerate}
\usepackage{enumitem}
\usepackage{amssymb}
\usepackage{helvet} % Default font is the helvetica postscript font
%\usepackage{newcent} % To change the default font to the new century schoolbook postscript font uncomment this line and comment the one above

%\usepackage{graphicx}

%\setlength{\textwidth}{5.1in} % Text width of the document
\setlength{\textwidth}{5.5in} % Text width of the document

\hypersetup{
     colorlinks   = true,
     urlcolor  = blue
}

\begin{document}

%----------------------------------------------------------------------------------------
%	NAME AND ADDRESS SECTION
%----------------------------------------------------------------------------------------

\moveleft.5\hoffset\centerline{\large\bf Anton Bobkov}
%\\ % Your name at the top{Control+Shift+F7}{Control+Shift+F5}
\moveleft\hoffset\vbox{\hrule width\resumewidth height 1pt}\smallskip % Horizontal line after name; adjust line thickness by changing the '1pt'
\section{\textsc{Contact Information}}
\begin{tabular}{l|l}
Graduate Student & {\it E-mail:}\\
Department of Mathematics & \href{mailto:antongml@gmail.com}{antongml@gmail.com}\\
University of California, Los Angeles & \href{mailto:bobkov@math.ucla.edu}{bobkov@math.ucla.edu}\\
Los Angeles, CA 90095-1555 USA & {\it Website:}\\
\phantom{lots of text that takes up a bit of space blah blah blah} & \url{www.math.ucla.edu/~bobkov/}\\
\phantom{lots of text that takes up a bit of space blah blah blah} & {\it Phone:} (408)813-6331
\end{tabular}


%----------------------------------------------------------------------------------------

\begin{resume}

%----------------------------------------------------------------------------------------
%	EDUCATION SECTION
%----------------------------------------------------------------------------------------

\section{\textsc{Education}}

\textbf{University of California, Los Angeles} {\sl (graduate)}\hfill \textbf{Fall 2011 to present}
{\sl PhD,} Mathematics (in progress)

\begin{itemize}
	\item GPA: 3.91
\end{itemize}

{\sl Advisor}: Matthias Aschenbrenner \\
{\sl Research interests}: Mathematical logic, model theory, NIP theories, VC-density

\textbf{University of California, Los Angeles} {\sl (undergraduate)}\hfill \textbf{Graduated Spring 2011}

\begin{itemize}
	\item {\sl B.S.} in Mathematics, {\sl B.A.} in Physics
	\item Sherwood Prize
	\item Departmental Highest Honors in Mathematics, College Honors
	\item GPA: 3.82 (Magna Cum Laude)
	\item William Lowell Putnam Mathematics Competition
		\begin{itemize}
			\item 2008 - score 30
			\item 2009 - score 19
		\end{itemize}
\end{itemize}


%----------------------------------------------------------------------------------------
%	Undergraduate Research
%----------------------------------------------------------------------------------------
 
\section{\textsc{Undergraduate Research}}

\textbf{Cryptography REU at Northern Kentucky University} \hfill \textbf{Summer 2009}\\
Implemented a variant of MXL algorithm in computational algebra system MAGMA \\

\textbf{Research assistant for Vladimir Vassiliev} \hfill \textbf{2008 - 2011}\\
Numerical simulations for AGIS gamma-ray telescope. This included forward and inverse kinematics for Stewart platform, ray casting, and high precision calibration.
% I have also worked on network interfacing with Gumstix boards using CORBA as well as installing and configuring a custom linux kernel. 

%----------------------------------------------------------------------------------------
%	TEACHING
%----------------------------------------------------------------------------------------
 
\section{\textsc{Teaching}}

% Intermediate C++ Programming, Linear Algebra, Calculus

%\begin{itemize}
	Math 31B: Integration and Infinite Series \hfill \textbf{2012 - 2013} \\
	Math 33A: Linear Algebra and Applications \hfill \textbf{2012 - 2013} \\
	PIC 10B: Intermediate Programming \hfill \textbf{Winter 2015,  Spring 2015} \\
	PIC 20A: Principles of Java Language with Applications \hfill \textbf{Spring 2015} \\
	PIC 40A: Introduction to Programming for Internet \hfill \textbf{Fall 2015} \\
	Math 115B: Linear Algebra \hfill \textbf{Winter 2016} \\
	Independent Programming Projects \hfill \textbf{Winter 2016}
%\end{itemize}


%----------------------------------------------------------------------------------------
%	PAPERS
%----------------------------------------------------------------------------------------
 
\section{\textsc{Papers}}

Bobkov, A. {\it VC-density for trees}, in preparation  \\
Bobkov, A. {\it Some VC-density computations for Shelah-Spencer graphs}, in preparation \\
Bobkov, A. {\it VC-density in $\mathbb{Q}_p$-reducts}, in preparation

\section{\textsc{Awards and Scholarships}}

\textbf{2011-2012:} \\

Fees: \$13,247.13 paid from unrestricted Graduate Division allocation \\
Stipend: \$21,000 paid from departmental funds (RTG Logic) \\
Stipend: \$4,000 paid from unrestricted Graduate Division allocation \\
Summer'11 Stipend: \$ 3000 from departmental funds (RTG Logic) \\
Summer'11 Stipend:\$3,000 from College funds \\

\textbf{2012-2013:} \\

Fees: \$14,372.68 paid through TA appointments \\
Fee remission balance: \$374.25 paid from unrestricted Graduate Division allocation  \\
Income: \$17,655.18 TA salary \\
Stipend: \$3,344.82 paid from unrestricted Graduate Division allocation \\				
  
\textbf{2013-2014:} \\

Fees: \$2,070.75 partial fees paid from unrestricted Graduate Division allocation \\
Fees: \$10,500 remainder of fees paid from departmental funds (RTG Logic) \\
Stipend:  \$ 21,000 paid from departmental funds (RTG Logic) \\
Stipend: \$4,000 paid from unrestricted Graduate Division allocation \\
Summer'13: \$4,000 paid from departmental funds (RTG Algebra) \\

\textbf{2014-2015:} \\

Fees: \$15,203.10 paid through TA \& GSR appointment \\
Fee remission balance: \$378.99 paid from unrestricted Graduate Division allocation  \\
Income: \$20,621.16 TA \& GSR salary \\
Stipend: \$378.84 paid from unrestricted Graduate Division allocation \\
  
  
\textbf{2015-2016:} \\

Fees: \$15,440.48 paid through TA appointments \\
TA Income: \$11,121.43 (Spring TA salary yet to be paid during Spring'16) \\
Summer'15: \$4,000 paid from unrestricted Graduate Division allocation \\




\end{resume}

\end{document}


%----------------------------------------------------------------------------------------
%	Software
%----------------------------------------------------------------------------------------
 
\section{\textsc{Software Experience}}

\textbf{Unix-like systems}\\
I am comfortable working in command line environment, including tasks such as
\begin{itemize}
	\item installing and managing web-server, repository server, ssh server
	\item code building, editing, and version control
\end{itemize}

\begin{tabular}{ll}
\textbf{Languages} & C++, C\#, bash, Java, PHP, MAGMA \\
\textbf{Code management} & CMake, Makefile, git, subversion, Visual Studio, Unity3D \\
\textbf{Standards} & TCP/IP, .NET, CORBA  \\
\end{tabular}

Python, HTML, CSS, JavaScript, PHP, SQL

%----------------------------------------------------------------------------------------
%	Projects
%----------------------------------------------------------------------------------------
 
\section{\textsc{Independent Projects}}
For more information and links visit \url{www.math.ucla.edu/~bobkov/projects.html}

\textbf{Burn and Turn} \hfill \textbf{2008 - 2011}\\
Cross-platform arcade style video game featured on \href{http://kotaku.com/5862197/burn-and-turn-combines-retro-arcade-stylings-with-tower-defense-for-combustible-fun}{Kotaku} and \href{http://indiegames.com/2011/10/trailer_burn_turn_robot_bear.html}{IndieGames}. It was coded in C++ and used OpenGL as a backend for graphics. It was created by a team of three people over a course of four years and released on iOS and Android markets.

\textbf{Self Balancing Robot} \hfill \textbf{Summer 2012}\\
A vertical self-balancing robot ran by an arduino controller coded in C++. A numerical simulation was used to determine weight distribution. Robot's position is determined by data from an accelerometer and a gyroscope combined through a Kalman filter. Balancing is done with a DC motor using PID controller.

\textbf{UCLA Graduate Student Wiki} \hfill \textbf{Summer 2014}\\
Official wiki for graduate math department at UCLA that maintains a database of qualifying exam problems. It is made on top of Semantic Media Wiki using custom extension written in PHP that allows to users to search, filter, and tag the solutions.

\textbf{Decentralized Online Game} \hfill \textbf{Fall 2014 - Present}\\
Exploration multiplayer online game that manages players and game data using peer-to-peer connections instead of relying on a central server. It is coded with C\# in Unity3D using standard TCP/IP network.

\section{\textsc{Sample Code}}
\url{https://gitorious.org/~antonbobkov}


