\documentclass{beamer}

\usepackage{../Research}

\newcommand{\F}{\mathcal F}
\newcommand{\curly}[1]{\left\{ #1 \right\}}


\title{A Tiny Example}
\author{Andrew Mertz and William Slough}
\date{June 15, 2005}


\begin{document}

\maketitle

\begin{frame}
	Suppose we have an infinite collection of sets $\F$.
	Take $n$ many of those sets.
	They generate a boolean algebra.
	Count the number of atoms in it.
	There can be at most $2^n$ atoms, though depending on the collection there may be much less.
	For a given $n$, out of all choices of $n$ sets, record the highest possible number of atoms generated.
	We define that to be a shatter function.
\end{frame}

\begin{frame}
	\frametitle{Second Slide}
	Contents of the second slide
\end{frame}

\end{document}