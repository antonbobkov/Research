\documentclass{amsart}

\usepackage{../AMC_style}	
\usepackage{../Research}

\newcommand{\F}{\mathcal F}
\newcommand{\curly}[1]{\left\{ #1 \right\}}

\DeclareMathOperator{\cx}{Complexity}

\begin{document}

\section{VC-dimension}

Suppose we have an infinite collection of sets $\F$.
Take $n$ many of those sets.
They generate a boolean algebra.
Count the number of atoms in it.
There can be at most $2^n$ atoms, though depending on the collection there may be much less.
For a given $n$, out of all choices of $n$ sets, record the highest possible number of atoms generated.
We define that to be a shatter function.

\begin{Definition}
	\begin{align*}
		\pi_\F(n) = \max \curly{ \text {\# of atoms in boolean algebra generated by $S$} \mid S \subset \F \text{ and } |S| = n}
	\end{align*}
\end{Definition}

\begin{Example}
	\begin{enumerate}
		\item Let $\F$ be a set of lines on a plane. Then
		\begin{align*}
			\pi_\F(0) &= 1 \\
			\pi_\F(1) &= 2 \\
			\pi_\F(2) &= 4 \\
			\pi_\F(3) &= 7 \\
			\pi_\F(4) &= 11 \\
			\pi_\F(n) &= n(n+1)/2 + 1
		\end{align*}
		\item Let $\F$ be a set of disks on a plane. Then
		\begin{align*}
			\pi_\F(0) &= 1 \\
			\pi_\F(1) &= 2 \\
			\pi_\F(2) &= 4 \\
			\pi_\F(3) &= 8
		\end{align*}
		\item Let $\F$ be a set of intervals on a line. Then
		\begin{align*}
			\pi_\F(0) &= 1 \\
			\pi_\F(1) &= 2 \\
			\pi_\F(2) &= 4 \\
			\pi_\F(3) &= 7
		\end{align*}
	\end{enumerate}
\end{Example}

\end{document}