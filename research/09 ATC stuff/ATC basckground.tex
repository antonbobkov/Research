\documentclass{amsart}

\usepackage{../AMC_style}	
\usepackage{../Research}

\newcommand{\F}{\mathcal F}
\newcommand{\curly}[1]{\left\{ #1 \right\}}

\DeclareMathOperator{\cx}{Complexity}

\begin{document}

\section{VC-dimension}

Suppose we have an infinite collection of sets $\F$.
Take $n$ many of those sets.
They generate a boolean algebra.
Count the number of atoms in it.
There can be at most $2^n$ atoms, though depending on the collection there may be much less.
For a given $n$, out of all choices of $n$ sets, record the highest possible number of atoms generated.
We define that to be a shatter function.

\begin{Definition}
	\begin{align*}
		\pi_\F(n) = \max \curly{ \text {\# of atoms in boolean algebra generated by $S$} \mid S \subset \F \text{ and } |S| = n}
	\end{align*}
\end{Definition}

\begin{Example}
	\begin{enumerate}
		\item Let $\F$ be a set of lines on a plane. Then
		\begin{align*}
			\pi_\F(n) &= n(n+1)/2 + 1
		\end{align*}
		\item Let $\F$ be a set of disks on a plane. Then
		\begin{align*}
			\pi_\F(n) &= n^2 - n + 2
		\end{align*}
		\item Let $\F$ be a set of balls in $\R^3$. Then
		\begin{align*}
			\pi_\F(n) &= n(n^2 - 3n + 8)/3
		\end{align*}
		\item Let $\F$ be a set of intervals on a line. Then
		\begin{align*}
			\pi_\F(n) &= 2n
		\end{align*}
		\item Let $\F$ be a set of half-planes. Then
		\begin{align*}
			\pi_\F(n) &= n(n+1)/2 + 1
		\end{align*}
		\item Let $\F$ be a collection of finite subsets of $\N$. Then
		\begin{align*}
			\pi_\F(n) &= 2^n
		\end{align*}
		\item Let $\F$ be a collection of polygons in a plane. Then
		\begin{align*}
			\pi_\F(n) &= 2^n
		\end{align*}
	\end{enumerate}
\end{Example}

\begin{Theorem} [Sauer-Shelah]
	Shatter function is either $2^n$ or bounded by a polynomial.
\end{Theorem}

\begin{Definition}
	Families of sets with polynomially bounded shatter functions are said to have a finite VC-dimension.
\end{Definition}

\begin{Definition}
	Suppose $\F$ has a finite VC-dimension.
	Let $k$ be the smallest real such that 
	\begin{align*}
		\pi_\F(n) = O(n^k)
	\end{align*}
	We define such $k$ to be the vc-density of $\F$.
\end{Definition}

\section{Model Theory}

Consider a structure with a language
\begin{align*}
	(\R, 0, 1, +, \cdot, \leq)
\end{align*}

We work with subsets of the underlying set definable by first-order formulas.
Those are called definable sets.

\begin{align*}
	\phi(x) &= 5 \leq x \leq 7.7 \vee x \leq 0\\
	\psi(x) &= \exists y \ y \cdot y = x \\
	\gamma(x) &= x \cdot x \cdot x \cdot x = 2 \\
\end{align*}

$\phi(x)$ defines the set $[5, 7.7] \cup (-\infty, 0]$ in the structure above.
$\psi(x)$ defines the set $[0, \infty)$ in the structure above.

\begin{enumerate}
	\item in $(\Q, \cdot)$ $\gamma(x)$ defines an empty subset
	\item in $(\R, \cdot)$ $\gamma(x)$ defines a subset with two elements
	\item in $(\C, \cdot)$ $\gamma(x)$ defines a subset with four elements
	\item in $(\mathbb H, \cdot)$ $\gamma(x)$ defines an infinite subset
\end{enumerate}

\begin{align*}
	\theta(x) = \forall y \exists z \ x \leq z \leq y
\end{align*}

\begin{enumerate}
	\item in $(\Q, \leq)$ $\theta(x)$ defines an empty subset
	\item in $(\N, \leq)$ $\theta(x)$ defines an empty subset
	\item in $(\Q^{\geq 0}, \leq)$ $\theta(x)$ defines the set $\{0\}$
\end{enumerate}

\begin{Definition}
	for a formula $\phi(x_1 \ldots x_n, y_1, \ldots y_n)$ we can plug in elements of our structure as parameters in places of $y$ variables. This gives us a collection of definable sets. 
\end{Definition}

\begin{Example}
	\begin{align*}
		\phi(x_1, x_2, y_1, y_2, y_3) = (x_1 - y_1)^2 + (x_2 - y_2)^2 \leq y_3^2
	\end{align*}
\end{Example}

In structure $(\R, +, \cdot, \leq)$ given $a,b,r \in \R$ the formula $\phi(x_1, x_2, a, b, r)$ defines a disk in $\R^2$ with radius $r$ with center $(a,b)$.

Thus all discs in $\R^2$ are defined uniformly by $\phi$.

What are the collection of sets we can consider when working with a model?

We can look at all definable subsets. That's not interesting, always has an infinite VC-dimension.
Uniformly definable 

\end{document}