%----------------------------------------------------------------------------------------
%	PACKAGES AND OTHER DOCUMENT CONFIGURATIONS
%----------------------------------------------------------------------------------------

\documentclass[final]{beamer}

%\usepackage[orientation=landscape,size=a1,scale=0.65,debug]{beamerposter} % Use the beamerposter package for laying out the poster
\usepackage[scale=0.9]{beamerposter} % Use the beamerposter package for laying out the poster

\usepackage{../Research}

\newcommand{\F}{\mathcal F}
\newcommand{\curly}[1]{\left\{ #1 \right\}}
\newcommand{\paren}[1]{\left( #1 \right)}
\newcommand{\defn}{\underline}

\DeclareMathOperator{\Sg}{Sg}
\DeclareMathOperator{\Bt}{It}
\DeclareMathOperator{\Pt}{Pt}
\DeclareMathOperator{\Ct}{Ct}
\DeclareMathOperator{\vecspan}{span}
\DeclareMathOperator{\vval}{val}
\DeclareMathOperator{\tval}{T-val}


\usetheme{confposter} % Use the confposter theme supplied with this template

\setbeamercolor{block title}{fg=ngreen,bg=white} % Colors of the block titles
\setbeamercolor{block body}{fg=black,bg=white} % Colors of the body of blocks
\setbeamercolor{block alerted title}{fg=white,bg=dblue!70} % Colors of the highlighted block titles
\setbeamercolor{block alerted body}{fg=black,bg=dblue!10} % Colors of the body of highlighted blocks
% Many more colors are available for use in beamerthemeconfposter.sty

%-----------------------------------------------------------
% Define the column widths and overall poster size
% To set effective sepwid, onecolwid and twocolwid values, first choose how many columns you want and how much separation you want between columns
% In this template, the separation width chosen is 0.024 of the paper width and a 4-column layout
% onecolwid should therefore be (1-(# of columns+1)*sepwid)/# of columns e.g. (1-(4+1)*0.024)/4 = 0.22
% Set twocolwid to be (2*onecolwid)+sepwid = 0.464
% Set threecolwid to be (3*onecolwid)+2*sepwid = 0.708

\newlength{\sepwid}
\newlength{\onecolwid}
%\setlength{\paperwidth}{54in} % A0 width: 46.8in
\setlength{\paperwidth}{48in} % A0 width: 46.8in
\setlength{\paperheight}{32in} % A0 height: 33.1in
%\setlength{\paperheight}{36in} % A0 height: 33.1in
%% \setlength{\paperwidth}{36in} % A0 width: 46.8in
%% \setlength{\paperheight}{24in} % A0 height: 33.1in

%\setlength{\sepwid}{0in} % Separation width (white space) between columns
\setlength{\sepwid}{0.024\paperwidth} % Separation width (white space) between columns
\setlength{\onecolwid}{0.3\paperwidth} % Width of one column
%% \setlength{\sepwid}{0.024\paperwidth} % Separation width (white space) between columns
%% \setlength{\onecolwid}{0.22\paperwidth} % Width of one column

\setlength{\topmargin}{-0.5in} % Reduce the top margin size
%-----------------------------------------------------------

\usepackage{graphicx}  % Required for including images

\usepackage{booktabs} % Top and bottom rules for tables

%----------------------------------------------------------------------------------------
%	TITLE SECTION 
%----------------------------------------------------------------------------------------

\title{VC-density in an additive reduct of $p$-adic numbers}
\author{Anton Bobkov}
\institute{UCLA}

%----------------------------------------------------------------------------------------

\begin{document}

\addtobeamertemplate{alertblock end}{}{\vspace*{2ex}} % White space under alertblocks
\addtobeamertemplate{alertblock alerted end}{}{\vspace*{2ex}} % White space under highlighted (alert) alertblocks

\setlength{\belowcaptionskip}{2ex} % White space under figures
\setlength{\belowdisplayshortskip}{2ex} % White space under equations

\begin{frame}[t] % The whole poster is enclosed in one beamer frame

\begin{columns}[t] % The whole poster consists of three major columns, the second of which is split into two columns twice - the [t] option aligns each column's content to the top

\begin{column}{\sepwid}\end{column} % Empty spacer column

\begin{column}{\onecolwid} % The first column

    \begin{block}{Abstract}
            Aschenbrenner et. al. computed a bound $\vc(n) \leq 2n - 1$ for the vc-density function in the field of $p$-adic numbers,
            but it is not known to be optimal.
            I investigate a certain $P$-minimal additive reduct of the field of $p$-adic numbers and
            compute an optimal bound $\vc(n) = n$ for it using a cell decomposition result of Leenknegt.
    \end{block}
    \begin{block}{VC-density}
	Let $M$ be a structure and $\Psi(x; y) = \curly{\phi_i(x; y)}$ a finite collection of formulas in $L(M)$.
        Define the \defn{shatter function} $\pi^M_\Psi \colon \N \arr \N$ of $\Psi$ as
        \begin{align*}
            \pi^M_\Psi(n) = \max \{ \text {number of $\Psi$-types over $B$ }
            \mid B \subset M^{|y|} \text{ with } |B| = n\}.
        \end{align*}
        Define the shatter function of a single formula $\phi$ as the shatter function of a one element collection $\curly{\phi}$.
        The shatter function only depends on the theory of $M$.
        The following theorem is an important result concerning a dichotomy for the growth of the shatter function.
        \begin{alertblock}{Theorem (Sauer-Shelah '72)}
            The shatter function either grows exponentially or is bounded by a polynomial.
        \end{alertblock}
        In fact, a formula $\phi(x; y)$ is NIP precisely when its shatter function grows polynomially.
        From now on work with NIP theories, that is all formulas will have shatter functions that grow polynomially.
        The following definition captures the degree of polynomial growth.

        For a collection of formulas $\Psi(x; y)$ in a model $M$ let $\vc^M(\Psi)$ be the infimum of all positive reals $r$ such that
            \begin{align*}
                \pi^M_\Psi(n) = O(n^r)
            \end{align*}
            Call $\vc^M(\Psi)$ the \defn{vc-density} of $\Psi$.
        As before for a single formula $\phi$ define $\vc(\phi)$ as the VC-density of a one element collection $\curly{\phi}$.
        This allows formula by formula analysis of the growth rate for the shatter function.
        More generally, we look at the bounds of VC-density for all the formulas in a given structure.

        Define the \defn{vc-function} $\vc^M \colon \N \arr \N$ to be the largest $\vc$-density achieved by formulas
        that define subsets of $M^n$.
            \begin{align*}
                \vc^M(n) = \sup \curly{ \vc^M(\phi) \mid \phi(x, y) \text{ with } |x| = n}
            \end{align*}        

        As before this only depends on the theory of $M$.
        There is a simple lower bound $\vc^M(n) \geq n$.
        More generally $\vc^M(n) \geq n\vc^M(1)$, and it is not known whether strict inequality can hold.
    \end{block}

    \begin{block}{Application to $p$-adic numbers}
        A common example of a non-stable NIP structure is the field $\Q_p$ of $p$-adic numbers.
	In \cite{density}, Aschenbrenner, Dolich, Haskell, Macpherson, and Starchenko show that this structure has $\vc(n) \leq 2n - 1$.
        My work improves that bound in a reduct of the full structure.
        In \cite{reduct}, Leenknegt analyzes the reduct of $p$-adic numbers to the language
        \begin{align*}
            \LL_{aff}  = \curly{+, -, \curly{\bar c}_{c \in \Q_p}, |, \curly{Q_{m,n}}_{m,n\in \N}}
        \end{align*}
        where $\bar c$ is a scalar multiplication by $c$,
        $a | b$ stands for $\vval a \leq \vval b$,
        and $Q_{m,n}$ is a unary predicate
        \begin{align*}
            Q_{m,n} = \bigcup_{k \in \Z} p^{km} (1 + p^n\Z_p).
        \end{align*}
	Note that $Q_{m,n}$ is a subgroup of the multiplicative group of $\Q_p$ with finitely many cosets.
        One can check that the extra relation symbols are definable in the full structure.
        Moreover, \cite{reduct} shows that $(\Q_p, \LL_{aff})$ is a $P$-minimal reduct,
        that is one-dimensional definable sets coincide with one-dimensional definable sets in the full structure.
        \begin{alertblock} {Theorem (B.)}
            $(\Q_p, \LL_{aff})$ has $\vc(n) = n$.
        \end{alertblock}
    \end{block}
    
\end{column} % End of the first column

\begin{column}{\sepwid}\end{column} % Empty spacer column

\begin{column}{\onecolwid}
     
    %% \cite{reduct} provides the following cell decomposition result
    %% \begin{Theorem}
    %%     Any formula $\phi(t, x)$  with $t$  singleton decomposes into the union of the following cells:
    %%     \begin{align*}
    %%             \curly{(t, x) \in K \times D \mid \vval a_1(x) \square_1 \vval (t - c(x)) \square_2 \vval a_2(x), t - c(x) \in \lambda Q_{m,n} }
    %%     \end{align*}
    %%     where $D$ is a cell of a smaller dimension,
    %%     $a_1, a_2, c$ are linear polynomials in  $x$,
    %%     $\square$ is $<$ or no condition,
    %%     $\lambda  \in\Q_p$.    
    %% \end{Theorem}
    \begin{block}{Proof outline}
        There is a cell decomposition result in \cite{reduct} that can be used to eliminate quantifiers
        \begin{alertblock} {Theorem}
            Any formula $\phi(x; y)$ in $(\Q_p, \LL_{aff})$ can be written as a boolean combination of formulas from the following collection
            \begin{align*}
                    \Psi(x; y) = &\curly{\vval (p_i(x) - c_i(y)) < \vval (p_j(x) - c_j(y))}_{i, j \in I} \cup \\
                    &\curly{p_i(x) - c_i(y) \in \lambda_k Q_{m,n}}_{i \in I , k \in K}
            \end{align*}
            where $I, K$ are finite index sets,
            each $p_i$ is a linear polynomial in $x$ without a constant term,
            each $c_i$ is a linear polynomial in $y$, and
            $\lambda_k \in \Q_p$.
        \end{alertblock}

        It is easy to show that $\vc(\phi) \leq \vc(\Psi)$.
        Therefore to show that $\vc(n) = n$ it suffices to bound $\vc(\Psi) \leq |x|$ for any such collection.
        More precisely, it is sufficient to show that if there is a parameter set $B$ of size $N$
        then the number of $\Psi$-types over $B$ is $O(N^{|x|})$.
                Fix a parameter set $B$ of size $N$.
                Consider a set $T = \curly{c_i(b) \mid b \in B, i \in I} \subset \Q_p$.
                View $T$ as a tree as follows.
                Branches through the tree are elements of $T$.
        For $c \in \Q_p, \alpha \in \Z$  define a \defn{ball} 
                        $B(c, \alpha) = \curly{c' \in \Q_p \mid \vval \paren{c' - c} \leq \alpha}$.
                With $T$ we associate the balls $B(t_1, \vval(t_1 - t_2))$ for all $t_1, t_2 \in T$.
                An \defn{interval} is two balls $B(t_1, v_1) \supset B(t_2, v_2)$ with no balls in between.
                An element $a \in \Q_p$ belongs to this interval if $a \in B(t_1, v_1) \backslash B(t_2, v_2)$.
                There are at most $2|T| = 2 N |I| = O(N)$ different intervals and they partition $\Q_p$.
                (See Figure 1).

                Suppose $a \in \Q_p$ lies in an interval $B(t_L, \alpha_L) \backslash B(t_U, \alpha_U)$.
                Define \defn{T-valuation} of $a$ to be $\tval(a) = \vval(a - t_U)$.    
                Define \defn{floor} of $a$ to be $F(a) = \alpha_L$.

            Suppose $a_1, a_2 \in \Q_p$ lie in our tree in the same interval $B(t_L, \alpha_L) \backslash B(t_U, \alpha_U)$.
            We say that $a_i$ is \defn{close to boundary} if $|\tval(a_i) - \alpha_L| \leq m$ or $|\tval(a_i) - \alpha_U| \leq m$.
            Otherwise we say that it is \defn{far from boundary}.
            We say $a_1, a_2$ have the same \defn{interval type} if one of the following holds (see Figure 2):
            \begin{itemize}
                \item Both $a_1, a_2$ are far from boundary and $a_1 - t_U, a_2 - t_U$ are in the same $Q_{m,n}$ coset.
                \item Both $a_1, a_2$ are close to boundary and $\vval(a_1 - a_2) > \tval(a_1) + n = \tval(a_2) + n$.
            \end{itemize}      

        One can check that for each interval there are at most $K = K(\Psi, Q_{m,n})$ many interval types (with $K$ not dependent on $B$ or the interval).

        \begin{alertblock}{Lemma}
                Suppose $c_1, c_2 \in \Q_p^{|x|}$ satisfy the follwing three conditions 
                \begin{itemize}
                    \item For all $i \in I$ $p_i(c_1)$ and $p_i(c_2)$ are in the same interval.
                    \item For all $i \in I$ $p_i(c_1)$ and $p_i(c_2)$ have the same interval type.
                    \item For all $i,j \in I$, $\tval(p_i(c_1)) > \tval(p_j(c_1))$ iff $\tval(p_i(c_2)) > \tval(p_j(c_2))$.
                \end{itemize}
                Then $c_1, c_2$ have the same $\Psi$-type over $B$.
        \end{alertblock}
    \end{block}    
\end{column} % End of column 2.1

\begin{column}{\sepwid}\end{column} % Empty spacer column

\begin{column}{\onecolwid}

    This gives us an upper bound on the number of types - there are at most $|I|!$ many choices for the order of $\tval$,
    $O(N)$ many choices for the interval for each $p_i$,
    and $K$ many choices for the interval type for each $p_i$,
    giving a total of $O(N^{|I|}) \cdot K^{|I|} \cdot |I|! = O(N^{|I|})$ many types (see Figure 3).
    This implies $\vc(\Psi) \leq |I|$.
    The biggest contribution to this bound are the choices among the $O(N)$ many intervals for each $p_i$ with $i \in I$.
    Are all of those choices realized?
    Intuitively there are $|x|$ many variables and $|I|$ many equations,
    so once we choose an interval for $|x|$ many $p_i$'s, the interval for the rest should be determined.
    This would give the required $\vc(\Psi) \leq |x|$ bound.
    The remainder of the poster is a more formal outline of this idea.

    \begin{block}{Reduction from $|I|$ to $|x|$}
        For $c \in \Q_p$ and $\alpha, \beta \in \Z$ define $c \midr [\alpha, \beta] \in \paren{\Z/p\Z}^{\beta - \alpha + 1}$
        to be the record of the coefficients of $c$ for the valuations between $\alpha, \beta$.
	More precisely write $c$ in its power series form
	\begin{align*}
		c = \sum_{\gamma \in \Z} c_\gamma p^\gamma \text{ with } c_\gamma \in \Z/p\Z
	\end{align*}
	Then $c \midr [\alpha, \beta]$ is just $(c_\alpha, c_{\alpha+1}, \ldots c_\beta)$.
        
        Alternative way to write $p_i(x)$ is $\vec p_i \cdot \vec x$,
        where $\vec p_i$ and $\vec x$ are vectors in $\Q_p^{|x|}$.
        \begin{alertblock}{Lemma}
            Suppose we have a finite collection of vectors $\curly{\vec p_i}_{i \in I}$ with each $\vec p_i \in \Q_p^{|x|}$.
            Suppose $J \subset I$ and $j \in I$ satisfy
                    $\vec p_j \in \vecspan \curly{\vec p_i}_{i \in J} $.
            and we have $\vec x \in \Q_p^{|x|}, \alpha \in \Z$ with
                    $\vval(\vec p_i \cdot \vec x) > \alpha \text{ for all } i \in J$.
            Then
                    $\vval(\vec p_j \cdot \vec x) > \alpha - \gamma$
            for some $\gamma \in \N$.
            Moreover $\gamma$ can be chosen independently from $J, j, \vec x, \alpha$ depending only on $\curly{\vec p_i}_{i \in I}$.
        \end{alertblock}          
        Let $f: \Q_p^{|x|} \arr \Q_p^I$ with $f(c) = (p_i(c))_{i \in I}$.
        Define the segment space $\Sg$ to be the image of $f$.
        Given a tuple $(a_i)_{i\in I}$ in the segment space,
        look at the corresponding floors $\curly{F(a_i)}_{i\in I}$ and T-valuations $\curly{\tval(a_i)}_{i\in I}$.
        Partition the segment space by the order types of $\{F(a_i)\}$ and $\curly{\tval(a_i)}$ (as subsets of $\Z$).
        Work in a fixed partition $\Sg'$.
        After relabeling we may assume that $F(a_1) \geq F(a_2) \geq \ldots$.
        Consider the (relabeled) sequence of vectors $\vec p_1, \vec p_2, \ldots, \vec p_I$.
        There is a unique subset $J \subset I$ such that all vectors with indices in $J$ are linearly independent,
        and all vectors with indices outside of $J$ are a linear combination of the preceding vectors.
        For any index $i \in I$ we call it independent if $i \in J$ and we call it dependent otherwise.

        Now, we define the following function
                $g_{\Sg'}: \Sg' \arr \Bt^I \times \Pt^J \times \Ct^{I - J}$.
        Let $a = (a_i)_{i\in I} \in \Sg'$.
        To define $g_{\Sg'}(a)$ we need to specify where it maps $a$ in each individual component of the product.

        For all $a_i$ record its interval type, giving the first component.

        For $a_j$ with $j \in J$, record the interval of $a_j$, giving the second component.

        For the third component do the following computation.
        Pick $a_i$ with $i$ dependent.
        Let $j$ be the largest independent index with $j < i$.
        Record $a_i \midr [F(a_j) - \gamma, F(a_j)]$.

        Combine $g_{\Sg'}$ for all the partitions to get a function $g: \Sg \arr \Bt^I \times \Pt^J \times \Ct^{I - J}$.
        \begin{alertblock}{Lemma}
            Suppose we have $c_1, c_2 \in \Q_p^{|x|}$ such that $f(c_1), f(c_2)$ are in the same partition and $g(f(c_1)) = g(f(c_2))$.
            Then $c_1, c_2$ have the same $\Psi$-type over $B$.
        \end{alertblock}          
        To get the desired bound one checks that the number of partitions multiplied by the size of the range of $g$ is $O(N^{|x|})$.
    \end{block}
    
    
    \begin{block}{References}
        \begin{thebibliography}{9}
            \bibitem{density}
                    M. Aschenbrenner, A. Dolich, D. Haskell, D. Macpherson, S. Starchenko,
                    \textit{Vapnik-Chervonenkis density in some theories without the independence property}, I,
                    Trans. Amer. Math. Soc. 368 (2016), 5889-5949
            \bibitem{reduct}
                    E. Leenknegt. \textit{Reducts of $p$-adically closed fields}, Archive for Mathematical logic, 53(3):285-306, 2014
        \end{thebibliography}
    \end{block}

\end{column} % End of column 2.2

\end{columns} % End of all the columns in the poster

\end{frame} % End of the enclosing frame

\end{document}
