	Suppose we have an (infinite) collection of sets $\F$. \\
	We define the \defn{shatter function} $\pi_\F \colon \N \arr \N$ of $\F$

	\begin{align*}
		\pi_\F(n) = \max \{ &\text {\# of atoms in the boolean algebra generated by $\mathcal S$} \\
		            &\mid \mathcal S \subset \F \text{ with } |\mathcal S| = n\}
	\end{align*}

	\begin{Theorem} [Sauer-Shelah '72]
		The shatter function is either $2^n$ or bounded by a polynomial.
	\end{Theorem}
	\begin{Definition}
		Suppose the growth of the shatter function of $\F$ is polynomial.
		Let $\vc(\F)$ be the infimum of all positive reals $r$ such that
		\begin{align*}
			\pi_\F(n) = O(n^r)
		\end{align*}
		Call $\vc(\F)$ the \defn{vc-density} of $\F$.
		If the shatter function grows exponentially, we let $\vc(\F) := \infty$.
	\end{Definition}

		Fix a formula $\phi(x_1 \ldots x_m, y_1, \ldots y_n) = \phi(\vec x, \vec y)$ and structure $M$.
		Plug in elements from $M$ for $y$ variables to get a family of definable sets in $M^m$.
		\begin{align*}
			\F^M_\phi = \curly{\phi(M^m, a_1, \ldots a_n) \mid a_1, \ldots a_n \in M}
		\end{align*}
		$\F^M_\phi$ is a \defn{uniformly definable family}. \\
		Define $\vc^M(\phi)$ to be the vc-density of the family $\F^M_\phi$.

	Given an NIP structure $M$ we define $\vc^M(n)$ to be the largest $\vc$-density achieved by uniformly definable families in $M^n$.
	\begin{align*}
		\vc^M(n) = \sup \curly{ \vc^M(\phi) \mid \phi(\vec x, \vec y) \text{ with } |\vec x| = n}
	\end{align*}
	Easy to show:
	\begin{align*}
		\vc^M(n) \geq n \cdot \vc^M(1) \geq n
	\end{align*}
                
