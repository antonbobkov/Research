\documentclass{amsart}

\usepackage{../AMC_style}	
\usepackage{../Research}
\usepackage{../Thm}

\usepackage{mathrsfs}

%\usepackage{setspace}
%\doublespacing

\usepackage[margin=.75in]{geometry}

\usepackage{pgfpages}
\pgfpagesuselayout{2 on 1}

\renewcommand{\AA}{\mathscr A}
  \newcommand{\II}{\mathscr I}
  \newcommand{\MM}{\mathscr M}

  \newcommand{\A}{\mathcal A}
  \newcommand{\B}{\mathcal B} 
\renewcommand{\C}{\mathcal C}
  \newcommand{\D}{\mathcal D}
  \newcommand{\F}{\mathcal F}
  \newcommand{\G}{\mathcal G}
\renewcommand{\H}{\mathcal H}
\renewcommand{\LL}{\mathcal L}
  \newcommand{\LLA}{\mathcal L_{aff}}
  \newcommand{\LLM}{\mathcal L_{Mac}}
  \newcommand{\M}{\mathcal M}

  \newcommand{\U}{\mathcal U}	

  
\newcommand{\curly}[1]{\left\{#1\right\}}
\newcommand{\paren}[1]{\left(#1\right)}
\newcommand{\abs}[1]{\left|#1\right|}

\providecommand{\floor}[1]{\left \lfloor #1 \right \rfloor }

\DeclareMathOperator{\Sg}{Sg}
\DeclareMathOperator{\Bt}{Bt}
\DeclareMathOperator{\Pt}{Pt}
\DeclareMathOperator{\Ct}{Ct}
\DeclareMathOperator{\vecspan}{span}
\DeclareMathOperator{\val}{val}



\title{VC-density in an additive reduct of $p$-adic numbers}
\author{Anton Bobkov}
\email{bobkov@math.ucla.edu}

\begin{document}

\begin{abstract}
        Aschenbrenner et. al. computed a bound $\vc(n) = 2n - 1$ for the VC density function in the field of $p$-adic numbers,
        but it is not known to be optimal.
	I investigate a certain $P$-minimal additive reduct of the field of $p$-adic numbers and
        using a cell decomposition result of Leenknegt I compute an optimal bound $\vc(n) = n$ for that structure.
\end{abstract}


\maketitle

VC density was introduced into model theory in \cite{density} by Aschenbrenner, Dolich, Haskell, MacPherson, and Starchenko as a natural notion of dimension for NIP theories.
In a NIP theory we can define the VC function

\begin{align*}
	\vc : \N \arr \N
\end{align*}

where $vc(n)$ measures complexity of the definable sets in an $n$-dimensional space.
The simplest possible behavior is $\vc(n) = n$ for all $n$.
\cite{density} computes an upper bound for this function to be $2n+1$, and it is not known whether it is optimal.
This same bound would hold in any reduct of $p$-adic numbers, so one may hope that the simplified structure of the reduct would allow a better bound.
In \cite{reduct}, Leenknegt provides a cell decomposition result for a certain P-minimal additive reduct of $p$-adic numbers.
Using that I'm able to improve the bound for the VC function, showing that $\vc(n) = n$.

%%%%%%%%%%%%%%%%%%%%%%%%%%%%%%%%

\section{VC-dimension and VC-density}

%%%%%%%%%%%%%%%%%%%%%%%%%%%%%%%%

Throughout this section we work with a collection $\F$ of subsets of $X$.
We call it a \defn{set system} $(X, \F)$.

\begin{Definition}
  \begin{itemize}
    \item $A \cap \F = \curly{A \cap F}_{F\in \F}$.
    \item For $A \subset X$ we say that $\F$ \defn{shatters} $A$ if $A \cap \F = \PP(A)$.
  \end{itemize}    
\end{Definition}  

\begin{Definition}
  We say $(X, \F)$ has VC-dimension $n$ if the largest set it shatters is of size $n$.
  If it can shatter arbitrarily large set we say that it has infinite VC-dimension.
\end{Definition}  

\begin{Definition}
  For $a \in X$ define $X_a = \curly{F \in \F \mid a \in F}$.
  Let $X^* = \curly{X_a}_{a \in X}$.
  We define $(\F, X^*)$ as the \defn{dual system} of $(X, \F)$.
  VC-dimension of a dual system is referred to as \defn{dual VC-dimension}.
\end{Definition}  


\begin{Lemma}
    A set system has finite VC-dimension if and only if its dual has finite VC-dimension.
\end{Lemma}  

\begin{Definition}
    Suppose we have a collection $\F$ of subsets of $X$.
    We define a \defn{shatter function} $\pi_\F(n)$ and \defn{dual shatter function} $\pi^*_\F(n)$
    \begin{align*}
            \pi_\F(n) &= \max \curly{|A \cap \F| \mid A \subset X \text{ and } |A| = n} \\
            \pi^*_\F(n) &= \max \curly{\text{number of atoms in Boolean algebra generated by B} \mid B \subset \F, |B| = n}
    \end{align*}
    Note that the dual shatter function is precisely the shatter function of the dual system.
\end{Definition}  

A simple upper bound is $\pi_\F(n) \leq 2^n$ (same for the dual).
In fact, if VC-dimension is infinite then $\pi_F(n) = 2^n$.

\begin{Theorem} [Sauer-Shelah]
  If the set system $(X, \F)$ has finite VC-dimension $d$ then $\pi_\F(n) \leq \choose{n}{d}$.    
\end{Theorem}

Thus systems where shatter function grows polynomially are precisely the systems of finite VC-dimension.
For set systems that grow polynomially we define VC-density as the degree of that polynomial.
More formally

\begin{Definition}
  Define \defn{vc-density} and \defn{dual vc-density} of $\F$ as
  \begin{align*}
          \vc(\F) &= \limsup_{n \to \infty}\frac{\log \pi_\F}{\log n} \\
          \vc^*(\F) &= \limsup_{n \to \infty}\frac{\log \pi^*_\F}{\log n}
  \end{align*}
\end{Definition}

In general, VC-density can be an arbitrary real number $\geq 1$.
Also note that shatter function that is bounded by polynomial doesn't have to be a polynomial.
There is an example of shatter function that grows like $n \log n$ (it has VC-density $1$).

So far the notions that we have defined are purely combinatorial.
We now adapt VC-dimension and VC-density to model theoretic context.

\begin{Definition}
  Work in a structure $M$.
  Let $\phi(x, y) \in \LL(M)$.

  \begin{itemize}
  \item For $b \in M$ let $\phi(M, b) = \{a \in M^{|x|} \mid \phi(a, b)\} \subseteq M^{|x|}$.
  \end{itemize}

  Now fix a finite collection of formulas $\Psi(x, y) = \curly{\phi_i(x, y)}$.

  \begin{itemize}
  \item Let $\Psi(M, M)= \{\phi_i(M, b) \mid \phi_i \in \Psi, b \in M^{|y|}\} \subseteq \PP(M^{|x|})$.
  \item Let $\F_\Psi = \Psi(M, M)$ forming a set system $(M^{|x|}, \F_\Psi)$.
  \item Define \defn{VC-dimension} of $\Psi$ to be the dual VC-dimension of $(M^{|x|}, \F_\Psi)$.
  \item Define \defn{VC-density} of $\Psi$, $\vc(\Psi)$ to be the dual VC-density of $(M^{|x|}, \F_\Psi)$.
  \end{itemize}

  We will also refer to the VC-density and VC-dimension of a single formula $\phi$
  viewing it as a one element collection $\curly{\phi}$.
\end{Definition}

Counting atoms of a Boolean algebra in model theoretic setting corresponds to counting types,
so it is instructive to rewrite shatter function in terms of number of types.

One can check that VC-dimension and VC-density of a formula are elementary notions,
so they only depend on the first-order theory of the structure.

\begin{Lemma}
    $\phi$ is NIP if and only if it has finite VC-dimension. 
\end{Lemma}

NIP theories thus are a natural context for study of VC-density.
There are examples of formulas having non-integer VC-density in an NIP theory,
however it is open whether one can have an irrational VC-density for a formula in an NIP theory.
In general, instead of working with a theory formula by formula we can look for uniform bounds:

\begin{Definition}
  For a given NIP structure $M$, define \defn{vc-function}
  \begin{align*}
    \vc^M(n) = \sup \{\vc(\phi(x, y)) \in \LL(M) \mid |x| = n\}
  \end{align*}
\end{Definition}

As before this definition is elementary, so it only depends on the theory of $M$.
One can easily check the following bounds
\begin{align*}
  \vc(1) &\geq 1 \\
  \vc(n) &\geq n\vc(1)
\end{align*}

However, it is not known whether the second inequality can be strict or whether $\vc(1) < \infty$ implies $\vc(n) < \infty$.

%%%%%%%%%%%%%%%%%%%%%%%%%%%%%%%%

\section{$p$-adic numbers}

%%%%%%%%%%%%%%%%%%%%%%%%%%%%%%%%

$P$-adic numbers are often studied in the language of Macintyre $\LLM$% = \curly{0, 1, +, -, \cdot, P_n}$.
which is a language of fields together with unary predicates $\curly{P_n}_{n \in \N}$ interpreted by

\begin{align*}
  P_n x \leftrightarrow \exist y \; y^n = x
\end{align*}  

Note that $P_n$ is a multiplicative subgroup of $\Q_p$ with finitely many cosets.

\begin{Theorem}
  $(\Q_p, \LLM)$ has quantifier elimination.
\end{Theorem}

There is also the following cell decomposition result

\begin{Theorem}
  Any formula $\phi(t, x)$ in $(\Q_p, \LLM)$ with $t$ singleton decomposes into the union of the following cells:
  \begin{align*}
          \curly{(t, x) \in K \times D \mid \vval a_1(x) \ \square_1 \vval (t - c(x)) \ \square_2 \vval a_2(x), t - c(x) \in \lambda P_n}
  \end{align*}
  where $D$ is a cell of a smaller dimension,
  $a_1(x), a_2(x), c(x)$ are $\emptyset$-definable,
  $\square$ is $<, \leq$ or no condition, and
  $\lambda  \in\Q_p$.    
\end{Theorem}  

In \cite{density}, Aschenbrenner, Dolich, Haskell, Macpherson, and Starchenko show that this structure has $\vc(n) \leq 2n - 1$,
however it is not known whether this bound is optimal.

In \cite{reduct}, Leenknegt analyzes the reduct of $p$-adic numbers to the language
\begin{align*}
    \LL_{aff}  = \curly{0, 1, +, -, \curly{\bar c}_{c \in \Q_p}, |, \curly{Q_{m,n}}_{m,n\in \N}}
\end{align*}
where $\bar c$ is a scalar multiplication by $c$,
$a | b$ stands for $\vval a \leq \vval b$,
and $Q_{m,n}$ is a unary predicate
\begin{align*}
    Q_{m,n} = \bigcup_{k \in \Z} p^{km} (1 + p^n\Z_p).
\end{align*}
Note that $Q_{m,n}$ is a subgroup of the multiplicative group of $\Q_p$ with finitely many cosets.
One can check that the extra relation symbols are definable in the full structure $(\Q_p, \LLM)$.
The following cell decomposition result is provided by \cite{reduct}

\begin{Theorem}
    Any formula $\phi(t, x)$ in $(\Q_p, \LL_{aff})$ with $t$ singleton decomposes into the union of the following cells:
    \begin{align*}
            \curly{(t, x) \in K \times D \mid \vval a_1(x) \ \square_1 \vval (t - c(x)) \ \square_2 \vval a_2(x), t - c(x) \in \lambda Q_{m,n} }
    \end{align*}
    where $D$ is a cell of a smaller dimension,
    $a_1(x), a_2(x), c(x)$ are linear polynomials,
    $\square$ is $<$ or no condition, and
    $\lambda  \in\Q_p$.
\end{Theorem}  

Moreover, \cite{reduct} shows that $(\Q_p, \LL_{aff})$ is a $P$-minimal reduct,
that is one-dimensional definable sets coincide with one-dimensional definable sets in the full structure.

I am able to compute $\vc$-function for this structure
\begin{Theorem} {Theorem (B.)}
    $(\Q_p, \LL_{aff})$ has $\vc(n) = n$.
\end{Theorem}



%%%%%%%%%%%%%%%%%%%%%%%%%%%%%%%%

\section{Cell Decomposition}

%%%%%%%%%%%%%%%%%%%%%%%%%%%%%%%%

\begin{Definition}
	Let
	\begin{align*}
		Q_{n,m} = \bigcup_{k \in \Z} p^{kn} (1 + p^m\Z_p) 
	\end{align*}
	It is a subgroup of the multiplicative group of $\Q_p$ with finitely many cosets.
\end{Definition}

We work with the reduct of $p$-adic numbers in the language $\LLA = \curly{\Q_p, \curly{R_{n,m}}_{n,m\in \N}, +, -, \curly{\bar c}_{c \in \Q_p} }$,
where $\bar c$ is a scalar multiplication by $c$, and $R_{n,m}$ is a predicate for cosets of $Q_{n,m}$
\begin{align*}
    Q_{n,m} = \bigcup_{k \in \Z} p^{kn} (1 + p^m\Z_p) 
\end{align*}


In \cite{reduct}, Leenknegt provides a cell decomposition result for this structure.
Any formula $\phi(t, x)$  with $t$  singleton decomposes as the union of the following cells:

\begin{align*}
	\curly{(t, x) \in K \times D \mid \val a_1(x) \square_1 \val (t - c(x)) \square_2 \val a_2(x), t - c(x) \in \lambda Q_{n',m'} }
\end{align*}

where $D$ is a cell of a smaller dimension, $a_1, a_2, c$ are linear polynomials in  $x$, $\square$ is $<$ or no condition, $\lambda  \in\Q_p$.

\begin{Lemma}
	For a formula $\phi(x)$ with $x = (t, \bar x)$ there exists a family of formulas $\Psi'(x)$
	\begin{align*}
		&\val \paren{q_i(x)} < \val \paren{q_j(x)} & i, j \in I \\
		&\val \paren{q_i(x)} \in \lambda_k Q_{n,m} & i \in I , k \in K \\
		&\bar x \in D_l & l \in L
	\end{align*}
	with $I, K, L$ finite,
	$D_l$ cells,
	$q_i$ linear polynomials,
	$\lambda_k \in \Q_p$, and
	$Q = Q_{n,m}$ for some $n,m$.
	Moreover we have that if $a, a' \in Q_p^{|x|}$ agree on all the formulas from $\Psi'$ then they agree on $\phi$.
\end{Lemma}

\begin{proof}
	To see that, apply cell decomposition theorem to $\phi(t, \bar x)$.
	Let $q_i$ enumerate all of the polynomials $a_1(\bar x), a_2(\bar x), t - c(\bar x)$ that show up in the cells.
	Let $D_l$ be the smaller cells for the $\bar x$ components that appear in the cells.
	Choose $n,m$ large enough to cover all $n', m'$ that come up in the cells for $Q_{n',m'}$.
	Choose $\lambda_k$ to go over all the cosets of $Q_{n,m}$.
\end{proof}

Applying this lemma inductively to smaller cells, we obtain a family $\Psi(x)$
\begin{align*}
		&\val \paren{q_i(x)} < \val \paren{q_j(x)} & i, j \in I \\
		&\val \paren{q_i(x)} \in \lambda_k Q_{n,m} & i \in I , k \in K
\end{align*}
with $I, K$ finite,
$q_i$ linear polynomials,
$\lambda_k \in \Q_p$, and
$Q = Q_{n,m}$ for some $n,m$.
Moreover whenever $a, a' \in Q_p^{|x|}$ agree on all the formulas from $\Psi$ then they agree on $\phi$.

Now fix a formula $\phi(x; y)$ for finding an upper bound of its VC-density.
Using the result above we can construct a family of formulas $\Psi(x; y)$ which can be now written as

\begin{align*}
	&\val (p_i(x) - c_i(y)) < \val (p_j(x) - c_j(y)) & i, j \in I \\
	&\val (p_i(x) - c_i(y)) \in \lambda_k Q & i \in I , k \in K
\end{align*}

where $I, K$ finite,
$p_i$ a homogeneous linear polynomials in $x$,
$c_i$ is a linear polynomial in $y$,
$\lambda_k \in \Q_p$, and
$Q = Q_{n,m}$ for some $n,m$
(to do this we simply split the polynomial $q_i$ into its $x$ part and into its $y$ part including the constant term).
Now for any parameter set $B$ we have that if $a, a'$ have the same $\Psi$-type over $B$ then they have the same $\phi$-type over $B$.
Thus it suffices to bound VC-density for $\Psi$.

%%%%%%%%%%%%%%%%%%%%%%%%%%%%%%%%

\section{Key Lemmas and Definitions}

%%%%%%%%%%%%%%%%%%%%%%%%%%%%%%%%
\begin{Definition}
	A tuple $p \in  \Q_p^{|x|}$ can be viewed as a vector $\vec p$, treating $\Q_p^{|x|}$ as a vector space over $\Q_p$.
\end{Definition}

We may rewrite our collection of formulas $\Psi(x, y)$ as

\begin{align*}
	&\val (\vec p_i \cdot \vec x) - c_i(y) < \val (\vec p_j \cdot \vec x) - c_j(y) & i, j \in I \\
	&\val (\vec p_i \cdot \vec x) - c_i(y) \in \lambda_k Q & i \in I , k \in K
\end{align*}

\begin{Lemma}	 \label{gamma}
	Suppose we have a collection of vectors $\curly{\vec p_i}_{i \in I}$ with each $\vec p_i \in \Q_p^{|x|}$.
	Pick a subset $J \subset I$ and $j \in I$ such that
	\begin{align*}
		\vec p_j \in \vecspan \curly{\vec p_i}_{i \in J} 
	\end{align*}
	Suppose we have $\vec x \in \Q_p^{|x|}, \alpha \in \Z$ with
	\begin{align*}
		\val(\vec p_i \cdot \vec x) > \alpha \text{ for all } i \in J
	\end{align*}
	Then
	\begin{align*}
		\val(\vec p_j \cdot \vec x) > \alpha - \gamma
	\end{align*}
	for some $\gamma \in \Z^{\geq 0}$.
	Moreover $\gamma$ can be chosen independently from $J, j, \vec x, \alpha$ depending only on $\curly{\vec p_i}_{i \in I}$, independent of their order.
\end{Lemma}
\begin{proof}
	Fix some $i, J$.
	For some $c_i$
	\begin{align*}
		\vec p_j &= \sum_{i \in J} c_i \vec p_i \\
		\vec p_j \cdot \vec x &= \sum_{i \in J} c_i \vec p_i \cdot \vec x
	\end{align*}
	We have
	\begin{align*}
		\val \paren{c_i \vec p_i \cdot \vec x} = \val \paren{c_i} + \val \paren{\vec p_i \cdot \vec x} > \val \paren{c_i} + \alpha
	\end{align*}
	Pick $\gamma = -\max \val \paren{c_i}$ or $0$ if all those values are positive.
	Then we have 
	\begin{align*}
		&\val \paren{c_i \vec p_i \cdot \vec x} > \alpha - \gamma &\text{ for all $i \in J$}\\
		&\sum_{i \in J} c_i \vec p_i \cdot \vec x > \alpha - \gamma
	\end{align*}
	This shows that we can pick such $\gamma$ for a given choice of $i, J$, but independent from $\alpha, \vec x$.
	To get a choice independent from $i, J$, go over all such eligible choices (of which there are finitely many as $I$ is finite),
	pick $\gamma$ for each, and then take the maximum of those values.
\end{proof}

\begin{Definition}
	For $c \in \Q_p, \alpha \in \Z$ we define an open ball 
	\begin{align*}
		B(c, \alpha) = \curly{c' \in \Q_p \mid \val \paren{c' - c} \leq \alpha}
	\end{align*}
\end{Definition}

\begin{Definition}
	Suppose we have a finite $T \subset \Q_p$.
	We view it as a tree as follows.
	Branches through the tree are elements of $T$.
	With this tree we associate open balls $B(t_1, \val(t_1 - t_2))$ for all $t_1, t_2 \in T$.
	An interval is two balls $B(t_1, v_1) \supset B(t_2, v_2)$ with no balls in between.
	An element $a \in \Q_p$ belongs to this interval if $a \in B(t_1, v_1) \backslash B(t_2, v_2)$.
	There are at most $2|T|$ different intervals and they partition the entire space.
	
	Fix a parameter set $B$ of size $N$.
	
	Consider a tree $T = \curly{c_i(b) \mid b \in B, i \in I}$
	It has at most $O(N) = N \cdot |I|$ many intervals.
	Denote the set of all intervals as $\Pt$.
	For the remainder of the paper we work with this tree.	
\end{Definition}

\begin{Definition}
	Let $c \in \Q_p$.
	It lies in the tree in one of the unique intervals $B(c_L, \alpha_L) \backslash B(c_U, \alpha_U)$.
	Define $F(c)$, the floor of $c$ to be $\alpha_L$.
\end{Definition}

\begin{Definition}
	We say $x, x' \in \Q_p$ have the same tree type if
	\begin{itemize}
		\item $\val(x - c_i(b)) < \val(x - c_j(b))$ iff $\val(x' - c_i(b)) < \val(x' - c_j(b))$ for all $i,j \in I, b \in B$
		\item $x + c_i(b)$ is in the same $Q$-coset as $x' + c_i(b)$ for all $i \in I, b \in B$
	\end{itemize}
\end{Definition}
 
\begin{Lemma} \label{sigh}
	Let $a, a' \in \Q_p^{|x|}$.
	If $p_i(a), p_i(a')$ have the same tree type for all $i \in I$, then $a, a'$ have the same $\Psi$-type.
\end{Lemma}
\begin{proof}
	Clear from the construction.
\end{proof}

\begin{Definition}
	For $c \in \Q_p$ and $\alpha, \beta \in \Z$ let $c \midr [\alpha, \beta] \in \paren{\Z/p\Z}^{\beta - \alpha}$ be the record of coefficients of $c$ for the valuations between $\alpha, \beta$.
	More precisely write $c$ in its power series form
	\begin{align*}
		c = \sum_{\gamma \in Z} c_\gamma p^\gamma \text{ with } c_\gamma \in \Z/p\Z
	\end{align*}
	Then $c \midr [\alpha, \beta]$ is just $(c_\alpha, c_{\alpha+1}, \ldots c_\beta)$.
\end{Definition}

The following lemma is an adaptation of lemma 7.4 in \cite{density}.

\begin{Lemma} \label{distance}
	For $n,m$ there exists $D = D(n,m) \in \Z$ such that for any $x,y,a \in \Q_p$ if
	\begin{align*}
		\val (x - c) = \val (y - c) < \val (x - y) - D
	\end{align*}
	then $x - c, y - c$ are in the same coset of $Q_{n,m}$.
\end{Lemma}
\begin{proof}
	Define that $a,b \in \Q_p$ are similar if $\val a = \val b$ and
	\begin{align*}
		a \midr [\val a, \val a + (m + n)] = b \midr [\val b, \val b + (m + n)]
	\end{align*}
	If $a,b$ are similar then
	\begin{align*}
		a \in Q_{n,m} \leftrightarrow b \in Q_{n,m}
	\end{align*}
	Moreover for any $\lambda \in \Q_p$, if $a,b$ are similar we would also have $a/\lambda, b/\lambda$ are similar.
	Thus if $a,b$ are similar, then they belong in the same coset of $Q_{n,m}$.
	If we pick $D = n + m$ then conditions of the lemma force $x - c, y - c$ to be similar.
\end{proof} 

The following construction is along the lines of lemmas 7.3, 7.5 of \cite{density}.

\begin{Definition}
	For two balls $B(a, \alpha), B(b, \beta)$ let $\gamma = \min(\alpha, \beta, \val(a - b))$.
	Define the distance between those two balls to be $|\alpha - \gamma| + |\beta - \gamma|$.
	In $\Q_p$ value group is discrete and residue field is finite, so there are finitely many balls at a fixed distance from a given ball.
	Near balls of $B(a, \alpha)$ are defined to be balls with distance $\D$ from $B(a, \alpha)$.
	Enumerate those as:
	\begin{align*}
		B_1(a, \alpha), B_2(c, \alpha), \ldots B_{N_D}(a, \alpha)
	\end{align*}
	Near balls partition the space
	\begin{align*}
		\curly{b \in \Q_p \mid |\val(a - b) - \alpha| \leq D}
	\end{align*}
\end{Definition}

\begin{Definition}
	Let $c \in \Q_p$.
	It lies in our tree in one of the intervals $B(c_L, \alpha_L) \backslash B(c_U, \alpha_U)$.
	Suppose $c$ lies in one of the near balls of $B(c_L, \alpha_L)$ or $B(c_U, \alpha_U)$.
	Then define its interval type to be the index of that near ball.
	Otherwise define its interval type to be the coset of $c - c_U$ of $Q$.
	Denote the space of all the possible branch types $\Bt$.
\end{Definition}

\begin{Lemma} \label{interval}
	If $a, a'$ are in the same interval and have the same interval type then they have the same tree type.
\end{Lemma}
\begin{proof}
	First part of the tree type definition is satisfied as $a, a'$ are in the same interval,
	so we only need to demonstrate that the corresponding $Q$-cosets match.
	Pick any element of our tree $c_i(b)$.
	We want to show that $a - c_i(b), a' - c_i(b)$ are in the same $Q$-coset.
		
	Suppose $a$ is in one of the near balls.
	As $a'$ has the same interval type, it has to be in the same near ball.
	By definition of the near ball we then have $\val(a - c_i(b)) = \val(a' - c_i(b)) < \val(a - a') - D$.
	Thus by Lemma \ref{distance} we have $a - c_i(b), a' - c_i(b)$ in the same $Q$-coset.
	
	Now, suppose both $a, a'$ aren't in any near balls.
	Label their interval as $B(c_L, \alpha_L) \backslash B(c_U, \alpha_U)$.
	Then we have 
	\begin{align*}
		\alpha_L + D < &\val (a - c_U) < \alpha_U - D \\
		\alpha_L + D < &\val (a' - c_U) < \alpha_U - D
	\end{align*}
	as otherwise one (both) of them would be in one of the near balls.
	We have either $\val(c_U - c_i(b)) \geq \alpha_U$ or $\val(c_U - c_i(b)) \leq \alpha_L$
	as otherwise it would contradict the definition of an interval.
	
	Suppose it is the first case $\val(c_U - c_i(b)) \geq \alpha_U$.
	Then
	\begin{align*}
		 \val(a - c_i(b)) = \val(a - c_U) < \alpha_U - D \leq \val(c_U - c_i(b)) - D
	\end{align*}
	so by Lemma \ref{distance} we have $a - c_i(b), a - c_U$ are in the same $Q$-coset.
	By a parallel argument we have $a' - c_i(b), a' - c_U$ are in the same $Q$-coset.
	As we are assuming $a, a'$ have the same tree type it implies that $a - c_U, a' - c_U$ are in the same $Q$-coset.
	Thus by transitivity we get that $a - c_i(b), a' - c_i(b)$ are in the same $Q$-coset.
	
	For the second case, suppose $\val(c_U - c_i(b)) \leq \alpha_L$.
	Then
	\begin{align*}
		\val(a - c_i(b)) = \val(c_U - c_i(b)) \leq \alpha_L < \val(a - c_U) - D
	\end{align*}
	so by Lemma \ref{distance} we have $a - c_i(b), c_U - c_i(b)$ are in the same $Q$-coset.
	By a parallel argument we have $a' - c_i(b), c_U - c_i(b)$ are in the same $Q$-coset.
	Thus by transitivity we get that $a - c_i(b), a' - c_i(b)$ are in the same $Q$-coset.
\end{proof}


%%%%%%%%%%%%%%%%%%%%%%%%%%%%%%%%

\section{Main Proof}

%%%%%%%%%%%%%%%%%%%%%%%%%%%%%%%%
Fix $\gamma$ corresponding to $\curly{\vec p_i}_{i \in I}$ according to Lemma \ref{gamma}.

\begin{Definition}
	Denote $\Z/p\Z^\gamma$ as $\Ct$.
\end{Definition}

\begin{Definition}
	Let $f: \Q_p^{|x|} \arr \Q_p^I$ with $f(\bar c) = (p_i(\bar c))_{i \in I}$.
	Define the segment space $\Sg$ to be the image of $f$.	
\end{Definition}


Given a tuple $(a_i)_{i\in I}$in the segment space look at the corresponding floors $\curly{F(a_i)}_{i\in I}$.
Those are ordered as elements of $\Z$.
Partition the segment space by order type of $\{F(a_i)\}$.
Work in a fixed partition $\Sg'$.
After relabeling we may assume that
\begin{align*}
	F(a_1) \geq F(a_2) \geq \ldots 
\end{align*}

Consider the (relabeled) sequence of vectors $\vec p_1, \vec p_2, \ldots, \vec p_I$.
There is a unique subset $J \subset I$ such that all vectors with indices in $J$ are linearly independent, and all vectors with indices outside of $J$ are a linear combination of preceding vectors.
For any index $i \in I$ we call it independent if $i \in J$ and we call it dependent otherwise.
%Let $m = |J|$.

Now, we define the following function
\begin{align*}
	g: \Sg' \arr \Bt^I \times \Pt^J \times \Ct^{I - J}
\end{align*}

Let $\bar a = (a_i)_{i\in I} \in \Sg'$.
To define $g(\bar a)$ we need to specify where it maps $\bar a$ in each individual component of the product.

For all $a_i$ record its interval type $\in \Bt$, giving the first component.

For $a_j$ with $j \in J$, record the interval of $a_j$, giving the second component.

For the third component do the following computation.
Pick $a_i$ with $i$ dependent.
Let $j$ be the largest independent index with $j < i$.
Record $a_i \midr [F(a_j) - \gamma, F(a_j)]$.



\begin{Lemma}
	For $\bar a, \bar a' \in \Sg'$ if $g(\bar a) = g(\bar a')$ then $a_i, a_i'$ have the same tree type for all $i \in I$.	
\end{Lemma}

\begin{proof}
	For each $i$ we show that $a_i, a_i'$ are in the same interval and have the same interval type, so the conclusion follows by Lemma \ref{interval}.
	$\Bt$ records the interval type of each element, so if $g(\bar a) = g(\bar a')$ then $a_i, a_i'$ have the same interval type for all $i \in I$.
	Thus it remains to show that $a_i, a_i'$ lie in the same interval for all $i \in I$.
	Suppose $i$ is an independent index.
	Then by construction, $\Pt$ records the interval for $a_i, a_i'$, so those have to belong to the same interval.
	Now suppose $i$ is dependent.
	Pick the largest $j < i$ such that $j$ is independent.
	We have $F(a_i) \leq F(a_j)$ and $F(a_i') \leq F(a_j')$.
	Moreover $F(a_j) = F(a_j')$ as they are mapped to the same interval (using the earlier part of the argument as $j$ is independent).
	
	\begin{Claim}
		$\val(a_i - a_i') > F(a_j) - \gamma$
	\end{Claim}
	\begin{proof}
		Let $\vec x, \vec x' \in \Q_p^{|x|}$ be some elements with
		\begin{align*}
			\vec p_k &\cdot \vec x = a_k \\
			\vec p_k &\cdot \vec x' = a_k' \text { for all } k \in I
		\end{align*}
		It is always possible to do that as $\bar a, \bar a' \in \Sg'$. 
		Let $J'$ be the set of the independent indices less than $i$.
		We have 
		\begin{align*}
			\val(a_k - a_k') > F(a_k) \text { for all } k \in J'
		\end{align*}
		as for the independent indices $a_k, a_k'$ lie in the same interval.
		\begin{align*}
			&\val(a_k - a_k') > F(a_j) \text { for all } k \in J' \text{ by monotonicity of $F(a_k)$} \\
			&\val(\vec p_k \cdot \vec x - \vec p_k \cdot \vec x') > F(a_j) \text { for all } k \in J' \\
			&\val(\vec p_k \cdot (\vec x - \vec x')) > F(a_j) \text { for all } k \in J' \\
		\end{align*}
		$J'$ and $i$ match the requirements of Lemma \ref {gamma} so we conclude
		\begin{align*}
			&\val(\vec p_i \cdot (\vec x - \vec x')) > F(a_j) - \gamma \\
			&\val(\vec p_i \cdot \vec x - \vec p_i \cdot \vec x') > F(a_j) - \gamma \\
			&\val(a_i - a_i')) > F(a_j) - \gamma
		\end{align*}
		as needed, finishing the proof of the claim.
	\end{proof}	
	Additionally $a_i, a_i'$ have the same image in $\Ct$ component, so we have
	\begin{align*}
		\val(a_i - a_i') > F(a_j) 
	\end{align*}
	As $F(a_i) \leq F(a_j)$, $a_i, a_i'$ have to lie in the same interval.	
\end{proof}

\begin{Corollary}
	$\Psi(x,y)$ has VC-density $\leq |x|$
\end{Corollary}

\begin{proof}
	Suppose we have $c, c' \in \Q_p^{|x|}$ such that $f(c), f(c')$ are in the same partition and $g(f(c)) = g(f(c'))$.
	Then by the previous lemma $p_i(c)$ has the same tree type as $p_i(c')$ for all $i\in I$.
	Then by Lemma \ref{sigh} $c, c'$ have the same $\Psi$-type.
	Thus the number of possible $\Psi$-types is bounded by the size of the range of $g$ times the number of possible partitions
	
	\begin{align*}
		\text{(number of partitions)} \cdot |Bt|^{|I|} \cdot |Pt|^{|J|} \cdot |Ct|^{|I-J|}
	\end{align*}
	
	We have
	
	\begin{align*}
		|\Bt| = N_D + \text {number of cosets of $Q$}
		|\Pt| &\leq N \cdot I^2 \text { (the only component dependent on $N$)} \\
		|\Ct| &= p^\gamma 
	\end{align*}
	and there are at most ${|I|}!$ many partitions of $\Sg$. 
	This gives us a bound
	
	\begin{align*}
		{|I|}! \cdot |Bt|^{|I|} \cdot (N \cdot {|I|}^2)^{|J|} \cdot p^{\gamma {|I-J|}} = O(N^{|J|})
	\end{align*}	
	
	Every $p_i$ is an element of a $|x|$-dimensional vector space, so there can be at most $|x|$ many independent vectors.
	Thus we have $|J| \leq |x|$ and the bound follows.
\end{proof}

\begin{Corollary}
	In the language $\LLA$ we have $\vc(n) = n$.
\end{Corollary}

\begin{proof}
	Previous lemma implies that $\vc(\phi) \leq \vc(\Psi) \leq |x|$.
	As choice of $\phi$ was arbitrary, this implies that VC-density of any formula is bounded by the arity of $x$.
\end{proof}

\begin{thebibliography}{9}
	\bibitem{density}
		M. Aschenbrenner, A. Dolich, D. Haskell, D. Macpherson, S. Starchenko,
		\textit{Vapnik-Chervonenkis density in some theories without the independence property}, I, preprint (2011)
	\bibitem{reduct}
		E. Leenknegt. \textit{Reducts of $p$-adically closed fields}, Archive for Mathematical logic, 53(3):285-306, 2014
\end{thebibliography}

\end{document}








