\documentclass{amsart}

\usepackage{../AMC_style}	
\usepackage{../Research}
\usepackage{../Thm}

\usepackage{mathrsfs}

%\usepackage{setspace}
%\doublespacing

\usepackage[margin=1in]{geometry}

\usepackage{pgfpages}
\pgfpagesuselayout{2 on 1}

\renewcommand{\AA}{\mathscr A}
  \newcommand{\II}{\mathscr I}
  \newcommand{\MM}{\mathscr M}

  \newcommand{\A}{\mathcal A}
  \newcommand{\B}{\mathcal B} 
\renewcommand{\C}{\mathcal C}
  \newcommand{\D}{\mathcal D}
\renewcommand{\H}{\mathcal H}
  \newcommand{\G}{\mathcal G}
\renewcommand{\LL}{\mathcal L_R}
	\newcommand{\M}{\mathcal M}

  \newcommand{\U}{\mathcal U}	

  \newcommand{\K}{\boldface K_\alpha}
\renewcommand{\S}{S_\alpha}

\newcommand{\curly}[1]{\left\{#1\right\}}
\newcommand{\paren}[1]{\left(#1\right)}
\newcommand{\abs}[1]{\left|#1\right|}

\providecommand{\floor}[1]{\left \lfloor #1 \right \rfloor }

\DeclareMathOperator{\Sg}{Sg}
\DeclareMathOperator{\Bt}{Bt}
\DeclareMathOperator{\Pt}{Pt}
\DeclareMathOperator{\Ct}{Ct}
\DeclareMathOperator{\vecspan}{span}
\DeclareMathOperator{\val}{val}



\title{VC-density in an additive reduct of p-adic numbers}
\author{Anton Bobkov}
\email{bobkov@math.ucla.edu}

\begin{document}

\begin{abstract}
	\cite{density} computed a bound $2n+1$ for the VC function in p-adic numbers, but it is not known to be optimal.
	I investigate a C-minimal additive reduct of p-adic numbers and using techniques of \cite{reduct} I compute an optimal bound $n$ for that structure.
\end{abstract}


\maketitle

VC density was introduced in \cite{density} by Aschenbrenner, Dolich, Haskell, MacPherson, and Starchenko as a natural notion of dimension for NIP theories.
In a NIP theory we can define the VC function

\begin{align*}
	\vc : \N \arr \N
\end{align*}

where $vc(n)$ measures complexity of the definable sets in an $n$-dimensional space.
The simplest possible behavior is $\vc(n) = n$ for all $n$.
\cite{density} computes an upper bound for this function to be $2n+1$, and it is not known whether it is optimal.
This same bound would hold in any reduct of p-adic numbers, so one may hope that the simplified structure of the reduct would allow a better bound.
In \cite{reduct}, Leenknegt provides a cell decomposition result for the C-minimal additive reduct of p-adic numbers.
Using that I'm able to improve the bound for the VC function, showing that $\vc(n) = n$.

%%%%%%%%%%%%%%%%%%%%%%%%%%%%%%%%

\section{Cell Decomposition}

%%%%%%%%%%%%%%%%%%%%%%%%%%%%%%%%


We work with the reduct of p-adic numbers in the language $\LL = \curly{\Q_p, \curly{Q_{n,m}}_{n,m\in \N}, +, -, \curly{\bar c}_{c \in K} }$,
where $\bar c$ is a scalar multiplication by $c$, and $Q_{n,m}$ is a unary predicate.

\begin{align*}
	Q_{n,m} = \curly{\bigcup_{k \in \Z} p^{kn} (1 + p^m\Z_p) }
\end{align*}

\cite{reduct} provides a cell decomposition result for this structure.
Any formula $\phi(t, x)$  with $t$  singleton decomposes as the union of the following cells:

\begin{align*}
	\curly{(t, x) \in K \times D \mid \val a_1(x) \square_1 \val (t - c(x)) \square_2 \val a_2(x), t - c(x) \in \lambda Q_{n,m} }
\end{align*}

where $D$ is a cell of a smaller dimension, $a_1, a_2, c$ are linear polynomials in  $x$, $\square$ is $<$ or no condition, $\lambda  \in\Q_p$.

\begin{Lemma}
	For a formula $\phi(x)$ with $x = (t, \bar x)$ there exists a family of formulas $\Psi'(x)$
	\begin{align*}
		&\val \paren{q_i(x)} < \val \paren{q_j(x)} & i, j \in I \\
		&\val \paren{q_i(x)} \in \lambda_k Q_{n,m} & i \in I , k \in K \\
		&\bar x \in D_l & l \in L
	\end{align*}
	with $I, K, L$ finite,
	$D_l$ cells,
	$q_i$ linear polynomials,
	$\lambda_k \in \Q_p$, and
	$Q = Q_{n,m}$ for some $n,m$.
	Moreover we have that if $a, a' \in Q_p^{|x|}$ agree on all the formulas from $\Psi'$ then they agree on $\phi$.
\end{Lemma}

\begin{proof}
	To see that, apply cell decomposition theorem to $\phi(t, \bar x)$.
	Let $q_i$ enumerate all of the polynomials $a_1(\bar x), a_2(\bar x), t - c(\bar x)$ that show up in the cells.
	Let $D_l$ be the smaller cells for the $\bar x$components that appear in the cells.
	Choose $n,m$ large enough to cover all $n', m'$ that come up in the cells for $Q_{n',m'}$.
	Choose $\lambda_k$ to go over all the cosets of $Q_{n,m}$.
\end{proof}

Applying this lemma inductively to smaller cells, we obtain a family $\Psi(x)$
\begin{align*}
		&\val \paren{q_i(x)} < \val \paren{q_j(x)} & i, j \in I \\
		&\val \paren{q_i(x)} \in \lambda_k Q_{n,m} & i \in I , k \in K
\end{align*}
with $I, K$ finite,
$q_i$ linear polynomials,
$\lambda_k \in \Q_p$, and
$Q = Q_{n,m}$ for some $n,m$.
Moreover whenever $a, a' \in Q_p^{|x|}$ agree on all the formulas from $\Psi$ then they agree on $\phi$.

Now fix a formula $\phi(x; y)$ for finding an upper bound of its VC-density.
Using the result above we can construct a family of formulas $\Psi(x; y)$ which can be now written as

\begin{align*}
	&\val p_i(x) - c_i(y) < \val p_j(x) - c_j(y) & i, j \in I \\
	&\val p_i(x) - c_i(y) \in \lambda_k Q & i \in I , k \in K
\end{align*}

where $I, K$ finite,
$p_i$ a homogeneous linear polynomials in $x$,
$c_i$ is a linear polynomial in $y$,
$\lambda_k \in \Q_p$, and
$Q = Q_{n,m}$ for some $n,m$
(to do this we simply split the polynomial $q_i$ into its $x$ part and into its $y$ part including the constant term).
Now for any parameter set $B$ we have that if $a, a'$ have the same $\Psi$-type over $B$ then they have the same $\phi$-type over $B$.
Thus it suffices to bound VC-density for $\Psi$.

%%%%%%%%%%%%%%%%%%%%%%%%%%%%%%%%

\section{Key Lemmas and Definitions}

%%%%%%%%%%%%%%%%%%%%%%%%%%%%%%%%
\begin{Definition}
	A tuple $p \in  \Q_p^{|x|}$ can be viewed as a vector $\vec p$, treating $\Q_p^{|x|}$ as a vector space over $\Q_p$.
\end{Definition}

We may rewrite our collection of formulas $\Psi(x, y)$ as

\begin{align*}
	&\val (\vec p_i \cdot \vec x) - c_i(y) < \val (\vec p_j \cdot \vec x) - c_j(y) & i, j \in I \\
	&\val (\vec p_i \cdot \vec x) - c_i(y) \in \lambda_k Q & i \in I , k \in K
\end{align*}

\begin{Lemma}	 \label{gamma}
	Suppose we have a collection of vectors $\curly{\vec p_i}_{i \in I}$ with each $\vec p_i \in \Q_p^{|x|}$.
	Pick a subset $J \subset I$ and $j \in I$ such that
	\begin{align*}
		\vec p_j \in \vecspan \curly{\vec p_i}_{i \in J} 
	\end{align*}
	Suppose we have $\vec x \in \Q_p^{|x|}, \alpha \in \Z$ with
	\begin{align*}
		\val(\vec p_i \cdot \vec x) > \alpha \text{ for all } i \in J
	\end{align*}
	Then
	\begin{align*}
		\val(\vec p_j \cdot \vec x) > \alpha - \gamma
	\end{align*}
	for some $\gamma \in \Z^{\geq 0}$.
	Moreover $\gamma$ can be chosen independently from $J, j, \vec x, \alpha$ depending only on $\curly{\vec p_i}_{i \in I}$, independent of their order.
\end{Lemma}
\begin{proof}
	INSERT PROOF HERE
\end{proof}

\begin{Definition}
	For $c \in \Q_p, \alpha \in \Z$ we define an open ball 
	\begin{align*}
		B(c, \alpha) = \curly{c' \in \Q_p \mid \val \paren{c' - c} \leq \alpha}
	\end{align*}
\end{Definition}

\begin{Definition}
	Suppose we have a finite $T \subset \Q_p$.
	We view it as a tree as follows.
	Branches through the tree are elements of $T$.
	With this tree we associate open balls $B(t_1, \val(t_1 - t_2))$ for all $t_1, t_2 \in T$.
	An interval is two balls $B(t_1, v_1) \supset B(t_2, v_2)$ with no balls in between.
	An element $a \in \Q_p$ belongs to this interval if $a \in B(t_1, v_1) \backslash B(t_2, v_2)$.
	There are at most $2|T|$ different intervals and they partition the entire space.
	
	Fix a parameter set $B$ of size $N$.
	
	Consider a tree $T = \curly{c_i(b) \mid b \in B, i \in I}$
	It has at most $O(N) = N \cdot |I|$ many intervals.
	Denote the set of all intervals as $\Pt$.
	For the remainder of the paper we work with this tree.	
\end{Definition}

\begin{Definition}
	Let $c \in \Q_p$.
	It lies in the tree in one of the unique intervals $B(c_L, \alpha_L) \backslash B(c_U, \alpha_U)$.
	Define $F(c)$, the floor of $c$ to be $\alpha_L$.
\end{Definition}


\begin{Definition}
	We say $a, a' \in \Q_p^{|x|}$ have the same $\Psi$-type if they have the same $\Psi$ type over $B$.	
\end{Definition}

\begin{Definition}
	We say $x, x' \in \Q_p$ have the same tree type if
	\begin{itemize}
		\item $x + c_i(b)$ is in the same $Q$-coset as $x' + c_i(b)$ for all $i \in I, b \in B$
		\item $\val(x + c_i(b)) < \val(x + c_j(b))$ iff $\val(x' + c_i(b)) < \val(x' + c_j(b))$ for all $i,j \in I, b \in B$
	\end{itemize}
\end{Definition}
 
\begin{Lemma} \label{sigh}
	Let $a, a' \in \Q_p^{|x|}$.
	If $p_i(a), p_i(a')$ have the same tree type for all $i \in I$, then $a, a'$ have the same $\Psi$-type.
\end{Lemma}
\begin{proof}
	INSERT PROOF HERE
\end{proof}

The following lemma is an adaptation of lemma 7.4 in \cite{density}.

\begin{Lemma}
	For $n,m$ there exists $D = D(n,m) \in \Z$ such that for any $x,y,a \in \Q_p$ if
	\begin{align*}
		\val (x - a) = \val (y - a) < \val (x - y) - D
	\end{align*}
	then $x - a, y - a$ are in the same coset of $Q_{n,m}$.
\end{Lemma}
\begin{proof}
	INSERT PROOF HERE
\end{proof}

Next definition is along the lines of lemma 7.5 of \cite{density}.

\begin{Definition}
	Using $D$ from the previous lemma define an enumeration of near balls
	\begin{align*}
		B_1(c, \alpha), B_2(c, \alpha), \ldots B_{N_D}(c, \alpha)
	\end{align*}
\end{Definition}

\begin{Definition}
	Let $c \in \Q_p$.
	It lies in our tree in one of the intervals $B(c_L, \alpha_L) \backslash B(c_U, \alpha_U)$.
	Suppose $c$ lies in one of the near balls corresponding to $B(c_L, \alpha_L)$ or $B(c_U, \alpha_U)$.
	Then define its interval type to be the index of that near ball.
	Otherwise define its interval type to be the coset of $c - c_U$ of $Q$.
	Denote the space of all the possible branch types $\Bt$.
	We have
	\begin{align*}
		|\Bt| = N_D + \text {number of cosets of $Q$}
	\end{align*}
	depending only on $\Psi$, independent from $B$.
	
\end{Definition}

\begin{Lemma} \label{interval}
	If $c, c'$ are in the same interval and have the same interval type then they have the same tree type.
\end{Lemma}
\begin{proof}
	INSERT PROOF HERE
\end{proof}

\begin{Definition}
	For $c \in \Q_p$ and $\alpha, \beta \in \Z$ let $c \midr [\alpha, \beta] \in \paren{\Z/p\Z}^{\beta - \alpha}$ be the record of coefficients of $c$ for the valuations between $\alpha, \beta$.
	More precisely write $c$ in its power series form
	\begin{align*}
		c = \sum_{\gamma \in Z} c_\gamma p^\gamma \text{ with } c_\gamma \in \Z/p\Z
	\end{align*}
	Then $c \midr [\alpha, \beta]$ is just $(c_\alpha, c_{\alpha+1}, \ldots c_\beta)$.
\end{Definition}
%%%%%%%%%%%%%%%%%%%%%%%%%%%%%%%%

\section{Main Proof}

%%%%%%%%%%%%%%%%%%%%%%%%%%%%%%%%
Fix $\gamma$ corresponding to $\curly{\vec p_i}_{i \in I}$ according to Lemma \ref{gamma}.

\begin{Definition}
	Denote $\Z/p\Z^\gamma$ as $\Ct$.
\end{Definition}

\begin{Definition}
	Let $f: \Q_p^{|x|} \arr \Q_p^I$ with $f(\bar c) = (p_i(\bar c))_{i \in I}$.
	Define the segment space $\Sg$ to be the image of $f$.	
\end{Definition}


Given a tuple $(a_i)_{i\in I}$in the segment space look at the corresponding floors $\curly{F(a_i)}_{i\in I}$.
Those are ordered as elements of $\Z$.
Partition the segment space by order type of $\{F(a_i)\}$.
Work in a fixed partition $\Sg'$.
After relabeling we may assume that
\begin{align*}
	F(a_1) \geq F(a_2) \geq \ldots 
\end{align*}

Consider the (relabeled) sequence of vectors $\vec p_1, \vec p_2, \ldots, \vec p_I$.
There is a unique subset $J \subset I$ such that all vectors with indices in $J$ are linearly independent, and all vectors with indices outside of $J$ are a linear combination of preceding vectors.
For any index $i \in I$ we call it independent if $i \in J$ and we call it dependent otherwise.
%Let $m = |J|$.

Now, we define the following function
\begin{align*}
	g: \Sg' \arr \Bt^I \times \Pt^J \times \Ct^{I - J}
\end{align*}

Let $\bar a = (a_i)_{i\in I} \in \Sg'$.
To define $g(\bar a)$ we need to specify where it maps $\bar a$ in each individual component of the product.

For all $a_i$ record its interval type $\in \Bt$, giving the first component.

For $a_j$ with $j \in J$, record the interval of $a_j$, giving the second component.

For the third component do the following computation.
Pick $a_i$ with $i$ dependent.
Let $j$ be the largest independent index with $j < i$.
Record $a_i \midr [F(a_j) - \gamma, F(a_j)]$.



\begin{Lemma}
	For $\bar a, \bar a' \in \Sg'$ if $g(\bar a) = g(\bar a')$ then $a_i, a_i'$ have the same tree type for all $i \in I$.	
\end{Lemma}

\begin{proof}
	For each $i$ we show that $a_i, a_i'$ are in the same interval and have the same interval type, so the conclusion follows by Lemma \ref{interval}.
	$\Bt$ records the interval type of each element, so if $g(\bar a) = g(\bar a')$ then $a_i, a_i'$ have the same interval type for all $i \in I$.
	Thus it remains to show that $a_i, a_i'$ lie in the same interval for all $i \in I$.
	Suppose $i$ is an independent index.
	Then by construction, $\Pt$ records the interval for $a_i, a_i'$, so those have to belong to the same interval.
	Now suppose $i$ is dependent.
	Pick the largest $j < i$ such that $j$ is independent.
	We have $F(a_i) \leq F(a_j)$ and $F(a_i') \leq F(a_j')$.
	Moreover $F(a_j) = F(a_j')$ as they are mapped to the same interval (using the earlier part of the argument as $j$ is independent).
	
	\begin{Claim}
		$\val(a_i - a_i') > F(a_j) - \gamma$
	\end{Claim}
	\begin{proof}
		Let $\vec x, \vec x' \in \Q_p^{|x|}$ be some elements with
		\begin{align*}
			\vec p_k &\cdot \vec x = a_k \\
			\vec p_k &\cdot \vec x' = a_k' \text { for all } k \in I
		\end{align*}
		It is always possible to do that as $\bar a, \bar a' \in \Sg'$. 
		Let $J'$ be the set of the independent indices less than $i$.
		We have 
		\begin{align*}
			\val(a_k - a_k') > F(a_k) \text { for all } k \in J'
		\end{align*}
		as for the independent indices $a_k, a_k'$ lie in the same interval.
		\begin{align*}
			&\val(a_k - a_k') > F(a_j) \text { for all } k \in J' \text{ by monotonicity of $F(a_k)$} \\
			&\val(\vec p_k \cdot \vec x - \vec p_k \cdot \vec x') > F(a_j) \text { for all } k \in J' \\
			&\val(\vec p_k \cdot (\vec x - \vec x')) > F(a_j) \text { for all } k \in J' \\
		\end{align*}
		$J'$ and $i$ match the requirements of Lemma \ref {gamma} so we conclude
		\begin{align*}
			&\val(\vec p_i \cdot (\vec x - \vec x')) > F(a_j) - \gamma \\
			&\val(\vec p_i \cdot \vec x - \vec p_i \cdot \vec x') > F(a_j) - \gamma \\
			&\val(a_i - a_i')) > F(a_j) - \gamma
		\end{align*}
		as needed, finishing the proof of the claim.
	\end{proof}	
	Additionally $a_i, a_i'$ have the same image in $\Ct$ component, so we have
	\begin{align*}
		\val(a_i - a_i') > F(a_j) 
	\end{align*}
	As $F(a_i) \leq F(a_j)$, $a_i, a_i'$ have to lie in the same interval.	
\end{proof}

\begin{Corollary}
	$\Psi(x,y)$ has VC-density $\leq |x|$
\end{Corollary}

\begin{proof}
	Suppose we have $c, c' \in \Q_p^{|x|}$ such that $f(c), f(c')$ are in the same partition and $g(f(c)) = g(f(c'))$.
	Then by the previous lemma $p_i(c)$ has the same tree type as $p_i(c')$ for all $i\in I$.
	Then by Lemma \ref{sigh} $c, c'$ have the same $\Psi$-type.
	Thus the number of possible $\Psi$-types is bounded by the size of the range of $g$ times the number of possible partitions
	
	\begin{align*}
		\text{(number of partitions)} \cdot |Bt|^{|I|} \cdot |Pt|^{|J|} \cdot |Ct|^{|I-J|}
	\end{align*}
	
	We have
	
	\begin{align*}
		|\Pt| &\leq N \cdot I^2 \text { (the only component dependent on $N$)} \\
		|\Ct| &= p^\gamma 
	\end{align*}
	and there are at most ${|I|}!$ many partitions of $\Sg$. 
	This gives us a bound
	
	\begin{align*}
		{|I|}! \cdot |Bt|^{|I|} \cdot (N \cdot {|I|}^2)^{|J|} \cdot p^{\gamma {|I-J|}} = O(N^{|J|})
	\end{align*}	
	
	Every $p_i$ is an element of a $|x|$-dimensional vector space, so there can be at most $|x|$ many independent vectors.
	Thus we have $|J| \leq |x|$ and the bound follows.
\end{proof}

\begin{Corollary}
	In the language $\LL$ we have $\vc(n) = n$.
\end{Corollary}

\begin{proof}
	Previous lemma implies that $\vc(\phi) \leq \vc(\Psi) \leq |x|$.
	As choice of $\phi$ was arbitrary, this implies that VC-density of any formula is bounded by the arity of $x$.
\end{proof}

\begin{thebibliography}{9}
	\bibitem{density}
		M. Aschenbrenner, A. Dolich, D. Haskell, D. Macpherson, S. Starchenko,
		\textit{Vapnik-Chervonenkis density in some theories without the independence property}, I, preprint (2011)
	\bibitem{reduct}
		insert citation
\end{thebibliography}

\end{document}








