\documentclass{amsart}

\usepackage{../AMC_style}	
\usepackage{../Research}
\usepackage{../Thm}

\usepackage{mathrsfs}



\renewcommand{\AA}{\mathscr A}
  \newcommand{\II}{\mathscr I}
  \newcommand{\MM}{\mathscr M}

  \newcommand{\A}{\mathcal A}
  \newcommand{\B}{\mathcal B}
\renewcommand{\C}{\mathcal C}
  \newcommand{\D}{\mathcal D}
\renewcommand{\H}{\mathcal H}
  \newcommand{\G}{\mathcal G}
  \newcommand{\M}{\mathcal M}

  \newcommand{\U}{\mathcal U}	

  \newcommand{\K}{\boldface K_\alpha}
\renewcommand{\S}{S_\alpha}

\newcommand{\curly}[1]{\left\{#1\right\}}
\newcommand{\paren}[1]{\left(#1\right)}
\newcommand{\abs}[1]{\left|#1\right|}

\providecommand{\floor}[1]{\left \lfloor #1 \right \rfloor }

\DeclareMathOperator{\Sg}{Sg}
\DeclareMathOperator{\Bt}{Bt}
\DeclareMathOperator{\Pt}{Pt}
\DeclareMathOperator{\Ct}{Ct}
\DeclareMathOperator{\vecspan}{span}
\DeclareMathOperator{\val}{val}

\title{reduct}
\author{Anton Bobkov}
\email{bobkov@math.ucla.edu}

\begin{document}

\maketitle

%\begin{Definition}
	%\begin{itemize}
		%\item For $a \in \Q_p$ and $\alpha < \beta \in \Z$ we denote $a \midr [\alpha, \beta]$ the part of the $p$-adic restricted to $[\alpha, \beta]$
		%\item By $a \midr \alpha$ we mean $a \midr (-\infty, \alpha)$.
		%\item By $a = b \ (\midr \alpha)$ we mean $a \midr \alpha = b \midr \alpha$.
	%\end{itemize}
%\end{Definition}
%
%\begin{Claim}
	%\begin{itemize}
		%\item If $a = b \ (\midr \alpha)$ and $c = d \ (\midr \alpha)$ then $a + c = b + d \ (\midr \alpha)$.
		%\item If $a = b \ (\midr \alpha)$ and $v(c) = 0$ then $ca = cb \ (\midr \alpha)$
		%\item More generally if $a = b \ (\midr \alpha)$ then $ca = cb \ (\midr \alpha + v(c))$
	%\end{itemize}
%\end{Claim}

%\begin{Claim}
	%Suppose we have a system 
	%\begin{align*}
		%\curly{ \sum \alpha_{ij} x_i }_{j \in J}
	%\end{align*}
	%with $j \in J$. There exists $\gamma$ with the following property.
	%Pick any subset $J' \subset J$ of equations and $\alpha$. Suppose we are given
	%\begin{align*}
		%\curly{ \sum \alpha_{ij} x_i = c_j \ (\midr \alpha)}_{j \in J'}
	%\end{align*}
	%and suppose those equations are enough to determine $\sum \alpha_{ij'} x_i$ for some $j' \in J$.
	%That is writing $\vec v_j = (\alpha_{1j}, \ldots, \alpha_{Ij})$ we have
	%that $v_{j'}$ is a linear combination of vectors $\{v_j\}_{j \in J'}$
	%Then $\sum \alpha_{ij'} x_i$ is determined up to $\alpha - \gamma$
%\end{Claim}

\begin{Claim}
	Suppose we have a collection of vectors $\curly{\vec p_i}_{i \in I}$ with each $\vec p_i \in \Q_p^m$.
	Pick a subset $J \subset I$ and $j \in I$ such that $\vec p_j \in \vecspan \curly{p_i}_{i \in J}$.
	Suppose we have $\vec x \in \Q_p^m, \alpha \in \Z$ with $\val(\vec p_i \cdot \vec x) > \alpha$ for all $i \in J$.
	Then $\val(\vec p_j \cdot \vec x) > \alpha - \gamma$ for some $\gamma \in \Z^{\geq 0}$.
	Moreover $\gamma$ can be chosen independent of choice of $J, j, \vec x, \alpha$ depending only on $\curly{\vec p_i}_{i \in I}$.
\end{Claim}

We work with a collection of formulas $\Psi(\bar x, \vec y)$ of the form

\begin{align*}
	\vec p_i \cdot \bar x + c_i(\vec y) &\in \lambda_i Q_i \\
	\val(\vec p_i \cdot \bar x + c_i(\bar y)) \; &\square_i \; v_i
\end{align*}

for $i \leq I$ with $|\bar x| = m$ with $Q_i = Q_{n_i, m_i}$ for some $n_i, m_i$.
We work with a parameter set $B$ of size $N$.
Consider a tree given by elements $c_i(b)$ with $b \in B, i \leq I$ with branching points representing levels $v_i$.
This tree has $O(N) = N \cdot I \cdot I$ many branches.

We call $\vec p_i$ segments.

For some $x, x' \in \M$ we say they have the same $\Psi$-type if they have the same $\Psi$ type over $B$.

For some $x, x' \in \M$ we say they have the same $Q$-type if
\begin{itemize}
	\item $x + c_i(b)$ is in the same $Q^i$-coset as $x' + c_i(b)$ for all $i \leq I, b \in B$
	\item $\val(x + c_i(b)) \square_i \; v_i$ iff $\val(x' + c_i(b)) \square_i \; v_i$ for all $i \leq I, b \in B$
\end{itemize}
 
\begin{Lemma}
	$c, c' \in \M^m$ have the same $\Psi$-type if all $p_i(c), p_i(c')$ have the same $Q^i$-type
\end{Lemma}

\begin{Lemma}
	For any $Q = Q_{n,m}$ there exists $\theta_Q$ such that for all $\theta \geq \theta_Q$ the following holds.
	Suppose we have $x,y,c \in \M$ such that
	\begin{align*}
		\val(x-y) - \theta > \val(x-c) = \val(y-c)
	\end{align*}
	Then $x-c$ and $y-c$ lie in the same coset of $Q$.
\end{Lemma}

\begin{Lemma}
	Pick $\theta$ sufficiently large to satisfy previous lemma for all $Q_i$. Define an enumeration of near balls
	\begin{align*}
		B_1(c, \alpha), B_2(c, \alpha), \ldots B_{NB}(c, \alpha)
	\end{align*}
\end{Lemma}

\begin{Definition}
	Let $c \in \M$.
	It lies in our tree between $(c_L, \alpha_L)$ and $(c_U, \alpha_U)$ branching out of $c_B$.
	Suppose $c$ lies in one of the near balls in a branching point above or below it.
	Then define its branch type to be the index of that near ball.
	Otherwise define its branch type to be the coset of $c - c_B$ of $Q_i$ for all $i \in I$.
\end{Definition}

\begin{Lemma}
	If $c, c'$ are in the same branch and have the same branch type then they have the same $Q$-type.
\end{Lemma}

\begin{Definition}
	For $c \in \M$ and $\alpha, \beta \in \Z$ let $c \midr [\alpha, \beta] \in \Z_p^{\beta - \alpha}$ be the record of coefficients of $c$ for valuations between $\alpha, \beta$. More precisely write $c$ in its power series form
	\begin{align*}
		c = \sum_{\gamma \in Z} c_\gamma p^\gamma \text{ with } c_\gamma \in \Z_p
	\end{align*}
	Then $c \midr [\alpha, \beta]$ is just $(c_\alpha, c_{\alpha+1}, \ldots c_\beta)$.
\end{Definition}

For any $c$ define $F(c)$, the floor of $c$ to be the valuation of the largest branching point below $c$.

Let $f: \M^n \arr \M^I$ with $f(\bar c) = (p_i(\bar c))_{i \leq I}$.
Define segment space $\Sg$ to be the image of $f$.

For some element $(a_i)$ in segment space look at floors $F(a_i)$.
Partition the segment space by order type of $\{F(a_i)\}$.
Work in a fixed partition $\Sg'$.
After relabeling we may assume that
\begin{align*}
	F(a_1) \geq F(a_2) \geq \ldots 
\end{align*}

Consider (relabeled) sequence of vectors $\vec p_1, \vec p_2, \ldots, \vec p_I$.
Choose the unique subset of linearly independent vectors $J \subset I$.
For any index $i \in I$ we call it independent if $i \in J$ and we call it dependent otherwise.

For all $a_i$ record its branch type.

For $a_i$ with $i$ independent, record the branch of $a_i$.

Pick $a_i$ with $i$ dependent.
Let $j$ be the largest independent index with $j < i$.
Record $a_i \midr [F(a_j) - \gamma, F(a_j)]$.

Combining all the records defines a function
\begin{align*}
	g: \Sg' \arr \Bt^I \times \Pt^m \times \Ct^I
\end{align*}

We claim that for $\bar a, \bar a' \in \Sg'$ if we have $g(\bar a) = g(\bar a')$ then all $a_i, a_i'$ have the same $Q$-type.

\begin{proof}
	Suppose we have $\bar a, \bar a' \in \Sg'$ that map to the same image by $g$.
	Suppose $i$ is independent.
	Then by construction, $a_i, a_i'$ map to the same branch of the tree and have the same branch type.
	Thus they have the same $Q$-type.
	Otherwise, suppose $i$ is dependent.
	Pick largest $j < i$ such that $j$ is independent.
	We have $F(a_i) \leq F(a_j)$ and $F(a_i') \leq F(a_j')$.
	Moreover $F(a_j) = F(a_j')$ as they are mapped to the same branch (as $j$ is independent).
	
	\begin{Claim}
		$\val(a_i - a_i') > F(a_j) - \gamma$
	\end{Claim}
	\begin{proof}
		Let $\bar x, \bar x' \in \Q_p^m$ be some elements with
		\begin{align*}
			\vec p_k &\cdot \bar x = a_k \\
			\vec p_k &\cdot \bar x' = a_k' \text { for all } k \leq I
		\end{align*}
		Let $J$ be the set of independent indices less than $i$.
		We have 
		\begin{align*}
			\val(a_k - a_k') > F(a_k) \text { for all } k \leq J
		\end{align*}
		as for independent indices $a_k, a_k'$ lie in the same branch.
		\begin{align*}
			&\val(a_k - a_k') > F(a_j) \text { for all } k \leq J \text{ by monotonicity of $F(a_k)$} \\
			&\val(\vec p_k \cdot \bar x - \vec p_k \cdot \bar x') > F(a_j) \text { for all } k \leq J \\
			&\val(\vec p_k \cdot (\bar x - \bar x')) > F(a_j) \text { for all } k \leq J \\
		\end{align*}
		By construction $\vec p_i \in \vecspan\{\vec p_k\}_{k \in J}$. So by the first claim we conclude
		\begin{align*}
			&\val(\vec p_i \cdot (\bar x - \bar x')) > F(a_j) - \gamma \\
			&\val(\vec p_i \cdot \bar x - \vec p_i \cdot \bar x') > F(a_j) - \gamma \\
			&\val(a_i - a_i')) > F(a_j) - \gamma
		\end{align*}
		as needed.
	\end{proof}
	
	So we have $\val(a_i - a_i')) > F(a_j) - \gamma$.
	Moreover we know that
	\begin{align*}
		a_i \midr [F(a_j) - \gamma, F(a_j)] = a_i' \midr [F(a_j) - \gamma, F(a_j)]
	\end{align*}
	This shows that $\val(a_i - a_i')) > F(a_j)$.

	As $F(a_i) \leq F(a_j)$, $a_i, a_i'$ have to lie in the same branch.
	They also agree on branch type.
	Thus they have the same $Q$-type.
\end{proof}

\end{document}








