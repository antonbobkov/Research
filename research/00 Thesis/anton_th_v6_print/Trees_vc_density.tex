\chapter{Trees}
In this \chapa we establish an optimal bound on the VC-density function in the theory of infinite trees.
This generalizes a result of Simon showing that trees are dp-minimal. 
 
We work with trees viewed as posets.
% More precisely, our structure is branches of an infinite tree in a langugage of $\leq$ with possibly finitely many colors.
Parigot in \cite{parigot_trees} showed that such structures have NIP.
This result was strengthened by Simon in \cite{simon_dp_min} showing that trees are dp-minimal.
The paper \cite{density} poses the following problem:

\begin{Problem} (see section 6.3 in \cite{density})
  Determine the VC-density function of each infinite tree.
\end{Problem}

Here we solve this problem by showing that any infinite model of the theory of trees has $\vc(n) = n$ for each $n$.

In Sec\-tion~1 we introduce proper subdivisions -- the main tool that we use to analyze trees.
We also prove some basic properties of proper subdivisions.
Sec\-tion~2 introduces the key constructions of proper subdivisions which will be used in the proof.
Sec\-tion~3 presents the proof of $\vc(n) = n$ via subdivisions.
In the concluding section we state open questions and outline future work.

The language of trees consists of a single binary predicate symbol $\leq$. The theory of trees states that $\leq$ defines a partial order and that for every element $a$ the set $\{x \mid x \leq a\}$ is linearly ordered by $\leq$. For visualization purposes we assume that trees grow upwards, with the smaller elements on the bottom and the larger elements on the top. If $a \leq b$ we will say that $a$ is below $b$ and $b$ is above $a$.

\begin{Definition}
  Work in a tree $\TTB = (T, \leq)$.
  For $x \in T$ let $T^{\leq x} = \{t \in T \mid t \leq x\}$ denote the set of all elements of $T$ below $x$.
  Two elements $a,b$ are said to be in same \defn{connected component} if $T^{\leq a} \cap T^{\leq b}$ is non-empty.
  The \defn{meet} of two elements $a,b$ of $T$ is the greatest element of $T^{\leq a} \cap T^{\leq b}$ (if one exists) and is denoted by $a \wedge b$.
\end{Definition}

The theory of \defn{meet trees} requires that any two elements in the same connected component have a meet. \defn{Colored trees} are trees with a finite number of colors added via unary predicates.

From now on assume that all trees are colored.
We allow our trees to be disconnected (so really, we work with forests) or finite unless otherwise stated.

\section{Proper Subdivisions: Definition and Properties}
We work with finite relational languages.
Given a formula we define its complexity as the depth of quantifiers used to build up the formula. More precisely:
% See for example \cite{ynm_notes} Definition 2D.4 pg.72.
\begin{Definition}
  Define the \defn{complexity} of a formula by induction:
  \begin{align*}
    &\cx(\text{q.f. formula}) = 0 \\
    &\cx(\exists x \phi(x)) = \cx(\phi(x)) + 1 \\
    &\cx(\phi \wedge \psi) = \max(\cx(\phi), \cx(\psi)) \\
    &\cx(\neg \phi) = \cx(\phi)
  \end{align*}
\end{Definition}
A simple inductive argument verifies that there are (up to logical equivalence) only finitely many formulas when the complexity
and free variables are fixed.
We will use the following notation for types:
\begin{Definition} Let $\BT$ be a structure, $A \subseteq B$ be a finite parameter set, $a,b$ be tuples in $\BT$, and $m, n$ be natural numbers.
  \begin{itemize}
  \item $\tp^n_{\BT}(a/A)$ will stand for the set of all $A$-formulas of complexity $\leq n$ that are true of $a$ in $\BT$.
    If $A = \emptyset$ we may also write this as $\tp^n_{\BT}(a)$.
    The subscript $\BT$ will be omitted as well if it is clear from context.
    Note that if $A$ is finite, then there are finitely many such formulas (up to equivalence and renaming free variables).
    The conjunction of finitely many formulas of complexity $\leq n$
    still has complexity $\leq n$ and so we can just associate a single formula with complexity $\leq n$ to every type $\tp^n_{\BT}(a/A)$
    defining the set of realizations of $\tp^n_{\BT}(a/A)$ in $\BT$.
  \item $\BT \models a \equiv^n_A b$ means that $a,b$ have the same type with complexity $\leq n$ over $A$ in $\BT$, i.e., $\tp^n_{\BT}(a/A) = \tp^n_{\BT}(b/A)$.
  \item $S^n_{\BT, m}(A)$ will stand for the set of all $m$-types of complexity $\leq n$ over $A$:
    \begin{align*}
      S^n_{\BT, m}(A) = \curly{\tp^n_{\BT}(a/A) \mid a \in B^m}.
    \end{align*}
  \end{itemize}
\end{Definition}

\begin{Definition} \ 
  \begin{itemize}
  \item Let $\AT$, $\BT$, $\TTB$ be structures in some (possibly different) finite relational languages. If the underlying sets $A, B$ of $\AT, \BT$ partition the underlying set $T$ of $\TTB$ (i.e. $T = A \sqcup B$), then we say that $(\AT, \BT)$ is a \defn{subdivision} of $\TTB$.
  \item A subdivision $(\AT, \BT)$ of $\TTB$ is called \defn{$n$-proper} if given $p,q \in \N$,  $a_1, a_2 \in A^p$ and $b_1, b_2 \in B^q$ with
    \begin{align*}
      \AT \models a_1 &\equiv_n a_2 \\
      \BT \models b_1 &\equiv_n b_2
    \end{align*}
    we have
    \begin{align*}
      \TTB \models a_1b_1 \equiv_n a_2b_2.
    \end{align*}
  \item A subdivision $(\AT, \BT)$ of $\TTB$ is called \defn{proper} if it is $n$-proper for all $n \in \N$.
  \end{itemize}
\end{Definition}

\begin{Lemma} \label{lm_subdivision}
  Consider a subdivision $(\AT, \BT)$ of $\TTB$. If $(\AT, \BT)$ is $0$-proper then it is proper.
\end{Lemma}

\begin{proof}
  We prove that the subdivision $(\AT, \BT)$ is $n$-proper for all $n$ by induction.
  The case $n = 0$ is given by the assumption.
  Suppose we have $\TTB \models \exists x \, \phi^n(x, a_1, b_1)$ where $\phi^n$ is some formula of complexity $n$. Let $a \in T$ witness the existential claim, i.e., $\TTB \models \phi^n(a, a_1, b_1)$. We can have $a \in A$ or $a \in B$. Without loss of generality assume $a \in A$. Let $\pp = \tp^n_{\AT} (a, a_1)$. Then we have 
  \begin{align*}
    \AT \models \exists x \, \tp^n_{\AT}(x, a_1) = \pp
  \end{align*}
  (with $\tp^n_{\AT}(x, a_1) = \pp$ a shorthand for $\phi_{\pp}(x)$ where $\phi_{\pp}$ is a formula that determines the type $\pp$).
  The formula $\tp^n_{\AT}(x, a_1) = \pp$ is of complexity $\leq n$ so $\exists x \, \tp^n_{\AT}(x, a_1) = \pp$ is of complexity $\leq n+1$. By the inductive hypothesis we have
  \begin{align*}
    \AT \models \exists x \, \tp^n_{\AT}(x, a_2) = \pp.
  \end{align*}
  Let $a'$ witness this existential claim, so that $\tp^n_{\AT}(a', a_2) = \pp$, hence $\tp^n_{\AT}(a', a_2) = \tp^n_{\AT}(a, a_1)$, that is,
  $\AT \models a'a_2 \equiv_n aa_1$. By the inductive hypothesis we therefore have
  $\TTB \models aa_1b_1 \equiv_n a'a_2b_2$; in particular $\TTB \models \phi^n(a', a_2, b_2)  \text { as } \TTB \models \phi^n(a, a_1, b_1)$,
  and $\TTB \models \exists x \phi^n(x, a_2, b_2)$.
\end{proof}

This lemma is general, but we will use it specifically applied to (colored) trees.
Suppose $\TTB$ is a (colored) tree in some language $\LL = \{\leq, \ldots\}$ expanding the language of trees by finitely many predicate symbols.
Suppose $\AT, \BT$ are some structures in languages $\LL_A, \LL_B$ which expand $\LL$, with the $\LL$-reducts of $\AT, \BT$ substructures of $\TTB$.
Furthermore suppose that $(\AT, \BT)$ is 0-proper.
Then by the previous lemma $(\AT, \BT)$ is a proper subdivision of $\TTB$.
From now on all the subdivisions we work with will be of this form.

\begin{Example} \label{ex_disc}
  Suppose a tree consists of two connected components $C_1, C_2$.
  Then those components $(C_1, \leq)$, $(C_2, \leq)$ viewed as substructures form a proper subdivision.
  To see this we only need to show that this subdivision is 0-proper.
  But that is immediate as any $c_1 \in C_1$ and $c_2 \in C_2$ are incomparable.
\end{Example}

\begin{Example} \label{ex_cone}
  Fix a tree $\TTB$ in the language $\{\leq\}$, and let $a \in T$. Let $B = \{t \in T \mid a < t\}$, $S = \{t \in T \mid t \leq a\}$, $A = T - B$. Then $(A, \leq, S)$ and $(B, \leq)$ form a proper subdivision, where $\LL_A$ has a unary predicate interpreted by $S$.
  To see this, again, we show that the subdivision is 0-proper.
  The only time $a \in A$ and $b \in B$ are comparable is when $a \in S$, and this is captured by the language.
  (See proof of Lemma \ref{subdivide} for more details.)
\end{Example}

\begin{Definition} For $\phi(x, y)$, $A \subseteq T^{|x|}$ and $B \subseteq T^{|y|}$
  \begin{itemize}
  \item let $\phi(A, b) = \{a \in A \mid \phi(a, b)\} \subseteq A$, and 
  \item let $\phi(A, B) = \{\phi(A, b) \mid b \in B\} \subseteq \PP(A)$.	
  \end{itemize}
\end{Definition}
Thus $\phi(A, B)$ is a collection of subsets of $A$ definable by $\phi$ with parameters from $B$. We notice the following bound when $A, B$ are parts of a proper subdivision.

\begin{Corollary} \label{cor_type_count}
  Let $\AT, \BT$ be a proper subdivision of $\TTB$ and $\phi(x,y)$ be a formula of complexity $n$. Then $|\phi(A^{|x|}, B^{|y|})| \leq |S^n_{\BT, |y|}|$.
\end{Corollary}

\begin{proof}
  Take some $a \in A^{|x|}$ and $b_1, b_2 \in B^{|y|}$ with $\tp^n_{\BT}(b_1) = \tp^n_{\BT}(b_2)$. We have $\BT \models b_1 \equiv_n b_2$ and (trivially) $\AT \models a \equiv_n a$. Thus  we have $\TTB \models ab_1 \equiv_n ab_2$, so $\TTB \models \phi(a, b_1) \leftrightarrow \phi(a, b_2)$. Since $a$ was arbitrary we have $\phi(A^{|x|}, b_1) = \phi(A^{|x|}, b_2)$ as different traces can only come from parameters of different types. Thus $|\phi(A^{|x|}, B^{|y|}) \leq |S^n_{\BT, |y|}|$.
\end{proof}

We note that the size of the type space $|S^n_{\BT, |y|}|$ can be bounded uniformly:

\begin{Definition} \label{def_type_count}
  Fix a (finite relational) language $\LL_B$. Let $N = N(n, m, \LL_B)$ be smallest integer such that for any structure $\BT$ in $\LL_B$ we have $|S^n_{\BT, m}| \leq N$. This integer exists as there is a finite number (up to logical equivalence) of possible formulas of complexity $\leq n$ with free variables $x_1,x_2, \ldots, x_m$.
  Note that $N(n, m, \LL_B)$ is increasing in all parameters:
  \begin{align*}
    n \leq n', m \leq m', \LL_B \subseteq \LL_B' \Rightarrow N(n, m, \LL_B) \leq N(n', m', \LL_B').
  \end{align*}
\end{Definition}

\section{Proper Subdivisions: Constructions}
Throughout this section, $\TTB = (T, \leq, C_1, \ldots, C_n)$ denotes a colored meet tree.
First, we describe several constructions of proper subdivisions that are needed for the proof of our main theorem. 
\begin{Definition}
  We use $E(b,c)$ to express that $b$ and $c$ are in the same connected component of $T$:
  \begin{align*}
    E(b, c) \ifff \exists x \, (b \geq x) \wedge (c \geq x).
  \end{align*}
\end{Definition}
\begin{Definition}
  Given an element $a$ of the tree $\TTB$ we call the set of all elements strictly above $a$, i.e., the set $T^{> a} = \{x \mid x > a\}$,
  the \defn{open cone} above $a$.
  Connected components of that cone can be thought of as \defn{closed cones} above $a$.
  With that interpretation in mind, the notation $E_a(b, c)$ means that $b$ and $c$ are in the same closed cone above $a$. More formally:
  \begin{align*}
    E_a(b, c) \ifff E(b,c) \text{ and } (b \wedge c) > a.
  \end{align*}
\end{Definition}

Fix a language for our colored tree $\LL = \{\leq, C_1, \ldots C_n\}$. Put  $\vec C = \curly{C_1, \ldots C_n}$.
Given $A \subseteq T$ we put $\vec C \cap A = \curly{C_1 \cap A, \ldots, C_n \cap A}$.
In that case $(A, \leq, \vec C \cap A)$ is a substructure of $\TTB$ (here as usual by abuse of notation $\leq$ denotes the restriction of the ordering $\leq$ of $T$ to $A$.)
For the following four definitions fix an expansion $\LLU = \LL \cup \{U\}$ of the language $\LL$ by a unary predicate symbol $U$.
We refer to the figures below for illustrations of these definitions.
% It is not always necessary to have this predicate but we keep it for the sake of uniformity.
% Structures denoted by $\AT$ in defintion below will have different languages $\LL_A$ (those are not as important in later applications).
% All the colors $\vec C$ are interpreted by colors in $\TTB$ by restriction.


	\tikzstyle{node}=[circle, draw]

	\tikzstyle{up}=[node, fill = white]
	\tikzstyle{c1}=[node, fill = white]
	\tikzstyle{md}=[node, fill = lightgray]
	\tikzstyle{c2}=[node, fill = white]
	\tikzstyle{dn}=[node, fill = white]
	\tikzstyle{ds}=[node, fill = white]
	\tikzstyle{ex}=[node, fill = white]
	\tikzstyle{nd}=[rectangle, draw]

	\begin{figure}[p]
		\begin{tikzpicture}
	\node[up] {}
		child[grow = south, level distance=10mm] {node[up] {}
			child[grow = south west, level distance=10mm]{node[up]{}	%left up
				[sibling distance=5mm]
				child{node[up]{}}
				child{node[up]{}}
			}
			child[level distance=15mm]{node[c1]{$c_1$} %main up
				[sibling distance=5mm]
				child [grow = -120, level distance=12mm] {node[md]{}	%1
					child[grow = -120]{node[md]{} %2
						child[grow = -150]{node[md]{}
							child{node[md]{}}
							child{node[md]{}}
						}
						child[grow = -120]{node[c2]{$c_2$} %3
							[grow = south]
							child{node[dn]{}}
							child{node[dn]{}
								child{node[dn]{}}
								child{node[dn]{}}
							}
						}
					}
					child[grow = south]{node[md]{}
						child[grow = -60]{node[md]{}}
					}
				}
				child[grow = -30, level distance=10mm]{node[ds]{}	%aux1 middle
					[sibling distance=5mm]
					child{node[ds]{}}
					child{node[ds]{}}
				}
				child [grow = -60, level distance=10mm] {node[ds]{} 	%aux2 middle
					[sibling distance=5mm]
					child{node[ds]{}}
					child{node[ds]{}}
				}
			}
			child [grow = south east, level distance=10mm] {node[up]{}	%right up
				[sibling distance=5mm]
				child{node[up]{}}
				child{node[up]{}}
			}
		}
		child[grow = east, level distance=40mm, white]{node[black, ex]{}
			[grow = south]
			[level distance=10mm]
			child[black]{node[ex]{}
				[sibling distance=5mm]
				child{node[ex]{}}
				child{node[ex]{}}
			}
			child[black] {node[ex]{}
				[sibling distance=5mm]
				child{node[ex]{}}
				child{node[ex]{}}
			}
		}
		;
		\draw [black, fill=white] (70mm,0) circle [radius=2mm];
		\node [right] at (72mm,0) {$A$};
		\draw [black, fill=lightgray] (70mm,-10mm) circle [radius=2mm]; 
		\node [right] at (72mm,-10mm) {$B$};
\end{tikzpicture}

		\caption{Proper subdivision  $(\AT, \BT) = (\AT^{c_1}_{c_2}, \BT^{c_1}_{c_2})$}
	\end{figure}
	
	\tikzstyle{up}=[node, fill = lightgray]
	\tikzstyle{c1}=[node, fill = white]
	\tikzstyle{md}=[node, fill = white]
	\tikzstyle{c2}=[node, fill = white]
	\tikzstyle{dn}=[node, fill = white]
	\tikzstyle{ds}=[node, fill = white]
	\tikzstyle{ex}=[node, fill = white]
	\tikzstyle{nd}=[rectangle, draw]

	\begin{figure}[p]
		\begin{tikzpicture}
	\node[up] {}
		child[grow = south, level distance=10mm] {node[up] {}
			child[grow = south west, level distance=10mm]{node[up]{}	%left up
				[sibling distance=5mm]
				child{node[up]{}}
				child{node[up]{}}
			}
			child[level distance=15mm]{node[c1]{$c_1$} %main up
				[sibling distance=5mm]
				child [grow = -120, level distance=12mm] {node[md]{}	%1
					child[grow = -120]{node[md]{} %2
						child[grow = -150]{node[md]{}
							child{node[md]{}}
							child{node[md]{}}
						}
						child[grow = -120]{node[c2]{$c_2$} %3
							[grow = south]
							child{node[dn]{}}
							child{node[dn]{}
								child{node[dn]{}}
								child{node[dn]{}}
							}
						}
					}
					child[grow = south]{node[md]{}
						child[grow = -60]{node[md]{}}
					}
				}
				child[grow = -30, level distance=10mm]{node[ds]{}	%aux1 middle
					[sibling distance=5mm]
					child{node[ds]{}}
					child{node[ds]{}}
				}
				child [grow = -60, level distance=10mm] {node[ds]{} 	%aux2 middle
					[sibling distance=5mm]
					child{node[ds]{}}
					child{node[ds]{}}
				}
			}
			child [grow = south east, level distance=10mm] {node[up]{}	%right up
				[sibling distance=5mm]
				child{node[up]{}}
				child{node[up]{}}
			}
		}
		child[grow = east, level distance=40mm, white]{node[black, ex]{}
			[grow = south]
			[level distance=10mm]
			child[black]{node[ex]{}
				[sibling distance=5mm]
				child{node[ex]{}}
				child{node[ex]{}}
			}
			child[black] {node[ex]{}
				[sibling distance=5mm]
				child{node[ex]{}}
				child{node[ex]{}}
			}
		}
		;
		\draw [black, fill=white] (70mm,0) circle [radius=2mm];
		\node [right] at (72mm,0) {$A$};
		\draw [black, fill=lightgray] (70mm,-10mm) circle [radius=2mm]; 
		\node [right] at (72mm,-10mm) {$B$};
\end{tikzpicture}

		\caption{Proper subdivision  $(\AT, \BT) = (\AT_{c_1}, \BT_{c_1})$}
	\end{figure}

	\tikzstyle{up}=[node, fill = white]
	\tikzstyle{c1}=[node, fill = white]
	\tikzstyle{md}=[node, fill = white]
	\tikzstyle{c2}=[node, fill = white]
	\tikzstyle{dn}=[node, fill = white]
	\tikzstyle{ds}=[node, fill = lightgray]
	\tikzstyle{ex}=[node, fill = white]
	\tikzstyle{nd}=[rectangle, draw]

	\begin{figure}[p]
		\begin{tikzpicture}
	\node[up] {}
		child[grow = south, level distance=10mm] {node[up] {}
			child[grow = south west, level distance=10mm]{node[up]{}	%left up
				[sibling distance=5mm]
				child{node[up]{}}
				child{node[up]{}}
			}
			child[level distance=15mm]{node[c1]{$c_1$} %main up
				[sibling distance=5mm]
				child [grow = -120, level distance=12mm] {node[md]{}	%1
					child[grow = -120]{node[md]{} %2
						child[grow = -150]{node[md]{}
							child{node[md]{}}
							child{node[md]{}}
						}
						child[grow = -120]{node[c2]{$c_2$} %3
							[grow = south]
							child{node[dn]{}}
							child{node[dn]{}
								child{node[dn]{}}
								child{node[dn]{}}
							}
						}
					}
					child[grow = south]{node[md]{}
						child[grow = -60]{node[md]{}}
					}
				}
				child[grow = -30, level distance=10mm]{node[ds]{}	%aux1 middle
					[sibling distance=5mm]
					child{node[ds]{}}
					child{node[ds]{}}
				}
				child [grow = -60, level distance=10mm] {node[ds]{} 	%aux2 middle
					[sibling distance=5mm]
					child{node[ds]{}}
					child{node[ds]{}}
				}
			}
			child [grow = south east, level distance=10mm] {node[up]{}	%right up
				[sibling distance=5mm]
				child{node[up]{}}
				child{node[up]{}}
			}
		}
		child[grow = east, level distance=40mm, white]{node[black, ex]{}
			[grow = south]
			[level distance=10mm]
			child[black]{node[ex]{}
				[sibling distance=5mm]
				child{node[ex]{}}
				child{node[ex]{}}
			}
			child[black] {node[ex]{}
				[sibling distance=5mm]
				child{node[ex]{}}
				child{node[ex]{}}
			}
		}
		;
		\draw [black, fill=white] (70mm,0) circle [radius=2mm];
		\node [right] at (72mm,0) {$A$};
		\draw [black, fill=lightgray] (70mm,-10mm) circle [radius=2mm]; 
		\node [right] at (72mm,-10mm) {$B$};
\end{tikzpicture}

		\caption{Proper subdivision  $(\AT, \BT) = (\AT^{c_1}_S, \BT^{c_1}_S)$ for $S = \{c_2\}$}
	\end{figure}

	\tikzstyle{up}=[node, fill = white]
	\tikzstyle{c1}=[node, fill = white]
	\tikzstyle{md}=[node, fill = white]
	\tikzstyle{c2}=[node, fill = white]
	\tikzstyle{dn}=[node, fill = white]
	\tikzstyle{ds}=[node, fill = white]
	\tikzstyle{ex}=[node, fill = lightgray]
	\tikzstyle{nd}=[rectangle, draw]

	\begin{figure}[p]
		\begin{tikzpicture}
	\node[up] {}
		child[grow = south, level distance=10mm] {node[up] {}
			child[grow = south west, level distance=10mm]{node[up]{}	%left up
				[sibling distance=5mm]
				child{node[up]{}}
				child{node[up]{}}
			}
			child[level distance=15mm]{node[c1]{$c_1$} %main up
				[sibling distance=5mm]
				child [grow = -120, level distance=12mm] {node[md]{}	%1
					child[grow = -120]{node[md]{} %2
						child[grow = -150]{node[md]{}
							child{node[md]{}}
							child{node[md]{}}
						}
						child[grow = -120]{node[c2]{$c_2$} %3
							[grow = south]
							child{node[dn]{}}
							child{node[dn]{}
								child{node[dn]{}}
								child{node[dn]{}}
							}
						}
					}
					child[grow = south]{node[md]{}
						child[grow = -60]{node[md]{}}
					}
				}
				child[grow = -30, level distance=10mm]{node[ds]{}	%aux1 middle
					[sibling distance=5mm]
					child{node[ds]{}}
					child{node[ds]{}}
				}
				child [grow = -60, level distance=10mm] {node[ds]{} 	%aux2 middle
					[sibling distance=5mm]
					child{node[ds]{}}
					child{node[ds]{}}
				}
			}
			child [grow = south east, level distance=10mm] {node[up]{}	%right up
				[sibling distance=5mm]
				child{node[up]{}}
				child{node[up]{}}
			}
		}
		child[grow = east, level distance=40mm, white]{node[black, ex]{}
			[grow = south]
			[level distance=10mm]
			child[black]{node[ex]{}
				[sibling distance=5mm]
				child{node[ex]{}}
				child{node[ex]{}}
			}
			child[black] {node[ex]{}
				[sibling distance=5mm]
				child{node[ex]{}}
				child{node[ex]{}}
			}
		}
		;
		\draw [black, fill=white] (70mm,0) circle [radius=2mm];
		\node [right] at (72mm,0) {$A$};
		\draw [black, fill=lightgray] (70mm,-10mm) circle [radius=2mm]; 
		\node [right] at (72mm,-10mm) {$B$};
\end{tikzpicture}

		\caption{Proper subdivision  $(\AT, \BT) = (\AT_S, \BT_S)$ for $S = \{c_1, c_2\}$}
	\end{figure}


\begin{Definition}
  Fix $c_1 < c_2$ in $T$. Let
  \begin{align*}
    B &= \{b \in T \mid E_{c_1}(c_2, b) \wedge \neg(b \geq c_2)\}, \\
    A &= T - B, \\
    T^{<c_1} &= \{t \in T \mid t < c_1\}, \\
    T^{<c_2} &= \{t \in T \mid t < c_2\}, \\
    S_B &= T^{<c_2} - T^{<c_1}, \\
    T^{\geq c_2} &= \{t \in T \mid c_2 \leq t\}.
  \end{align*}
  Define structures $\AT^{c_1}_{c_2} = (A, \leq, \vec C \cap A, T^{<c_1}, T^{\geq c_2})$ in a language expanding $\LL$ by two unary predicate symbols and
  $\BT^{c_1}_{c_2} = (B, \leq, \vec C \cap B, S_B)$ in the language $\LLU$ (as defined above).
  Note that $c_1, c_2 \notin B$.
\end{Definition}


\begin{Definition}
  Fix $c$ in $T$. Let
  \begin{align*}
    B &= \{b \in T \mid \neg(b \geq c) \wedge E(b,c)\}, \\
    A &= T - B, \\
    T^{<c_1} &= \{t \in T \mid t < c\}.
  \end{align*}
  Define structures $\AT_{c} = (A, \leq, \vec C \cap A)$ in the language $\LL$ and $\BT_{c} = (B, \leq, \vec C\cap B, T^{<c_1})$ in the language $\LLU$.
  Note that $c \notin B$. (cf. example \ref{ex_cone}).
\end{Definition}

\begin{Definition}
  Fix $c$ in $T$ and  a finite subset $S \subseteq T$. Let
  \begin{align*}
    B &= \{b \in T \mid (b > c) \text{ and for all $s \in S$ we have } \neg E_c(s, b)\}, \\
    A &= T - B, \\
    T^{\leq c_1} &= \{t \in T \mid t \leq c\}.
  \end{align*}
  Define structures $\AT^{c}_{S} = (A, \leq, \vec C\cap A, T^{\leq c_1})$ and $\BT^{c}_{S} = (B, \leq, \vec C\cap B, B)$ both in the language $\LLU$.
  Note that $c \notin B$ and $S \cap B = \emptyset$.
\end{Definition}

\begin{Definition}
  Fix  a finite subset $S \subseteq T$. Let
  \begin{align*}
    B &= \{b \in T \mid \text{ for all $s \in S$ we have } \neg E(s, b)\}, \\
    A &= T - B.
  \end{align*}
  Define structures $\AT_{S} = (A, \leq, \vec C \cap A)$ in the language $\LL$ and $\BT_{S} = (B, \leq, \vec C\cap B, B)$ in the language $\LLU$.
  (cf. example \ref{ex_disc})
\end{Definition}

Note that we forced the structures $\BT^{c_1}_{c_2}, \BT_c, \BT^{c}_{S}, \BT_{S}$ to have the same language $\LLU$.
This is done for uniformity to simplify Lemma \ref{lm_partition_bound}.
By comparison, the corresponding structures denoted by $\AT$ with decorations have different languages.

\begin{Lemma} \label{subdivide}
  The pairs of structures defined above are all proper subdivisions of $\TTB$.
\end{Lemma}

\begin{proof}
  We only show this holds for the pair $(\AT, \BT) = (\AT^{c_1}_{c_2} ,\BT^{c_1}_{c_2})$.
  The other cases follow by a similar argument.
  The sets $A,B$ partition $T$ by definition, so $(A,B)$ is a subdivision of $\TTB$.
  To show that it is proper, by Lemma \ref{lm_subdivision} we only need to check that it is $0$-proper. Suppose we have
  \begin{align*}
    a &= (a_1, a_2, \ldots, a_p) \in A^p, \\
    a' &= (a_1', a_2', \ldots, a_p') \in A^p,  \\
    b &= (b_1, b_2, \ldots, b_q) \in B^q,  \\
    b' &= (b_1', b_2', \ldots, b_q') \in B^q, 
  \end{align*}
  with $\AT \models a \equiv_0 a'$ and $\BT \models b \equiv_0 b'$.
  We need to show that $ab$ has the same quantifier-free type in $\TTB$ as $a'b'$.
  Any two elements in $T$ can be related in the four following ways:
  \begin{align*}
    x &= y, \\
    x &< y, \\
    x &> y, \text{ or } \\
    x&,y \text{ are incomparable.}
  \end{align*}
  We need to check that  for all $i,j$ the same relations hold for $(a_i, b_j)$ as do for $(a_i', b_j')$.
  
  \begin{itemize}
  \item It is impossible that $a_i = b_j$ as they come from disjoint sets.
  \item Suppose $a_i < b_j$. This forces $a_i \in T^{<c_1}$ thus $a_i' \in T^{<c_1}$ and $a_i' < b_j'$.
  \item Suppose $a_i > b_j$. This forces $b_j \in S_B$ and $a \in T^{\geq c_2}$, thus $b_j' \in S_B$ and $a_i' \in T^{\geq c_2}$, so $a_i' > b_j'$.
  \item Suppose $a_i$ and $b_j$ are incomparable. Two cases are possible:
    \begin{itemize}
    \item $b_j \notin S_B$ and $a_i \in T^{\geq c_2}$. Then $b_j' \notin S_B$ and $a_i' \in T^{\geq c_2}$ making $a_i', b_j'$ incomparable.
    \item $b_j \in S_B$, $a_i \notin T^{\geq c_2}$, $a_i \notin T^{<c_1}$. Similarly this forces $a_i', b_j'$ to be incomparable.
    \end{itemize}
  \end{itemize}
  Also we need to check that $ab$ has the same colors as $a'b'$. But that is immediate as having the same color in a substructure means having the same color in the tree.
\end{proof}

\section{Main proof}
The basic idea for the proof is as follows.
Suppose we have a formula with $q$ parameters over a parameter set of size $n$.
We are able to split our parameter space into $O(n)$ many sets
by designating subdivisions described in the previous section based on the parameter set
(see figures in the previous section for an example of a partition of the tree into four sets using parameter set $\curly{c_1, c_2}$).
Each of the $q$ parameters can come from any of those $O(n)$ components giving us $O(n)^q$ many choices for parameter configuration.
When every parameter is coming from a fixed component, the number of definable sets is constant and in fact is uniformly bounded above by some $N$
(independent of the parameter set).
This gives us at most $N \cdot O(n)^q$ possibilities for different definable sets.

First, we generalize Corollary \ref{cor_type_count}.
(This is required for computing VC-density of formulas $\phi(x, y)$ with $|y| > 1$).

\begin{Lemma} \label{lm_partition_bound}
  Consider a finite collection $(\AT_i, \BT_i)_{i \leq n}$ satisfying the following properties:
  \begin{itemize}
  \item $(\AT_i, \BT_i)$ is either a proper subdivision of $\TTB$ or $A_i = T$ and $B_i = \{b_i\}$,
  \item all $\BT_i$ have the same language $\LLU$, and 
  \item the sets $\curly{B_i}_{i \leq n}$ are pairwise disjoint.
  \end{itemize}
  Let $A = \bigcap_{i \in I} A_i$.
  Fix a formula $\phi(x, y)$ of complexity $m$ . Let $N = N(m, |y|, \LLU)$ be as in Definition \ref{def_type_count}.
  Let $B = B_1^{i_1} \times B_2^{i_2} \times \ldots \times B_n^{i_n} \subseteq T^{|y|}$ where $i_1 + i_2 + \ldots + i_n = |y|$
  (some of the indices can be zero). Then we have the following bound:
  \begin{align*}
    \phi(A^{|x|}, B) \leq N^{|y|}.
  \end{align*}
\end{Lemma}

\begin{proof}
  We show this result by counting types.
  \begin{Claim}
    Suppose we have
    \begin{align*}
      b_1, b_1' &\in B_1^{i_1} \text{ with } b_1 \equiv_m b_1' \text { in } \BT_1, \\
      b_2, b_2' &\in B_2^{i_2} \text{ with } b_2 \equiv_m b_2' \text { in } \BT_2, \\
                &\cdots \\
      b_n, b_n' &\in B_n^{i_n} \text{ with } b_n \equiv_m b_n' \text { in } \BT_n.
    \end{align*}
    Then
    \begin{align*}
      \phi(A^{|x|}, b_1, b_2, \ldots b_n) \iff \phi(A^{|x|}, b_1', b_2', \ldots b_n').
    \end{align*}
  \end{Claim}
  \begin{proof}
    Define $\bar b_i = (b_1, \ldots, b_i, b_{i+1}', \ldots, b_n') \in B$ for $i = 0, \ldots, n$.
    We have
    \begin{align*}      
      \phi(A^{|x|}, \bar b_i) \iff \phi(A^{|x|}, \bar b_{i+1}),
    \end{align*}
    as either $(\AT_{i+1}, \BT_{i+1})$ is $m$-proper
    or $\BT_{i+1}$ is a singleton, and the implication is trivial.
    (Notice that $b_i \in \AT_j$ for $j \neq i$ by disjointness assumption.)
    Thus, by induction we get $\phi(A^{|x|}, \bar b_0) \iff \phi(A^{|x|}, \bar b_n)$ as needed.
  \end{proof}
  Thus $\phi(A^{|x|}, B)$ only depends on the choice of the types for the tuples:
  \begin{align*}
    |\phi(A^{|x|}, B)| \leq |S^m_{\BT_1, i_1}| \cdot |S^m_{\BT_2, i_2}| \cdot \ldots \cdot |S^m_{\BT_n, i_n}|
  \end{align*}
  Now for each type space we have an inequality
  \begin{align*}
    |S^m_{\BT_j, i_j}| \leq N(m, i_j, \LLU) \leq N(m, |y|, \LLU) \leq N.
  \end{align*}
  (For singletons $|S^m_{\BT_j, i_j}| = 1 \leq N$). Only non-zero indices contribute to the product and there are at most $|y|$ of those (by the equality $i_1 + i_2 + \ldots + i_n = |y|$). Thus we have
  \begin{align*}
    |\phi(A^{|x|}, B)| \leq N^{|y|}
  \end{align*}
  as needed.
\end{proof}

For subdivisions to work out properly, we will need to pass to subsets closed under meets. We observe that the closure under meets doesn't add too many new elements:
\begin{Lemma} \label{lm_meet}
  Suppose $S \subseteq T$ is a finite subset of size $n \geq 1$ in a meet tree and $S'$ is its closure under meets. Then $|S'| \leq 2n - 1$.
\end{Lemma}
\begin{proof}
  We can partition $S$ into connected components and prove the result separately for each component. Thus we may assume all elements of $S$ lie in the same connected component. We prove the claim by induction on $n$. The base case $n = 1$ is clear. Suppose we have $S$ of size $k$ with closure of size at most $2k - 1$. Take a new point $s$, and look at its meets with all the elements of $S$.
  Those are linearly ordered, so we can pick the smallest one, $s'$.
  Then $S \cup \{s, s'\}$ is closed under meets.
\end{proof}

Putting all of those results together we are able to compute the VC-density of formulas in meet trees:
\begin{Theorem}
  Let $\TTB$ be an infinite (colored) meet tree and $\phi(x, y)$ a formula with $|x| = p$ and $|y| = q$. Then $\vc(\phi) \leq q$.
\end{Theorem}

\begin{proof}
  Pick a finite subset of $S_0 \subset T^p$ of size $n$. Let $S_1 \subset T$ consist of the components of the elements of $S_0$. Let $S \subset T$ be the closure of $S_1$ under meets. Using Lemma \ref{lm_meet} we have $|S| \leq 2|S_1| \leq 2p|S_0| = 2pn = O(n)$. We have $S_0 \subseteq S^p$, so $|\phi(S_0, T^q)| \leq |\phi(S^p, T^q)|$. Thus it is enough to show $|\phi(S^p, T^q)| = O(n^q)$.
  
  Label $S = \{c_i\}_{i \in I}$ with $|I| \leq 2pn$. For every $c_i$ we construct two subdivisions in the following way. We have that $c_i$ is either minimal in $S$ or it has a predecessor in $S$ (greatest element less than $c$). If it is minimal, construct $(\AT_{c_i}, \BT_{c_i})$. If there is a predecessor $p$, construct $(\AT^p_{c_i}, \BT^p_{c_i})$. For the second subdivision let $G$ be all the elements in $S$ greater than $c_i$ and construct $(\AT^c_G, \BT^c_G)$. So far we have constructed two subdivisions for every $i \in I$. Additionally construct $(\AT_S, \BT_S)$. We end up with a finite collection of proper subdivisions $(\AT_j, \BT_j)_{j \in J}$ with $|J| = 2|I| + 1$. Before we proceed, we note the following two lemmas describing our subdivisions.
  
  \begin{Lemma}
    For all $j \in J$ we have $S \subseteq A_j$. Thus $S \subseteq \bigcap_{j \in J} A_j$ and $S^p \subseteq \bigcap_{j \in J} (A_j)^p$. 
  \end{Lemma}
  
  \begin{proof}
    Check this for each possible choice of subdivision. Cases for subdivisions of the type $\AT_S, \AT^c_G, \AT_c$ are easy.
    Suppose we have a subdivision $(\AT, \BT) = (\AT^{c_1}_{c_2}, \BT^{c_1}_{c_2})$.
    We need to show that $B \cap S = \emptyset$.
    By construction we have $c_1, c_2 \notin B$.
    Suppose we have some other $c \in S$ with $c \in B$.
    We have $E_{c_1}(c_2, c)$, i.e., there is some $b$ such that $b > c_1$, $b \leq c_2$ and $b \leq c$.
    Consider the meet $c \wedge c_2$.
    We have $c \wedge c_2 \geq b > c_1$.
    Also as $\neg (c \geq c_2)$ we have $c \wedge c_2 < c_2$.
    To summarize: $c_2 > c \wedge c_2 > c_1$.
    But this contradicts our construction as $S$ is closed under meets,
    so $c \wedge c_2 \in S$ and $c_1$ is supposed to be a predecessor of $c_2$ in $S$.
  \end{proof}
  
  \begin{Lemma}
    $\{B_j\}_{j \in J}$ is a partition of $T - S$, i.e., $T = \bigsqcup_{j \in J} B_j \sqcup S$.
  \end{Lemma}
  
  \begin{proof}
    This more or less follows from the choice of subdivisions. Pick any $b \in T - S$.
    Let $a$ be the minimal element of $S$ with $a > b$, and let $c$ be the maximal element of $S$ with $c < b$ (if such elements exist).
    %Take all the elements in $S$ greater than $b$ and take the minimal one $a$. Take all the elements in $S$ less than $b$ and take the maximal one $c$ (possible as $S$ is closed under meets).
    Also let $G$ be the set of elements of $S$ incomparable to $b$.
    If both $a$ and $c$ exist we have $b \in \BT^a_c$. If only the upper bound exists we have $b \in \BT^a_G$. If only the lower bound exists we have $b \in \BT_c$. If neither exists we have $b \in \BT_G$.
  \end{proof}
  Note that those two lemmas imply $S = \bigcap_{j \in J} A_j$.
  
  % careful application of note - have different languages and has to be > 1 
  For the one-dimensional case $q = 1$ we don't need to do any more work. We have partitioned the parameter space into $|J| = O(n)$ many pieces and over each piece the number of definable sets is uniformly bounded. By Corollary \ref{cor_type_count} we have that $|\phi((A_j)^p, B_j)| \leq N$ for any $j \in J$ (letting $N = N(n_\phi, q, \LL \cup \{S\})$ where $n_\phi$ is the complexity of $\phi$ and $S$ is a unary predicate). Compute
  % describe steps
  \begin{align*}
    |\phi(S^p, T)|
    &= \left|\bigcup_{j \in J} \phi(S^p, B_j) \cup \phi(S^p, S)\right| \leq \\
    &\leq \sum_{j \in J} |\phi(S^p, B_j)| + |\phi(S^p, S)| \leq \\
    &\leq \sum_{j \in J} |\phi((A_j)^p, B_j)| + |S| \leq \\
    &\leq \sum_{j \in J}N + |I| \leq \\
    &\leq (4pn + 1)N + 2pn = (4pN + 2p)n + N = O(n).
  \end{align*}
  
  The basic idea for the general case $q \geq 1$ is that we have $q$ parameters and $|J| = O(n)$ many components to pick each parameter from, giving us $|J|^q = O(n^q)$ choices for the parameter configuration, each giving a uniformly constant number of definable subsets of $S$. (If every parameter is picked from a fixed component, Lemma \ref{lm_partition_bound} provides a uniform bound). This yields $\vc(\phi) \leq q$ as needed. The rest of the proof is stating this idea formally.
  
  First, we extend our collection of subdivisions $(\AT_j, \BT_j)_{j \in J}$ by the following singleton sets.
  For each $c_i \in S$ let $A_i = T, B_i = \{c_i\}$ and add $(\AT_i, \BT_i)$ to our collection with $U$ in the language $\LLU$ of $B_i$ interpreted arbitrarily.
  We end up with a new collection $(\AT_k, \BT_k)_{k \in K}$ indexed by some set $K$ with $|K| = |I| + |J|$ (we added $|S|$ many new pairs). Now $\curly{B_k}_{k \in K}$ partitions $T$, so $T = \bigsqcup_{k \in K} B_k$ and $S = \bigcap_{j \in J} A_j = \bigcap_{k \in K} A_k$. For $(k_1, k_2, \ldots k_q) = \vec k \in K^q$ denote 
  \begin{align*}
    B_{\vec k} = B_{k_1} \times B_{k_2} \times \ldots \times B_{k_q}.
  \end{align*}
  Then we have the following identity
  \begin{align*}
    T^q = (\bigsqcup_{k \in K} B_k)^q = \bigsqcup_{\vec k \in K^q} B_{\vec k}.
  \end{align*}
  Thus we have that $\{B_{\vec k}\}_{\vec k \in K^q}$ partition $T^q$. Compute
  \begin{align*}
    |\phi(S^p, T^q)|
    &= \left|\bigcup_{\vec k \in K^q} \phi(S^p, B_{\vec k}) \right| \leq \\
    &\leq \sum_{\vec k \in K^q} |\phi(S^p, B_{\vec k})|.
  \end{align*}
  We can bound $|\phi(S^p, B_{\vec k})|$ uniformly using Lemma \ref{lm_partition_bound}. The family $(\AT_k, \BT_k)_{k \in K}$ satisfies the requirements of the lemma and $B_{\vec k}$ looks like $B$ in the lemma after possibly permuting some variables in $\phi$.
  (For example we would need to permute $B_{(2,1,2)} = B_2 \times B_1 \times B_2$ into $B_1 \times B_2^2$ so it has the appropriate form for Lemma \ref{lm_partition_bound}.)
  Applying the lemma we get
  \begin{align*}
    |\phi(S^p, B_{\vec k})| \leq N^q
  \end{align*}
  with $N$ only depending on $q$ and the complexity of $\phi$. We complete our computation:
  \begin{align*}
    |\phi(S^p, T^q)|
    &\leq \sum_{\vec k \in K^q} |\phi(S^p, B_{\vec k})| \leq \sum_{\vec k \in K^q} N^q \leq \\
    &\leq |K^q| N^q \leq (|J| + |I|)^q N^q \leq \\
    &\leq (4pn + 1 + 2pn)^q N^q = N^q (6p + 1/n)^q n^q = O(n^q).
  \end{align*}
\end{proof}
\begin{Corollary}
  In the theory of infinite (colored) meet trees we have $vc(n) = n$ for all $n$.
\end{Corollary}
We get the general result for trees that aren't necessarily meet trees via an easy application of interpretability.
\begin{Corollary}
  In the theory of infinite (colored) trees we have $vc(n) = n$ for all $n$.
\end{Corollary}
\begin{proof}
  Let $\TTB'$ be a tree. We can embed it in a larger tree $\TTB$ that is closed under meets. Expand $\TTB$ by an extra color and interpret it by coloring the subset $\TTB'$. Thus we can interpret $\TTB'$ in $T$. By Corollary 3.17 in \cite{density} we get that $\vc^{\TTB'}(n) \leq \vc^T(1 \cdot n) = n$ thus $\vc^{\TTB'}(n) = n$ as well.
\end{proof}

\section{Conclusion}
This settles the question of determining VC-density function for infinite trees.
Lacking a quantifier elimination result in a natural language, a lot is still not known.
One can try to adapt these techniques to compute the VC-density on the formula by formula basis:
% of a fixed formula, and see if it can take non-integer values.
\begin{openq}
  What is the VC-density of individual formulas in infinite trees?
  Can it take non-integer values?
\end{openq}
It is also not known whether trees have the VC 1 property
(see Definition 5.2 in \cite{density}; this is a quantified formulation of uniform definability of types over finite sets).
It seems that our techniques can be used to show that the VC 2 property holds but this doesn't give the optimal VC-density function.
\begin{openq}
  For which $n$ do infinite trees have the VC $n$ property?
\end{openq}
One can also try to apply similar techniques to more general classes of partially ordered sets.
For example, the VC-density function is not known for lattices, and it is also not known whether lattices are dp-minimal.
Similarly, relaxing the ordering condition,
one can look at nicely behaved families of graphs, such as planar graphs or flat graphs (see \cite{stable_graphs}).
Those are known to be dp-minimal (see Theorem \ref{flat_dp_thm}), so one would expect a simple VC-density function.
It is this author's hope that the techniques developed in this \chapa can be adapted
to yield fruitful results for such more general classes of structures.