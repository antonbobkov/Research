\chapter{An investigation into regularity and free entropy.}
In this section we collect some results dealing with the regularity of non-commutative random variables.
We also show how some of the ideas from the study of free entropy, such as those in \cite{Voiculescu1999101}, and ideas from the study of free unitary Brownian motion coming from \cite{biane1997free} may be used to gain further insight into bi-free probability.

\section{Regularity results.}
We begin by recalling several useful theorems from the literature, which we will use as a starting point for our results.

\subsection{Useful results from the literature.}
This first result is paraphrased to suit our purposes.
\begin{theorem}[\cite{anderson2008law}*{Theorem 2.9}]
	Suppose that $\mu$ is a non-atomic probability measure with algebraic Cauchy transform.
	Then $\mu$ has density $f$ with respect to Lebesgue measure which fails to exist at only finitely many points, and for some $d > 0$ and every $a \in \R$ satisfies
	$$\lim_{x\to a}(x-a)^{1-d} f(x) < \infty.$$
	In particular, if $1 < p < (1-d)^{-1}$, we have $f \in L^p(\R)$.
	\label{thm:alginlp}
\end{theorem}

\begin{theorem}[\cite{2014arXiv1406.6664A}*{Theorem 1}]
	\label{thm:algebraicandfree}
	Let $(\A, \varphi)$ be a non-commutative probability space.
	Let $x_1, \ldots, x_n \in \A$ be freely independent self-adjoint non-commutative random variables.
	Let $$X = X^* \in M_N(\C)\otimes\C\ang{x_1, \ldots, x_n} \subset M_N(\C)\otimes\A$$ be a self-adjoint matrix polynomial.
	If the laws of $x_1, \ldots, x_n$ are algebraic, then so is the law of $X$.
\end{theorem}

\begin{theorem}[\cite{Charlesworth2016}*{Theorem 3}]
	\label{thm:noatoms}
	Let $(M, \tau)$ be a finite tracial von Neumann algebra, and $y_1, \ldots, y_n \in M$.
	Suppose that Voiculescu's free entropy dimension $\delta^*(y_1, \ldots, y_n) = n$.
	Then for any self-adjoint non-constant non-commutative polynomial $P$, the spectral measure of $y = P(y_1, \ldots, y_n)$ has no atoms.
\end{theorem}
We will not define Voiculescu's free entropy dimension, but instead state that the condition in the above theorem is satisfied if $\chi(y_1, \ldots, y_n) > -\infty$ or $\Phi^*(y_1, \ldots, y_n) < \infty$.

\section{Algebraicity and finite entropy.}
We the help of the following proposition, we are able to combine the above results to show that finite entropy is preserved under polynomial convolutions for sufficiently smooth variables.


\begin{proposition}
	\label{prop:lpentropy}
	Suppose that $y = y^*$ is a self-adjoint variable in a finite tracial von Neumann algebra $(M, \tau)$.
	Further suppose that the spectral distribution of $y$ is Lebesgue absolutely continuous, with density $f$.
	If $f \in L^p(\R)$ for some $p > 1$, then $\chi(y) > -\infty$.
\end{proposition}
\begin{proof}
	Using interpolation and the fact that $\norm{f}_1 = 1$, we may assume that $p < 2$.
	From Theorem~\ref{thm:singlevariableentropy}, we know that the free entropy of $y$ is given by
	$$\chi(y) = \iint_{\R^2}\log\abs{s-t} \,d\mu_y(s)\,d\mu_y(t) + \frac34 + \frac12\log2\pi,$$
	so it suffices to bound the integral above.
	Let $L(t) := 1_{(-4M, 4M)}(t)\log\abs{t}$, where $M = \norm{y}$ (so that the support of $\mu_y$ is contained in $[-M, M]$).
	Now for $s \in \R$, we have
	$$f(s) \int f(t) \log\abs{s-t}\,dt = f(s)\int f(t)L(s-t)\,dt = f(s)(f\star L)(s).$$
	We compute:
	\begin{align*}
		\abs{\iint \log\abs{t-s}\,d\mu(t)\,d\mu(s)}
		&= \abs{\int f(s) \int f(t)\log\abs{s-t}\,dt\,ds}\\
		&= \abs{\int f(s)(f\star L)(s)\,ds}\\
		&\leq \norm{f\cdot(f\star L)}_1\\
		&\leq \norm{f}_p\norm{f\star L}_q,
	\end{align*}
	where $1 = \frac1p+\frac1q$, by H\"older's inequality; note that $q > 2$ since $p < 2$.
	It suffices, now, to show that $f\star L \in L^q(\R)$; for this, we appeal to Young's inequality.
	Indeed, if $s = \frac q2 > 1$, then $1 + \frac 1q = \frac1p+\frac1s$ and so we have
	$$\norm{f \star L}_q \leq \norm{f}_p\norm{L}_s.$$
	Yet $\norm{L}_s < \infty$ for any $1 \leq s < \infty$, so we conclude $\abs{\chi(x)} < \infty$.
\end{proof}


\begin{corollary}
	Suppose that $y_1, \ldots, y_n$ are freely independent, self-adjoint, and algebraic, with $\chi(y_i) > -\infty$.
	Then if $P$ is a non-constant polynomial and $y = P(y_1, \ldots, y_n)$, we have $\chi(y) > -\infty$.
\end{corollary}

\begin{proof}
	From Theorem~\ref{thm:algebraicandfree}, we know that $y$ is algebraic.
	As $y_1, \ldots, y_n$ are free, we have $\chi(y_1, \ldots, y_n) = \chi(y_1) + \ldots + \chi(y_n) > -\infty$, and so Theorem~\ref{thm:noatoms} informs us that the spectral measure of $y$ has no atoms.
	Then Theorem~\ref{thm:alginlp} tells us that Proposition~\ref{prop:lpentropy} applies, and we conclude that $\chi(y) > -\infty$.
\end{proof}

It is tempting to believe that the above corollary is true far more generally, such as when $y_1, \ldots, y_n$ is merely a system of variables with $\chi(y_1, \ldots, y_n) > -\infty$.
Indeed, one expects that polynomial convolution is an operation which should be well-behaved.
However, we are not aware of a proof that works in this full generality.


\subsection{Properties of spectral measures.}
We will show in this section that certain strong regularity properties on the generators of an algebra do ensure weaker regularity of elements of that algebra.
We begin by establishing several useful lemmata which will allow us to gain some control over operators based on their spectral measures.
The first result may be found in \cite{voiculescu1979some}*{Section 4}, though our narrower version of it essentially goes back to Kato~\cite{kato1966perturbation}.
We sketch a proof here for completeness.
\begin{lemma}
	\label{lem:singularcontractioncommutation}
	Let $x = x^* \in B(\cH)$ be a self-adjoint operator, and assume that the spectral measure of $x$ is not Lebesgue absolutely continuous.
	Then there exists a sequence $T_n$ of finite-rank operators which satisfy the following properties:
	\begin{itemize}
		\item $0 \leq T_n \leq 1$;
		\item $T_n \to p$ weakly, where $p$ is the spectral projection of $x$ corresponding to the support of the Lebesgue-singular part of its spectral measure; and
		\item $\norm{[T_n, x]}_1 \to 0$.
	\end{itemize}
\end{lemma}


\begin{proof}
	Replacing $x$ by $pxp$, we may assume that $x$ has singular spectral measure.
	Let us further assume that the spectrum of $x$ is contained in $[0,1]$ (renormalizing and shifting $x$ if necessary) and that $\cH = L^2([0,1], \mu_x)$.
	Fix $n > 0$ and let $E_1, \ldots, E_k$ be a disjoint collection of Borel subsets of $[0,1]$ such that
	$$\sum_{i=1}^k\operatorname{diam}(E_i) < \frac1n \qquad\text{but}\qquad \mu_x\paren{\bigcup_{i=1}^k E_i} > 1-\frac1n.$$
	Now let $Q_i$ be the rank $1$ projection onto the function $1_{E_i}$, and $P_n = Q_1 + \ldots + Q_k$ a rank $k$ projection.
	Plainly $P_n \to 1$ weakly, by our choice of $E_i$.
	On the other hand, 
	$$[P_n, x] = \sum_{i=1}^k [Q_i, x].$$
	Now $[Q_i, x]$ has rank at most two, while $\norm{[Q_i, x]} \leq \operatorname{diam}(E_i) < \frac1n$; hence 
	$$\norm{[P_n, x]}_1 \leq \sum_{i=1}^k \norm{[Q_i, x]}_1 \leq 2\operatorname{diam}(E_i) \leq \frac2n \to 0.$$
\end{proof}

The next two lemmata are similar to each other in spirit; each allows us to conclude that the vanishing of certain derivative-like quantities implies that a variable must be constant.
First, though, we establish some convenient notation.
\begin{notation}
	In what follows, given $x, y \in \A$, we will denote $y\otimes x =: (x\otimes y)^\sigma$, and extend this map linearly to all of $\A\otimes\A$.
	Recall also the notation we established in Section~\ref{ss:freeentropyintro}: if $X$ is an $\A$-$\A$-bimodule, $\xi\in X$, and $x, y \in \A$, then $(x\otimes y)\#\xi = x\cdot\xi\cdot y$.
\end{notation}

\begin{lemma}
	\label{lem:zero-der}
	Let $(\A, \tau)$ be a non-commutative probability space with faithful tracial state $\tau$, generated by algebraically free self-adjoint elements $y_1, \ldots, y_n$.
	Take $y \in \A$ a polynomial.
	Let $\partial_i : \A\otimes\A \to \A$ be the free difference quotients, as in Definition~\ref{defn:freedifferencequotients}.
	Then $y \in \operatorname{Alg}\paren{y_1, \ldots, \hat{y}_i, \ldots, y_n}$ if and only if
	$$\paren{\partial_i y}^\sigma\#y^* = 0.$$
	Consequently, if the above equation holds for each $i$, $y \in \C$.
\end{lemma}

\begin{proof}
	One direction is immediate as $\operatorname{Alg}(y_1, \ldots, \hat y_i, \ldots, y_n) \subseteq \ker\partial_i$.

	Let $\mathcal{N}_i : \A \to \A$ be the number operator associated to $y_i$, the linear map which multiplies each monomial by its $y_i$-degree.
	Observe that
	$$(\partial_i y)\# y_i = \mathcal{N}_i(y),$$
	as each monomial $m$ in $y$ contributes $\sum_{m = ay_ib} (a\otimes b)\# y_i = \mathcal{N}_i(m)$.

	Suppose, then, that $(\partial_i y)^\sigma\#y^* = 0$, so $y_i(\partial_i y)^\sigma\#y^* = 0$ as well.
	Let $\varphi_\lambda : A \to A$ be the algebra homomorphism given by $\varphi_\lambda(y_i) = \lambda y_i$, $\varphi_\lambda(y_j) = y_j$ for $j \neq i$, which exists as $y_1, \ldots, y_n$ are algebraically free.
	We compute the following:
	$$
	0
	= \tau\circ\varphi_\lambda(0)
	= \tau\circ\varphi_\lambda\left(y_i(\partial_i y)^\sigma\#y^*\right)
	= \tau\circ\varphi_\lambda\left(y^*(\partial_i y)\# y_i\right)
	= \tau\circ\varphi_\lambda\left(y^* \mathcal{N}_i(y)\right).
	$$

	Now, suppose $\deg_{y_i}(y) = d$, and take $x, z \in A$ so that $x$ is $y_i$-homogeneous of degree $d$, $\deg_{y_i}(z) < d$, and $y = x + z$.
	Then:
	$$
	0
	= \tau\circ\varphi_\lambda\left( y^*\mathcal{N}_i(y) \right)
	= \tau\circ\varphi_\lambda(dx^*x) + \tau\circ\varphi_\lambda(z^*\mathcal{N}_i(y) + x^*\mathcal{N}_i(z))
	= d\lambda^{2d}\tau(x^*x) + \bigO{\lambda^{2d-1}}.
	$$
	Thus $d\lambda^{2d}\tau(x^*x) = 0$, and as $\tau$ is faithful, either $x = 0$ (in which case $y = 0$) or $d = 0$ (in which case $\deg_{y_i}(y) = 0$).
	Either way, $y \in \operatorname{Alg}(y_1, \ldots, \hat y_i, \ldots, y_n)$.

	Repeated application of the above yields the final claim.
\end{proof}

\begin{lemma}
	\label{lem:zero-der-again}
	Let $y_1, \ldots, y_n$ be algebraically free self-adjoint elements which generate a finite tracial von Neumann algebra $(M, \tau)$, and $\A$ be the algebra they generate.
	Suppose further that $\delta^*(y_1, \ldots, y_n) = n$.
	Once again, let $\partial_i : \A\otimes\A \to \A$ be the free difference quotients.
	Suppose that $N: \A \to \A$ is a linear combination of number operators $\cN_i$, so for $x \in \A$,
	$$N(x) = \sum_{i=1}^n a_i(\partial_ix)\# y_i.$$
	Let $y = y^* \in \A$ be an eigenvector of $N$.
	Then if for some spectral projection $p \in \cP(W^*(y))$ and for each $1 \leq i \leq n$ with $a_i \neq 0$ we have
	$$(\partial_i y)^\sigma\#(py) = 0,$$
	it follows that $N(y) = 0$.
\end{lemma}

\begin{proof}
	As in the proof of Lemma \ref{lem:zero-der}, we compute:
	$$
	0 = \sum_{i=1}^n a_i\tau\left(y_i(\partial_iy)^\sigma\#(py)\right)
	= \tau\left( py \sum_{i=1}^n a_i(\partial_i y)\#y_i \right)
	= \tau\left( py N(y) \right)
	= \lambda \tau\left( pyyp \right).
	$$
	If $\lambda \neq 0$, it follows that $pyyp = 0$, hence $py = 0$.
	But then by Theorem~\ref{thm:noatoms} we have that $y$ is constant and each $\mathcal{N}_i(y) = 0$.
	In either case, $N(y) = 0$.
\end{proof}

We now ready to state and prove the main result of this section.
\begin{theorem}
	Let $y_1, \ldots, y_n$ be algebraically free self-adjoint elements which generate a finite tracial von Neumann algebra $(M, \tau)$, and $\A$ be the algebra they generate.
	Suppose further that $y_1, \ldots, y_n$ admit a dual system $R_1, \ldots, R_n$ as in Definition~\ref{defn:dualsystem}.
	Take $y = y^* \in \A$ be a non-constant polynomial evaluated at $(y_1, \ldots, y_n)$.
	Then the spectral measure of $y$ is not singular with respect to Lebesgue measure.
	Moreover, if there is $N$ is in the positive linear span of the number operators $\cN_i$ such that each $\cN_i$ has a non-zero coefficient and $y$ is an eigenvector of $N$, then the spectral measure of $y$ is Lebesgue absolutely continuous.
\end{theorem}

\begin{proof}
	Assume to the contrary that the spectral measure of $y$ is not absolutely continuous with respect to Lebesgue measure.
	By Lemma~\ref{lem:singularcontractioncommutation}, we may choose $0\leq T_n \leq 1$ finite rank operators with $T_n\to p$ weakly and $\norm{[T_n,y]}_1 \to 0$, where $p$ is the spectral projection onto the singular part of $y$.
	Let $J : L^2(M)\to L^2(M)$ be Tomita's conjugation operator, defined on $x \in M$ by $J(x) = x^*$, and extended by continuity to $L^2(M)$.

	Fix $x \in M$.
	Note that by definition of $R_j$, $[R_j,y_k]=\partial_j (y_k) \# P_1$; since both $[R_j,\cdot]$ and $\partial_j(\cdot)\#P_1$ are derivations, it follows that $[R_j,y]=\partial_j(y)\#P_1$.
	We now compute as follows:
	\begin{align*}
		0
		&= \lim_{n\to\infty} \norm{ Jx^*J R_i y }_\infty \norm{ [T_n, y] }_1 \\
		&\geq \lim_{n\to\infty} \abs{ \operatorname{Tr}( Jx^*J R_i y [T_n, y]) } \\
		&= \lim_{n\to\infty} \abs{ \operatorname{Tr}(Jx^*J [y, R_i] yT_n) } \\
		&= \lim_{n\to\infty} \abs{ \operatorname{Tr}(Jx^*J ((\partial_i y)\# P_1) yT_n) } \\
		&= \abs{ \operatorname{Tr}( Jx^*J P_1 ((\partial_i y)^\sigma \# (yp)) ) } \\
		&= \abs{ \tau(((\partial_i y)^\sigma\#(yp))x) }
	\end{align*}
	Here we used: the inequality $\norm{ Tw }_1 \leq \norm{ T}_1 \norm{ w}_\infty$, the trace property and commutation between $Jx^*J$ and $y$, the equality $[y,R_i] =-\partial_i(y)\#P_1$, the fact that $P_1$ is finite rank (so that we can pass to the limit $T_n\to p$) and finally the equality $\tau(z) = \langle z1, 1\rangle =\operatorname{Tr} (zP_1)$.

	We conclude that $(\partial_i y)^\sigma\#(yp) = 0$ as $\tau$ is faithful and as $x$ was arbitrary.
	Then if $p = 1$, Lemma~\ref{lem:zero-der} implies that $y$ is constant, which is absurd; thus the spectral measure of $y$ cannot be singular with respect to Lebesgue measure.

	Further, if we have $N$ as in the statement of the theorem, then Lemma~\ref{lem:zero-der-again} implies that $N(y) = 0$.
	It follows that for any non-zero monomial $m$ in $y$, $N(m) = 0$.
	As $m$ is an eigenvector of each $\cN_i$ and each has a non-negative coefficient in $N$, we learn that $\cN_i(m) = 0$.
	Thus $m$ has zero degree for each $y_i$, and so $m \in \C$.
	We conclude that $y \in \C$, a contradiction.
\end{proof}




\section{Bi-free unitary Brownian motion.}
Our aim in this section is to study multiplicative bi-free Brownian motion, as an analogue to the free unitary Brownian motion introduced by Biane \cite{biane1997free}.
Many related results in the free case were obtained in the context of a tracial von Neumann algebra, allowing the arguments to be simplified; unfortunately that luxury is not available to us in the context of bi-free probability as we are not aware of an appropriate analogue of traciality.
As the following example demonstrates, simply asking that the state on the non-commutative probability space be tracial is too restrictive.

\begin{example}
	Suppose $\paren{\A_\ell^{(\iota)}, \A_r^{(\iota)}}_{\iota\in\set{\makeaball{0},\makeaball{1}}}$ are bi-free pairs of faces in a non-commutative probability space $(\A, \varphi)$.
	Then we have for $x \in \A_\ell^{(\makeaball{0})}$, $w \in \A_r^{(\makeaball{0})}$, $y \in \A_\ell^{(\makeaball{1})}$, and $z \in \A_r^{(\makeaball{1})}$ that
	\[
		\begin{tikzpicture}[baseline]
			\def\colours{{0, 0, 1, 1}}
			\def\sidez{{1, -1,-1,1,1}}
			\def\opnames{{"$w$", "$x$", "$y$", "$z$"}}

			\begin{scope}[shift={(-\textwidth*0.25,0)}]
				\def\ordr{{1,2,3,0}}
				\bnc [n=4,colourzfrompalette=\colours,sidez=\sidez,order=\ordr,labelz=\opnames] 
				\node [below, scale=0.7] (vp1) at (bc) {$\varphi(xyzw) = \varphi(xw)\varphi(y)\varphi(z) + \varphi(x)\varphi(w)\varphi(yz) - \varphi(x)\varphi(w)\varphi(y)\varphi(z)$};
			\end{scope}
			\begin{scope}[shift={(\textwidth*0.25,0)}]
				\def\ordr{{0,1,2,3}}
				\bnc [n=4,colourzfrompalette=\colours,sidez=\sidez,order=\ordr,labelz=\opnames] 
				\node [below, scale=0.7] (vp2) at (bc) {$\varphi(wxyz) = \varphi(wx)\varphi(yz)$.};
			\end{scope}
			\path (vp1) -- node [scale=0.7] {while} (vp2);
		\end{tikzpicture}
	\]
	Note that these two terms fail to be equal even when $(x, w)$, $(y, z)$ are a bi-free standard semicircular system with $\varphi(wx) = \varphi(yz) = 1$, as the left expression vanishes while the right equals $1$.
\end{example}

\subsection{Free Brownian motion.}
We take some time to review the concept of free Brownian motion, which is the free analogue of the Gaussian process acting on a Hilbert space.
A more complete description of free Brownian motion and free stochastic calculus may be found in \cite{voiculescu1992free}.

\begin{definition}
	A \emph{free Brownian motion} in a non-commutative probability space $(\A, \varphi)$ is a non-commutative stochastic process $(S(t))_{t\geq0}$ such that:
	\begin{itemize}
		\item the increments of $S(t)$ are free: for $0\leq t_1 < \cdots < t_k$, the collection $S(t_2)-S(t_1), \ldots, S(t_k)-S(t_{k-1})$ are freely independent; and
		\item the process is stationary, with semicircular increments: for $0 \leq s < t$, $S(t)-S(s)$ is semicircular with variance $t-s$.
	\end{itemize}
\end{definition}

Free Brownian motion can be modelled on a Fock space \cite{voiculescu1992free}. Indeed, suppose
$$\mathcal{F}\paren{L^2(\R_{\geq0})} := \C\Omega \oplus \bigoplus_{n\geq1} L^2(\R_{\geq0})^{\otimes n}.$$
Let $\xi_t = 1_{[0, t]}$, and define $S(t) = l(\xi_t) + l^*(\xi_t)$.
Then $(S(t))_t$ is a free Brownian motion.

Free unitary Brownian motion was initially introduced by Biane in \cite{biane1997free} as a multiplicative analogue of the (additive) free Brownian motion above.
It's definition makes reference to a certain family of measures $(\nu_t)_{t\geq0}$ supported on $\mathbb{T}$, introduced by Bercovici and Voiculescu in \cite{bercovici1992levy}.
In particular, $\nu_t$ has the property that for $t, s \geq 0$, $\nu_t\boxtimes\nu_s = \nu_{t+s}$.
We do not require the particular details of its introduction and so will eschew them.

\begin{definition}
	A \emph{free unitary Brownian motion} in a non-commutative probability space $(\A, \varphi)$ is a non-commutative stochastic process $(U(t))_{t\geq0}$ such that:
	\begin{itemize}
		\item the (left) multiplicative increments of $U(t)$ are free: for $0 \leq t_1 < \cdots < t_k$, the increments given by $U^*(t_1)U(t_2), U^*(t_2)U(t_3), \ldots, U^*(t_{n-1})U(t_n)$ are freely independent; and
		\item the process is stationary with increments prescribed by $\nu_\cdot$: the distribution of $U^*(t)U(s)$ depends only on $s-t$, and is in fact $\nu_{s-t}$.
	\end{itemize}
\end{definition}

It was shown in \cite{biane1997free} that if $S(t)$ is a Fock space realization of a free additive Brownian motion and $U(t)$ the solution to the free stochastic differential equation
$$dU(t) = iU(t)\,dS(t) - \frac12 U(t)\,dt$$
with $U(0) = 1$, then $U(t)$ is a free unitary Brownian motion.
Moreover, the moments of a free unitary Brownian motion were computed: for $n > 0$,
$$\varphi(U(t)^n) = \sum_{k=0}^{n-1}(-1)^k\frac{t^k}{k!}n^{k-1}{n \choose k+1}e^{-nt/2}.$$
A consequence is that free unitary Brownian motion converges in distribution as $t\to\infty$ to a Haar unitary, i.e., a unitary $u_\infty$ with $\varphi(u_\infty^{k}) = \delta_{k=0}$ for $k \in \Z$.
Another important results from \cite{biane1997free} is the following bound: for some $K > 0$ and any $t > 0$,
$$\norm{U(t)-e^{-t/2}} \leq K\sqrt{t}.$$
We have already identified the bi-free analogue of Haar unitaries in Subsection~\ref{ssec:haarunitary}, which we will use to motivate our approach to a bi-free version of Brownian motion: we want a process which tends to the distribution of a Haar pair of unitaries, so that conjugating by the process asymptotically creates bi-freeness and can therefore be seen as a sort of liberation.





\subsection{The free liberation derivation.}
Suppose that $A, B$ are algebraically free unital sub-algebras generating a tracial non-commutative probability space $(\A, \tau)$.
In \cite{Voiculescu1999101}, Voiculescu defined the derivation $\delta_{A:B} : \A \to \A\otimes\A$ to be a linear map satisfying the Leibniz rule such that $\delta_{A:B}(a) = a\otimes1-1\otimes a$ for $a \in A$ and $\delta_{A:B}(b) = 0$ for $b \in B$.
It was shown that $A$ and $B$ are freely independent if and only if $(\tau\otimes\tau)\circ\delta_{A:B} \equiv 0$.
Moreover, the derivation $\delta_{A:B}$ relates to how the joint distribution of $A$ and $B$ changes as $A$ is perturbed by unitary free Brownian motion.
\begin{proposition}[\cite{Voiculescu1999101}*{Proposition 5.6}]
	Let $A, B$ be two unital $*$-subalgebras in $(\A, \tau)$ and let $\paren{U(t)}_{t\geq0}$ be a unitary free Brownian motion, which is freely independent of $A\vee B$.
	If $a_j \in A$ and $b_j \in B$ for $1\leq j \leq n$, then
	\begin{align*}
		\tau\paren{U(\epsilon)a_1U(\epsilon)^*b_1\cdots U(\epsilon)a_nU(\epsilon)^*b_n} 
		&= 
		\frac\epsilon2(\tau\otimes\tau)\left(\delta_{A:B}\left(\sum_{k=1}^n a_kb_k\cdots a_nb_na_1b_1\cdots a_{k-1}b_{k-1} \right.\right.\\
		&\qquad\qquad\qquad- \left.\left.\sum_{k=1}^n b_ka_{k+1}b_{k+1}\cdots a_nb_na_1b_1\cdots b_{k-1}a_k\right)\right) \\
		&\qquad+\tau(a_1b_1\cdots a_nb_n) + \bigO{\epsilon^2}.
	\end{align*}
\end{proposition}

Important to the proof of the above proposition, and of use to us here also, is the following approximation result.
\begin{proposition}[\cite{Voiculescu1999101}*{Proposition 1.4}]
	\label{prop:approxubm}
	Let $A$ be a $W^*$-subalgebra, $(U(t))_t$ a unitary free Brownian motion, and $S$ a $(0,1)$-semicircular element in $(M, \tau)$ so that $A$ and $(U(t))_t$ are $*$-free and $A$ and $S$ are also free. If $a_j \in A$ and $\alpha_j \in \set{1,-1}$, then we have
	$$\tau\paren{\prod_{1\leq j \leq n}^{\rightarrow} a_j U(t)^{\alpha_j}}
	= \tau\paren{\prod_{1\leq j \leq n}^{\rightarrow} a_j\paren{\paren{1-\frac{t}{2}}+i\alpha_j\sqrt{t}S}} + \mathcal{O}\paren{t^2},$$
	where the products place the terms in order from left to right.
\end{proposition}
Although the proposition was stated in terms of a tracial $W^*$-probability space, traciality was not needed in the proof.

\subsection{A bi-free analogue to the liberation derivation.}
\renewcommand{\e}{\epsilon} %this is gonna be painful to fix... TODO
For the remainder of this section, we will always be working in the context of a family of pairs of faces $\paren{(\A_\ell^{(\iota)}, \A_r^{(\iota)})}_{\iota\in\I}$ generating a non-commutative probability space $(\A, \varphi)$.
We will further denote by $\A_\ell$ and $\A_r$ the algebras generated by $\set{\A_\ell^{(\iota)} : \iota \in \I}$ and $\set{\A_r^{(\iota)} : \iota \in \I}$ respectively, and by $\A^{(\iota)}$ the algebra generated by $\A_\ell^{(\iota)}$ and $\A_r^{(\iota)}$.
Moreover, we assume that there are no algebraic relations between $\A^{(i)}$ and $\A^{(j)}$ other than $[\A^{(i)}_\ell, \A^{(j)}_r] = 0$ when $i \neq j$, and possibly $[\A^{(i)}_\ell, \A^{(i)}_r] = 0$.
In particular, we want to ensure that given a product $z_1\cdots z_n$ we can determine the $\chi$-order of the variables.

Suppose $\chi : [n] \to \set{\ell, r}$ and let $1 \leq i, j \leq n$ with $i \preceq_\chi j$.
We denote $[i, j]_\chi := \set{k : i \preceq_\chi k \preceq_\chi j}$ the $\chi$-interval between $i$ and $j$, and define analogously $[i, j)_\chi$, $(i, j]_\chi$, and $(i, j)_\chi$.
Likewise we define $[i, \infty)_\chi := \set{k : 1 \leq k \leq n, i \preceq k}$ and analogously the other rays.

We will also change our conventions on the use of $\iota$ from earlier.
For the remainder of this section, $\iota$ will always be an element of $\I$, never a map; we will use $\gem$ to denote a map $\gem: [n] \to \I$.

\begin{definition}
	Fix $\iota \in \I$.
	We define a map
	$$\taur_{\A^{(\iota)} : \bigvee_{j \in \I\setminus\set{\iota}} \A^{(j)}} : \A \to \A \otimes \A$$
	as follows.
	Given $\chi : [n] \to \slr$, $\gem: [n] \to \I$, and $z_i \in \A^{(\gem(i))}_{\chi(i)}$,
	%\begin{dmath*}
	$$
	\taur_{\A^{(\iota)} : \bigvee_{j \in \I\setminus\set{\iota}} \A^{(j)}}(z_1\cdots z_n)
	= \sum_{i \in \e^{-1}(\iota)} \sum_{\substack{j\in\e^{-1}(\iota)\\i \preceq_\chi j}}
	z_{[i,j]_\chi^c} \otimes z_{[i,j]_\chi}
	- z_{[i,j)_\chi^c}\otimes z_{[i,j)_\chi}
	- z_{(i,j]_\chi^c}\otimes z_{(i,j]_\chi}
	+ z_{(i,j)_\chi^c}\otimes z_{(i,j)_\chi}.
	$$
	%\end{dmath*}
	We now extend this definition by linearity to all of $\A$.
	When context makes our intent clear, we will sometimes write $\taur_\iota$ for $\taur_{\A^{(\iota)} : \bigvee_{j \in \I\setminus\set{\iota}} \A^{(j)}}$.

	The subscript $A : B$ is meant to mimic that in the free situation, and the basic properties present there still hold: $B \subset \ker\taur_{A:B}$ and for $a \in A$, $\taur_{A:B}(a) = 1\otimes a-a\otimes 1$.
	However, $\taur_{A:B}$ is not a derivation, even when restricted to the left or right faces of $A$ and $B$.
\end{definition}
\begin{lemma}
	\label{lem:taur}
	$\taur_\iota$ is well-defined.
	Moreover, the only terms which do not cancel in the sum defining $\taur_\iota$ are those in which no $\gem$-monochromatic $\chi$-interval is split across a tensor sign.
\end{lemma}
\begin{proof}
	Our assumptions about the lack of algebraic relations in $\A$ mean that the only ambiguity in writing a product $z_1\cdots z_n$ comes from grouping or failing to group adjacent terms, and commuting left and right terms; the latter has no impact on $\taur_\iota$ because it does not change the $\chi$-ordering of the variables.
	Notice that if $i \prec_\chi i_+$ are consecutive under the $\chi$-ordering and both contribute to the sum, then all intervals with $i$ as an open left endpoint are intervals with $i_+$ as a closed left endpoint and have opposite sign in their contributions to the two terms; likewise, all intervals with $i$ as a closed right endpoint are intervals with $i_+$ as an open right endpoint and again cancel.
	Hence the value of $\taur_\iota$ does not change if a product is written differently, and the only terms which do not cancel are those with the tensor sign falling between two $\e$-monochromatic $\chi$-intervals (or one such interval and the edge of the product), exactly one of which is $\iota$-coloured.
\end{proof}

In essence, $\taur_\iota$ acts by adding one term for each $\chi$-interval with endpoints either before or after terms coming from $\A^{(\iota)}$, consisting of the product of the terms not in that interval tensored with the product of the terms in the interval.
The sign is chosen so that if the division comes before both chosen nodes or after both chosen nodes the term counts negatively, and otherwise counts positively.
Notice that when $i = j$, the terms corresponding to $[i,i)_\chi$ and $(i,i)_\chi$ cancel and only one term contributing $-z_1\cdots z_n \otimes 1$ survives.

The liberation gradient $\delta_{A^{(\iota)}:B}$ can be expressed in a similar manner:
$$\delta_{A:B}(z_1\cdots z_n) = \sum_{i \in \gem^{-1}(\iota)} -z_{(-\infty, i)}\otimes z_{(-\infty, i)^c} + z_{(-\infty, i]} \otimes z_{(-\infty, i]^c}.$$

\begin{example}
	Let $\chi:[n]\to\slr$ and $\gem:[n]\to\I$ be as in Example~\ref{ex:vaccine}.
	Then $\taur_{\makeaball{1}}(z_1\cdots z_{10})$ is a sum of the following eight terms:
	\[\begin{tikzpicture}[baseline]
			\def\colours{{0, 0, 0, 1, 0, 1, 1, 0, 0, 0}}
			\def\sidez{{1,-1,1,1,-1,-1,-1,1,-1,1}}

			\begin{scope}[shift={(-\textwidth*0.375,0)},scale=1]
				\draw[thick] (-1,0.125) -- (-1, -2.375) --
				node[below,scale=2/3] {$-z_1\cdots z_{10}\otimes 1$}
				(1,-2.375) -- (1,0.125);
				\foreach \y in {0,...,9} {
					\pgfmathtruncatemacro{\nodename}{\y+1}
					\pgfmathtruncatemacro{\sd}{\sidez[\y]}
					\pgfmathparse{\palette[\colours[\y]]}
					\def\clr{\pgfmathresult}
					\node (ball\nodename) [draw, shade, circle, ball color=\clr, inner sep=0.07cm*2/3] at (\sd, -\y*1/4) {};
					\ifthenelse{\sd=1}{\node[scale=2/3,right] at (\sd, -\y*0.25) {\nodename}}
					{\node[scale=2/3,left] at (\sd, -\y*0.25) {\nodename}};
				}
				\draw (-1.1, -1.125) -- (-.8, -1.125);
			\end{scope}
			\begin{scope}[shift={(-\textwidth*0.125,0)},scale=1]
				\draw[thick] (-1,0.125) -- (-1, -2.375) --
				node[below,scale=2/3] {$z_1\cdots z_5 z_8z_9z_{10}\otimes z_6z_7$}
				(1,-2.375) -- (1,0.125);
				\foreach \y in {0,...,9} {
					\pgfmathtruncatemacro{\nodename}{\y+1}
					\pgfmathtruncatemacro{\sd}{\sidez[\y]}
					\pgfmathparse{\palette[\colours[\y]]}
					\def\clr{\pgfmathresult}
					\node (ball\nodename) [draw, shade, circle, ball color=\clr, inner sep=0.07cm*2/3] at (\sd, -\y*1/4) {};
					\ifthenelse{\sd=1}{\node[scale=2/3,right] at (\sd, -\y*0.25) {\nodename}}
					{\node[scale=2/3,left] at (\sd, -\y*0.25) {\nodename}};
				}
				\draw (-1.1, -1.125) -- (-1, -1.125) to [in=90,out=0] (-.8, -1.375) to [in=0,out=270] (-1, -1.625) -- (-1.1, -1.625);
			\end{scope}
			\begin{scope}[shift={(\textwidth*0.125,0)},scale=1]
				\draw[thick] (-1,0.125) -- (-1, -2.375) --
				node[below,scale=2/3] {$-z_1\cdots z_5\otimes z_6\cdots z_{10}$}
				(1,-2.375) -- (1,0.125);
				\foreach \y in {0,...,9} {
					\pgfmathtruncatemacro{\nodename}{\y+1}
					\pgfmathtruncatemacro{\sd}{\sidez[\y]}
					\pgfmathparse{\palette[\colours[\y]]}
					\def\clr{\pgfmathresult}
					\node (ball\nodename) [draw, shade, circle, ball color=\clr, inner sep=0.07cm*2/3] at (\sd, -\y*1/4) {};
					\ifthenelse{\sd=1}{\node[scale=2/3,right] at (\sd, -\y*0.25) {\nodename}}
					{\node[scale=2/3,left] at (\sd, -\y*0.25) {\nodename}};
				}
			%\draw plot [smooth] coordinates {(-1.1, -1.125) (-0.5, -1.125) (0, -1) (0.5, -0.875) (1.1, -0.875)};
				\draw (-1.1, -1.125) -- ++(0.6,0) .. controls +(.5,0) and +(-.5, 0) .. (0.5, -0.875) -- ++(0.6,0);
			\end{scope}
			\begin{scope}[shift={(\textwidth*0.375,0)},scale=1]
				\draw[thick] (-1,0.125) -- (-1, -2.375) --
				node[below,scale=2/3] {$z_1z_2z_3z_5\otimes z_4z_6\cdots z_{10}$}
				(1,-2.375) -- (1,0.125);
				\foreach \y in {0,...,9} {
					\pgfmathtruncatemacro{\nodename}{\y+1}
					\pgfmathtruncatemacro{\sd}{\sidez[\y]}
					\pgfmathparse{\palette[\colours[\y]]}
					\def\clr{\pgfmathresult}
					\node (ball\nodename) [draw, shade, circle, ball color=\clr, inner sep=0.07cm*2/3] at (\sd, -\y*1/4) {};
					\ifthenelse{\sd=1}{\node[scale=2/3,right] at (\sd, -\y*0.25) {\nodename}}
					{\node[scale=2/3,left] at (\sd, -\y*0.25) {\nodename}};
				}
			%\draw plot [smooth] coordinates {(-1.1, -1.125) (-0.5, -1.125) (0, -.875) (0.5, -0.625) (1.1, -0.625)};
				\draw (-1.1, -1.125) -- ++(0.6,0) .. controls +(.5,0) and +(-.5, 0) .. (0.5, -0.625) -- ++(0.6,0);
			\end{scope}
			\begin{scope}[yshift=-3cm]
				\begin{scope}[shift={(-\textwidth*0.375,0)},scale=1]
					\draw[thick] (-1,0.125) -- (-1, -2.375) --
					node[below,scale=2/3] {$z_1\cdots z_7\otimes z_8z_9z_{10}$}
					(1,-2.375) -- (1,0.125);
					\foreach \y in {0,...,9} {
						\pgfmathtruncatemacro{\nodename}{\y+1}
						\pgfmathtruncatemacro{\sd}{\sidez[\y]}
						\pgfmathparse{\palette[\colours[\y]]}
						\def\clr{\pgfmathresult}
						\node (ball\nodename) [draw, shade, circle, ball color=\clr, inner sep=0.07cm*2/3] at (\sd, -\y*1/4) {};
						\ifthenelse{\sd=1}{\node[scale=2/3,right] at (\sd, -\y*0.25) {\nodename}}
						{\node[scale=2/3,left] at (\sd, -\y*0.25) {\nodename}};
					}
			%\draw plot [smooth] coordinates {(-1.1, -1.625) (-0.5, -1.625) (0, -1.25) (0.5, -0.875) (1.1, -0.875)};
					\draw (-1.1, -1.625) -- ++(0.6,0) .. controls +(.5,0) and +(-.5, 0) .. (0.5, -0.875) -- ++(0.6,0);
				\end{scope}
				\begin{scope}[shift={(-\textwidth*0.125,0)},scale=1]
					\draw[thick] (-1,0.125) -- (-1, -2.375) --
					node[below,scale=2/3] {$-z_1z_2z_3z_5z_6z_7\otimes z_4z_8z_9z_{10}$}
					(1,-2.375) -- (1,0.125);
					\foreach \y in {0,...,9} {
						\pgfmathtruncatemacro{\nodename}{\y+1}
						\pgfmathtruncatemacro{\sd}{\sidez[\y]}
						\pgfmathparse{\palette[\colours[\y]]}
						\def\clr{\pgfmathresult}
						\node (ball\nodename) [draw, shade, circle, ball color=\clr, inner sep=0.07cm*2/3] at (\sd, -\y*1/4) {};
						\ifthenelse{\sd=1}{\node[scale=2/3,right] at (\sd, -\y*0.25) {\nodename}}
						{\node[scale=2/3,left] at (\sd, -\y*0.25) {\nodename}};
					}
			%\draw plot [smooth] coordinates {(-1.1, -1.625) (-0.5, -1.625) (0, -1.125) (0.5, -0.625) (1.1, -0.625)};
					\draw (-1.1, -1.625) -- ++(0.6,0) .. controls +(.5,0) and +(-.5, 0) .. (0.5, -0.625) -- ++(0.6,0);
				\end{scope}
				\begin{scope}[shift={(\textwidth*0.125,0)},scale=1]
					\draw[thick] (-1,0.125) -- (-1, -2.375) --
					node[below,scale=2/3] {$z_1z_2z_3z_5\cdots z_{10}\otimes z_4$}
					(1,-2.375) -- (1,0.125);
					\foreach \y in {0,...,9} {
						\pgfmathtruncatemacro{\nodename}{\y+1}
						\pgfmathtruncatemacro{\sd}{\sidez[\y]}
						\pgfmathparse{\palette[\colours[\y]]}
						\def\clr{\pgfmathresult}
						\node (ball\nodename) [draw, shade, circle, ball color=\clr, inner sep=0.07cm*2/3] at (\sd, -\y*1/4) {};
						\ifthenelse{\sd=1}{\node[scale=2/3,right] at (\sd, -\y*0.25) {\nodename}}
						{\node[scale=2/3,left] at (\sd, -\y*0.25) {\nodename}};
					}
					\draw (1.1, -0.875) -- (1, -0.875) to [in=270,out=180] (.8, -0.75) to [in=180,out=90] (1, -0.625) -- (1.1, -0.625);
				\end{scope}
				\begin{scope}[shift={(\textwidth*0.375,0)},scale=1]
					\draw[thick] (-1,0.125) -- (-1, -2.375) --
					node[below,scale=2/3] {$-z_1\cdots z_{10}\otimes 1$}
					(1,-2.375) -- (1,0.125);
					\foreach \y in {0,...,9} {
						\pgfmathtruncatemacro{\nodename}{\y+1}
						\pgfmathtruncatemacro{\sd}{\sidez[\y]}
						\pgfmathparse{\palette[\colours[\y]]}
						\def\clr{\pgfmathresult}
						\node (ball\nodename) [draw, shade, circle, ball color=\clr, inner sep=0.07cm*2/3] at (\sd, -\y*1/4) {};
						\ifthenelse{\sd=1}{\node[scale=2/3,right] at (\sd, -\y*0.25) {\nodename}}
						{\node[scale=2/3,left] at (\sd, -\y*0.25) {\nodename}};
					}
					\draw (1.1, -0.875) -- (0.8, -0.875);
				\end{scope}
			\end{scope}
	\end{tikzpicture}\]
\end{example}

\begin{theorem}
	Let the notation be as above, and suppose $\I = \set{1,2}$.
	Then $(\A_\ell^{(1)}, \A_r^{(1)})$ and $(\A_\ell^{(2)}, \A_r^{(2)})$ are bi-free if and only if $(\varphi\otimes\varphi)\circ\taur_{1} \equiv 0$.
\end{theorem}

\begin{proof}
	Suppose first that bi-freeness holds.
	Note that for $\lambda \in \C$, $\taur_1(\lambda) = 0$, so it suffices to check the condition on products $z_1\cdots z_n$ with $z_i \in \A_{\chi(i)}^{(\gem(i))}$ and each maximal $\gem$-monochromatic $\chi$-interval centred, since an arbitrary term may be written as a sum of such terms.
	However, by Lemma~\ref{lem:taur} we know that each term in $\taur_1(z_1\cdots z_n)$ is a tensor product with zero or more centred $\chi$-intervals occurring on each side of the tensor. The vaccine condition from bi-freeness then tells us that $\varphi\otimes\varphi$ of such a term is $0$, and so $(\varphi\otimes\varphi)\circ\taur_1(z_1\cdots z_n) = 0$.

	Now, suppose that bi-freeness fails, and let $z_1, \ldots, z_n$ be an example of the failure of vaccine with a minimum number of terms.
	Then the only terms which possibly fail to vanish under $\varphi\otimes\varphi$ from $\taur_1$ are those of the form $z_1\cdots z_n\otimes 1$ or $1\otimes z_1\cdots z_n$ (the rest being ones to which vaccine should apply, which are of shorter length and so not counterexamples by minimality).
	The term $z_1\cdots z_n\otimes1$ occurs once per $1$-coloured $\chi$-interval with negative sign, while $1\otimes z_1\cdots z_n$ occurs once with positive sign if the $\chi$-first and $\chi$-last variables are both in $\A^{(1)}$, and not at all otherwise; let $k$ be the number of $1$-coloured $\chi$-intervals, and $d = 1$ if the $\chi$-first and $\chi$-last variables are in $\A^{(1)}$, with $d = 0$ otherwise.
	Hence $\varphi\otimes\varphi(\taur_1(z_1\cdots z_n)) = (d-k)\varphi(z_1\cdots z_n) \neq 0$ unless $k = d$; but if $k = d$ either there is one $1$-interval which is $[n]$, or there are no $1$-intervals; in either case, $\gem$ is constant and so $z_1\cdots z_n$ cannot actually be a counterexample of vaccine.
\end{proof}



\subsection{Bi-free unitary Brownian motion.}
We are now ready to introduce a notion of bi-free unitary Brownian motion.

\begin{definition}
	A pair of free stochastic processes $(U_\ell(t), U_r(t))_{t\geq0}$ is a \emph{bi-freely liberating unitary Brownian motion} (or \emph{bi-flu Brownian motion}) if:
	\begin{itemize}
		\item the multiplicative increments are bi-free: if $0\leq t_1 < \cdots < t_n$, then the family of pairs of faces $((U_\ell^*(t_\iota)U_\ell(t_{\iota+1}), U_r(t_{\iota+1})U_r^*(t_\iota))_{\iota=1}^{n-1}$ is bi-free;
		\item $(U_\ell(t))_{t\geq0}$ and $(U^*_r(t))_{t\geq0}$ are each free unitary Brownian motions, and for all $t > 0$ the $*$-distribution of the pair $(U_\ell(t), U_r(t))$ matches that of $(U_\ell(t), U_\ell^*(t))$; and
		\item the distribution is stationary: the moments of $(U_\ell^*(s)U_\ell(t), U_r(t)U_r^*(s))$ depend only on $t-s$.
	\end{itemize}
\end{definition}
We have qualified this as a liberating unitary Brownian motion because it asymptotically introduces bi-free independence without modifying the distributions of the faces being liberated; we reserve the possibility of using ``bi-free unitary Brownian motion'' more broadly, such as for processes with different covariance between the left and right faces.

We will show that once again a bi-flu Brownian motion may be realized from an additive Brownian motion.
\begin{lemma}
	Suppose that $(S_\ell(t))_{t\geq0}$ is a free Brownian motion in a tracial von Neumann algebra $(M, \tau)$, and $J : L^2(M) \to L^2(M)$ is the Tomita operator defined on $M$ by $J(x) = x^*$ and extended continuously to $L^2(M)$.
	Let $S_r(t) = JS_\ell(t)J \in M'$.
	Then if $(U_\ell(t), U_r(t))$ are solutions to the stochastic differential equations
	$$dU_\ell(t) = iU_\ell(t)\,dS_\ell(t) - \frac12 U_\ell(t)\,dt \qquad\text{and}\qquad dU_r(t) = -iU_r(t)\,dS_r(t) - \frac12 U_r(t)\,dt,$$
	with initial conditions $U_\ell(0) = 1 = U_r(0)$, the pair $(U_\ell(t), U_r(t))$ is a bi-flu Brownian motion.
	Moreover, $(U_\ell(t), U_r(t))$ converges in distribution as $t \to \infty$ to a Haar pair of unitaries.
\end{lemma}

\begin{proof}
	We find immediately that $U_\ell(t)$ is a unitary free Brownian motion.
	Note that integrating a stochastic process $\omega_t\sharp dX_t$ comes down to finding a limit in $L^2(\A)$ of approximations of the form $\sum \theta_{t_k}(x_{t_k}-x_{t_k-1})\phi_{t_k}$, where $\sum \theta_{t_k}\otimes \phi_{t_k}$ approximates $\omega_t$.
	It follows that $d(JX_t^*J) = J(dX_t)^*J$, and in particular, $JdS_r(t)J = dS_\ell(t)$.
	Conjugating the equation for $dU_r(t)$ above, we find
	$$
	d(JU_r(t)J)
	= i\paren{JU_r(t)J}JdS_r(t)J - \frac12\paren{JU_r(t)J}\,dt 
	= i\paren{JU_r(t)J}dS_\ell(t) - \frac12\paren{JU_r(t)J}\,dt.
	$$
	Thus $JU_r(t)J$ satisfies the same differential equation as $U_\ell$, whence the two are equal.
	We conclude that $U_r(t)$ corresponds to right multiplication in the standard representation on $L^2(M)$ by $U_\ell^*(t)$.
	The remaining properties of bi-flu Brownian motion now follow readily from the free properties possessed by $(U_\ell(t))_{t\geq0}$, using, essentially, the techniques of Theorem~\ref{thm:bipartitefreetobifree}.
\end{proof}

We find that conjugating by bi-flu Brownian motion leads to bi-freeness as $t\to\infty$, much like in the free case, and this allows to think of this as a sort of bi-free liberation process.
A strange consequence is the following: suppose that $X, Y \in L^\infty(\Omega, \mu) \subset \A$ are classical random variables, and $((U_\ell(t), U_r(t))_{t\geq 0}$ a bi-flu Brownian motion in $\A$, bi-free from $(X, Y)$.
Then $X$ commutes in distribution with $Y$, $U_r(t)$, and $U_r^*(t)$, so in particular, $X$ and $U_r(t)YU_r^*(t)$ become independent as $t\to\infty$ while always generating a commutative probability space.
One finds that
\begin{align*}
	\varphi(f(X)U_r(t)g(Y)U_r^*(t))
	&= \varphi(f(X)g(Y))\varphi(U_r(t))\varphi(U_r^*(t)) \\
	&\qquad+ \varphi(f(X))\varphi(g(Y))\paren{1-\varphi(U_r(t))\varphi(U_r^*(t))} \\
	&= \varphi(f(X)g(Y))e^{-t} + \varphi(f(X))\varphi(g(Y))\paren{1-e^{-t}}.
\end{align*}

We will demonstrate a connection between liberation and the map $\taur$, but first we need a bi-free version of Proposition~\ref{prop:approxubm}.

\begin{lemma}
	\label{lem:ubmest}
	Suppose $(\A_\ell, \A_r)$ is a pair of faces in $\A$ and $\paren{U_\ell(t), U_r(t)}$ is a bi-flu Brownian motion, bi-free from $(\A_\ell, \A_r)$.
	Suppose further that $(S_\ell, S_r)$ is a pair of semicircular variables with covariance matrix containing a $1$ in every entry, also bi-free from $(\A_\ell, \A_r)$.
	Let $\chi : [n] \to \set{\ell, r}$, and for $1 \leq j \leq n$, take $a_j \in \A_{\chi_j}$ and $\alpha_j \in \set{1, 0, -1}$.
	Define $\psi : [n] \to \set{1, -1}$ by $\psi(j) = \alpha_j$ if $\chi(j) = \ell$, and $\psi(j) = -\alpha_j$ otherwise.
	Then we have
	$$\varphi\paren{\prod_{1\leq j \leq n}^{\rightarrow} a_jU_{\chi(j)}(t)^{\alpha_j}}
	= \varphi\paren{\prod_{1\leq j \leq n}^{\rightarrow} a_j\paren{\paren{1-\abs{\alpha_j}\frac{t}{2}}+i\psi(j)\sqrt{t}S_{\chi(j)}}} + \mathcal{O}\paren{t^2},$$
\end{lemma}

Essentially, this lemma tells us that the pair $(U_\ell(t), U_r(t))$ behaves in $*$-distribution to order $t$ the same as the pair $\paren{1-\frac{t}2 + i\sqrt{t}S_\ell, 1-\frac{t}2 - i\sqrt{t}S_r}$.

\begin{proof}
	We proceed along the same lines as in the proof of Proposition~\ref{prop:approxubm}.
	Let $I = \set{j : \alpha_j \neq 0}$, and write $m := \abs{I}$.
	Since the $*$-distribution of $(U_\ell(t), U_r(t))$ is the same as that of $(U_\ell(t), U_\ell^*(t))$, one can check that for any sequence $j_1 < \ldots < j_k$ of terms in $I$,
	$$\varphi\paren{(U_{\chi(j_1)}(t)^{\alpha_{j_1}}-e^{-t/2})\cdots(U_{\chi(j_k)}(t)^{\alpha_{j_k}}-e^{-t/2})}
	= -\delta_{k=2}\psi(j_1)\psi(j_2)t + \bigO{t^2}.$$
	This follows from the fact that the same is true in the free case, which was used in the original proof of Proposition~\ref{prop:approxubm} (\emph{cf.} \cite{Voiculescu1999101}).

	Now for each $j \in I$, we rewrite $U_{\chi(j)}(t)^{\alpha_j}$ as $\paren{U_{\chi(j)}(t)^{\alpha_j} - e^{-t/2}} + e^{-t/2}$, and expand the product.
	As we have the estimate $\norm{U_{\chi(j)}(t)^{\alpha_j} - e^{-t/2}} \leq K\sqrt{t}$, we find that only terms where at most three of these are chosen will contribute more than $\bigO{t^2}$.
	But by the above argument, terms with one or three such differences are $\bigO{t^2}$ under $\varphi$;
	then only terms which contribute are those where precisely zero or two $\paren{U_{\chi(j)}(t)^{\alpha_j} - e^{-t/2}}$ terms are chosen.
	Hence, if we abbreviate
	$$Z(t) := \paren{\sum_{\substack{1 \preceq_\chi p \prec_\chi q \preceq_\chi n\\p, q \in I}} \varphi\paren{a_1\cdots a_p (U_{\chi(p)}^{\alpha_p}-e^{-t/2}) a_{p+1}\cdots a_q (U_{\chi(q)}^{\alpha_q} - e^{-t/2}) a_{q+1}\cdots a_n}},$$
	we have
	\begin{align*}
		\varphi\paren{\prod_{1\leq j \leq n}^{\rightarrow} a_jU_{\chi(j)}(t)^{\alpha_j}}
		&= \varphi(a_1\cdots a_n)e^{-mt/2} + \bigO{t^2} - e^{-(m-2)t/2}Z(t) \\
		&= \varphi(a_1\cdots a_n)e^{-mt/2} + \bigO{t^2} \\
		&\qquad- t e^{-(m-2)t/2} \paren{\sum_{\substack{1 \preceq_\chi p \prec_\chi q \preceq_\chi n\\p, q \in I}} \varphi(a_{(p, q]_\chi})\varphi(a_{(p, q]_\chi^c})\psi(p)\psi(q)} \\
		&= \varphi(a_1\cdots a_n)\paren{1 - m\frac{t}2} + \bigO{t^2} \\
		&\qquad- t\paren{\sum_{\substack{1 \preceq_\chi p \prec_\chi q \preceq_\chi n\\p, q \in I}} \varphi(a_{(p, q]_\chi})\varphi(a_{(p, q]_\chi^c})\psi(p)\psi(q)}.
	\end{align*}
	Here the second equality may require some justification.
	One can verify that it is correct by considering the expansion in terms of cumulants; the terms corresponding to partitions with blocks of mixed colour or partitions that do not connect the $U$ terms both vanish, and we are left with all the bi-non crossing partitions which have the two joined. Summing over these, in turn, produces the product of the two moments claimed.

	Next we turn our attention to the right hand side of the equation.
	Notice that the pair $(S_\ell, S_r)$ has the same distribution as $(-S_\ell, -S_r)$ while both are bi-free from $(\A_\ell, \A_r)$, so replacing $\sqrt{t}$ by $-\sqrt{t}$ does not change the value and thus we are in fact working with a power series in $t$ rather than $\sqrt{t}$.
	Since the constant term is clearly correct, we need only establish that the $t$ term agrees.
	Contributions to the linear term come either from selecting a single $\frac{t}{2}$ in the product (together these contribute $-m\frac{t}{2}\varphi(a_1\cdots a_n)$) or from selecting a pair indices to include the semicircular terms from.
	But now
	$$\varphi(a_1\cdots a_{p}(i\psi(p)\sqrt{t})S_{\chi(p)}a_{p+1}\cdots a_{q} (i\psi(q)\sqrt{t})S_{\chi(q)} a_{q+1}\cdots a_n)
	= -t\psi(p)\psi(q)\varphi(a_{(p, q]_\chi})\varphi(a_{(p, q]^c_\chi}).
	$$
	Summing over the terms from which semi-circular elements may be selected, which is to say those with indices coming from $I$, we see the two sides of the claimed equation agree at order $t$, also. 
\end{proof}

\begin{theorem}
	Suppose $(\A^{(\iota)}_\ell, \A^{(\iota)}_r)_{\iota \in \set{\makeaball{0}, \makeaball{1}}}$ are algebraically-free pairs of faces in a non-commutative probability space $(\A, \varphi)$, which is bi-free from the bi-flu Brownian motion $(U_\ell(t), U_r(t))$.
	Given $\chi : [n]$, $\gem : [n]\to\set{\makeaball{0}, \makeaball{1}}$, and $x_i \in \A_{\chi(i)}$, set
	$$z_i{(t)} = \left\{\begin{array}{l@{\emph{ if }}l} % emph to prevent italics...
			x_i & \e(i) = \makeaball{0}\\
			U_{\chi(i)}(t) x_i U_{\chi(i)}^*(t) & \e(i) = \makeaball{1}.
	\end{array}\right.$$
	Then we have the following estimate:
	$$
	\varphi(z_1{(t)}\cdots z_n{(t)}) = \varphi(x_1\cdots x_n) + t\varphi\otimes\varphi\paren{\taur_{\makeaball{1}}(x_1\cdots x_n)} + \bigO{t^2}.
	$$
\end{theorem}

\begin{proof}
	We first apply Lemma~\ref{lem:ubmest} to replace $U_\ell(t)^{\pm1}$ by $1-\frac{t}{2} \pm i\sqrt{t}S_\ell$ and $U_r(t)^{\pm1}$ by $1-\frac{t}{2}\mp i\sqrt{t}S_r$, for some $(S_\ell, S_r)$ bi-free from $(\A_\ell, \A_r)$ as in Lemma~\ref{lem:ubmest}.
	Again, as the distribution of $(S_\ell, S_r)$ matches that of $(-S_\ell, -S_r)$, we find that we are dealing with a power series in $t$; further, it is evident that the constant term is correct.
	We therefore consider contributions to the linear term.

	However, note that these precisely correspond to the terms in the definition of $\taur_{\makeaball{1}}$.
	Indeed, we notice that when $i \prec_\chi j$ with $\e(i) = \e(j) = \makeaball{1}$, selecting the $S$ terms on either side of $x_i$ and $x_j$ contribute a total of
	$$t\paren{\varphi(x_{[i,j]_\chi^c})\varphi(x_{[i,j]_\chi}) - \varphi(x_{[i,j)_\chi^c})\varphi(x_{[i,j)_\chi})
	- \varphi(x_{(i,j]_\chi^c})\varphi(x_{(i,j]_\chi}) + \varphi(x_{(i,j)_\chi^c})\varphi(x_{(i,j)_\chi})}.$$
	The signs occur because the signs of $S$'s $\chi$-before their respective elements, or $\chi$-after, always match.
	This accounts for all the contributions coming from selecting two semicircular variables when expanding the product; what's left are the terms corresponding to selecting a $-\frac{t}{2}$ term, so each $x_i$ coming from \makeaball{1} winds up contributing $-t\varphi(x_1\cdots x_n)$ in total.
	Yet this precisely matches the contribution to $\taur_{\makeaball{1}}$ corresponding to selecting the empty terms with $i = j$.
	We conclude that the linear term in $\varphi(z_1(t)\cdots z_n(t))$ is precisely $t\varphi\otimes\varphi\paren{\taur_{\makeaball{1}}(x_1\cdots x_n)}$.
\end{proof}

In \cite{Voiculescu1999101}, Voiculescu used the free liberation process to define the liberation gradient and a mutual non-microstates free entropy.
Thus our approach here may be seen as taking the first steps towards a non-microstates theory of bi-free entropy.
\renewcommand{\e}{\iota}
