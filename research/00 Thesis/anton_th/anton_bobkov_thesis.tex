   
\documentclass [PhD] {uclathes}
 
\usepackage{tikz}
\usetikzlibrary{positioning}
\usetikzlibrary{calc}


\usepackage{enumitem}  
\usepackage{mathrsfs} 

\newcommand{\X}{\mathcal X}
\newcommand{\Y}{\mathcal Y}


\DeclareMathOperator{\TT}{\mathcal T}
\DeclareMathOperator{\PR}{P}
\DeclareMathOperator{\cl}{cl}
\newcommand{\CS}{\mathcal S}

\DeclareMathOperator{\cx}{Complexity}
\newcommand{\K}{\boldface K_\alpha}
\renewcommand{\S}{S_\alpha}

\DeclareMathOperator{\I}{\mathcal I}
\DeclareMathOperator{\J}{\mathcal J}
\DeclareMathOperator{\acl}{acl}
\DeclareMathOperator{\Aut}{Aut}

\renewcommand{\SS}{\mathbb S}


\renewcommand{\AA}{\mathscr A}
\newcommand{\BB}{\mathscr B}
\newcommand{\DD}{\mathscr D}
\newcommand{\II}{\mathscr I}
\newcommand{\MM}{\mathbb M}

\newcommand{\A}{\mathcal A}
\newcommand{\B}{\mathcal B} 
\renewcommand{\C}{\mathcal C}
\newcommand{\D}{\mathcal D}
\newcommand{\F}{\mathcal F}
\newcommand{\G}{\mathcal G}
\renewcommand{\H}{\mathcal H}
\renewcommand{\LL}{\mathcal L}
\newcommand{\LLA}{\mathcal L_{aff}}
\newcommand{\LLM}{\mathcal L_{Mac}}
\newcommand{\M}{\mathcal M}

\newcommand{\U}{\mathcal U}	

\DeclareMathOperator{\Sg}{Sg}
\DeclareMathOperator{\It}{Tp}
\DeclareMathOperator{\Sub}{Sub}
\DeclareMathOperator{\Ct}{Ct}
\DeclareMathOperator{\vecspan}{span}
\DeclareMathOperator{\val}{val}
\DeclareMathOperator{\vval}{val}
\DeclareMathOperator{\tval}{T-val}
\DeclareMathOperator{\tfl}{T-fl}
\DeclareMathOperator{\inti}{I}

\newcommand{\interval}{\inti(t, \alpha_L, \alpha_U)}

\newcommand{\GG}{\mathbb G}
\newcommand{\GGY}{\GG^{|y|}}
\newcommand{\AX}{A^{|x|}}
\newcommand{\BA}{\bar A}

\DeclareMathOperator{\diag}{diag}

\newcommand{\DB}{\mathbb D}
\newcommand{\ppp}{\partial}
\newcommand{\BM}{\bar M_{j-1}}
\newcommand{\E}{\mathscr E}
\DeclareMathOperator{\ind}{ind}
\newcommand{\vcind}{\vc_{\ind}}
\newcommand{\Ind}{\mathscr I}


\newcommand{\curly}[1]{\left\{#1\right\}}
\newcommand{\paren}[1]{\left(#1\right)}
\newcommand{\abs}[1]{\left|#1\right|}
\newcommand{\agl}[1]{\left\langle #1 \right\rangle}

\providecommand{\floor}[1]{\left \lfloor #1 \right \rfloor }

\usepackage{soul}
\newcommand{\defn}{\ul}

%\newenvironment{openq}{\paragraph{Open Question:}}{}
\renewcommand{\subset}{\subseteq}

\DeclareMathOperator{\TTB}{\boldface T}
\DeclareMathOperator{\AT}{\boldface A}
\DeclareMathOperator{\BT}{\boldface B}


\newcommand{\LLU}{\LL^*}


                         % personal LaTeX macros

%%%%%%%%%%%%%%%%%%%%%%%%%%%%%%%%%%%%%%%%%%%%%%%%%%%%%%%%%%%%%%%%%%%%%%
%
% Usually things live in separate flies.
%
% \input {prelim}                           % preliminary page info

%%%%%%%%%%%%%%%%%%%%%%%%%%%%%%%%%%%%%%%%%%%%%%%%%%%%%%%%%%%%%%%%%%%%%%%%
%                                                                      %
%                          PRELIMINARY PAGES                           %
%                                                                      %
%%%%%%%%%%%%%%%%%%%%%%%%%%%%%%%%%%%%%%%%%%%%%%%%%%%%%%%%%%%%%%%%%%%%%%%%

\title          {VC-density Computations \\
                in Various Model-Theoritic Structures}
\author         {Anton Bobkov} 
\department     {Mathematics}
% Note:  degreeyear should be optional, but as of  5-Feb-96
% it seems required or you get a year of ``2''.   -johnh
\degreeyear     {2017}

%%%%%%%%%%%%%%%%%%%%%%%%%%%%%%%%%%%%%%%%%%%%%%%%%%%%%%%%%%%%%%%%%%%%%%%%

\chair          {Matthias Aschenbrenner}
\member         {Yiannis N.\ Moschovakis}
\member         {Igor Pak} 
\member         {Vladimir V.\ Vassiliev}

%%%%%%%%%%%%%%%%%%%%%%%%%%%%%%%%%%%%%%%%%%%%%%%%%%%%%%%%%%%%%%%%%%%%%%%%

\dedication     {\textsl{To my family and friends\\
                who have been unerringly supportive \\
                throughout my career path}}
% \dedication     {\textsl{To my mother \ldots \\
%                 who---among so many other things--- \\
%                 saw to it that I learned to touch-type \\
%                 while I was still in elementary school}}

%%%%%%%%%%%%%%%%%%%%%%%%%%%%%%%%%%%%%%%%%%%%%%%%%%%%%%%%%%%%%%%%%%%%%%%%

% \acknowledgments {(Acknowledgments omitted for brevity.)}

%%%%%%%%%%%%%%%%%%%%%%%%%%%%%%%%%%%%%%%%%%%%%%%%%%%%%%%%%%%%%%%%%%%%%%%%

% \vitaitem   {1974--1975}
%                 {Campus computer center ``User Services'' programmer and
%                 consultant, Stanford Center for Information Processing,
%                 Stanford University, Stanford, California.}
% \vitaitem   {1974--1975}
%                 {Programmer, Housing Office, Stanford University.
%                 Designed a major software system for assigning
%                 students to on-campus housing.
%                 With some later improvements, it is still in use.}
% \vitaitem   {1975}
%                 {B.S.~(Mathematics) and A.B.~(Music),
%                 Stanford University.}
% \vitaitem   {1977}
%                 {M.A.~(Music), UCLA, Los Angeles, California.}
% \vitaitem   {1977--1979}
%                 {Teaching Assistant, Computer Science Department, UCLA.
%                 Taught sections of Engineering 10 (beginning computer
%                 programming course) under direction of Professor Leon
%                 Levine.
%                 During summer 1979, taught a beginning programming
%                 course as part of the Freshman Summer Program.}
% \vitaitem   {1979}
%                 {M.S.~(Computer Science), UCLA.}
% \vitaitem   {1979--1980}
%                 {Teaching Assistant, Computer Science Department, UCLA.}
% \vitaitem   {1980--1981}
%                 {Research Assistant, Computer Science Department, UCLA.}
% \vitaitem   {2007--2011}
%                 {Programmer/Analyst, Computer Science Department, UCLA.}

%%%%%%%%%%%%%%%%%%%%%%%%%%%%%%%%%%%%%%%%%%%%%%%%%%%%%%%%%%%%%%%%%%%%%%%%

% \publication    {\textsl{MADHOUS Reference Manual.}
%                 Stanford University, Dean of Student Affairs
%                 (Residential Education Division), 1978.
%                 Technical documentation for the MADHOUS
%                 software system used to assign students to
%                 on-campus housing.}

%%%%%%%%%%%%%%%%%%%%%%%%%%%%%%%%%%%%%%%%%%%%%%%%%%%%%%%%%%%%%%%%%%%%%%%%

            \abstract       {
  Aschenbrenner et. al. studied vc-density in model-theoretic context.
  We investigate it further by computing in some common structures: trees, Shelah-Spencer graphs, and an additive reduct of $p$-adic numbers.
  We show that in the theory of infinite trees the vc-function is optimal.
  This generalizes a result of Simon showing that the trees are dp-minimal. 
  In Shelah-Spencer graphs we provide an upper bound on a formula-by-formula basis and show that there isn't a uniform lower bound,
  forcing the vc-function to be infinite.
  In addition we show that Shelah-Spencer graphs do not have a finite dp-rank,
  in particular they are not dp-minimal.
  There is a linear bound for the vc-density function in the field of $p$-adic numbers,
  but this bound is not known to be optimal.
  We investigate a certain $P$-minimal additive reduct of the field of $p$-adic numbers and
  use a cell decomposition result of Leenknegt to compute an optimal bound for that structure.
            }

%%%%%%%%%%%%%%%%%%%%%%%%%%%%%%%%%%%%%%%%%%%%%%%%%%%%%%%%%%%%%%%%%%%%%%%%
 

\begin {document} 
\makeintropages

\chapter{Inroduction and Preliminaries}

\section{Inroduction}

My research concentrates on the concept of VC-density, a recent notion of rank in NIP theories.
The study of a structure in model theory usually starts with quantifier elimination, followed by a finer analysis of definable functions and interpretability.
The study of VC-density goes one step further, looking at a structure of the asymptotic growth of finite definable families.
In the most geometric examples, VC-density coincides with the natural notion of dimension.
However, no geometric structure is required for the definition of VC-density, thus we can get some notion of geometric dimension for families of sets given without any geometric context.

In 2013, Aschenbrenner et al. investigated and developed a notion of VC-density for NIP structures, an analog of geometric dimension in an abstract setting \cite{density}. Their applications included a bound for p-adic numbers, an object of great interest and a very active area of research in mathematics. My research concentrates on improving and expanding techniques of that paper to improve the known bounds as well as computing VC-density for other NIP structures of interest. I am able to obtain new bounds for the additive reduct of p-adic numbers, treese, and certain families of graphs. Recent research by Chernikov and Starchenko in 2015 \cite{regularity} suggests that having good bounds on VC-density in p-adic numbers opens a path for applications to incidence combinatorics (e.g. Szemeredi-Trotter theorem).

The concept of VC-dimension was first introduced in 1971 by Vapnik and Chervonenkis for set systems in a probabilistic setting (see \cite{density}). The theory grew rapidly and found wide use in geometric combinatorics, computational learning theory, and machine learning. Around the same time Shelah was developing the notion of NIP ("not having the independence property"), a natural tameness property of (complete theories of) structures in model theory. In 1992 Laskowski noticed the connection between the two: theories where all uniformly definable families of sets have finite VC-dimension are exactly NIP theories. It is a wide class of theories including algebraically closed fields, differentially closed fields, modules, free groups, o-minimal structures, and ordered abelian groups. A variety of valued fields fall into this category as well, including the p-adic numbers.

P-adic numbers were first introduced by Hensel in 1897, and over the following century a powerful theory was developed around them with numerous applications across a variety of disciplines, primarily in number theory, but also in physics and computer science. In 1965 Ax, Kochen and Ershov axiomatized the theory of p-adic numbers and proved a quantifier elimination result. A key insight was to connect properties of the value group and residue field to the properties of the valued field itself. In 1984 Denef proved a cell decomposition result for more general valued fields. This result was soon generalized to p-adic subanalytic and rigid analytic extensions, allowing for the later development of a more powerful technique of motivic integration. The conjunction of those model theoretic results allowed to solve a number of outstanding open problems in number theory (e.g., Artin's Conjecture on p-adic homogeneous forms).

In 1997, Karpinski and Macintyre computed VC-density bounds for o-minimal structures and asked about similar bounds for p-adic numbers. VC-density is a concept closely related to VC-dimension. It comes up naturally in combinatorics with relation to packings, Hamming metric, entropic dimension and discrepancy.  VC-density is also the decisive parameter in the Epsilon-Approximation Theorem, which is one of the crucial tools for applying VC theory in computational geometry. In a model theoretic setting it is computed for families of uniformly definable sets.  In 2013, Aschenbrenner, Dolich, Haskell, Macpherson, and Starchenko computed a bound for VC-density in p-adic numbers and a number of other NIP structures \cite{density}. They observed connections to dp-rank and dp-minimality, notions first introduced by Shelah. In well behaved NIP structures families of uniformly definable sets tend to have VC-density bounded by a multiple of their dimension, a simple linear behavior. In a lot of cases including p-adic numbers this bound is not known to be optimal. My research concentrates on improving those bounds and adapting those techniques to compute VC-density in other common NIP structures of interest to mathematicians.

Some of the other well behaved NIP structures are Shelah-Spencer graphs and flat graphs. Shelah-Spencer graphs are limit structures for random graphs arising naturally in a combinatorial context. Their model theory was studied by Baldwin, Shi, and Shelah in 1997, and later work of Laskowski in 2006 \cite{laskowski} have provided a quantifier simplification result.  Flat graphs were first studied by Podewski-Ziegler in 1978, showing that those are stable \cite{stable_graphs}, and later results gave a criterion for super stability. Flat graphs also come up naturally in combinatorics in work of Nesetril and Ossona de Mendez \cite{nowhere}. %Shelah-Spencer graphs and flat graphs are both subclasses of NIP theories, extremely well behaved model theoretically.

The first chapter of my dissertation concentrates on Shelah-Spencer graphs. I have shown that they have infinite dp-rank, so they are poorly behaved as NIP structures. I have also shown that one can obtain arbitrarily high VC-density when looking at uniformly definable families in a fixed dimension. However I'm able to bound VC-density of individual formulas in terms of edge density of the graphs they define.

The second chapter of my dissertation concentrates on graphs and graph-like structures. I have answered an open question from \cite{density}, computing VC-density for trees viewed as a partial order. The main idea is to adapt a technique of Parigot \cite{parigot_trees} to partition trees into weakly interacting parts, with simple bounds of VC-density on each. Similar partitions come up in the Podewski-Ziegler analysis of flat graphs \cite {stable_graphs}. I am able to use that technique to show that flat graphs are dp-minimal, an important first step before establishing bounds on VC-density. 

The third chapter of my dissertation deals with p-adic numbers and valued fields. I have shown that VC-density is linear for an additive reduct of p-adic numbers (using a cell decomposition result from the work of Leenknegt in 2013 \cite{reduct}). I will explore other reducts described in that paper, to see if my techniques apply to those as well.

\section{Basic Model Theory}

This section goes through the basics of the model theory used throughout this text.
It is meant to be used mostly as a reference on the notation as opposed to a comprehensive summary.
For a complete and more thorough introduction to the material, we refer the reader to \noref.
We begin with a short summary of languages, formulas, and structures:

\begin{Definition} \;
  \begin{itemize}
  \item A \defn{language} is a collection of predicate, function, and constant symbols.
  \item Fix language $\LL$ and a collection of variables.
    A \defn{term} is an expression constructed out of constants, variables, and functions.
  \item An \defn{atomic formula} is an expression constructed out of equality symbol or a predicate applied to terms.
  \item A \defn{(first-order) formula} is an expression constructed out of atomic formulas using boolean connectives
    $\wedge, \vee, \neg$ and quantifiers $\exists, \forall$.
    We denote such a formula as $\phi(x)$ where $x$ is a tuple of \defn{free variables},
    that is the variables used in $\phi$ that are not bound by quantifiers.
    Abusing notation, we denote $\LL$ to be the set of all such formulas (so $\phi \in \LL$).
  \item A formula without free variables is called a \defn{sentence}.
  \item A \defn{quantifier-free} formula is a formula that doesn't contain any quantifiers.
  \item A \defn{structure} $\MM$ consists of an infinite universe $M$ and functions, predicates, and constants matching those of $\LL$.
  \item For a variable tuple $x$, let $|x|$ be the arity of the tuple.
  \item Suppose we have a formula $\phi(x)$, structure $\MM$, and $a \in M^{|x|}$.
    Then we say that $\MM$ \defn{models} $\phi(a)$, denoted as $\MM \models \phi(a)$,
    if formula $\phi$ holds $\MM$ when we plug in $a$ into $x$.
  \item Suppose we have a structure $\MM$ and $A \subseteq M$.
    Then $\LL(A)$ denotes an expansion of $\LL$ by constant symbols correspoding to elements in $A$.
    The structure $\MM$ then can be viewed as a $\LL(A)$-structure with the appropriate interpretations.
    Formulas $\phi \in \LL(A)$ will be referred to as \defn{formulas with parameters from $A$} or simply as \defn{$A$-formulas}.
    In this context $A$ is usually referred to as a \defn{parameter set}.
  \item A \defn{theory} is a collection of sentences.
  \item For a theory $T$ and a structure $\MM$, we say that $\MM$ \defn{models} $T$,
    or that $\MM$ is a \defn{model} of $T$, if $\MM$ models every sentence in $T$.
  \item For a structure $\MM$, a \defn{theory of $\MM$} is a collection of all sentences that are modelled by $\MM$.
  \item A theory is called \defn{complete} if it is a theory of some structure $\MM$.
  \end{itemize}  
\end{Definition}

Throughout this text we often confuse complete theories with their models.
This is justfied for properties that can be described by a collection of first-order sentences.
Then a theory has this property if and only any (all) models have this property.
An example of that is a notion of stability.

Stability is a deep subject, with a lot of theory developed around it.
We present a definitnon for completeness and refer the reader to \noref for a more complete introcudtion.

\begin{Definition}
  \begin{itemize}
  \item Suppose we have a structure $\MM$.
    The formula $\phi(x,y)$ is called \defn{unstable} if for all natural $n$
    there exist $a_i \in M^{|x|}, b_i \in M^{|y|}$ for $0 \leq i \leq n$ such that
    \begin{align*}
      \MM \models \phi(a_i, b_j) \iff i \leq j.
    \end{align*}
  \item A formula is \defn{stable} if it is not unstable.
  \item A structure $\MM$ is \defn{stable} if all of its formulas are stable.
  \item A complete theory $T$ is \defn{stable} if any (all) of its models are stable.
  \end{itemize}
\end{Definition}

Definable sets are subsets of our structure given by formulas.
More precisely:
\begin{Definition}
  Suppose we have a structure $\MM$, a paramter set $A \subseteq M$ and an $A$-formula $\phi(x)$.
  Then
  \begin{align*}
    \phi(M^{|x|}) = \curly{m \in M^{|x|} \mid \MM \models \phi(m)}.
  \end{align*}
  is referred as an \defn{$A$-definable} subset of $M^{|x|}$ defined by $\phi$.
\end{Definition}

More generally, we will need a slightly more refined notion of a trace:
\begin{Definition}
  Suppose we have a structure $\MM$, a formula $\phi(x, y)$, tuples $a \in M^{|x|}, b \in M^{|y|}$, and
  sets $A \subseteq M^{|x|}, B \subseteq M^{|y|}$. 
  Define
  \begin{align*}
    \phi(A, b) &= \curly{a \in A \mid \MM \models \phi(a,b)}, \\
    \phi(a, B) &= \curly{b \in B \mid \MM \models \phi(a,b)}.
  \end{align*}
  These sets will be informally referred to as traces.
\end{Definition}

Types is one of the main tools of study in model theory:
\begin{Definition}
  Suppose $\MM$ is a structure, $B \subseteq M$.
  Also fix a variable tuple $x$.
  \begin{itemize}
  \item \defn{A partial type over $B$} is a collection of formulas in variable $x$ with parameters from $B$.
  \item A partial type $p(x)$ has a realization in $\MM$ if there exists $a \in M^{|x|}$ such that
    $\MM \models \phi(a)$ for all $\phi(x) \in p(x)$.
  \item A partial type is \defn{consistent} if its every finite subset of formulas has a realization.
  \item Suppose $a \in M^{|x|}$ and $\Delta \subseteq \LL(B)$ a collection of formulas in $x$.
    Define the \defn{$\Delta$-type of $a$ over $B$} to be a collection of formulas $\phi(x) \in \Delta$
    such that $\MM \models \phi(a)$.
    Denote it as $\tp_{\Delta}(a/B)$.
  \item Suppose $a \in M^{|x|}$.
    Define the \defn{type of $a$ over $B$} as the $\Delta$-type of $a$ over $B$ for $\Delta = \LL(B)$.
    Denote it as $\tp(a/B)$.
  \end{itemize}
\end{Definition}

Saturated structures is another important construction that we will be using.
Generally speaking, a lot of the model theory is done inside of saturated structures as it simplifies a lot of constructions.
The definition is as follows:
\begin{Definition}
  Let $\kappa$ be a cardinal.
  A structure $\MM$ is called $\kappa$-saturated if for all $B \subset M$ with $|B| < \kappa$
  we have that all consistent partial types over $B$ are realized in $\MM$.
\end{Definition}

Indiscernible sequences will be useful to us to describe dp-rank and dp-minimality.
They come up often in model theory as a way to leverage symmetry present in sequences and sets.
\begin{Definition}
  \begin{itemize}
  \item Suppose we have a sequence $(a_i)_{i \in \I}$ where $\I$ is an ordered index set.
    For $\J \subset \I$ the expression $a_{\J}$ denotes  a tuple obtained by concatenation of the sequence $(a_j)_{j \in \J}$
    (the sequence is ordered using the order of $\I$).    
  \item Suppose $\MM$ is a structure, $B \subseteq M$, and $\I$ is an oredred index set. 
    A sequence $(a_i)_{i \in \I}$ is called \defn{indiscernible over $B$} if
    for any two subsets $\J_1, \J_2 \subseteq \I$ of equal length we have 
    \begin{align*}
      \tp(a_{J_1}/B) = \tp(a_{J_2}/B).
    \end{align*}
  \item If we use the same definition, but allow tuples $a_{J_1}, a_{J_2}$ to be concatenated in arbitrary order,
    then we obtain the definition a sequence that is \defn{totally indiscernible over $B$} (alternatively a totally indiscernible set).
  \end{itemize}
\end{Definition}

Here is an important property of indiscernible sequences in stable thoeries:
\begin{Lemma}[see \noref]
  A structure is stable if and only if every indiscernible sequence is totally indiscernible.
\end{Lemma}

Sometimes instead of starting with an indiscernible sequence, we wish to construct one from a sequence with some degree of symmetry:
\begin{Lemma} [see \noref]
  Work in a $\aleph_1$-saturated structure $\MM$.
  Suppose $B \subset M$.
  Fix a variable tuple $x$ and a collection of formulas $\Delta(x_1, \ldots, x_n)$ with $|x_1| = |x|$.
  Suppose we can find an arbitrarily long sequence $(a_i)_{i \in \I}$ with $a_i \in M^{|x|}$ such that
  for any subset $\J \subseteq \I$ of length $n$ we have
  \begin{align*}
    \MM \models \Delta(a_{\J}).
  \end{align*}
  Then there exists an infinite indiscernible sequence $(a_i')_{i \in \omega}$ with
  \begin{align*}
    \MM \models \Delta(a_1', a_2', \ldots, a_n').
  \end{align*}
\end{Lemma}

Instead of working with types directly, it is often more convenient to work with automorphisms:
\begin{Definition}
  Suppose $\MM$ is a structure and $A \subset M$.
  An \defn{automorphism} of $\MM$ over $A$ is a bijection of $f \colon M \arr M$
  that fixes $A$ and preserves constants, relations, and functions of $\MM$.
  We use notation $f \in \Aut(\MM/A)$.
\end{Definition}

The following result is easy to show directly from the definition:
\begin{Lemma} [see \noref]
  Suppose $\MM$ is a structure, $A \subset M$, and $f \in \Aut(\MM/A)$.
  Suppose also that $a,b \in M^n$ such that $f(a) = b$.
  Then $tp(a/A) = tp(b/A)$.
\end{Lemma}

The converse of this result holds in a special type of structure:
\begin{Definition}
  Let $\MM$ be a structure and $\kappa$ a cardinal.
  Then $\MM$ is called \defn{$\kappa$-homogeneous} if for all $A \subseteq M$ with $|A| < \kappa$
  we have that if for $a, b \in M^n$ if $tp(a/A) = tp(b/A)$ then
  there exists $f \in \Aut(\MM/A)$ such that $f(a) = b$.
\end{Definition}

Luckily, for a given theory one can always find a model sufficiently saturated and homogeneous:
\begin{Lemma} [see \noref]
  Let $T$ be a complete theory and $\kappa$ a cardinal.
  There exists a model of $T$ that is $\kappa$-saturated and $\kappa$-homogeneous.
\end{Lemma}


\section{VC-dimension and VC-density}
Throughout this section we work with a collection $\F$ of subsets of an infinite set $X$.
We call the pair $(X, \F)$ a \defn{set system}.

\begin{Definition} \ 
  \begin{itemize} 
  \item Given a subset $A$ of $X$, we define the set system $(A, A \cap \F)$
    where $A \cap \F = \curly{A \cap F \mid F\in \F}$.
  \item For $A \subseteq X$ we say that $\F$ \defn{shatters} $A$ if $A \cap \F = \PP(A)$ (the power set of $A$).
  \end{itemize}    
\end{Definition}  

\begin{Definition}
  We say $(X, \F)$ has \defn{VC-dimension} $n$ if the largest subset of $X$ shattered by $\F$ is of size $n$.
  If $\F$ shatters arbitrarily large subsets of $X$, we say that $(X, \F)$ has infinite VC-dimension.
  We denote the VC-dimension of $(X, \F)$ by $\VC(X, \F)$.
\end{Definition}  

\begin{Note}
  We may drop $X$ from the notation $\VC(X, \F)$, as the VC-dimension doesn't depend on the base set and is determined by $(\bigcup \F, \F)$.
\end{Note}
Set systems of finite VC-dimension tend to have good combinatorial properties,
and we consider set systems with infinite VC-dimension to be poorly behaved.

Another natural combinatorial notion is that of the dual system of a set system:
\begin{Definition}
  For $a \in X$ define $X_a = \curly{F \in \F \mid a \in F}$.
  Let $\F^* = \curly{X_a \mid a \in X}$.
  We call $(\F, \F^*)$ the \defn{dual system} of $(X, \F)$.
  The VC-dimension of the dual system of $(X, \F)$ is referred to as the \defn{dual VC-dimension} of $(X, \F)$ and denoted by $\VC^*(\F)$.
  (As before, this notion doesn't depend on $X$.)
\end{Definition}  

\begin{Lemma} [see 2.13b in \cite{ash7}]
  A set system $(X, \F)$ has finite VC-dimension if and only if its dual system has finite VC-dimension.
  More precisely
  \begin{align*}
    \VC^*(\F) \leq 2^{1+\VC(\F)}.
  \end{align*}
\end{Lemma}

For a more refined notion of complexity of $(X, \F)$ we look at the traces of our family on finite sets:
\begin{Definition}
  Define the \defn{shatter function} $\pi_\F \colon \N \arr \N$ of $\F$ and the \defn{dual shatter function} $\pi^*_\F \colon \N \arr \N$ of $\F$ by 
  \begin{align*}
    \pi_\F(n) &= \max \curly{|A \cap \F| \mid A \subseteq X \text{ and } |A| = n} \\
    \pi^*_\F(n) &= \max \curly{\text{atoms($B$)} \mid B \subseteq \F, |B| = n}
  \end{align*}
  where atoms($B$) is the number of atoms in the boolean algebra of sets generated by $B$.
  Note that the dual shatter function is precisely the shatter function of the dual system: $\pi^*_\F = \pi_{\F^*}$.
\end{Definition}  

A simple upper bound is $\pi_\F(n) \leq 2^n$ (same for the dual).
If the VC-dimension of $\F$ is infinite then clearly $\pi_\F(n) = 2^n$ for all $n$. Conversely we have the following remarkable fact:
\begin{Theorem} [Sauer-Shelah '72, see \cite{sauer}, \cite{shelah}]
  If the set system $(X, \F)$ has finite VC-dimension $d$ then $\pi_\F(n) \leq {n \choose \leq d}$ for all $n$, where
  ${n \choose \leq d} = {n \choose d} + {n \choose d - 1} + \ldots + {n \choose 1}$.    
\end{Theorem}

Thus the systems with a finite VC-dimension are precisely the systems where the shatter function grows polynomially.
The VC-density of $\F$ quantifies the growth of the shatter function of $\F$: 
\begin{Definition}
  Define the \defn{VC-density} and \defn{dual VC-density} of $\F$ as
  \begin{align*}
    \vc(\F) &= \limsup_{n \to \infty}\frac{\log \pi_\F(n)}{\log n} \in \R^{\geq 0} \cup \curly{+\infty},\\
    \vc^*(\F) &= \limsup_{n \to \infty}\frac{\log \pi^*_\F(n)}{\log n}\in \R^{\geq 0} \cup \curly{+\infty}.
  \end{align*}
\end{Definition}

Generally speaking a shatter function that is bounded by a polynomial doesn't itself have to be a polynomial.
Proposition 4.12 in \cite{density} gives an example of a shatter function that grows like $n \log n$ (so it has VC-density $1$).

So far the notions that we have defined are purely combinatorial.
We now adapt VC-dimension and VC-density to the model theoretic context.

\begin{Definition}
  Work in a first-order structure $\MM$.
  Fix a finite collection of formulas $\Phi(x, y)$ in the language $\LL(M)$ of $\MM$.

  \begin{itemize}
  \item For $\phi(x, y) \in \LL(M)$ and $b \in M^{|y|}$ let 
    \begin{align*}
      \phi(M^{|x|}, b) = \{a \in M^{|x|} \mid \phi(a, b)\} \subseteq M^{|x|}.
    \end{align*}
  \item Let $\Phi(M^{|x|}, M^{|y|})= \{\phi(M^{|x|}, b) \mid \phi \in \Phi, b \in M^{|y|}\} \subseteq \PP(M^{|x|})$.
  \item Let $\F_\Phi = \Phi(M^{|x|}, M^{|y|})$, giving rise to a set system $(M^{|x|}, \F_\Phi)$.
  \item Define the \defn{VC-dimension} $\VC(\Phi)$ of $\Phi$ to be the VC-dimension of $(M^{|x|}, \F_\Phi)$, similarly for the dual.
  \item Define the \defn{VC-density} $\vc(\Phi)$ of $\Phi$ to be the VC-density of $(M^{|x|}, \F_\Phi)$, similarly for the dual.
  \end{itemize}

  We will also refer to the VC-density and VC-dimension of a single formula $\phi$
  viewing it as a one element collection $\Phi = \curly{\phi}$.
\end{Definition}

Counting atoms of a boolean algebra in a model theoretic setting corresponds to counting types,
so it is instructive to rewrite the shatter function in terms of types.

\begin{Definition} 
  \begin{align*}
    \pi^*_\Phi(n) &= \max \curly{\text{number of $\Phi$-types over $B$} \mid B \subseteq M, |B| = n}.
  \end{align*}
  Here a $\Phi$-type over $B$ is a maximal consistent collection of formulas of the form $\phi(x, b)$ or $\neg\phi(x, b)$
  where $\phi \in \Phi$ and $b \in B$.
\end{Definition}

The functions $\pi^*_{\Phi}$ and $\pi^*_{\F_\Phi}$ do not have to agree,
as one fixes the number of generators of a boolean algebra of sets and the other fixes the size of the parameter set.
However, as the following lemma demonstrates, they both give the same asymptotic definition of dual VC-density.

\begin{Lemma} \label{count_types}
  \begin{align*}
    \vc^*(\Phi) &= \text{degree of polynomial growth of $\pi^*_\Phi(n)$}  = \limsup_{n \to \infty}\frac{\log \pi^*_\Phi(n)}{\log n}.
  \end{align*}  
\end{Lemma}

\begin{proof}
  With a parameter set $B$ of size $n$, we get at most $|\Phi|n$ sets $\phi(M^{|x|}, b)$ with $\phi \in \Phi, b \in B$.
  We check that asymptotically it doesn't matter whether we look at growth of boolean algebra of sets generated by
  $n$ or by $|\Phi|n$ many sets.
  We have:
  \begin{align*}
    \pi^*_{\F_\Phi}\paren{n} \leq \pi^*_\Phi(n) \leq \pi^*_{\F_\Phi}\paren{|\Phi|n}.
  \end{align*}
  Hence:
  \begin{align*}
    &\vc^*(\Phi) \leq \limsup_{n \to \infty}\frac{\log \pi^*_\Phi(n)}{\log n} \leq \limsup_{n \to \infty}\frac{\log \pi^*_{\F_\Phi}\paren{|\Phi|n}}{\log n} = \\
    & = \limsup_{n \to \infty}\frac{\log \pi^*_{\F_\Phi}\paren{|\Phi|n}}{\log |\Phi|n} \frac{\log |\Phi|n}{\log n} =
      \limsup_{n \to \infty}\frac{\log \pi^*_{\F_\Phi}\paren{|\Phi|n}}{\log |\Phi|n} \leq \\
    &\leq \limsup_{n \to \infty}\frac{\log \pi^*_{\F_\Phi}\paren{n}}{\log n} = \vc^*(\Phi).
  \end{align*}
\end{proof} 

One can check that the shatter function and hence VC-dimension and VC-density of a formula are elementary notions,
so they only depend on the first-order theory of the structure $\MM$.

NIP theories are a natural context for studying VC-density.
In fact we can take the following as the definition of NIP:
\begin{Definition}
  Define $\phi$ to be NIP if it has finite VC-dimension in a theory $T$.
  A theory $T$ is NIP if all the formulas in $T$ are NIP.
\end{Definition}

In a general combinatorial context (for arbitrary set systems),
VC-density can be any real number in $0 \cup [1, \infty)$ (see \cite{ash8}).
Less is known if we restrict our attention to NIP theories.
Proposition 4.6 in \cite{density} gives examples of formulas that have non-integer rational VC-density in an NIP theory,
however it is open whether one can get an irrational VC-density in this model-theoretic setting.

Instead of working with a theory formula by formula, we can look for a uniform bound for all formulas:
\begin{Definition} \label{vc_fn_def}
  For a given NIP structure $\MM$, define the \defn{VC-density function}
  \begin{align*}
    \vc^\MM(n) &= \sup \{\vc^*(\phi(x, y)) \mid \phi \in \LL(M), |x| = n\} \\
             &= \sup \{\vc(\phi(x, y)) \mid \phi \in \LL(M), |y| = n\} \in \R^{\geq 0} \cup \curly{+\infty}.
  \end{align*}
\end{Definition}

As before this definition is elementary, so it only depends on the theory of $\MM$.
We omit the superscript $\MM$ if it is understood from the context.
One can easily check the following bounds:
\begin{Lemma} [Lemma 3.22 in \cite{density}] \label{vcone}
  We have $\vc(1) \geq 1$ and $\vc(n) \geq n\vc(1)$.  
\end{Lemma}

However, it is not known whether the second inequality can be strict or even just whether $\vc(1) < \infty$ implies $\vc(n) < \infty$.


Dp-rank is a common measure used in study of NIP theories, with dp-minimality being a special case.
Those notions originated in \cite{shelah_dp}, and further studied in \cite{dp_add}, showing, for example, that dp-rank is additive.
Here it is easiest for us to define dp-rank in terms of vc-density over indiscernible sequences.

\begin{Definition} \label{def_dp}
  \begin{itemize}
  \item
    Work in a $\aleph_1$-saturated first-order structure $M$.
    Fix a finite collection of formulas $\Phi(x, y)$ in the language of $M$.
    Suppose $A = (a_i)_{i \in \omega}$ is an indiscernible sequence with each $a_i \in M^{|x|}$.
    Let
    \begin{align*}
      \Ind(A, \Phi) = \{\phi(\bigcup_{i \in \N} a_i, b) \mid \phi \in \Phi, b \in M^{|y|}\} \subseteq \PP(M^{|x|}).     
    \end{align*}
    This gives rise to a set system $(M^{|x|}, \Ind(A, \Phi))$.
  \item Define
    \begin{align*}
      \vcind(\Phi) = \sup \curly{\vc(\Ind(A, \Phi)) \mid A = (a_i)_{i \in \N} \text{ is indiscernible}}.
    \end{align*}
  \item \defn{Dp-rank} of an $\aleph_1$-saturated structure $M$ is $\leq n$ if $\vcind(\phi) \leq n$ for all formulas $\phi$.
  \item \defn{Dp-rank} of a theory $T$ is $\leq n$ if dp-rank is $\leq n$ for any (all) $\aleph_1$-saturated model of $T$.
  \item A theory $T$ is said to have finite dp-rank if its dp-rank is $\leq n$ for some $n$.
  \item A theory $T$ is \defn{dp-minimal} if its dp-rank $\leq 1$.
  \end{itemize}
\end{Definition}

Refer to \cite{guingona} for the connection between to the classical definition of dp-rank and the definition given here.


  We show that for the theory of infinite trees we have $\vc(n) = n$ for all $n$.
  This generalizes a result of Simon in \cite{simon_dp_min} showing that the trees are dp-minimal. 


VC-density was studied in \cite{density} by Aschenbrenner, Dolich, Haskell, MacPherson, and Starchenko as a natural notion of dimension for NIP theories. In an NIP theory we can define a vc-function

\begin{align*}
  \vc : \N \arr \R \cup \curly{\infty},
\end{align*}

where $\vc(n)$ measures the worst-case complexity of families of definable sets in an $n$-dimensional space. Simplest possible behavior is $\vc(n) = n$ for all $n$. Theories with the property that $\vc(1) = 1$ are known to be dp-minimal, i.e., having the smallest possible dp-rank. In general, it is not known whether there can be a dp-minimal theory which doesn't satisfy $\vc(n)=n$.

In this paper we work with trees viewed as posets.
% More precisely, our structure is branches of an infinite tree in a langugage of $\leq$ with possibly finitely many colors.
Parigot in \cite{parigot_trees} showed that such structures have NIP.
This result was strengthened by Simon in \cite{simon_dp_min} showing that trees are dp-minimal.
The paper \cite{density} poses the following problem:

\begin{Problem} (\cite{density} p. 47)
  Determine the VC density function of each (infinite) tree.
\end{Problem}

Here we settle this question by showing that any model of the theory of trees has $\vc(n) = n$.

Section 1 of the paper consists of a basic introduction to the concepts of VC-dimension and vc-density.
In Section 2 we introduce proper subdivisions -- the main tool that we use to analyze trees.
We also prove some basic properties of proper subdivisions.
Section 3 introduces the key constructions of proper subdivisions in tree which will be used in the proof.
Section 4 presents the proof of $\vc(n) = n$ via the subdivisions.

We use notation $a \in T^n$ for the tuples of size $n$. For a variable $x$ or a tuple $a$ we denote their arity by $|x|$ and $|a|$ respectively.

The language for the trees consists of a single binary predicate $\{\leq\}$. The theory of trees states that $\leq$ defines a partial order and for every element $a$ the set $\{x \mid x < a\}$ is linearly ordered by $<$. For visualization purposes we assume that trees grow upwards, with the smaller elements on the bottom and the larger elements on the top. If $a \leq b$ we will say that $a$ is below $b$ and $b$ is above $a$.

\begin{Definition}
  Work in a tree $\TT = (T, \leq)$.
  For $x \in T$ let $I(x) = \{t \in T \mid t \leq x\}$ denote all the elements below $x$.
  The \emph{meet} of two tree elements $a,b$ is the greatest element of $I(a) \cap I(b)$ (if one exists) and is denoted by $a \wedge b$.
\end{Definition}

The theory of \defn{meet trees} requires that any two elements in the same connected component have a meet. \defn{Colored trees} are trees with a finite number of colors added via unary predicates.

From now on assume that all trees are colored.
We allow our trees to be disconnected (so really, we work with forests) or finite unless otherwise stated.

\section{Proper Subdivisions: Definition and Properties}

We work with finite relational languages.
Given a formula we define its complexity as the depth of quantifiers used to build up the formula. More precisely:
% See for example \cite{ynm_notes} Definition 2D.4 pg.72.
\begin{Definition}
  Define \emph{complexity} of a formula by induction:
  \begin{align*}
    &\cx(\text{q.f. formula}) = 0 \\
    &\cx(\exists x \phi(x)) = \cx(\phi(x)) + 1 \\
    &\cx(\phi \wedge \psi) = \max(\cx(\phi), \cx(\psi)) \\
    &\cx(\neg \phi) = \cx(\phi)
  \end{align*}
\end{Definition}
A simple inductive argument verifies that there are (up to equivalence) only finitely many formulas when the complexity and the number of free variables are fixed.
We will use the following notation for types:
\begin{Definition} Let $\B$ be a structure, $A \subset B$ be a finite parameter set, and $a,b$ be tuples in $\B$, and $m, n$ be natural numbers.
  \begin{itemize}
  \item $\tp^n_{\B}(a/A)$ will stand for the set of all $A$-formulas of complexity $\leq n$ that are true of $a$ in $\B$.
    If $A = \emptyset$ we may also write this as $\tp^n_{\B}(a)$.
    The subscript $\B$ will be omitted as well if it is clear from context.
    Note that if $A$ is finite, there are finitely many such formulas (up to equivalence).
    The conjunction of those formulas still has complexity $\leq n$ and so we can just associate a single formula to every type $\tp^n_{\B}(a/A)$.
  \item $\B \models a \equiv^n_A b$ means that $a,b$ have the same type with complexity $\leq n$ over $A$ in $\B$, i.e., $\tp^n_{\B}(a/A) = \tp^n_{\B}(b/A)$.
  \item $S^n_{\B, m}(A)$ will stand for the set of all $m$-types of complexity $\leq n$ over $A$:
    \begin{align*}
      S^n_{\B, m}(A) = \curly{\tp^n_{\B}(a/A) \mid a \in B^m}.
    \end{align*}
  \end{itemize}
\end{Definition}

\begin{Definition}
  \begin{itemize}
  \item Let $\A$, $\B$, $\TT$ be structures in some (possibly different) finite relational languages. If the underlying sets $A, B$ of $\A, \B$ partition the underlying set $T$ of $\TT$ (i.e. $T = A \sqcup B$), then we say that $(\A, \B)$ is a \emph{subdivision} of $\TT$.
  \item A subdivision $(\A, \B)$ of $\TT$ is called \emph{$n$-proper} if given $p,q \in \N$,  $a_1, a_2 \in A^p$ and $b_1, b_2 \in B^q$ with
    \begin{align*}
      \A \models a_1 &\equiv_n a_2 \\
      \B \models b_1 &\equiv_n b_2
    \end{align*}
    we have
    \begin{align*}
      \TT \models a_1b_1 \equiv_n a_2b_2.
    \end{align*}
  \item A subdivision $(\A, \B)$ of $\TT$ is called \emph{proper} if it is $n$-proper for all $n \in \N$.
  \end{itemize}
\end{Definition}

\begin{Lemma} \label{lm_subdivision}
  Consider a subdivision $(\A, \B)$ of $\TT$. If $(\A, \B)$ is $0$-proper then it is proper.
\end{Lemma}

\begin{proof}
  We prove that the subdivision is $n$-proper for all $k$ by induction.
  The case $n = 0$ is given by the assumption.
  Suppose we have $\TT \models \exists x \, \phi^n(x, a_1, b_1)$ where $\phi^n$ is some formula of complexity $n$. Let $a \in T$ witness the existential claim, i.e., $\TT \models \phi^n(a, a_1, b_1)$. We can have $a \in A$ or $a \in B$. Without loss of generality assume $a \in A$. Let $\pp = \tp^n_{\A} (a, a_1)$. Then we have 
  \begin{align*}
    \A \models \exists x \, \tp^n_{\A}(x, a_1) = \pp
  \end{align*}
  (with $\tp^n_{\A}(x, a_1) = \pp$ a shorthand for $\phi_{\pp}(x)$ where $\phi_{\pp}$ is a formula that determines the type $\pp$).
  The formula $\tp^n_{\A}(x, a_1) = \pp$ is of complexity $\leq k$ so $\exists x \, \tp^n_{\A}(x, a_1) = \pp$ is of complexity $\leq k+1$. By the inductive hypothesis we have
  \begin{align*}
    \A \models \exists x \, \tp^n_{\A}(x, a_2) = \pp.
  \end{align*}
  Let $a'$ witness this existential claim, so that $\tp^n_{\A}(a', a_2) = \pp$, hence $\tp^n_{\A}(a', a_2) = \tp^n_{\A}(a, a_1)$, that is,
  $\A \models a'a_2 \equiv_n aa_1$. By the inductive hypothesis we therefore have
  $\TT \models aa_1b_1 \equiv_n a'a_2b_2$; in particular $\TT \models \phi^n(a', a_2, b_2)  \text {as } \TT \models \phi^n(a, a_1, b_1)$,
  and $\TT \models \exists x \phi^n(x, a_2, b_2)$.
\end{proof}

This lemma is general, but we will use it specifically applied to (colored) trees.
Suppose $\TT$ is a (colored) tree in some language $\LL = \{\leq, \ldots\}$.
Suppose $\A, \B$ are some structures in languages $\LL_A, \LL_B$ which expand $\LL$, with the $\LL$-reducts of $\A, \B$ substructures of $\TT$.
Furthermore suppose that $(\A, \B)$ is 0-proper.
Then by the previous lemma $(\A, \B)$ is a proper subdivision of $\TT$.
From now on all the subdivisions we work with will be of this form.

\begin{Example} \label{ex_disc}
  Suppose a tree consists of two connected components $C_1, C_2$.
  Then those components $(C_1, \leq)$, $(C_2, \leq)$ viewed as substructures form a proper subdivision.
  To see that we only need to show that this subdivision is 0-proper.
  But that is immediate as any $c_1 \in C_1$ and $c_2 \in C_2$ are incomparable.
\end{Example}

\begin{Example} \label{ex_cone}
  Fix a tree $\TT$ in the language $\{\leq\}$ and $a \in T$. Let $B = \{t \in T \mid a < t\}$, $S = \{t \in T \mid t \leq a\}$, $A = T - B$. Then $(A, \leq, S)$ and $(B, \leq)$ form a proper subdivision, where $\LL_A$ has a unary predicate interpreted by $S$.
  To see this, again, we show that the subdivision is 0-proper.
  The only time $a \in A$ and $b \in B$ are comparable is when $a \in S$, and this is captured by the language.
  (See proof of Lemma \ref{subdivide} for more details.)
\end{Example}

\begin{Definition} For $\phi(x, y)$, $A \subseteq T^{|x|}$ and $B \subseteq T^{|y|}$
  \begin{itemize}
  \item let $\phi(A, b) = \{a \in A \mid \phi(a, b)\} \subseteq A$, and 
  \item let $\phi(A, B) = \{\phi(A, b) \mid b \in B\} \subseteq \PP(A)$.	
  \end{itemize}
\end{Definition}
Thus $\phi(A, B)$ is a collection of subsets of $A$ definable by $\phi$ with parameters from $B$. We notice the following bound when $A, B$ are parts of a proper subdivision.

\begin{Corollary} \label{cor_type_count}
  Let $\A, \B$ be a proper subdivision of $\TT$ and $\phi(x,y)$ be a formula of complexity $n$. Then $|\phi(A^{|x|}, B^{|y|})|$ is bounded by $|S^n_{\B, |y|}|$.
\end{Corollary}

\begin{proof}
  Take some $a \in A^{|x|}$ and $b_1, b_2 \in B^{|y|}$ with $\tp^n_{\B}(b_1) = \tp^n_{\B}(b_2)$. We have $\B \models b_1 \equiv_n b_2$ and (trivially) $\A \models a \equiv_n a$. Thus  we have $\TT \models ab_1 \equiv_n ab_2$, so $T \models \phi(a, b_1) \leftrightarrow \phi(a, b_2)$. Since $a$ was arbitrary we have $\phi(A^{|x|}, b_1) = \phi(A^{|x|}, b_2)$ as different traces can only come from parameters of different types. Thus $|\phi(A^{|x|}, B^{|y|}) \leq |S^n_{\B, |y|}|$.
\end{proof}

We note that the size of the type space $|S^n_{\B, |y|}|$ can be bounded uniformly:

\begin{Definition} \label{def_type_count}
  Fix a (finite relational) language $\LL_B$. Let $N = N(n, m, \LL_B)$ be smallest integer such that for any structure $\B$ in $\LL_B$ we have $|S^n_{\B, m}| \leq N$. This integer exists as there is a finite number (up to equivalence) of possible formulas of complexity $\leq n$ with $m$ free variables.
  Note that $N(n, m, \LL_B)$ is increasing in all variables:
  \begin{align*}
    n \leq n', m \leq m', \LL_B \subseteq \LL_B' \Rightarrow N(n, m, \LL_B) \leq N(n', m', \LL_B')
  \end{align*}
\end{Definition}

\section{Proper Subdivisions: Constructions}

Throughout this section, $\TT$ denotes a colored meet tree.
First, we describe several constructions of proper subdivisions that are needed for the proof. 

\begin{Definition}
  We use $E(b,c)$ to express that $b$ and $c$ are in the same connected component:
  \begin{align*}
    E(b, c) \ifff \exists x \, (b \geq x) \wedge (c \geq x).
  \end{align*}
\end{Definition}
\begin{Definition}
  Given an element $a$ of the tree we call the sets fo all the elements above $a$, i.e. the set $\{x \mid x \geq a\}$,
  the \emph{closed cone} above $a$.
  Connected components of that cone can be thought of as \emph{open cones} above $a$.
  With that interpretation in mind, the notation $E_a(b, c)$ means that $b$ and $c$ are in the same open cone above $a$. More formally:
  \begin{align*}
    E_a(b, c) \ifff E(b,c) \text{ and } (b \wedge c) > a.
  \end{align*}
\end{Definition}

Fix a language $\LL$ for a colored tree $\LL = \{\leq, C_1, \ldots C_n\} = \{\leq, \vec C\}$.
In the following four definitions structures denoted by $\B$ are going to be in the same language $\LL_B = \LL \cup \{U\}$ with $U$ a unary predicate.
It is not always necessary to have this predicate but we keep it for the sake of uniformity.
Structures denoted by $\A$ will have different languages $\LL_A$ (those are not as important in later applications).
% All the colors $\vec C$ are interpreted by colors in $\TT$ by restriction.


	\tikzstyle{node}=[circle, draw]

	\tikzstyle{up}=[node, fill = white]
	\tikzstyle{c1}=[node, fill = white]
	\tikzstyle{md}=[node, fill = lightgray]
	\tikzstyle{c2}=[node, fill = white]
	\tikzstyle{dn}=[node, fill = white]
	\tikzstyle{ds}=[node, fill = white]
	\tikzstyle{ex}=[node, fill = white]
	\tikzstyle{nd}=[rectangle, draw]

	\begin{figure}[p]
		\begin{tikzpicture}
	\node[up] {}
		child[grow = south, level distance=10mm] {node[up] {}
			child[grow = south west, level distance=10mm]{node[up]{}	%left up
				[sibling distance=5mm]
				child{node[up]{}}
				child{node[up]{}}
			}
			child[level distance=15mm]{node[c1]{$c_1$} %main up
				[sibling distance=5mm]
				child [grow = -120, level distance=12mm] {node[md]{}	%1
					child[grow = -120]{node[md]{} %2
						child[grow = -150]{node[md]{}
							child{node[md]{}}
							child{node[md]{}}
						}
						child[grow = -120]{node[c2]{$c_2$} %3
							[grow = south]
							child{node[dn]{}}
							child{node[dn]{}
								child{node[dn]{}}
								child{node[dn]{}}
							}
						}
					}
					child[grow = south]{node[md]{}
						child[grow = -60]{node[md]{}}
					}
				}
				child[grow = -30, level distance=10mm]{node[ds]{}	%aux1 middle
					[sibling distance=5mm]
					child{node[ds]{}}
					child{node[ds]{}}
				}
				child [grow = -60, level distance=10mm] {node[ds]{} 	%aux2 middle
					[sibling distance=5mm]
					child{node[ds]{}}
					child{node[ds]{}}
				}
			}
			child [grow = south east, level distance=10mm] {node[up]{}	%right up
				[sibling distance=5mm]
				child{node[up]{}}
				child{node[up]{}}
			}
		}
		child[grow = east, level distance=40mm, white]{node[black, ex]{}
			[grow = south]
			[level distance=10mm]
			child[black]{node[ex]{}
				[sibling distance=5mm]
				child{node[ex]{}}
				child{node[ex]{}}
			}
			child[black] {node[ex]{}
				[sibling distance=5mm]
				child{node[ex]{}}
				child{node[ex]{}}
			}
		}
		;
		\draw [black, fill=white] (70mm,0) circle [radius=2mm];
		\node [right] at (72mm,0) {$A$};
		\draw [black, fill=lightgray] (70mm,-10mm) circle [radius=2mm]; 
		\node [right] at (72mm,-10mm) {$B$};
\end{tikzpicture}

		\caption{Proper subdivision  $(\AT, \BT) = (\AT^{c_1}_{c_2}, \BT^{c_1}_{c_2})$}
	\end{figure}
	
	\tikzstyle{up}=[node, fill = lightgray]
	\tikzstyle{c1}=[node, fill = white]
	\tikzstyle{md}=[node, fill = white]
	\tikzstyle{c2}=[node, fill = white]
	\tikzstyle{dn}=[node, fill = white]
	\tikzstyle{ds}=[node, fill = white]
	\tikzstyle{ex}=[node, fill = white]
	\tikzstyle{nd}=[rectangle, draw]

	\begin{figure}[p]
		\begin{tikzpicture}
	\node[up] {}
		child[grow = south, level distance=10mm] {node[up] {}
			child[grow = south west, level distance=10mm]{node[up]{}	%left up
				[sibling distance=5mm]
				child{node[up]{}}
				child{node[up]{}}
			}
			child[level distance=15mm]{node[c1]{$c_1$} %main up
				[sibling distance=5mm]
				child [grow = -120, level distance=12mm] {node[md]{}	%1
					child[grow = -120]{node[md]{} %2
						child[grow = -150]{node[md]{}
							child{node[md]{}}
							child{node[md]{}}
						}
						child[grow = -120]{node[c2]{$c_2$} %3
							[grow = south]
							child{node[dn]{}}
							child{node[dn]{}
								child{node[dn]{}}
								child{node[dn]{}}
							}
						}
					}
					child[grow = south]{node[md]{}
						child[grow = -60]{node[md]{}}
					}
				}
				child[grow = -30, level distance=10mm]{node[ds]{}	%aux1 middle
					[sibling distance=5mm]
					child{node[ds]{}}
					child{node[ds]{}}
				}
				child [grow = -60, level distance=10mm] {node[ds]{} 	%aux2 middle
					[sibling distance=5mm]
					child{node[ds]{}}
					child{node[ds]{}}
				}
			}
			child [grow = south east, level distance=10mm] {node[up]{}	%right up
				[sibling distance=5mm]
				child{node[up]{}}
				child{node[up]{}}
			}
		}
		child[grow = east, level distance=40mm, white]{node[black, ex]{}
			[grow = south]
			[level distance=10mm]
			child[black]{node[ex]{}
				[sibling distance=5mm]
				child{node[ex]{}}
				child{node[ex]{}}
			}
			child[black] {node[ex]{}
				[sibling distance=5mm]
				child{node[ex]{}}
				child{node[ex]{}}
			}
		}
		;
		\draw [black, fill=white] (70mm,0) circle [radius=2mm];
		\node [right] at (72mm,0) {$A$};
		\draw [black, fill=lightgray] (70mm,-10mm) circle [radius=2mm]; 
		\node [right] at (72mm,-10mm) {$B$};
\end{tikzpicture}

		\caption{Proper subdivision  $(\AT, \BT) = (\AT_{c_1}, \BT_{c_1})$}
	\end{figure}

	\tikzstyle{up}=[node, fill = white]
	\tikzstyle{c1}=[node, fill = white]
	\tikzstyle{md}=[node, fill = white]
	\tikzstyle{c2}=[node, fill = white]
	\tikzstyle{dn}=[node, fill = white]
	\tikzstyle{ds}=[node, fill = lightgray]
	\tikzstyle{ex}=[node, fill = white]
	\tikzstyle{nd}=[rectangle, draw]

	\begin{figure}[p]
		\begin{tikzpicture}
	\node[up] {}
		child[grow = south, level distance=10mm] {node[up] {}
			child[grow = south west, level distance=10mm]{node[up]{}	%left up
				[sibling distance=5mm]
				child{node[up]{}}
				child{node[up]{}}
			}
			child[level distance=15mm]{node[c1]{$c_1$} %main up
				[sibling distance=5mm]
				child [grow = -120, level distance=12mm] {node[md]{}	%1
					child[grow = -120]{node[md]{} %2
						child[grow = -150]{node[md]{}
							child{node[md]{}}
							child{node[md]{}}
						}
						child[grow = -120]{node[c2]{$c_2$} %3
							[grow = south]
							child{node[dn]{}}
							child{node[dn]{}
								child{node[dn]{}}
								child{node[dn]{}}
							}
						}
					}
					child[grow = south]{node[md]{}
						child[grow = -60]{node[md]{}}
					}
				}
				child[grow = -30, level distance=10mm]{node[ds]{}	%aux1 middle
					[sibling distance=5mm]
					child{node[ds]{}}
					child{node[ds]{}}
				}
				child [grow = -60, level distance=10mm] {node[ds]{} 	%aux2 middle
					[sibling distance=5mm]
					child{node[ds]{}}
					child{node[ds]{}}
				}
			}
			child [grow = south east, level distance=10mm] {node[up]{}	%right up
				[sibling distance=5mm]
				child{node[up]{}}
				child{node[up]{}}
			}
		}
		child[grow = east, level distance=40mm, white]{node[black, ex]{}
			[grow = south]
			[level distance=10mm]
			child[black]{node[ex]{}
				[sibling distance=5mm]
				child{node[ex]{}}
				child{node[ex]{}}
			}
			child[black] {node[ex]{}
				[sibling distance=5mm]
				child{node[ex]{}}
				child{node[ex]{}}
			}
		}
		;
		\draw [black, fill=white] (70mm,0) circle [radius=2mm];
		\node [right] at (72mm,0) {$A$};
		\draw [black, fill=lightgray] (70mm,-10mm) circle [radius=2mm]; 
		\node [right] at (72mm,-10mm) {$B$};
\end{tikzpicture}

		\caption{Proper subdivision  $(\AT, \BT) = (\AT^{c_1}_S, \BT^{c_1}_S)$ for $S = \{c_2\}$}
	\end{figure}

	\tikzstyle{up}=[node, fill = white]
	\tikzstyle{c1}=[node, fill = white]
	\tikzstyle{md}=[node, fill = white]
	\tikzstyle{c2}=[node, fill = white]
	\tikzstyle{dn}=[node, fill = white]
	\tikzstyle{ds}=[node, fill = white]
	\tikzstyle{ex}=[node, fill = lightgray]
	\tikzstyle{nd}=[rectangle, draw]

	\begin{figure}[p]
		\begin{tikzpicture}
	\node[up] {}
		child[grow = south, level distance=10mm] {node[up] {}
			child[grow = south west, level distance=10mm]{node[up]{}	%left up
				[sibling distance=5mm]
				child{node[up]{}}
				child{node[up]{}}
			}
			child[level distance=15mm]{node[c1]{$c_1$} %main up
				[sibling distance=5mm]
				child [grow = -120, level distance=12mm] {node[md]{}	%1
					child[grow = -120]{node[md]{} %2
						child[grow = -150]{node[md]{}
							child{node[md]{}}
							child{node[md]{}}
						}
						child[grow = -120]{node[c2]{$c_2$} %3
							[grow = south]
							child{node[dn]{}}
							child{node[dn]{}
								child{node[dn]{}}
								child{node[dn]{}}
							}
						}
					}
					child[grow = south]{node[md]{}
						child[grow = -60]{node[md]{}}
					}
				}
				child[grow = -30, level distance=10mm]{node[ds]{}	%aux1 middle
					[sibling distance=5mm]
					child{node[ds]{}}
					child{node[ds]{}}
				}
				child [grow = -60, level distance=10mm] {node[ds]{} 	%aux2 middle
					[sibling distance=5mm]
					child{node[ds]{}}
					child{node[ds]{}}
				}
			}
			child [grow = south east, level distance=10mm] {node[up]{}	%right up
				[sibling distance=5mm]
				child{node[up]{}}
				child{node[up]{}}
			}
		}
		child[grow = east, level distance=40mm, white]{node[black, ex]{}
			[grow = south]
			[level distance=10mm]
			child[black]{node[ex]{}
				[sibling distance=5mm]
				child{node[ex]{}}
				child{node[ex]{}}
			}
			child[black] {node[ex]{}
				[sibling distance=5mm]
				child{node[ex]{}}
				child{node[ex]{}}
			}
		}
		;
		\draw [black, fill=white] (70mm,0) circle [radius=2mm];
		\node [right] at (72mm,0) {$A$};
		\draw [black, fill=lightgray] (70mm,-10mm) circle [radius=2mm]; 
		\node [right] at (72mm,-10mm) {$B$};
\end{tikzpicture}

		\caption{Proper subdivision  $(\AT, \BT) = (\AT_S, \BT_S)$ for $S = \{c_1, c_2\}$}
	\end{figure}


\begin{Definition}
  Fix $c_1 < c_2$ in $T$. Let
  \begin{align*}
    B &= \{b \in T \mid E_{c_1}(c_2, b) \wedge \neg(b \geq c_2)\}, \\
    A &= T - B, \\
    S_1 &= \{t \in T \mid t < c_1\}, \\
    S_2 &= \{t \in T \mid t < c_2\}, \\
    S_B &= S_2 - S_1, \\
    T_A &= \{t \in T \mid c_2 \leq t\}.
  \end{align*}
  Define structures $\A^{c_1}_{c_2} = (A, \leq, \vec C \cap A, S_1, T_A)$
  where $\vec C \cap A = \curly{C_1 \cap A, \ldots, C_n \cap A}$
  and $\B^{c_1}_{c_2} = (B, \leq, \vec C \cap B, S_B)$ where $\LL_A$ is an expansion of $\LL$ by two unary predicate symbols (and $\LL_B$ as defined above). Note that $c_1, c_2 \notin B$.
\end{Definition}


\begin{Definition}
  Fix $c$ in $T$. Let
  \begin{align*}
    B &= \{b \in T \mid \neg(b \geq c) \wedge E(b,c)\}, \\
    A &= T - B, \\
    S_1 &= \{t \in T \mid t < c\}.
  \end{align*}
  Define structures $\A_{c} = (A, \leq, \vec C \cap A)$ and $\B_{c} = (B, \leq, \vec C\cap B, S_1)$ where $\LL_A = \LL$ (and $\LL_B$ as defined above). Note that $c \notin B$. (cf example \ref{ex_cone}).
\end{Definition}

\begin{Definition}
  Fix $c$ in $T$ and  a finite subset $S \subseteq T$. Let
  \begin{align*}
    B &= \{b \in T \mid (b > c) \text{ and for all $s \in S$ we have } \neg E_c(s, b)\}, \\
    A &= T - B, \\
    S_1 &= \{t \in T \mid t \leq c\}.
  \end{align*}
  Define structures $\A^{c}_{S} = (A, \leq, \vec C\cap A, S_1)$ and $\B^{c}_{S} = (B, \leq, \vec C\cap B, B)$ where $\LL_A$ is an expansion of $\LL$ by a single unary predicate (and $U \in \LL_B$ vacuously interpreted by $B$). Note that $c \notin B$ and $S \cap B = \emptyset$.
\end{Definition}

\begin{Definition}
  Fix  a finite subset $S \subseteq T$. Let
  \begin{align*}
    B &= \{b \in T \mid \text{ for all $s \in S$ we have } \neg E(s, b)\}, \\
    A &= T - B.
  \end{align*}
  Define structures $\A_{S} = (A, \leq)$ and $\B_{S} = (B, \leq, \vec C\cap B, B)$ where $\LL_A = \LL$ (and $U \in \LL_B$ vacuously interpreted by $B$). Note that $S \cap B = \emptyset$. (cf. example \ref{ex_disc})
\end{Definition}

\begin{Lemma} \label{subdivide}
  The pairs of structures defined above are all proper subdivisions of $\TT$.
\end{Lemma}

\begin{proof}
  We only show this holds for the pair $(\A, \B) = (\A^{c_1}_{c_2} ,\B^{c_1}_{c_2})$.
  The other cases follow by a similar argument.
  The sets $A,B$ partition $T$ by definition, so $(A,B)$ is a subdivision of $\TT$.
  To show that it is proper, by Lemma \ref{lm_subdivision} we only need to check that it is $0$-proper. Suppose we have
  \begin{align*}
    a &= (a_1, a_2, \ldots, a_p) \in A^p, \\
    a' &= (a_1', a_2', \ldots, a_p') \in A^p,  \\
    b &= (b_1, b_2, \ldots, b_q) \in B^q,  \\
    b' &= (b_1', b_2', \ldots, b_q') \in B^q. 
  \end{align*}
  with $\A \models a \equiv_0 a'$ and $\B \models b \equiv_0 b'$.
  We need to show that $ab$ has the same quantifier-free type in $\TT$ as $a'b'$.
  Any two elements in $T$ can be related in the four following ways:
  \begin{align*}
    x &= y, \\
    x &< y, \\
    x &> y, \text{ or } \\
    x&,y \text{ are incomparable.}
  \end{align*}
  We need to check that  for all $i,j$ the same relations hold for $(a_i, b_j)$ as do for $(a_i', b_j')$.
  
  \begin{itemize}
  \item It is impossible that $a_i = b_j$ as they come from disjoint sets.
  \item Suppose $a_i < b_j$. This forces $a_i \in S_1$ thus $a_i' \in S_1$ and $a_i' < b_j'$.
  \item Suppose $a_i > b_j$. This forces $b_j \in S_B$ and $a \in T_A$, thus $b_j' \in S_B$ and $a_i' \in T_A$, so $a_i' > b_j'$.
  \item Suppose $a_i$ and $b_j$ are incomparable. Two cases are possible:
    \begin{itemize}
    \item $b_j \notin S_B$ and $a_i \in T_A$. Then $b_j' \notin S_B$ and $a_i' \in T_A$ making $a_i', b_j'$ incomparable.
    \item $b_j \in S_B$, $a_i \notin T_A$, $a_i \notin S_1$. Similarly this forces $a_i', b_j'$ to be incomparable.
    \end{itemize}
  \end{itemize}
  Also we need to check that $ab$ has the same colors as $a'b'$. But that is immediate as having the same color in a substructure means having the same color in the tree.
\end{proof}

\section{Main proof}

The basic idea for the proof is as follows.
Suppose we have a formula with $q$ parameters over a parameter set of size $n$.
We are able to split our parameter space into $O(n)$ many partitions. Each of $q$ parameters can come from any of those $O(n)$ partitions giving us $O(n)^q$ many choices for parameter configuration. When every parameter is coming from a fixed partition the number of definable sets is constant and in fact is uniformly bounded above by some $N$. This gives us at most $N \cdot O(n)^q$ possibilities for different definable sets.

First, we generalize Corollary \ref{cor_type_count}. (This is required for computing vc-density for formulas $\phi(x, y)$ with $|y| > 1$).

\begin{Lemma} \label{lm_partition_bound}
  Consider a finite collection $(\A_i, \B_i)_{i \leq n}$ satisfying the following properties:
  \begin{itemize}
  \item $(\A_i, \B_i)$ is either a proper subdivision of $\TT$ or $A_i = T$ and $B_i = \{b_i\}$,
  \item all $\B_i$ have the same language $\LL_B$,
  \item sets $\curly{B_i}_{i \leq n}$ are pairwise disjoint.
  \end{itemize}
  Let $A = \bigcap_{i \in I} A_i$.
  Fix a formula $\phi(x, y)$ of complexity $m$ . Let $N = N(m, |y|, \LL_B)$ be as in Definition \ref{def_type_count}. Consider any $B \subseteq T^{|y|}$ of the form
  \begin{align*}
    B = B_1^{i_1} \times B_2^{i_2} \times \ldots \times B_n^{i_n} \text { with } i_1 + i_2 + \ldots + i_n = |y|.
  \end{align*}
  (some of the indeces can be zero). Then we have the following bound:
  \begin{align*}
    \phi(A^{|x|}, B) \leq N^{|y|}.
  \end{align*}
\end{Lemma}

\begin{proof}
  We show this result by counting types.
  \begin{Claim}
    Suppose we have
    \begin{align*}
      b_1, b_1' &\in B_1^{i_1} \text{ with } b_1 \equiv_m b_1' \text { in } \B_1, \\
      b_2, b_2' &\in B_2^{i_2} \text{ with } b_2 \equiv_m b_2' \text { in } \B_2, \\
                &\cdots \\
      b_n, b_n' &\in B_n^{i_n} \text{ with } b_n \equiv_m b_n' \text { in } \B_n.
    \end{align*}
    Then
    \begin{align*}
      \phi(A^{|x|}, b_1, b_2, \ldots b_n) \iff \phi(A^{|x|}, b_1', b_2', \ldots b_n').
    \end{align*}
  \end{Claim}
  \begin{proof}
    Define $\bar b_i = (b_1, \ldots, b_i, b_{i+1}', \ldots, b_n') \in B$ for $i \in [0..n]$.
    (That is, a tuple where first $i$ elements are without prime, and elements after that are with a prime.)
    We have $\phi(A^{|x|}, \bar b_i) \iff \phi(A^{|x|}, \bar b_{i+1})$ as either $(\A_{i+1}, \B_{i+1})$ is $m$-proper
    or $\B_{i+1}$ is a singleton, and the implication is trivial.
    (Notice that $b_i \in \A_j$ for $j \neq i$ by disjointness assumption.)
    Thus, by induction we get $\phi(A^{|x|}, \bar b_0) \iff \phi(A^{|x|}, \bar b_n)$ as needed.
  \end{proof}
  Thus $\phi(A^{|x|}, B)$ only depends on the choice of the types for the tuples:
  \begin{align*}
    |\phi(A^{|x|}, B)| \leq |S^m_{\B_1, i_1}| \cdot |S^m_{\B_2, i_2}| \cdot \ldots \cdot |S^m_{\B_n, i_n}|
  \end{align*}
  Now for each type space we have an inequality
  \begin{align*}
    |S^m_{\B_j, i_j}| \leq N(m, i_j, \LL_B) \leq N(m, |y|, \LL_B) \leq N
  \end{align*}
  (For singletons $|S^m_{\B_j, i_j}| = 1 \leq N$). Only non-zero indices contribute to the product and there are at most $|y|$ of those (by the equality $i_1 + i_2 + \ldots + i_n = |y|$). Thus we have
  \begin{align*}
    |\phi(A^{|x|}, B)| \leq N^{|y|}
  \end{align*}
  as needed.
\end{proof}

For subdivisions to work out properly, we will need to work with subsets closed under meets. We observe that the closure under meets doesn't add too many new elements.

% MAYBE: write a more detailed proof
\begin{Lemma} \label{lm_meet}
  Suppose $S \subseteq T$ is a finite subset of size $n \geq 1$ in a meet tree and $S'$ is its closure under meets. Then $|S'| \leq 2n - 1$.
\end{Lemma}
\begin{proof}
  We can partition $S$ into connected components and prove the result separately for each component. Thus we may assume all elements of $S$ lie in the same connected component. We prove the claim by induction on $n$. The base case $n = 1$ is clear. Suppose we have $S$ of size $k$ with closure of size at most $2k - 1$. Take a new point $s$, and look at its meets with all the elements of $S$. Pick the smallest one, $s'$. Then $S \cup \{s, s'\}$ is closed under meets.
\end{proof}

Putting all of those results together we are able to compute the $\vc$-density of formulas in meet trees.

\begin{Theorem}
  Let $\TT$ be an infinite (colored) meet tree and $\phi(x, y)$ a formula with $|x| = p$ and $|y| = q$. Then $\vc(\phi) \leq q$.
\end{Theorem}

\begin{proof}
  Pick a finite subset of $S_0 \subset T^p$ of size $n$. Let $S_1 \subset T$ consist of the components of the elements of $S_0$. Let $S \subset T$ be the closure of $S_1$ under meets. Using Lemma \ref{lm_meet} we have $|S| \leq 2|S_1| \leq 2p|S_0| = 2pn = O(n)$. We have $S_0 \subseteq S^p$, so $|\phi(S_0, T^q)| \leq |\phi(S^p, T^q)|$. Thus it is enough to show $|\phi(S^p, T^q)| = O(n^q)$.
  
  Label $S = \{c_i\}_{i \in I}$ with $|I| \leq 2pn$. For every $c_i$ we construct two partitions in the following way. We have that $c_i$ is either minimal in $S$ or it has a predecessor in $S$ (greatest element less than $c$). If it is minimal, construct $(\A_{c_i}, \B_{c_i})$. If there is a predecessor $p$, construct $(\A^p_{c_i}, \B^p_{c_i})$. For the second subdivision let $G$ be all the elements in $S$ greater than $c_i$ and construct $(\A^c_G, \B^c_G)$. So far we have constructed two subdivisions for every $i \in I$. Additionally construct $(\A_S, \B_S)$. We end up with a finite collection of proper subdivisons $(\A_j, \B_j)_{j \in J}$ with $|J| = 2|I| + 1$. Before we proceed, we note the following two lemmas describing our partitions.
  
  \begin{Lemma}
    For all $j \in J$ we have $S \subseteq A_j$. Thus $S \subseteq \bigcap_{j \in J} A_j$ and $S^p \subseteq \bigcap_{j \in J} (A_j)^p$. 
  \end{Lemma}
  
  \begin{proof}
    Check this for each possible choice of partition. Cases for partitions of the type $\A_S, \A^c_G, \A_c$ are easy. Suppose we have a partition $(\A, \B) = (\A^{c_1}_{c_2}, \B^{c_1}_{c_2})$. We need to show that $B \cap S = \emptyset$. By construction we have $c_1, c_2 \notin B$. Suppose we have some other $c \in S$ with $c \in B$. We have $E_{c_1}(c_2, c)$ i.e. there is some $b$ such that $(b > c_1)$, $(b \leq c_2)$ and $(b \leq c)$. Consider the meet $(c \wedge c_2)$. We have $(c \wedge c_2) \geq b > c_1$. Also as $\neg (c \geq c_2)$ we have $(c \wedge c_2) < c_2$. To summarize: $c_2 > (c \wedge c_2) > c_1$. But this contradicts our construction as $S$ is closed under meets, so $(c \wedge c_2) \in S$ and $c_1$ is supposed to be a predecessor of $c_2$ in $S$.
  \end{proof}
  
  \begin{Lemma}
    $\{B_j\}_{j \in J}$ is a disjoint partition of $T - S$ i.e. $T = \bigsqcup_{j \in J} B_j \sqcup S$
  \end{Lemma}
  
  \begin{proof}
    This more or less follows from the choice of partitions. Pick any $b \in S - T$. Take all the elements in $S$ greater than $b$ and take the minimal one $a$. Take all the elements in $S$ less than $b$ and take the maximal one $c$ (possible as $S$ is closed under meets). Also take all the elements in $S$ incomparable to $b$ and denote them $G$. If both $a$ and $c$ exist we have $b \in \B^a_c$. If only the upper bound exists we have $b \in \B^a_G$. If only the lower bound exists we have $b \in \B_c$. If neither exists we have $b \in \B_G$.
  \end{proof}
  
  \begin{Note}
    Those two lemmas imply $S = \bigcap_{j \in J} A_j$.
  \end{Note}
  
  \begin{Note}
    % careful application of note - have different languages and has to be > 1 
    For one-dimensional case $q = 1$ we don't need to do any more work. We have partitioned the parameter space into $|J| = O(n)$ many pieces and over each piece the number of definable sets is uniformly bounded. By Corollary \ref{cor_type_count} we have that $|\phi((A_j)^p, B_j)| \leq N$ for any $j \in J$ (letting $N = N(n_\phi, q, \LL \cup \{S\})$ where $n_\phi$ is the complexity of $\phi$ and $S$ is a unary predicate). Compute
    % describe steps
    \begin{align*}
      |\phi(S^p, T)|
      &= \left|\bigcup_{j \in J} \phi(S^p, B_j) \cup \phi(S^p, S)\right| \leq \\
      &\leq \sum_{j \in J} |\phi(S^p, B_j)| + |\phi(S^p, S)| \leq \\
      &\leq \sum_{j \in J} |\phi((A_j)^p, B_j)| + |S| \leq \\
      &\leq \sum_{j \in J}N + |I| \leq \\
      &\leq (4pn + 1)N + 2pn = (4pN + 2p)n + N = O(n)
    \end{align*}
  \end{Note}
  Basic idea for the general case $q \geq 1$ is that we have $q$ parameters and $|J| = O(n)$ many partitions to pick each parameter from giving us $|J|^q = O(n^q)$ choices for the parameter configuration, each giving a uniformly constant number of definable subsets of $S$. (If every parameter is picked from a fixed partition, Lemma \ref{lm_partition_bound} provides a uniform bound). This yields $\vc(\phi) \leq q$ as needed. The rest of the proof is stating this idea formally.
  
  First, we extend our collection of subdivisions $(\A_j, \B_j)_{j \in J}$ by the following singleton sets. For each $c_i \in S$ let $B_i = \{c_i\}$ and $A_i = T$ and add $(\A_i, \B_i)$ to our collection with $\LL_B$ the language of $B_i$ interpreted arbitrarily. We end up with a new collection $(\A_k, \B_k)_{k \in K}$ indexed by some $K$ with $|K| = |J| + |I|$ (we added $|S|$ new pairs). Now $\curly{B_k}_{k \in K}$ partitions $T$, so $T = \bigsqcup_{k \in K} B_k$ and $S = \bigcap_{j \in J} A_j = \bigcap_{k \in K} A_k$. For $(k_1, k_2, \ldots k_q) = \vec k \in K^q$ denote 
  \begin{align*}
    B_{\vec k} = B_{k_1} \times B_{k_2} \times \ldots \times B_{k_q}
  \end{align*}
  Then we have the following identity
  \begin{align*}
    T^q = (\bigsqcup_{k \in K} B_k)^q = \bigsqcup_{\vec k \in K^q} B_{\vec k}
  \end{align*}
  Thus we have that $\{B_{\vec k}\}_{\vec k \in K^q}$ partition $T^q$. Compute
  \begin{align*}
    |\phi(S^p, T^q)|
    &= \left|\bigcup_{\vec k \in K^q} \phi(S^p, B_{\vec k}) \right| \leq \\
    &\leq \sum_{\vec k \in K^q} |\phi(S^p, B_{\vec k})|
  \end{align*}
  We can bound $|\phi(S^p, B_{\vec k})|$ uniformly using Lemma \ref{lm_partition_bound}. $(\A_k, \B_k)_{k \in K}$ satisfies the requirements of the lemma and $B_{\vec k}$ looks like $B$ in the lemma after possibly permuting some variables in $\phi$. Applying the lemma we get
  \begin{align*}
    |\phi(S^p, B_{\vec k})| \leq N^q
  \end{align*}
  with $N$ only depending on $q$ and complexity of $\phi$. We complete our computation
  \begin{align*}
    |\phi(S^p, T^q)|
    &\leq \sum_{\vec k \in K^q} |\phi(S^p, B_{\vec k})| \leq \\
    &\leq \sum_{\vec k \in K^q} N^q \leq \\
    &\leq |K^q| N^q \leq \\
    &\leq (|J| + |I|)^q N^q \leq \\
    &\leq (4pn + 1 + 2pn)^q N^q = N^q (6p + 1/n)^q n^q = O(n^q)
  \end{align*}
\end{proof}
\begin{Corollary}
  In the theory of infinite (colored) meet trees we have $vc(n) = n$ for all $n$.
\end{Corollary}
We get the general result for the trees that aren't necessarily meet trees via an easy application of interpretability.
\begin{Corollary}
  In the theory of infinite (colored) trees we have $vc(n) = n$ for all $n$.
\end{Corollary}
\begin{proof}
  Let $\TT'$ be a tree. We can embed it in a larger tree $\TT$ that is closed under meets. Expand $\TT$ by an extra color and interpret it by coloring the subset $\TT'$. Thus we can interpret $\TT'$ in $T$. By Corollary 3.17 in \cite{density} we get that $\vc^{\TT'}(n) \leq \vc^T(1 \cdot n) = n$ thus $\vc^{\TT'}(n) = n$ as well.
\end{proof}

This settles the question of $vc$-function for trees. Lacking a quantifier elimination result, a lot is still not known.
One can try to adapt these techniques to compute the vc-density of a fixed formula, and see if it can take non-integer values.
It is also not known whether trees have VC 1 property (see \cite{density} 5.2 for the definition).
Our techniques can be used to show that VC 2 property holds but this doesn't give the optimal vc-function.

One can also try to apply similar techniques to more general classes of partially ordered sets.
For example, vc-density values are not known for lattices.
Similarly, dropping the order, one can look at nicely behaved families of graphs, such as planar graphs or flat graphs.
Those are known to be dp-minimal, so one would expect a simple vc-function.
It is this author's hope that the techniques developed in this paper can be adapted to yield fruitful results for a more general class of structures.

\documentclass{amsart}

\usepackage{../AMC_style}	
\usepackage{../Research}
\usepackage{../Thm}

\usepackage{mathrsfs}
\usepackage{pgfpages} 
\usepackage{setspace}

\doublespacing
%% \usepackage[margin=.75in]{geometry}
%% \pgfpagesuselayout{2 on 1}

\renewcommand{\AA}{\mathscr A}
\newcommand{\BB}{\mathscr B}
\newcommand{\DD}{\mathscr D}
\newcommand{\II}{\mathscr I}
\newcommand{\MM}{\mathscr M}

\newcommand{\F}{\mathcal F}
\renewcommand{\LL}{\mathcal L}
\newcommand{\LLA}{\mathcal L_{aff}}
\newcommand{\LLM}{\mathcal L_{Mac}}

\newcommand{\defn}{\underline}



\newcommand{\A}{\mathcal A}
\newcommand{\B}{\mathcal B}
\renewcommand{\C}{\mathcal C}
\newcommand{\D}{\mathcal D}
\renewcommand{\H}{\mathcal H}
\newcommand{\G}{\mathcal G}
\newcommand{\M}{\mathcal M}
\newcommand{\U}{\mathcal U}	
\newcommand{\X}{\mathcal X}
\newcommand{\Y}{\mathcal Y}

\newcommand{\K}{\boldface K_\alpha}
\renewcommand{\S}{S_\alpha}

\newcommand{\curly}[1]{\left\{#1\right\}}
\newcommand{\paren}[1]{\left(#1\right)}
\newcommand{\abs}[1]{\left|#1\right|}
\newcommand{\agl}[1]{\left\langle #1 \right\rangle}

\providecommand{\floor}[1]{\left \lfloor #1 \right \rfloor }

% \DeclareMathOperator{\dim}{dim}

\title{Some vc-density computations in Shelah-Spencer graphs}
\author{Anton Bobkov}
\email{bobkov@math.ucla.edu}

\begin{document}

\begin{abstract}
  We investigate vc-density in Shelah-Spencer graphs.
  We provide an upper bound on formula-by-formula basis and show that there isn't a uniform lower bound forcing the vc-function to be infinite.
\end{abstract}

\maketitle

%%%%%%%%%%%%%%%%%%%%%%%%%%%%%%%%%%%%%%%%%%%%%%%%%%%%%%%%%%%%%%%%%%%%%%%%%%%%%%%%%%%%%%%%%%%%%%%%%%%%%%%%%%%%%%%%% 

VC-density was studied in \cite{density} by Aschenbrenner, Dolich, Haskell, MacPherson, and Starchenko as a natural notion of dimension for NIP theories.
In a complete NIP theory $T$ we can define the vc-function

\begin{align*}
  \vc^T = \vc : \N \arr \R \cup \curly{\infty}
\end{align*}

where $\vc(n)$ measures the worst-case complexity of families of definable sets in an $n$-fold Cartesian power of the underlying set of a model of $T$
(see \ref{vc_fn_def} below for a precise definition of $\vc^T$).
The simplest possible behavior is $\vc(n) = n$ for all $n$. Theories with the property that $\vc(1) = 1$ are known to be dp-minimal, i.e., having the smallest possible dp-rank. It is not known whether there can be a dp-minimal theory which doesn't satisfy $\vc(n)=n$
(see \cite{density}, diagram on pg. 41).

In this paper, we investigate vc-density of definable sets in Shelah-Spencer graphs.
In our description of Shelah-Spencer graphs we follow closely the treatment in \cite{laskowski}.
A Shelah-Spencer graph is a limit of random structures $G(n, n^{-\alpha})$ for an irrational $\alpha \in (0,1)$.
$G(n, n^{-\alpha})$ is a random graph on $n$ vertices with edge probability $n^{-\alpha}$.

Our first result is that in Shelah-Spencer graphs
\begin{align*}
    \vc(n) = \infty
\end{align*}
which implies that they are not dp-minimal.
Our second result is providing an upper bound on a vc-density for a formula $\phi$
\begin{align*}
    \vc(\phi) \leq K(\phi) \frac{Y(\phi)}{\epsilon(\phi)}    
\end{align*}
where $K(\phi), Y(\phi), \epsilon(\phi)$ are paramters easily computable from the quantifier free form of $\phi$.

Chapter 1 introduces basic facts about VC-dimension and vc-density.
More can be found in \cite{density}.
Chapter 2 summarizes notation and basic facts concerning Shelah-Spencer graphs.
We direct the reader to \cite{laskowski} for a more in-depth treatment.
In chapter 3 we introduce some measure of dimension for quantifier free formulas as well as proving some elementary facts about it.
Chapter 4 computes a lower bound for vc-density to demonstrate that $\vc(n) = \infty$.
Chapter 5 computes an upper bound for vc-density on a formula-by-formula basis.

% This struc is axiomatized by $S_\alpha$.
% Our ambient model is $\MM$.
% Notations we use are $\delta(\A), \delta(\A/\B), \A \leq \B$ as well as notions of $N$-strong substructure, minimal extension, chain minimal extension, minimal pair, and $N$-strong closure.

\section{VC-dimension and VC-density}
Throughout this section we work with a collection $\F$ of subsets of an infinite set $X$.
We call the pair $(X, \F)$ a \defn{set system}.

\begin{Definition} \ 
  \begin{itemize} 
  \item Given a subset $A$ of $X$, we define the set system $(A, A \cap \F)$
    where $A \cap \F = \curly{A \cap F \mid F\in \F}$.
  \item For $A \subseteq X$ we say that $\F$ \defn{shatters} $A$ if $A \cap \F = \PP(A)$ (the power set of $A$).
  \end{itemize}    
\end{Definition}  

\begin{Definition}
  We say $(X, \F)$ has \defn{VC-dimension} $n$ if the largest subset of $X$ shattered by $\F$ is of size $n$.
  If $\F$ shatters arbitrarily large subsets of $X$, we say that $(X, \F)$ has infinite VC-dimension.
  We denote the VC-dimension of $(X, \F)$ by $\VC(X, \F)$.
\end{Definition}  

\begin{Note}
  We may drop $X$ from the notation $\VC(X, \F)$, as the VC-dimension doesn't depend on the base set and is determined by $(\bigcup \F, \F)$.
\end{Note}
Set systems of finite VC-dimension tend to have good combinatorial properties,
and we consider set systems with infinite VC-dimension to be poorly behaved.

Another natural combinatorial notion is that of the dual system of a set system:
\begin{Definition}
  For $a \in X$ define $X_a = \curly{F \in \F \mid a \in F}$.
  Let $\F^* = \curly{X_a \mid a \in X}$.
  We call $(\F, \F^*)$ the \defn{dual system} of $(X, \F)$.
  The VC-dimension of the dual system of $(X, \F)$ is referred to as the \defn{dual VC-dimension} of $(X, \F)$ and denoted by $\VC^*(\F)$.
  (As before, this notion doesn't depend on $X$.)
\end{Definition}  

\begin{Lemma} [see 2.13b in \cite{ash7}]
  A set system $(X, \F)$ has finite VC-dimension if and only if its dual system has finite VC-dimension.
  More precisely
  \begin{align*}
    \VC^*(\F) \leq 2^{1+\VC(\F)}.
  \end{align*}
\end{Lemma}

For a more refined notion of complexity of $(X, \F)$ we look at the traces of our family on finite sets:
\begin{Definition}
  Define the \defn{shatter function} $\pi_\F \colon \N \arr \N$ of $\F$ and the \defn{dual shatter function} $\pi^*_\F \colon \N \arr \N$ of $\F$ by 
  \begin{align*}
    \pi_\F(n) &= \max \curly{|A \cap \F| \mid A \subseteq X \text{ and } |A| = n} \\
    \pi^*_\F(n) &= \max \curly{\text{atoms($B$)} \mid B \subseteq \F, |B| = n}
  \end{align*}
  where atoms($B$) is the number of atoms in the boolean algebra of sets generated by $B$.
  Note that the dual shatter function is precisely the shatter function of the dual system: $\pi^*_\F = \pi_{\F^*}$.
\end{Definition}  

A simple upper bound is $\pi_\F(n) \leq 2^n$ (same for the dual).
If the VC-dimension of $\F$ is infinite then clearly $\pi_\F(n) = 2^n$ for all $n$. Conversely we have the following remarkable fact:
\begin{Theorem} [Sauer-Shelah '72, see \cite{sauer}, \cite{shelah}]
  If the set system $(X, \F)$ has finite VC-dimension $d$ then $\pi_\F(n) \leq {n \choose \leq d}$ for all $n$, where
  ${n \choose \leq d} = {n \choose d} + {n \choose d - 1} + \ldots + {n \choose 1}$.    
\end{Theorem}

Thus the systems with a finite VC-dimension are precisely the systems where the shatter function grows polynomially.
The VC-density of $\F$ quantifies the growth of the shatter function of $\F$: 
\begin{Definition}
  Define the \defn{VC-density} and \defn{dual VC-density} of $\F$ as
  \begin{align*}
    \vc(\F) &= \limsup_{n \to \infty}\frac{\log \pi_\F(n)}{\log n} \in \R^{\geq 0} \cup \curly{+\infty},\\
    \vc^*(\F) &= \limsup_{n \to \infty}\frac{\log \pi^*_\F(n)}{\log n}\in \R^{\geq 0} \cup \curly{+\infty}.
  \end{align*}
\end{Definition}

Generally speaking a shatter function that is bounded by a polynomial doesn't itself have to be a polynomial.
Proposition 4.12 in \cite{density} gives an example of a shatter function that grows like $n \log n$ (so it has VC-density $1$).

So far the notions that we have defined are purely combinatorial.
We now adapt VC-dimension and VC-density to the model theoretic context.

\begin{Definition}
  Work in a first-order structure $\MM$.
  Fix a finite collection of formulas $\Phi(x, y)$ in the language $\LL(M)$ of $\MM$.

  \begin{itemize}
  \item For $\phi(x, y) \in \LL(M)$ and $b \in M^{|y|}$ let 
    \begin{align*}
      \phi(M^{|x|}, b) = \{a \in M^{|x|} \mid \phi(a, b)\} \subseteq M^{|x|}.
    \end{align*}
  \item Let $\Phi(M^{|x|}, M^{|y|})= \{\phi(M^{|x|}, b) \mid \phi \in \Phi, b \in M^{|y|}\} \subseteq \PP(M^{|x|})$.
  \item Let $\F_\Phi = \Phi(M^{|x|}, M^{|y|})$, giving rise to a set system $(M^{|x|}, \F_\Phi)$.
  \item Define the \defn{VC-dimension} $\VC(\Phi)$ of $\Phi$ to be the VC-dimension of $(M^{|x|}, \F_\Phi)$, similarly for the dual.
  \item Define the \defn{VC-density} $\vc(\Phi)$ of $\Phi$ to be the VC-density of $(M^{|x|}, \F_\Phi)$, similarly for the dual.
  \end{itemize}

  We will also refer to the VC-density and VC-dimension of a single formula $\phi$
  viewing it as a one element collection $\Phi = \curly{\phi}$.
\end{Definition}

Counting atoms of a boolean algebra in a model theoretic setting corresponds to counting types,
so it is instructive to rewrite the shatter function in terms of types.

\begin{Definition} 
  \begin{align*}
    \pi^*_\Phi(n) &= \max \curly{\text{number of $\Phi$-types over $B$} \mid B \subseteq M, |B| = n}.
  \end{align*}
  Here a $\Phi$-type over $B$ is a maximal consistent collection of formulas of the form $\phi(x, b)$ or $\neg\phi(x, b)$
  where $\phi \in \Phi$ and $b \in B$.
\end{Definition}

The functions $\pi^*_{\Phi}$ and $\pi^*_{\F_\Phi}$ do not have to agree,
as one fixes the number of generators of a boolean algebra of sets and the other fixes the size of the parameter set.
However, as the following lemma demonstrates, they both give the same asymptotic definition of dual VC-density.

\begin{Lemma} \label{count_types}
  \begin{align*}
    \vc^*(\Phi) &= \text{degree of polynomial growth of $\pi^*_\Phi(n)$}  = \limsup_{n \to \infty}\frac{\log \pi^*_\Phi(n)}{\log n}.
  \end{align*}  
\end{Lemma}

\begin{proof}
  With a parameter set $B$ of size $n$, we get at most $|\Phi|n$ sets $\phi(M^{|x|}, b)$ with $\phi \in \Phi, b \in B$.
  We check that asymptotically it doesn't matter whether we look at growth of boolean algebra of sets generated by
  $n$ or by $|\Phi|n$ many sets.
  We have:
  \begin{align*}
    \pi^*_{\F_\Phi}\paren{n} \leq \pi^*_\Phi(n) \leq \pi^*_{\F_\Phi}\paren{|\Phi|n}.
  \end{align*}
  Hence:
  \begin{align*}
    &\vc^*(\Phi) \leq \limsup_{n \to \infty}\frac{\log \pi^*_\Phi(n)}{\log n} \leq \limsup_{n \to \infty}\frac{\log \pi^*_{\F_\Phi}\paren{|\Phi|n}}{\log n} = \\
    & = \limsup_{n \to \infty}\frac{\log \pi^*_{\F_\Phi}\paren{|\Phi|n}}{\log |\Phi|n} \frac{\log |\Phi|n}{\log n} =
      \limsup_{n \to \infty}\frac{\log \pi^*_{\F_\Phi}\paren{|\Phi|n}}{\log |\Phi|n} \leq \\
    &\leq \limsup_{n \to \infty}\frac{\log \pi^*_{\F_\Phi}\paren{n}}{\log n} = \vc^*(\Phi).
  \end{align*}
\end{proof} 

One can check that the shatter function and hence VC-dimension and VC-density of a formula are elementary notions,
so they only depend on the first-order theory of the structure $\MM$.

NIP theories are a natural context for studying VC-density.
In fact we can take the following as the definition of NIP:
\begin{Definition}
  Define $\phi$ to be NIP if it has finite VC-dimension in a theory $T$.
  A theory $T$ is NIP if all the formulas in $T$ are NIP.
\end{Definition}

In a general combinatorial context (for arbitrary set systems),
VC-density can be any real number in $0 \cup [1, \infty)$ (see \cite{ash8}).
Less is known if we restrict our attention to NIP theories.
Proposition 4.6 in \cite{density} gives examples of formulas that have non-integer rational VC-density in an NIP theory,
however it is open whether one can get an irrational VC-density in this model-theoretic setting.

Instead of working with a theory formula by formula, we can look for a uniform bound for all formulas:
\begin{Definition} \label{vc_fn_def}
  For a given NIP structure $\MM$, define the \defn{VC-density function}
  \begin{align*}
    \vc^\MM(n) &= \sup \{\vc^*(\phi(x, y)) \mid \phi \in \LL(M), |x| = n\} \\
             &= \sup \{\vc(\phi(x, y)) \mid \phi \in \LL(M), |y| = n\} \in \R^{\geq 0} \cup \curly{+\infty}.
  \end{align*}
\end{Definition}

As before this definition is elementary, so it only depends on the theory of $\MM$.
We omit the superscript $\MM$ if it is understood from the context.
One can easily check the following bounds:
\begin{Lemma} [Lemma 3.22 in \cite{density}] \label{vcone}
  We have $\vc(1) \geq 1$ and $\vc(n) \geq n\vc(1)$.  
\end{Lemma}

However, it is not known whether the second inequality can be strict or even just whether $\vc(1) < \infty$ implies $\vc(n) < \infty$.


%%%%%%%%%%%%%%%%%%%%%%%%%%%%%%%%%%%%%%%%%%%%%%%%%%%%%%%%%%%%%%%%%%%%%%%%%%%%%%%%%%%%%%%%%%%%%%%%%%%%%%%%%%%%%%%%% 
\section{Graph Combinatorics}

We denote graph by $\A$, set of its vertices by $A$.
\begin{Definition}
  Fix $\alpha \in (0,1)$, irrational.
  \begin{itemize}
  \item For a finite graph $\A$ let
    \begin{align*}
      \delta(\A) = |A| - \alpha e(\A)
    \end{align*}
    where $e(\A)$ is the number of edges in $\A$.

  \item For finite $\A,\B$ with $\A \subseteq \B$ define $\delta(\B/\A) = \delta(\B) - \delta(\A)$.
  \item We say that $\A \leq \B$ if $\A \subseteq \B$ and $\delta(\A'/\B) > 0$ for all $\A \subseteq \A' \subsetneq \B$.

  \item We say that finite $\A$ is \defn{positive} if for all $\A' \subseteq \A$ we have $\delta(\A') \geq 0$.

  \item We work in theory $S_\alpha$ axiomatized by
    \begin{itemize}
    \item Every finite substructure is positive.
    \item For a model $\MM$ given $\A \leq \B$ every embedding $f : \A \arr \MM$ extends to $g: \B \arr \MM$.
    \end{itemize}

  \item For $\A, \B$ positive, $(\A, \B)$ is called a \defn{minimal pair} if
    $\A \subseteq \B$, $\delta(\B/\A) < 0$ but $\delta(\A'/\A) \geq 0$ for all proper $\A \subseteq \A' \subsetneq \B$.

  \item $\agl{\A_i}_{i \leq m}$ is called a \defn{minimal chain} if $(\A_i, \A_i+1)$ is a minimal pair (for all $i < m$).

  \item For a positive $\A$ let $\delta_\A(\bar x)$ be the atomic diagram of $\A$. For positive $\A \subset \B$ let 
    \begin{align*}
      \Psi_{\A,\B}(\bar x) = \delta_\A(\bar x) \wedge \exists \bar y \; \delta_\B(\bar x, \bar y).
    \end{align*}
    Such formula is called a \defn{chain-minimal extension formula} if in addition we have that there is a minimal chain starting at 
    $\A$ and ending in $\B$.
    Denote such formulas as $\Psi_{\agl{\M_i}}$.
  \end{itemize}
\end{Definition}

\begin{Theorem} [5.6 in \cite{laskowski}]
  $S_\alpha$ admits quantifier elimination down to boolean combination of chain-minimal extension formulas.
\end{Theorem}


%%%%%%%%%%%%%%%%%%%%%%%%%%%%%%%%%%%%%%%%%%%%%%%%%%%%%%%%%%%%%%%%%%%%%%%%%%%%%%%%%%%%%%%%%%%%%%%%%%%%%%%%%%%%%%%%% 
\section{Basic Definitions and Lemmas}

Fix tuples $x = (x_1, \ldots x_n), y = (y_1, \ldots, y_m)$.
We refer to chain-minimal extension formulas as basic formulas.
Let $\phi_{\agl{\M_i}}(x, y)$ be a basic formula.

\begin{Definition}
  Define $\X$ to be the graph on vertices $\{x_i\}$ with edges as defined by $\phi_{\agl{\M_i}}$.
  Similarly define $\Y$.
  We define those abstractly, i.e. on a new set of vertices disjoint from $\MM$.
\end{Definition}

Note that $\X$, $\Y$ are positive as they are subgraphs of $\M_0$.
As usual $X, Y$ will refer to vertices of those graphs.

We restrict our attention to formulas that define no edges between $X$ and $Y$.

\begin{Note} \label{note_edges}
  We can handle edges between $x$ and $y$ as separate elements of the minimal chain extension.
\end{Note}

\begin{Definition} \label{def_basic}
  For a basic formula $\phi = \phi_{\agl{\M_i}_{i \leq k}}(x, y)$ let
  \begin{itemize}
  \item $\epsilon_i(\phi) = -\dim \paren{M_i/M_{i-1}}$.
  \item $\epsilon_L(\phi) = \sum_{[1..k]} \epsilon_i(\phi)$.
  \item $\epsilon_U(\phi) = \min_{[1..k]} \epsilon_i(\phi)$.
  \item Let $\Y'$ be a subgraph of $\Y$ induced by vertices of $\Y$ that are connected to $M_k - (X \cup Y)$.
  \item Let $Y(\phi) = \dim (\Y')$.
    In particular if $\Y = \Y'$ and $\Y$ is disconnected then $Y(\phi)$ is just the arity of the tuple $y$.
  \end{itemize}
\end{Definition}

We conclude this section by stating a couple of technical lemmas that will be useful in our proofs later.
\begin{Lemma}
  Suppose we have a set $B$ and a minimal pair $(M, A)$ with $A \subset B$ and $\dim(M/A) = -\epsilon$.
  Then either $M \subseteq B$ or $\dim((M \cup B)/B) < -\epsilon$.
\end{Lemma}

\begin{proof}
  By diamond construction
  \begin{align*}
    \dim((M \cup B)/B) \leq \dim(M / (M \cap B))
  \end{align*}
  and 
  \begin{align*}
    \dim(M / (M \cap B)) &= \dim (M/A) - \dim(M / (M \cap B)) \\
    \dim (M/A) &= -\epsilon \\
    \dim(M / (M \cap B)) &> 0
  \end{align*}
\end{proof}



\begin{Lemma}	\label{chain_lemma}
  Suppose we have a set $B$ and a minimal chain $M_n$ with $M_0 \subset B$ and dimensions $-\epsilon_i$.
  Let $\epsilon$ be the minimal of $\epsilon_i$.
  Then either $M_n \subseteq B$ or $\dim((M_n \cup B)/B) < -\epsilon$.
\end{Lemma}


\begin{proof}
  Let $\bar M_i = M_i \cup B$

  \begin{align*}
    \dim(\bar M_n/B) = \dim(\bar M_n/\bar M_{n-1}) + \ldots + \dim(\bar M_2/\bar M_1) + \dim(\bar M_1/B)
  \end{align*}

  Either $M_n \subseteq B$ or one of the summands above is nonzero.
  Apply previous lemma.
\end{proof}

\begin{Lemma} \label{chain_intersect}
  Suppose we have a minimal chain $M_n$ with dimensions $-\epsilon_i$.
  Let $\epsilon$ be the sum of all $\epsilon_i$.
  Suppose we have some $B$ with $B \subseteq M_n$.
  Then $\dim B / (M_0 \cap B) \geq -\epsilon$.
\end{Lemma}

\begin{proof}
  Let $B_i = B \cap M_i$.
  We have $\dim B_{i+1}/B_i \geq \dim M_{i+1}/M_i$ by minimality.
  $\dim B / (M_0 \cap B) = \dim B_n / B_0 = \sum \dim B_{i+1}/B_i \geq -\epsilon$.
\end{proof}

%%%%%%%%%%%%%%%%%%%%%%%%%%%%%%%%%%%%%%%%%%%%%%%%%%%%%%%%%%%%%%%%%%%%%%%%%%%%%%%%%%%%%%%%%%%%%%%%%%%%%%%%%%%%%%%%% 
\section{Lower bound}

As a simplification for our lower bound computation we assume that all the basic formulas involved we have $\Y' = \Y$ (see Definition \ref{def_basic}).

We work with formulas that are boolean combinations of basic formulas written in disjunctive-conjunctive form.
First, we extend our definition of $\epsilon$.

\begin{Definition}[Negation]
  If $\phi$ is a basic formula, then define
  \begin{align*}
    \epsilon_L(\neg \phi) &= \epsilon_L(\phi)
  \end{align*}
\end{Definition}

\begin{Definition}[Conjunction]
  Take a collection of formulas $\phi_i(x, y)$ where each $\phi_i$ is positive or negative basic formula.
  If both positive and negative formulas are present then $\epsilon_L(\phi) = \infty$.
  We don't have a lower bound for that case.
  If different formulas define $\X$ or $\Y$ differently then $\epsilon_L(\phi) = \infty$.
  In that case of the conflicting definitions would make the formula have no realizations.
  Otherwise
  \begin{align*}
    \epsilon_L(\bigwedge \phi_i) &= \sum \epsilon_L(\phi_i)
  \end{align*}
\end{Definition}

\begin{Definition} [Disjunction]
  Take a collection of formulas $\psi_i$ where each instance is a conjunction of positive and negative instances of basic formulas that agree on $\X$ and $\Y$. % can generalize?
  \begin{align*}
    \epsilon_L(\bigvee \psi_i) &= \min \epsilon_L(\psi_i).
  \end{align*}
\end{Definition}

\begin{Theorem}
  For a formula $\phi$ as above
  \begin{align*}
    \vc \phi \geq \floor{\frac{Y(\phi)}{\epsilon_L(\phi)}}
  \end{align*}
  where $Y(\phi)$ is $Y(\psi)$ for $\psi$ one the basic components of $\phi$ (all basic componenets agree on $\Y$).
\end{Theorem}

\begin{proof}
  First work with a formula that is a conjunction of positive basic formulas

  \begin{align*}
    \psi = \bigwedge_{j \leq J} \phi_j.
  \end{align*}
  Then as we defined above
  \begin{align*}
    \epsilon_L(\psi) = \sum \epsilon_L(\phi_j)
  \end{align*}
  
  Let $\phi$ be one of the basic formulas in $\psi$ with a chain $\agl{M_i}_{i \leq k}$.
  Let $K_\phi = |M_k|$ i.e. the size of the extension.
  Let $K$ be the largest such size among all $\phi_i$.

  Let $n$ be the integer such that $n \epsilon_L(\psi) < Y$ and $(n+1) \epsilon_L(\psi) > Y$.

  Label $\Y$ by an tuple $b$.

  Pick parameter set $A \subset \MM$ such that 

  \begin{align*}
    A = \bigcup_{i<N} b_i
  \end{align*}

  a disjoint union where each $b_i$ is an ordered tuple of size $|x|$ connected according to $\psi$.
  We also require $A$ to be $N \cdot I \cdot K$-strong.

  Fix $n$ arbitrary elements out of $A$, label them $a_j$.

  For each $\phi_i$, $a_j$ pick an abstract realization $M_{ij}$ of $\phi_i$ over $(a_j, b)$
  (abstract meaning disjoint from $\MM$).
  
  Let $\bar M$ be an abstract disjoint union of all those realizations.

  \begin{Claim}
    $(A \cap \bar M) \leq \bar M$.
  \end{Claim}
  \begin{proof}
    Consider some $(A \cap \bar M) \subseteq B \subseteq \bar M$.
    Let $B_{ij} = B \cap M_{ij} \subseteq M_{ij}$.
    Then $B_{ij}$'s are disjoint over $\bar A = A \cup b$.
    In particular $\dim B / (\bar A \cap B) = \sum \dim B_{ij} / (\bar A \cap B_{ij})$.
    $\dim B_{ij} / \bar A \geq -\epsilon_L(\phi_i)$ by Lemma \ref{chain_intersect}.
    Thus $\dim B / (\bar A \cap B) \geq -n\epsilon(\psi)$.
    Thus $\dim B / (A \cap B) \geq \dim(B) - n\epsilon(\psi)$.
    By construction we have $Y - n\epsilon_L(\psi) > 0$ as needed.
  \end{proof}

  $|\bar M| \leq N \cdot I \cdot K$ and $A$ is $\leq N \cdot I \cdot K$-strong.
  Thus a copy of $\bar M$ can be embedded over $A$ into our ambient model $\MM$.
  Our choice of $b_i$'s was arbitrary, so we get ${N \choose n}$ choices out of $N|x|$ many elements.
  Thus we have $O(|A|^n)$ many traces.

  \begin{Lemma}
    There are arbitrarily large sets with properties of $A$.
  \end{Lemma}
  
  \begin{proof}
    $A$ is positive, as each of its disjoint components is positive. Thus $0 \leq A$.
    We apply proposition 4.4 in Laskoswki paper, embedding $A$ into our structure $\MM$ while avoiding all nonpositive extensions of size at most $N \cdot I \cdot K$.
  \end{proof}

  This shows

  \begin{align*}
    \vc \psi \geq n = \floor{\frac{Y}{\epsilon_L}}
  \end{align*}

  Now consider the formula which is a conjunction consists of negative basic formulas
  \begin{align*}
    \psi = \bigwedge \neg \phi_i
  \end{align*}
  Let
  \begin{align*}
    \bar \psi = \bigwedge \phi_i
  \end{align*}

  Do the construction above for $\bar \psi$ and suppose its trace is $X \subset A$ for some $b$.
  Then over $b$ the same construction gives trace $(A - X)$ for $\psi$. Thus we get as many traces.
  
  Finally consider a formula which is a disjunction of formulas considered above.
  Choose the one with the smallest $\epsilon_L$, this yields the lower bound for the entire formula.
\end{proof}

\begin{Claim}[4.1 in \cite{laskowski}]
  We can find a minimal extension $M$ with arbitrarily small dimension.
\end{Claim}

\begin{Corollary}
  This shows that the vc-function is infinite in Shelah-Spencer random graphs.
  \begin{align*}
    \vc(n) = \infty
  \end{align*}
  In particular, this implies that Shelah-Spencer graphs are not dp-minimal.
\end{Corollary}

%%%%%%%%%%%%%%%%%%%%%%%%%%%%%%%%%%%%%%%%%%%%%%%%%%%%%%%%%%%%%%%%%%%%%%%%%%%%%%%%%%%%%%%%%%%%%%%%%%%%%%%%%%%%%%%%% 
\section{Upper bound}

We bound the number of types of some finite collection of formulas $\Psi(\vec x, \vec y) = \curly{\phi_i(\vec x, \vec y)}_{i\in I}$ over a parameter set $B$ of size $N$,
where $\phi_i$ is a basic formula.

Fix a formula $\phi$ from our collection.
Suppose it defines a minimal chain extension over $\{x, y\}$. 
Record the size of that extension as $K(\phi)$ and its total dimension $\epsilon(\phi) = \epsilon_U(\phi)$.
Define dimension of that formula $D(\phi) = |\vec y| \frac{K(\phi)}{\epsilon(\phi)}$
Define dimension of the entire collection as $D(\Psi) = \max_{i \in I} D(\phi_i)$

In general we have parameter set $B \subset \MM^{|y|}$, however without loss of generality we may work with
a parameter set $B^{|y|}$, with $B \subset \MM$.

Let $S = \floor{D(\Psi)}$.

For our proof to work we also need $B$ to be $S$-strong.
We can achieve this by taking (the unique) $S$-strong closure of $B$.
If size of $B$ is $N$ then the size of its closure is $O(N)$.	%elaborate
So without loss of generality we can assume that $B$ is $S$-strong.

\begin{Definition}
  A \defn{witness} of $a$ is a union of realizations of the existential formulas $\phi_i(a, b)$ for all $i, b$ so that the formula holds.
\end{Definition}

\begin{Definition}
  For sets $C, B$ define the boundary of $C$ over $B$
  \begin{align*}
    \partial(C, B) = \curly{b \in B \mid \text{there is an edge between $b$ and element of $C - B$}}
  \end{align*}
\end{Definition}

\begin{Definition}
  For each $a$ pick some $\bar M_a$ to be its witness.
  Define two quantities
  \begin{itemize}
  \item $\partial_a$ is the boundary $\partial(\bar M_a, B \cup a)$
  \item Suppose $G_1, G_2$ are some subgraphs of our model and $a_1 \subset G_1, a_2 \subset G_2$ finite tuples of vertices.
    Call $f \colon (G_1, a_1) \arr (G_2, a_2)$ a $\partial$-isomorphism if it is a graph isomorphism,
    $f$ and $f^{-1}$ are constant on $B$, and
    $f(a_1) = a_2$.
  \item Define $\II_a$ as the $\partial$-isomorphism class of $(\bar M_a, a)$.
  \end{itemize}
\end{Definition}

\begin{Lemma} \label {bound_trace}
  If $\II_{a_1} = \II_{a_2}$ then $a_1, a_2$ have the same $\Psi$-type over $B$.
\end{Lemma}

\begin{proof}
  Fix a $\partial$-isomorphism $f \colon (\bar M_{a_1}, a_1) \arr (\bar M_{a_1}, a_2)$.
  Suppose we have $\phi(a_1, b)$ for some $b \in B$.
  Pick witness of this existential formula $M_1 \subset \bar M_{a_1}$.
  Then $f(M_1)$ is a witness for  $\phi(a_2, b)$.
\end{proof}

Thus to bound the number of traces it is sufficient to bound the number of possibilities for $\II_a$.

\begin{Theorem} \label{main_bound}
  \begin{align*}
    |\partial_a| &\leq D(\Psi) \\ 
    |\bar M_b - \bar A| &\leq D(\Psi)
  \end{align*}
\end{Theorem}

\begin{Corollary}
  \begin{align*}
    \vc(\phi) \leq K(\phi) \frac{Y(\phi)}{\epsilon(\phi)}
  \end{align*}
\end{Corollary}

\begin{proof}
  We count possible $\partial$-isomorphism classes $\II_b$.
  Let $W = K(\phi) \frac{Y(\phi)}{\epsilon(\phi)}$.
  If the parameter set $A$ is of size $N$ then there are $N \choose W$ choices for boundary $\partial_b$.
  On top of the boundary there are at most $W$ extra vertices and $(2W)^2$ extra edges.
  Thus there are at most
  \begin{align*}
    W \cdot 2^{(2W)^2}
  \end{align*}
  configurations up to a graph isomorphism.
  In total this gives us 
  \begin{align*}
    {N \choose W} \cdot W \cdot 2^{(2W)^2} = O(N^W)
  \end{align*}
  options for $\partial$-isomorphism classes.
  By Lemma \ref{bound_trace} there are at most $O(N^W)$ many traces, giving the required bound.
\end{proof}

\begin{proof} \textit{(of Theorem \ref{main_bound})}
  Fix some $b$-trace $A_b$. Enumerate $A_b = \{a_1, \ldots, a_I\}$.

  Let $M_i / \{a_i, b\}$ be a witness of $\phi(a_i, b)$ for each $i \leq I$.
  Let $\bar M_i = \bigcup_{j < i} M_j$.
  Let $\bar M = \bigcup M_i$, a witness of $A_b$
  
  \begin{Claim}
    \begin{align*}
      &\abs{\partial(M_i M, \bar A) - \partial(M, \bar A)} \leq |M_i| = K(\phi)\\
      &\dim(M_i M \bar A / M \bar A) > -\epsilon(\phi)
    \end{align*}
  \end{Claim}
  
  \begin{Definition}
    $(j-1, j)$ is called a \defn{jump} if some of the following conditions happen
    \begin{itemize}
    \item New vertices are added outside of $\bar A$ i.e.
      \begin{align*}
        \bar M_j - \bar A \neq \bar M_{j-1} - \bar A
      \end{align*}
    \item New vertices are added to the boundary, i.e.
      \begin{align*}
        \partial(\bar M_j, \bar A) \neq \partial(\bar M_{j-1}, \bar A)
      \end{align*}
    \end{itemize}
  \end{Definition}

  \begin{Definition}
    We now let $m_i$ count all jumps below $i$
    % Let $d_i = \dim(\bar M_i/A)$.
    \begin{align*}
      m_i = \abs{\curly{j < i \mid (j-1, j) \text{ is a jump}}}
    \end{align*}
  \end{Definition}

  \begin{Lemma} \label{ub_lemma}
    \begin{align*}
      \dim(\bar M_i / \bar A) &\leq -m_i \cdot \epsilon(\phi) \\
      |\partial(\bar M_i, \bar A)| &\leq m_i \cdot K(\phi) \\
      |\bar M_j - \bar A| &\leq m_i \cdot K(\phi)
    \end{align*}
  \end{Lemma}

  \begin{proof} \textit{(of Lemma \ref{ub_lemma})}
    Proceed by induction.
    Second and third propositions are clear.
    For the first proposition base case is clear.
    
    Induction step.
    Suppose $\bar M_j \cap (A \cup b) = \bar M_{j+1}$ and $\partial(\bar M_j, A) = \partial(\bar M_{j+1}, A)$.
    Then $m_i = m_{i+1}$ and the quantities don't change.
    Thus assume at least one of these equalities fails.
    
    Apply Lemma \ref{chain_lemma} to $\bar M_j \cup (A \cup b)$ and $(M_{j+1}, a_{j+1}b)$.
    There are two options
    
    \begin{itemize}
    \item $\dim(\bar M_{j+1} \cup (A \cup b) / \bar M_i \cup (A \cup b)) \leq -\epsilon_U$.
      This implies the proposition.
    \item $M_{j+1} \subset \bar M_j \cup (A \cup b)$.
      Then by our assumption it has to be $\partial(\bar M_j, A) \neq \partial(\bar M_{j+1}, A)$.
      There are edges between $M_{j+1} \cap (\partial(\bar M_{j+1}, A) - \partial(\bar M_j, A))$ so they contribute some negative dimension $\leq \epsilon_U$.
    \end{itemize}
    This ends the proof for Lemma \ref{ub_lemma}.
  \end{proof}
  \textit{(Proof of Theorem \ref{main_bound} continued)}
  First part of lemma \ref{ub_lemma} implies that we have $\dim(\bar M / \bar A) \leq -m_I \cdot \epsilon(\phi)$.
  The requirement of $A$ to be $S$-strong forces
  \begin{align*}
    m_I \cdot \epsilon(\phi) &< Y(\phi) \\
    m_I  &< \frac{Y(\phi)}{\epsilon(\phi)} \\
  \end{align*}
  % Let $W = \frac{K(\phi)Y(\phi)}{\epsilon(\phi)}$
  Applying the rest of \ref{ub_lemma} gives us
  \begin{align*}
    |\partial(\bar M, A)| &\leq m_I \cdot K(\phi) \leq \frac{K(\phi)Y(\phi)}{\epsilon(\phi)} \\
    |\bar M \cap A| &\leq m_I \cdot K(\phi) \leq \frac{K(\phi)Y(\phi)}{\epsilon(\phi)}
  \end{align*}
  as needed.
  This ends the proof for Theorem \ref{main_bound}.
\end{proof}

So far we have computed an upper bound for a single basic formula $\phi$.

To bound an arbitrary formula, write it as a boolean combination of basic formulas $\phi_i$ (via quantifier elimination)
It suffices to bound vc-density for collection of formulas $\{\phi_i\}$ to obtain a bound for the original formula.

In general work with a collection of basic formulas $\{\phi_i\}_{i \in I}$.
The proof generalizes in a straightforward manner.
Instead of $A^{|x|}$ we now work with $A^{|x|} \times I$ separating traces of different formulas.
Formula with the largest quantity $Y(\phi)\frac{K(\phi)}{\epsilon(\phi)}$ contributes the most to the vc-density.
Thus we have
\begin{align*}
  \Phi &= \{\phi_i\}_{i \in I} \\
  \vc(\Phi) &\leq   \max_{i \in I} Y(\phi_i) \frac{K(\phi_i)}{\epsilon_{\phi_i}}
\end{align*}

%%%%%%%%%%%%%%%%%%%%%%%%%%%%%%%%%%%%%%%%%%%%%%%%%%%%%%%%%%%%%%%%%%%%%%%%%%%%%%%%%%%%%%%%%%%%%%%%%%%%%%%%%%%%%%%%% 

\begin{thebibliography}{9}

\bibitem{density}
  M. Aschenbrenner, A. Dolich, D. Haskell, D. Macpherson, S. Starchenko,
  \textit{Vapnik-Chervonenkis density in some theories without the independence property}, I,
  Trans. Amer. Math. Soc. 368 (2016), 5889-5949
  
\bibitem{laskowski}
  Michael C. Laskowski, \textit{A simpler axiomatization of the Shelah-Spencer almost sure theories},
  Israel J. Math. \textbf{161} (2007), 157–186. MR MR2350161	

\end{thebibliography}

\end{document}



% Aschenbrenner et. al. computed a linear bound for the VC-density function in the field of $p$-adic numbers,
% but this bound is not known to be optimal.

% $P$-adic numbers are a simple, yet a very deep construction.
% They were only discovered a hundred years ago, but could have been studied in classical mathematics when number theory was just forming.
% Their construction is simple enough to explain at the undergraduate level, yet has a very rich number theoretic structure.
% Normally the real numbers are constructed by first taking rational numbers in decimal form and allowing infinite decimal sequences after the decimal point.
% Letting decimals be infinite before the decimal point yields a well behaved mathematical object as well, but with a drastically different behavior from real numbers, now depending on the base in which the decimals were written.
% When the base is a prime number $p$, this constructs $p$-adic numbers.
% These were first studied exclusively within number theory, but later found applications in other areas of math, physics, and computer science.

% My research will allow for a finer understanding of the finite structure of polynomially definable sets in $p$-adic numbers.
% Model theory began with G\"odel and Malcev in the 1930s, but first matured as a subject in the work of Abraham Robinson, Tarski, Vaught, and others in the 1950s.
% Model theory studies sets definable by first order formulas in a variety of mathematical objects.
% Restricting to subsets definable by simple formulas gives access to an array of powerful techniques such as indiscernible sequences and nonstandard extensions.
% These allow insights not otherwise accessible by classical methods.
% Nonstandard real numbers, for example, formalize the notion of infinitesimals.
% Model theory is an extremely flexible field with applications in many areas of mathematics including algebra, analysis, geometry, number theory, and combinatorics as well as some applications to computer science and quantum mechanics.

\chapter{An Additive Reduct of the $P$-adic Numbers}
For $\Q_p$ in the language $\LLM$ of Macintyre, the paper \cite{density} computes an upper bound for the VC-density function to be $2n-1$, and it is not known whether it is optimal for $n \geq 2$.
This same bound holds in any reduct of the field of $p$-adic numbers, but one may expect that the simplified structure of suitable reducts would allow a better bound.
In this \chapa we investigate a reduct of the field of $p$-adic numbers and
use a cell decomposition result of Leenknegt to compute an optimal bound for that structure.

In \cite{reduct}, Leenknegt provides a cell decomposition result for a certain $P$-minimal additive reduct of the field of $p$-adic numbers.
Using this result, we improve the bound for the VC-density function, showing that in Leenknegt's structure $\vc(n) = n$ for each $n$.
Using Definition \ref{def_dp} this also gives a proof that this structure is dp-minimal which is more direct than using that the field of $p$-adics is dp-minimal.

In Sec\-tion~1 we recall some basic facts about the theory of $p$-adic numbers.
Here we also introduce the reduct which we will be working with.
Sec\-tion~2 sets up basic definitions and lemmas that will be needed for the proof.
We define trees and intervals and show how they help with VC-density calculations. 
Sec\-tion~3 finishes the proof.
In the concluding section we state open questions and discuss future work.

\section{$P$-adic numbers}
The field $\Q_p$ of $p$-adic numbers is often studied in the language of Macintyre 
  \begin{align*}
	\LLM = \curly{0, 1, +, -, \cdot, |, \{P_n\}_{n \in \N}}
  \end{align*}
which is a language $\curly{0, 1, +, -, \cdot}$ of rings together with unary predicates $P_n$ interpreted in $\Q_p$ so as to satisfy
\begin{align*}
  P_n x \leftrightarrow \exists y \; y^n = x
\end{align*}
and a divisibility relation where $a|b$ holds in $\Q_p$ when $\vval a \leq \vval b$.

Note that $P_n\backslash \curly{0}$ is a multiplicative subgroup of $\Q_p$ with finitely many cosets.
\begin{Theorem} [see \cite{mac76}]
  The $\LLM$-structure $\Q_p$ has quantifier elimination.
\end{Theorem}

There is also a cell decomposition result for definable sets in this structure:
\begin{Definition}
  Define \defn{$k$-cells} recursively as follows.
  A \defn{$0$-cell} is the singleton $\Q_p^0$.
  A \defn{$(k+1)$-cell} is a subset of $\Q_p^{k+1}$ of the following form:
  \begin{align*}
    \curly{(x, t) \in D \times \Q_p \mid \vval a_1(x) \ \square_1 \vval (t - c(x)) \ \square_2 \vval a_2(x), t - c(x) \in \lambda P_n}
  \end{align*}
  where $D$ is a $k$-cell,
  $a_1(x), a_2(x), c(x)$ are definable functions $D \arr \Q_p$,
  each of $\square_i$ is $<, \leq$ or no condition,
  $n \in \N$,
  and
  $\lambda \in \Q_p$.    
\end{Definition}

\begin{Theorem} [see \cite{den84}]
  Any definable subset of $\Q_p^k$ defined by an $\LLM$-formula decomposes into a finite disjoint union of $k$-cells.
\end{Theorem}  

In \cite{density}, Aschenbrenner, Dolich, Haskell, Macpherson, and Starchenko show that $\Q_p$ as $\LLM$-structure satisfies $\vc(n) \leq 2n - 1$
for each $n \geq 1$, however it is not known whether this bound is optimal.

In \cite{reduct}, Leenknegt analyzes the reduct of $\Q_p$ to the language
\begin{align*}
  \LL_{aff}  = \curly{0, 1, +, -, \curly{\bar c}_{c \in \Q_p}, |, \curly{Q_{m,n}}_{m,n\in \N}}
\end{align*}
where $\bar c$ denotes the scalar multiplication by $c$,
$a | b$ as above stands for $\vval a \leq \vval b$,
and $Q_{m,n}$ is a unary predicate interpreted as
\begin{align*}
  Q_{m,n} = \bigcup_{k \in \Z} p^{km} (1 + p^n\Z_p).
\end{align*}
Note that $Q_{m,n} \backslash \curly{0}$ is a subgroup of the multiplicative group of $\Q_p$ with finitely many cosets.
One can check that these extra relation symbols are definable in the $\LLM$-structure $\Q_p$.
The paper \cite{reduct} provides a cell decomposition result based on the following notion of cell:

\begin{Definition} \label{cell}
  A \defn{$0$-cell} is the singleton $\Q_p^0$.
  A \defn{$(k+1)$-cell} is a subset of $\Q_p^{k+1}$ of the following form:
  \begin{align*}
    \curly{(x, t) \in D \times \Q_p \mid \vval a_1(x) \ \square_1 \vval (t - c(x)) \ \square_2 \vval a_2(x), t - c(x) \in \lambda Q_{m,n} }
  \end{align*}
  where $D$ is a $k$-cell, called the \defn{base} of the cell,
  $a_1(x), a_2(x), c(x)$ are polynomials of degree $\leq 1$, called the \defn{defining polynomials},
  each of $\square_1, \square_2$ is $<$ or no condition,
  $m,n \in \N$,
  and
  $\lambda  \in\Q_p$.
  We call $Q_{m,n}$ the \defn{defining predicate} of our cell.
\end{Definition}

\begin{Theorem}[Leenknegt '12] \label{leen}
  Any definable subset of $\Q_p^k$ defined by an $\LL_{aff}$-formula decomposes into a finite disjoint union of $k$-cells.
  %Any formula $\phi(x, t)$ in $(\Q_p, \LL_{aff})$ with $|x| = n$ and $|t| = 1$ decomposes into a union of $(k+1)$-cells.
\end{Theorem}  

Moreover, \cite{reduct} shows that $\LLA$-structure $\Q_p$ is a $P$-minimal reduct
of the $\LLM$-structure $\Q_p$,
that is, the one-variable definable sets of the $\LLA$-structure $\Q_p$
coincide with the one-variable definable sets of the full structure $\LLM$-structure $\Q_p$.

The main result of this \chapa is the computation of the VC-density function for this structure:
\begin{Theorem} \label{main_theorem}
  The $\LLA$-structure $\Q_p$ satisfies $\vc(n) = n$ for all $n$.
\end{Theorem}
 
Unlike the bound on the VC-density function of the $\LLM$-structure $\Q_p$ from \cite{density}
which was obtained via a quantified version of uniform definability of types over finite sets,
we will directly count the number of $\phi$-types over a finite set of parameters.

%%%%%%%%%%%%%%%%%%%%%%%%%%%%%%%% 

\section{Key Lemmas and Definitions}
To show that $\vc(n) = n$ it suffices to bound $\vc^*(\phi) \leq |x|$ for every $\LL_{aff}$-formula $\phi(x, y)$.
Fix such a formula $\phi(x, y)$.
Instead of working with this formula directly, we first simplify it using quantifier elimination.
The required quantifier elimination result can be easily obtained from cell decomposition:
\begin{Lemma} \label {quantifier_elimination}
  Any $\LLA$-formula $\phi(x, y)$ is equivalent in the $\LLA$-structure $\Q_p$ to a boolean combination of formulas from a collection
  \begin{align*}
    \Phi(x, y) = &\curly{\vval (p_i(x) - c_i(y)) < \vval (p_j(x) - c_j(y))}_{i, j \in I} \cup \\
                 &\curly{p_i(x) - c_i(y) \in \lambda_k Q_{m,n}}_{i \in I , k \in K}
  \end{align*}
  of $\LLA$-formulas
  where $I, K$ are finite index sets,
  each $p_i$ is a degree $\leq 1$ polynomial in $x$ without a constant term,
  each $c_i$ is a degree $\leq 1$ polynomial in $y$,
  $m,n \in \N$,
  and
  $\lambda_k \in \Q_p$.
\end{Lemma}

\begin{proof}
  Let $l = |x| + |y|$.
  Using Theorem \ref{leen} partition the subset of $\Q_p^l$ defined by $\phi$ to obtain $\DD^l$, a collection of $l$-cells.
  Let $\DD^{l-1}$ be the collection of the bases of the cells in $\DD^l$.
  Similarly, construct by induction $\DD^i$ for each $0 \leq j < l$,
  where $\DD^j$ is the collection of $j$-cells which are the bases of cells in $\DD^{j+1}$.
  Set
  \begin{align*}
    &m = \prod \curly{m' \mid Q_{m',n'} \text{ is the defining predicate of a cell in $\DD^j$ for $0 \leq j \leq l$} }, \\
    &n = \max \curly{n' \mid Q_{m',n'} \text{ is the defining predicate of a cell in $\DD^j$ for $0 \leq j \leq l$} }.
  \end{align*}
  % Choose $m,n$ large enough to cover all $m', n'$ for $Q_{m',n'}$ that show up in the cells of $\DD$.
  This way, if $a, a'$ are in the same coset of the definable predicate
  $Q_{m',n'}$ of a cell in $\DD^j$ ($0 \leq j \leq l$),
  then they are in the same coset of $Q_{m,n}$.
  Choose $\curly{\lambda_k}_{k \in K}$ to range over all representations of cosets of $Q_{m,n}$.
  Let $q_i(x, y)$ enumerate all of the defining polynomials $a_1(x), a_2(x), t - c(x)$ that show up in the cells of $\DD^j$ for any $j$.
  All of those are polynomials of degree $\leq 1$ in the variables $x, y$.
  We can split each of them as $q_i(x,y) = p_i(x) - c_i(y)$ where the constant term of $q_i$ is substituted by $c_i$.
  This gives us the appropriate finite collection $\Phi$ of formulas.
  From the cell decomposition it is easy to see that when $a, a'$ have the same $\Phi$-type,
  then they have the same $\phi$-type.
  Thus $\phi$ can be written as a boolean combination of formulas from $\Phi$.
\end{proof}

\begin{Lemma}
  Let $\Phi(x, y)$ be a finite collection of formulas.
  If $\phi$ can be written as a boolean combination of formulas from $\Phi$ then $\vc^* (\phi) \leq \vc^* (\Phi)$.
\end{Lemma}
\begin{proof}
  If $a,a'$ have the same $\Phi$-type over $B$, then they have the same $\phi$-type over $B$, where $B$ is some parameter set.
  Therefore the number of $\phi$-types is bounded by the number of $\Phi$-types.
  The bound follows from Lemma \ref{count_types}.
\end{proof}

For the remainder of the \chapa fix $\Phi(x, y)$ to be a collection of formulas as in Lemma \ref{quantifier_elimination}.
By the previous lemma, to show that $\vc^*(\phi) \leq |x|$, it suffices to bound $\vc^*(\Phi) \leq |x|$.
More precisely, it is sufficient to show that given a parameter set $B$ of size $N$,
the number of $\Phi$-types over $B$ is $O(N^{|x|})$.
Fix such a parameter set $B$ and work with it from now on.
We will compute a bound for the number of $\Phi$-types over $B$.

Consider the finite set $T = T(\Phi, B) = \curly{c_i(b) \mid b \in B, i \in I} \subseteq \Q_p$.
In this definition %$B$ is the parameter set that we have fixed 
$c_i(b)$ come from the collection of formulas $\Phi$ (see Lemma \ref{quantifier_elimination}).
%from the quantifier elimination above.
View $T$ as a tree as follows:
\begin{Definition} \ 
  \begin{itemize}
  \item For $c \in \Q_p, \alpha \in \Z$  define the (open) \defn{ball} 
    \begin{align*}
      B(c, \alpha) = \curly{c' \in \Q_p \mid \vval \paren{c' - c} > \alpha}
    \end{align*}
    of radius $\alpha$ and center $c$.
    We also let $B(c, -\infty) = \Q_p$ and $B(c, +\infty) = \emptyset$.
  \item Define the collection of balls $\BB = \curly{B(t_1, \vval(t_1 - t_2))}_{t_1, t_2 \in T}$.
    Note that $\BB$ is a (directed) boolean algebra of sets in $\Q_p$.
    We refer to the atoms in that algebra as \defn{intervals}.
    Note that the intervals partition $\Q_p$ so any element $a \in \Q_p$ belongs to a unique interval.
  \item Let's introduce some notation for the intervals.
    For $t \in T$ and $\alpha_L, \alpha_U \in \Z \cup \curly{-\infty, +\infty}$ define
    \begin{align*}
      \interval = B(t, \alpha_L) \backslash \bigcup \curly{B(t', \alpha_U) \mid t' \in T, \vval(t' - t) \geq \alpha_U}
    \end{align*}
    (this is sometimes referred to as the swiss cheese construction).
    One can check that every interval is of the form $\interval$ for some values of $t, \alpha_L, \alpha_U$.
    The quantities $\alpha_L, \alpha_U$ are uniquely determined by the interval $\interval$,
    while $t$ might not be.
  \item Intervals are a natural construction for trees, however we will require a more refined notion to make Lemma \ref{main_lemma} below work.
    Define a larger collection of balls 
    \begin{align*}
      \BB' = \BB \cup \curly{B(c_i(b), \vval(c_j(b) - c_k(b)))}_{i,j,k \in I, b \in B}.  
    \end{align*}
    Similarly to the previous definition, we define a \defn{subinterval} to be an atom of the boolean algebra generated by $\BB'$.
    Subintervals refine intervals.
    Moreover, as before, each subinterval can be written as $\interval$ for some values of $t, \alpha_L, \alpha_U$.
    As before, $\alpha_L, \alpha_U$ are uniquely determined by the subinterval $\interval$,
    while $t$ might not be.
  \end{itemize}
\end{Definition}

\tikzstyle{node}=[circle, draw]

\tikzstyle{up}=[draw,shape=circle,fill=blue,scale=0.5]
\tikzstyle{c1}=[node, fill = white]
\tikzstyle{md}=[node, fill = lightgray]
\tikzstyle{c2}=[node, fill = white]
\tikzstyle{dn}=[node, fill = white]
\tikzstyle{ds}=[node, fill = white]
\tikzstyle{nd}=[rectangle, draw]

\tikzstyle{l}=[level distance=50mm]
\tikzstyle{m}=[level distance=35mm]
\tikzstyle{s}=[level distance=20mm]

\tikzstyle{ww}=[sibling distance=50mm]
\tikzstyle{wt}=[sibling distance=20mm]
\tikzstyle{w}=[sibling distance=15mm]
\tikzstyle{a}=[sibling distance=10mm]
\tikzstyle{n}=[sibling distance=5mm]

\tikzstyle{ex}=[draw,shape=circle,fill=black,scale=0.1]
\tikzstyle{lb}=[]

\newcommand{\bloffset}{(-1,1)}
\newcommand{\broffset}{(1,1)}
\newcommand{\tloffset}{\bloffset}
\newcommand{\troffset}{\broffset}

\begin{figure}[h]
  \begin{tikzpicture}[scale=.6]%, , every node/.style={transform shape}]
    \def\boffset{0}
    \def\toffset{\boffset}
        \node[ex]{}
    [grow=north]
    child[black, s]{node(AAN)[ex]{}
      child[l,ww]{node(BBN)[ex]{}
        child[m,w]{node[ex]{}
          child[s,n]{node[ex]{}
          }
          child[s,n]{node[ex]{}
          }
        }
        child[s,w]{node[ex]{}
          child[m,n]{node[ex]{}
            child[s,a]{node[lb]{$\ldots$}
            }
            child[s,a]{node[lb]{$c_i(b)$}
            }
            child[s,a]{node[lb]{$\ldots$}
            }
          }
          child[s,n]{node[ex]{}
          }
        }
      }
      child[m,ww]{node[ex]{}
        child[s,n]{node[ex]{}
        }
        child[s,n]{node[ex]{}
          child[s,n]{node[ex]{}
          }
          child[m,n]{node[ex]{}
            child[s,wt]{node[lb]{$c_3(b_5)$}
            }
            child[s,wt]{node[lb]{$c_4(b_3)$}
            }
          }
        }
      }
    };

    \coordinate (bl) at (-1,2);
    \coordinate (br) at (2,1);
    \coordinate (tl) at (bl);
    \coordinate (tr) at (br);
    \def\offset{1.25};

    \coordinate (AA) at (${1-\boffset}*(AAN) + \boffset*(BBN)$);
    \coordinate (BB) at ($\toffset*(AAN) + {1-\toffset}*(BBN)$);

    \path[fill=gray!50] (AA) -- ++(bl) --  ($(BB) + (tl)$) -- (BB) -- ++(tr) -- ($(AA) + (br)$) -- (AA);

    % \draw ($(AA) + (bl)$) -- (AA) -- ++(br);
    % \draw ($(BB) + (tl)$) -- (BB) -- ++(tr);
    \draw (AA) -- (BB);

    \draw[dashed] ($(AA) + \offset*(bl)$) -- (AA) -- ($(AA) + \offset*(br)$);
    \draw[dashed] ($(BB) + \offset*(tl)$) -- (BB) -- ($(BB) + \offset*(tr)$);
  
  \end{tikzpicture}
  \begin{tikzpicture}[scale=.6]%, , every node/.style={transform shape}]
    \def\boffset{.33}
    \def\toffset{\boffset}
        \node[ex]{}
    [grow=north]
    child[black, s]{node(AAN)[ex]{}
      child[l,ww]{node(BBN)[ex]{}
        child[m,w]{node[ex]{}
          child[s,n]{node[ex]{}
          }
          child[s,n]{node[ex]{}
          }
        }
        child[s,w]{node[ex]{}
          child[m,n]{node[ex]{}
            child[s,a]{node[lb]{$\ldots$}
            }
            child[s,a]{node[lb]{$c_i(b)$}
            }
            child[s,a]{node[lb]{$\ldots$}
            }
          }
          child[s,n]{node[ex]{}
          }
        }
      }
      child[m,ww]{node[ex]{}
        child[s,n]{node[ex]{}
        }
        child[s,n]{node[ex]{}
          child[s,n]{node[ex]{}
          }
          child[m,n]{node[ex]{}
            child[s,wt]{node[lb]{$c_3(b_5)$}
            }
            child[s,wt]{node[lb]{$c_4(b_3)$}
            }
          }
        }
      }
    };

    \coordinate (bl) at (-1,2);
    \coordinate (br) at (2,1);
    \coordinate (tl) at (bl);
    \coordinate (tr) at (br);
    \def\offset{1.25};

    \coordinate (AA) at (${1-\boffset}*(AAN) + \boffset*(BBN)$);
    \coordinate (BB) at ($\toffset*(AAN) + {1-\toffset}*(BBN)$);

    \path[fill=gray!50] (AA) -- ++(bl) --  ($(BB) + (tl)$) -- (BB) -- ++(tr) -- ($(AA) + (br)$) -- (AA);

    % \draw ($(AA) + (bl)$) -- (AA) -- ++(br);
    % \draw ($(BB) + (tl)$) -- (BB) -- ++(tr);
    \draw (AA) -- (BB);

    \draw[dashed] ($(AA) + \offset*(bl)$) -- (AA) -- ($(AA) + \offset*(br)$);
    \draw[dashed] ($(BB) + \offset*(tl)$) -- (BB) -- ($(BB) + \offset*(tr)$);
  
  \end{tikzpicture}
  \caption{A typical interval (left) and subinterval (right) on a tree $T = \{c_i(b) \mid i \in I, b \in B\}$.}
\end{figure}


Subintervals are fine enough to make Lemma \ref{main_lemma} below work while coarse enough to be $O(N)$ few:
\begin{Lemma} \label{interval_count}\ 
  \begin{itemize}
  \item 
    There are at most $2|T| = 2 N |I| = O(N)$ different intervals.
  \item 
    There are at most $2|T| + |B| \cdot |I|^3 = O(N)$ different subintervals.
  \end{itemize}
\end{Lemma}

\begin{proof}
  Each new element in the tree $T$ adds at most two intervals to the total count,
  so by induction there can be at most $2|T|$ many intervals.
  Each new ball in $\BB' \backslash \BB$ adds at most one subinterval to the total count,
  so by induction there are at most $|\BB' \backslash \BB|$ more subintervals than there are intervals.
\end{proof}


\begin{Definition}
  Suppose $a \in \Q_p$ lies in the interval $\interval$. 
  Define the \defn{T-valuation} of $a$ to be $\tval(a) = \vval(a - t)$.    
\end{Definition}

This is a natural notion having the following properties:
\begin{Lemma}  \label{tval} \ 
  \begin{enumerate}[label=(\alph*)]
  \item $\tval(a)$ is well-defined, independent of choice of $t$ to represent the interval.
  \item If $a \in \Q_p$ lies in the subinterval $\interval$,
    then $\tval(a) = \vval(a - t)$.
  \item If $a \in \Q_p$ lies in the (sub)interval $\interval$ 
    then $\alpha_L < \tval(a) \leq \alpha_U$.
  \item For any $a \in \Q_p$ lying in the (sub)interval $\interval$ and $t' \in T$:
    \begin{itemize}
    \item If $\vval(t - t') \geq \alpha_U$, then $\vval(a - t') = \tval(a)$. 
    \item If $\vval(t - t') \leq \alpha_L$, then $\vval(a - t') = \vval(t - t') \paren{\leq \alpha_L < \tval(a)}$. 
    \end{itemize}
  \end{enumerate}
\end{Lemma}


\begin{proof}
  (a)-(c) are clear.
  For (d) fix $t' \in T$ and suppose $a \in \Q_p$ lies in the subinterval $\inti(t, \alpha_L', \alpha_U')$.
  This  subinterval lies inside of a unique interval $\interval$ for some choice of $\alpha_L, \alpha_U$ and
  by the definition of intervals (or more specifically $\BB$):
  \begin{align*}
    \vval(t - t') \geq \alpha_U &\iff \vval(t - t') \geq \alpha_U',\\
    \vval(t - t') \geq \alpha_L &\iff \vval(t - t') \geq \alpha_L'.
  \end{align*}
  Therefore without loss of generality we may assume that $a \in \Q_p$ lies in an interval $\interval$.
  By (c) and the definition of intervals one of the three following cases has to hold.
  
  Case 1: $\vval(t - t') \geq \alpha_U$ and $\tval(a) < \alpha_U$. Then
  \begin{align*}
    \vval(t - t') \geq \alpha_U > \tval(a) = \vval(a - t),
  \end{align*}
  thus $\vval(a - t') = \vval(a - t) = \tval(a)$ as needed.

  Case 2: $\vval(t - t') \geq \alpha_U$ and $\tval(a) = \alpha_U$. Then
  \begin{align*}
    \tval(a) = \vval(a - t) = \vval(t - t') \geq \alpha_U,
  \end{align*}
  thus $\vval(a - t') \geq \alpha_U$.
  The interval $\interval$ is disjoint from the ball $B(t', \alpha_U)$,
  so $a \notin B(t', \alpha_U)$, that is, $\val(a - t') \leq \alpha_U$.
  Combining this with the previous inequality we get that $\val(a - t') = \alpha_U = \tval(a)$ as needed.

  Case 3: $\vval(t - t') \leq \alpha_L$. Then
  \begin{align*}
    \vval(t - t') \leq \alpha_L < \tval(a) = \vval(a - t),
  \end{align*}
  thus $\vval(a - t') = \vval(t - t')$ as needed. 
\end{proof}




\begin{Definition}
  Suppose $a \in \Q_p$ lies in the subinterval $\interval$.
  We say that $a$ is \defn{far from the boundary} (tacitly: of $\interval$) if 
    \begin{align*}
	\alpha_L + n \leq \tval(a) \leq \alpha_U - n.
    \end{align*}
  Here $n$ is as in Lemma \ref{quantifier_elimination}.
  Otherwise we say that it is \defn{close to the boundary} (of $\interval$).
\end{Definition}

\begin{Definition}
  Suppose $a_1, a_2 \in \Q_p$ lie in the same subinterval $\interval$.
  We say $a_1, a_2$ have the same \defn{subinterval type} if one of the following holds:
  \begin{itemize}
  \item Both $a_1, a_2$ are far from the boundary and $a_1 - t, a_2 - t$ are in the same $Q_{m,n}$-coset.
    (Here $Q_{m,n}$ is as in Lemma \ref{quantifier_elimination}.)
  \item Both $a_1, a_2$ are close to the boundary and 
    \begin{align*}
	  \tval(a_1) = \tval(a_2) \leq \vval(a_1 - a_2) - n.
    \end{align*}
  \end{itemize}      
\end{Definition}


\begin{Definition}
	For $c \in \Q_p$ and $\alpha, \beta \in \Z, \alpha < \beta$ define $c \midr [\alpha, \beta)$
  to be the record of the coefficients of $c$ for the valuations between $[\alpha, \beta)$.
  More precisely write $c$ in its power series form
  \begin{align*}
    c = \sum_{\gamma \in \Z} c_\gamma p^\gamma \text{ with } c_\gamma \in \curly{0,1, \ldots, p-1}.
  \end{align*}
  Then $c \midr [\alpha, \beta)$ is just $(c_\alpha, c_{\alpha+1}, \ldots c_{\beta - 1}) \in \curly{0,1, \ldots, p-1}^{\beta - \alpha}$.
\end{Definition}

The following lemma is an adaptation of Lemma 7.4 in \cite{density}.
\begin{Lemma} \label{distance}
  Fix $m,n \in \N$.
  For any $x,y,c \in \Q_p$, if
  \begin{align*}
    \val (x - c) = \val (y - c) \leq \val (x - y) - n,
  \end{align*}
  then $x - c, y - c$ are in the same coset of $Q_{m,n}$.
\end{Lemma}
\begin{proof}
  Call $a,b \in \Q_p$ \defn{similar} if $\val a = \val b$ and
  \begin{align*}
    a \midr [\val a, \val a + n) = b \midr [\val b, \val b + n).
  \end{align*}
  If $a,b$ are similar then
  \begin{align*}
    a \in Q_{m,n} \iff b \in Q_{m,n}.
  \end{align*}
  Moreover for any $\lambda \in \Q_p^\times$, if $a,b$ are similar then so are $\lambda a, \lambda b$.
  Thus if $a,b$ are similar, then they belong to the same coset of $Q_{m,n}$.
  The hypothesis of the lemma force $x - c, y - c$ to be similar, thus belonging to the same coset.
\end{proof} 


\begin{Lemma} \label{interval_type_count}
  For each subinterval there are at most $K = K(Q_{m,n})$ many subinterval types 
  (with $K$ not depending on $B$ or on the subinterval).  
\end{Lemma}

\begin{proof}
  Let $a, a' \in \Q_p$ lie in the same subinterval $\interval$.

  Suppose $a, a'$ are far from the boundary.
  Then they have the same subinterval type if $a - t, a' - t$ are in the same $Q_{m,n}$-coset.
  So the number of such subinterval types is bounded by the number of $Q_{m,n}$-cosets.

  Suppose $a, a'$ are close to the boundary and
  \begin{align*}
    &\tval(a) - \alpha_L = \tval(a') - \alpha_L < n \text { and}\\
    &a \midr [\tval(a), \tval(a) + n) = a' \midr [\tval(a'), \tval(a') + n).
  \end{align*}
  Then $a, a'$ have the same subinterval type.
  Such a subinterval type is thus determined by $\tval(a) - \alpha_L$ and the tuple $a \midr [\tval(a), \tval(a) + n)$,
  therefore there are at most $n p^n$ many such types.

  A similar argument works for $a$ with $\alpha_U - \tval(a) \leq n$.

  Adding all this up we get that there are at most
  \begin{align*}
    K = \text{(number of $Q_{m,n}$ cosets)} + 2 n p^n  
  \end{align*}
  many subinterval types.
\end{proof}

The following critical lemma relates tree notions to $\Phi$-types.
\begin{Lemma} \label{main_lemma}
  Suppose $d, d' \in \Q_p^{|x|}$ satisfy the following three conditions:
  \begin{itemize}
  \item For all $i \in I$ $p_i(d)$ and $p_i(d')$ are in the same subinterval.
  \item For all $i \in I$ $p_i(d)$ and $p_i(d')$ have the same subinterval type.
  \item For all $i,j \in I$, $\tval(p_i(d)) > \tval(p_j(d))$ iff $\tval(p_i(d')) > \tval(p_j(d'))$.
  \end{itemize}
  Then $d, d'$ have the same $\Phi$-type over $B$.
\end{Lemma}
\begin{proof}
  There are two kinds of formulas in $\Phi$
  (see Lemma \ref{quantifier_elimination}).
  First we show that $d, d'$ agree on formulas of the form $p_i(x) - c_i(y) \in \lambda_k Q_{m,n}$.
  It is enough to show that for every $i \in I, b \in B$, $p_i(d) - c_i(b), p_i(d') - c_i(b)$ are in the same $Q_{m,n}$-coset.
  Fix such $i, b$.
  For brevity let $a = p_i(d), a' = p_i(d')$ and $Q = Q_{m,n}$.
  We want to show that $a - c_i(b), a' - c_i(b)$ are in the same $Q$-coset.
  
  Suppose $a, a'$ are close to the boundary.
  Then $\tval(a) = \tval(a') \leq \val(a - a') - n$.
  Using Lemma \ref{tval}d, we have
  \begin{align*}
    \val(a - c_i(b)) = \val(a' - c_i(b)) \leq \tval(a) \leq \val(a - a') - n.
  \end{align*}
  Lemma \ref{distance} shows that $a - c_i(b), a' - c_i(b)$ are in the same $Q$-coset.
  
  Now, suppose both $a, a'$ are far from the boundary.
  Let $\interval$ be the interval containing $a, a'$.
  Then we have 
  \begin{align*}
    \alpha_L + n \leq &\val (a - t) \leq \alpha_U - n, \\
    \alpha_L + n \leq &\val (a' - t) \leq \alpha_U - n
  \end{align*}
  (as being far from the subinterval's boundary also makes $a,a'$ far from interval's boundary).
  We have either $\val(t - c_i(b)) \geq \alpha_U$ or $\val(t - c_i(b)) \leq \alpha_L$ (as otherwise it would contradict the definition of intervals, or more specifically $\BB$).
  
  Suppose it is the first case $\val(t - c_i(b)) \geq \alpha_U$.
  Then using Lemma \ref{tval}d
  \begin{align*}
    \val(a - c_i(b)) = \val(a - t) \leq \alpha_U - n \leq \val(t - c_i(b)) - n.
  \end{align*}
  So by Lemma \ref{distance} elements $a - c_i(b), a - t$ are in the same $Q$-coset.
  By an analogous argument, $a' - c_i(b), a' - t$ are in the same $Q$-coset.
  As $a, a'$ have the same subinterval type, $a - t, a' - t$ are in the same $Q$-coset.
  Thus by transitivity we get that $a - c_i(b), a' - c_i(b)$ are in the same $Q$-coset.
  
  For the second case, suppose $\val(t - c_i(b)) \leq \alpha_L$.
  Then using Lemma \ref{tval}d
  \begin{align*}
    \val(a - c_i(b)) = \val(t - c_i(b)) \leq \alpha_L \leq \val(a - t) - n,
  \end{align*}
  so by Lemma \ref{distance} elements $a - c_i(b), t - c_i(b)$ are in the same $Q$-coset.
  Similarly $a' - c_i(b), t - c_i(b)$ are in the same $Q$-coset.
  Thus by transitivity we get that $a - c_i(b), a' - c_i(b)$ are in the same $Q$-coset.

  Next, we need to show that $d, d'$ agree on formulas of the form
  $\vval (p_i(x) - c_i(y)) < \vval (p_j(x) - c_j(y))$ 
  (again, referring to the presentation in Lemma \ref{quantifier_elimination}).
  Fix $i,j \in I, b \in B$.
  We would like to show the following equivalence: 
  \begin{multline} \label {eq:order_equation}
    \vval (p_i(d) - c_i(b)) < \vval (p_j(d) - c_j(b)) \iff \\
    \iff \vval (p_i(d') - c_i(b)) < \vval (p_j(d') - c_j(b))
  \end{multline}

  Suppose $p_i(d), p_i(d')$ are in the subinterval $\inti(t_i, \alpha_i, \beta_i)$ and 
  $p_j(d), p_j(d')$ are in the subinterval $\inti(t_j, \alpha_j, \beta_j)$.
  Lemma \ref{tval}d yields the following four cases.

  Case 1:
  \begin{align*}
    &\vval (p_i(d) - c_i(b)) = \vval (p_i(d') - c_i(b)) = \vval(t_i - c_i(b)) \\
    &\vval (p_j(d) - c_j(b)) = \vval (p_j(d') - c_j(b)) = \vval(t_j - c_j(b))
  \end{align*}
  Then it is clear that the equivalence \eqref{eq:order_equation} holds.

  Case 2:
  \begin{align*}
    &\vval (p_i(d) - c_i(b)) = \tval(p_i(d)) \text{ and } \vval (p_i(d') - c_i(b)) = \tval(p_i(d')) \\
    &\vval (p_j(d) - c_j(b)) = \tval(p_j(d)) \text{ and } \vval (p_j(d') - c_j(b)) = \tval(p_j(d'))
  \end{align*}
  Then the equivalence \eqref{eq:order_equation} holds by the third hypothesis of the lemma (that order of T-valuations is preserved).

 Case 3:
  \begin{align*}
    &\vval (p_i(d) - c_i(b)) = \vval (p_i(d') - c_i(b)) = \vval(t_i - c_i(b)) \\
    &\vval (p_j(d) - c_j(b)) = \tval(p_j(d)) \text{ and } \vval (p_j(d') - c_j(b)) = \tval(p_j(d'))
  \end{align*}
  If $p_j(d), p_j(d')$ are close to the boundary,
  then $\tval(p_j(d)) = \tval(p_j(d'))$ and the equivalence \eqref{eq:order_equation} clearly holds.
  Suppose then that $p_j(d), p_j(d')$ are far from the boundary.
  \begin{align*}
    \alpha_j + n \leq &\tval(p_j(d)), \tval(p_j(d')) \leq \beta_j - n \\
    \alpha_j < &\tval(p_j(d)), \tval(p_j(d')) < \beta_j
  \end{align*}
  and $\vval(t_i - c_i(b))$ lies outside of the $(\alpha_j, \beta_j)$
  by the definition of subinterval (more specifically definition of $\BB'$).
  Therefore \eqref{eq:order_equation} has to hold.
  (Note that we always have
  \begin{align*}
    \tval(p_j(d)), \tval(p_j(d')) \in (\alpha_j, \beta_j]
  \end{align*}
   by Lemma \ref{tval}c, so 
  we only need the condition on being far from the boundary to avoid the edge case of equality to $\beta_j$.)

  Case 4:
  \begin{align*}
    &\vval (p_i(d) - c_i(b)) = \tval(p_i(d)) \text{ and } \vval (p_i(d') - c_i(b)) = \tval(p_i(d')) \\
    &\vval (p_j(d) - c_j(b)) = \vval (p_j(d') - c_j(b)) = \vval(t_j - c_j(b)).
  \end{align*}
  Similar to case 3 (switching $i,j$).
\end{proof}



  The previous lemma gives us an upper bound on the number of types - there are at most $|2I|!$ many choices for the order of $\tval$,
  $O(N)$ many choices for the subinterval for each $p_i$,
  and $K$ many choices for the subinterval type for each $p_i$ (where $K$ is as in Lemma \ref{interval_type_count}),
  giving a total of $O(N^{|I|}) \cdot K^{|I|} \cdot |I|! = O(N^{|I|})$ many types.
  This implies $\vc^*(\Phi) \leq |I|$.
  The biggest contribution to this bound are the choices among the $O(N)$ many subintervals for each $p_i$ with $i \in I$.
  Are all of those choices realized?
  Intuitively there are $|x|$ many variables and $|I|$ many equations,
  so once we choose a subinterval for $|x|$ many $p_i$'s, the subintervals for the rest should be determined.
  This would give the required bound $\vc^*(\Phi) \leq |x|$.
  The next section outlines this idea formally.



%%%%%%%%%%%%%%%%%%%%%%%%%%%%%%%% 

\section{Main Proof}
Given a homogenenous linear polynomial $p(x)$ with coefficients in $\Q_p$ and $c \in \Q_p^{|x|}$,
an alternative way to write $p(c)$ is as a scalar product $\vec p \cdot \vec c$,
where $\vec p$ and $\vec c$ are vectors in $\Q_p^{|x|}$.% (as $p(x)$ is homogeneous linear).

\begin{Lemma}	 
  Suppose we have a finite collection of vectors $\curly{\vec p_j}_{j \in J}$ with each $\vec p_j \in \Q_p^{|x|}$.
  Suppose $\vec p \in \Q_p^{|x|}$ satisfies $\vec p \in \vecspan \curly{\vec p_j}_{j \in J}$,
  and we have $\vec c \in \Q_p^{|x|}, \alpha \in \Z$ with $\val(\vec p_j \cdot \vec c) > \alpha \text{ for all } j \in J$.
  Then $\val(\vec p \cdot \vec c) > \alpha - \gamma$ for some $\gamma \in \N$.
  Moreover $\gamma$ can be chosen independently from $\vec c, \alpha$ depending only on $\curly{\vec p_j}_{j \in J}$.
\end{Lemma}

\begin{proof}
  For some $c_j \in \Q_p$ for $j \in J$ we have $\vec p = \sum_{j \in J} c_j \vec p_j$,
  hence $\vec p \cdot \vec c = \sum_{j \in J} c_j \vec p_j \cdot \vec c$.
  Thus
  \begin{align*}
    \val \paren{c_j \vec p_j \cdot \vec c} = \val \paren{c_j} + \val \paren{\vec p_j \cdot \vec c} > \val \paren{c_j} + \alpha.
  \end{align*}
  % Pick $\gamma = -\max \val \paren{c_i}$ or $0$ if all those values are positive.
  Let $\gamma = \max(0, -\max_{j \in J} \val \paren{c_j})$.
  % Let $\gamma = -\min(0, \max_{j \in J} \val \paren{c_j})$.
  % Let $\gamma = \max(0, \min -\val \paren{c_j})$.
  Then we have 
  \begin{align*}
    &\val(\vec p \cdot \vec c) =
      \val \paren{\sum_{j \in J} c_j \vec p_j \cdot \vec c} \geq \\
      \geq &\min_{j \in J} \val\paren{\sum_{j \in J} c_j \vec p_j \cdot \vec c} >
      \min_{j \in J} \val(c_j) + \alpha \geq
      \alpha - \gamma
  \end{align*}
  as required.
\end{proof}

\begin{Corollary}	 \label{gamma}
  Suppose we have a finite collection of vectors $\curly{\vec p_i}_{i \in I}$ with each $\vec p_i \in \Q_p^{|x|}$.
  Suppose $J \subseteq I$ and $i \in I$ satisfy $\vec p_i \in \vecspan \curly{\vec p_j}_{j \in J}$,
  and we have $\vec c \in \Q_p^{|x|}, \alpha \in \Z$ with $\val(\vec p_j \cdot \vec c) > \alpha \text{ for all } j \in J$.
  Then $\val(\vec p_i \cdot \vec c) > \alpha - \gamma$
  for some $\gamma \in \N$.
  Moreover $\gamma$ can be chosen independently from $J, j, \vec c, \alpha$ depending only on $\curly{\vec p_i}_{i \in I}$.
\end{Corollary}
\begin{proof}
  The previous lemma shows that we can pick such $\gamma$ for a given choice of $i, J$, but independent from $\alpha, \vec c$.
  To get a choice independent from $i, J$, go over all such eligible choices 
  ($i$ ranges over $I$ and $J$ ranges over subsets of $I$),
  pick $\gamma$ for each, and then take the maximum of those values.  
\end{proof}

Recall that we have confined to work with collection $\Phi(x, y)$ of formulas from Lemma \ref{quantifier_elimination}.
Fix $\gamma$ according to Corollary \ref{gamma} corresponding to $\curly{\vec p_i}_{i \in I}$ given by $\Phi$.
(The lemma above is a general result, but we only use it applied to the vectors given by $\Phi$.)

\begin{Definition}
  Suppose $a \in \Q_p$ lies in the subinterval $\interval$.
  Define the \defn{$T$-floor} of $a$ to be $\tfl(a) = \alpha_L$.
\end{Definition}

\begin{Definition}
  Let $f: \Q_p^{|x|} \arr \Q_p^I$ with $f(c) = (p_i(c))_{i \in I}$.
  Define the segment space $\Sg$ to be the image of $f$.
  Equivalently:
  \begin{align*}
    \Sg = \curly{(p_i(c))_{i \in I} \mid c \in \Q_p^{|x|}} \subseteq \Q_p^I.
  \end{align*}
\end{Definition}

Without loss of generality, we may assume that $I = \curly{1,2, \ldots, k}$ (that is the formulas are labeled by consecutive natural numbers).
Given a tuple $(a_i)_{i\in I}$ in the segment space,
look at the corresponding $T$-floors $\curly{\tfl(a_i)}_{i\in I}$ and $T$-valuations $\curly{\tval(a_i)}_{i\in I}$.
Partition the segment space by the order types of $\{\tfl(a_i)\}_{i\in I}$ and $\curly{\tval(a_i)}_{i\in I}$ (as subsets of $\Z$).

Work in a fixed set $\Sg'$ of the partition.
After relabeling the $p_i$ we may assume that
\begin{align*}
  \tfl(a_1) \geq \tfl(a_2) \geq \ldots \text { for all $a_i \in \Sg'$}.
\end{align*}

Consider the (relabeled) sequence of vectors $\vec p_1, \vec p_2, \ldots, \vec p_I$.
There is a unique subset $J \subseteq I$ such that the set of all vectors with indices in $J$ is linearly independent,
and all vectors with indices outside of $J$ are a linear combination of preceding vectors.
(We can pick those using a greedy algorithm for finding a linearly independent subset of vectors.)
We call indices in $I$ \defn{independent} and we call the indices in $I \setminus J$ \defn{dependent}.


\begin{Definition} \ 
  \begin{itemize}
  \item Denote $\curly{0,1, \ldots, p-1}$ as \defn{$\Ct$}.
    %Note that $|\Ct| = p^\gamma$.
  \item Let \defn{$\It$} be the space of all subinterval types.
    By Lemma \ref{interval_type_count} we have $|\It| \leq K$.
  \item Let \defn{$\Sub$} be the space of all subintervals.
    By Lemma \ref{interval_count} we have $|\Sub| \leq 3 |I|^2 \cdot N = O(N)$.
  \end{itemize}
\end{Definition}

\begin{Definition}
  Now, we define a function
  \begin{align*}
    g_{\Sg'}: \Sg' \arr \It^I \times \Sub^J \times \Ct^{I \backslash J}
  \end{align*}
  as follows:
  
  Let $a = (a_i)_{i\in I} \in \Sg'$.
  To define $g_{\Sg'}(a)$ we need to specify where it maps $a$ in each individual component of the product.

  For each $a_i$ record its subinterval type, giving the first component in $\It^I$.

  For $a_j$ with $j \in J$, record the subinterval of $a_j$, giving the second component in $\Sub^J$.

  For the third component (an element of $\Ct^{I \backslash J}$) do the following computation.
  Pick $a_i$ with $i$ dependent.
  Let $j$ be the largest independent index with $j < i$.
  Record $a_i \midr [\tfl(a_j) - \gamma, \tfl(a_j))$.

  Combine $g_{\Sg'}$ for all sets $\Sg'$ in our partition of $\Sg$ to get a function 
  \begin{align*}
    g: \Sg \arr \It^I \times \Sub^J \times \Ct^{I \backslash J}.  
  \end{align*}
\end{Definition}

\begin{Lemma}
  Suppose we have $c, c' \in \Q_p^{|x|}$ such that $f(c), f(c')$ are in the same
  set $\Sg'$ of the partition of $\Sg$ and $g(f(c)) = g(f(c'))$.
  Then $c, c'$ have the same $\Phi$-type over $B$.
\end{Lemma}

\newcommand{\pvec}[1]{\vec{#1}\mkern2mu\vphantom{#1}}

\begin{proof}
  Let $a_i = \vec p_i \cdot \vec c$ and $a_i' = \vec p_i \cdot \pvec c'$ so that
  \begin{align*}
    f(c) &= (p_i(c))_{i \in I} = (\vec p_i \cdot \vec c)_{i \in I} = (a_i)_{i \in I}, \\
    f(c') &= (p_i(c'))_{i \in I} = (\vec p_i \cdot \pvec c')_{i \in I} = (a_i')_{i \in I}.
  \end{align*}
  For each $i$ we show that $a_i, a_i'$ are in the same subinterval and have the same subinterval type,
  so the conclusion follows by Lemma \ref{main_lemma}
  (the tuples $f(c), f(c')$ are in the same partition ensuring the proper order of $T$-valuations for the 3rd condition of the lemma).
  $\It$ records the subinterval type of each element, so if $g(\bar a) = g(\bar a')$ then $a_i, a_i'$ have the same subinterval type for all $i \in I$.
  Thus it remains to show that $a_i, a_i'$ lie in the same subinterval for all $i \in I$.
  Suppose $i$ is an independent index.
  Then by construction, $\Sub$ records the subinterval for $a_i, a_i'$, so those have to belong to the same subinterval.
  Now suppose $i$ is dependent.
  Pick the largest $j < i$ such that $j$ is independent.
  We have $\tfl(a_i) \leq \tfl(a_j)$ and $\tfl(a_i') \leq \tfl(a_j')$.
  Moreover $\tfl(a_j) = \tfl(a_j')$ as $a_j, a_j'$ lie in the same subinterval (using the earlier part of the argument as $j$ is independent).
  
  \begin{Claim}
    We have $\val(a_i - a_i') > \tfl(a_j) - \gamma$.
  \end{Claim}
  \begin{proof}
    Let $K$ be the set of the independent indices less than $i$.
    Note that by the definition for dependent indices we have $\vec p_i \in \vecspan \curly{\vec p_k}_{k \in K}$.
    We also have 
    \begin{align*}
      \val(a_k - a_k') > \tfl(a_k) \text { for all } k \in K
    \end{align*}
    as $a_k, a_k'$ lie in the same subinterval (using the earlier part of the argument as $k$ is independent).
    Now $\val(a_k - a_k') > \tfl(a_j)$  for all $k \in K$ by monotonicity of $\tfl(a_k)$.
    Moreover $a_k - a_k' = \vec p_k \cdot \vec c - \vec p_k \cdot \pvec c' = \vec p_k \cdot (\vec c - \pvec {c}')$.
    Combining the two, we get that $\val(\vec p_k \cdot (\vec c - \pvec {c}')) > \tfl(a_j)$ for all $k \in K$.
    Now observe that $K \subseteq I, i \in I, \vec c - \pvec {c}' \in \Q_p^{|x|}, \tfl(a_j) \in \Z$
    satisfy the requirements of Lemma \ref {gamma}, so we apply it to obtain
    $\val(\vec p_i \cdot (\vec c - \pvec {c}')) > \tfl(a_j) - \gamma$.
    Similar to before, we have $\vec p_i \cdot (\vec c - \pvec {c}') = \vec p_i \cdot \vec c - \vec p_i \cdot \pvec {c}' = a_i - a_i'$.
    Therefore we can conclude that $\val(a_i - a_i') > \tfl(a_j) - \gamma$
    as needed, finishing the proof of the claim.
  \end{proof}	
  Additionally $a_i, a_i'$ have the same image in the $\Ct$ component, so we have $\val(a_i - a_i') > \tfl(a_j)$.
  We now would like to show that $a_i, a_i'$ lie in the same subinterval.
  As $\tfl(a_i) \leq \tfl(a_j)$, $\tfl(a_i') \leq \tfl(a_j')$ and $\tfl(a_j) = \tfl(a_j')$ we have that
  $\val(a_i - a_i') > \tfl(a_i)$ and $\val(a_i - a_i') > \tfl(a_i')$.
  Suppose that $a_i$ lies in the subinterval $\inti(t, \tfl(a_i), \alpha_U)$
  and that $a_i'$ lies in the subinterval $\inti(t', \tfl(a_i'), \alpha_U')$.
  Without loss of generality assume that $\tfl(a_i) \leq \tfl(a_i')$.
  As $\val(a_i - a_i') > \tfl(a_i')$, this implies that $a_i \in B(a_i', \tfl(a_i'))$.
  Then $a_i \in B(t', \tfl(a_i'))$ as $\vval(a_i - t') > \tfl(a_i')$.
  This implies that $B(t, \tfl(a_i)) \cap B(t', \tfl(a_i')) \neq \emptyset$ as they both contain $a_i$.
  As balls are directed, the non-zero intersection means that one ball has to be contained in another.
  Given our assumption that $\tfl(a_i) \leq \tfl(a_i')$, we have $B(t, \tfl(a_i)) \subseteq B(t', \tfl(a_i'))$.
  For the subintervals to be disjoint we need 
  $\inti(t, \tfl(a_i), \alpha_U) \cap B(t', \tfl(a_i')) = \emptyset$.
  But $\val(t' - a_i) > \tfl(a_i')$ implying that $a_i \in \inti(t, \tfl(a_i), \alpha_U) \cap B(t', \tfl(a_i'))$ giving a contradiction.
  Therefore the subintervals coincide finishing the proof.
\end{proof}

\begin{Corollary}
  The dual VC-density of $\Phi(x,y)$ is $\leq |x|$.
\end{Corollary}

\begin{proof}
  Suppose we have $c, c' \in \Q_p^{|x|}$ such that $f(c), f(c')$ are in the same partition and $g(f(c)) = g(f(c'))$.
  Then by the previous lemma $c, c'$ have the same $\Phi$-type.
  Thus the number of possible $\Phi$-types is bounded by the size of the range of $g$ times the number of possible partitions
  \begin{align*}
    \text{(number of partitions)} \cdot |\It|^{|I|} \cdot |\Sub|^{|J|} \cdot |\Ct|^{|I-J|}.
  \end{align*}
  There are at most $\paren{|2I|!}^2$ many partitions of $\Sg$,
  so in the product above, the only component dependent on $B$ is
  \begin{align*}
    |\Sub|^{|J|} \leq (N \cdot 3{|I|}^2)^{|J|} = O(N^{|J|}).
  \end{align*}	
  Every $p_i$ is an element of a $|x|$-dimensional vector space, so there can be at most $|x|$ many independent vectors.
  Thus we have $|J| \leq |x|$ and the bound follows.
\end{proof}

\begin{Corollary} [Theorem \ref{main_theorem}]
  The $\LL_{aff}$-structure $\Q_p$ satisfies $\vc(n) = n$ for each $n$.
\end{Corollary}

\begin{proof}
  The previous lemma implies that $\vc^*(\phi) \leq \vc^*(\Phi) \leq |x|$.
  As our choice of $\phi$ was arbitrary, this implies that the VC-density of any formula is bounded by $|x|$.
\end{proof}

\section{Conclusion}
This proof relies heavily on the linearity of the defining polynomials $a_1, a_2, c$ in the cell decomposition result (see Definition \ref{cell}).
Linearity is used to separate the $x$ and $y$ variables as well as
for Corollary \ref{gamma} to reduce the number of independent factors from $|I|$ to $|x|$.
The paper \cite{reduct} has cell decomposition results for more expressive reducts of $\Q_p$,
including, for example, restricted multiplication.
While our results don't apply to them directly,
it is this author's hope that similar techniques can be used to also compute the VC-density function for those structures.
\begin{openq}
  Compute the VC-density function for $\Q_p$-reducts studied in \cite{reduct}.
\end{openq}

Another interesting question is whether the reduct studied in this \chapa has the VC 1 property (see Definition 5.2 in \cite{density}).
If so, this would imply the linear VC-density bound directly.
The techniques used in the paper \cite{density} make it seem likely that the reduct has VC 2 property (just as the $\LLM$-structure $\Q_p$).
While there are techniques for showing that a structure has a given VC $n$ property,
less is known about showing that a structure doesn't have a given VC $n$ property.
Perhaps the simple structure of the $\LLA$-reduct can help understand this phenomenon better.
\begin{openq}
  For which $n$ does the $\LLA$-structure $\Q_p$ have the VC $n$ property?
\end{openq}
 
\chapter{Dp-minimality in Flat Graphs}

In this chapter we show that the theory of superflat graphs is dp-minimal.

\section{Preliminaries}

Superflat graphs were introduced in \cite{stable_graphs} as a natural class of stable graphs. Here we present a direct proof showing dp-minimality.

First, we introduce some basic graph-theoretic definitions.
\begin{Definition}
  Work in a possibly infinite graph $\GG$. Let $A, B, S, V \subseteq G$ where $G$ is the set of vertices of $G$.
  \begin{enumerate}
  \item A \defn{path} is a subgraph of $\GG$ with distinct vertices $v_0, v_1, \ldots, v_n$ and an edge between $v_{i-1}, v_i$ for all $i = 1, \ldots n$.
    It is called a path from $A$ to $B$ if $v_0 \in A$ and $v_n \in B$.
    A \defn{length} of such a path is $n$.
  \item Two paths are \defn{disjoint} if, excluding endpoints, they have no vertices in common.
  \item For $a,b \in G$ define the \defn{distance} $d(a,b)$ to be the length of the shortest path from $a$ to $b$ in $G$.
    If no such path exists then the distance is infinite.
  \item For $a,b \in G - A$ define $d_A(a,b)$ to be the distance between $a$ and $b$ in the subgraph of $\GG$
    induced on the vertices $G - A$. Equivalently it is the shortest path between $a$ and $b$ that avoids vertices in $A$.
  \item We say that $S$ \defn{separates} $A$ from $B$ if there exists $a \in A - S$, $b \in B - S$, with $d_S(a,b) = \infty$.
  \item We say that $A$ \defn{separates} $V$ if it separates $V$ from itself.
  \item We say that $V$ has \defn{connectivity} $n$ if there is a set of size $n$ that separates $V$,
    but there are no sets of size $n-1$ that separate $V$.
  \item Suppose $V$ has connectivity $n$. \defn{A connectivity hull} of $V$ is defined to be the union of all sets of size $n$ separating $V$.
  \end{enumerate}
\end{Definition}

In \cite{infinite_megner} we find a generalization of Megner's Theorem for infinite graphs:

\begin{Theorem} \label{megner}
  Let $A$ and $B$ be two sets of vertices in a possibly infinite graph. Then there exists a set $P$ of disjoint paths from $A$ to $B$, and a set $S$ of vertices separating $A$ from $B$, such that $S$ consists of a choice of precisely one vertex from each path in $P$.
\end{Theorem}

We use the following easy consequences:

\begin{Corollary} \label{cr_disjoint_paths}
  Let $V$ be a subset of vertices of a graph $\GG$ with connectivity $n$. Then there exists a set of $n$ disjoint paths from $V$ into itself.
\end{Corollary}

\begin{Corollary} \label{cr_hull_finite}
  With assumptions as above, the connectivity hull of $V$ is finite.
\end{Corollary}

\begin{proof}
  All the separating sets have to have exactly one vertex in each of those paths. 
\end{proof}

\begin{Definition} \ 
  \begin{itemize}
    \item A graph $K^m_n$ denotes a graph obtained from a complete graph on $n$ vertices by adding $m$ vertices to every edge.
    \item A graph is called \defn{superflat} if for every $m \in \N$ there is $n \in \N$ such that the graph avoids $K^m_n$ as a subgraph. 
  \end{itemize}
\end{Definition}

Theorem 2 in \cite{stable_graphs} gives a useful characterization of the superflat graphs.

\begin{Theorem} \label{th_superflat_equivalence}
  The following are equivalent:
  \begin{enumerate}
  \item $\GG$ is superflat.
  \item For every $n \in \N$ and an infinite set $A \subseteq G$, there exists a finite $B \subseteq G$ and infinite $A' \subseteq A$ such that for all $x,y \in A'$ we have $d_{B}(x, y) > n$.
  \end{enumerate}
\end{Theorem}

Roughly, in superflat graphs every infinite set contains a sparse infinite subset (possibly after throwing away finitely many nodes).

\section{Indiscernible sequences}

Fix an uncoutable cardinal $\kappa$.
Work in a superflat graph $\SS$ that is $\kappa$-saturated and strongly $\kappa$-homogeneous.
Fix a parameter set $A \subset S$ with $|A| < \kappa$.
Let $I = (a_i)_{i \in \I}$ be a countable $A$-indiscernible sequence.% of singletons (that is $|a_i| = 1$).
Stability implies that $I$ is totally indiscernible (see Lemma \ref{totally}).

\begin{Definition} \ 
  \begin{itemize}
  \item For a subsequence $\J \subset \I$ let $a(\J)$ denote the tuple obtained by concatenating $(a_j)_{j \in \J}$.
  \item Let $m$ be the arity of elements of $I$, that is $a_i \in S^m$.
    We call a set $H \subseteq S$ \defn{uniformly definable} from $I$ if there
    is a formula $\phi(x, y_1, \ldots, y_k)$ with $|y_i| = m$
    such that for every $\J \subseteq \I$ of size $k$ we have $H = \phi(G, a(\J))$.
  \end{itemize}
\end{Definition}

First suppose that $I$ consists of singletons, that is $a_i \in S$.

\begin{Definition}
  Let $V \subseteq S$. Define $P_n(V)$, a subgraph of $\SS$, to be a union of all paths of length $\leq n$ between the vertices of $V$.
\end{Definition}

\begin{Lemma} \label{lm_bump}
  Let $n \in \N$.
  There exists a finite set $B \subset S$ such that
  \begin{align*}
    \forall i \neq j \ d_B(a_i, a_j) > n.
  \end{align*}
\end{Lemma}

\begin{proof}
  By a \ref{th_superflat_equivalence} we can find an infinite $\J \subseteq \I$ and a finite set $B'$
  such that each pair from $J = (a_j)_{j \in \J}$ has distance $>n$ over $B'$.
  By total indiscernibility there exists an automorphism mapping $J$ to $I$ and fixing $A$.
  Image of $B'$ under this automorphism is the required set $B$.
\end{proof}

In other words, $B$ separates $I$ when viewed inside the subgraph $P_n(I)$.
This shows that $I$ has finite connectivity in $P_n(I)$.
Applying Corollary \ref{cr_hull_finite} we obtain that the connectivity hull of $I$ in $P_n(I)$ is finite.


\begin{Definition}
  Given a graph $\GG$ and $V \subseteq G$ define $H(\GG, V) \subseteq G$ to be the connectivity hull of $V$ in $\GG$.
  Note that if $V$ is finite, we have that $H(P_n(V), V)$ is $V$-definable.
\end{Definition}

\begin{Lemma} \label{lm_uniform}
  Let $H$ be the connectivity hull of $I$ inside of graph $P_n(I)$, that is $H = H(P_n(I), I)$.
  Then $H$ is uniformly definable from $I$ in $\SS$.
\end{Lemma}

\begin{proof}
  Using total indiscernability we may assume without the loss of generality that $I$ is indexed by $\N$.
  Let $I_i = \{a_0, a_1, \ldots, a_{i-1}\}$ a finite segment of the sequence.
  Let $N$ be the connectivity of $I$ inside of $P_n(I)$.
  % Define $H_i = H(P_n(I_i), I_i)$.
  % It is $I_i$-definable.

  First note that any finite set $H \subseteq P_n(I)$ will be contained in $P_n(I_i)$ for large enough $i$.
  Every element of $H$ is inside a path of length $\leq n$ and endpoints of that path are eventually going to be inside $I_i$.
  (Here the assumption that $I$ is enumerated by $\N$ is important.)

  Vertices $a_0, a_1$ cannot be separated by less than $N$ elements inside of $P_n(I)$
  (as this would contradict connectivity being $N$).
  Thus by Theorem \ref{megner} there are $N$ disjoint paths inside of $P_n(I)$ connecting $a_0$ to $a_1$.
  For large enough $i$, say $i \geq M_1$, all these paths are contained inside of $P_n(I_i)$.
  Those paths witness that vertices $a_0, a_1$ cannot be separated by less than $N$ elements inside of $P_n(I_i)$.
  As the set $P_n(I_i)$ is $I_i$-definable and $I$ is indiscernible,
  we have that no two vertices can be separated by less than $N$ elements inside of $P_n(I_i)$
  Thus $I_i$ has connectivity $\geq N$ inside of $P_n(I_i)$ for $i \geq M_1$.

  Consider a set $S$ of size $N$ that separates $I$ inside of $P_n(I)$.
  This is witnessed by two elements of $I$ that are separated.
  There are finitely many such sets $S$ as connectivity hull is finite.
  Thus for large enough $i$, say $i \geq M_2$, for each such $S$
  the segment $I_i$ contains a pair of vertices witnessing that $S$ is a separating set.

  Corollary \ref{cr_disjoint_paths} tells us that there are finitely many paths between elements of $V$ such that
  $H(P_n(I), I)$ is inside the union of those paths.
  For large enough $i$, say $i \geq M_3$, $P_n(I_i)$ will contain all of those paths, and thus $H(P_n(I), I) \subseteq P_n(I_i)$.

  Combine those three observations.
  Let $M = \max(M_1, M_2, M_3)$.
  Then for $i \geq M$ the set $P_n(I_i)$ contains all the $N$-element sets separating $I$ in $P_n(I)$,
  those sets separate $I_i$ in $P_n(I_i)$,
  and the connectity of $I_i$ in $P_n(I_i)$ is at most $N$.
  But this means that the connectity of $I_i$ in $P_n(I_i)$ has to be exactly $N$, and $H(P_n(I), I) \subseteq H(P_n(I_i), I_i)$.

  For $i \geq M$ define
  \begin{align*}
    E_i = \bigcap_{j = M}^{i} H(P_n(I_j), I_j).
  \end{align*}
  We have $H(P_n(I), I) \subseteq E_i$ and $E_i$ is a decreasing chain.
  Suppose $H(P_n(I), I) \subsetneq H(P_n(I_M), I_M)$, that is ${H(P_n(I_M), I_M) - H(P_n(I), I)} \neq \emptyset$.
  Then there exists a set $S$ of size $N$ that separates $I_M$ in $P_n(I_M)$ but does not separate $I$ in $P_n(I)$.
  Thus there has to be a finite subgraph of $P_n(I)$ disjoint from $S$ that connects all the elements of $I_M$ (witness of failure of separation).
  For large enough $i$, say $i \geq M_S$, this subgraph is contained in $P_n(I_i)$.
  There are finitely many possibilities for $S$ (as connectivity hull of $I_M$ in $P_n(I_M)$ is finite).
  Let $M_4 = \max_S(M_S)$.
  Then for $i \geq \max(M_4, M)$ we have
  \begin{align*}
    H(P_n(I_i), I_i) \cap \paren{H(P_n(I_M), I_M) - H(P_n(I), I)} = \emptyset,    
  \end{align*}
  and thus $E_i = H(P_n(I), I)$.
  As $E_i$ is $I_i$-definable, this shows that $H(P_n(I), I)$ is $I_i$-definable.
  Now we need to show uniform definability.
  Suppose $I'$ is a subsequence of $I$ of length $i$.
  There is an automorphism mapping $I_i$ to $I'$ that fixes $I$ setwise.
  But then it has to fix $H(P_n(I), I)$ setwise.
  Then it maps $I_i$-definition of $H(P_n(I), I)$ to a $I'$-definition of $H(P_n(I), I)$.
  As $I'$ was arbitrary this shows uniformity.
\end{proof}

\begin{Corollary} \label{inf_dis}
  Let $H_n = H(P_n(I), I)$.
  Then 
  \begin{align*}
    \forall i \neq j \ d_{H_n}(a_i, a_j) > n.
  \end{align*}
\end{Corollary}

\begin{proof}
  The set $H_n$ separates $I$ inside of $P_n(I)$.
  In particular there exist $i \neq j$ such that $d_{H_n}(a_i, a_j) = \infty$ inside $P_n(I)$.
  This means that $d_{H_n}(a_i, a_j) > n$ inside of $\SS$.
  But then by total indiscernibility and using the fact that $H_n$ is uniformly $I$-definable, this holds for all $i \neq j$.
\end{proof}

\newcommand{\tpl}[2]{{#1}^{(#2)}}

We would like to start working with tuples now instead of singletons.
We need some notation to extract individual elements of a tuple:
\begin{Definition}
  Suppose $a = (a_1, \ldots, a_m)$ is a tuple of arity $m$.
  Let $\tpl{a}{j}$ denote the $i$'th component, that is $\tpl{a}{j} = a_j$.
\end{Definition}

More generally, now suppose that $I$ consists of tuples of arity $m$, that is $a_i \in S^m$.

\begin{Definition} \ 
  \begin{itemize}
  \item We would like extract $j$'th components out of elements of $I$.
    Let $\tpl{I}{j} = (\tpl{a_i}{j})_{i \in \I}$, an $A$-indiscernible sequence of singletons.
  \item Let $\tpl{H_n}{j} = H(P_n(\tpl{I}{j}), \tpl{I}{j})$.
  \item Let
    \begin{align*}
      B_n = \bigcup_{i = 1}^{n} \bigcup_{j = 1}^{m} \tpl{H_n}{j}.
    \end{align*}
    Note that $B_n$ is finite as each $\tpl{H_n}{j}$ is finite by Corollary \ref{cr_hull_finite}.
  \end{itemize}
\end{Definition}

\begin{Lemma} \label{cr_bump}
  The sequence $I$ is indiscernible over the $A \cup B_n$.
\end{Lemma}

\begin{proof}
  By Lemma \ref{lm_uniform} the set $\tpl{H_n}{j}$ is uniformly $\tpl{I}{j}$-definable.
  Thus it is uniformly $I$-definable.
  Then $B_n$ is a finite union of uniformly $I$-definable sets, thus also uniformly $I$-definable.

  By uniform definability there is a formula $\phi(z, w_1, \ldots , w_k)$ with $|z| = 1$ and $|w_i| = m$ such that
  for any subsequence $\J \subset \I$ of length $k$ we have $\phi(G, a(\J)) = B_n$.
  Fix such a subsequence $\J$.

  Let $\psi(x_1, \ldots, x_l ,y)$ be an arbitrary $A$-formula with $|x_i| = m$. 
  Consider a collection of traces (i.e a collection of subsets of $B_n^{|y|}$)
  \begin{align*}
    \curly{\psi(a(\J'), B_n^{|y|}) \mid \text{ $\J'$ a subsequence of $\I$ of length $l$ disjoint from $\J$}}.
  \end{align*}
  If two of the traces are distinct, then by indiscernibility all of them are (using the fact that $B_n$ is uniformly definable).
  But that is impossible as $B_n$ is finite and thus has finitely many subsets.
  Thus all such traces are identical.
  As the choice of $\J$ was arbitrary, we can drop the condition that $\J'$ is disjoint from $\J$.
  This shows that for any $\J_1, \J_2 \subset \I$ of length $l$ and $h \in B_n^{|y|}$ we have
  \begin{align*}
    \SS \models \psi(a(\J_1), h) \iff \SS \models \psi(a(\J_2), h)    .
  \end{align*}
  As the choice of $\psi$ was arbitrary, this shows that $I$ is indiscernible over $A \cup B_n$ as needed.
\end{proof}

\begin{Definition}
  For tuples $a,b$ of the same arity $m$ and $B \subset S$ define
  \begin{align*}
    d_B(a,b) = \min_{1 \leq i,j \leq m} d_B(\tpl{a}{i}, \tpl{b}{j}).
  \end{align*}
\end{Definition}

\begin{Lemma} \label{inf_dis_gen}
  \begin{align*}
    \forall i \neq j \ d_{B_n}(a_i, a_j) > n/2.
  \end{align*}
\end{Lemma}

\begin{proof}
  Towards a contradiction suppose we have some $i \neq j$ and $k, l$ such that
  \begin{align*}
    d_{B_n}(\tpl{a_i}{k}, \tpl{a_j}{l}) \leq n/2.
  \end{align*}
  As $B_n$ is uniformly $I$-definable, by total indiscernability we have that this inequality holds for all $i \neq j$.
  Assuming for convenience that $I$ is enumerated by naturals, let $b_1 = \tpl{a_1}{k}$, $b_2 = \tpl{a_2}{l}$, $b_3 = \tpl{a_3}{k}$
  (note the superscripts).
  Then we have
  \begin{align*}
    &d_{B_n}(b_1, b_2) \leq n/2, \\
    &d_{B_n}(b_3, b_2) \leq n/2.
  \end{align*}
  By the triangle inequality
  \begin{align*}
    &d_{B_n}(b_1, b_3) \leq n, \\
    &d_{B_n}(\tpl{a_1}{k}, \tpl{a_3}{k}) \leq n.
  \end{align*}
  But this is a contradiction, as Lemma \ref{inf_dis} gives us
  \begin{align*}
    \forall i \neq j \ d_{\tpl{H_n}{k}}(\tpl{a_i}{k}, \tpl{a_j}{k}) > n
  \end{align*}
  and we have $\tpl{H_n}{k} \subseteq B_n$.
\end{proof}

\begin{Corollary} \label{inf_dis}
  There is a countable $B$ such that $I$ is indiscernible over $A \cup B$ and
  \begin{align*}
    \forall i \neq j \ d_B(a_i, a_j) = \infty.
  \end{align*}
\end{Corollary}

\begin{proof}
  Let $B_n$ as above. By Lemma \ref{inf_dis_gen} we have 
  \begin{align*}
    \forall i \neq j \ d_{B_n}(a_i, a_j) > n,
  \end{align*}
  and $I$ is indiscernible over $A \cup B_n$ by Lemma \ref{cr_bump}.
  Let $B = \bigcup_{n \in \N} B_n$.
  Then
  \begin{align*}
    \forall i \neq j \ d_{B}(a_i, a_j) = \infty.
  \end{align*}
  As $B_n \subseteq B_{n+1}$, the sequence $I$ is indiscernible over $A \cup B$ as needed.
\end{proof}

Thus $I$ can be upgraded to have infinite distance over its parameter set.

\section{Superflat graphs are dp-minimal}

\begin{Definition}
  For $B \subseteq S$ define an equivalence relation $\sim_B$ on $S - B$.
  Two vertices $b,c$ are $\sim_B$-equivalent if $d_B(b,c)$ is finite.
\end{Definition}

\begin{Lemma}
  Fix tuples $a,b,c$ in $S$, with $a,b$ having the same arity.
  Also let $B \subseteq S$.
  Suppose $\tp(a/B) = \tp(b/B)$ and $d_B(a, c) = d_B(b, c) = \infty$.
  Then $\tp(a/Bc) = \tp(b/Bc)$.
\end{Lemma}

\begin{proof}
  Suppose $a = (a_1, a_2, \ldots, a_m)$ and $b = (b_1, b_2, \ldots, b_m)$.
  Define $X_j$ to be the $\sim_B$-equivalence class of $a_j$ or $X_j = \emptyset$ if $a_j \in B$.
  Similarly define $Y_j$ for $b_j$.
  There is an automorphism $f$ of $\SS$ fixing $B$ with $f(a) = b$.
  It's easy to see that $f(X_j) = Y_j$ setwise.
  We would like to define a function $g \colon S \arr S$.
  For each $j$ let $g = f$ on $X_j$.
  Additionally if $X_j \neq Y_j$ then also let $g = f^{-1}$ on $Y_j$.
  Define $g$ to be identity on the rest of $S$.
  It is easy to check that $g$ is a well-defined automorphism fixing $Bc$ that maps $a$ to $b$.
  This shows that $\tp(a/Bc) = \tp(b/Bc)$.
\end{proof}

\begin{Lemma} \label{exclude}
  % Let $G$ be a flat graph with $(a_i)_{i\in\Q}$ indiscernible over $A$ and $b \in G$.
  Let $b \in G$.
  There exists $c \in \I$ such that all $(a_i)_{i\in\{\I - c\}}$ have the same type over $Ab$.
\end{Lemma}

\begin{proof}
  Use Corollary \ref{inf_dis} to find $B \supseteq A$ such that $I$ is indiscernible over $B$ and has infinite distance over $B$.
  All the tuples of the indiscernible sequence fall into distinct $\sim_B$-equivalence classes.
  If $b \in B$ we are done.
  Otherwise, there can be at most one element of the sequence that is in the same $\sim_B$-equivalence class as $b$.
  Exclude that element from the sequence.
  Remaining sequence elements are all infinitely far away from $b$ over $B$.
  By the previous lemma we have that elements of indiscernible sequence all have the same type over $Bb$ as needed.
\end{proof}


\begin{Theorem} \label{flat_dp_thm}
  Flat graphs are dp-minimal.
\end{Theorem}
\begin{proof}
  It suffices to show that $\SS$ is dp-minimal.
  Using Lemma \ref{dp_min_simon}, by total indiscernibility
  it is enough to show that if $b \in S$ and $I$ is a countable sequence indiscernible over $\emptyset$,
  then one element can be excluded from $I$, so that the remaining elements have the same type over $b$.
  But this is precisely Lemma \ref{exclude}.
\end{proof}

\section{Conclusion}
The determination of dp-minimality is the first step towards establishing bounds on vc-density.
It is this author's hope that the simple structure of flat graphs yields nicely behaved vc-density.
We pose the following question for the future work:
\begin{openq} 
  What are bounds on vc-function $\vc(n)$ in flat graphs?
  In particular do we have $\vc(1) = 1$ or $\vc(n) = n \vc(1)$?
  Are the bounds better in specific classes of flat graphs,
  such as planar graphs, graphs with bounded tree-width, or graphs excluding certain classes of subgraphs?
\end{openq}


 
\bibliography {thesis_ref}    % bibliography references
\bibliographystyle {uclathes_my}

\end {document}