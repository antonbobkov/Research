\chapter{Dp-minimality in Superflat Graphs}
Superflat graphs were introduced in \cite{stable_graphs} as a natural class of stable graphs.
This family of graphs comes up in a combinatorial context as nowhere dense graphs, see \cite{adleradler} and \cite{nowhere}.
In this chapter we prove that superflat graphs are dp-minimal.
It may be possible to prove dp-minimality by combining a characterization of dp-minimality in stable theories studied in \cite{dpstable}
with the results on forking in superflat graphs given in \cite{ivanov}.
Here, however, we present a direct proof using the characterization of dp-minimality by Lemma \ref{dp_min_simon}.

Sec\-tion~1 gives all the necessary combinatorial and model-theoretic definitions.
In addition, we list several basic results involving connectivity hulls and superflat graphs.
In Sec\-tion~2 we study how to expand parameter sets of indiscernible sequences to increase distance between the elements of those sequences.
Sec\-tion~3 applies a special case of this result to show dp-mimality via Lemma \ref{dp_min_simon}.
In the concluding section we outline directions for the future work.

\section{Preliminaries}
First, we introduce some basic graph-theoretic definitions.
\begin{Definition}
  Work in a possibly infinite graph $\GG$. Let $A, B, S, V \subseteq G$ where $G$ is the set of vertices of $\GG$.
  \begin{enumerate}
  \item A \defn{path} is a subgraph of $\GG$ with distinct vertices $v_0, v_1, \ldots, v_n$ and an edge between $v_{i-1}, v_i$ for all $i = 1, \ldots n$.
    It is called a path from $A$ to $B$ if $v_0 \in A$ and $v_n \in B$.
    The \defn{length} of such a path is $n$.
  \item Two paths are \defn{disjoint} if, excluding endpoints, they have no vertices in common.
  \item For $a,b \in G$ define the \defn{distance} $d(a,b)$ between $a$ and $b$ to be the length of the shortest path from $a$ to $b$ in $G$.
    If no such path exists then the distance is infinite.
  \item For $a,b \in G - A$ define $d_A(a,b)$ to be the distance between $a$ and $b$ in the subgraph of $\GG$
    induced on the set of vertices $G - A$. Equivalently it is the shortest path between $a$ and $b$ that avoids vertices in $A$.
  \item We say that $S$ \defn{separates} $A$ from $B$ if there exists $a \in A - S$, $b \in B - S$, with $d_S(a,b) = \infty$.
  \item We say that $A$ \defn{separates} $V$ if it separates $V$ from itself.
  \item We say that $V$ has \defn{connectivity} $n$ if there is a set of size $n$ that separates $V$,
    but there are no sets of size $n-1$ that separate $V$.
  \item Suppose $V$ has connectivity $n$. The \defn{connectivity hull} of $V$ is defined to be the union of all sets of size $n$ separating $V$.
  \end{enumerate}
\end{Definition}

In \cite{infinite_megner} we find a generalization of Menger's Theorem for infinite graphs:

\begin{Theorem} \label{megner}
  Let $A$ and $B$ be two sets of vertices in a possibly infinite graph. Then there exists a set $P$ of disjoint paths from $A$ to $B$, and a set $S$ of vertices separating $A$ from $B$, such that $S$ consists of a choice of precisely one vertex from each path in $P$.
\end{Theorem}

We use the following easy consequences:

\begin{Corollary} \label{cr_disjoint_paths}
  Let $V$ be a subset of vertices of a graph $\GG$ with connectivity $n$. Then there exists a set of $n$ disjoint paths from $V$ into itself.
\end{Corollary}

\begin{Corollary} \label{cr_hull_finite}
  With assumptions as above, the connectivity hull of $V$ is finite.
\end{Corollary}

\begin{proof}
  All the separating sets have to have exactly one vertex in each of those paths. 
\end{proof}

\begin{Definition} \ 
  \begin{itemize}
  \item $K^m_n$ denotes the graph obtained from a complete graph on $n$ vertices by adding $m$ vertices to every edge.
    $K^m_\infty$ denotes the same construction on a complete graph with an infinite countable number of vertices.
    \item A graph is called \defn{flat} if for every $m \in \N$ the graph avoids $K^m_\infty$ as a subgraph. 
    \item A graph is called \defn{superflat} if for every $m \in \N$ there is $n \in \N$ such that the graph avoids $K^m_n$ as a subgraph. 
  \end{itemize}
\end{Definition}
It is easy to see by compactness that a graph is superflat if and only if there is an elementary extension which is flat.
By the same line of reasoning, in uncountably saturated structures notions of flatness and superflatness coincide.

Theorem 2 in \cite{stable_graphs} gives a useful characterization of the superflat graphs.

\begin{Theorem} \label{th_superflat_equivalence}
  The following are equivalent:
  \begin{enumerate}
  \item $\GG$ is superflat.
  \item For every $n \in \N$ and an infinite set $A \subseteq G$, there exists a finite $B \subseteq G$ and infinite $A' \subseteq A$ such that for all $x,y \in A'$ we have $d_{B}(x, y) > n$.
  \end{enumerate}
\end{Theorem}

Roughly, in superflat graphs every infinite set contains a sparse infinite subset (possibly after throwing away finitely many nodes).

We also note the stability result:
\begin{Theorem} [see Corollary 10 in \cite{stable_graphs}]
  Every superflat graph is stable.
\end{Theorem}

\section{Indiscernible sequences}

Fix an uncountable cardinal $\kappa$.
Work in a superflat graph $\SS$ that is $\kappa$-saturated and strongly $\kappa$-homogeneous.
Fix a parameter set $A \subset S$ with $|A| < \kappa$.
Let $I = (a_i)_{i \in \I}$ be a countable $A$-indiscernible sequence.
Stability implies that $I$ is totally indiscernible (see Lemma \ref{totally}).

\begin{Definition} \ 
  \begin{itemize}
  \item For a subsequence $\J \subset \I$ let $a(\J)$ denote the tuple obtained by concatenating $(a_j)_{j \in \J}$.
  \item Let $m$ be the arity of elements of $I$, that is, $a_i \in S^m$.
    We call a set $H \subseteq S$ \defn{uniformly definable} from $I$ if there
    is a formula $\phi(x, y_1, \ldots, y_k)$ with $|y_i| = m$
    such that for every $\J \subseteq \I$ of size $k$ we have $H = \phi(G, a(\J))$.
  \end{itemize}
\end{Definition}

First suppose that $I$ consists of singletons, that is $a_i \in S$.

\begin{Definition}
  Let $V \subseteq S$. Define $P_n(V)$, a subgraph of $\SS$, to be the union of all paths of length $\leq n$ between the vertices of $V$.
\end{Definition}

\begin{Lemma} \label{lm_bump}
  Let $n \in \N$.
  There exists a finite set $B \subset S$ such that
  \begin{align*}
    \forall i \neq j \ d_B(a_i, a_j) > n.
  \end{align*}
\end{Lemma}

\begin{proof}
  By a \ref{th_superflat_equivalence} we can find an infinite $\J \subseteq \I$ and a finite set $B'$
  such that each pair from $J = (a_j)_{j \in \J}$ has distance $>n$ over $B'$.
  By total indiscernibility there exists an automorphism mapping $J$ to $I$ and fixing $A$.
  The image of $B'$ under this automorphism is the required set $B$.
\end{proof}

In other words, $B$ separates $I$ when viewed inside the subgraph $P_n(I)$.
This shows that $I$ has finite connectivity in $P_n(I)$.
Applying Corollary \ref{cr_hull_finite} we obtain that the connectivity hull of $I$ in $P_n(I)$ is finite.


\begin{Definition}
  Given a graph $\GG$ and $V \subseteq G$ define $H(\GG, V) \subseteq G$ to be the connectivity hull of $V$ in $\GG$.
  Note that if $V$ is finite, then $H(P_n(V), V)$ is $V$-definable.
\end{Definition}

\begin{Lemma} \label{lm_uniform}
  Let $H$ be the connectivity hull of $I$ inside the graph $P_n(I)$, that is, $H = H(P_n(I), I)$.
  Then $H$ is uniformly definable from $I$ in $\SS$.
\end{Lemma}

\begin{proof}
  Using total indiscernibility we may assume without the loss of generality that $I$ is indexed by $\N$.
  Let $I_i = \paren{a_0, a_1, \ldots, a_{i-1}}$ a finite subsequence of the sequence $I$.
  Let $N$ be the connectivity of $I$ inside of $P_n(I)$.
  % Define $H_i = H(P_n(I_i), I_i)$.
  % It is $I_i$-definable.

  First note that any finite set $H \subseteq P_n(I)$ will be contained in $P_n(I_i)$ for large enough $i$.
  Every element of $H$ is inside a path of length $\leq n$ and endpoints of that path are eventually going to be inside $I_i$.
  (Here the assumption that $I$ is enumerated by $\N$ is important.)

  Vertices $a_0, a_1$ cannot be separated by less than $N$ elements inside of $P_n(I)$
  (as this would contradict connectivity being $N$).
  Thus by Theorem \ref{megner} there are $N$ disjoint paths inside of $P_n(I)$ connecting $a_0$ to $a_1$.
  For large enough $i$, say $i \geq M_1$, all these paths are contained inside of $P_n(I_i)$.
  Those paths witness that vertices $a_0, a_1$ cannot be separated by less than $N$ elements inside of $P_n(I_i)$.
  As the set $P_n(I_i)$ is $I_i$-definable and $I$ is indiscernible,
  we have that no two vertices can be separated by less than $N$ elements inside of $P_n(I_i)$
  Thus $I_i$ has connectivity $\geq N$ inside of $P_n(I_i)$ for $i \geq M_1$.

  Consider a set $S$ of size $N$ that separates $I$ inside of $P_n(I)$.
  This is witnessed by two elements of $I$ that are separated.
  There are finitely many such sets $S$ as connectivity hull is finite.
  Thus for large enough $i$, say $i \geq M_2$, for each such $S$
  the segment $I_i$ contains a pair of vertices witnessing that $S$ is a separating set.

  Corollary \ref{cr_disjoint_paths} tells us that there are finitely many paths between elements of $V$ such that
  $H(P_n(I), I)$ is inside the union of those paths.
  For large enough $i$, say $i \geq M_3$, $P_n(I_i)$ will contain all of those paths, and thus $H(P_n(I), I) \subseteq P_n(I_i)$.

  Combine those three observations.
  Let $M = \max(M_1, M_2, M_3)$.
  Then for $i \geq M$ the set $P_n(I_i)$ contains all the $N$-element sets separating $I$ in $P_n(I)$,
  those sets separate $I_i$ in $P_n(I_i)$,
  and the connectivity of $I_i$ in $P_n(I_i)$ is at most $N$.
  But this means that the connectivity of $I_i$ in $P_n(I_i)$ has to be exactly $N$, and $H(P_n(I), I) \subseteq H(P_n(I_i), I_i)$.

  For $i \geq M$ define
  \begin{align*}
    E_i = \bigcap_{j = M}^{i} H(P_n(I_j), I_j).
  \end{align*}
  We have $H(P_n(I), I) \subseteq E_i$ and $E_i$ is a decreasing chain.
  Suppose $H(P_n(I), I) \subsetneq H(P_n(I_M), I_M)$, that is ${H(P_n(I_M), I_M) - H(P_n(I), I)} \neq \emptyset$.
  Then there exists a set $S$ of size $N$ that separates $I_M$ in $P_n(I_M)$ but does not separate $I$ in $P_n(I)$.
  Thus there has to be a finite subgraph of $P_n(I)$ disjoint from $S$ that connects all the elements of $I_M$ (witness of failure of separation).
  For large enough $i$, say $i \geq M_S$, this subgraph is contained in $P_n(I_i)$.
  There are finitely many possibilities for $S$ (as connectivity hull of $I_M$ in $P_n(I_M)$ is finite).
  Let $M_4 = \max_S(M_S)$.
  Then for $i \geq \max(M_4, M)$ we have
  \begin{align*}
    H(P_n(I_i), I_i) \cap \paren{H(P_n(I_M), I_M) - H(P_n(I), I)} = \emptyset,    
  \end{align*}
  and thus $E_i = H(P_n(I), I)$.
  As $E_i$ is $I_i$-definable, this shows that $H(P_n(I), I)$ is $I_i$-definable.
  Now we need to show uniform definability.
  Suppose $I'$ is a subsequence of $I$ of length $i$.
  There is an automorphism mapping $I_i$ to $I'$ that fixes $I$ setwise.
  But this automorphism has to fix $H(P_n(I), I)$ setwise,
  so it maps an $I_i$-definition of $H(P_n(I), I)$ to an $I'$-definition of $H(P_n(I), I)$.
  As $I'$ was arbitrary this shows uniformity.
\end{proof}

\begin{Corollary} \label{inf_dis}
  Let $H_n = H(P_n(I), I)$.
  Then 
  \begin{align*}
    \forall i \neq j \ d_{H_n}(a_i, a_j) > n.
  \end{align*}
\end{Corollary}

\begin{proof}
  The set $H_n$ separates $I$ inside of $P_n(I)$.
  In particular there exist $i \neq j$ such that $d_{H_n}(a_i, a_j) = \infty$ inside $P_n(I)$.
  This means that $d_{H_n}(a_i, a_j) > n$ inside of $\SS$.
  But then by total indiscernibility and using the fact that $H_n$ is uniformly $I$-definable, this holds for all $i \neq j$.
\end{proof}

\newcommand{\tpl}[2]{{#1}^{(#2)}}

We would like to start working with tuples now instead of singletons.
We need some notation to extract individual elements of a tuple:
\begin{Definition}
  Suppose $a = (a_1, \ldots, a_m)$ is a tuple of arity $m$.
  Let $\tpl{a}{j}$ denote the $i$'th component, that is $\tpl{a}{j} = a_j$.
\end{Definition}

More generally, now suppose that $I$ consists of tuples of arity $m$, that is $a_i \in S^m$.

\begin{Definition} \ 
  \begin{itemize}
  \item We would like extract $j$'th components out of elements of $I$.
    Let $\tpl{I}{j} = (\tpl{a_i}{j})_{i \in \I}$, an $A$-indiscernible sequence of singletons.
  \item Let $\tpl{H_n}{j} = H(P_n(\tpl{I}{j}), \tpl{I}{j})$.
  \item Let
    \begin{align*}
      B_n = \bigcup_{i = 1}^{n} \bigcup_{j = 1}^{m} \tpl{H_n}{j}.
    \end{align*}
    Note that $B_n$ is finite as each $\tpl{H_n}{j}$ is finite by Corollary \ref{cr_hull_finite}.
  \end{itemize}
\end{Definition}

\begin{Lemma} \label{cr_bump}
  The sequence $I$ is indiscernible over $A \cup B_n$.
\end{Lemma}

\begin{proof}
  By Lemma \ref{lm_uniform} the set $\tpl{H_n}{j}$ is uniformly $\tpl{I}{j}$-definable.
  Thus it is uniformly $I$-definable.
  Then $B_n$ is a finite union of uniformly $I$-definable sets, thus also uniformly $I$-definable.

  By uniform definability there is a formula $\phi(z, w_1, \ldots , w_k)$ with $|z| = 1$ and $|w_i| = m$ such that
  for any subsequence $\J \subset \I$ of length $k$ we have $\phi(G, a(\J)) = B_n$.
  Fix such a subsequence $\J$.

  Let $\psi(x_1, \ldots, x_l ,y)$ be an arbitrary $A$-formula with $|x_i| = m$. 
  Consider the collection of traces (i.e a collection of subsets of $B_n^{|y|}$)
  \begin{align*}
    \curly{\psi(a(\J'), B_n^{|y|}) \mid \text{ $\J'$ a subsequence of $\I$ of length $l$ disjoint from $\J$}}.
  \end{align*}
  If two of the traces are distinct, then by indiscernibility all of them are (using the fact that $B_n$ is uniformly definable).
  But that is impossible as $B_n$ is finite and thus has finitely many subsets.
  Thus all such traces are identical.
  As the choice of $\J$ was arbitrary, we can drop the condition that $\J'$ is disjoint from $\J$.
  This shows that for any $\J_1, \J_2 \subset \I$ of length $l$ and $h \in B_n^{|y|}$ we have
  \begin{align*}
    \SS \models \psi(a(\J_1), h) \iff \SS \models \psi(a(\J_2), h)    .
  \end{align*}
  As the choice of $\psi$ was arbitrary, this shows that $I$ is indiscernible over $A \cup B_n$ as needed.
\end{proof}

\begin{Definition}
  For tuples $a,b$ of the same arity $m$ and $B \subset S$ define
  \begin{align*}
    d_B(a,b) = \min_{1 \leq i,j \leq m} d_B(\tpl{a}{i}, \tpl{b}{j}).
  \end{align*}
\end{Definition}

\begin{Lemma} \label{inf_dis_gen}
  \begin{align*}
    \forall i \neq j \ d_{B_n}(a_i, a_j) > n/2.
  \end{align*}
\end{Lemma}

\begin{proof}
  Towards a contradiction suppose we have some $i \neq j$ and $k, l$ such that
  \begin{align*}
    d_{B_n}(\tpl{a_i}{k}, \tpl{a_j}{l}) \leq n/2.
  \end{align*}
  As $B_n$ is uniformly $I$-definable, by total indiscernibility we have that this inequality holds for all $i \neq j$.
  Assuming for convenience that $I$ is enumerated by naturals, let $b_1 = \tpl{a_1}{k}$, $b_2 = \tpl{a_2}{l}$, $b_3 = \tpl{a_3}{k}$
  (note the superscripts).
  Then we have
  \begin{align*}
    &d_{B_n}(b_1, b_2) \leq n/2, \\
    &d_{B_n}(b_3, b_2) \leq n/2.
  \end{align*}
  By the triangle inequality
  \begin{align*}
    &d_{B_n}(b_1, b_3) \leq n, \\
    &d_{B_n}(\tpl{a_1}{k}, \tpl{a_3}{k}) \leq n.
  \end{align*}
  But this is a contradiction, as Lemma \ref{inf_dis} gives us
  \begin{align*}
    \forall i \neq j \ d_{\tpl{H_n}{k}}(\tpl{a_i}{k}, \tpl{a_j}{k}) > n
  \end{align*}
  and we have $\tpl{H_n}{k} \subseteq B_n$.
\end{proof}

\begin{Corollary} \label{inf_dis}
  There is a countable $B$ such that $I$ is indiscernible over $A \cup B$ and
  \begin{align*}
    \forall i \neq j \ d_B(a_i, a_j) = \infty.
  \end{align*}
\end{Corollary}

\begin{proof}
  Let $B_n$ as above. By Lemma \ref{inf_dis_gen} we have 
  \begin{align*}
    \forall i \neq j \ d_{B_n}(a_i, a_j) > n,
  \end{align*}
  and $I$ is indiscernible over $A \cup B_n$ by Lemma \ref{cr_bump}.
  Let $B = \bigcup_{n \in \N} B_n$.
  Then
  \begin{align*}
    \forall i \neq j \ d_{B}(a_i, a_j) = \infty.
  \end{align*}
  As $B_n \subseteq B_{n+1}$, the sequence $I$ is indiscernible over $A \cup B$ as needed.
\end{proof}

Thus $I$ can be upgraded to have infinite distance over its parameter set.

\section{Superflat graphs are dp-minimal}

\begin{Definition}
  For $B \subseteq S$ define an equivalence relation $\sim_B$ on $S - B$ as follows:
  two vertices $b,c$ are $\sim_B$-equivalent if $d_B(b,c)$ is finite.
\end{Definition}

\begin{Lemma}
  Fix tuples $a,b,c$ in $S$, with $a,b$ having the same arity.
  Also let $B \subseteq S$.
  Suppose $\tp(a/B) = \tp(b/B)$ and $d_B(a, c) = d_B(b, c) = \infty$.
  Then $\tp(a/Bc) = \tp(b/Bc)$.
\end{Lemma}

\begin{proof}
  Suppose $a = (a_1, a_2, \ldots, a_m)$ and $b = (b_1, b_2, \ldots, b_m)$.
  Define $X_j$ to be the $\sim_B$-equivalence class of $a_j$ or $X_j = \emptyset$ if $a_j \in B$.
  Similarly define $Y_j$ for $b_j$.
  There is an automorphism $f$ of $\SS$ fixing $B$ with $f(a) = b$.
  It's easy to see that $f(X_j) = Y_j$ setwise.
  We would like to define a function $g \colon S \arr S$.
  For each $j$ let $g = f$ on $X_j$.
  Additionally if $X_j \neq Y_j$ then also let $g = f^{-1}$ on $Y_j$.
  Define $g$ to be identity on the rest of $S$.
  It is easy to check that $g$ is a well-defined automorphism fixing $Bc$ that maps $a$ to $b$.
  This shows that $\tp(a/Bc) = \tp(b/Bc)$.
\end{proof}

\begin{Lemma} \label{exclude}
  % Let $G$ be a flat graph with $(a_i)_{i\in\Q}$ indiscernible over $A$ and $b \in G$.
  Let $b \in G$.
  There exists $c \in \I$ such that all $(a_i)_{i \in \I - c}$ have the same type over $Ab$.
\end{Lemma}

\begin{proof}
  Use Corollary \ref{inf_dis} to find $B \supseteq A$ such that $I$ is indiscernible over $B$ and has infinite distance over $B$.
  All the tuples of the indiscernible sequence fall into distinct $\sim_B$-equivalence classes.
  If $b \in B$ we are done.
  Otherwise, there can be at most one element of the sequence that is in the same $\sim_B$-equivalence class as $b$.
  Exclude that element from the sequence.
  Remaining sequence elements are all infinitely far away from $b$ over $B$.
  By the previous lemma we have that elements of indiscernible sequence all have the same type over $Bb$ as needed.
\end{proof}


\begin{Theorem} \label{flat_dp_thm}
  Superflat graphs are dp-minimal.
\end{Theorem}
\begin{proof}
  It suffices to show that $\SS$ is dp-minimal.
  Using Lemma \ref{dp_min_simon}, by total indiscernibility
  it is enough to show that if $b \in S$ and $I$ is a countable sequence indiscernible over $\emptyset$,
  then one element can be excluded from $I$, so that the remaining elements have the same type over $b$.
  But this is precisely Lemma \ref{exclude}.
\end{proof}

\section{Conclusion}
The determination of dp-minimality is the first step towards establishing bounds on VC-density.
It is this author's hope that the simple structure of superflat graphs yields nicely behaved VC-density.
We pose the following question for the future work:
\begin{openq} 
  What are bounds on VC-density function $\vc(n)$ in superflat graphs?
  In particular do we have $\vc(1) = 1$ or $\vc(n) = n \vc(1)$?
  Are the bounds better in specific classes of superflat graphs,
  such as planar graphs, graphs with bounded tree-width, or graphs excluding certain classes of subgraphs?
\end{openq}
