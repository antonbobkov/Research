\chapter{Dp-minimality in Flat Graphs}

In this chapter we show that the theory of superflat graphs is dp-minimal.

\section{Preliminaries}

Superflat graphs were introduced in \cite{stable_graphs} as a natural class of stable graphs. Here we present a direct proof showing dp-minimality.

First, we introduce some basic graph-theoretic definitions.
\begin{Definition}
  Work in a possibly infinite graph $\GG$. Let $A, B, S, V \subseteq G$ where $G$ is the set of vertices of $G$.
  \begin{enumerate}
  % \item $G' = G[V]$ is called the \defn{induced} subgraph of $G$ \defn{spanned} by $V$. It is obtained as a subgraph of $G$ by taking all edges between vertices in $V$.
  \item A \defn{path} is a subgraph of $\GG$ with distinct vertices $v_0, v_1, \ldots, v_n$ and an edge between $v_{i-1}, v_i$ for all $i = 1, \ldots n$.
    It is called a path from $A$ to $B$ if there is an edge between $v_0$ and a vertex in $A$ and there is an edge between $v_n$ and a vertex in $B$.
    A \defn{length} of such a path is $n+2$.
  \item For $a,b \in V(G)$ define the \defn{distance} $d(a,b)$ to be the length of the shortest path between $a$ and $b$ in $G$.
  \item For $a,b \in V(G)$ define $d_A(a,b)$ to be the distance between $a$ and $b$ in $G[V(G) - A]$. Equivalently it is the shortest path between $a$ and $b$ that avoids vertices in $A$.
  \item We say that $S$ \defn{separates} $A$ from $B$ if there exist $a \in A - S$, $b \in B - S$, with $d_S(a,b) = \infty$.
  \item We say that $A$ \defn{separates} $V$ if it separates $V$ from itself.
  \item We say that $V$ has \defn{connectivity} $n$ if there is a set of size $n$ that separates $V$,
    but there are no sets of size $n-1$ that separate $V$.
  \item Suppose $V$ has connectivity $n$. \defn{A connectivity hull} of $V$ is defined to be the union of all sets separating $V$ of size $n$.
  \end{enumerate}
\end{Definition}

In \cite{infinite_megner} we find a generalization of Megner's Theorem for infinite graphs:

\begin{Theorem} \label{megner}
  Let $A$ and $B$ be two sets of vertices in a possibly infinite graph. Then there exists a set $P$ of disjoint paths from $A$ to $B$, and a set $S$ of vertices separating $A$ from $B$, such that $S$ consists of a choice of precisely one vertex from each path in $P$.
\end{Theorem}

We use the following easy consequence:

\begin{Corollary} \label{cr_disjoint_paths}
  Let $V$ be a subset of vertices of a graph $\GG$ with connectivity $n$. Then there exists a set of $n$ disjoint paths from $V$ into itself.
\end{Corollary}

\begin{Corollary} \label{cr_hull_finite}
  With assumptions as above, the connectivity hull of $V$ is finite.
\end{Corollary}

\begin{proof}
  All the separating sets have to have exactly one vertex in each of those paths. 
\end{proof}

\begin{Definition} \ 
  \begin{itemize}
    \item A graph $K^m_n$ denotes a graph obtained from a complete graph on $n$ vertices with $m$ vertices added to every edge.
    \item A graph is called \defn{superflat} if for every $m \in \N$ there is $n \in \N$ such that the graph avoids $K^m_n$ as a subgraph. 
  \end{itemize}
\end{Definition}

Theorem 2 in \cite{stable_graphs} gives a useful characterization of the superflat graphs.

\begin{Theorem} \label{th_superflat_equivalence}
  The following are equivalent:
  \begin{enumerate}
  \item $\GG$ is superflat.
  \item For every $n \in \N$ and an infinite set $A \subseteq G$, there exists a finite $B \subseteq G$ and infinite $A' \subseteq A$ such that for all $x,y \in A'$ we have $d_{B}(x, y) > n$.
  \end{enumerate}
\end{Theorem}

Roughly, in superflat graphs every infinite set contains a sparse infinite subset (possibly after throwing away finitely many nodes).

\section{Indiscernible sequences}

Fix an uncoutable cardinal $\kappa$.
Work in a superflat graph $\SS$ that is $\kappa$-saturated and strongly $\kappa$-homogeneous.
Fix a parameter set $A \subset S$ with $|A| < \kappa$.
Let $I = (a_i)_{i \in \I}$ be a countable $A$-indiscernible sequence.% of singletons (that is $|a_i| = 1$).
Stability implies that $I$ is totally indiscernible (see Lemma \ref{totally}).

\begin{Definition}
  % Let $m$ be the arity of elements of $I$, that is $a_i \in S^m$.
  We call a set $H \subseteq S$ \defn{uniformly definable} from $I$ if there
  is a formula $\phi(x, y_1, \ldots, y_k)$ such that for every $J \subseteq I$ of size $k$ we have
  \begin{align*}
    H = \{g \in G \mid \phi(g, J)\},
  \end{align*}
  where $J$ is considered a tuple.
\end{Definition}

First suppose that $I$ consists of singletons, that is $a_i \in S$.

\begin{Definition}
  Let $V \subseteq S$. Define $P_n(V)$, a subgraph of $\SS$, to be a union of all paths of length $\leq n$ between the vertices of $V$.
\end{Definition}

\begin{Lemma} \label{lm_bump}
  Fix $n \in \N$.
  Then there exists a finite set $B$ such that
  \begin{align*}
    \forall i \neq j \ d_B(a_i, a_j) > n.
  \end{align*}
\end{Lemma}

\begin{proof}
  By a \ref{th_superflat_equivalence} we can find an infinite $\J \subseteq \I$ and a finite set $B'$
  such that each pair from $J = (a_j : j \in \J)$ has distance $>n$ over $B'$.
  Using total indiscernibility we have an automorphism sending $J$ to $I$ fixing $A$.
  Image of $B'$ under this automorphism is the required set $B$.
\end{proof}

In other words, $B$ separates $I$ when viewed inside the subgraph $P_n(I)$.
This shows that $I$ has finite connectivity in $P_n(I)$.
Applying Corollary \ref{cr_hull_finite} we obtain that the connectivity hull of $I$ in $P_n(I)$ is finite.


\begin{Definition}
  Given a graph $\GG$ and $V \subseteq G$ define $H(\GG, V) \subseteq G$ to be the connectivity hull of $V$ in $\GG$.
  Note that if $V$ is finite, we have that $H(P_n(V), V)$ is $V$-definable.
\end{Definition}

\begin{Lemma} \label{lm_uniform}
  Let $H$ be the connectivity hull of $I$ inside of graph $P_n(I)$, that is $H = H(\P_n(I), I)$.
  Then $H$ is uniformly definable from $I$ in $\SS$.
\end{Lemma}

\begin{proof}
  Using total indiscernability we may assume without the loss of generality that $I$ is indexed by $\N$.
  Let $I_i = \{a_0, a_1, \ldots, a_{i-1}\}$ a finite segment of the sequence.
  Let $N$ be the connectivity of $I$ inside of $P_n(I)$.
  % Define $H_i = H(P_n(I_i), I_i)$.
  % It is $I_i$-definable.

  First note that any finite set $H \subseteq P_n(I)$ will be contained in $P_n(I_i)$ for large enough $i$.
  Every element of $H$ is inside a path of length $\leq n$ and endpoints of that path are eventually going to be inside $I_i$.
  Here the assumption that $I$ is enumerated by $\N$ is important.

  Vertices $a_0, a_1$ cannot be separated by less than $N$ elements inside of $P_n(I)$.
  By Theorem \ref{megner} there are $N$ disjoint paths inside of $P_n(I)$ connecting $a_0$ to $a_1$.
  For large enough $i$, say $i \geq M_1$, all these paths are contained inside of $P_n(I_i)$.
  Those paths witness that vertices $a_0, a_1$ cannot be separated by less than $N$ elements inside of $P_n(I_i)$.
  Thus by indiscernibility, no two vertices can be separated by less than $N$ elements inside of $P_n(I_i)$
  (using the fact that $P_n(I_i)$ is $I_i$-definable).
  Thus $I_i$ has connectivity $\geq N$ inside of $P_n(I_i)$ for $i \geq M_1$.

  Consider a set $S$ of size $N$ that separates $I$ inside of $P_n(I)$.
  This is witnessed by two elements of $I$ that are separated.
  There are finitely many such sets $S$ as connectivity hull is finite.
  Thus for large enough $i$, say $i \geq M_2$, for each such $S$
  the segment $I_i$ contains a pair of vertices witnessing that $S$ is a separating set.

  The Corollary \ref{cr_disjoint_paths} tells us that there are finitely many paths between elements of $V$ such that
  $H(P_n(I), I)$ is inside the union of those paths.
  For large enough $i$, say $i \geq M_3$, $P_n(I_i)$ will contain all of those paths, and thus $H(P_n(I), I) \subseteq P_n(I_i)$.

  Combine those three observations.
  Let $M = \max(M_1, M_2, M_3)$.
  Then for $i \geq M$ the set $P_n(I_i)$ contains all the $N$-element sets separating $I$ in $P_n(I)$,
  those sets separate $I_i$ in $P_n(I_i)$,
  and the connectity of $I_i$ in $P_n(I_i)$ is at most $N$.
  But this means that the connectity of $I_i$ in $P_n(I_i)$ has to be exactly $N$, and $H(P_n(I), I) \subseteq H(P_n(I_i), I_i)$.

  For $i \geq M$ define
  \begin{align*}
    E_i = \bigcap_{j = M}^{i} H(P_n(I_j), I_j).
  \end{align*}
  We have $H(P_n(I), I) \subseteq E_i$ and $E_i$ is a decreasing chain.
  Suppose $H(P_n(I), I) \subsetneq H(P_n(I_M), I_M)$, that is ${H(P_n(I_M), I_M) - H(P_n(I), I)} \neq \emptyset$.
  Then there exists a set $S$ of size $N$ that separates $I_M$ in $P_n(I_M)$ but does not separate $I$ in $P_n(I)$.
  There is a finite subgraph of $P_n(I)$ disjoint from $S$ that connects all the elements of $I_M$.
  For large enough $i$, say $i \geq M_S$, this subgraph is contained in $P_n(I_i)$.
  There are finite many possibilities for $S$ (as connectivity hull of $I_M$ in $P_n(I_M)$ is finite).
  Let $M_4 = \max_S(M_S)$.
  Then for $i \geq M_4$ we have
  \begin{align*}
    H(P_n(I_i), I_i) \cap \paren{H(P_n(I_M), I_M) - H(P_n(I), I)} = \emptyset,    
  \end{align*}
  and thus $E_i = (P_n(I), I)$.
  As $E_i$ is $I_i$-definable, we have that $(P_n(I), I)$ is $I_i$-definable.
  We need to show uniform definability.
  Suppose $I'$ is a subsequence of $I$ of length $i$.
  There is an automorphism mapping $I_i$ to $I'$ that is $I$-invariant.
  Thus it is $H(P_n(I), I)$-invariant and maps $I_i$-definition of $H(P_n(I), I)$ to a $I'$-definition of $H(P_n(I), I)$.
  As $I'$ was arbitrary this shows uniformity.
\end{proof}

\begin{Corollary} \label{inf_dis}
  Let $H_n = H(P_n(I), I)$.
  Then 
  \begin{align*}
    \forall i \neq j \ d_{H_n}(a_i, a_j) > n.
  \end{align*}
\end{Corollary}

\begin{proof}
  The set $H_n$ separates $I$ inside of $P_n(I)$.
  In particular there exist $i \neq j$ such that $d_{H_n}(a_i, a_j) = \infty$ inside $P_n(I)$.
  This means that $d_{H_n}(a_i, a_j) > n$ inside of $\SS$.
  But then by total indiscernibility and using the fact that $H_n$ is uniformly $I$-definable, this holds for all $i \neq j$.
\end{proof}

\newcommand{\tpl}[2]{{#1}^{(#2)}}

We would like to start working with tuples now, instead of singletons.
We need some notation to extract individual elements of a tuple:
\begin{Definition}
  Suppose $a = (a_1, \ldots, a_m)$ is a tuple of arity $m$.
  Let $\tpl{a}{j}$ denote the $i$'th component, that is $\tpl{a}{j} = a_j$.
\end{Definition}

More generally, now suppose that $I$ consists of tuples of arity $m$, that is $a_i \in S^m$.

\begin{Definition}
  \begin{itemize}
  \item We would like extract $j$'th components out of elements of $I$.
    Let $\tpl{I}{j} = (\tpl{a_i}{j})_{i \in \I}$, an $A$-indiscernible sequence of singletons.
  \item Let $\tpl{H_n}{j} = H(P_n(\tpl{I}{j}), \tpl{I}{j})$.
  \item Let
    \begin{align*}
      B_n = \bigcup_{i = 1}^{n} \bigcup_{j = 1}^{m} \tpl{H_n}{j}.
    \end{align*}
  \end{itemize}
\end{Definition}

\begin{Lemma} \label{cr_bump}
  The sequence $I$ is indiscernible over the $A \cup B_n$.
\end{Lemma}

\begin{proof}
  By Lemma \ref{lm_uniform} the set $\tpl{H_n}{j}$ is uniformly $\tpl{I}{j}$-definable.
  Thus it is uniformly $I$-definable.
  Then $B_n$ is a finite union of uniformly $I$-definable sets, thus also uniformly $I$-definable.

  For a subsequence $\J \subset \I$ let $a(\J)$ denote the tuple obtained by concatenating $(a_j)_{j \in \J}$.
  By uniform definability there is a formula $\phi(z, w_1, \ldots , w_k)$ with $|z| = 1$ and $|w_i| = m$ such that
  for any subsequence $\J \subset \I$ of length $k$ we have $\phi(G, a(\J)) = B_n$.
  Fix such a subsequence $\J$.

  Let $\psi(x_1, \ldots, x_l ,y)$ be an arbitrary $A$-formula with $|x_i| = m$. 
  Consider the collection of traces $\psi(a(\J'), B_n^{|y|})$ for subsequences $\J' \subset \I$ of length $l$ and disjoint from $\J$.
  If two of the traces are distinct, then by indiscernibility all of them are.
  But that is impossible as $B_n$ is finite and thus has finitely many subsets.
  Thus all such traces are identical.
  As the choice of $\J$ was arbitrary, we can drop the condition that $\J'$ is disjoint from $\J$.
  This shows that for any $\J_1, \J_2 \subset \I$ of length $l$ and $h \in B_n^{|y|}$ we have $\psi(a(\J_1), h) \iff \psi(a(\J_2), h)$.
  As choice of $\psi$ was arbitrary, this shows that $I$ is indiscernible over $A \cup B_n$ as needed.
\end{proof}

\begin{Definition}
  For tuples $a,b$ of the same arity $m$ and $B \subset S$ define
  \begin{align*}
    d_B(a,b) = \min_{1 \leq i,j \leq m} d_B(\tpl{a}{i}, \tpl{b}{j}).
  \end{align*}
\end{Definition}

\begin{Lemma} \label{inf_dis_gen}
  \begin{align*}
    \forall i \neq j \ d_{B_n}(a_i, a_j) > n/2.
  \end{align*}
\end{Lemma}

\begin{proof}
  Towards a contradiction suppose we have some $i \neq j$ and $k, l$ such that
  \begin{align*}
    d_{B_n}(\tpl{a_i}{k}, \tpl{a_j}{l}) \geq n/2.
  \end{align*}
  As $B_n$ is uniformly $I$-definable, by total indiscernability we have that this inequality holds for all $i \neq j$.
  Let $b_1 = \tpl{a_1}{k}$, $b_2 = \tpl{a_2}{l}$, $b_3 = \tpl{a_3}{k}$ (again, assuming for convenience that $I$ is enumerated by naturals).
  Then we have
  \begin{align*}
    &d_{B_n}(b_1, b_2) \geq n/2, \\
    &d_{B_n}(b_3, b_2) \geq n/2.
  \end{align*}
  By triangle inequality
  \begin{align*}
    &d_{B_n}(b_1, b_3) \geq n, \\
    &d_{B_n}(\tpl{a_1}{k}, \tpl{a_3}{k}) \geq n.
  \end{align*}
  But this is a contradiction, as Lemma \ref{inf_dis} gives us
  \begin{align*}
    \forall i \neq j \ d_{\tpl{H_n}{k}}(\tpl{a_i}{k}, \tpl{a_j}{k}) > n
  \end{align*}
  and we have $\tpl{H_n}{k} \subseteq B_n$.
\end{proof}

\begin{Corollary} \label{inf_dis}
  There is a countable $B$ such that $I$ is indiscernible over $A \cup B$ and
  \begin{align*}
    \forall i \neq j \ d_B(a_i, a_j) = \infty.
  \end{align*}
\end{Corollary}

\begin{proof}
  Let $B_n$ as above. By Lemma \ref{inf_dis_gen} we have 
  \begin{align*}
    \forall i \neq j \ d_{B_n}(a_i, a_j) > n,
  \end{align*}
  and $I$ is indiscernible over $A \cup B_n$ by Lemma \ref{cr_bump}.
  Let $B = \bigcup_{n \in \N} B_n$.
  Then
  \begin{align*}
    \forall i \neq j \ d_{B}(a_i, a_j) = \infty.
  \end{align*}
  As $B_i \subseteq B_{i+1}$, the sequence $I$ is indiscernible over $A \cup B$ as needed.
\end{proof}

That is $I$ can be upgraded to have infinite distance over its parameter set.

\section{Superflat graphs are dp-minimal}

\begin{Definition}
  For $B \subseteq S$ define an equivalence relation $\sim_B$ on $S - B$.
  Two vertices $b,c$ are $\sim_B$-equivalent if $d_B(b,c)$ is finite.
\end{Definition}

\begin{Lemma}
  Fix tuples $a,b,c$ in $S$, with $a,b$ having the same arity.
  Also let $B \subseteq S$.
  Suppose $\tp(a/B) = \tp(b/B)$ and $d_B(a, c) = d_B(b, c) = \infty$.
  Then $\tp(a/Bc) = \tp(b/Bc)$.
\end{Lemma}

\begin{proof}
  Suppose $a = (a_1, a_2, \ldots, a_m)$ and $b = (b_1, b_2, \ldots, b_m)$.
  Define $X_j$ to be the $\sim_B$-equivalence class of $a_j$ or $\emptyset$ if $a_j \in B$.
  Similarly define $Y_j$ for $b_j$.
  There is an automorphism $f$ of $\SS$ fixing $B$ mapping $a$ to $b$.
  It's easy to see that $f(X_j) = Y_j$.
  We would like to define a function $g \colon S \arr S$.
  For each $j$ let $g = f$ on $X_j$.
  Additionally if $X_j \neq Y_j$ then also let $g = f^{-1}$ on $Y_j$.
  Let $g$ be identity on the rest of $S$.
  It is easy to check that $g$ is a well-defined automorphism fixing $Bc$ that maps $a$ to $b$.
  This shows that $\tp(a/Bc) = \tp(b/Bc)$.
\end{proof}

\begin{Lemma} \label{exclude}
  % Let $G$ be a flat graph with $(a_i)_{i\in\Q}$ indiscernible over $A$ and $b \in G$.
  Suppose $c \in G$.
  Then there exists $c \in \I$ such that all $(a_i)_{i\in\{\I - c\}}$ have the same type over $Ab$.
\end{Lemma}

\begin{proof}
  Use Corollary \ref{inf_dis} to find $B \supseteq A$ such that $I$ is indiscernible over $B$ and has infinite distance over $B$.
  All the tuples of the indiscernible sequence fall into distinct $\sim_B$-equivalence classes.
  If $b \in B$ we are done.
  Otherwise, there can be at most one element of the sequence that is in the same $\sim_B$-equivalence class as $b$.
  Exclude that element from the sequence.
  Remaining sequence elements are all infinitely far away from $b$ over $B$.
  By the previous lemma we have that elements of indiscernible sequence all have the same type over $Bb$ as needed.
\end{proof}


\begin{Corollary}
  Flat graphs are dp-minimal.
\end{Corollary}
\begin{proof}
  Using Lemma \ref{dp_min_simon}, by total indiscernibility
  it is enough to show that if $I$ is a countable sequence indiscernible over $\emptyset$ and $b \in S$,
  then one element can be excluded from $I$, so that the remaining elements have the same type over $b$.
  But this is precisely Lemma \ref{exclude}.
\end{proof}

\section{Conclusion}
The determination of dp-minimality is the first step towards establishing bounds on vc-density.
It is this author's hope that the simple structure of flat graphs yields nicely behaved vc-density.
We pose the following question for the future work:
\begin{openq}
  What are bounds on vc-function $\vc(n)$ in flat graphs?
  In particular do we have $\vc(1) = 1$ or $\vc(n) = n \vc(1)$?
  Are the bounds better in specific classes of flat graphs,
  such as planar graphs, graphs with bounded tree-width, or graphs excluding certain types of subgraphs?
\end{openq}
