\chapter{Inroduction and Preliminaries}

\section{Inroduction}

My research concentrates on the concept of VC-density, a recent notion of rank in NIP theories.
The study of a structure in model theory usually starts with quantifier elimination, followed by a finer analysis of definable functions and interpretability.
The study of VC-density goes one step further, looking at a structure of the asymptotic growth of finite definable families.
In the most geometric examples, VC-density coincides with the natural notion of dimension.
However, no geometric structure is required for the definition of VC-density, thus we can get some notion of geometric dimension for families of sets given without any geometric context.

In 2013, Aschenbrenner et al. investigated and developed a notion of VC-density for NIP structures, an analog of geometric dimension in an abstract setting \cite{density}. Their applications included a bound for p-adic numbers, an object of great interest and a very active area of research in mathematics. My research concentrates on improving and expanding techniques of that paper to improve the known bounds as well as computing VC-density for other NIP structures of interest. I am able to obtain new bounds for the additive reduct of p-adic numbers, treese, and certain families of graphs. Recent research by Chernikov and Starchenko in 2015 \cite{regularity} suggests that having good bounds on VC-density in p-adic numbers opens a path for applications to incidence combinatorics (e.g. Szemeredi-Trotter theorem).

The concept of VC-dimension was first introduced in 1971 by Vapnik and Chervonenkis for set systems in a probabilistic setting \cite{vc71}.
The theory grew rapidly and found wide use in geometric combinatorics, computational learning theory, and machine learning.
Around the same time Shelah was developing the notion of NIP ("not having the independence property"),
a natural tameness property of (complete theories of) structures in model theory \cite{shelah_nip}.
In 1992 Laskowski noticed the connection between the two: theories where all uniformly definable families of sets have finite VC-dimension are exactly NIP theories \cite{laskowski92}.
It is a wide class of theories including algebraically closed fields, differentially closed fields, modules, free groups, o-minimal structures, and ordered abelian groups.
A variety of valued fields fall into this category as well, including the p-adic numbers.

P-adic numbers were first introduced by Hensel in 1897 in \cite{hensel}, and over the following century a powerful theory was developed around them with numerous applications across a variety of disciplines, primarily in number theory, but also in physics and computer science.
In 1965 Ax, Kochen \cite{ak1} and Ershov \cite{er1} axiomatized the theory of p-adic numbers and proved a quantifier elimination result.
A key insight was to connect properties of the value group and residue field to the properties of the valued field itself.
In 1984 Denef proved a cell decomposition result for more general valued fields \cite{den84}.
This result was soon generalized to p-adic subanalytic and rigid analytic extensions, allowing for the later development of a more powerful technique of motivic integration.
The conjunction of those model theoretic results allowed to solve a number of outstanding open problems in number theory (e.g., Artin's Conjecture on p-adic homogeneous forms).

In 1997, Karpinski and Macintyre computed VC-density bounds for o-minimal structures and asked about similar bounds for p-adic numbers \cite{karp97}.
VC-density is a concept closely related to VC-dimension.
It comes up naturally in combinatorics with relation to packings, Hamming metric, entropic dimension and discrepancy.
 VC-density is also the decisive parameter in the Epsilon-Approximation Theorem, which is one of the crucial tools for applying VC theory in computational geometry.
In a model theoretic setting it is computed for families of uniformly definable sets.
 In 2013, Aschenbrenner, Dolich, Haskell, Macpherson, and Starchenko computed a bound for VC-density in p-adic numbers and a number of other NIP structures \cite{density}.
They observed connections to dp-rank and dp-minimality, notions first introduced by Shelah.
In well behaved NIP structures families of uniformly definable sets tend to have VC-density bounded by a multiple of their dimension, a simple linear behavior.
In a lot of cases including p-adic numbers this bound is not known to be optimal.
My research concentrates on improving those bounds and adapting those techniques to compute VC-density in other common NIP structures of interest to mathematicians.

Some of the other well behaved NIP structures are Shelah-Spencer graphs and flat graphs.
Shelah-Spencer graphs are limit structures for random graphs arising naturally in a combinatorial context.
Their model theory was studied by Baldwin, Shi, and Shelah in 1997 \cite{shi}, \cite{baldwin}.
Later work of Laskowski in 2006 \cite{laskowski} has provided a quantifier simplification result.
 Flat graphs were first studied by Podewski-Ziegler in 1978, showing that those are stable \cite{stable_graphs}, and later results gave a criterion for super stability.
Flat graphs also come up naturally in combinatorics in work of Nesetril and Ossona de Mendez \cite{nowhere}.
%Shelah-Spencer graphs and flat graphs are both subclasses of NIP theories, extremely well behaved model theoretically.

The first chapter of my dissertation concentrates on Shelah-Spencer graphs.
I have shown that they have infinite dp-rank, so they are poorly behaved as NIP structures.
I have also shown that one can obtain arbitrarily high VC-density when looking at uniformly definable families in a fixed dimension.
However I'm able to bound VC-density of individual formulas in terms of edge density of the graphs they define.

The second chapter of my dissertation concentrates on graphs and graph-like structures.
I have answered an open question from \cite{density}, computing VC-density for trees viewed as a partial order.
The main idea is to adapt a technique of Parigot \cite{parigot_trees} to partition trees into weakly interacting parts, with simple bounds of VC-density on each.
Similar partitions come up in the Podewski-Ziegler analysis of flat graphs \cite {stable_graphs}.
I am able to use that technique to show that flat graphs are dp-minimal, an important first step before establishing bounds on VC-density.


The third chapter of my dissertation deals with p-adic numbers and valued fields.
I have shown that VC-density is linear for an additive reduct of p-adic numbers (using a cell decomposition result from the work of Leenknegt in 2013 \cite{reduct}).
I will explore other reducts described in that paper, to see if my techniques apply to those as well.

\section{Basic Model Theory}

This section goes through the basics of the model theory used throughout this text.
It is meant to be used mostly as a reference on the notation as opposed to a comprehensive summary.
For a complete and more thorough introduction to the material, we refer the reader to Chapters 1 and 2 of \cite{tent}.
We begin with a short summary of languages, formulas, and structures:

\begin{Definition} \;
  \begin{itemize}
  \item A \defn{language} is a collection of predicate, function, and constant symbols.
  \item Fix language $\LL$ and a collection of variables.
    A \defn{term} is an expression constructed out of constants, variables, and functions.
  \item An \defn{atomic formula} is an expression constructed out of equality symbol or a predicate applied to terms.
  \item A \defn{(first-order) formula} is an expression constructed out of atomic formulas using boolean connectives
    $\wedge, \vee, \neg$ and quantifiers $\exists, \forall$.
    We denote such a formula as $\phi(x)$ where $x$ is a tuple of \defn{free variables},
    that is the variables used in $\phi$ that are not bound by quantifiers.
    Abusing notation, we denote $\LL$ to be the set of all such formulas (so $\phi \in \LL$).
  \item A formula without free variables is called a \defn{sentence}.
  \item A \defn{quantifier-free} formula is a formula that doesn't contain any quantifiers.
  \item A \defn{structure} $\MM$ consists of an infinite universe $M$ and functions, predicates, and constants matching those of $\LL$.
  \item For a variable tuple $x$, let $|x|$ be the arity of the tuple.
  \item Suppose we have a formula $\phi(x)$, structure $\MM$, and $a \in M^{|x|}$.
    Then we say that $\MM$ \defn{models} $\phi(a)$, denoted as $\MM \models \phi(a)$,
    if formula $\phi$ holds $\MM$ when we plug in $a$ into $x$.
  \item Suppose we have a structure $\MM$ and $A \subseteq M$.
    Then $\LL(A)$ denotes an expansion of $\LL$ by constant symbols correspoding to elements in $A$.
    The structure $\MM$ then can be viewed as a $\LL(A)$-structure with the appropriate interpretations.
    Formulas $\phi \in \LL(A)$ will be referred to as \defn{formulas with parameters from $A$} or simply as \defn{$A$-formulas}.
    In this context $A$ is usually referred to as a \defn{parameter set}.
  \item A \defn{theory} is a collection of sentences.
  \item For a theory $T$ and a structure $\MM$, we say that $\MM$ \defn{models} $T$,
    or that $\MM$ is a \defn{model} of $T$, if $\MM$ models every sentence in $T$.
  \item For a structure $\MM$, a \defn{theory of $\MM$} is a collection of all sentences that are modelled by $\MM$.
  \item A theory is called \defn{complete} if it is a theory of some structure $\MM$.
  \end{itemize}  
\end{Definition}

Throughout this text we often confuse complete theories with their models.
This is justfied for properties that can be described by a collection of first-order sentences.
Then a theory has this property if and only any (all) models have this property.
An example of that is a notion of stability.

Stability is a deep subject, with a lot of theory developed around it.
We won't work with it directly, but it is a property of some of the structures we study.
We present a definitnon for completeness and refer the reader to Chapter 8 of \cite{tent} or to \cite{pillay} for a more complete introduction.

\begin{Definition}
  \begin{itemize}
  \item Suppose we have a structure $\MM$.
    The formula $\phi(x,y)$ is called \defn{unstable} if for all natural $n$
    there exist $a_i \in M^{|x|}, b_i \in M^{|y|}$ for $0 \leq i \leq n$ such that
    \begin{align*}
      \MM \models \phi(a_i, b_j) \iff i \leq j.
    \end{align*}
  \item A formula is \defn{stable} if it is not unstable.
  \item A structure $\MM$ is \defn{stable} if all of its formulas are stable.
  \item A complete theory $T$ is \defn{stable} if any (all) of its models are stable.
  \end{itemize}
\end{Definition}

Definable sets are subsets of our structure given by formulas.
More precisely:
\begin{Definition}
  Suppose we have a structure $\MM$, a paramter set $A \subseteq M$ and an $A$-formula $\phi(x)$.
  Then
  \begin{align*}
    \phi(M^{|x|}) = \curly{m \in M^{|x|} \mid \MM \models \phi(m)}.
  \end{align*}
  is referred as an \defn{$A$-definable} subset of $M^{|x|}$ defined by $\phi$.
\end{Definition}

More generally, we will need a slightly more refined notion of a trace:
\begin{Definition}
  Suppose we have a structure $\MM$, a formula $\phi(x, y)$, tuples $a \in M^{|x|}, b \in M^{|y|}$, and
  sets $A \subseteq M^{|x|}, B \subseteq M^{|y|}$. 
  Define
  \begin{align*}
    \phi(A, b) &= \curly{a \in A \mid \MM \models \phi(a,b)}, \\
    \phi(a, B) &= \curly{b \in B \mid \MM \models \phi(a,b)}.
  \end{align*}
  These sets will be informally referred to as traces.
\end{Definition}

Types is one of the main tools of study in model theory:
\begin{Definition}
  Suppose $\MM$ is a structure, $B \subseteq M$.
  Also fix a variable tuple $x$.
  \begin{itemize}
  \item \defn{A partial type over $B$} is a collection of formulas in variable $x$ with parameters from $B$.
  \item A partial type $p(x)$ has a realization in $\MM$ if there exists $a \in M^{|x|}$ such that
    $\MM \models \phi(a)$ for all $\phi(x) \in p(x)$.
  \item A partial type is \defn{consistent} if its every finite subset of formulas has a realization.
  \item Suppose $a \in M^{|x|}$ and $\Delta \subseteq \LL(B)$ a collection of formulas in $x$.
    Define the \defn{$\Delta$-type of $a$ over $B$} to be a collection of formulas $\phi(x) \in \Delta$
    such that $\MM \models \phi(a)$.
    Denote it as $\tp_{\Delta}(a/B)$.
  \item Suppose $a \in M^{|x|}$.
    Define the \defn{type of $a$ over $B$} as the $\Delta$-type of $a$ over $B$ for $\Delta = \LL(B)$.
    Denote it as $\tp(a/B)$.
  \end{itemize}
\end{Definition}

Saturated structures is another important construction that we will be using.
Generally speaking, a lot of the model theory is done inside of saturated structures as it simplifies a lot of constructions.
The definition is as follows:
\begin{Definition}
  Let $\kappa$ be a cardinal.
  A structure $\MM$ is called $\kappa$-saturated if for all $B \subset M$ with $|B| < \kappa$
  we have that all consistent partial types over $B$ are realized in $\MM$.
\end{Definition}

Indiscernible sequences will be useful to us to describe dp-rank and dp-minimality.
They come up often in model theory as a way to leverage symmetry present in sequences and sets.
\begin{Definition}
  \begin{itemize}
  \item Suppose we have a sequence $(a_i)_{i \in \I}$ where $\I$ is an ordered index set.
    For $\J \subset \I$ the expression $a_{\J}$ denotes  a tuple obtained by concatenation of the sequence $(a_j)_{j \in \J}$
    (the sequence is ordered using the order of $\I$).    
  \item Suppose $\MM$ is a structure, $B \subseteq M$, and $\I$ is an oredred index set. 
    A sequence $(a_i)_{i \in \I}$ is called \defn{indiscernible over $B$} if
    for any two subsets $\J_1, \J_2 \subseteq \I$ of equal length we have 
    \begin{align*}
      \tp(a_{J_1}/B) = \tp(a_{J_2}/B).
    \end{align*}
  \item If we use the same definition, but allow tuples $a_{J_1}, a_{J_2}$ to be concatenated in arbitrary order,
    then we obtain the definition a sequence that is \defn{totally indiscernible over $B$} (alternatively a totally indiscernible set).
  \end{itemize}
\end{Definition}

Here is an important property of indiscernible sequences in stable thoeries:
\begin{Lemma}[see Lemma 9.1.1 in \cite{tent}] \label{totally}
  If a structure is stable then every indiscernible sequence is totally indiscernible.
\end{Lemma}

Sometimes instead of starting with an indiscernible sequence, we wish to construct one from a sequence with some degree of symmetry:
\begin{Lemma} [see Lemma 5.1.3 in \cite{tent}]
  Work in a $\aleph_1$-saturated structure $\MM$.
  Suppose $B \subset M$.
  Fix a variable tuple $x$ and a collection of formulas $\Delta(x_1, \ldots, x_n)$ with $|x_1| = |x|$.
  Suppose we can find an arbitrarily long sequence $(a_i)_{i \in \I}$ with $a_i \in M^{|x|}$ such that
  for any subset $\J \subseteq \I$ of length $n$ we have
  \begin{align*}
    \MM \models \Delta(a_{\J}).
  \end{align*}
  Then there exists an infinite indiscernible sequence $(a_i')_{i \in \omega}$ with
  \begin{align*}
    \MM \models \Delta(a_1', a_2', \ldots, a_n').
  \end{align*}
\end{Lemma}

Instead of working with types directly, it is often more convenient to work with automorphisms:
\begin{Definition}
  Suppose $\MM$ is a structure and $A \subset M$.
  An \defn{automorphism} of $\MM$ over $A$ is a bijection of $f \colon M \arr M$
  that fixes $A$ and preserves constants, relations, and functions of $\MM$.
  We use notation $f \in \Aut(\MM/A)$.
\end{Definition}

The following result is easy to show directly from the definition:
\begin{Lemma}
  Suppose $\MM$ is a structure, $A \subset M$, and $f \in \Aut(\MM/A)$.
  Suppose also that $a,b \in M^n$ such that $f(a) = b$.
  Then $tp(a/A) = tp(b/A)$.
\end{Lemma}

The converse of this result holds in a special type of structure:
\begin{Definition}
  Let $\MM$ be a structure and $\kappa$ a cardinal.
  Then $\MM$ is called \defn{strongly $\kappa$-homogeneous} if for all $A \subseteq M$ with $|A| < \kappa$
  we have that if for $a, b \in M^n$ if $tp(a/A) = tp(b/A)$ then
  there exists $f \in \Aut(\MM/A)$ such that $f(a) = b$.
\end{Definition}

Luckily, for a given theory one can always find a model sufficiently saturated and homogeneous:
\begin{Lemma} [see Theorem 6.1.7 in \cite{tent}]
  Let $T$ be a complete theory and $\kappa$ a cardinal.
  There exists a model of $T$ that is $\kappa$-saturated and strongly $\kappa$-homogeneous.
\end{Lemma}


\section{VC-dimension and VC-density}
Throughout this section we work with a collection $\F$ of subsets of an infinite set $X$.
We call the pair $(X, \F)$ a \defn{set system}.

\begin{Definition} \ 
  \begin{itemize} 
  \item Given a subset $A$ of $X$, we define the set system $(A, A \cap \F)$
    where $A \cap \F = \curly{A \cap F \mid F\in \F}$.
  \item For $A \subseteq X$ we say that $\F$ \defn{shatters} $A$ if $A \cap \F = \PP(A)$ (the power set of $A$).
  \end{itemize}    
\end{Definition}  

\begin{Definition}
  We say $(X, \F)$ has \defn{VC-dimension} $n$ if the largest subset of $X$ shattered by $\F$ is of size $n$.
  If $\F$ shatters arbitrarily large subsets of $X$, we say that $(X, \F)$ has infinite VC-dimension.
  We denote the VC-dimension of $(X, \F)$ by $\VC(X, \F)$.
\end{Definition}  

\begin{Note}
  We may drop $X$ from the notation $\VC(X, \F)$, as the VC-dimension doesn't depend on the base set and is determined by $(\bigcup \F, \F)$.
\end{Note}
Set systems of finite VC-dimension tend to have good combinatorial properties,
and we consider set systems with infinite VC-dimension to be poorly behaved.

Another natural combinatorial notion is that of the dual system of a set system:
\begin{Definition}
  For $a \in X$ define $X_a = \curly{F \in \F \mid a \in F}$.
  Let $\F^* = \curly{X_a \mid a \in X}$.
  We call $(\F, \F^*)$ the \defn{dual system} of $(X, \F)$.
  The VC-dimension of the dual system of $(X, \F)$ is referred to as the \defn{dual VC-dimension} of $(X, \F)$ and denoted by $\VC^*(\F)$.
  (As before, this notion doesn't depend on $X$.)
\end{Definition}  

\begin{Lemma} [see 2.13b in \cite{ash7}]
  A set system $(X, \F)$ has finite VC-dimension if and only if its dual system has finite VC-dimension.
  More precisely
  \begin{align*}
    \VC^*(\F) \leq 2^{1+\VC(\F)}.
  \end{align*}
\end{Lemma}

For a more refined notion of complexity of $(X, \F)$ we look at the traces of our family on finite sets:
\begin{Definition}
  Define the \defn{shatter function} $\pi_\F \colon \N \arr \N$ of $\F$ and the \defn{dual shatter function} $\pi^*_\F \colon \N \arr \N$ of $\F$ by 
  \begin{align*}
    \pi_\F(n) &= \max \curly{|A \cap \F| \mid A \subseteq X \text{ and } |A| = n} \\
    \pi^*_\F(n) &= \max \curly{\text{atoms($B$)} \mid B \subseteq \F, |B| = n}
  \end{align*}
  where atoms($B$) is the number of atoms in the boolean algebra of sets generated by $B$.
  Note that the dual shatter function is precisely the shatter function of the dual system: $\pi^*_\F = \pi_{\F^*}$.
\end{Definition}  

A simple upper bound is $\pi_\F(n) \leq 2^n$ (same for the dual).
If the VC-dimension of $\F$ is infinite then clearly $\pi_\F(n) = 2^n$ for all $n$. Conversely we have the following remarkable fact:
\begin{Theorem} [Sauer-Shelah '72, see \cite{sauer}, \cite{shelah}]
  If the set system $(X, \F)$ has finite VC-dimension $d$ then $\pi_\F(n) \leq {n \choose \leq d}$ for all $n$, where
  ${n \choose \leq d} = {n \choose d} + {n \choose d - 1} + \ldots + {n \choose 1}$.    
\end{Theorem}

Thus the systems with a finite VC-dimension are precisely the systems where the shatter function grows polynomially.
The VC-density of $\F$ quantifies the growth of the shatter function of $\F$: 
\begin{Definition}
  Define the \defn{VC-density} and \defn{dual VC-density} of $\F$ as
  \begin{align*}
    \vc(\F) &= \limsup_{n \to \infty}\frac{\log \pi_\F(n)}{\log n} \in \R^{\geq 0} \cup \curly{+\infty},\\
    \vc^*(\F) &= \limsup_{n \to \infty}\frac{\log \pi^*_\F(n)}{\log n}\in \R^{\geq 0} \cup \curly{+\infty}.
  \end{align*}
\end{Definition}

Generally speaking a shatter function that is bounded by a polynomial doesn't itself have to be a polynomial.
Proposition 4.12 in \cite{density} gives an example of a shatter function that grows like $n \log n$ (so it has VC-density $1$).

So far the notions that we have defined are purely combinatorial.
We now adapt VC-dimension and VC-density to the model theoretic context.

\begin{Definition}
  Work in a first-order structure $\MM$.
  Fix a finite collection of formulas $\Phi(x, y)$ in the language $\LL(M)$ of $\MM$.

  \begin{itemize}
  \item For $\phi(x, y) \in \LL(M)$ and $b \in M^{|y|}$ let 
    \begin{align*}
      \phi(M^{|x|}, b) = \{a \in M^{|x|} \mid \phi(a, b)\} \subseteq M^{|x|}.
    \end{align*}
  \item Let $\Phi(M^{|x|}, M^{|y|})= \{\phi(M^{|x|}, b) \mid \phi \in \Phi, b \in M^{|y|}\} \subseteq \PP(M^{|x|})$.
  \item Let $\F_\Phi = \Phi(M^{|x|}, M^{|y|})$, giving rise to a set system $(M^{|x|}, \F_\Phi)$.
  \item Define the \defn{VC-dimension} $\VC(\Phi)$ of $\Phi$ to be the VC-dimension of $(M^{|x|}, \F_\Phi)$, similarly for the dual.
  \item Define the \defn{VC-density} $\vc(\Phi)$ of $\Phi$ to be the VC-density of $(M^{|x|}, \F_\Phi)$, similarly for the dual.
  \end{itemize}

  We will also refer to the VC-density and VC-dimension of a single formula $\phi$
  viewing it as a one element collection $\Phi = \curly{\phi}$.
\end{Definition}

Counting atoms of a boolean algebra in a model theoretic setting corresponds to counting types,
so it is instructive to rewrite the shatter function in terms of types.

\begin{Definition} 
  \begin{align*}
    \pi^*_\Phi(n) &= \max \curly{\text{number of $\Phi$-types over $B$} \mid B \subseteq M, |B| = n}.
  \end{align*}
  Here a $\Phi$-type over $B$ is a maximal consistent collection of formulas of the form $\phi(x, b)$ or $\neg\phi(x, b)$
  where $\phi \in \Phi$ and $b \in B$.
\end{Definition}

The functions $\pi^*_{\Phi}$ and $\pi^*_{\F_\Phi}$ do not have to agree,
as one fixes the number of generators of a boolean algebra of sets and the other fixes the size of the parameter set.
However, as the following lemma demonstrates, they both give the same asymptotic definition of dual VC-density.

\begin{Lemma} \label{count_types}
  \begin{align*}
    \vc^*(\Phi) &= \text{degree of polynomial growth of $\pi^*_\Phi(n)$}  = \limsup_{n \to \infty}\frac{\log \pi^*_\Phi(n)}{\log n}.
  \end{align*}  
\end{Lemma}

\begin{proof}
  With a parameter set $B$ of size $n$, we get at most $|\Phi|n$ sets $\phi(M^{|x|}, b)$ with $\phi \in \Phi, b \in B$.
  We check that asymptotically it doesn't matter whether we look at growth of boolean algebra of sets generated by
  $n$ or by $|\Phi|n$ many sets.
  We have:
  \begin{align*}
    \pi^*_{\F_\Phi}\paren{n} \leq \pi^*_\Phi(n) \leq \pi^*_{\F_\Phi}\paren{|\Phi|n}.
  \end{align*}
  Hence:
  \begin{align*}
    &\vc^*(\Phi) \leq \limsup_{n \to \infty}\frac{\log \pi^*_\Phi(n)}{\log n} \leq \limsup_{n \to \infty}\frac{\log \pi^*_{\F_\Phi}\paren{|\Phi|n}}{\log n} = \\
    & = \limsup_{n \to \infty}\frac{\log \pi^*_{\F_\Phi}\paren{|\Phi|n}}{\log |\Phi|n} \frac{\log |\Phi|n}{\log n} =
      \limsup_{n \to \infty}\frac{\log \pi^*_{\F_\Phi}\paren{|\Phi|n}}{\log |\Phi|n} \leq \\
    &\leq \limsup_{n \to \infty}\frac{\log \pi^*_{\F_\Phi}\paren{n}}{\log n} = \vc^*(\Phi).
  \end{align*}
\end{proof} 

One can check that the shatter function and hence VC-dimension and VC-density of a formula are elementary notions,
so they only depend on the first-order theory of the structure $\MM$.

NIP theories are a natural context for studying VC-density.
In fact we can take the following as the definition of NIP:
\begin{Definition}
  Define $\phi$ to be NIP if it has finite VC-dimension in a theory $T$.
  A theory $T$ is NIP if all the formulas in $T$ are NIP.
\end{Definition}

In a general combinatorial context (for arbitrary set systems),
VC-density can be any real number in $0 \cup [1, \infty)$ (see \cite{ash8}).
Less is known if we restrict our attention to NIP theories.
Proposition 4.6 in \cite{density} gives examples of formulas that have non-integer rational VC-density in an NIP theory,
however it is open whether one can get an irrational VC-density in this model-theoretic setting.

Instead of working with a theory formula by formula, we can look for a uniform bound for all formulas:
\begin{Definition} \label{vc_fn_def}
  For a given NIP structure $\MM$, define the \defn{VC-density function}
  \begin{align*}
    \vc^\MM(n) &= \sup \{\vc^*(\phi(x, y)) \mid \phi \in \LL(M), |x| = n\} \\
             &= \sup \{\vc(\phi(x, y)) \mid \phi \in \LL(M), |y| = n\} \in \R^{\geq 0} \cup \curly{+\infty}.
  \end{align*}
\end{Definition}

As before this definition is elementary, so it only depends on the theory of $\MM$.
We omit the superscript $\MM$ if it is understood from the context.
One can easily check the following bounds:
\begin{Lemma} [Lemma 3.22 in \cite{density}] \label{vcone}
  We have $\vc(1) \geq 1$ and $\vc(n) \geq n\vc(1)$.  
\end{Lemma}

However, it is not known whether the second inequality can be strict or even just whether $\vc(1) < \infty$ implies $\vc(n) < \infty$.


Dp-rank is a common measure used in study of NIP theories, with dp-minimality being a special case.
Those notions originated in \cite{shelah_dp}, and further studied in \cite{dp_add}, showing, for example, that dp-rank is additive.
Here it is easiest for us to define dp-rank in terms of vc-density over indiscernible sequences.

\begin{Definition} \label{def_dp}
  \begin{itemize}
  \item
    Work in a $\aleph_1$-saturated first-order structure $M$.
    Fix a finite collection of formulas $\Phi(x, y)$ in the language of $M$.
    Suppose $A = (a_i)_{i \in \omega}$ is an indiscernible sequence with each $a_i \in M^{|x|}$.
    Let
    \begin{align*}
      \Ind(A, \Phi) = \{\phi(\bigcup_{i \in \N} a_i, b) \mid \phi \in \Phi, b \in M^{|y|}\} \subseteq \PP(M^{|x|}).     
    \end{align*}
    This gives rise to a set system $(M^{|x|}, \Ind(A, \Phi))$.
  \item Define
    \begin{align*}
      \vcind(\Phi) = \sup \curly{\vc(\Ind(A, \Phi)) \mid A = (a_i)_{i \in \N} \text{ is indiscernible}}.
    \end{align*}
  \item \defn{Dp-rank} of an $\aleph_1$-saturated structure $M$ is $\leq n$ if $\vcind(\phi) \leq n$ for all formulas $\phi$.
  \item \defn{Dp-rank} of a theory $T$ is $\leq n$ if dp-rank is $\leq n$ for any (all) $\aleph_1$-saturated model of $T$.
  \item A theory $T$ is said to have finite dp-rank if its dp-rank is $\leq n$ for some $n$.
  \item A theory $T$ is \defn{dp-minimal} if its dp-rank $\leq 1$.
  \end{itemize}
\end{Definition}

Refer to \cite{guingona} for the connection between to the classical definition of dp-rank and the definition given here.
