\chapter{Shelah-Spencer Graphs}



  We investigate vc-density in Shelah-Spencer graphs.
  We provide an upper bound on a formula-by-formula basis and show that there isn't a uniform lower bound,
  forcing the vc-function to be infinite.
  In addition we show that Shelah-Spencer graphs do not have a finite dp-rank,
  in particular they are not dp-minimal.


VC-density was studied in \cite{density} by Aschenbrenner, Dolich, Haskell, MacPherson, and Starchenko as a natural notion of dimension for NIP theories.
In a complete NIP theory $T$ we can define the vc-function

\begin{align*}
  \vc^T = \vc : \N \arr \R \cup \curly{\infty}
\end{align*}

where $\vc(n)$ measures the worst-case complexity of families of definable sets in an $n$-fold Cartesian power of the underlying set of a model of $T$
(see \ref{vc_fn_def} below for a precise definition of $\vc^T$).
We always have $\vc(n) \geq n$ for each $n$, and the simplest possible behavior is $\vc(n) = n$ for all $n$. Theories with the property that $\vc(1) = 1$ are known to be dp-minimal, i.e., having the smallest possible dp-rank (see Definition \ref{def_dp}). It is not known whether there can be a dp-minimal theory which doesn't satisfy $\vc(n)=n$
(see \cite{density}, diagram in section 5.3).


In this paper, we investigate vc-density of definable sets in Shelah-Spencer graphs.
First major model-theoretic breakthrough for these structures was made in \cite{spencer}.
In our description of Shelah-Spencer graphs we follow closely the treatment in \cite{laskowski}.
A Shelah-Spencer graph is a limit of random structures $G(n, n^{-\alpha})$ for an irrational $\alpha \in (0,1)$.
Here $G(n, n^{-\alpha})$ is a random graph on $n$ vertices with edge probability $n^{-\alpha}$.

Our first result is that in Shelah-Spencer graphs
\begin{align*}
  \vc(n) = \infty \text{ for each } n.
\end{align*}
We also show that Shelah-Spencer graphs don't have a finite dp-rank, which in particular implies that they are not dp-minimal.
Our second result provides an upper bound on the vc-density of a given formula $\phi(x, y)$:
\begin{align*}
  \vc(\phi) \leq D(\phi)
\end{align*}
where $D(\phi)$ is an expression involving $|y|$ and number of vertices and edges defined by $\phi$.

Section 1 introduces basic facts about VC-dimension and vc-density.
More can be found in \cite{density}.
Section 2 summarizes notation and basic facts concerning Shelah-Spencer graphs.
We direct the reader to \cite{laskowski} for a more in-depth treatment.
In section 3 we introduce key lemmas that will be useful in our proofs.
Section 4 computes a lower bound for vc-density to demonstrate that $\vc(n) = \infty$.
Here we also do computations involving dp-rank.
Section 5 computes an upper bound for vc-density on a formula-by-formula basis.



%%%%%%%%%%%%%%%%%%%%%%%%%%%%%%%%%%%%%%%%%%%%%%%%%%%%%%%%%%%%%%%%%%%%%%%%%%%%%%%%%%%%%%%%%%%%%%%%%%%%%%%%%%%%%%%%% 
\section{Graph Combinatorics}

Throughout this paper $A, B, C, M$ will denote finite graphs, and $\DB$ will be used to denote potentially infinite graphs.
For a graph $\A$ the set of its vertices is denoted by $v(\A)$, and the set of its edges by $e(\A)$.
Number of vertices of $\A$ will be denoted as $|\A|$.
Subgraph always means induced subgraph and $A \subseteq B$ means that $A$ is a subgraph of $B$.
For two subgraphs $\A, \B$ of a larger graph, the union $\A \cup \B$ denotes the graph induced by $v(\A) \cup v(\B)$.
Similarly, $A - B$ means a subgraph of $A$ induced by the vertices of $v(A) - v(B)$.
For $A \subseteq B \subseteq D$ and $A \subseteq C \subseteq D$,
graphs $B,C$ are said to be \defn{disjoint over $A$} if $v(B) - v(A)$ is disjoint from $v(C) - v(A)$
and there are no edges from $v(B) - v(A)$ to $v(C) - v(A)$ in $D$.

For the remainder of the paper fix $\alpha \in (0,1)$, irrational.
\begin{Definition} \ 
  \begin{itemize}
  \item For a graph $\A$ let $\dim(\A) = |\A| - \alpha |e(\A)|$.
  \item For $\A,\B$ with $\A \subseteq \B$ define $\dim(\B/\A) = \dim(\B) - \dim(\A)$.
  \item We say that $\A \leq \B$ if $\A \subseteq \B$ and $\dim(\A'/A) > 0$ for all $\A \subsetneq \A' \subseteq \B$.
  \item Define $\A$ to be \defn{positive} if for all $\A' \subseteq \A$ we have $\dim(\A') \geq 0$.
  \item We work in theory $S_\alpha$ in the language of graphs axiomatized by:
    \begin{itemize}
    \item Every finite substructure is positive.
    \item Given a model $\GG$ and graphs $\A \leq \B$, every embedding $f : \A \arr \GG$ extends to an embedding $g: \B \arr \GG$.
    \end{itemize}
    (Here an embedding maps edges to edges and nonedges to nonedges.)
    This theory is complete and stable (see 5.7 and 7.1 in \cite{laskowski}).
    From now on fix an ambient model $\GG \models S_\alpha$.
    This will be the only infinite graph we work with.
  \item For $\A, \B$ positive, $(\A, \B)$ is called a \defn{minimal pair} if
    $\A \subseteq \B$, $\dim(\B/\A) < 0$ but $\dim(\A'/\A) \geq 0$ for all proper $\A \subseteq \A' \subsetneq \B$.
    We call $B$ a \defn{minimal extension} of $A$.
    The dimension of a minimal pair is defined as $|\dim(B/A)|$.
  \item A sequence $\agl{M_i}_{0 \leq i \leq n}$ is called a \defn{minimal chain} if $(M_i, M_{i+1})$ is a minimal pair for all $0 \leq i < n$.
  \item For a graph $\A$ with the tuple of vertices $x$ let $\diag_\A(x)$ be the atomic diagram of $\A$,
    i.e. the first-order formula recording whether there is an edge between every pair of vertices.
  \item Given $\A \subseteq \B$ let 
    \begin{align*}
      \phi_{\A,\B}(x) = \diag_\A(x) \wedge \exists z \; \diag_\B(x, z).
    \end{align*}
    Any graph isomorphic to $\B$ is called a \defn{witness} of $\phi_{A,B}$.
  \item A formula $\phi_{A,B}$ is called a \defn{basic formula}
    if there is a minimal chain $\agl{M_i}_{0 \leq i \leq n}$
    such that $A = M_0$ and $B = M_n$.
  \end{itemize}
\end{Definition}

\begin{Theorem} [Quantifier elimination, 5.6 in \cite{laskowski}]
  In theory $S_\alpha$ every formula is equivalent to a boolean combination of basic formulas.
\end{Theorem}

\begin{Definition}
  A graph $S \subseteq \DB$ is called \defn{$N$-strong} if for any $S \subseteq T \subseteq D$ with $|T| - |S| \leq N$ we have $S \leq T$.
\end{Definition}

%%%%%%%%%%%%%%%%%%%%%%%%%%%%%%%%%%%%%%%%%%%%%%%%%%%%%%%%%%%%%%%%%%%%%%%%%%%%%%%%%%%%%%%%%%%%%%%%%%%%%%%%%%%%%%%%% 
\section{Basic Definitions and Lemmas}

\begin{Definition} \label{def_basic}
  Suppose $\phi(x, y)$ is a basic formula.
  Define $\X$ to be the graph on vertices $x$ with edges defined by $\phi$.
  Similarly define $\Y$.
  Note that $\X$, $\Y$ are positive.
  Additionally, let $\Y'$ be a subgraph of $\Y$ induced by vertices of $\Y$ that are connected to $W - (X \cup Y)$, where $W$ is a witness of $\phi$.
\end{Definition}

\begin{Definition} \label{def_e}
  Suppose $A, B$ are subgraphs of $\D$ such that $v(A), v(B)$ are disjoint.
  Then define $\E(A, B)$ to be the number of edges between the vertices in $v(A)$ and the vertices in $v(B)$.
\end{Definition}

We will require the following lemmas from \cite{laskowski}:

\begin{Lemma} \label{diamond} [See 2.3 in \cite{laskowski}]
  Let $A, B \subseteq \DB$.
  Then
  \begin{align*}
    \dim(A \cup B / A) \leq \dim(\B / A \cap B).
  \end{align*}
  Moreover, 
  \begin{align*}
    \dim(A \cup B / A) = \dim(\B / A \cap B) - \alpha E,
  \end{align*}
  % where $E$ is the number of edges connecting the vertices of $A \cup B - A$ to the vertices of $A - A \cap B$.
  where $E$ is the number of edges connecting the vertices of $B - A$ to the vertices of $A - B$.
\end{Lemma}

\begin{Lemma} \label{las_min} [See 4.1 in \cite{laskowski}]
  Suppose $A$ is a positive graph, with at least $1/\alpha + 2$ vertices.
  Then for any $\epsilon > 0$ there exists a graph $B$ such that $(A, B)$ is a minimal pair with dimension $\leq \epsilon$.
  Moreover, every vertex in $A$ is connected to a vertex in $B - A$.
\end{Lemma}

\begin{Lemma} \label{las_str} [See 4.4 in \cite{laskowski}]
  Suppose $A$ is a positive graph, and $\G$ a model of $S_\alpha$.
  Then for any integer $S$ there exists an embedding $f \colon A \arr \G$ such that $f(A)$ is $S$-strong in $\G$.
\end{Lemma}
    
\begin{Lemma} \label{las_closure} [See 3.8 in \cite{laskowski}]
  For all $S > 0$ there exists $M = M(S, \alpha) \in \N$ with the following property.
  Suppose $A \subseteq \G$ where $\G$ is a model of $S_\alpha$.
  Then there exists $B$ with $A \subseteq B \subseteq \G$ such that $B$ is $S$-strong in $\GG$ and $|B| \leq M|A|$.
\end{Lemma}

We conclude this section by stating a couple of technical lemmas that will be useful in our proofs later.

\begin{Lemma} \label{minimal_over_set}
  Work in an ambient graph $\DB$.
  Suppose we have a set $B$ and a minimal pair $(A, M)$ with $A \subseteq B$ and $\dim(M/A) = -\epsilon$.
  Then either $M \subseteq B$ or $\dim(M \cup B/B) < -\epsilon$.
\end{Lemma}

\begin{proof}
  By Lemma \ref{diamond}
  \begin{align*}
    \dim(M \cup B/B) \leq \dim(M / M \cap B),
  \end{align*}
  and as $A \subseteq M \cap B \subseteq M$
  \begin{align*}
    \dim (M/A) = \dim(M / M \cap B) + \dim(M \cap B / A).
  \end{align*}
  In addition we are given $\dim (M/A) = -\epsilon$.
  If $M \not\subseteq B$ then $A \subseteq M \cap B \subsetneq M$ and by minimality $\dim(M \cap B / A) > 0$.
  Combining the inequalities above we obtain the desired result:
  \begin{align*}
    \dim(M \cup B/B) \leq \dim(M / M \cap B) = \dim (M/A) - \dim(M \cap B / A) < -\epsilon.
  \end{align*}
\end{proof}

\begin{Lemma}	\label{chain_lemma}
  Work in an ambient graph $\DB$.
  Suppose we have a set $B$ and a minimal chain  $\agl{M_i}_{0 \leq i \leq n}$ with dimensions
  \begin{align*}
    \dim(M_{i+1}/M_i) = -\epsilon_i
  \end{align*}
  and $M_0 \subseteq B$.
  Let $\epsilon = \min_{0 \leq i \leq n} \epsilon_i$.
  Then either $M_n \subseteq B$ or $\dim((M_n \cup B)/B) < -\epsilon$.
\end{Lemma}

\begin{proof}
  Let $\bar M_i = M_i \cup B$. Then:
  \begin{align*}
    \dim(\bar M_n/B) = \dim(\bar M_n/\bar M_{n-1}) + \ldots + \dim(\bar M_2/\bar M_1) + \dim(\bar M_1/B).
  \end{align*}
  Either $M_n \subseteq B$ or at least one of the summands above is nonzero.
  Apply previous lemma.
\end{proof}

\begin{Lemma} \label{minimal_subset}
  Suppose we have a minimal pair $(A, M)$ with dimension $\epsilon$.
  Suppose we have some $B \subseteq M$.
  Then $\dim B / (A \cap B) \geq -\epsilon$.
  Moreover if $B \cup A \neq M$ then $\dim B / (A \cap B) \geq 0$.
\end{Lemma}

\begin{proof}
  We have $\dim (B \cup A / A) \leq \dim B / (A \cap B)$ by Lemma \ref{diamond}.
  As $A \subseteq B \cup A \subseteq M$ we have $\dim (B \cup A / A) \geq -\epsilon$ by minimality.
  Moreover, minimality implies that it is positive if $B \cup A \neq M$.
\end{proof}

\begin{Lemma} \label{chain_intersect}
  Suppose we have a minimal chain  $\agl{M_i}_{0 \leq i \leq n}$ with dimensions
  \begin{align*}
    \dim(M_{i+1}/M_i) = -\epsilon_i.
  \end{align*}
  Let $\epsilon$ be the sum of all $\epsilon_i$.
  Suppose we have a graph $B$ with $B \subseteq M_n$.
  Then $\dim B / (M_0 \cap B) \geq -\epsilon$.
\end{Lemma}

\begin{proof}
  Let $B_i = B \cap M_i$.
  We have $\dim B_{i+1}/B_i \geq \dim M_{i+1}/M_i$ by the previous lemma.
  Thus
  \begin{align*}
    \dim B / (M_0 \cap B) = \dim B_n / B_0 = \sum \dim B_{i+1}/B_i \geq -\epsilon.
  \end{align*}
\end{proof}

%%%%%%%%%%%%%%%%%%%%%%%%%%%%%%%%%%%%%%%%%%%%%%%%%%%%%%%%%%%%%%%%%%%%%%%%%%%%%%%%%%%%%%%%%%%%%%%%%%%%%%%%%%%%%%%%% 
\section{Lower bound}

In this section we restrict our attention to the following family of basic formulas $\phi(x,y)$:
\begin{itemize}
%\item Graphs defined by $x,y$ are $\X, \Y$.
\item All formulas have $\Y' = \Y$ (see Definition \ref{def_basic}).
\item All formulas define no edges between $X$ and $Y$.
\item Minimal chain of $\phi(x,y)$ consists of one step, that is we only have one minimal extension as opposed to a chain of minimal extensions.
\item The dimension of that minimal extension is smaller than $\alpha$.
\end{itemize}

We obtain a lower bound for the formulas that are boolean combinations of basic formulas written in the disjunctive-conjunctive form.
First, define $\epsilon_L(\phi)$.

\begin{Definition} 
  For a basic formula $\phi = \phi_{\agl{M_i}_{0 \leq i \leq n}}(x, y)$ let
  \begin{itemize}
  \item $\epsilon_i(\phi) = -\dim \paren{M_i/M_{i-1}}$.
  \item $\epsilon_L(\phi) = \sum_1^{n} \epsilon_i(\phi)$.
  \end{itemize}
\end{Definition}

\begin{Definition}[Negation]
  If $\phi$ is a basic formula, then define
  \begin{align*}
    \epsilon_L(\neg \phi) &= \epsilon_L(\phi).
  \end{align*}
\end{Definition}

\begin{Definition}[Conjunction]
  Take a collection of formulas $\phi_i(x, y)$ where each $\phi_i$ is a positive or a negative basic formula.
  If both positive and negative formulas are present then $\epsilon_L(\phi) = \infty$.
  We don't have a lower bound for that case.
  If different formulas define $\X$ or $\Y$ differently then $\epsilon_L(\phi) = \infty$.
  In the case of conflicting definitions the formula would have no realizations.
  Otherwise let
  \begin{align*}
    \epsilon_L\paren{\bigwedge \phi_i} &= \sum \epsilon_L(\phi_i).
  \end{align*}
\end{Definition}

\begin{Definition} [Disjunction]
  Take a collection of formulas $\psi_i$ where each instance is a conjunction as above all agreing on $\X$ and $\Y$.
  Then
  \begin{align*}
    \epsilon_L\paren{\bigvee \psi_i} &= \min \epsilon_L(\psi_i).
  \end{align*}
\end{Definition}
\begin{Theorem} \label{main_lower}
  For a formula $\psi$ as above we have
  \begin{align*}
    \vc \psi \geq \floor{\frac{Y(\psi)}{\epsilon_L(\psi)}},
  \end{align*}
  where $Y(\psi)$ is $\dim(Y)$ (as all basic componenets agree on $\Y$).
\end{Theorem}
\begin{proof}
  First, work with a formula that is a conjunction of positive basic formulas $\psi = \bigwedge_{i \in I} \phi_i$.
  Then as we have defined above
  \begin{align*}
    \epsilon_L(\psi) = \sum_{i \in I} \epsilon_L(\phi_i).
  \end{align*}
  If $W_i$ is a witness of $\phi_i$, let $S_i = |W_i|$.
  Let $n_1$ be the largest natural number such that
  \begin{align*}
    n_1 \epsilon_L(\psi) < Y(\psi).
  \end{align*}
  Let $\epsilon'$ be the smallest value among $\epsilon_L(\phi_i)$.
  Suppose it corresponds to the formula $\phi'$.
  Let $n_2$ be the largest natural number such that
  \begin{align*}
    n_1 \epsilon_L(\psi) + n_2 \epsilon' < Y(\psi).
  \end{align*}

  Fix some $N > n_1 + n_2$.
  Let 
  \begin{align*}
    J = \curly{0 \leq j < N} \subseteq \N.
  \end{align*}
  Let $a_j$ be a graph isomorphic to $\X$ for each $j \in J$, pairwise disjoint.
  Let $A = \bigcup_{1 \leq j \leq N} a_j$.
  Let 
  \begin{align*}
    S = |Y| + (n_1 + n_2 + 1) \sum_{i \in I} S_i.
  \end{align*}

  By Lemma \ref{las_str} the graph $A$ can be embedded into $\GG$ as an $S$-strong graph. 
  Abusing notation, we identify $A$ with this embedding.
  Thus we have $A \subseteq \GG$, $S$-strong. 

  Let $J_1, J_2$ be disjoint subsets of $J$, of sizes $n_1, n_2$ respectively.
  Let $b$ be a graph isomorphic to $\Y$.
  For each $i \in I, j \in J_1$ let $W_{ij}$ be a witness of $\phi_i(a_j, b)$.
  (Note that then $(a_j \cup b, W_{ij})$ is a minimal pair.)
  For each $j \in J_1$ let $W_j$ be a union of $\curly{W_{ij}}_{i \in I}$ disjoint over $a_j \cup b$.
  For each $j \in J_2$ let $W_{j}$ be a witness of $\phi'(a_j, b)$.
  Let $W'$ be a union of $\curly{W_j}_{j \in J_1 \cup J_2}$ disjoint over $b$.
  Let $W$ be a union of $W'$ and $A$ disjoint over $\curly{a_j}_{j \in J_1 \cup J_2}$.
  \begin{Claim}
    We have $A \leq W$.
  \end{Claim}
  \begin{proof}
    Consider some $A \subsetneq B \subseteq W$.
    We need to show $\dim (B/A) > 0$.
    Let $\BA = A \cup b$.
    We have
    \begin{align*}
      \dim(B/A) = \dim(B/ B \cap \BA) + \dim(B \cap \BA / A).
    \end{align*}
    Let $B_{ij} = B \cap W_{ij}$.
    Let $B_{j} = B \cap W_{j}$.
    To unify indices, relabel all the graphs above as $\curly{B_k}_{k \in K}$ for some index set $K$.
    By the construction of $W$ we have
    \begin{align*}
      \dim(B/ B \cap \BA) = \sum_{k \in K} \dim(B_k/ B_k \cap \BA).
    \end{align*}
    Fix $k$.
    We have $B_k \subseteq W_k$, where $W_k$ is a minimal extension of $M^k_0 = a \cup b$ for some $a \in A$.
    Let $\epsilon_k$ be the dimension of this minimal extension.
    We have $\dim(B_k / B_k \cap \BA) = \dim(B_k / a \cup (B \cap b))$.

    Case 1: $B \cap b = b$.
    Then $M_0^k \subseteq B_k \subseteq W_k$ and
    \begin{align*}
      \dim(B_k / a \cup (B \cap b)) = \dim (B_k/M_0^k).
    \end{align*}
    By minimality of $(M_0^k, B_k)$ we have $\dim (B_k/M_0^k) \geq -\epsilon_k$.
    Thus
    \begin{align*}
      \dim(B/ B \cap \BA) \geq - \sum_{k \in K} \epsilon_k = -\paren{n_1 \epsilon_L(\psi) + n_2 \epsilon'}.
    \end{align*}
    In addition
    \begin{align*}
      \dim(B \cap \BA / A) = \dim (b) = Y(\psi).
    \end{align*}
    Combining the two, we get
    \begin{align*}
      \dim(B/A) \geq Y(\psi) - \paren{n_1 \epsilon_L(\psi) + n_2 \epsilon'},
    \end{align*}
    which is positive by the construction of $n_1, n_2$ as needed.
    
    Case 2: $B \cap b \subsetneq b$.
    \begin{Claim} We have $\dim(B_k / B_k \cap \BA) > 0$.
    \end{Claim}
    \begin{proof}
      Recall that $\dim(B_k / B_k \cap \BA) = \dim(B_k / a \cup (B \cap b))$.
      First, suppose that $B_k \cup M_0^k \neq W_k$.
      Then by Lemma \ref{minimal_subset} we get the required inequality.
      Thus we may assume that $B_k \cup M_0^k = W_k$.
      By Lemma \ref{diamond} we have
      \begin{align*}
        \dim(B_k \cup M_0^k / M_0^k) = \dim(B_k / B_k \cap M_0^k) - \alpha E,
      \end{align*}
      where $E$ is the number of edges connecting the vertices of
      $B_k - M_0^k = B_k \cup M_0^k - M_0^k$ to the vertices of $M_0^k - B_k = M_0^k - B_k \cap M_0^k$.
      Noting that $B_k \cup M_0^k = W_k$, $\dim{W_k / M_0^k} = -\epsilon_k$, and $B_k \cap M_0^k = a \cup (B \cap b)$
      we may rewrite the equality above as
      \begin{align*}
        \dim(B_k / a \cup (B \cap b)) = \alpha E - \epsilon,
      \end{align*}
      and $E$ is the number of edges connecting the vertices of $W_k - M_0^k$ to the vertices of $M_0^k - a \cup (B \cap b)$.
      As $\Y = \Y'$ and $B \cap b \subsetneq b$ we must have $E \geq 1$.
      But then as $\alpha > \epsilon$ we have $\dim(B_k / a \cup (B \cap b)) > 0$ as needed.
    \end{proof}
    Now, recall that
    \begin{align*}
      \dim(B/A) = \dim(B \cap \BA / A) + \sum_{k \in K} \dim(B_k/ B_k \cap \BA).
    \end{align*}
    By the claim above each of $\dim(B_k/ B_k \cap \BA) > 0$, thus
    \begin{align*}
      \dim(B/A) > \dim(B \cap \BA / A).
    \end{align*}
    In addition
    \begin{align*}
      \dim(B \cap \BA / A) = \dim (B \cap b) \geq 0,
    \end{align*}
    as $b$ is postive.
    Thus $\dim (B/A) > 0$ as needed.
  \end{proof}

  As $A \leq W$ and $A \subseteq \GG$, we can embed $W$ into $\GG$ over $A$.
  Abusing notation again, we identify $W$ with its embedding $A \leq W \subseteq \GG$.
  In particular, now we have $b \in \GG$.
  Also note that
  \begin{align*}
    \dim(W/A) &= Y(\psi) - \paren{n_1 \epsilon_L(\psi) + n_2 \epsilon'}, \\
    |W| - |A| &\leq |b| + (n_1 + n_2) \sum_{i \in I} S_i.
  \end{align*}

  \begin{Lemma} We have
    \begin{align*}
      \curly{a_j}_{j \in J_1} \subseteq \psi(A, b) \subseteq \curly{a_j}_{j \in J_1 \cup J_2}.
    \end{align*}
  \end{Lemma}
  \begin{proof}
    First inclusion $\curly{a_j}_{j \in J_1} \subseteq \psi(A, b)$ is immediate from the construction of $W$,
    as $W_{ij}$ witnesses that $\phi_i(a_j, b)$ holds.
    For the second inclusion, suppose that there is $a \in A - \curly{a_j}_{j \in J_1 \cup J_2}$ such that $\psi(a,b)$ holds.
    Let $W' \subseteq \GG$ be a witness of $\phi_1(a,b)$.
    First, note that the case $W' \subseteq W$ is impossible
    as there are no edges between $a$ and $W - a$, but there are edges between $a$ and $W' - a$.
    Thus assume $W' \not\subseteq W$.
    As $(a \cup b, W')$ is minimal, by Lemma \ref{minimal_over_set} we have $\dim (W' \cup W / W) < -\epsilon_1$.
    Therefore
    \begin{align*}
      \dim(W' \cup W / A) = \dim (W' \cup W / W) + \dim(W/A) < Y(\psi) - \paren{n_1 \epsilon_L(\psi) + n_2 \epsilon'} - \epsilon_1,
    \end{align*}
    which is negative by the construction of $n_1, n_2$.
    Thus $A \not\leq W \cup W'$, as then it would have a positive dimension.
    Additionally,
    \begin{align*}
      |W' \cup W| - |A| \leq |W' - W| + |W| - |A| \leq S_1 + |b| + (n_1 + n_2) \sum_{i \in I} S_i \leq S,
    \end{align*}
    but then this contradicts that $A$ is $S$-strong, as then we would have $A \leq W \cup W'$.
  \end{proof}

  In the construction of $W$ we have chosen indices $J_1, J_2$ arbitrarily.
  In particular, suppose we let $J_2$ to be the last $n_2$ indices of $J$ and
  $J_1$ an arbitrary $n_1$-element subset of the first $N - n_2$ elements of $J$.
  Each of those choices would then yield a different trace $\psi(A, b)$ by the lemma above.
  Thus $\psi(A, M^{|y|}) \geq {N - n_2 \choose n_1}$ and therefore $\vc(\psi) \geq n_1$.
  By the definition of $n_1$ we have $n_1 = \floor{\frac{Y(\psi)}{\epsilon_L(\psi)}}$, so this proves the theorem for $\psi$.
 
  Now consider a formula which is a conjunction consisting of negative basic formulas $\psi = \bigwedge_{i \in I} \neg \phi_i$.
  Let $\bar \psi = \bigwedge_{i \in I} \phi_i$.
  Do the construction above for $\bar \psi$ and suppose its trace is $X \subseteq A$ for some $b$.
  Then over $b$ the same construction gives trace $(A - X)$ for $\psi$. Thus we get as many traces as above, and the same bound.
  
  Finally consider a formula which is a disjunction of formulas considered above $\theta = \bigvee_{k \in K} \psi_k$.
  Choose the one with the smallest $\epsilon_L$, say $\psi_k$, and repeat the construction above for $\psi_k$.
  Any trace we obtain is automatically a trace for $\theta$, and thus we get as many traces as above, and the same bound.
\end{proof}

\begin{Corollary}
  VC-function is infinite in Shelah-Spencer random graphs:
  \begin{align*}
    \vc(n) = \infty.
  \end{align*}
\end{Corollary}

\begin{proof}
  Let $A$ be a graph consisting of $1/\alpha + 2 + n$ disconnected vertices.
  Fix $\epsilon > 0$.
  By Lemma \ref{las_min}, there exists $B$ such that $(A, B)$ is minimal with dimension $\leq \epsilon$.
  Consider a basic formula $\psi_{A, B}(x, y)$ where $|x| = 1/\alpha + 2$ and $|y| = n$.
  Then by the theorem above $\vc(n) \geq \vc (\psi_{A,B}) \geq \frac{n}{\epsilon}$.
  As $\epsilon$ was arbitrary, this number can be made arbitrarily large, giving $vc(n) = \infty$ as needed.
\end{proof}

\begin{Corollary}
  Shelah-Spencer random graphs don't have finite dp-rank.
  In particular they are not dp-minimal.
\end{Corollary}

\begin{proof}
  We would like to modify the proof of Theorem \ref{main_lower} such that $A$ is indiscernible.
  Note that in the proof we can construct sets $A = \curly{a_j}_{j \in J}$ of arbitrary length.
  Moreover for every finite $J' \subseteq J$, the set $A = \curly{a_j}_{j \in J'}$ is still $S$-strong.
  Thus we can find an infinite set $A = \curly{a_j}_{j \in \N}$ indiscernible and $S$-strong.
  Repeating the construction of the corollary above,
  we can obtain a formula with an arbitrarily large vc-density over the indiscernible sequence $A$.
\end{proof}



%%%%%%%%%%%%%%%%%%%%%%%%%%%%%%%%%%%%%%%%%%%%%%%%%%%%%%%%%%%%%%%%%%%%%%%%%%%%%%%%%%%%%%%%%%%%%%%%%%%%%%%%%%%%%%%%% 
\section{Upper bound}

Consider a basic formula $\phi(x,y)$ with a minimal chain  $\agl{M_i}_{0 \leq i \leq n_{\phi}}$ with dimensions  $\dim(M_{i+1}/M_i) = -\epsilon_i$.
Define
\begin{align*}
  \epsilon(\phi) &= \min \curly{\epsilon_i}_{0 \leq i \leq n_\phi}\\
  K(\phi) &= |M_{n_\phi}|.
\end{align*}
Now consider a finite collection of basic formulas
\begin{align*}
  \Phi = \Phi(\vec x, \vec y) = \curly{\phi_i(\vec x, \vec y)}_{i\in I}.
\end{align*}
Define
\begin{align*}
  \epsilon(\Phi) &= \min \curly{\epsilon(\phi_i)}_{i \in I} \cup \curly{\alpha}, \\
  K(\Phi) &= \max \curly{K(\phi_i)}_{i \in I},\\
  D_1(\Phi) &= \frac{K(\Phi)}{\epsilon(\Phi)}, \\
  D(\Phi) &= |y| D_1(\Phi).\\
\end{align*}
We claim that
\begin{Theorem} \label{upper}
  If $\phi$ is a boolean combination of formulas from $\Phi$, then $\vc(\phi) \leq D(\Phi)$.
\end{Theorem}
Let
\begin{align*}
  S = \left\lceil{\paren{\frac{|y|}{\epsilon(\phi)} + 1} K(\phi)}\right\rceil.
\end{align*}
Suppose we have a finite parameter set $A_0 \subseteq \GG^{|x|}$ with $|A_0| = N_0$.
We would like to bound $\phi(A_0, \GGY)$.
Let $A_1 \subseteq \GG$ consist of the components of the elements of $A_0$.
Then $|A_1| \leq |x| N_0$.
Using Lemma \ref{las_closure} let $A$ be a graph $A_0 \subseteq A \subseteq \GG$, $S$-strong in $\GG$.
Let $N = |A|$.
We have $N \leq |x| N_0 M$ (where $M$ is the constant from the Lemma \ref{las_closure}).
As $A_0 \subseteq \AX$ we have
\begin{align*}
  \abs{\phi(A_0, \GGY)} \leq \abs{\phi(\AX, \GGY)}.
\end{align*}
Therefore it suffices to bound $\abs{\phi(\AX, \GGY)}$.

\begin{Definition}
  For $A \subseteq \GG^{|x|}, B \subseteq \GG^{|y|}$ define
  \begin{align*}
    \Phi(A, B) = \curly{(a, i) \in A \times I \mid \GG \models \phi_i(a, b)} \subseteq A \times I
  \end{align*}  
\end{Definition}

\begin{Lemma}
  For $A \subseteq \GG^{|x|}, B \subseteq \GG^{|y|}$
  if $\phi$ is a boolean combination of formulas from $\Phi$ then
  \begin{align*}
    \abs{\phi(A, B)} \leq \abs{\Phi(A, B)}
  \end{align*}  
\end{Lemma}
\begin{proof}
  Clear, as for all $a \in A, b \in B$ the set $\Phi(a,b)$ determines the truth value of $\phi(a,b)$.
\end{proof}

Thus it suffices to bound  $\abs{\Phi(\AX, \GGY)}$.

\begin{Definition} \;
  \begin{itemize}
  \item For all $i \in I, a \in \AX, b \in \GGY$ if $\phi_i(a, b)$ holds fix $W^i_{a,b} \subseteq \GG$, a witness of this formula.
  \item For $b \in \GGY$ let 
    \begin{align*}
      W_b = \bigcup \curly{W^i_{a,b} \mid a \in \AX, i \in I, \GG \models \phi_i(a,b)}.
    \end{align*}
  \item For sets $C, B \subset \GG$ define the \defn{boundary} of $C$ over $B$
    \begin{align*}
      \partial(C, B) = \curly{b \in B \mid \E(b, C - B) \neq \emptyset}
    \end{align*}
    (see Definition \ref{def_e} for $\E$).
  \item For $b \in \GGY$ let $\partial_b \subseteq A$ be the boundary $\partial(W_b, A)$.
  \item For $b \in \GGY$ let $\bar W_b = (W_b - A) \cup \ppp_b$.
  \item For $b_1, b_2 \in \GGY$ we say that $b_1 \sim b_2$ if $\ppp_{b_1} = \ppp_{b_2}$,
    $b_1 \cap A = b_2 \cap A$,
    and there exists a graph isomorphism from $\bar W_{b_1} \cup b_1$ to $\bar W_{b_2} \cup b_2$ that fixes $\ppp_{b_1}$ and
    maps $b_1$ to $b_2$.
    One easily checks that this defines an equivalence relation.
  \item For $b \in \GGY$ define $\II_b$ to be the $\sim$-equivalence class of $b$.
  \end{itemize}
\end{Definition}

\begin{Lemma} \label {bound_trace}
  For $b_1, b_2 \in \GGY$ if $b_1 \sim b_2$ then $\Phi(\AX, b_1) = \Phi(\AX, b_2)$.
\end{Lemma}

\begin{proof}
  Fix a graph isomorphism $\bar f \colon \bar W_{b_1} \cup b_1 \arr \bar W_{b_2} \cup b_2$.
  Extend it to an isomorphism $f \colon W_{b_1} \cup A \arr W_{b_2} \cup A$ by being an identity map on the new vertices.
  Suppose $\GG \models \phi_i(a, b_1)$ for some $a \in \AX$.
  Then $f(W^i_{a, b_1})$ is a witness for  $\phi_i(a, b_2)$ (though not necessarily equal to $W^i_{a, b_2}$)
  and thus $\GG \models \phi_i(a, b_2)$.
  As $a$ was arbitrary, this proves the equality of the traces.
\end{proof}

Thus to bound the number of traces it is sufficient to bound the number of possibilities for $\II_b$.

\begin{Theorem} \label{main_bound}
  Suppose we have $b \in \GGY$.
  Let $Y = \abs{b - A}$.
  Then
  \begin{align*}
    |\partial_b| &\leq Y D_1(\phi) \\ 
    |\bar W_b| &\leq 3 Y D_1(\phi)
  \end{align*}
\end{Theorem}

From this theorem we get the desired result:
\begin{Corollary} (Theorem \ref{upper})
  If $\phi$ is a boolean combination of formulas from $\Phi$, then $\vc(\phi) \leq D(\Phi)$.
\end{Corollary}

\begin{proof}
  We count possible equivalence classes of $\sim$.
  This essentially boils down to bounding possibilities for $\partial_b$, $b \cap A$, and number of isomorphism classes of $W_b$.
  Fix $b \in \GGY$ and let $Y = \abs{b - A}$.
  Let $D = Y D_1(\Phi)$.
  By the first part of Theorem \ref{main_bound} there are $N \choose D$ choices for boundary $\partial_b$.
  By the second part of Theorem \ref{main_bound} there are at most $3D$ vertices in $\bar W_b$.
  Thus to determine the graph $\bar W_b$ we need to choose how many vertices it has and then decide where edges go.
  This amounts to at most $3D 2^{(3D)^2}$ choices.
  Additionally there are $N \choose |y| - Y$ choices for $b \cap A$.
  In total this gives us at most
  \begin{align*}
    &{N \choose D} \cdot {N \choose |y| - Y} \cdot 3D 2^{(3D)^2} = O(N^{D + |y| - Y}) = \\
    &= O(N^{Y D_1(\Phi) + |y| - Y}) = O(N^{|y| D_1(\Phi)}) = O(N^{D(\Phi)})
  \end{align*}
  choices (second to last inequality uses $D_1(\Phi) \geq 1$).
  By Lemma \ref{bound_trace} we have $\abs{\Phi(\AX, \GGY)} = O(N^{D(\Phi)})$.
  Recall that 
  \begin{align*}
    \abs{\phi(A_0, \GGY)} \leq \abs{\Phi(\AX, \GGY)}.    
  \end{align*}
  Therefore we have
  \begin{align*}
    \abs{\phi(A_0, \GGY)} &= O(N^{D(\Phi)}) = O(\paren{|x| N_0 M}^{D(\Phi)}) = \\
    &= O(\paren{|x| M}^{D(\Phi)} N_0^{D(\Phi)}) = O(N_0^{D(\Phi)}).
  \end{align*}
  As $A_0$ was arbitrary, this shows that $\vc(\phi) \leq D(\Phi)$ as needed.
  (Note that throughout this proof we are using the fact that quantities $D_1(\Phi), D(\Phi), M$ are completely determined by $\Phi$,
  thus independent from $A_0$.)
\end{proof}

\begin{proof} \textit{(of Theorem \ref{main_bound})}

  The graph $W_b$ is a union of witnesses of the from $W_{a,b}$ for some $a \in \AX, b \in \GGY$.
  Enumerate all of them as $\curly{W_j}_{1 \leq j \leq J}$.
  Define $M_j = \bigcup_1^j W_{j'}$ for $1 \leq j \leq J$ and let $M_0 = b$.
  Let $\BA = A \cup b$.
  \begin{Definition}
    For $0 \leq j \leq J$ define:
    \begin{itemize}
    \item Let $v_j = 1$ if new vertices are added to $M_j$ outside of $\BA$, that is if
      $M_j - \BA \neq M_{j-1} - \B$,
      and let it be $0$ otherwise.
    \item Let %$E_j = \curly{a \in A - W_j \mid  \E(a, M_j - A) \neq \emptyset}$.
      $E_j = \partial(A - W_j, M_j - A)$.
    \item Let
      \begin{align*}
        m_j = \sum_{j' = 0}^j (v_j + |E_j|).
      \end{align*}
    \end{itemize}
    (Here assume $M_{-1} = \emptyset$.)
  \end{Definition}

  \begin{Lemma} \label{ubd_lemma}
    For $0 \leq j \leq J$ we have
    \begin{align*}
      |\partial(M_j, A)| \leq |E_0| + m_j K(\Phi) 
    \end{align*}
  \end{Lemma}

  \begin{proof} %\textit{(of Lemma \ref{ub_lemma})}
    Proceed by induction. The base case $j = 0$ is clear.
    For an induction step suppose that
    \begin{align*}
      |\partial(M_{j-1}, A)| \leq m_{j-1}  K(\Phi)
    \end{align*}
    holds.
    Let
    \begin{align*}
      \delta_1 &= \partial(M_j, A) - \partial(M_{j-1}, A) = \\
               &=\curly{a \in A \mid  \E(a, M_j - A) \neq \emptyset \text{ and } \E(a, M_{j-1} - A) = \emptyset}.
    \end{align*}
    If $M_j - A = M_{j-1} - A$ then $\delta_1 = \emptyset$ and we are done as $m_j$ is increasing.
    Suppose not.
    We have $|\delta_1| = |\delta_1 \cap W_j| + |\delta_1 - W_j|$, and
    \begin{align*}
      \delta_1 - W_j = \curly{a \in A - W_j \mid \E(a, M_j - A) \neq \emptyset \text{ and } \E(a, M_{j-1} - A) = \emptyset}.
    \end{align*}
    But then it's clear that $\delta_1 - W_j \subseteq E_j$ as
    \begin{align*}
      &W_j - M_{j-1} - A \subseteq M_j - A, \\
      &(W_j - M_{j-1} - A) \cap (M_{j-1} - A) = \emptyset.
    \end{align*}
    As $b \in M_{j-1}$ and $M_j - A \neq M_{j-1} - A$, then $M_j - \BA \neq M_{j-1} - \BA$, and thus $v_j = 1$. 
    Therefore we have
    \begin{align*}
      |\delta_1| &= |\delta_1 \cap W_j| + |\delta_1 - W_j| \leq |W_j| + |E_j| \leq \\
      &\leq K(\Phi) + |E_j|
      \leq (v_j + |E_j|) K(\Phi)  \leq (m_j - m_{j-1}) K(\Phi),
    \end{align*}
    as needed.
  \end{proof}

  \begin{Lemma} \label{ub_lemma}
    For $0 \leq j \leq J$ we have
    \begin{align*}
      |M_j - \BA| \leq \sum_{j'=0}^j v_{j'} K(\Phi)
    \end{align*}
  \end{Lemma}

  \begin{proof} %\textit{(of Lemma \ref{ub_lemma})}
    Proceed by induction. The base case $j = 0$ is clear.
    For an induction step suppose that
    \begin{align*}
      |M_{j-1} - \BA| \leq \sum_{j'=0}^{j-1} v_{j'} K(\Phi)
    \end{align*}
    holds.
    If $M_j - \BA = M_{j-1} - \BA$ then the inequality is immediate as $v_j \geq 0$.
    Therefore assume this is not the case, so $v_j = 1$ and $|M_j - A| - |M_{j-1} - A| \leq |W_j| \leq v_j K(\Phi)$, and so we get the required inequality.
    \begin{align*}
    \end{align*}
  \end{proof}
  
  \begin{Lemma} \label{ubdim_lemma}
    For $0 \leq j \leq J$ we have
    \begin{align*}
      \dim(M_j \cup \BA / \BA) \leq -m_j  \epsilon(\Phi),
    \end{align*}
  \end{Lemma}
  \begin{proof}
    Proceed by induction. Base case $j = 0$ is clear.    
    For an induction step suppose that
    \begin{align*}
      \dim(M_{j-1} \cup \BA / \BA) \leq  - m_{j-1}  \epsilon(\Phi)
    \end{align*}
    holds.
    We have
    \begin{align*}
      \dim(M_j \cup \BA / \BA) &= \dim(M_j \cup \BA / M_{j-1} \cup \BA) + \dim(M_{j-1} \cup \BA / \BA) \leq \\
      &\leq \dim(M_j \cup \BA / M_{j-1} \cup \BA) - m_{j-1}  \epsilon(\Phi).
    \end{align*}
    Let $\BM = M_{j-1} \cup \BA$.
    By Lemma \ref{diamond}
    \begin{align*}
      \dim(M_j \cup \BA / M_{j-1} \cup \BA) = \dim(W_j \cup \BM / \BM) = \dim(W_j / W_j \cap \BM) - e \alpha
    \end{align*}
    where $e$ is the number of edges connecting the vertices of $\BM - W_j$ to the vertices of $W_j - \BM$.
    Recall that       $E_j = \partial(A - W_j, M_j - A)$.
    We have $A - W_j \subseteq \BM - W_j$ (as $A \subseteq \BM$) and $W_j - M_{j-1} - A = W_j - \BM$ (as for $j > 1$, we have $b \subseteq M_{j-1}$).
    Thus $|E_j| \leq e$, and we get 
    \begin{align*}
      \dim(M_j \cup \BA / M_{j-1} \cup \BA) \leq \dim(W_j / W_j \cap \BM) - |E_j| \alpha.
    \end{align*}
    If $W_j \subseteq \BM$ then $\dim(W_j / W_j \cap \BM) = 0$.
    If not, then by Lemma \ref{chain_lemma} we have $\dim(W_j / W_j \cap \BM) \leq - \epsilon(\Phi)$.
    Either way, we have $\dim(W_j / W_j \cap \BM) \leq - v_j \epsilon(\Phi)$.
    Using this and the fact that $\epsilon(\Phi) \leq \alpha$, we obtain
    \begin{align*}
      \dim(M_j \cup \BA / M_{j-1} \cup \BA) \leq - v_j \epsilon(\Phi) - |E_j| \epsilon(\Phi) = -(m_j - m_{j-1})\epsilon(\Phi).
    \end{align*}
    Finally,
    \begin{align*}
      \dim(M_j \cup \BA / \BA) &\leq \dim(M_j \cup \BA / M_{j-1} \cup \BA) - m_{j-1}  \epsilon(\Phi) \leq \\
      &\leq  -(m_j - m_{j-1})\epsilon(\Phi) - m_{j-1}  \epsilon(\Phi) =  - m_j  \epsilon(\Phi),
    \end{align*}
    as needed.
  \end{proof}
  \textit{(Proof of Theorem \ref{main_bound} continued)}
  For any $0 \leq j \leq J$ we have
  \begin{align*}
    \dim(M_j \cup A / A) &= \dim(\BA / A) + \dim(M_j \cup \BA / \BA) \\
    &\leq Y - |E_0|\alpha + \dim(M_j \cup \BA / \BA).
  \end{align*}
  Lemma \ref{ubdim_lemma} gives us
  \begin{align*}
    \dim(M_j \cup \BA / \BA) \leq -m_j  \epsilon(\Phi).
  \end{align*}
  Thus
  \begin{align*}
    \dim(M_j \cup A / A) \leq Y - |E_0| \alpha - m_j  \epsilon(\Phi).
  \end{align*}
  Suppose $j$ is an index such that
  \begin{align*}
    &Y - |E_0| \alpha - m_j  \epsilon(\Phi) \geq 0, \\
    &Y - |E_0| \alpha - m_{j+1}  \epsilon(\Phi) < 0
  \end{align*}
  if one exists.
  Then 
  \begin{align*}
    m_j \leq \frac{Y - |E_0| \alpha}{\epsilon(\Phi)}.
  \end{align*}
  By Lemma \ref{ub_lemma} we have
  \begin{align*}
    \abs{M_{j+1} - A} &\leq \paren{\sum_{j'=1}^{j+1} v_{j'}} K(\Phi) \leq (m_j + 1) K(\Phi) \\
                     &\leq \paren{\frac{Y - |E_0| \alpha}{\epsilon(\Phi)} + 1} K(\Phi) \leq S.
  \end{align*}
  This is a contradiction, as $A$ is $S$-strong and $\dim(M_{j+1} \cup A / A)$ is negative.
  Thus $Y - |E_0| \alpha - m_j  \epsilon(\Phi) \geq 0$ for all $j \leq J$.
  In particular $Y - |E_0| \alpha - m_J  \epsilon(\Phi) \geq 0$, so $m_J \leq \frac{Y - |E_0| \alpha}{\epsilon(\Phi)}$.
  Noting that $M_J = W_b$, Lemma \ref{ubd_lemma} gives us 
  \begin{align*}
      |\ppp_b| = |\partial(W_b, A)| \leq |E_0| + m_J  K(\Phi) \leq |E_0| + K(\Phi) \frac{Y - |E_0| \alpha}{\epsilon(\Phi)}.
  \end{align*}
  As $K(\Phi) \geq 1$ and $\epsilon(\Phi) \geq \alpha$, we get
  \begin{align*}
      |\ppp_b| \leq K(\Phi) \frac{Y}{\epsilon(\Phi)} = Y D_1(\Phi).
  \end{align*}
  But this is precisely the first inequality we need to prove.
  For the second inequality, Lemma \ref{ub_lemma} gives us
  \begin{align*}
    \abs{W_b - \BA} &\leq Y + \paren{\sum_{j'=0}^J v_{j'}} K(\Phi) \leq Y + m_J K(\Phi) \leq \\
    &\leq Y + K(\Phi) \frac{Y}{\epsilon(\Phi)} \leq 2 Y D_1(\Phi).
  \end{align*}
  Thus we have
  \begin{align*}
      |\bar W_b| \leq \abs{W_b - A} + \abs{\ppp_b} \leq 3 Y D_1(\Phi),
  \end{align*}
  as needed.
  This ends the proof for Theorem \ref{main_bound}.
\end{proof}


%%%%%%%%%%%%%%%%%%%%%%%%%%%%%%%%%%%%%%%%%%%%%%%%%%%%%%%%%%%%%%%%%%%%%%%%%%%%%%%%%%%%%%%%%%%%%%%%%%%%%%%%%%%%%%%%% 
\section{Conclusion}

This paper computes upper and lower bounds for certain types of formulas in Shelah-Spencer graphs.
The bounds are not tight: in the best case scenario for a basic formula $\phi(x,y)$ defining a minimal extension of
dimension $\epsilon$ we have
\begin{align*}
  \frac{|y|}{\epsilon} \leq \vc(\phi) \leq K \frac{|y|}{\epsilon},
\end{align*}
where $K$ is the number of vertices in the minimal extension.
Thus there is a multiple of $K$ gap between lower and upper bounds.
It is this author's hope that a refinement of presented techniques can yield better estimates of the vc-density.
One potential direction towards this goal is to have a closer study on
how multiple minimal extensions can intersect without increasing overall dimension.

Note that this paper doesn't answer the question whether there can be exotic values for vc-density of individual formulas,
such as non-integer or irrational values.
A better bound can help address this question.

Another observation is that while $\vc(n) = \infty$ there seems to be a good structural behavior of the vc-density for individual formulas.
This perhaps suggests that the vc-function is not the best tool to describe behaviour of the definable sets in Shelah-Spencer graphs,
and some more refined measure might be required.
One potential way to do this is to separate the formulas based on values of $K(\phi), \epsilon(\phi)$.
Once those are bounded, vc-density seems to be well-behaved.
This author hopes to explore this further in his future work.
