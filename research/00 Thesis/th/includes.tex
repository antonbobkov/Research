\usepackage{amsrefs}
\usepackage{amsmath}
\usepackage{amsfonts}
\usepackage{amssymb}
\usepackage{amsthm}
%\usepackage{color}
\usepackage{xcolor}
%\usepackage{breqn} %For dmath
\usepackage{mathrsfs} %mathscr?
\usepackage{wasysym} %\eighthnote

\usepackage{enumitem} % For fancy enumerations...

\usepackage{hyperref}

\usepackage{tikz}
\usetikzlibrary{shapes,snakes,calc,arrows,intersections}
\usetikzlibrary{decorations.pathreplacing,shapes.geometric}
\usetikzlibrary{calc,positioning}

\DeclareSymbolFont{AMSb}{U}{msb}{m}{n}
\DeclareMathSymbol{\R}{\mathalpha}{AMSb}{"52}
\DeclareMathSymbol{\C}{\mathalpha}{AMSb}{"43}
\DeclareMathSymbol{\F}{\mathalpha}{AMSb}{"46}
\DeclareMathSymbol{\E}{\mathalpha}{AMSb}{"45}
\DeclareMathSymbol{\N}{\mathalpha}{AMSb}{"4E}
\DeclareMathSymbol{\T}{\mathalpha}{AMSb}{"54}
\DeclareMathSymbol{\Z}{\mathalpha}{AMSb}{"5A}

\newcommand{\set}[1]{\left\{#1\right\}}
\newcommand{\paren}[1]{\left(#1\right)}
\newcommand{\sq}[1]{\left[#1\right]}
\newcommand{\abs}[1]{\left|#1\right|}
\newcommand{\norm}[1]{\left\|#1\right\|}
\newcommand{\ang}[1]{\left\langle#1\right\rangle}

\renewcommand{\P}{\mathbb{P}}
\newcommand{\A}{\mathcal{A}}
\newcommand{\I}{\mathcal{I}}
\newcommand{\cE}{\mathcal{E}}
\newcommand{\cF}{\mathcal{F}}
\newcommand{\cH}{\mathcal{H}}
\newcommand{\cP}{\mathcal{P}}
\newcommand{\cB}{\mathcal{B}}
\newcommand{\cC}{\mathcal{C}}
\newcommand{\cJ}{\mathcal{J}}
\newcommand{\cK}{\mathcal{K}}
\newcommand{\cL}{\mathcal{L}}
\newcommand{\cN}{\mathcal{N}}
\newcommand{\cS}{\mathcal{S}}
\newcommand{\cX}{\mathcal{X}}
\newcommand{\cY}{\mathcal{Y}}
\newcommand{\cZ}{\mathcal{Z}}
\newcommand{\sD}{\mathscr{D}}
\newcommand{\NC}{\mathcal{NC}}
\newcommand{\BNC}{\mathcal{BNC}}
\newcommand{\LR}{LR}

\newcommand{\bigO}[1]{\mathcal{O}\paren{#1}}

\newcommand{\ocX}{\mathring{\cX}}
\newcommand{\oX}{\mathring{X}}
\newcommand{\oV}{\mathring{V}}

\newcommand{\taur}{\text{\taurus}}
\newcommand{\gem}{\text{\gemini}}

\newcommand{\st}{\mathop{\ast}}
\newcommand{\stst}{\mathop{\ast\ast}}

\newtheorem{theorem}{Theorem}
\newtheorem{proposition}[theorem]{Proposition}
\newtheorem{lemma}[theorem]{Lemma}
\newtheorem{corollary}[theorem]{Corollary}

\theoremstyle{definition}
\newtheorem{definition}[theorem]{Definition}
\newtheorem{example}[theorem]{Example}
\newtheorem{remark}[theorem]{Remark}
\newtheorem{construction}[theorem]{Construction}
\newtheorem{notation}[theorem]{Notation}

\numberwithin{theorem}{section}


\newcommand{\fpf}{\paren{(\A_\ell^{(\iota)}, \A_r^{(\iota)})}_{\iota\in\I}}
\newcommand{\slr}{\set{\ell, r}}
\newcommand{\precc}{\prec_\chi}

\newcommand{\lat}{{\operatorname{lat}}}
\newcommand{\llat}{<_{\lat}}
\newcommand{\llateq}{\leq_{\lat}}
\newcommand{\glat}{>_{\lat}}
\newcommand{\glateq}{\geq_{\lat}}

\newcommand{\alg}{{\operatorname{alg}}}

%%%%%%
% Temporary for amalgamated thingy; get rid of these later...
%%%%%%
\newcommand{\lrleq}{\llateq}
\newcommand{\e}{\iota}
%%%%%%
% End temporary...
%%%%%%

\newcommand{\makeaball}[1]{\begin{tikzpicture}\pgfmathparse{\palette[#1]}\shade[ball color=\pgfmathresult] (0, 0) circle (0.07);\end{tikzpicture}}

%%%%%%%%%%%%%%%%
% Colours
%%%%%%%%%%%%%%%%
\def\greyscale{0}
\ifthenelse{\greyscale>0}{
	\definecolor{iblue}{rgb}{0.75, 0.75, 0.75}
	\definecolor{igreen}{rgb}{0, 0, 0}
	\definecolor{ired}{rgb}{0.55, 0.55, 0.55}
	\definecolor{izure}{rgb}{0.3, 0.3, 0.3}
}{
	\definecolor{iblue}{rgb}{0.2, 0.3, 1.0}
	\definecolor{igreen}{rgb}{0.4, 1.0, 0.3}
	\definecolor{ired}{rgb}{0.8, 0.2, 0.2}
	\definecolor{izure}{rgb}{0.3, 0.8, 1.0}
}

\def\palette{{"iblue", "igreen", "ired", "izure"}}
\colorlet{pal0}{iblue}
\colorlet{pal1}{igreen}
\colorlet{pal2}{ired}
\colorlet{pal3}{izure}


\makeatletter
\pgfkeys{/bnc/.cd,
	n/.code		= {\def\bnc@n{#1}},
	sidez/.code	= {\def\bnc@sidez{#1}},
	labelz/.code	= {\def\bnc@labelz{#1}},
	colourz/.code	= {\def\bnc@colourz{#1}},
	colourzfrompalette/.code	= {\def\bnc@colourzfrompalette{#1}},
	nodez/.code	= {\def\bnc@nodez{#1}},
	drawbase/.code	= {\def\bnc@drawbase{#1}},
	order/.code	= {\def\bnc@order{#1}}
	}
\newcommand{\bnc}[1][]{
	\pgfkeys{/bnc/.cd,
		n = 0,
		sidez = {},
		labelz = {},
		colourz = {},
		colourzfrompalette = {},
		nodez = {},
		drawbase = 1,
		order = {}
	}
	\pgfqkeys{/bnc}{#1}
	\draw[thick] (-1,0.25) coordinate (tl) -- coordinate (cl) ++(0, -0.5*\bnc@n) coordinate (bl)
			++(2,0) coordinate (br) -- coordinate (cr) ++(0, 0.5*\bnc@n) coordinate (tr);
	\ifthenelse{\bnc@drawbase=0}{}{\draw[thick] (bl) -- coordinate (bc) (br);}
	\pgfmathtruncatemacro{\looplimit}{\bnc@n-1}
	\foreach \w in {0, ..., \looplimit} {
		\ifthenelse{\equal{\bnc@order}{}}{\pgfmathtruncatemacro{\y}{\w}}{\pgfmathtruncatemacro{\y}{\bnc@order[\w]}}
		\pgfmathtruncatemacro{\sd}{\bnc@sidez[\y]}

		\ifthenelse{\sd=0}{}{

			\ifthenelse{\equal{\bnc@colourz}{}}{\pgfmathparse{"black"}}{
				\pgfmathparse{\bnc@colourz[\y]}
			}
			\ifthenelse{\equal{\bnc@colourzfrompalette}{}}{}{
				\pgfmathparse{\palette[\bnc@colourzfrompalette[\y]]}}
			\colorlet{clr}{\pgfmathresult}
			\ifthenelse{\equal{\bnc@nodez}{}}{\pgfmathtruncatemacro{\nodename}{\w+1}}{
				\pgfmathparse{\bnc@nodez[\y]}
				\edef\nodename{\pgfmathresult}}
			\ifthenelse{\equal{\bnc@labelz}{}}{\pgfmathtruncatemacro{\nodelabel}{\w+1}}{
				\pgfmathparse{\bnc@labelz[\y]}
				\edef\nodelabel{\pgfmathresult}}

			\node (ball\nodename) [draw, shade, circle, ball color=clr, inner sep=0.07cm] at (\sd, -\w*0.5) {};
			\ifthenelse{\sd=1}{\node[right]}{\node[left]} at (ball\nodename) {\nodelabel};
		}
	}
}
\makeatother
