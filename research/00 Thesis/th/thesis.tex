\documentclass [PhD] {uclathes}

\usepackage{amsrefs}
\usepackage{amsmath}
\usepackage{amsfonts}
\usepackage{amssymb}
\usepackage{amsthm}
%\usepackage{color}
\usepackage{xcolor}
%\usepackage{breqn} %For dmath
\usepackage{mathrsfs} %mathscr?
\usepackage{wasysym} %\eighthnote

\usepackage{enumitem} % For fancy enumerations...

\usepackage{hyperref}

\usepackage{tikz}
\usetikzlibrary{shapes,snakes,calc,arrows,intersections}
\usetikzlibrary{decorations.pathreplacing,shapes.geometric}
\usetikzlibrary{calc,positioning}

\DeclareSymbolFont{AMSb}{U}{msb}{m}{n}
\DeclareMathSymbol{\R}{\mathalpha}{AMSb}{"52}
\DeclareMathSymbol{\C}{\mathalpha}{AMSb}{"43}
\DeclareMathSymbol{\F}{\mathalpha}{AMSb}{"46}
\DeclareMathSymbol{\E}{\mathalpha}{AMSb}{"45}
\DeclareMathSymbol{\N}{\mathalpha}{AMSb}{"4E}
\DeclareMathSymbol{\T}{\mathalpha}{AMSb}{"54}
\DeclareMathSymbol{\Z}{\mathalpha}{AMSb}{"5A}

\newcommand{\set}[1]{\left\{#1\right\}}
\newcommand{\paren}[1]{\left(#1\right)}
\newcommand{\sq}[1]{\left[#1\right]}
\newcommand{\abs}[1]{\left|#1\right|}
\newcommand{\norm}[1]{\left\|#1\right\|}
\newcommand{\ang}[1]{\left\langle#1\right\rangle}

\renewcommand{\P}{\mathbb{P}}
\newcommand{\A}{\mathcal{A}}
\newcommand{\I}{\mathcal{I}}
\newcommand{\cE}{\mathcal{E}}
\newcommand{\cF}{\mathcal{F}}
\newcommand{\cH}{\mathcal{H}}
\newcommand{\cP}{\mathcal{P}}
\newcommand{\cB}{\mathcal{B}}
\newcommand{\cC}{\mathcal{C}}
\newcommand{\cJ}{\mathcal{J}}
\newcommand{\cK}{\mathcal{K}}
\newcommand{\cL}{\mathcal{L}}
\newcommand{\cN}{\mathcal{N}}
\newcommand{\cS}{\mathcal{S}}
\newcommand{\cX}{\mathcal{X}}
\newcommand{\cY}{\mathcal{Y}}
\newcommand{\cZ}{\mathcal{Z}}
\newcommand{\sD}{\mathscr{D}}
\newcommand{\NC}{\mathcal{NC}}
\newcommand{\BNC}{\mathcal{BNC}}
\newcommand{\LR}{LR}

\newcommand{\bigO}[1]{\mathcal{O}\paren{#1}}

\newcommand{\ocX}{\mathring{\cX}}
\newcommand{\oX}{\mathring{X}}
\newcommand{\oV}{\mathring{V}}

\newcommand{\taur}{\text{\taurus}}
\newcommand{\gem}{\text{\gemini}}

\newcommand{\st}{\mathop{\ast}}
\newcommand{\stst}{\mathop{\ast\ast}}

\newtheorem{theorem}{Theorem}
\newtheorem{proposition}[theorem]{Proposition}
\newtheorem{lemma}[theorem]{Lemma}
\newtheorem{corollary}[theorem]{Corollary}

\theoremstyle{definition}
\newtheorem{definition}[theorem]{Definition}
\newtheorem{example}[theorem]{Example}
\newtheorem{remark}[theorem]{Remark}
\newtheorem{construction}[theorem]{Construction}
\newtheorem{notation}[theorem]{Notation}

\numberwithin{theorem}{section}


\newcommand{\fpf}{\paren{(\A_\ell^{(\iota)}, \A_r^{(\iota)})}_{\iota\in\I}}
\newcommand{\slr}{\set{\ell, r}}
\newcommand{\precc}{\prec_\chi}

\newcommand{\lat}{{\operatorname{lat}}}
\newcommand{\llat}{<_{\lat}}
\newcommand{\llateq}{\leq_{\lat}}
\newcommand{\glat}{>_{\lat}}
\newcommand{\glateq}{\geq_{\lat}}

\newcommand{\alg}{{\operatorname{alg}}}

%%%%%%
% Temporary for amalgamated thingy; get rid of these later...
%%%%%%
\newcommand{\lrleq}{\llateq}
\newcommand{\e}{\iota}
%%%%%%
% End temporary...
%%%%%%

\newcommand{\makeaball}[1]{\begin{tikzpicture}\pgfmathparse{\palette[#1]}\shade[ball color=\pgfmathresult] (0, 0) circle (0.07);\end{tikzpicture}}

%%%%%%%%%%%%%%%%
% Colours
%%%%%%%%%%%%%%%%
\def\greyscale{0}
\ifthenelse{\greyscale>0}{
	\definecolor{iblue}{rgb}{0.75, 0.75, 0.75}
	\definecolor{igreen}{rgb}{0, 0, 0}
	\definecolor{ired}{rgb}{0.55, 0.55, 0.55}
	\definecolor{izure}{rgb}{0.3, 0.3, 0.3}
}{
	\definecolor{iblue}{rgb}{0.2, 0.3, 1.0}
	\definecolor{igreen}{rgb}{0.4, 1.0, 0.3}
	\definecolor{ired}{rgb}{0.8, 0.2, 0.2}
	\definecolor{izure}{rgb}{0.3, 0.8, 1.0}
}

\def\palette{{"iblue", "igreen", "ired", "izure"}}
\colorlet{pal0}{iblue}
\colorlet{pal1}{igreen}
\colorlet{pal2}{ired}
\colorlet{pal3}{izure}


\makeatletter
\pgfkeys{/bnc/.cd,
	n/.code		= {\def\bnc@n{#1}},
	sidez/.code	= {\def\bnc@sidez{#1}},
	labelz/.code	= {\def\bnc@labelz{#1}},
	colourz/.code	= {\def\bnc@colourz{#1}},
	colourzfrompalette/.code	= {\def\bnc@colourzfrompalette{#1}},
	nodez/.code	= {\def\bnc@nodez{#1}},
	drawbase/.code	= {\def\bnc@drawbase{#1}},
	order/.code	= {\def\bnc@order{#1}}
	}
\newcommand{\bnc}[1][]{
	\pgfkeys{/bnc/.cd,
		n = 0,
		sidez = {},
		labelz = {},
		colourz = {},
		colourzfrompalette = {},
		nodez = {},
		drawbase = 1,
		order = {}
	}
	\pgfqkeys{/bnc}{#1}
	\draw[thick] (-1,0.25) coordinate (tl) -- coordinate (cl) ++(0, -0.5*\bnc@n) coordinate (bl)
			++(2,0) coordinate (br) -- coordinate (cr) ++(0, 0.5*\bnc@n) coordinate (tr);
	\ifthenelse{\bnc@drawbase=0}{}{\draw[thick] (bl) -- coordinate (bc) (br);}
	\pgfmathtruncatemacro{\looplimit}{\bnc@n-1}
	\foreach \w in {0, ..., \looplimit} {
		\ifthenelse{\equal{\bnc@order}{}}{\pgfmathtruncatemacro{\y}{\w}}{\pgfmathtruncatemacro{\y}{\bnc@order[\w]}}
		\pgfmathtruncatemacro{\sd}{\bnc@sidez[\y]}

		\ifthenelse{\sd=0}{}{

			\ifthenelse{\equal{\bnc@colourz}{}}{\pgfmathparse{"black"}}{
				\pgfmathparse{\bnc@colourz[\y]}
			}
			\ifthenelse{\equal{\bnc@colourzfrompalette}{}}{}{
				\pgfmathparse{\palette[\bnc@colourzfrompalette[\y]]}}
			\colorlet{clr}{\pgfmathresult}
			\ifthenelse{\equal{\bnc@nodez}{}}{\pgfmathtruncatemacro{\nodename}{\w+1}}{
				\pgfmathparse{\bnc@nodez[\y]}
				\edef\nodename{\pgfmathresult}}
			\ifthenelse{\equal{\bnc@labelz}{}}{\pgfmathtruncatemacro{\nodelabel}{\w+1}}{
				\pgfmathparse{\bnc@labelz[\y]}
				\edef\nodelabel{\pgfmathresult}}

			\node (ball\nodename) [draw, shade, circle, ball color=clr, inner sep=0.07cm] at (\sd, -\w*0.5) {};
			\ifthenelse{\sd=1}{\node[right]}{\node[left]} at (ball\nodename) {\nodelabel};
		}
	}
}
\makeatother


\title          {On bi-free probability and free entropy.}
\author         {Ian Lorne Charlesworth}
\department     {Mathematics}
% Note:  degreeyear should be optional, but as of  5-Feb-96
% it seems required or you get a year of ``2''.   -johnh
\degreeyear     {2017}

%%%%%%%%%%%%%%%%%%%%%%%%%%%%%%%%%%%%%%%%%%%%%%%%%%%%%%%%%%%%%%%%%%%%%%%%

\chair          {Dimitri~Y.~Shlyakhtenko}
\member         {Igor~Pak} %TODO: Igor is a non-certifying member. Should he still be included?
\member         {Sorin~Popa}
\member         {Milos~D.~Ercegovac}

%%%%%%%%%%%%%%%%%%%%%%%%%%%%%%%%%%%%%%%%%%%%%%%%%%%%%%%%%%%%%%%%%%%%%%%%

\dedication     {...(todo: dedication goes here)...}

%%%%%%%%%%%%%%%%%%%%%%%%%%%%%%%%%%%%%%%%%%%%%%%%%%%%%%%%%%%%%%%%%%%%%%%%

\acknowledgments {...(todo: acknowledgements go here)...}

%%%%%%%%%%%%%%%%%%%%%%%%%%%%%%%%%%%%%%%%%%%%%%%%%%%%%%%%%%%%%%%%%%%%%%%%

\vitaitem{2007-2012}{B.Math. Honours Pure Mathematics \& Honours Computer Science, University of Waterloo, Waterloo, Ontario, Canada.}
\vitaitem{2012-2016}{Natural Sciences and Engineering Research Council of Canada Postgraduate Scholarship.}
\vitaitem{2012-2017}{Ph.D. in Mathematics, University of California, Los Angeles, Los Angeles, California, U.S.A.}
\vitaitem{2016-2017}{U.C.L.A. Dissertation Year Fellowship.}

%%%%%%%%%%%%%%%%%%%%%%%%%%%%%%%%%%%%%%%%%%%%%%%%%%%%%%%%%%%%%%%%%%%%%%%%

\publication{ A. Nica, I. Charlesworth, and M. Panju. \emph{Analyzing Query Optimization Process: Portraits of Join Enumeration Algorithms}. IEEE 28th International Conference on Data Engineering (2012).}
\publication{ I. Charlesworth, B. Nelson, and P. Skoufranis. \emph{On Two-faced Families of Non-commutative Random Variables}, Canadian Journal of Mathematics \textbf{67} (2015), no. 6, 1290-1325.}
\publication{ I. Charlesworth, B. Nelson, and P. Skoufranis. \emph{Combinatorics of Bi-Freeness with Amalgamation}, Communications in Mathematical Physics \textbf{338} (2015), 801-847.}
\publication{ I. Charlesworth and D. Shlyakhtenko. \emph{Free Entropy Dimension and Regularity of Non-commutative Polynomials}, Journal of Functional Analysis \textbf{271} (2016), no. 8, 2274-2292.}
\publication{ I. Charlesworth. \emph{An alternating moment condition for bi-freeness}. arXiv preprint 1611.01262 (2016).}
\publication{ I. Charlesworth, K. Dykema, F. Sukochev, and D. Zanin. \emph{Joint spectral distributions and invariant subspaces for commuting operators in a finite von Neumann algebra}. arXiv preprint 1703.05695 (2017).}

%%%%%%%%%%%%%%%%%%%%%%%%%%%%%%%%%%%%%%%%%%%%%%%%%%%%%%%%%%%%%%%%%%%%%%%%

\abstract{
	Free probability is a non-commutative analogue of probability theory.
	Recently, Voiculescu has introduced bi-free probability, a theory which aims to study simultaneously ``left'' and ``right'' non-commutative random variables, such as those arising from the left and right regular representations of a countable group.
	We introduce combinatorial techniques to characterise bi-free independence, generalising results of Nica and Speicher from the free setting to the bi-free setting.
	In particular, we develop the lattice of bi-non-crossing partitions which is deeply tied to the action of bi-freely independent random variables on a free product space.
	We use these techniques to show that a conjecture of Mastnak and Nica holds, and bi-free independence is equivalent to the vanishing of mixed bi-free cumulants vanishing.
	Moreover, we extend the theory into the operator-valued setting, introducing operator-valued cumulants which correspond to bi-freeness with amalgamation in the same way.
	Finally, we investigate regularity problems in algebras of non-commuting random variables.
	Using operator theoretic techniques show that in an algebra generated by non-commutative random variables which admit a dual system, any self-adjoint element with spectral measure singular with respect to Lebesgue measure is a multiple of $1$.
	We are also able to slightly improve on prior results in the literature and show that any non-constant self-adjoint polynomial evaluated at a set of non-commutative random variables which are free, algebraic, and have finite free entropy must produce a variable with finite free entropy.
}

%%%%%%%%%%%%%%%%%%%%%%%%%%%%%%%%%%%%%%%%%%%%%%%%%%%%%%%%%%%%%%%%%%%%%%%%



\begin {document}
\makeintropages


%
% introduction.tex
% Copyright (C) 1995 by John Heidemann, <johnh@isi.edu>.
% $Id: demo2intr.tex,v 1.1 1996/01/12 18:13:58 johnh Exp $
%

\chapter{Introduction}

In classical probability, in the context of a probability space $(\Omega,\mc{F},\mathbb{P})$ a random variable is a measurable function $X\colon \Omega\to \R$ and the \emph{moments} of a random variable are the quantities
	\begin{align*}
		E(X^n):=\int_\Omega X(\omega)^n d\mathbb{P}(\omega)\qquad n\geq 0,
	\end{align*}
which capture a great deal of information about the random variable. A random variable $X$ is often studied via its \emph{law}, which is a measure $\mu_X$ on $\R$ that completely characterizes $X$. In particular,
	\begin{align*}
		\mathbb{P}(a\leq X\leq b) = \mu_X([a,b])\qquad \forall -\infty<a\leq b<\infty,
	\end{align*}
and the law describes the moments of $X$:
	\begin{align*}
		E(X^n) = \int_\R t^n\ d\mu_X(t) \qquad \forall n\geq 0.
	\end{align*}
When considering several random variables $X_1,\ldots, X_n\colon \Omega\to \R$, their \emph{joint law} is a measure $\mu_{(X_1,\ldots, X_n)}$ on $\R^n$ satisfying
	\begin{align*}
		\mathbb{P}(X_i\in [a_i,b_i]\colon i\in\{1,\ldots,n\}) = \mu_{(X_1,\ldots, X_n)}( [a_1,b_1]\times\cdots \times [a_n,b_n]),
	\end{align*}
and for any polynomial $p\in \C[t_1,\ldots, t_n]$
	\begin{align*}
		\int_{\Omega^n} p(X_1(\omega_1),\ldots, X_n(\omega_n))\ d\mathbb{P}(\omega_1)\cdots d\mathbb{P}(\omega_n) = \int_{\R^n} p(t_1,\ldots, t_n) d\mu_{(X_1,\ldots, X_n)}(t_1,\ldots, t_n).
	\end{align*}
	
In free probability (or non-commutative probability), the context is usually a unital algebra $A$ and a positive linear functional $\phi\colon A\to \C$ satisfying $\phi(1)=1$. The elements $a\in A$ are thought of as \emph{non-commutative random variables}, and evaluation in $\phi$ corresponds to integration against $d\mathbb{P}$ in classical probability in the sense that the moments of $a$ are given by $\phi(a^n)$, $n\geq 0$. In fact, if $a=a^*$ is self-adjoint then there exists a measure $\mu_a$ on $\R$ which describes the moments of $a$:
	\begin{align*}
		\phi(a^n) = \int_\R t^n\ d\mu_a(t)\qquad \forall n\geq 0.
	\end{align*}
Thus the measure $\mu_a$ is thought of as the law of $a$.

However, if $a$ is not self-adjoint its moments are no longer necessarily described by a measure. In this case the \emph{law} of a general non-commutative random variable refers to the collection of its moments $\{\phi(a^n)\}_{n\geq 0}$. More precisely, it is a linear functional $\phi_a$ on complex polynomials on an abstract indeterminate $t$. For $p\in \C[t]$, if we write $p(a)$ for the polynomial evaluated at $t=a$ then $\phi_a$ is defined by
	\begin{align*}
		\phi_a(p):=\phi(p(a))\qquad \forall p\in \C[t].
	\end{align*}
	
More generally, the \emph{joint law} of an $n$-tuple of non-commutative random variables $(a_1,\ldots, a_n)$ in  $A^n$ is a linear functional $\phi_{(a_1,\ldots, a_n)}$ on non-commutative polynomials in abstract non-commutating indeterminates $t_1,\ldots, t_n$. For $p\in \C\<t_1,\ldots, t_n\>$, if we write $p(a_1,\ldots, a_n)$ for the polynomial evaluated at $t_1=a_1,\ldots, t_n=a_n$ then $\phi_{(a_1,\ldots, a_n)}$ is defined by
	\begin{align*}
		\phi_{(a_1,\ldots, a_n)}(p)=\phi( p(a_1,\ldots, a_n))\qquad p\in \C\<t_1,\ldots, t_n\>.
	\end{align*}

It is often the case that the $*$-algebra is either a $C^*$-algebra or a von Neumann algebra, in which case $(A,\phi)$ has additional structure. For example, if $A=M$ is a $\mathrm{II}_1$ factor, then $\phi$ is usually taken to be the unique tracial state $\tau$ on $M$. In this work, however, we shall consider non-tracial von Neumann algebras equipped with a faithful, normal, non-tracial state.

Transport, in classical probability, refers to a map $T\colon \Omega_1\to \Omega_2$ between two probability spaces $(\Omega_i,\mc{F}_i,\mathbb{P}_i)$, $i\in\{1,2\}$, such that
	\begin{align*}
		\mathbb{P}_1(T^{-1}(S))=\mathbb{P}_2(S)\qquad \forall S\in\mc{F}_2;
	\end{align*}
that is, $T_*\mathbb{P}_1=\mathbb{P}_2$. In particular, $T$ induces a measure preserving map via precomposition:
	\begin{align*}
		L^\infty(\Omega_2,\mathbb{P}_2)\ni f\mapsto f\circ T\in L^\infty(\Omega_1,\mathbb{P}_1).
	\end{align*}
Moreover, if $X_1,\ldots, X_n$ are random variables on $\Omega_2$ with joint law $\mu_{(X_1,\ldots, X_n)}$, then $X_1\circ T,\ldots, X_n\circ T$ are random variables on $\Omega_1$ with the same joint law. In this case we describe $T$ as transport from $\mu_{(X_1,\ldots, X_n)}$ to $\mu_{(X_1\circ T,\ldots, X_n\circ T)}$.

Free transport is the analogue of this latter notion. Let $(\mc{M},\theta)$ and $(\mc{N},\psi)$ be two von Neumann algebra probability spaces with faithful normal states, and let $X:=(X_1,\ldots, X_n)\in \mc{M}^n$ and $Z:=(Z_1,\ldots, Z_n)\in \mc{N}^n$ be two $n$-tuples of non-commutative random variables with joint laws $\theta_X$ and $\psi_Z$, respectively. Then transport from $\theta_X$ to $\psi_Z$  is an $n$-tuple $Y=(Y_1,\ldots, Y_n) \in W^*(X_1,\ldots, X_n)^n$ whose joint law with with respect $\theta$, say $\theta_Y$, is the same as $\psi_Z$:
	\begin{align*}
		\theta_Y(p) = \psi_Z(p)\qquad \forall p\in \C\<t_1,\ldots, t_n\>.
	\end{align*}
In particular, the densely defined map
	\begin{align*}
		W^*(Z_1,\ldots, Z_n)\ni p(Z_1,\ldots, Z_n) \mapsto p(Y_1,\ldots, Y_n)\in W^*(X_1,\ldots, X_n)\qquad p\in \C\<t_1,\ldots, t_n\>,
	\end{align*}
extends to a state-preserving embedding $W^*(Z_1,\ldots, Z_n)\hookrightarrow W^*(X_1,\ldots, X_n)$.

Transport maps are abundant in classical probability because of Brenier's monotone transport theorem \cite{Bre91}: if the joint law of classical random variables $X_1,\ldots, X_n$ is the standard Gaussian distribution on $\R^n$:
	\begin{align*}
		\frac{d\mu_{(X_1,\ldots, X_n)}}{dm_n}(t_1,\ldots, t_n) =\frac{1}{\sqrt{(2\pi)^n}} \text{exp}\left(-\frac{1}{2}\sum_{j=1}^n t_j^2\right)
	\end{align*}
(here $m_n$ is the Lebesgue measure on $\R^n$), then there exists transport from $\mu_{(X_1,\ldots, X_n)}$ to any other joint law $\mu_{(Z_1,\ldots, Z_N)}$ satisfying some technical conditions (Lebesgue absolutely continuous, finite second moment, etc.). Moreover, the transport map $T$ can be taken to be monotone: $T=\nabla G$ for some convex function $G$. In free probability, transport is much harder to come by.

In \cite{GS14}, by solving a free analogue of the Monge-Amp\'{e}re equation, Guionnet and Shlyakhtenko obtained transport from the joint law of free semi-circular random variables $X_1,\ldots, X_n \in M_{s.a.}$ in a tracial von Neumann algebra $(M,\tau)$ to certain perturbations of this joint law, which we will discuss below. A semi-circular random variable $X\in M$ is an element whose distribution with respect to $\tau$, $\tau_X$, satisfies
	\begin{align*}
		\frac{d\tau_X}{d t}(t) = \chi_{[-2,2]}(t) \frac{1}{2\pi} \sqrt{4 - t^2}.
	\end{align*}
Free semi-circular variables are the non-commutative analogue of independent Gaussian random variables, insomuch as Voiculescu's free central limit theorem (\emph{cf.} \cite{Voi91}) is precisely the classical central limit theorem with independence and Gaussian random variables replaced by free independence and semi-circular random variables, respectively. Hence this result of Guionnet and Shlyakhtenko can be viewed as a non-commutative analogue of Brenier's monotone transport theorem. Furthermore, if sufficient control on the transport variables is maintained then the state-preserving embedding guaranteed by free transport is in fact a state-preserving $*$-isomorphism. Consequently, this result provided criterion for when an $n$-tuple of non-commutative random variables generate the free group factor $L \mathbb{F}_n = W^*(X_1,\ldots, X_n)$.

The non-commutative joint laws to which Guionnet and Shlyakhtenko obtained transport to were perturbations of $\tau_{(X_1,\ldots, X_n)}$ in the following sense. The trace $\tau$ satisfies a ``free Gibbs state'' condition with respect to a ``Gaussian potential'' $V_0=\frac{1}{2}\sum X_j^2$:
	\begin{align*}
		\tau( \D(V_0) \cdot P) = \tau\otimes\tau^{op}(\J P)\qquad P\in \C\<X_1,\ldots, X_n\>^n,
	\end{align*}
where $\D$ and $\J$ are non-commutative differential operators (\emph{cf.} subsection \ref{tensor_product_notation}). Suppose $Z_1,\ldots, Z_n$ are self-adjoint elements from another tracial von Neumann algebra $(\tilde{M}, \tilde{\tau})$ whose joint law satisfies this free Gibbs state condition for some other potential $V\in W^*(Z_1,\ldots, Z_n)$. Guionnet and Shlyakhtenko showed in \cite{GS14} that provided $V$ is a convergent power series in $Z_1,\ldots, Z_n$ which is close in some Banach norm (\emph{cf.} subsection \ref{setup}) to $V_0$ (when considering both as formal power series) then transport from $\tau_{(X_1,\ldots, X_n)}$ to $\tilde{\tau}_{(Z_1,\ldots, Z_n)}$ exists. Moreover, by requiring $V$ to be closer to $V_0$ if necessary, it follows that $W^*(X_1,\ldots, X_n)\cong W^*(Z_1,\ldots, Z_n)$.
 
In the commutative case (i.e. $n=1$), the free Gibbs state condition amounts to saying that if $\eta$ is the semi-circle law ($\frac{d\eta}{dt}(t)=\chi_{[-2,2]}(t) \frac{1}{2\pi}\sqrt{4-t^2}$) and $V(t)=\frac{1}{2} t^2 + W(t)$ for $W$ analytic on a disk of radius $R$ and small $\|\cdot\|_\infty$-norm then
	\begin{align*}
		\int_\R V'(t) f(t)\ d\eta(t) = \int_\R \int_\R \frac{f(s) - f(t)}{s-t} d\eta(s) d\eta(t)
	\end{align*}
for all $f$ which are analytic on the disk of radius $R$. In general, a measure satisfying this equation is called a Gibbs state with potential $V$.

Given a potential $V$ close to $V_0$ and starting with the non-commutative free Gibbs state condition, Guionnet and Shlyakhtenko produced an equivalent condition which is amenable to a fixed point argument. Using this latter condition they show the existence of $Y_1,\ldots, Y_n$ power series in the $X_1,\ldots, X_n$ whose joint law with respect to $\tau$ satisfies the free Gibbs state condition with potential $V$. Then a result of Guionnet and Maurel-Segala in \cite{GM06} implies that this condition is uniquely satisfied by a joint law (again provided $V$ and $V_0$ are sufficiently close). Hence $Y_1,\ldots, Y_n$ serve as transport variables for any other $n$-tuple $(Z_1,\ldots, Z_n)$ whose joint law satisfies the free Gibbs condition with potential $V$.

In Chapter \ref{non-tracial transport chapter} we adapt the transport result of Guionnet and Shlyakhtenko to the context of a von Neumann algebra $M$ with a (not necessarily tracial) state $\varphi$ on $M$. The random variables $X_1,\ldots, X_N$ are no longer assumed to be free; instead their joint law is assumed to be a free quasi-free state and they generate the free Araki-Woods factor $\Gamma(\H_\R, U_t)''$ (\emph{cf.} \cite{Shl97}). While the state is no longer tracial, in this case there at least exists a positive matrix $A\in M_N(\C)$ such that
	\begin{align*}
		\varphi(X_jX_k) = \varphi\left(X_k \sum_{\ell = 1}^N [A]_{j\ell} X_\ell\right).
	\end{align*}
Really, $A$ here is encoding the action of the modular operator $\Delta_\varphi$ arising from the Tomita-Takesaki theory for $\varphi$. The Gaussian potential $V_0$ is replaced by
	\begin{align*}
		V_0:= \frac{1}{2} \sum_{j,k=1}^n \left[\frac{1+A}{2}\right]_{jk} X_k X_j.
	\end{align*}
Using the same strategy as in \cite{GS14} and making non-tracial adaptations along the way, we construct for potentials $V$ close to $V_0$ transport variables $Y_1,\ldots, Y_N$ for any other joint law which is the free Gibbs state with potential $V$. This produces a criterion for when an $N$-tuple of non-commutative random variables generate the free Araki-Woods factor $\Gamma(\H_\R,U_t)''$.


In Chapter \ref{application} we consider our first application of non-tracial free transport. Hiai developed in \cite{Hia03} a generalization of Shlyakhtenko's algebras $\Gamma(\H_\R, U_t)$ from \cite{Shl97}, called $q$-deformed Araki-Woods algebras. Letting $A$ be the generator of the one-parameter family of unitary operators $U_t=A^{it}$ $t\in \R$, Hiai was able to show the von Neumann algebras $\Gamma_q(\H_\R,U_t)''$ are factors and produced a type classification, but only in the case that $A$ has either infinitely many mutually orthogonal eigenvectors or none at all. In particular, when the Hilbert space $\H_\R$ is finite dimensional the questions of factoriality and type classification remained unanswered. An application of our result in Section \ref{application} yields $\Gamma_q(\H_\R,U_t)''\cong \Gamma(\H_\R,U_t)''$ for small $|q|$, and hence we are able to settle these questions using Theorem 6.1 in \cite{Shl97}.



In Chapter \ref{planar_algebras_chapter} we consider our second application of non-tracial free transport. Despite the relatively innocuous definition of a subfactor, Jones showed in \cite{Jon83}, \cite{Jon99}, and \cite{Jon00} that there is in fact an incredibly rich structure underlying the inclusion of one $\II_1$ factor in another. Suppose $1_B\in A\subset B$ is an inclusion of $\II_1$ factors with trace $\Tr_B$ on $B$ and trace $\Tr_A=\Tr_B\mid_A$ on $A$. Letting $e_A$ be the orthogonal projection of $L^2(B,\Tr_B)$ onto $L^2(A,\Tr_A)$, we can consider the von Neumann algebra $\<B,e_A\>\subset \B(L^2(B,\Tr_B))$ generated by $B$ and $e_A$. If the index of $A$ inside $B$
	\begin{align*}
		[B\colon A]:=\dim_A L^2(B,\Tr_B)
	\end{align*}
is finite, then $\<B,e_A\>$ is also a $\II_1$ factor with trace $\Tr_{\<B,e_A\>}$ that restricts to $\Tr_B$ on $B$. Moreover, we have $[\<B,e_A\>\colon B]=[B\colon A]$. The von Neumann algebra $\<B,e_A\>$ is called the basic construction for $A$ and $B$. Clearly this process may be iterated and doing so yields the Jones tower:
	\begin{align*}
		A_0 \subset A_1 \subset_{e_1} A_2 \subset_{e_2} A_3 \subset_{e_3} \cdots
	\end{align*}
where $A_0=A$, $A_1=B$, and $e_1=e_A$. The standard invariant of $A\subset B$ is then the lattice of higher relative commutants:
	\[
	\begin{tikzpicture}[thick, scale=.5]
	\node at (0,0) {$A_0'\cap A_0$};
	
	\node at (2.25,0) {$\subset$};
	
	\node at (4.5, 0) {$A_0'\cap A_1$};
	\node at (4.5,-1) {\rotatebox{90}{$\subset$}};
	\node at (4.5,-2) {$A_1'\cap A_1$};
	
	\node at (6.75,0) {$\subset$};
	\node at (6.75,-2) {$\subset$};
	
	\node at (9,0) {$A_0'\cap A_2$};
	\node at (9,-1) {\rotatebox{90}{$\subset$}};
	\node at (9,-2) {$A_1'\cap A_2$};
	
	\node at (11.75,0) {$\subset\cdots $};
	\node at (11.75,-2) {$\subset\cdots $};
	
	\end{tikzpicture}
	\]
By studying $\lambda$-lattices, Popa found necessary and sufficient conditions for when such lattices are the standard invariant of a $\II_1$-subfactor $A\subset B$, and for each such lattice provided a construction of a canonical subfactor whose standard invariant recovered the lattice (\emph{cf.} \cite{Pop93}, \cite{Pop95}, and \cite{Pop02}).

Collecting these relative commutants as $\mc{P}:=\{\mc{P}_{n,\pm}\}_{n\geq 0}$, where for each $n\geq 0$
	\begin{align*}
		\mc{P}_{n,+}&:= A_0'\cap A_n\\
		\mc{P}_{n,-}&:= A_1' \cap A_{n+1},
	\end{align*}
defines a subfactor planar algebra. More generally, a planar algebra is a collection of graded vector spaces $\{P_{n,\pm}\}_{n\geq 0}$ which admits an action by planar tangles: diagrams which encode multilinear maps. A subfactor planar algebra is a planar algebra which satisfies some additional analytic properties.

In \cite{GJS10} Guionnet, Jones, and Shlyakhtenko use free probabilistic methods to construct a subfactor with $\mc{P}$ as its standard invariant, and hence is an alternative approach to Popa's earlier result. Given a subfactor planar algebra $\mc{P}$, for each $k\geq 0$ one can turn $Gr_k^+\mc{P}=\oplus_{n\geq k} \mc{P}_{n,+}$ into a $*$-algebra with a trace $Tr_{k,+}$ defined by a particular pairing with Temperley-Lieb diagrams.  Then each $Gr_k^+\mc{P}$ embeds into the bounded operators on a Hilbert space and generates a $\text{II}_1$ factor $M_{k,+}$. Moreover, one can define inclusion maps $i^{k-1}_k\colon M_{k-1,+}\to M_{k,+}$ so that the standard invariant associated to the subfactor inclusion $i_k^{k-1}(M_{k-1,+})\subset M_{k,+}$ (for any $k\geq 1$) recovers $\mc{P}$. The embedding relies on the fact that a subfactor planar algebra $\mc{P}$ always embeds into the planar algebra of a bipartite graph $\mc{P}^\Gamma$ (\emph{cf.} \cite{Jon00}, \cite{JP11}, and \cite{MW10}). 

It turns out that $Gr_0^+\mc{P}$ embeds as a subalgebra of a free Araki-Woods factor. In Chapter \ref{planar_algebras_chapter} we show that the free transport machinery can be encoded via planar tangles and provide an application of free transport to finite depth subfactor planar algebras.

Let $\mc{P}$ be a finite depth subfactor planar algebra and $Tr\colon \mc{P}\to\C$ be the state induced by the Temperley-Lieb diagrams via duality. By using the transport construction methods of Chapter \ref{non-tracial transport chapter}, we show that we can perturb the embedding constructed in \cite{GJS10} to make it state-preserving for states on $\mc{P}$ which are ``close'' to $Tr$. Moreover, the von Neumann algebra generated by the subfactor planar algebra via this embedding is unchanged. In this context, if $\mc{P}$ embeds into $\mc{P}^\Gamma$ and $\mu$ is the Perron-Frobenius eigenvector for the bipartite graph $\Gamma$, then the generator $A$ associated to the free Araki-Woods factor will be determined by $\mu$.

The free transport methods in \cite{GS14} and Chapter \ref{non-tracial transport chapter} apply only to joint laws of finitely many non-commutative random variables. Since each edge in the graph $\Gamma$ will correspond to a non-commutative random variable, we can only consider finite depth subfactor planar algebras with these methods.


\chapter{Bi-free independence.}
\label{ch:bfi}

The goal of this chapter is to make an examination of bi-free independence as introduced by Voiculescu in \cite{voiculescu2014free}, and as introduced in Subsection~\ref{ss:introbifree}.
We will show that the combinatorial bi-free independence of Mastnak and Nica is equivalent to bi-free independence, and also develop a vanishing moment condition inspired by the one mentioned in Proposition~\ref{prop:freemomentsvanish}.
To do this, we will need to introduce some combinatorial machinery.

\section{Some combinatorial structures.}
We need two main combinatorial structures for our arguments: bi-non-crossing partitions, and shaded $LR$-diagrams.

\subsection{Bi-non-crossing partitions.}
\begin{definition}
	Let $\chi: [n]\to\slr$, and let $s_\chi \in \cS_n$ be as above: if $\chi^{-1}(\ell) = \set{i_1 < \cdots < i_k}$ and $\chi^{-1}(r) = \set{i_{k+1} > \cdots > i_n}$, then $s_\chi(j) := i_j$.
	Then $\chi$ induces an ordering on $[n]$ via $i \precc j$ if and only if $s_\chi^{-1}(i) < s_\chi^{-1}(j)$.
	We will also denote intervals in this ordering by $[i, j]_\chi := \set{k \in [n] : i \preceq_\chi k \preceq_\chi j}$ (and use similar notation for open and half-open intervals), and refer to such sets as $\chi$-intervals.

	The set of \emph{bi-non-crossing partitions with respect to $\chi$} (or \emph{$\chi$-non-crossing partitions}) is equal to the set of non-crossing partitions on the ordered set $\set{(i, \chi(i)) : i \in [n]}$, with the ordering given by $\precc$ on the first coordinate.
	We denote the set of such partitions by $\BNC(\chi)$.
	It will be convenient for us to think of elements of $\BNC(\chi)$ as partitions of $[n]$, and this is accomplished by projecting onto the first coordinate; from here on we will always make this identification implicitly.
\end{definition}

The reason we have formally make $\pi$ include the information from $\chi$ is two-fold: we wish to be able to recover $\chi$ from $\pi$, and we wish $\BNC(\chi)$ to be formally disjoint for different $\chi$.
However, this technicality is almost never worth calling attention to.
We also remark that $\chi$-non-crossing partitions correspond precisely to those which may be drawn without crossings between a pair of vertical lines, with labelled nodes added to the left or right line according to $\chi$.
In this picture, the ordering $\precc$ corresponds to reading the labels down the left line and then up the right line.
If we have such a diagram drawn using horizontal and vertical lines so that every block of $\pi$ contains at most one vertical line segment, we will refer to that segment as the \emph{spine} of the block it corresponds to, and the horizontal pieces as \emph{ribs}.

\begin{example}
	\label{ex:bnc}
	Suppose $\chi : [8] \to \slr$ is such that $\chi^{-1}(\ell) = \set{1, 3, 4, 6, 8}$ and $\chi^{-1}(r) = \set{2, 5, 7}$.
	\[\begin{tikzpicture}[baseline]
		\draw[thick] (-1,0.25) -- (-1, -3.75) -- (1,-3.75) -- (1,0.25);

		\def\sidez{{-1,1,-1,-1,1,-1,1,-1}}
		\foreach \y in {0,...,7} {
			\pgfmathtruncatemacro{\nodename}{\y+1}
			\pgfmathtruncatemacro{\sd}{\sidez[\y]}
			\node (ball\nodename) [draw, shade, circle, ball color=black, inner sep=0.07cm] at (\sd, -\y*0.5) {};
			\ifthenelse{\sd=1}{\node[right] at (\sd, -\y*0.5) {\nodename}}
					{\node[left] at (\sd, -\y*0.5) {\nodename}};
		}

		\draw [thick] (ball1) -- (ball1 -| 0, 0) |- (ball3);
		\draw [thick] (ball2) -- (ball2 -| 0.5, 0) |- (ball4);
		\draw [thick] (ball5) -- (ball5 -| 0, 0) |- (ball8);
		\draw [thick] (ball6) -- (ball6 -| 0, 0) |- (ball7);
	\end{tikzpicture}\]
	Then $\set{\set{1, 3}, \set{2, 4}, \set{5, 6, 7, 8}} \in \BNC(\chi)$ even though it is not a non-crossing partition.
	We also have the ordering
	$1 \precc 3 \precc 4 \precc 6 \precc 8 \precc 7 \precc 5 \precc 2$.
\end{example}

Finally, observe that $\BNC(\chi) = \set{\pi \in \cP(n) : s_\chi^{-1}\cdot \pi \in \NC(n)}$; thus we have the following expression for the bi-free cumulants:
$$\varphi(z_1\cdots z_n) = \sum_{\pi \in \BNC(\chi)} \kappa_\pi(z_1, \ldots, z_n).$$

\subsection{Shaded $LR$-diagrams.}
\label{ss:shadedlrdiagrams}
We will now introduce a set of diagrams, called \emph{shaded $LR$-diagrams}, which will be of use when computing joint moments of bi-free variables.
In this context, a \emph{diagram} consists of the following data:
\begin{itemize}
	\item $n \in \N_0$;
	\item a map $\chi : [n] \to \slr$;
	\item a map $\iota : [n] \to \I$;
	\item a $\chi$-non-crossing partition $\pi$ which in $\cP(n)$ satisfies $\pi \prec \set{\iota^{-1}(j) : j \in \I}$; and
	\item a subset $S \subset \pi$ which are \emph{outer} in the sense that if $B \in S$ and $C \in \pi$, and there are $i,k \in C$, $j \in B$ with $i \precc j \precc k$, then $B = C$.
\end{itemize}
The diagrammatic representation will be given by drawing the partition $\pi$, colouring the blocks of $\pi$ according to $\iota$, and extending the spines of blocks in $S$ to the top of the diagram.
Note that the outerness assumption on the blocks in $S$ ensures that this can be done in a non-crossing fashion; moreover, it allows us to order the blocks in $S$ by $\precc$.
We will often abbreviate the condition $\pi \prec \set{\iota^{-1}(j) : j \in \I}$ simply as $\pi \prec \iota$.
Given $B \in \pi$, we will often write $\iota(B)$ as shorthand for the value $\iota(b)$ for $b \in B$.

\begin{example}
	\label{ex:lrdiagram1}
	Let $\chi : [8] \to \slr$ be as in Example~\ref{ex:bnc}, define $\iota$ by the sequence
	{\def\clrz{{0,1,0,1,0,0,0,0}}
		\foreach \y in {0,...,7} {
			\ifthenelse{\y = 0}{(}{}
			\makeaball{\clrz[\y]}
			\ifthenelse{\y < 7}{,}{)}
		}
	},
	and let $S = \set{\set{1, 3}, \set{2, 4}}$.
	Then we have the following drawing of this diagram:
	\[\begin{tikzpicture}[baseline]
		\draw[thick] (-1,0.25) -- (-1, -3.75) -- (1,-3.75) -- (1,0.25);

		\def\sidez{{-1,1,-1,-1,1,-1,1,-1}}
		\def\clrz{{0,1,0,1,0,0,0,0}}
		\foreach \y in {0,...,7} {
			\pgfmathtruncatemacro{\nodename}{\y+1}
			\pgfmathtruncatemacro{\sd}{\sidez[\y]}
			\pgfmathparse{\palette[\clrz[\y]]}
			\colorlet{clr}{\pgfmathresult}
			\node (ball\nodename) [draw, shade, circle, ball color=clr, inner sep=0.07cm] at (\sd, -\y*0.5) {};
			\ifthenelse{\sd=1}{\node[right] at (\sd, -\y*0.5) {\nodename}}
					{\node[left] at (\sd, -\y*0.5) {\nodename}};
		}

		\pgfmathparse{\palette[0]}
		\colorlet{clr}{\pgfmathresult}
		\draw [clr, thick] (ball1) -- (ball1 -| 0, 0) |- (ball3);
		\draw [clr, thick] (ball1) -| (0, 0.25);
		\draw [clr, thick] (ball5) -- (ball5 -| 0, 0) |- (ball8);
		\draw [clr, thick] (ball6) -- (ball6 -| 0, 0) |- (ball7);

		\pgfmathparse{\palette[1]}
		\colorlet{clr}{\pgfmathresult}
		\draw [clr, thick] (ball2) -- (ball2 -| 0.5, 0) |- (ball4);
		\draw [clr, thick] (ball2) -| (0.5, 0.25);
	\end{tikzpicture}\]
\end{example}

We now construct the set of shaded $LR$-diagrams recursively.
First, we insist that the unique diagram with $n = 0$ (hence no nodes) is a shaded $LR$-diagram.
Now, suppose $D = (n, \chi, \iota, \pi, S)$ is a shaded $LR$-diagram, and let $\hat\chi : [n+1]\to\slr$, $\hat\iota : [n+1]\to\I$ be such that for $i > 1$, $\hat\chi(i+1) = \chi(i)$ and $\hat\iota(i+1) = \iota(i)$.
We prescribe $\hat\pi$ in the following way:
\begin{itemize}
	\item $(i+1) \sim_{\hat\pi} (j+1)$ if and only if $i \sim_\pi j$;
	\item if $S = \emptyset$, then $1$ is a singleton in $\hat\pi$;
	\item if $B \in S$ is the block with spine closest to the $\hat\chi(1)$ side of the diagram and the colour of $B$ matches $\hat\iota(1)$, then $1\sim_{\hat\pi} (j+1)$ for every $j \in B$;
	\item if neither of the above conditions occurs, $1$ is a singleton in $\hat\pi$. 
\end{itemize}
Notice that $\hat\pi \in \BNC(\hat\chi)$ as $\pi\in\BNC(\chi)$ and all blocks of $S$ are outer; notice also that the block containing $1$ in $\hat\pi$ is also outer.
Finally, let $\hat S$ be the set of blocks in $S$ which, after relabelling, are still blocks of $\hat\pi$; that is, if $1$ was added to a pre-existing block of $\pi$, we exclude it from $\hat S$.
Then if $B_1 \in \hat\pi$ is the block containing $1$, both $(n+1, \hat\chi, \hat\iota, \hat\pi, \hat S)$ and $(n+1, \hat\chi, \hat\iota, \hat\pi, \hat S\cup \set{B_1})$ are shaded $LR$-diagrams.
The set of shaded $LR$-diagrams, $\LR$, is the smallest set which is closed under the above extension.
Notice that each shaded $\LR$-diagram on $n+1$ nodes arises through this construction from a unique $\LR$-diagram on $n$ nodes, which can be recovered simply by deleting the top part of the diagram.

We define some particular subsets of $\LR$:
\begin{align*}
	\LR_k &:= \set{(n, \chi, \iota, \pi, S) \in \LR : \abs{S} = k},\\
	\LR(\chi, \iota) &:= \set{(n, \chi', \iota', \pi, S) \in \LR : \chi' = \chi, \iota' = \iota},\\
	\LR_k(\chi, \iota) &:= \LR_k \cap \LR(\chi, \iota).
\end{align*}

\begin{example}
	Let $D = (8, \chi, \iota, \pi, S)$ be as in Example~\ref{ex:lrdiagram1}.
	Suppose we wish to extend $D$ with $\hat\iota(1) = \makeaball{1}$.
	Then we have the following four diagrams as options, depending on the choice of $\hat\chi(1)$.
	\[
		\begin{tikzpicture}[baseline]
			\foreach \d in {0,...,3}{
				\begin{scope}[xshift = -6cm + 4*\d cm]
					\draw[thick] (-1,0.25) -- (-1, -4.25) -- (1,-4.25) -- (1,0.25);

					\def\sidez{{0,-1,1,-1,-1,1,-1,1,-1}}
					\def\clrz{{1,0,1,0,1,0,0,0,0}}
					\foreach \y in {0,...,8} {
						\pgfmathtruncatemacro{\nodename}{\y+1}
					\ifthenelse{\y>0}{
						\pgfmathtruncatemacro{\sd}{\sidez[\y]}
					}{
						\pgfmathtruncatemacro{\sd}{floor(\d/2)*2-1}
					}
					\pgfmathparse{\palette[\clrz[\y]]}
					\colorlet{clr}{\pgfmathresult}
					\node (\d ball\y) [draw, shade, circle, ball color=clr, inner sep=0.07cm] at (\sd, -\y*0.5) {};
					\ifthenelse{\sd=1}{\node[right] at (\sd, -\y*0.5) {\nodename}}
						{\node[left] at (\sd, -\y*0.5) {\nodename}};
					}

					\pgfmathparse{\palette[0]}
					\colorlet{clr}{\pgfmathresult}
					\draw [clr, thick] (\d ball1) -- (\d ball1 -| 0, 0) |- (\d ball3);
					\draw [clr, thick] (\d ball1) -| (0, 0.25);
					\draw [clr, thick] (\d ball5) -- (\d ball5 -| 0, 0) |- (\d ball8);
					\draw [clr, thick] (\d ball6) -- (\d ball6 -| 0, 0) |- (\d ball7);

					\pgfmathparse{\palette[1]}
					\colorlet{clr}{\pgfmathresult}
					\coordinate (\d righttop) at ($ (0.5, 0.25) - 0.25*floor(\d/2)*(0, 1) $);
					\draw [clr, thick] (\d ball2) -- (\d ball2 -| 0.5, 0) |- (\d ball4);
					\draw [clr, thick] (\d ball2) -| (\d righttop);
				\end{scope}}

			\pgfmathparse{\palette[1]}
			\colorlet{clr}{\pgfmathresult}
			\draw [clr, thick] (0ball0) -- ++(0.3,0);
			\draw [clr, thick] (1ball0) -| ( {$ (1ball0) + (0.5,0) $} |- 0,0.25 );
			\draw [clr, thick] (2ball0) -| (2righttop);
			\draw [clr, thick] (3ball0) -| (3righttop |- 0,0.25);
		\end{tikzpicture}\]
	The first and last diagrams lie in $\LR_2$, the second lies in $\LR_3$, and the third lies in $\LR_1$.
\end{example}

Notice that we can only increase the number of spines reaching the top of the diagram when we add a node whose colour does not match that of the extended spine nearest it; thus the spines reaching the top of the diagram will always be of alternating colours.

\begin{definition}
	Fix $n \in \N$, and let $\chi : [n]\to\slr$ and $\iota:[n]\to\I$.
	We define the set $\BNC(\chi, \iota)$ as
	$$\BNC(\chi, \iota) :=  \set{\pi \in \BNC(\chi) : (n, \chi, \iota, \pi, \emptyset) \in \LR}.$$
\end{definition}

Note that even if $\iota$ is constant on the blocks of $\pi \in \BNC(\chi)$, it is not necessarily true that $\pi \in \BNC(\chi, \iota)$!
This is because we have asked that $\pi$ can be constructed through a certain process which connects blocks of the same colour when both are on the same level and such a connection maintains the bi-non-crossing property.
\begin{example}
	If $n = 3$, $\iota \equiv \makeaball{0}$, and $\chi \equiv \ell$, $\pi = \set{\set{1,3}, \set{2}} \notin \BNC(\chi, \iota)$.
	\[\begin{tikzpicture}
		\draw[thick] (-1,0.25) -- ++(0, -1.5) -- ++(2,0) -- ++(0,1.5);

		\def\sidez{{-1,-1,-1}}
		\def\clrz{{0,0,0}}
		\foreach \y in {0,1,2} {
			\pgfmathtruncatemacro{\nodename}{\y+1}
			\pgfmathtruncatemacro{\sd}{\sidez[\y]}
			\pgfmathparse{\palette[\clrz[\y]]}
			\colorlet{clr}{\pgfmathresult}
			\node (ball\nodename) [draw, shade, circle, ball color=clr, inner sep=0.07cm] at (\sd, -\y*0.5) {};
			\ifthenelse{\sd=1}{\node[right] at (\sd, -\y*0.5) {\nodename}}
					{\node[left] at (\sd, -\y*0.5) {\nodename}};
		}

		\pgfmathparse{\palette[0]}
		\colorlet{clr}{\pgfmathresult}
		\draw [clr, thick] (ball1) -- ++(1,0) |- (ball3);
		\draw [clr, thick] (ball2) -- ++(0.3,0);
	\end{tikzpicture}\]
	If it were being constructed as an $\LR$-diagram, it would be necessary to connect $2$ to the spine extending upwards from $3$ when it is added.
\end{example}

In order to better discuss these conditions, we introduce some terminology:
\begin{definition}
	Let $\pi \in \BNC(\chi)$, and $B, C \in \pi$.
	Then $B$ and $C$ are said to be \emph{piled} if $\max(\min(B), \min(C)) \leq \min(\max(B), \max(C))$.
	Diagrammatically, this means that there is necessarily some horizontal level on which both the spines of $B$ and $C$ are present.

	A third block $D \in \pi$ \emph{separates} $B$ from $C$ if it is piled with both, equal to neither, and its spine lies between the spines of $B$ and $C$.
	Equivalently, $D$ is piled with $B$ and $C$, and there are $j, k \in D$ so that $B \subset (j, k)_\chi$ while $C \cap [j, k]_\chi = \emptyset$, or vice versa.

	Finally, $B$ and $C$ are \emph{tangled} if they are piled and no block separates them.

	We extend these notions to $\LR$ in the obvious way.
\end{definition}

We can now succinctly state the condition on partitions being in $\BNC(\chi, \iota)$: there must be no two tangled blocks of the same colour.
We also remark that any element $\pi \in \BNC(\chi)$ lies in $\BNC(\chi, \iota)$ for some $\iota : [n] \to \I$ provided $\abs\I \geq 2$: one may first colour a block of $\pi$, then colour all blocks it is tangled with in the opposite colour, and continue in this fashion until $\pi$ is coloured or there are no more restrictions, at which point another uncoloured block may be coloured.
Since the relation of entanglement does not allow for any cycles (any given non-singleton block splits the diagram into two or more regions, and only it may be tangled with blocks in more than one of these regions), this will always produce a consistent colouring.
We conclude
$$\BNC(\chi) = \bigcup_{\iota\in\I^n} \BNC(\chi, \iota).$$

\begin{definition}
	Suppose $\pi, \sigma \in \BNC(\chi)$.
	We say $\sigma$ is a \emph{lateral refinement} of $\pi$ and write $\sigma \llat \pi$ if no two piled blocks of $\sigma$ are contained in the same block of $\pi$.
	Such refinements correspond to making lateral ``cuts'' along the spines of blocks of $\pi$.
\end{definition}

\begin{example}
	Suppose $\chi = (\ell, r, \ell, r)$ and $\pi = 1_\chi$.
	Then $\set{\set{1, 2}, \set{3, 4}} \llat \pi$ while $\set{\set{1, 3}, \set{2, 4}} \not\llat \pi$.

	\[\begin{tikzpicture}
		\def\sidez{{-1,1,-1,1}}

		\foreach \d in {-2, 2} {
			\begin{scope}[xshift=\d cm]
				\draw[thick] (-1,0.25) -- ++(0, -2) -- ++(2,0) -- ++(0,2);
				\foreach \y in {0,1,2,3} {
					\pgfmathtruncatemacro{\nodename}{\y+1}
					\pgfmathtruncatemacro{\sd}{\sidez[\y]}
					\node (\d ball\nodename) [draw, shade, circle, ball color=black, inner sep=0.07cm] at (\sd, -\y*0.5) {};
					\ifthenelse{\sd=1}{\node[right] at (\sd, -\y*0.5) {\nodename}}
							{\node[left] at (\sd, -\y*0.5) {\nodename}};
				}
			\end{scope}
		}

		\draw [thick] (-2ball1) -- ++(1,0) |- (-2ball2);
		\draw [thick] (-2ball3) -- ++(1,0) |- (-2ball4);

		\draw[thick] (2ball1) -- ++(2/3,0) |- (2ball3);
		\draw[thick] (2ball2) -- ++(-2/3,0) |- (2ball4);
	\end{tikzpicture}\]

\end{example}

\section{Computing joint moments of bi-free variables.}
We will now make explicit the connection between shaded $\LR$-diagrams and joint moments of bi-free variables.
Let $(V^{(\iota)}, \oV^{(\iota)}, \xi^{(\iota)})_{\iota\in\I}$ be a family of vector spaces with specified state vectors, and let $\pi_\ell^{(\iota)}, \pi_r^{(\iota)}$ be the left and right representations of $L(V^{(\iota)})$ on $(V, \oV, \xi) := \st_{\iota\in\I}(V^{(\iota)}, \oV^{(\iota)}, \xi^{(\iota)})$.
Also, define $\psi^{(\iota)} : V^{(\iota)} \to \C$ as the functional with $v \in \oV^{(\iota)} + \psi^{(\iota)}\xi^{(\iota)}$, and let $p^{(\iota)}(x) = \psi^{(\iota)}(x)\xi^{(\iota)}$, and $\psi, p$ in a similar fashion for $V$.
Notice that, for $T \in L(V^{(\iota)})$, $p^{(\iota)}Tp^{(\iota)} = \varphi_{\xi^{(\iota)}}(T)p^{(\iota)}$.

%Fix $n \in \N$, $\chi : [n] \to \slr$ and $\iota : [n] \to \I$, and let $T_1, \ldots, T_n \in \A$ be such that $T_i \in \A_{\chi(i)}^{\iota(i)}$ for each $i \in [n]$.
%Now we will associate to every shaded $\LR$-diagram a vector in $V$; therefore, let $D = (n, \chi, \iota, \pi, S) \in \LR(\chi, \iota)$.
%Let $B \in \pi$, and let $\theta$ be such that $\iota(b) = \theta$ for $b \in B$.
Let $n \in \N$.
We define a map $\Xi$ which takes as arguments $B \subset [n]$ and $T_1, \ldots, T_n \in \bigcup_{\iota\in\I}L(V^{(\iota)})$ with the requirement that all operators corresponding to elements of $B$ come from a single $L^(V^{(\theta)})$, and produces a vector in $V^{(\theta)}$ as follows:
with $B = \set{k_1 < \cdots < k_{|B|}}$,
$$\Xi(B; T_1, \ldots, T_n) := T_{k_1}(1-p^{(\theta)})T_{k_2}(1-p^{(\theta)})T_{k_3}\cdots(1-p^{(\theta)})T_{k_{|B|}}\xi^{(\theta)}.$$

Given a map $\iota : [n] \to \I$, we say a sequence of operators $T_1, \ldots, T_n \in \bigcup_{\iota\in\I}L(V^{(\iota)})$ is consistent with $\iota$ if $T_i \in L(V^{(\iota)})$ for each $i \in [n]$.
We will now define a map $\Psi$ which takes as arguments a shaded $LR$-diagram $D = (n, \chi, \iota, \pi, S)$ and a sequence of operators $T_1, \ldots, T_n$ consistent with $\iota$, and produces an element of $V$.
%Each block $B$ in $\pi$ will contribute a scalar factor if $B \notin S$, and a vector factor if $B \in S$.
%In particular, if $S = \set{B_1, \ldots, B_k}$ with $B_1 \precc B_2 \precc\cdots\precc B_k$, then we will have
%$$\Psi(D; T_1, \ldots, T_n) \in \left\{\begin{array}{ll}\oV^{(\iota(B_1))} \otimes \cdots \otimes \oV^{(\iota(b_k))} & \text{ if } k > 0\\ \C\xi & \text{ if } k = 0\end{array}\right..$$
%Each block $B$ in $\pi \setminus S$ will contribute a scalar factor
%$$\lambda_B := \varphi_{\xi^{(\iota)}}\paren{\Xi(B; T_1, \ldots, T_n)}.$$
Let $S = \set{B_1, \ldots, B_k}$ with $B_1 \precc \cdots \precc B_k$.
Then
$$\Psi(D; T_1, \ldots, T_n) :=
\paren{\prod_{B \in \pi\setminus S}\psi^{(\iota)}\paren{\Xi(B; T_1, \ldots, T_n)}}
\paren{\bigotimes_{i=1}^k (1-p^{(\iota(B_i))})\Xi(B_i; T_1, \ldots, T_n)}.
$$
Here the tensor product should be interpreted as ordered from least $i$ to largest, and if $k = 0$, the empty tensor product factor should be read as the vector $\xi$.

\begin{example}
	Let $D = (8, \chi, \iota, \pi, S)$ be as in Example~\ref{ex:lrdiagram1}, and take $T_1, \ldots, T_8$ consistent with $\iota$.
	Then
	\begin{align*}
		\Psi(D; T_1, \ldots, T_8)
		&= \psi^{(\makeaball{0})}\paren{T_5(1-p^{(\makeaball{0})})T_6(1-p^{(\makeaball{0})})T_7(1-p^{(\makeaball{0})})T_8\xi^{(\makeaball{0})}}\\
		&\qquad\cdot(1-p^{(\makeaball{0})})T_1(1-p^{(\makeaball{0})})T_3\xi^{(\makeaball{0})}
		\otimes (1-p^{(\makeaball{1})})T_2(1-p^{(\makeaball{1})})T_4\xi^{(\makeaball{1})}.
	\end{align*}
\end{example}




\begin{proposition}
	Fix $\chi : [n] \to \slr$ and $\iota : [n] \to \I$.
	Let $T_1, \ldots, T_n$ be consistent with $\iota$.
	Then
	$$
	\pi_{\chi(1)}^{(\iota(1))}(T_1)\cdots \pi_{\chi(n)}^{(\iota(n))}(T_n)\xi
	= \sum_{D \in \LR(\chi, \iota)} \Psi(D; T_1, \ldots, T_n).
	$$
	Moreover,
	$$
	\varphi(\pi_{\chi(1)}^{(\iota(1))}(T_1)\cdots \pi_{\chi(n)}^{(\iota(n))}(T_n))
	= \sum_{\sigma \in \BNC(\chi)}\sq{\sum_{\substack{\pi \in \BNC(\chi, \iota) \\ \sigma \llateq \pi}} (-1)^{\abs{\pi}-\abs{\sigma}}}
	\varphi_\sigma(T_1, \ldots, T_n).
	$$
\end{proposition}

\begin{proof}
	We will establish the first identity by induction on $n$.
	The case $n = 0$ is immediate since if $D$ is the empty diagram, $\Xi(D;) = \xi$.
	Therefore suppose $n > 0$; our induction hypothesis applied to $T_2, \ldots, T_n$ tells us that, with $\hat\chi(j) = \chi(j+1)$ and $\hat\iota(j) = \iota(j+1)$ for $1 \leq j < n$,
	$$
	\pi_{\chi(2)}^{(\iota(2))}(T_2)\cdots \pi_{\chi(n)}^{(\iota(n))}\xi
	= \sum_{D \in \LR(\hat\chi, \hat\iota)} \Psi(D; T_2, \ldots, T_n).
	$$

	Fix $D = (n-1, \hat\chi, \hat\iota, \pi, S) \in \LR(\hat\chi, \hat\iota)$, and suppose for the moment that $\chi(1) = \ell$.
	Let $D_0, D_1 \in \BNC(\chi, \iota)$ be the two diagrams constructed from $D$, where $D_1$ has a spine extending from $1$ to the top and $D_0$ does not.

	Recall the bijections $W_j : V^{(j)}\otimes V(\ell, j) \to V$ from Subsection~\ref{ssec:freeind}.
	If $S$ is empty or its $\precc$-least element (i.e., the left-most spine reaching the top) is $B$ with $\iota(B) \neq \iota(1)$, then $w := \Psi(D; T_2, \ldots, T_n) \in V(\ell, \iota(1))$; hence
	\begin{align*}
		\pi_\ell^{(\iota(1))}(T_1)w
		&= W_j\paren{(T_1\otimes 1)(\xi^{(\iota(1))}\otimes w) }\\
		& = \psi^{(\iota(1))}(T_1) w + \paren{(1-p^{(\iota(1))}) T_1\xi^{(\iota(1))}} \otimes w \\
		& = \psi^{(\iota(1))}\paren{\Xi(\set{1}; T_1, \ldots, T_n)} w + \paren{(1-p^{(\iota(1))})\Xi(\set{1}; T_1, \ldots, T_n)}\otimes w \\
		& = \Psi(D_0; T_1, \ldots, T_n) + \Psi(D_1; T_1, \ldots, T_n).
	\end{align*}

	On the other hand, suppose the $\precc$-least element of $S$ is $\hat B$ with $\iota(\hat B) = \iota(1)$; let $B = \set{1}\cup\set{j+1 : j \in \hat B}$.
	Let $v := \Xi(\hat B; T_2, \ldots, T_n) \in \oV^{(\iota(1))}$, and let $w \in V(\ell, \iota(1))$ be so that $W_{(\iota(1))}(v\otimes w) = \Psi(D; T_2, \ldots, T_n)$.
	Since $v \in \oV{(\iota(1))}$, we have $(1-p^{(\iota(1))})v = v$.
	Then we have
	\begin{align*}
		\pi_\ell^{(\iota(1))}(T_1)\Psi(D; T_2, \ldots, T_n)
		&= W_j\paren{(T_1\otimes 1)(v \otimes w)} \\
		& = \psi^{(\iota(1))} (T_1(1-p^{(\iota(1))})v)w + (1-p^{(\iota(1))})T_1(1-p^{(\iota(1))})v \otimes w \\
		& = \psi^{(\iota(1))} \paren{\Xi(B; T_1, \ldots, T_n)}w + (1-p^{(\iota(1))})\Xi(B; T_1, \ldots, T_n)\otimes w \\
		& = \Psi(D_0; T_1, \ldots, T_n) + \Psi(D_1; T_1, \ldots, T_n).
	\end{align*}

	As each $D' \in \LR(\chi, \iota)$ arises from exactly one diagram in $\LR(\hat\chi, \hat\iota)$, we have
	$$
	\pi_{\ell}^{(\iota(1))}(T_1)\cdots \pi_{\chi(n)}^{(\iota(n))}\xi
	= \sum_{D \in \LR(\chi, \iota)} \Psi(D; T_1, \ldots, T_n).
	$$
	The argument for $\chi(1) = r$ is the same, except we use the decomposition $V \cong V(r, \iota(1)) \otimes V^{(\iota(1))}$ and work with the $\precc$-greatest element of $S$ (i.e., the furthest right spine which reaches the top).

	We now must establish the second claimed identity, which will follow from the first.
	Indeed, notice that
	$$\varphi(\pi_{\chi(1)}^{(\iota(1))}(T_1)\cdots \pi_{\chi(n)}^{(\iota(n))}(T_n)) = \psi(\pi_{\chi(1)}^{(\iota(1))}(T_1)\cdots \pi_{\chi(n)}^{(\iota(n))}(T_n)\xi).$$
	Applying $\psi$ to the first equation, we see that the only terms on the right hand side which survive correspond to diagrams in $\LR_0(\chi, \iota)$.

	Let $D = (n, \chi, \iota, \pi, \emptyset) \in \LR_0(\chi, \iota)$ be such a diagram, and take $B = \set{k_1 < \cdots < k_b} \in \pi$.
	Then
	\begin{align*}
		&\psi^{(\iota(B))}\paren{\Xi(B; T_1, \ldots, T_n)} \\
		&\qquad = \psi^{(\iota(B))}\paren{T_{k_1}(1-p^{(\theta)})T_{k_2}(1-p^{(\theta)})T_{k_3}\cdots(1-p^{(\theta)})T_{k_{|B|}}\xi^{(\theta)}} \\
		&\qquad = \sum_{m\geq 0} \sum_{1\leq q_1 < \cdots < q_m \leq b-1} (-1)^m\varphi_{\xi^{(\iota(B))}}\paren{T_{k_1}\cdots T_{k_{q_1}}}\cdots\varphi_{\xi^{(\iota(B))}}\paren{T_{k_{q_m+1}}\cdots T_{k_b}} \xi
	\end{align*}
	Each term in the sum corresponds to a lateral refinement of $\pi$ with cuts only in the block $B$, with sign determined by the parity of the number of cuts.
	Then there is a correspondence between lateral refinements of $\pi$ and terms in
	$$\prod_{B\in\pi} \paren{\sum_{m\geq 0} \sum_{1\leq q_1 < \cdots < q_m \leq b-1} (-1)^m\varphi_{\xi^{(\iota(B))}}\paren{T_{k_1}\cdots T_{k_{q_1}}}\cdots\varphi_{\xi^{(\iota(B))}}\paren{T_{k_{q_m+1}}\cdots T_{k_b}} \xi},$$
	and we conclude
	\begin{align*}
		&\psi\paren{\Psi(D; T_1, \ldots, T_n)} \\
		&\qquad = \prod_{B\in\pi} \paren{\sum_{m\geq 0} \sum_{1\leq q_1 < \cdots < q_m \leq b-1} (-1)^m\varphi_{\xi^{(\iota(B))}}\paren{T_{k_1}\cdots T_{k_{q_1}}}\cdots\varphi_{\xi^{(\iota(B))}}\paren{T_{k_{q_m+1}}\cdots T_{k_b}} \xi} \\
		&\qquad = \sum_{\substack{\sigma \in \BNC(\chi) \\ \sigma \llateq \pi}} (-1)^{\abs{\pi} - \abs{\sigma}}\varphi_\sigma(T_1, \ldots, T_n).
	\end{align*}
	Summing over $D \in \LR_0(\chi, \iota)$ (or equivalently, $\pi \in \BNC(\chi, \iota)$) and reversing the order of the summations yields the desired identity.
\end{proof}

\begin{corollary}
	\label{cor:bifreelats}
	Let $\fpf$ be a family of pairs of faces in a non-commutative probability space $(\A, \varphi)$.
	Then $\fpf$ are bi-freely independent if and only if for every $n \in \N$, $\chi : [n] \to \slr$, $\iota : [n] \to \I$, and $z_1, \ldots, z_n \in \A$ with $z_i \in \A_{\chi(i)}^{(\iota(i))}$, we have
	$$\varphi(z_1\cdots z_n)
	= \sum_{\sigma \in \BNC(\chi)}\sq{\sum_{\substack{\pi \in \BNC(\chi, \iota) \\ \sigma \llateq \pi}} (-1)^{\abs{\pi} - \abs{\sigma}}} \varphi_\sigma(z_1, \ldots, z_n).$$
\end{corollary}

\begin{proof}
	If $\fpf$ are bi-free, this follows from applying the previous proposition to the representation guaranteed by the definition of bi-freeness.

	Conversely, suppose that the above equation holds whenever appropriate.
	Let
	$$\mu = \stst_{\iota\in\I}\varphi^{(\iota)} : \st_{\iota\in\I} \paren{\A_{\ell}^{(\iota)} \st \A_r^{(\iota)}} \to \C$$
	be the state with respect to which $\fpf$ are bi-free in $\st_{\iota\in\I}\paren{\A_{\ell}^{(\iota)} \st \A_r^{(\iota)}}$ such that $\mu|_{\A_\ell^{(\iota)}\st\A_r^{(\iota)}} = \varphi^{(\iota)}$.
	Then we have
	\begin{align*}
		\varphi(z_1\cdots z_n)
		&= \sum_{\sigma \in \BNC(\chi)}\sq{\sum_{\substack{\pi \in \BNC(\chi, \iota) \\ \sigma \llateq \pi}} (-1)^{\abs{\pi} - \abs{\sigma}}} \varphi_\sigma(z_1, \ldots, z_n) \\
		&= \sum_{\sigma \in \BNC(\chi)}\sq{\sum_{\substack{\pi \in \BNC(\chi, \iota) \\ \sigma \llateq \pi}} (-1)^{\abs{\pi} - \abs{\sigma}}} \mu_\sigma(z_1, \ldots, z_n) \\
		&= \mu(z_1\cdots z_n).
	\end{align*}
	But then $\fpf$ are bi-free with respect to $\varphi$ since they are bi-free with respect to $\mu$.
\end{proof}



\section{The incident algebra on bi-non-crossing partitions.}
The main goal of this section is to develop the appropriate M\"obius function for bi-non-crossing partitions, with an eye towards applying it to the bi-free cumulant functionals.

\begin{definition}
	The incidence algebra on the lattice of bi-non-crossing partitions consists of the following set of functions:
	$$IA(\BNC) := \set{f : \coprod_{n\geq1}\coprod_{\chi : [n]\to\slr} \BNC(\chi)\times\BNC(\chi) \to \C \middle| f(\sigma, \pi) = 0 \text{ unless } \sigma \leq \pi}.$$ 
	Once again, we define the convolution product:
	$$(f\star g)(\sigma, \pi) := \sum_{\sigma \leq \rho \leq \pi} f(\sigma, \rho)g(\rho, \pi).$$
\end{definition}
It is immediately verified that the convolution is associative, essentially because
$$\sum_{\sigma \leq \rho \leq \pi}\sum_{\rho \leq \rho' \leq \pi} = \sum_{\sigma\leq\rho\leq\rho'\leq\pi} = \sum_{\sigma\leq\rho'\leq\pi}\sum_{\sigma\leq\rho\leq\rho'}.$$


\subsection{Multiplicative functions.}
We wish to demonstrate that the same kind of decomposition into products of full intervals holds in the bi-non-crossing case as in the non-crossing case, so that we can make sense of multiplicative functions.
However, we want this decomposition to be aware of more than just the lattice structure; it should be consistent with the choice of sides.
We adapt the decomposition of Proposition~\ref{prop:ncfactoring} from the free setting as follows.

Suppose that we have $\sigma \leq \pi \in \BNC(\chi)$.
We first split both partitions according to the blocks of $\pi$:
$$[\sigma, \pi] \longrightarrow \prod_{B \in \pi} [\sigma|_B, \set{B}],$$
where $\sigma|_B$ and $\set{B}$ are thought of as elements of $\BNC(\chi|_B)$; thus we will from here on assume $\pi = 1_\chi$.

Now, if possible, let $B = \set{i_1 \prec_\chi \cdots \prec_\chi i_k}$ be a block of $\sigma$ so that there is some $a \in [n]$ and $1 \leq j < k$ so that $i_j \prec_\chi a \prec_\chi i_{j+1}$.
 this case, let $I_1' = (i_j, i_{j+1})_{\prec_\chi}$, $I_0 = [n] \setminus I_1$, and $I_1 = I_1' \cup \set{\max(i_j, i_{j+1})}$; then take $\sigma_0$ and $\sigma_1'$ to be the sub-partitions of $\sigma$ corresponding to the sets $I_0$ and $I_1'$, and $\sigma_1 = \sigma_1' \cup \set{\max(i_j, i_{j+1})}$.
We then have
$$\sq{\sigma, 1_\chi} \longrightarrow \sq{\sigma_0, 1_{\chi|_{I_0}}} \times \sq{\sigma_1, 1_{\chi|_{I_1}}}.$$
If $\sigma_0 = 1_{\chi|_{I_0}}$, we exclude the left hand term above.

Lastly, if $\sigma$ consists entirely of $\chi$-intervals, we retract each interval to its lowest node.
That is, if $\sigma = \set{B_1, \ldots, B_m}$, $i_j = \max(B_j)$, and $i_1 < \cdots < i_m$, then taking $\chi'(j) = \chi(i_j)$ we have
$$[\sigma, 1_\chi] \longrightarrow [0_{\chi'}, 1_{\chi'}].$$

\begin{example}
	Letting a diagram below stand for the interval between it and the corresponding maximum partition,
	\[
		\begin{tikzpicture}
			\def\sdz{{0,-1,-1,-1,1,-1,1,1,-1,-1,-1,1,-1,1,1}}
			\def\labelz{{"","","","","","","","","","","","","","",""}}
			\def\clrz{{"black","black","black","black","black","black","black","black","black!.3","black!.3","black!.3","black!.3","black!.3","black!.3","black!.3"}}
			\begin{scope}[xshift=0cm]
				\def\ord{{1,2,3,4,5,6,7}}
				\bnc[n=7,sidez=\sdz,colourz=\clrz,labelz=\labelz,order=\ord]
				\foreach \y in {1,2,6} {\draw [thick] (ball\y) -| (ball6 -| 0.2,0);}
				\foreach \y in {5,7} {\draw [thick] (ball\y) -| (ball7 -| -0.2,0);}
				\coordinate (oldcr) at (cr);
			\end{scope}

			\begin{scope}[xshift=3cm]
				\def\ord{{1,2,0,4,0,6,0}}
				\bnc[n=7,sidez=\sdz,colourz=\clrz,labelz=\labelz,order=\ord]
				\foreach \y in {1,2,6} {\draw [thick] (ball\y) -| (ball6 -| 0.2,0);}
				\node at ($ (oldcr) ! .5 ! (cl) $) {$\cong$};
				\coordinate (oldcr) at (cr);
			\end{scope}

			\begin{scope}[xshift=6cm]
				\def\ord{{8,9,3,0,5,6,7}}
				\bnc[n=7,sidez=\sdz,colourz=\clrz,labelz=\labelz,order=\ord]
				\foreach \y in {5,7} {\draw [thick] (ball\y) -| (ball7 -| -0.2,0);}
				\foreach \y in {1,2,6} {\draw [thick,dotted] (ball\y) -| (ball6 -| 0.2,0);}
				\node at ($ (oldcr) ! .5 ! (cl) $) {$\times$};
				\coordinate (oldcr) at (cr);
			\end{scope}

			\begin{scope}[xshift=9cm]
				\def\ord{{8,9,0,4,0,6,0}}
				\bnc[n=7,sidez=\sdz,colourz=\clrz,labelz=\labelz,order=\ord]
				\node at ($ (oldcr) ! .5 ! (cl) $) {$\cong$};
				\coordinate (oldcr) at (cr);
				\node [below] (lbl) at (bc) {$[0_{(r,r)}, 1_{(r,r)}]$};
				\foreach \y in {1,2,6} {\draw [thick,dotted] (ball\y) -| (ball6 -| 0.2,0);}
			\end{scope}

			\begin{scope}[xshift=12cm]
				\def\ord{{8,9,3,0,12,6,7}}
				\bnc[n=7,sidez=\sdz,colourz=\clrz,labelz=\labelz,order=\ord]
				\node at ($ (oldcr) ! .5 ! (cl) $) {$\times$};
				\node [below] (lbl2) at (bc) {$[0_{(\ell,r,r)}, 1_{(\ell,r,r)}]$};
				\foreach \y in {1,2,6} {\draw [thick,dotted] (ball\y) -| (ball6 -| 0.2,0);}
				\foreach \y in {5,7} {\draw [thick,dotted] (ball\y) -| (ball7 -| -0.2,0);}
				\path (lbl) -- node {$\times$} (lbl2);
			\end{scope}
		\end{tikzpicture}
	\]
\end{example}


% \begin{proposition}
% 	Let $n \in \N$, $\chi : [n] \to \slr$, and $\sigma, \pi \in \BNC(\chi)$ with $\sigma \leq \pi$.
% 	Then there is $k \in \N$, $1 \leq m_1 \leq \cdots \leq m_k \leq n$, and functions $\beta_j : [m_j] \to \slr$ so that
% 	$$[\sigma, \pi] \cong \prod_{j=1}^k \BNC(\beta_j).$$
% 	Moreover, both $k$ and the sequence $m_1, \ldots, m_k$ up to adding additions $1$'s at the beginning.
% \end{proposition}
% 
% \begin{proof}
% 	This follows immediately once we recognize that our product is aware of no left-right structure, and $\BNC(\chi)$ is isomorphic \emph{as a lattice} to $\NC(n)$.
% 	Then we could decompose $\BNC(\chi) \cong \NC(m_1) \times\cdots\times\NC(m_k)$ and take $\beta_i \equiv \ell$.
% 	We really want something with a bit more structure, though\ldots
% 
% 	{\huge TODO: Note to self: we really want something stronger than a product of lattices here to account for chirality.}
% \end{proof}
\begin{proposition}
	The decomposition described above provides an isomorphism of lattices.
\end{proposition}

\begin{proof}
	We note that as lattices, $[\sigma, \pi] \cong [s_\chi\cdot\sigma, s_\chi\cdot\pi]$.
	After performing this identification, every step of the decomposition above corresponds to one in the decomposition of $[s_\chi\cdot\sigma, s_\chi\cdot\pi]$ in the lattice of non-crossing partitions, as in \cite{speicher1994}*{Proposition 1} (restated in Proposition~\ref{prop:ncfactoring}).
	All we have done is make choices of sides for the new nodes which are added, and for nodes obtained by contracting blocks, but we have done so in such a way as to make them consistent with the original lattice.
\end{proof}

\begin{definition}
	A function $f \in IA(\BNC)$ is said to be \emph{multiplicative} if whenever $\sigma, \pi \in \BNC(\chi)$ are such that $[\sigma, \pi]$ decomposes as
	$$\prod_{j=1}^m \BNC(\chi_j),$$
	it follows that
	$$f(\sigma, \pi) = \prod_{j=1}^m f(0_{\chi_j}, 1_{\chi_j}).$$
\end{definition}

\begin{lemma}
	The convolution of multiplicative functions is multiplicative.
\end{lemma}
The proof is identical to that of Proposition~\ref{prop:mcmism}.

\begin{definition}
	Mirroring the free case, we define some particular elements of $IA(\BNC)$:
	\[
		\delta_{\BNC}(\sigma, \pi) := \left\{\begin{array}{ll}1 & \text{ if } \sigma = \pi \\ 0 & \text{ otherwise}\end{array}\right.,
			\qquad\text{and}\qquad
		\zeta_{\BNC}(\sigma, \pi) := \left\{\begin{array}{ll}1 & \text{ if } \sigma \leq \pi \\ 0 & \text{ otherwise}\end{array}\right..
			\]
	We further define the \emph{M\"obius function} $\mu_{\BNC}$ by the relation $\mu_{\BNC} \star \zeta_{\BNC} = \delta_{\BNC}$.
\end{definition}

To check that $\zeta_{\BNC}$ is actually invertible, one may appeal to the fact that $\zeta_{\NC}$ is.
Indeed, if $\sigma, \pi \in \BNC(\chi)$, then $\zeta_{\BNC}(\sigma, \pi) = \zeta_{\NC}(s_\chi^{-1}\cdot\sigma, s_\chi^{-1}\cdot\pi)$ (as the lattice $\BNC(\chi)$ is isomorphic to $\NC(n)$, with isomorphism given by $s_\chi^{-1}$).
Hence if one defines $\mu_{\BNC}(\sigma, \pi) := \mu_{\NC}(s_\chi^{-1}\cdot\sigma, s_\chi^{-1}\cdot\pi)$ it follows that $\mu_{\BNC}\star\zeta_{\BNC} = \delta_{\BNC} = \zeta_{\BNC}\star\mu_{\BNC}$.

Further, notice that $\delta_{\BNC}$ is the convolutive identity, and $\delta_{\BNC}$ and $\zeta_{\BNC}$ are clearly multiplicative.
We also have $\mu_{\BNC}$ is multiplicative, and this may be verified by once again appealing to its relation to $\mu_{\NC}$.

\begin{remark}
	\label{rem:cumulantformula}
	Suppose that $a_\ell, a_r \in \A$ are elements in a non-commutative probability space.
	Let $m, k \in IA(\BNC)$ be the unique multiplicative functions with $m(0_\chi, 1_\chi) = \varphi(a_{\chi(1)}\cdots a_{\chi(n)})$ and $k(0_\chi, 1_\chi) = \kappa_\chi(a_{\chi(1)}, \ldots, a_{\chi(n)})$.
	Then we once again have the relation $m = k\star\zeta_{\BNC}$ and $k = m\star\mu_{\BNC}$.
	Likewise, we once again have
	$$
		\kappa_\pi(a_1, \ldots, a_n) = \sum_{\substack{\sigma \in \BNC(\chi) \\ \sigma \leq \pi}} \varphi_\sigma(a_1, \ldots, a_n) \mu_{\BNC}(\sigma, \pi).
	$$
\end{remark}


\section{Unifying bi-free independence.}
\label{sec:unibifree}
We are now ready to start the main thrust of our argument to relate the concepts of bi-free independence and combinatorial bi-free independence.

\subsection{Summation considerations.}
Our first goal is to put the condition from Corollary~\ref{cor:bifreelats} into a more convenient form.
Having introduced the M\"obius function for the lattice $\BNC$, we are now able to do so.

\begin{proposition}
	\label{prop:lattomob}
	Let $\chi : [n] \to \slr$ and $\iota : [n] \to \I$.
	Then for every $\sigma \in \BNC(\chi)$ such that $\sigma \leq \iota$,
	$$\sum_{\substack{\pi\in\BNC(\chi, \iota)\\\sigma\llateq\pi}} (-1)^{\abs{\pi}-\abs{\sigma}}
	= \sum_{\substack{\pi\in\BNC(\chi)\\\sigma\leq\pi\leq\iota}} \mu_{\BNC}(\sigma, \pi).$$
\end{proposition}

Our strategy is to first establish a simple case by appealing to free probability, and then reduce all other cases to it.

\begin{lemma}
	\label{lem:leftchiworks}
	The formula in Proposition~\ref{prop:lattomob} holds when $\chi \equiv \ell$.
\end{lemma}

\begin{proof}
	Notice that if $\chi\equiv\ell$, we have $\BNC(\chi) = \NC(n)$ and $\mu_{\BNC}|_{\BNC(\chi)\times\BNC(\chi)} = \mu_{\NC}|_{\NC(n)\times\NC(n)}$.
	Now if $(\A_\ell^{(\iota)})_{\iota\in\I}$ are taken to be free, we have for any $z_1, \ldots, z_n$ with $z_i \in \A_\ell^{(\iota(i))}$ that
	\begin{align*}
		\varphi(z_1\cdots z_n)
		&= \sum_{\pi\in\NC(n)}\kappa_\pi(z_1, \ldots, z_n) \\
		&= \sum_{\substack{\pi\in\NC(n)\\\pi\leq\iota}} \kappa_\pi(z_1, \ldots, z_n) \\
		&= \sum_{\substack{\pi\in\NC(n)\\\pi\leq\iota}} \sum_{\substack{\sigma\in\NC(n)\\\sigma\leq\pi}} \mu_{\NC}(\sigma, \pi)\varphi_\sigma(z_1, \ldots, z_n) \\
		&= \sum_{\substack{\sigma\in\NC(n)\\\sigma\leq\iota}} \paren{\sum_{\substack{\pi\in\NC(n)\\\sigma\leq\pi\leq\iota}} \mu_{\NC}(\sigma, \pi)} \varphi_\sigma(z_1, \ldots, z_n). \\
	\end{align*}
	On the other hand, $(\A_\ell^{(\iota)}, \C)_{\iota\in\I}$ are bi-free, so we find
	\begin{align*}
		\varphi(z_1\cdots z_n)
		&= \sum_{\sigma \in \BNC(\chi)}\sq{\sum_{\substack{\pi \in \BNC(\chi, \iota) \\ \sigma \llateq \pi}} (-1)^{\abs{\pi} - \abs{\sigma}}} \varphi_\sigma(z_1, \ldots, z_n) \\
		&= \sum_{\sigma \in \NC(n)}\sq{\sum_{\substack{\pi \in \BNC(\chi, \iota) \\ \sigma \llateq \pi}} (-1)^{\abs{\pi} - \abs{\sigma}}} \varphi_\sigma(z_1, \ldots, z_n). \\
	\end{align*}
	Now, fix $\pi \in \NC(n)$ with $\pi \leq \iota$, and construct variables $y_1, \ldots, y_n$ so that the appropriate collections are free and the only pure moments of these variables which do not vanish are those precisely corresponding to the blocks of $\pi$.
	Then only the term corresponding to $\pi$ will survive in each of the above sums, and we conclude
	$$
	\sum_{\substack{\pi\in\BNC(\chi)\\\sigma\leq\pi\leq\iota}} \mu_{\BNC}(\sigma, \pi)
	=
	\sum_{\substack{\pi\in\NC(n)\\\sigma\leq\pi\leq\iota}} \mu_{\NC}(\sigma, \pi)
	=
	\sum_{\substack{\pi \in \BNC(\chi, \iota) \\ \sigma \llateq \pi}} (-1)^{\abs{\pi} - \abs{\sigma}}.
	$$
\end{proof}

\begin{lemma}
	\label{lem:bifreeltor}
	Let $\chi, \vec\chi : [n] \to \slr$ be so that $\chi(j) = \vec\chi(j)$ for $1\leq j < n$, $\chi(n) = \ell$, and $\vec\chi(n) = r$.
	Take $\iota : [n] \to \I$ and let $\sigma \in \BNC(\chi)$ be such that $\sigma \leq \iota$; let $\vec\sigma \in \BNC\paren{\vec\chi}$ be the $\vec\chi$-non-crossing partition with the same blocks as $\sigma$.
	Then
	$$\sum_{\substack{\pi\in\BNC(\chi, \iota)\\\sigma\llateq\pi}} (-1)^{\abs\pi - \abs\sigma}
	= \sum_{\substack{\vec\pi \in \BNC(\vec\chi, \iota) \\ \vec\sigma \llateq \vec\pi}} (-1)^{\abs{\vec\pi}-\abs{\vec\sigma}},$$
	and
	$$\sum_{\substack{\pi \in \BNC(\chi) \\ \sigma \leq \pi \leq \iota}} \mu_{\BNC}(\sigma, \pi)
	= \sum_{\substack{\vec\pi \in \BNC(\vec\chi) \\ \vec\sigma \leq \vec\pi \leq \iota}} \mu_{\BNC}(\vec\sigma, \vec\pi).$$
\end{lemma}


\begin{proof}
	Notice that there is a correspondence between $\BNC(\chi)$ and $\BNC(\vec\chi)$ given by identifying elements which have the same block structure, and moreover this also holds for $\BNC(\chi, \iota)$ and $\BNC(\vec\chi, \iota)$.
	Moreover, this identification preserves the partial orderings $\leq$ and $\llateq$, as well as $\mu_{\BNC}$.
	Thus both identities hold.
\end{proof}


\begin{lemma}
	\label{lem:leftdownrightup}
	Let $\chi: [n] \to \slr$ and $k \in [n-1]$ be so that $\chi(k) = \ell$ and $\chi(k+1) = r$.
	Take $\iota : [n] \to \I$ and let $\sigma \in \BNC(\chi)$ be such that $\sigma \leq \iota$.
	Now, define $\chi' : [n] \to \slr$ and $\iota' : [n] \to \I$ as follows:
	$$
	\chi'(j) = \left\{\begin{array}{ll} \chi(k+1) & \text{ if } j = k \\
		\chi(k) & \text{ if } j = k + 1 \\
	\chi(j) & \text{ otherwise} \end{array}\right.,
	\qquad\text{ and }\qquad
	\iota'(j) = \left\{\begin{array}{ll} \iota(k+1) & \text{ if } j = k \\
		\iota(k) & \text{ if } j = k + 1 \\
	\iota(j) & \text{ otherwise} \end{array}\right..
	$$
	Let $\sigma' \in \BNC(\chi')$ be the bi-non-crossing partition obtained by interchanging $k$ and $k+1$ in $\sigma$ (note that $\sigma' \leq \iota'$).
	Then
	$$\sum_{\substack{\pi\in\BNC(\chi, \iota)\\\sigma\llateq\pi}} (-1)^{\abs\pi - \abs\sigma}
	= \sum_{\substack{\pi' \in \BNC(\chi', \iota') \\ \sigma' \llateq \pi'}} (-1)^{\abs{\pi'}-\abs{\sigma'}},$$
	and
	$$\sum_{\substack{\pi \in \BNC(\chi) \\ \sigma \leq \pi \leq \iota}} \mu_{\BNC}(\sigma, \pi)
	= \sum_{\substack{\pi' \in \BNC(\chi') \\ \sigma' \leq \pi' \leq \iota'}} \mu_{\BNC}(\sigma', \pi').$$
\end{lemma}

\begin{proof}
	Once again, observe that the correspondence between $\BNC(\chi)$ and $\BNC(\chi')$ obtained by exchanging $k$ and $k+1$ is a bijection which preserves the partial orders $\leq$ and $\llateq$, and leaves $\mu_{\BNC}$ invariant; this follows because $s_\chi = s_{\chi'}$.
	Moreover, $\pi \leq \iota$ if and only if $\pi' \leq \iota'$, so the second claimed equality holds.
	However, it is not necessarily the case that the bijection takes $\BNC(\chi, \iota)$ to $\BNC(\chi', \iota')$, as it may take a partition $\pi \in \BNC(\chi)$ which has no tangled blocks of the same colour to one which does, or vice versa.

	%	First we claim that the only way terms can appear on one side of the sum and not the other is when $k$ and $k+1$ are in distinct blocks of $\sigma$ and are not seperated.
	%	Indeed, notice that the only way a lateral refinement can cease to be lateral (or a non-lateral refinement can become lateral) is if it involves cutting between the nodes $k$ and $k+1$; for this to happen, it must be the case that $k \sim_\pi k+1$ and $k \nsim_\sigma k+1$.
	%	The second condition forces $k$ and $k+1$ to be disconnected in $\sigma$.
	%	If they are in seperated blocks in $\sigma$, they must be connected to all seperating blocks in $\pi$ as $\pi$ is $\chi$-non-crossing; but then $\sigma$ cannot be a lateral refinement of $\pi$.
	%	On the other hand, the only way that we may have $\pi \in \BNC(\chi, \iota)$ and $\pi' \notin \BNC(\chi', \iota')$ is if two blocks of the same colour become entangled in $\pi'$.
	%	As the only nodes which are changed are $k$ and $k+1$, this requires that $k$ and $k+1$ be of the same colour yet in distinct blocks of $\pi$ which are not seperated.
	Note that this correspondence only fails when either exactly one of $\sigma \llateq \pi$ and $\sigma'\llateq\pi'$ holds (which requires that $k \nsim_\sigma k+1$ and $k\sim_\pi k+1$, which ensures $\iota(k) = \iota(k+1)$) or exactly one of $\pi \in \BNC(\chi, \iota)$ and $\pi' \in \BNC(\chi', \iota')$ holds (which requires that blocks of the same colour are entangled in exactly one of $\pi$ and $\pi'$, so $k \nsim_\pi k+1$ but $\iota(k) = \iota(k+1)$).
	Thus the two sums agree unless $k \nsim_\sigma k+1$ and $\iota(k) = \iota(k+1)$.
	Therefore let us assume we are in this case.

	We will split the sum on the left hand side of the first equation into three sums: one over partitions where $k$ and $k+1$ are in separated blocks of $\pi$; one over terms with $k\sim_\pi k+1$; and one over the rest.
	$$
	\sum_{\substack{\pi\in\BNC(\chi, \iota)\\\sigma\llateq\pi}} (-1)^{\abs\pi - \abs\sigma}
	= \paren{\sum_{\substack{\pi\in\BNC(\chi, \iota)\\\sigma\llateq\pi \\ k, k+1 \text{ separated in } \pi}}
	+ \sum_{\substack{\pi\in\BNC(\chi, \iota)\\\sigma\llateq\pi \\ k \sim_\pi k+1}} 
	+ \sum_{\substack{\pi\in\BNC(\chi, \iota)\\\sigma\llateq\pi \\ k \nsim_\pi k+1 \text{ in unseparated blocks of } \pi}} } (-1)^{\abs\pi - \abs\sigma}.
	$$
	We now claim that the second and third sums cancel.
	Indeed, if $k$ and $k+1$ are in blocks which are distinct and not separated in $\pi$, then it must be the case that $\pi \leq \set{[1,k], [k+1, n]}$: $k$ may not be connected to a lower node, nor $k+1$ to a higher, or else they would be tangled and of the same colour; meanwhile, no other block may have nodes both less than and greater than $k$, or else it would separate the two.
	In particular, the partition $\rho$ obtained by joining the blocks containing $k$ and $k+1$ is still an element of $\BNC(\chi, \iota)$ and still has $\sigma\llateq\rho$ as the only additional cut required is a lateral one between $k$ and $k+1$.
	Further, it is clear that $\abs{\rho} = \abs{\pi}-1$ so their contributions to the sums cancel.
	As this correspondence is easily inverted, the contributions of the two sums vanish, and we are left with
	$$
	\sum_{\substack{\pi\in\BNC(\chi, \iota)\\\sigma\llateq\pi}} (-1)^{\abs\pi - \abs\sigma}
	= \sum_{\substack{\pi\in\BNC(\chi, \iota)\\\sigma\llateq\pi \\ k, k+1 \text{ separated in } \pi}} (-1)^{\abs\pi - \abs\sigma}.
	$$

	But now if $\pi \in \BNC(\chi, \iota)$ is such that $k$ and $k+1$ are in distinct separated blocks of $\pi$, then $k$ and $k+1$ are in distinct separated blocks of $\pi'$ and in particular none of the conditions we are concerned with can fail: $\sigma' \llateq\pi'$ as no cuts are necessary between $k$ and $k+1$, which is the only region where changes have been made; and no blocks have become entangled which were not before as the only blocks which have changed are separated.
	Likewise every $\pi' \in \BNC(\chi', \iota')$ with $k$ and $k+1$ in separated blocks comes from such a $\pi \in \BNC(\chi, \iota)$.
	We conclude
	\begin{align*}
		\sum_{\substack{\pi\in\BNC(\chi, \iota)\\\sigma\llateq\pi}} (-1)^{\abs\pi - \abs\sigma}
		&= \sum_{\substack{\pi\in\BNC(\chi, \iota)\\\sigma\llateq\pi \\ k, k+1 \text{ separated in } \pi}} (-1)^{\abs\pi - \abs\sigma} \\
		&= \sum_{\substack{\pi'\in\BNC(\chi', \iota')\\\sigma'\llateq\pi' \\ k, k+1 \text{ separated in } \pi'}} (-1)^{\abs{\pi'} - \abs{\sigma'}}
		= \sum_{\substack{\pi'\in\BNC(\chi', \iota')\\\sigma'\llateq\pi'}} (-1)^{\abs{\pi'} - \abs{\sigma'}}.
	\end{align*}

	%	%We will show that if exactly on of the conditions $\pi \in \BNC(\chi, \iota)$ and $\pi' \in \BNC(\chi', \iota')$ occurs, it must be the case that $\iota(k) = \iota(k+1)$, $k$ and $k+1$ are contained in distinct blocks $B_0, B_1 \subset \sigma$ (respectively), $B_0$ and $B_1$ are not seperated in $\sigma$.
	%
	%	%Let us examine the partitions $\pi \in \BNC(\chi)$ which may cause problems.
	%	%The only way that two blocks can become entangled or disentangled is if $k$ and $k+1$ are in different blocks $\pi$ which are not separated, while one of the blocks is contained in $[1, k+1]$ and the other is contained in $[k, n]$.
	%	%In particular, since $\sigma \llateq \pi$, issues can only arise when $k \nsim_\sigma k+1$, $\iota(k) = \iota(k+1)$.
	%%
	%%%	Let $S := \set{\pi \in \BNC(\chi, \iota) : \pi' \notin \BNC(\chi', \iota')}$, and $S' := \set{\pi \in \BNC(\chi)\setminus \BNC(\chi, \iota) : \pi' \in \BNC(\chi', \iota')}$.
	%%%	We then have
	%%%	$$\sum_{\substack{\pi\in\BNC(\chi, \iota)\\\sigma\llateq\pi}} (-1)^{\abs\pi - \abs\sigma}
	%%%	- \sum_{\substack{\pi' \in \BNC(\chi', \iota') \\ \sigma' \llateq \pi'}} (-1)^{\abs{\pi'}-\abs{\sigma'}}
	%%%	= \sum_{\substack{\pi\in S\\\sigma\llateq\pi}} (-1)^{\abs\pi - \abs\sigma}
	%%%	- \sum_{\substack{\pi' \in S' \\ \sigma' \llateq \pi'}} (-1)^{\abs{\pi'}-\abs{\sigma'}}.$$
	%%%	We claim that as subsets of $\cP(n)$, $S$ and $S'$ are equal, and that  so the right hand side vanishes.
	%%%	Indeed, suppose $\pi \in S$.
	%%%
	%%%	Notice that if $\pi \in S$, we must have that $k$ and $k+1$ are in distinct blocks $B_0, B_1 \in \pi$ which are not seperated so that $B_0 \subset [1, k]$ and $B_1 \subset [k+1, n]$, and $\iota(k) = \iota(k+1)$.
	%%%	Now $\pi \in \BNC(\chi')$ since 
	%%%
	%%%	Indeed, $\pi \in \BNC(\chi, \iota)$ ensures $\pi' \in \BNC(\chi')$ and $\pi' \leq \iota'$, so the only thing that can go wrong to prevent $\pi' \in \BNC(\chi', \iota')$ is that $\pi'$ has two tangled blocks of the same colour.
	%%%	Since the only blocks which may be tangled in $\pi'$ without being tangled in $\pi$ are those containing $k$ and $k+1$, we conclude that they must have been entangled by the operation.
	%%%
	%%	Let $S := \set{\pi \in \BNC(\chi, \iota) : \pi' \notin \BNC(\chi', \iota')}$, and $S' := \set{\pi \in \BNC(\chi)\setminus \BNC(\chi, \iota) : \pi' \in \BNC(\chi', \iota')}$.
	%%	We claim that as subsets of $\cP(n)$, $S$ and $S'$ are equal.
	%%	Indeed, suppose $\pi \in S$.
	%%	Since $\pi' \in \BNC(\chi')$ and $\pi' \leq \iota'$, the fact that $\pi' \notin \BNC(\chi', \iota')$ tells us that $\pi'$ must have two tangled blocks of the same colour.
	%%	But the only way blocks may become entangled is if $k, k+1$ are contained in distinct blocks $B_0, B_1 \in \pi$ which are neither seperated nor tangled.
	%%	(Not seperated so that they may become tangled in $\pi'$; not tangled since $\pi \in \BNC(\chi, \iota)$.)
	%%	In particular, this means that $\pi \leq \set{[1, k], [k+1, n]}$ and the action of switching the sides of $k$ and of $k+1$ cannot introduce any crossings, so $\pi \in \BNC(\chi')$.
	%%
	%
	%	Let
	%	\begin{align*}
	%		S &:= \set{\pi \in \BNC(\chi, \iota) : \pi' \notin \BNC(\chi', \iota')},\text{ and }\\
	%		S' &:= \set{\pi \in \BNC(\chi)\setminus \BNC(\chi, \iota) : \pi' \in \BNC(\chi', \iota')}.
	%	\end{align*}
	%	We claim that as subsets of $\cP(n)$, $S$ and $S'$ are equal.
	%	Indeed, suppose $\pi \in S$.
	%	Since $\pi' \in \BNC(\chi')$ and $\pi' \leq \iota'$, the fact that $\pi' \notin \BNC(\chi', \iota')$ tells us that $\pi'$ must have two tangled blocks of the same colour.
	%	But the only way blocks may become entangled is if $k, k+1$ are contained in distinct blocks $B_0, B_1 \in \pi$ which are neither seperated nor tangled.
	%	In particular, this means that $\pi \leq \set{[1, k], [k+1, n]}$ and the action of switching the sides of $k$ and of $k+1$ cannot introduce any crossings, so $\pi \in \BNC(\chi')$.
	%	Moreover, the fact that $\pi$ splits in this manner means that nothing becomes entangled when we reinterpret $\pi$ as an element of $\BNC(\chi')$; thus $\pi \in \BNC(\chi')$.
	%	Finally, it follows that the preimage of $\pi \in \BNC(\chi')$ in $\BNC(\chi)$, is equal as a partition to $\pi'$.
	%	But $\pi' \notin \BNC(\chi', \iota')$ as the blocks of $k$ and $k+1$ are tangled, yet this means also that $\pi' \notin \BNC(\chi, \iota)$.
	%	Hence $\pi \in S'$, and $S \subseteq S'$.
	%	That $S' \subseteq S$ follows from similar reasoning.
	%	\[
	%		\begin{tikzpicture}
	%			\foreach \block in {-1, 1} {
	%				\foreach \top in {-1, 1} {
	%					\begin{scope}[xshift=\block*\top*4cm, yshift=\block*2cm]
	%						\pgfmathtruncatemacro{\nom}{\block+1}
	%						\pgfmathtruncatemacro{\nomos}{1-\block}
	%						\draw [dashed, thick, name path=aaa\nom] (-1, 1.25) to[in=270, out = 270] (1,1.25);
	%						\draw [dashed, thick, name path=aaa\nomos] (-1, -1.25) to[in=90, out = 90] (1,-1.25);
	%						\foreach \x in {-1, 1} {
	%							\draw[thick,dotted] (\x, 1.25) -- ++(0, -2.5);
	%							\draw[thick] (\x, 0.75) -- ++(0, -1.5);
	%						}
	%						\node (ball1) [draw, shade, circle, ball color=black, inner sep=0.07cm] at (\top, 0.25) {};
	%						\node (ball2) [draw, shade, circle, ball color=black, inner sep=0.07cm] at (-\top, -0.25) {};
	%						\ifthenelse{\top=1}{\node[right]}{\node[left]} at (ball1) {$k$};
	%						\ifthenelse{\top=-1}{\node[right]}{\node[left]} at (ball2) {$k+1$};
	%
	%						\path [name path=lin\nom] (0.5, -1.25) -- (0.5, 1.25);
	%						\path [name path=lin\nomos] (-0.5, -1.25) -- (-0.5, 1.25);
	%						\draw [name intersections={of=aaa2 and lin0}, thick] (ball1) -- ++(-0.5*\top, 0) -- (intersection-1);
	%						\draw [name intersections={of=aaa0 and lin2}, thick] (ball2) -- ++(0.5*\top, 0) -- (intersection-1);
	%
	%						\coordinate (lbl\top\block) at (0,-1.5*\block);
	%					\end{scope}
	%				}
	%			}
	%
	%			\node[scale=0.7] at (lbl-1-1) {$\pi' \in \BNC(\chi)$};
	%			\node[scale=0.7] at (lbl1-1) {$\pi' \in \BNC(\chi')$};
	%			\node[scale=0.7] at (lbl-11) {$\pi \in \BNC(\chi, \iota)$};
	%			\node[scale=0.7] at (lbl11) {$\pi\in\BNC(\chi', \iota')$};
	%		\end{tikzpicture}
	%	\]
	%
	%	Lastly, we will show that if $\sigma\llateq\pi$ for some $\pi \in S$, then $\sigma'\llateq\pi$ as well.
	%	Indeed, as $k$ and $k+1$ are in different blocks of $\pi$, they must also be in different blocks of $\sigma$, hence of $\sigma'$.
	%	Then the manipulations to these two nodes 
\end{proof}

\begin{example}
	We take a moment here to show that the correspondence in the proof above can indeed fail to carry $\BNC(\chi, \iota)$ onto $\BNC(\chi', \iota')$, and so our proof is not more complicated than necessary.
	Let $\chi$ be given by the sequence $(\ell, \ell, r)$ and $\iota \equiv 1$.
	Then $\chi'$ corresponds to the sequence $(\ell, r, \ell)$.
	Yet if $\sigma = \set{\set{1, 2}, \set{3}}$ then $\sigma' = \set{\set{1,3}, \set{2}}$, and we have $\sigma \in \BNC(\chi, \iota)$ while $\sigma' \notin \BNC(\chi', \iota')$: the issue is that in $\sigma'$, $\set{2}$ is tangled with $\set{1,3}$, even though $\set{1,2}$ and $\set{3}$ were not tangled in $\sigma$.
	\[\begin{tikzpicture}
		\foreach \d in {-2, 2} {
			\begin{scope}[xshift=\d cm]
				\draw[thick] (-1,0.25) -- ++(0, -1.5) --node[below]{$\ifthenelse{\d=-2}{\phantom{\sigma'}\sigma\phantom{\sigma'}}{\sigma'}$} ++(2,0) -- ++(0,1.5);

				\def\sidez{{-1,\d/2,-\d/2}}
				\def\clrz{{0,0,0}}
				\foreach \y in {0,1,2} {
					\pgfmathtruncatemacro{\nodename}{\y+1}
					\pgfmathtruncatemacro{\sd}{\sidez[\y]}
					\pgfmathparse{\palette[\clrz[\y]]}
					\colorlet{clr}{\pgfmathresult}
					\node (ball\nodename) [draw, shade, circle, ball color=clr, inner sep=0.07cm] at (\sd, -\y*0.5) {};
					\ifthenelse{\sd=1}{\node[right] at (\sd, -\y*0.5) {\nodename}}
							{\node[left] at (\sd, -\y*0.5) {\nodename}};
				}

				\pgfmathparse{\palette[0]}
				\colorlet{clr}{\pgfmathresult}
				\pgfmathtruncatemacro{\conx}{(\d+10)/4}
				\pgfmathtruncatemacro{\disx}{(-\d+10)/4}
				\draw [clr, thick] (ball1) -- ++(1,0) |- (ball\conx);
				\draw [clr, thick] (ball\disx) -- ++(-0.3,0);
			\end{scope}
		}
	\end{tikzpicture}\]
	It is likewise easy to obtain examples of $\sigma \in \BNC(\chi)\setminus\BNC(\chi, \iota)$ with $\sigma' \in \BNC(\chi', \iota')$.
\end{example}


\begin{proof}[Proof of Proposition~\ref{prop:lattomob}]
	By Lemma~\ref{lem:leftchiworks}, we have that the equation holds whenever $\chi \equiv \ell$.
	By repeatedly applying Lemmata~\ref{lem:leftchiworks} and \ref{lem:bifreeltor}, we conclude that holds for all other choices of $\chi$ as well.
\end{proof}

\begin{corollary}
	\label{cor:bifreemob}
	Let $\fpf$ be a family of pairs of faces in a non-commutative probability space $(\A, \varphi)$.
	Then $\fpf$ are bi-freely independent if and only if for every $n \in \N$, $\chi : [n] \to \slr$, $\iota : [n] \to \I$, and $z_1, \ldots, z_n \in \A$ with $z_i \in \A_{\chi(i)}^{(\iota(i))}$, we have
	$$\varphi(z_1\cdots z_n)
	= \sum_{\sigma \in \BNC(\chi)}\sq{\sum_{\substack{\pi \in \BNC(\chi) \\ \sigma \leq \pi \leq \iota}} \mu_{\BNC}(\sigma, \pi)} \varphi_\sigma(z_1, \ldots, z_n).$$
\end{corollary}
\begin{proof}
	This follows immediately from Proposition~\ref{prop:lattomob} and Corollary~\ref{cor:bifreelats}.
\end{proof}

\subsection{Bi-free independence is equivalent to combinatorial bi-free independence.}
\begin{theorem}
	\label{thm:biequiv}
	Let $\fpf$ be a family of pairs of faces in a non-commutative probability space $(\A, \varphi)$.
	Then $\fpf$ are bi-freely independent if and only if they are combinatorially bi-freely independent.
\end{theorem}

\begin{proof}
	Suppose $\fpf$ are bi-free.
	We will show that all mixed bi-free cumulants vanish by induction on $n$, with the case $n=1$ being vacuously satisfied.
	Therefore fix $n\in\N$ and $\chi : [n] \to \slr$, and let $\iota : [n]\to\I$ be non-constant; we assume that all mixed cumulants with fewer than $n$ arguments vanish.
	Take $z_1, \ldots, z_n \in \A$ so that $z_i \in \A_{\chi(i)}^{(\iota(i))}$.
	By Corollary~\ref{cor:bifreemob}, we have
	\begin{align*}
		\varphi(z_1\cdots z_n)
		&= \sum_{\sigma \in \BNC(\chi)}\sq{\sum_{\substack{\pi \in \BNC(\chi) \\ \sigma \leq \pi \leq \iota}} \mu_{\BNC}(\sigma, \pi)} \varphi_\sigma(z_1, \ldots, z_n) \\
		&= \sum_{\substack{\pi \in \BNC(\chi) \\ \pi \leq \iota}} \sum_{\substack{\sigma\in \BNC(\chi) \\ \sigma \leq \pi}} \mu_{\BNC}(\sigma, \pi)\varphi_\sigma(z_1, \ldots, z_n) \\
		&= \sum_{\substack{\pi \in \BNC(\chi) \\ \pi \leq \iota}} \kappa_\pi(z_1, \ldots, z_n).
	\end{align*}
	On the other hand, using the moment-cumulant formula we have
	$$
	\varphi(z_1\cdots z_n)
	= \sum_{\pi\in\BNC(\chi)} \kappa_\pi(z_1, \ldots, z_n).
	$$
	But now any $\pi \in \BNC(\chi)$ with $\pi \nleq \iota$ and $|\pi| > 1$ is a product of the cumulants corresponding to its blocks, at least one of which must be mixed, and so $\kappa_\pi(z_1, \ldots, z_n) = 0$.
	Hence
	$$
	\varphi(z_1\cdots z_n)
	= \kappa_{1_\chi}(z_1, \ldots, z_n) + \sum_{\substack{\pi \in \BNC(\chi) \\ \pi \leq \iota}} \kappa_\pi(z_1, \ldots, z_n)
	= \kappa_{1_\chi}(z_1, \ldots, z_n) + \varphi(z_1\cdots z_n).
	$$
	We conclude that $\kappa_{1_\chi}(z_1, \ldots, z_n) = 0$.

	Now, for the other direction, suppose $\fpf$ are combinatorially bi-free.
	Once again, let $n \in \N$, $\chi : [n]\to\slr$, and $\iota: [n] \to \I$.
	Using successively the moment-cumulant formula, combinatorial bi-free independence, and the formula for computing cumulants from Remark~\ref{rem:cumulantformula}, we find:
	\begin{align*}
		\varphi(z_1\cdots z_n)
		&= \sum_{\pi \in \BNC(\chi)} \kappa_\pi(z_1, \ldots, z_n)
		= \sum_{\substack{\pi\in\BNC(\chi) \\ \pi \leq \iota}} \kappa_\pi(z_1, \ldots, z_n) \\
		&= \sum_{\substack{\pi\in\BNC(\chi) \\ \pi \leq \iota}} \sum_{\substack{\sigma\in\BNC(\chi) \\ \sigma\leq\pi}} \mu_{\BNC}(\sigma, \pi)\varphi_\sigma(z_1, \ldots, z_n) \\
		&= \sum_{\sigma \in \BNC(\chi)}\paren{\sum_{\substack{\pi \in \BNC(\chi) \\ \sigma\leq\pi\leq\iota}} \mu_{\BNC}(\sigma, \pi)} \varphi_\sigma(z_1, \ldots, z_n).
	\end{align*}
	Then by Corollary~\ref{cor:bifreemob}, $\fpf$ are bi-freely independent.
\end{proof}

\subsection{Voiculescu's universal bi-free polynomials.}
In \cite{voiculescu2014free}*{Section 5}, Voiculescu provided a non-constructive proof of universal polynomials characterising bi-free independence; using Theorem~\ref{thm:biequiv} we are able to produce them concretely.

We will first introduce some notation to ease things slightly.
Suppose $\paren{(z_i^{(\iota)})_{i\in I}, (z_j^{(\iota)})_{j\in J}}_{\iota\in\I}$ are pairs of two-faced families of non-commutative random variables, and suppose $\alpha : [n] \to I \coprod J$, $\iota : [n] \to \I$.
Then we write
$$\varphi_\alpha(z^{(1)}) := \varphi(z_{\alpha(1)}^{(1)}\cdots z_{\alpha(n)}^{(1)})
\qquad\text{and}\qquad
\varphi_\alpha(z^\iota) := \varphi(z_{\alpha(1)}^{(\iota(1))}\cdots z_{\alpha(n)}^{(\iota(n))}).$$
We also denote by $\chi_\alpha : [n] \to \slr$ the map such that $\chi(k) = \ell$ if and only if $\alpha(k) \in I$.

\begin{proposition}
	%%	Let $z' = \paren{(z_i')_{i \in I}, (z_j')_{j\in J}}$ and $z'' = \paren{(z_i'')_{i \in I}, (z_j'')_{j\in J}}$ denote a bi-free pair of two-faced families of non-commutative random variables in some non-commutative probability space $(\A, \varphi)$ and let $\alpha: [n] \to I\coprod J$, $\iota: [n] \to \set{', ''}$ be given.
	%%	Then there is a universal polynomial $P_{\alpha, \iota}$ so that
	%%	$$\varphi(z_{\alpha(1)}^{\iota(1)}\cdots z_{\alpha(n)}^{\iota(n)})
	%%	 = 
	%	For each $\a : [n] \to I \coprod J$ and $\iota : [n] \to \I$ we define polynomials $P_{\a, \iota}$ on indeterminates $X^{(\iota)}_K$ indexed by non-empty subsets $K \subseteq [n]$ via the formula
	%	$$P_{\a,\iota} := \sum_{\sigma\in\BNC\paren{\a, \iota}}\sq{\sum_{\substack{\pi\in\BNC(\chi_\a) \\ \sigma \leq \pi \leq \iota}} \mu_{\BNC}(\sigma, \pi)}\prod_{V \in \sigma}X^{(\iota(V))}_V,$$
	%	where $\chi_\a(k) = \ell$ if $\a(k) \in I$, and $r$ otherwise.
	%	Then if $\paren{(z_i^{(\iota)})_{i\in I}, (z_j^{(\iota)})_{j\in J}}_{\iota\in\I}$ is a collection of bi-free two-faced families of non-commutative random variables in some non-commutative probability space $(\A, \varphi)$, we have
	%	$\varphi(z_1^{\iota(1)}\cdots z_n^{(\iota(n))})$ is equal to $P_{\alpha, \iota}$ evaluated at
	%	$$X_{k_1 < \cdots < k_t}^{(a)} = \varphi(z_{k_1}^{(a)}\cdots z_{k_t}^{(a)}).$$
	%
	%	Furthermore, if $\I = \set{1,2}$ and
	%	$$Q_\alpha := \sum_{\iota : [n] \to \I} P_{\alpha, \iota},$$
	%	then
	%	$$Q_\alpha = X^{(1)}_{[n]} + X^{(2)}_{[n]} + \sum_{\substack{\iota : [n] \to \I \\ \iota \text{ non-constant}}} P_{\alpha, \iota},$$
	%	and
	%	$$\varphi\paren{(z^{(1)}_{\alpha(1)} + z^{(2)}_{\alpha(1)}\cdots(z^{(1)}_{\alpha(n)}+ z^{(2)}_{\alpha(n)})} Q(
	For each $\iota : [n] \to \I$ and $\alpha : [n] \to I\coprod J$ we define the polynomial $P_{\alpha,\iota}$ on indeterminates $X^{(\iota)}_K$ indexed by non-empty subsets $K\subset [n]$ by the formula
	$$
	P_{\alpha,\iota} := \sum_{\sigma\in \BNC(\chi_\alpha,\iota)} \sq{\sum_{\substack{\pi\in \BNC(\chi_\alpha)\\ \sigma\leq\pi\leq\iota}} \mu_{\BNC}(\sigma, \pi)} \prod_{V\in \sigma} X_V^{\iota(V)}.
	$$
	Then for $\paren{(z^{(\kappa)}_i)_{i\in I}, (z^{(\kappa)}_j)_{j\in J}}_{\kappa\in\I}$ a bi-free collection of two-faced families in $(\A,\varphi)$ we have
	\begin{align*}
		\varphi_\alpha(z^\iota) = P_{\alpha,\iota}((z^{(\kappa)})_{\kappa\in\I})
	\end{align*}
	where $P_{\alpha,\iota}((z^{(\kappa)})_{\kappa\in\I})$ is given by evaluating $P_{\alpha,\iota}$ at $X_{\set{k_1<\cdots<k_r}}^{\kappa}=\varphi(z_{\alpha(k_1)}^\kappa\cdots z_{\alpha(k_r)}^\kappa)$ for $\kappa \in \I$.

	Furthermore, if $\I = \set{1,2}$, if $((z_i)_{i\in I}, (z_j)_{j\in J})$ are given by $z_t = z_t^{(1)} + z_t^{(2)}$, and if
	$$Q_\alpha := \sum_{\iota : [n] \to \I} P_{\alpha, \iota},$$
	then
	$$Q_\alpha = X^{(1)}_{[n]} + X^{(2)}_{[n]} + \sum_{\substack{\iota : [n] \to \I \\ \iota \text{ non-constant}}} P_{\alpha, \iota},$$
	and
	$$\varphi_\alpha(z) = Q_\alpha(z^{(1)}, z^{(2)})$$
	where $Q_\alpha(z^{(1)},z^{(2)})$ is $Q_\alpha$ evaluated at the same point as the $P_{\alpha,\iota}$ above.
\end{proposition}

\begin{proof}
	The claim that $\varphi_\alpha(z^\iota)$ may be computed by evaluating $P_{\alpha, \iota}$ at the correct point is the content of Corollary~\ref{cor:bifreemob}.
	The claim that $\varphi_\alpha(z) = Q_\alpha(z^{(1)}, z^{(2)})$ is obtained by expanding
	$$\varphi_\alpha(z) = \varphi\paren{(z_{\alpha(1)}^{(1)} + z_{\alpha(1)}^{(2)})\cdots (z_{\alpha(n)}^{(1)}+z_{\alpha(n)}^{(2)})}
	= \sum_{\iota : [n] \to \I} \varphi_{\alpha}(z^\iota),$$
	and then applying the first claim.

	To establish the final piece of the proposition, that $Q_\alpha$ has the presentation claimed, it suffices to show that $P_{\alpha, \iota} = X_{[n]}^{(\iota)}$ when $\iota$ is constant.
	But in this case,
	$$\sum_{\substack{\pi\in\BNC(\chi_\alpha) \\ \sigma \leq \pi \leq \iota}} \mu_{\BNC}(\sigma, \pi)
	= \sum_{\pi \in \BNC(\chi_\alpha)} \mu_{\BNC}(\sigma, \pi) = \delta_{\BNC}(\sigma, 1),$$
	and we have that the only term surviving in $P_{\alpha, \iota}$ is $X_{[n]}^{(\iota)}$.
\end{proof}

\begin{proposition}
	For $\alpha : [n] \to I \coprod J$, recursively define polynomials $R_\alpha$ on indeterminates $X_K$ indexed by non-empty subsets $K\subseteq [n]$ by the formula
	$$
	R_\alpha = \sum_{\pi\in BNC(\chi_\alpha)} \mu_{\BNC}(\pi, 1) \prod_{V \in \pi} X_V.
	$$
	If $X_K$ is given degree $|K|$, then $R_\alpha$ is homogeneous with degree $n$.

	Given a two-faced family $((z_i)_{i\in I}, (z_j)_{j\in J})$ in a non-commutative probability space and setting $R_\alpha(z)$ to be $R_\alpha$ evaluated with $X_{\set{k_1<\cdots<k_r}} = \varphi(z_{\alpha(k_1)}\cdots z_{\alpha(k_n)})$, we have $R_\alpha(z) = \kappa_\alpha(z)$.
	Moreover, if $\paren{(z^{(\kappa)}_i)_{i\in I}, (z^{(\kappa)}_j)_{j\in J}}_{\kappa\in\set{1,2}}$ are bi-free, and as before $z_t = z_t^{(1)}+z_t^{(2)}$, we have $R_\alpha(z) = R_\alpha(z^{(1)}) + R_\alpha(z^{(2)})$; that is, $R_\alpha$ has the cumulant property of being additive over bi-free families.
\end{proposition}

\begin{proof}
	The identity for $R_\alpha$ follows directly from the formula in Remark~\ref{rem:cumulantformula}.
	The cumulant property then follows since $\kappa$ possesses it.
\end{proof}

The polynomials $P_{\alpha, \iota}$, $Q_\alpha$, and $R_\alpha$ are precisely the universal polynomials from Propositions 2.18, 5.2, 5.6, and 5.7 of \cite{voiculescu2014free}.


\section{An alternating moment condition for bi-free independence.}
So far we have still not presented a characterisation of bi-free independence as straightforward as the characterisation of free independence mentioned in Proposition~\ref{prop:freemomentsvanish}: that families are free if and only if alternating products of centred variables are centred.
In this section we will develop an analogous condition for bi-free independence.

\begin{definition}
	Let $\fpf$ be a family of pairs of faces in a non-commutative probability space $(\A, \varphi)$.
	We say the family has the \emph{vanishing alternating centred $\chi$-interval Eigenschaft} (which we will abbreviate as \emph{vaccine}) if whenever:
	\begin{itemize}
		\item $n \geq 1$, $\chi : [n] \to \slr$, $\iota : [n] \to \I$, and
		\item $z_1, \ldots, z_n \in \A$ are such that:
			\begin{itemize}
				\item $z_i \in \A_{\chi(i)}^{(\iota(i))}$; and
				\item whenever $\set{i_1 < \cdots < i_k}$ is a maximal $\iota$-monochromatic $\chi$-interval, $\varphi(z_{i_1}\cdots z_{i_k}) = 0$,
			\end{itemize}
	\end{itemize}
	it follows that $\varphi(z_1\cdots z_n) = 0$.
\end{definition}

\begin{example}
	\label{ex:vaccine}
	Suppose $\chi$ is such that $\chi^{-1}(\ell) = \set{2,5,6,7,8}$, $\chi^{-1}(r) = \set{1, 3, 4, 9, 10}$, and $\iota$ corresponds to the colouring below (i.e.,
	$\iota^{-1}\paren{\makeaball{0}} = \set{1,2,3,5,8,9,10}$ and
	$\iota^{-1}\paren{\makeaball{1}} = \set{4,6,7}$).
	\[\begin{tikzpicture}[baseline]
		\draw[thick] (-1,0.25) -- (-1, -4.75) -- (1,-4.75) -- (1,0.25);

		\def\colours{{0, 0, 0, 1, 0, 1, 1, 0, 0, 0}}
		\def\sidez{{1,-1,1,1,-1,-1,-1,1,-1,1}}
		\foreach \y in {0,...,9} {
			\pgfmathtruncatemacro{\nodename}{\y+1}
			\pgfmathtruncatemacro{\sd}{\sidez[\y]}
			\pgfmathparse{\palette[\colours[\y]]}
			\def\clr{\pgfmathresult}
			\node (ball\nodename) [draw, shade, circle, ball color=\clr, inner sep=0.07cm] at (\sd, -\y*0.5) {};
			\ifthenelse{\sd=1}{\node[right] at (\sd, -\y*0.5) {\nodename}}
					{\node[left] at (\sd, -\y*0.5) {\nodename}};
		}
		\draw[thick] (ball2)++(-0.45,0) -- ++(-0.2,0) |- ($ (ball5) + (-0.45,0) $);
		\draw[thick] (ball6)++(-0.45,0) -- ++(-0.2,0) |- ($ (ball7)+(-0.45,0) $);
		\draw[thick] (ball9)++(-0.45,0) -- ++(-0.2,0) |- (0,-5) -| ($ (ball8)+(0.65,0) $) -- ++(-0.2,0);
		\draw[thick] (ball4)++(0.45,0) -- ++(0.2,0);
		\draw[thick] (ball3)++(0.45,0) -- ++(0.2,0) |- ($ (ball1)+(0.45,0) $);
	\end{tikzpicture}\]
	The maximal $\iota$-monochromatic $\chi$-intervals are $\set{2, 5}, \set{6,7}, \set{8,9,10}, \set{4}$, and $\set{1,3}$.
	Vaccine would imply $\varphi(z_1\cdots z_{10}) = 0$ whenever $z_1, \ldots, z_{10}$ are chosen corresponding to $\chi$ and $\iota$ with $$0 = \varphi(z_2z_5) = \varphi(z_6z_7) = \varphi(z_8z_9z_{10}) = \varphi(z_4) = \varphi(z_1z_3).$$
	The reason this condition becomes more complicated than in the free case amounts to the fact that we cannot replace $z_1z_3$ or $z_8z_9z_{10}$ with single elements of either the left or right faces; in the latter case, because neither face may contain an appropriate operator, and in the former because $z_2$ may not commute with $z_3$ or $z_1$ and so the two may not be moved next to each other.
\end{example}

\subsection{The equivalence.}
\begin{lemma}
	\label{lem:bifreeimpliesvaccine}
	Let $\paren{(\A_\ell^{(\iota)}, \A_r^{(\iota)})}_{\iota \in \I}$ be a family of pairs of faces in a non-commutative probability space $(\A, \varphi)$.
	Then the family has vaccine{} if the pairs of faces are bi-free.
\end{lemma}

\begin{proof}
	Let $n \geq 1$, $\chi : [n] \to \slr$, and $\iota : [n] \to \I$, and denote by $\cJ$ the set of maximal $\iota$-monochromatic $\chi$-intervals in $[n]$.
	Note that $\cJ \in BNC(\chi)$ may be thought of as a $\chi$-non-crossing partition in its own right, which will sometimes be of use notationally.
	For $P \subset [n]$, let $m(P)$ denote the minimum element of $BNC(\chi)$ containing $P$ as a block, so all blocks of $m(P)$ except $P$ are singletons.
	Let $b : \set{\pi \in BNC(\chi) : \pi \leq \iota} \to \cJ$ be a function with the following properties:
	\begin{itemize}
		\item if $\pi \in BNC(\chi)$, $j \in b(\pi)$, and $j \sim_\pi k$, then $k \in b(\pi)$ (i.e., the interval $b(\pi)$ is isolated in $\pi$: $\pi \leq \set{b(\pi), b(\pi)^c}$); and
		\item if $\pi, \sigma \in BNC(\chi)$ satisfy $\pi\vee m(b(\pi)) = \sigma \vee m(b(\pi))$ then $b(\pi) = b(\sigma)$ (i.e., any partition obtained from $\pi$ by only modifying the part of $\pi$ in $b(\pi)$ is mapped to the same $\chi$-interval by $b$).
	\end{itemize}
	For example, one could take $b(\pi)$ to be the $\chi$-minimal element of $\cJ$ which is isolated in $\pi$.
	Any partition $\pi \in BNC(\chi)$ with $\pi \leq \iota$ must leave one element of $\cJ$ isolated; indeed, if one takes $\pi \vee \cJ \leq \iota$ (the element of $BNC(\chi)$ obtained from $\pi$ by joining all points lying in the same $\iota$-monochromatic $\chi$-intervals), it must contain a $\chi$-interval (as any non-crossing partition, in particular $s_\chi^{-1}\cdot (\pi\vee\cJ)$, must contain an interval) and since $\cJ \leq \pi \vee \cJ \leq \iota$, any interval it contains must be isolated and maximal $\iota$-monochromatic.
	Then this same interval is isolated in $\pi$.

	Denote $S(B) = \set{\pi \in BNC(\chi) : B \in \pi}$ the set of $\chi$-non-crossing partitions in which $B$ is a block.
	Note that if $\sigma \in S(B)$ and $\rho \in S(B^c)$, then $\sigma \wedge \rho$ is a partition with blocks under $B$ corresponding to $\rho$ and blocks outside of $B$ corresponding to $\sigma$.
	Further, any partition $\pi \in BNC(\chi)$ with $\pi \leq \set{B, B^c}$ may be expressed in this form: $\pi \vee m(B) \in S(B)$, $\pi \vee m(B^c) \in S(B^c)$, and $(\pi \vee m(B))\wedge(\pi\vee m(B^c)) = \pi$.

	Now, let $n \in \N$, $\chi : [n] \to \slr$, and $\iota : [n] \to \I$, and take $z_1, \ldots, z_n$ as in the definition of vaccine.
	Using the moment-cumulant formula and the vanishing of mixed cumulants from bi-freeness, we have
	\begin{align*}
		\varphi(z_1\cdots z_n)
		&= \sum_{\substack{\pi \in BNC(\chi)\\ \pi \leq \iota}} \kappa_\pi(z_1, \ldots, z_n)\\
		&= \sum_{B \in \cJ} \sum_{\pi \in b^{-1}(B)} \kappa_\pi(z_1, \ldots, z_n)\\
		&= \sum_{B \in \cJ} \sum_{\substack{\sigma \in S(B)\\ b(\sigma) = B}}\sum_{\rho \in S(B^c)} \kappa_{\sigma\wedge\rho}(z_1, \ldots, z_n)\\
		&= \sum_{B \in \cJ} \sum_{\substack{\sigma \in S(B)\\ b(\sigma) = B}}\paren{\sum_{\rho \in BNC(\chi|_B)}\kappa_\rho(z_1, \ldots, z_n)} \kappa_{\sigma\setminus \set{B}}(z_1, \ldots, z_n)\\
		&= \sum_{B \in \cJ} \sum_{\substack{\sigma \in S(B)\\ b(\sigma) = B}}\varphi(z_B) \kappa_{\sigma\setminus \set{B}}(z_1, \ldots, z_n)\\
		&= 0.
	\end{align*}
	Essentially, in the sum of cumulants representing $\varphi(z_1\cdots z_n)$, we have grouped terms together by isolated intervals, and used the fact that when we sum over the entire lattice of bi-non-crossing partitions over one of these intervals, we recover the moment corresponding to that interval, which is zero by assumption.
\end{proof}
We remark that a bi-free family enjoys an even stronger property: one may drop the requirement that the first and last $\chi$-intervals be centred, provided that there are at least three maximal $\iota$-monochromatic $\chi$-intervals.
Our proof above hinged on being able to find an isolated maximal $\chi$-interval in any $\chi$-non-crossing partition; but notice that not only does such an interval always exist, there is always one which is neither the first nor last interval.

\begin{lemma}
	\label{lem:vaccineimpliesbifree}
	Let $\paren{(\A_\ell^{(\iota)}, \A_r^{(\iota)})}_{\iota \in \I}$ be a family of pairs of faces in a non-commutative probability space $(\A, \varphi)$.
	Then the pairs of faces are bi-free if the family has vaccine{}.
\end{lemma}

\begin{proof}
	We will show that vaccine uniquely specifies mixed moments in terms of pure ones.
	Let $n \geq 1$, $\chi : [n] \to \slr$, $\iota : [n] \to \I$, and suppose $z_1, \ldots, z_n \in \A$ with $z_i \in \A_{\chi(i)}^{(\iota(i))}$.
	Denote by $\cJ$ the set of maximal $\iota$-monochromatic $\chi$-intervals in $[n]$.

	For each $I = \set{a_1 < \cdots < a_j} \in \cJ$, let $\lambda_I$ be a (complex) root of the polynomial $\varphi\paren{(z_{a_1} - w)\cdots (z_{a_j} - w)}$.
	Then if $f : [n] \to \cJ$ is the unique map so that $i \in f(i)$ for every $i$, we have
	$$\varphi\paren{(z_1 - \lambda_{f(1)})\cdots(z_n-\lambda_{f(n)})} = 0,$$
	as the $\lambda$'s were chosen precisely to make the vaccine property apply.
	Expanding this equation gives us an expression for $\varphi(z_1\cdots z_n)$ in terms of mixed moments with at most $n-1$ terms; by recursively applying the same procedure we find an expression for $\varphi(z_1\cdots z_n)$ in terms of pure moments.

	Now, for $\iota \in \I$ let $\varphi^{(\iota)}$ be the restriction of $\varphi$ to $\ang{\A_\ell^{(\iota)}, \A_r^{(\iota)}}$, and $\mu = \substack{**\\ \iota \in \I}\varphi^{(\iota)}$ their bi-free product distribution, which by Lemma~\ref{lem:bifreeimpliesvaccine} also has vaccine.
	We then find that the same expressions for joint moments in terms of pure ones hold under $\mu$ as under $\varphi$, which is to say that $\paren{(\A_\ell^{(\iota)}, \A_r^{(\iota)}}_{\iota\in\I}$ are bi-free.
\end{proof}

\begin{theorem}
	Let $\paren{(\A_\ell^{(\iota)}, \A_r^{(\iota)})}_{\iota \in \I}$ be a family of pairs of faces in a non-commutative probability space $(\A, \varphi)$.
	Then the family has vaccine{} if and only if the pairs of faces are bi-free.
\end{theorem}


\subsection{Conditional bi-freeness.}
\label{ss:condbifree}
We remark that this condition may be extended to other settings in bi-free probability.
Conditional bi-freeness was studied by Gu and Skoufranis in \cite{gu2016conditionally}, building off of conditional free independence which was introduced by Bo\.zejko, Leinert, and Speicher \cite{bozejko1996convolution}.
We will show that conditional bi-free independence also admits a characterization in terms of $\chi$-intervals.
First, though, we take the time to introduce some notation.

Suppose that $\pi \in BNC(\chi)$.
A a block $B \in \pi$ is said to be \emph{inner} if there is another block $C \in \pi$ and $j, k \in C$ so that for every $i \in B$, $j \prec_\chi i \prec_\chi k$; a block which is not inner is said to be \emph{outer}.
This corresponds with our earlier use of the term ``outer'' in Subsection~\ref{ss:shadedlrdiagrams}.

Let $(\A, \varphi)$ be a non-commutative probability space, and $\theta$ a state on $\A$.
The conditional cumulants with respect to $(\theta, \varphi)$ are multi-linear functionals $\cK_\chi : \A^n \to \C$ defined by the requirement that for any $z_1, \ldots, z_n \in \A$,
$$\theta(z_1\cdots z_n) = \sum_{\pi\in BNC(\chi)}\paren{\prod_{\substack{V\in\pi\\V \text{ inner}}} \kappa_{\chi|_V}\paren{(z_1, \ldots, z_n)|_V}}\paren{\prod_{\substack{V\in\pi\\V\text{ outer}}}\cK_{\chi|_V}\paren{(z_1, \ldots, z_n)|_V}}.$$
Here $\kappa$ represents the usual bi-free cumulants taken with respect to $\varphi$.
For $\pi \in BNC(\chi)$, we will denote by $\cK_\pi(z_1, \ldots, z_n)$ the term in the above sum corresponding to $\pi$, a product of $\kappa_{\chi|_V}$ terms and $\cK_{\chi|_V}$ terms.
We will say a family $\paren{\A_\ell^{(\iota)}, \A_r^{(\iota)}}_{\iota\in\I}$ is \emph{conditionally bi-free} in $(\A, \theta, \varphi)$ if it is bi-free with respect to $\varphi$ and all mixed conditional cumulants vanish; it was shown in \cite{gu2016conditionally} that this is equivalent to their definition in terms of free product representations, and moreover, that being conditionally bi-free uniquely specifies the mixed $\theta$-moments in terms of the pure $\theta$-moments and $\varphi$-moments.


\begin{theorem}
	Let $(\A, \varphi)$ be a non-commutative probability space and $\theta : \A \to \C$ a state on $\A$.
	Suppose $\paren{\A_\ell^{(\iota)}, \A_r^{(\iota)}}_{\iota\in\I}$ is a family of pairs of faces in $\A$.
	Then the family is conditionally bi-free if and only if whenever:
	\begin{itemize}
		\item $n \geq 1$, $\chi : [n] \to \set{\ell, r}$, $\iota : [n] \to \I$,
		\item $\cJ$ is the set of maximal $\iota$-monochromatic $\chi$-intervals, and
		\item $z_1, \ldots, z_n \in \A$ are such that:
			\begin{itemize}
				\item $z_i \in \A_{\chi(i)}^{(\iota(i))}$; and
				\item $\varphi(z_J) = 0$ for each $J \in \cJ$
			\end{itemize}
	\end{itemize}
	it follows that
	$$\varphi(z_1\cdots z_n) = 0 \qquad\text{and}\qquad \theta(a_1\cdots a_n) = \prod_{J \in \cJ} \theta(z_J).$$
\end{theorem}

\begin{proof}
	By the same argument as in the proof of Lemma~\ref{lem:vaccineimpliesbifree} it follows that the conditions assumed above suffice to uniquely specify all mixed $\varphi$- and $\theta$-moments in terms of pure $\varphi$- and $\theta$-moments; hence if we can show that conditionally bi-free families satisfy this condition the proof will be complete.
	The condition on $\varphi$ is precisely vaccine, so we need only show that our expression for mixed $\theta$-moments is correct.

	We take an approach similar to that of Lemma~\ref{lem:bifreeimpliesvaccine} for deducing the value of $\theta$.
	We claim that the only terms which contribute to the value of $\theta$ in the cumulant expansion are those corresponding to partitions $\pi < \cJ$, where once again $\cJ$ is the set of maximal $\iota$-monochromatic $\chi$-intervals.
	Towards this end, let $b : \set{\pi \in BNC(\chi) : \pi \leq \iota} \to \cJ$ be as in Lemma~\ref{lem:bifreeimpliesvaccine}, with the additional constraint that $b$ picks interior intervals whenever $\pi \nleq \cJ$.
	That is, $b$ should have the following properties:
	\begin{itemize}
		\item if $\pi \in BNC(\chi)$, $j \in b(\pi)$, and $j \sim_\pi k$, then $k \in b(\pi)$ (i.e., the interval $b(\pi)$ is isolated in $\pi$: $\pi \leq \set{b(\pi), b(\pi)^c}$)
		\item if $\pi, \sigma \in BNC(\chi)$ satisfy $\pi\vee m(b(\pi)) = \sigma \vee m(b(\pi))$ then $b(\pi) = b(\sigma)$ (i.e., any partition obtained from $\pi$ by only modifying the part of $\pi$ in $b(\pi)$ is mapped to the same $\chi$-interval by $b$); and
		\item if $\pi \in BNC(\chi)$ and $\pi \nleq \cJ$, then $b(\pi)$ is an inner block in $\pi \vee \cJ$.
	\end{itemize}
	Such functions exist: for example, one could take $b(\pi)$ to be the $\chi$-minimal element of $\cJ$ which is inner and isolated in $\pi$, if such exists, and the $\chi$-minimal element of $\cJ$ otherwise.
	Note that if $\pi \nleq \cJ$, $\pi$ must connect two intervals in $\cJ$ and so there must be an inner block in $\cJ\vee\pi$ between these two intervals.
	As before, let $S(B) = \set{\pi\in BNC(\chi) : B \in \pi}$, and set $S_i(B) = \set{\pi\in S(B) : B \text{ inner in } \pi\vee\cJ}$.
	We now compute much as in the proof of Lemma~\ref{lem:bifreeimpliesvaccine}.
	Supposing $n \in \N$, $\chi : [n] \to \slr$, and $\iota : [n] \to \I$, and with $z_1, \ldots, z_n$ meeting the hypotheses of the lemma:
	\begin{align*}
		\theta(z_1\cdots z_n)
		&= \sum_{\substack{\pi\in BNC(\chi)\\ \pi\leq \iota}} \cK_\pi(z_1, \ldots, z_n)\\
		&= \sum_{B \in \cJ}\paren{\sum_{\substack{\pi\in b^{-1}(B)\\ B \text{ inner in } \pi\vee\cJ}}\cK_\pi(z_1, \ldots, z_n) + \sum_{\substack{\pi\in b^{-1}(B)\\ B \text{ outer in } \pi\vee\cJ}}\cK_\pi(z_1, \ldots, z_n)}\\
		&= \sum_{B\in\cJ} \paren{\sum_{\pi \in S_i(B) \cap b^{-1}(B)} \varphi(z_B)\cK_{\pi\setminus\set{B}}(z_1, \ldots, z_n) + \sum_{\substack{\pi\in b^{-1}(B)\\ B \text{ outer in } \pi\vee\cJ}}\cK_\pi(z_1, \ldots, z_n)}\\
		&= \sum_{\substack{\pi\in b^{-1}(B)\\ B \text{ outer in } \pi\vee\cJ}}\cK_\pi(z_1, \ldots, z_n)\\
		&= \sum_{\substack{\pi \in BNC(\chi)\\\pi\leq\cJ}} \cK_\pi(z_1, \ldots, z_n)\\
		&= \prod_{J\in\cJ}\sum_{\pi_J \in BNC(\chi|_J)} \cK_{\pi_J}\paren{(z_1, \ldots, z_n)|_J}\\
		&= \prod_{J\in\cJ}\theta(z_J).
	\end{align*}
	Here in the last few lines we have noted that summing over all partitions sitting under $\cJ$ is the same as summing over partitions sitting under each interval individually, and then taking the product; this is valid since every term in $\cK_\pi$ is a product of terms corresponding to blocks, and each block must be contained in a single interval in $\cJ$.
\end{proof}


\section{Bi-free multiplicative convolution.}
\label{sec:bimultconv}
Much as in the free setting, we can view use bi-free probability to describe a multiplicative convolution on laws.
\begin{definition}
	If $\mu_\iota : \C\ang{X^{(\iota)}_\ell, X^{(\iota)}_r} \to \C$ are states for $\iota \in \set{1,2}$, there is a a unique state $\mu : \C\ang{X_\ell^{(1)}, X_\ell^{(2)}, X_r^{(1)}, X_r^{(2)}} \to \C$ so that $\mu$ restricted to the algebra generated by $X_\ell^{(\iota)}, X_r^{(\iota)}$ is equal to $\mu_\iota$ and the pairs $(X_{\ell}^{(\iota)}, X_r^{(\iota)})$ are bi-free.
	This allows us to define $\mu_1 \boxtimes\boxtimes_K \mu_2 : \C\ang{X_\ell, X_r} \to \C$ via $$\mu_1\boxtimes\boxtimes_K \mu_2\paren{f(X_\ell, X_r)} = \mu\paren{f(X_\ell^{(1)}X_\ell^{(2)}, X_r^{(2)}X_r^{(1)})}.$$
\end{definition}
There is of course a second option for defining the multiplicative convolution, which would be to sue $X_r^{(1)}X_r^{(2)}$ in the second coordinate; the corresponding operation is denoted by $\boxtimes\boxtimes$ and has been studied by Voiculescu in \cite{voiculescu2016freeiii} and Skoufranis in \cite{skoufranis2016combinatorial}.
As we will see, though, the operation $\boxtimes\boxtimes_K$ fits more naturally with the combinatorics of bi-free independence.


% %We remark here that in certain cases the convolution $\boxtimes\boxtimes_K$ takes laws of commutative random variables to commutative random variables.
% %In particular, if $X_{\chi}^{(1)}$ and $X_{\chi}^{(2)}$ are either both positive random variables, or both take values on the unit circle $\T$ (and so correspond to unitary operators) then the variable 
% Before we delve into our examination of the law of $(X_\ell^{(1)}X_\ell^{(2)}, X_r^{(2)}X_r^{(1)})$, we remark that it is much more difficult to have its distribution corresopnd to the law of a (commutative) random variable.
% Indeed, to use the techniques from the free setting, we require a much more restrictive assumption on the distribution $\mu_1$ and $\mu_2$.
% \begin{proposition}
% 	Suppose that $\mu_\iota : \C[X_\ell^{(\iota)}, X_r^{(\iota)}] \to \C$ are laws of pairs of (commutative) random variables, such that $X_\ell^{(\iota)}, X_r^{(\iota)} \geq 0$ and $\mu_\iota\paren{(X_\ell^{(\iota)})^{1/2} (X_r^{(\iota)})^{1/2} }^2 = \mu_\iota\paren{X_\ell^{(\iota)}}\mu_\iota\paren{X_r^{(\iota)}}$ for $\iota \in \set{1,2}$.
% 	Then $\mu_1\boxtimes\boxtimes_K\mu_2$ is the law of a pair of (commutative) positive random variables.
% \end{proposition}
% 
% \begin{proof}
% %	Write $\mu = \mu_1\boxtimes\boxtimes_K\mu_2 : \C\ang{X^{(1)}_\ell X^{(2)}_\ell, X^{(2)}_r X^{(1)}_r} \to \C$.
% %	Using the same techniques as in free probability, we know that $X^{(1)}_\ell X^{(2)}_\ell \sim Y_\ell := (X^{(1)}_\ell)^{1/2}X^{(2)}_\ell(X^{(1)}_\ell)^{1/2}$ and $X^{(2)}_rX^{(1)}_r \sim Y_r := (X^{(1)}_r)^{1/2}X^{(2)}_r(X^{(1)}_r)^{1/2}$ individually in distribution; we wish to establish that this also does not effect the distribution of the two variables taken together.
% %	Note that by bi-freeness and our assumptions on the individual families, both left variables commute with both right variables, and so $[Y_\ell, Y_r] = 0$.
% %	Now given any monomial $p$ in $Y_\ell$ and $Y_r$, we can use this commutativity to re-write it as follows:
% %	$$p = Y_\ell^kY_r^{k'} = (X^{(1)}_\ell)^{1/2}(X_\ell^{(2)}X_\ell^{(1)})^k (X_\ell^{(1)})^{1/2} (X_r^{(1)})^{1/2} (X_r^{(2)}X_r^{(1)})^{k'} (X_r^{(1)})^{1/2}.$$
% %	Let $\chi = (\ell, \ell, \ell, r, r, r)$; we then compute
% %	$$\varphi(p) = \sum_{\pi \in \BNC(\chi)}\kappa_\pi\paren{ (X^{(1)}_\ell)^{1/2}, (X_\ell^{(2)}X_\ell^{(1)})^k, (X^{(1)}_\ell)^{1/2}, (X^{(1)}_r)^{1/2}, (X_r^{(2)}X_r^{(1)})^{1/2}, (X^{(1)}_r)^{1/2} }.$$
% %	.......
% 	Let $\mu : \C\ang{X_\ell^{(1)}, X_\ell^{(2)}, X_r^{(1)}, X_r^{(2)}} \to \C$ be the state so that $\mu = \mu_\iota$ when restricted to $\C\ang{X_\ell^{(\iota)},X_r^{(\iota)}}$ which makes the two families free, so that $\mu|_{\C\ang{X^{(1)}_\ell X^{(2)}_\ell, X^{(2)}_r X^{(1)}_r}} = \mu_1\boxtimes\boxtimes_K \mu_2$.
% 	Note that because we have assumed that both left variables commute with both right variables, it suffices to describe all moments of the form $\mu\paren{(X^{(1)}_\ell X^{(2)}_\ell)^k (X^{(2)}_r X^{(1)}_r)^{k'}}$.
% 	We will show that the pair $(X^{(1)}_\ell X^{(2)}_\ell, X^{(2)}_r X^{(1)}_r)$ has the same law as the pair $(Y_\ell, Y_r)$ where $Y_\ell := (X^{(1)}_\ell)^{1/2}X^{(2)}_\ell(X^{(1)}_\ell)^{1/2}$ and $Y_r := (X^{(1)}_r)^{1/2}X^{(2)}_r(X^{(1)}_r)^{1/2}$; note that from the free case it follows that the moments agree for $k = 0$ or $k' = 0$.
% 	Therefore considering a moment with $k, k' > 0$, since each monochromatic interval consists of a single variable, we find by Theorem~\ref{lem:bifreeimpliesvaccine} that
% 	\begin{align*}
% 		\mu\paren{(X^{(1)}_\ell X^{(2)}_\ell)^k (X^{(2)}_r X^{(1)}_r)^{k'}}
% 		&= \paren{\mu(X^{(1)}_\ell)\mu(X^{(2)}_\ell)}^k \paren{\mu(X^{(2)}_r)\mu(X^{(1)}_r)}^{k'} \\
% 		&= \paren{\mu(X^{(1)}_\ell)\mu(X^{(2)}_\ell)}^{k-1} \paren{\mu(X^{(2)}_r)\mu(X^{(1)}_r)}^{k'-1} \mu\paren{(X_\ell^{(1)})^{1/2} (X_r^{(1)})^{1/2} }^2.
% 	\end{align*}
% 	But applying Theorem~\ref{lem:bifreeimpliesvaccine} (and the remark following its proof) a second time, we find that
% 	$$ \mu\paren{Y_\ell^k Y_r^{k'}}
% 	= \paren{\mu(X^{(1)}_\ell)\mu(X^{(2)}_\ell)}^{k-1} \paren{\mu(X^{(2)}_r)\mu(X^{(1)}_r)}^{k'-1} \mu\paren{(X_\ell^{(1)})^{1/2} (X_r^{(1)})^{1/2} }^2.$$
% 	Thus the two agree.
% 	But now $(Y_\ell, Y_r)$ is a pair of positive (hence self-adjoint) commuting operators and so their law corresponds to a measure on their spectrum.
% \end{proof}

\subsection{The Kreweras complement.}
To describe the law $\mu_1\boxtimes\boxtimes_K\mu_2$, we will adapt the Kreweras complement approach of Nica and Speicher in \cite{nicaspeicher1997} to our setting.

\begin{definition}
	Let $\pi \in \BNC(\chi)$.
	Then the \emph{Kreweras complement of $\pi$} is the $\chi$-non-crossing partition $K_{\BNC}(\pi)$ defined by
	$$K_{\BNC}(\pi) := s_\chi \cdot K_{\NC}(s_\chi^{-1}\cdot \pi).$$
	Equivalently, if $\bar{K}_{\BNC}(\pi)$ is the maximum partition on $\set{\bar1, \ldots, \bar n}$ such $\pi \cup \bar{K}_{\BNC}(\pi)$ is bi-non-crossing on the set $\set{1, \ldots, n, \bar1, \ldots,\bar n}$ with choice of side given by $\chi'(j) = \chi(j) = \chi'(\bar{j})$ and ordering such that $j < \bar j$ on the left and $\bar j < j$ on the right, then $K_{\BNC}(\pi) := \set{B \subset [n] : \set{\bar{j} : j \in B} \in \bar{K}_{\BNC}(\pi)}$.
\end{definition}

\begin{remark}
	\label{rem:postkreweras}
	We point out that $\bar{K}_\BNC(\pi)$ is the unique partition on $\set{\bar1, \ldots, \bar n}$ such that $\pi\cup\bar{K}_\BNC(\pi)$ is $\chi'$-non-crossing and, if $\beta = \set{\set{j, \bar{j}} : j \in [n]} \in \BNC(\chi')$, then $\paren{\pi\cup\bar{K}_\BNC(\pi)} \vee \beta = 1_{\chi'}$.
	Indeed, the fact that $\bar{K}_\BNC(\pi)$ is maximum ensures that any larger partition is not $\chi'$-non-crossing, while any smaller partition must fail to connect two nodes which are connected by $\bar{K}_\BNC(\pi)$, which will then remain disconnected in $\rho\cup\pi \vee \beta$.
\end{remark}

\begin{example}
	Suppose that $\chi$ is given by the sequence $(\ell, r, \ell, \ell, r, \ell, r, r)$ and $$\pi = \set{\set{1}, \set{2,4,8}, \set{3}, \set{5,7}, \set{6}} \in \BNC(\chi).$$
	Then $K_{\BNC}(\pi) = \set{\set{1,2,3}, \set{4,6}, \set{5,8}, \set{7}} \in \BNC(\chi)$, as illustrated below.
	\[\begin{tikzpicture}[baseline]
		\draw[thick] (-1,0.25) -- (-1, -3.75) -- (1,-3.75) -- (1,0.25);

		\def\sidez{{-1,1,-1,-1,1,-1,1,1}}
		\foreach \y in {0,...,7} {
			\pgfmathtruncatemacro{\nodename}{\y+1}
			\pgfmathtruncatemacro{\sd}{\sidez[\y]}
			\node (ball\nodename) [draw, shade, circle, ball color=black, inner sep=0.07cm] at (\sd, -\y*0.5) {};
			\ifthenelse{\sd=1}{\node[right] at (\sd, -\y*0.5) {\nodename}}
					{\node[left] at (\sd, -\y*0.5) {\nodename}};
		}

		\draw [thick] (ball2) -- ++(-1, 0) |- (ball8);
		\draw [thick] (ball4) -- ++(1, 0);
		\draw [thick] (ball5) -- ++(-0.5, 0) |- (ball7);

		\begin{scope}[color=iblue]
			\def\sidez{{-1,1,-1,-1,1,-1,1,1}}
			\foreach \y in {0,...,7} {
				\pgfmathtruncatemacro{\nodename}{\y+1}
				\pgfmathtruncatemacro{\sd}{\sidez[\y]}
				\node (nball\nodename) [draw, shade, circle, ball color=iblue, color=black, inner sep=0.07cm] at (\sd, -\y*0.5+\sd*0.2) {};
				\ifthenelse{\sd=1}{\node[right=0.2] at (\sd, -\y*0.5+\sd*0.2) {$\bar{\nodename}$}}
						{\node[left=0.2] at (\sd, -\y*0.5+\sd*0.2) {$\bar{\nodename}$}};
			}

			\draw [thick] (nball1) -- ++(0.5, 0) |- (nball2);
			\draw [thick] (nball3) -- ++(0.5, 0) |- (nball2);
			\draw [thick] (nball4) -- ++(0.5, 0) |- (nball6);
			\draw [thick] (nball5) -- ++(-0.75, 0) |- (nball8);

		\end{scope}
	\end{tikzpicture}\]
\end{example}

We notice that since $K_{\NC}$ is order reversing while $s_\chi$ is order-preserving on $\cP(n)$, $K_{\BNC}$ is order-reversing.
Then for $\pi \in \BNC(\chi)$, we have $[\pi, 1_\chi] \cong \sq{K_{\BNC}(1_\chi), K_{\BNC}(\pi)} = \sq{0_\chi, K_{\BNC}(\pi)}$.
Then if $f, g \in IA(\BNC)$ are multiplicative,
$$(f\star g)(0_\chi, 1_\chi) = \sum_{\pi \in \BNC(\chi)} f(0_\chi, \pi)g(0_\chi, K_{\BNC}(\pi)) = (g\star f)(0_\chi, 1_\chi);$$
hence $f\star g = g\star f$.



\subsection{Cumulants of products.}
Let $n \in \N$, and suppose $0 = k_0 < k_1 < \cdots < k_m = n$.
%For $\chi : [m] \to \slr$, define $\hat\chi : [n] \to \slr$ so that if $j \in [n]$ and $t \in [m]$ is the minimum such that $j \leq k_t$, then $\hat\chi(j) = \chi(t)$.
For $\chi : [m] \to \slr$, define $\hat\chi : [n] \to \slr$ to be constant on intervals $(k_{j-1}, k_{j}]$ with $\hat\chi(k_j) = \chi(j)$.
There is an embedding of $\BNC(\chi) \hookrightarrow \BNC(\hat\chi)$ via $\pi \mapsto \hat\pi$, the unique partition with $k_j \sim_{\hat\pi} k_i$ if and only if $j \sim_\pi i$ and $\set{(k_{j-1}, k_j] : 1 \leq j \leq m} \leq \hat\pi$; note that this partition of intervals is precisely $\hat{0_\chi}$, while $\hat{1_\chi} = 1_{\hat\chi}$.
Further,
$$\set{\hat\pi : \pi \in \BNC(\chi)} = \sq{\hat{0_\chi}, \hat{1_\chi}} = \sq{\hat{0_\chi}, 1_{\hat\chi}} \subseteq\BNC(\hat\chi)$$
is an interval of $\BNC(\hat\chi)$ and $\pi \mapsto \hat\pi$ is a lattice morphism, so $\mu_\BNC(\sigma, \pi) = \mu_\BNC(\hat\sigma, \hat\pi)$ for $\sigma, \pi \in \BNC(\chi)$.

We will now exploit a useful combinatorial fact (see, e.g., \cite{nica2006lectures}*{Proposition 10.11}): if functions $f, g : \BNC(\hat\chi) \to \C$ (or any finite lattice) are related via
$$f(\pi) = \sum_{\substack{\sigma\in\BNC(\chi) \\ \sigma\leq\pi}} g(\sigma),$$
then
$$\sum_{\substack{\tau\in\BNC(\hat\chi) \\ \sigma\leq\tau\leq\pi}} f(\tau)\mu_\BNC(\tau, \pi) = \sum_{\substack{\omega\in\BNC(\hat\chi) \\ \omega \vee \sigma = \pi}} g(\omega).$$

\begin{proposition}
	\label{prop:multicumulantplicationthingywhateverihatelabelsnowsosueme}
	Let $(\A, \varphi)$ be a non-commutative probability space, $n \in \N$, $0 = k_0 < k_1 < \cdots < k_m = n$, and $\chi : [m] \to \slr$.
	If $\pi \in \BNC(\chi)$ and $T_j \in \A_{\hat\chi(j)}$ for $j \in [n]$, then
	$$\kappa_\pi\paren{T_1\cdots T_{k_1}, T_{k_1+1} \cdots T_{k_2}, \ldots, T_{k_{m-1}+1}\cdots T_{k_m}}
	= \sum_{\substack{\sigma \in \BNC(\hat\chi) \\ \sigma \vee \hat{0_\chi} = \hat\pi}} \kappa_\sigma(T_1, \ldots, T_n).$$
\end{proposition}

\begin{proof}
	Expressing the cumulant above in terms of moments via the moment-cumulant relation mentioned in Remark~\ref{rem:cumulantformula}, we find
	\begin{align*}
		&\kappa_\pi\paren{T_1\cdots T_{k_1}, T_{k_1+1} \cdots T_{k_2}, \ldots, T_{k_{m-1}+1}\cdots T_{k_m}} \\
		&\qquad= \sum_{\substack{\tau\in\BNC(\chi)\\\tau\leq\pi}} \varphi_\tau\paren{T_1\cdots T_{k_1}, T_{k_1+1} \cdots T_{k_2}, \ldots, T_{k_{m-1}+1}\cdots T_{k_m}} \mu_\BNC(\tau, \pi) \\
		&\qquad= \sum_{\substack{\tau \in \BNC(\chi) \\ \tau\leq\pi}} \varphi_{\hat\tau}(T_1, \ldots, T_n)\mu_{\BNC}(\hat\tau, \hat\pi) \\
		&\qquad= \sum_{\substack{\sigma\in\BNC(\hat\chi) \\ \hat{0_\chi} \leq \sigma \leq \hat\pi}} \varphi_\sigma(T_1, \ldots, T_n) \mu_\BNC(\sigma, \hat\pi) \\
		&\qquad= \sum_{\substack{\sigma\in\BNC(\hat\chi) \\ \sigma\vee\hat{0_\chi} = \hat\pi}} \kappa_\sigma(T_1, \ldots, T_n).
	\end{align*}
\end{proof}


\subsection{Multiplicative convolution.}
\begin{theorem}
	\label{thm:multiconv}
	Let $(\A, \varphi)$ be a non-commutative probability space, and $\paren{(z_i^{(\iota)})_{i\in I}, (z_j^{(\iota)})_{j\in J}}_{\iota\in\set{1,2}}$ be bi-free; set $z_i = z_i^{(1)}z_i^{(2)}$ for $i \in I$, and $z_j = z_j^{(2)}z_j^{(1)}$ for $j \in J$.
	Then for every $\alpha : [n] \to I\coprod J$, we have
	$$\kappa_{\chi_\alpha}(z_{\alpha(1)}, \ldots, z_{\alpha(n)})
	= \sum_{\pi\in\BNC(\chi_\alpha)}\kappa_\pi\paren{z_{\alpha(1)}^{(1)}, \ldots, z_{\alpha(n)}^{(1)}}\kappa_{K_\BNC(\pi)}\paren{z_{\alpha(1)}^{(2)}, \ldots, z_{\alpha(n)}^{(2)}}.$$
\end{theorem}
Our proof here is inspired by the presentation of the free version found in \cite{nica2006lectures}*{Theorem 14.4}.
\begin{proof}
	Define $\widehat{\alpha} : [2n] \to I \coprod J$ by $\widehat{\alpha}(2k-1) = \widehat{\alpha}(2k) = \alpha(k)$ for $k \in [n]$, and define $\iota : [2n] \to \set{1,2}$ by
	\[
		\iota(2k-1)
	= \left\{
		\begin{array}{ll}
			1 & \text{if } \alpha(k) \in I
			\\
				2 & \text{if } \alpha(k) \in J
		\end{array} \right.
	\qquad \text{ and }\qquad \iota(2k) = 
	\left\{
		\begin{array}{ll}
			2 & \text{if } \alpha(k) \in I
			\\
			1 & \text{if } \alpha(k) \in J
		\end{array} \right..
	\]
	Using Proposition~\ref{prop:multicumulantplicationthingywhateverihatelabelsnowsosueme} we see that
	\[
		\kappa_\chi(z_{\alpha(1)}, \ldots, z_{\alpha(n)}) = \sum_{\substack{\pi \in \BNC(\chi_{\widehat{\alpha}}) \\ \pi \vee \sigma = 1_{\chi_{\widehat{\alpha}}}}} \kappa_\pi\paren{z^{\iota(1)}_{\alpha(1)}, z^{\iota(2)}_{\alpha(1)}, \ldots, z^{\iota(2n-1)}_{\alpha(n)}, z^{\iota(2n)}_{\alpha(n)}}
	\]
	where $\sigma = \set{(1,2), (3,4), \ldots, (2n-1, 2_n)}$.
	By bi-freeness and Theorem~\ref{thm:biequiv}, the mixed cumulants above vanish, and the only partitions which survive correspond to pairs $\pi^{(1)}, \pi^{(2)}$ such that $\pi^{(i)}$ is a bi-non-crossing partition on the nodes in $\iota^{-1}(i)$, such that the pair taken together remain bi-non-crossing on $[2n]$, and $\paren{\pi^{(1)}\cup\pi^{(2)}} \vee \sigma = 1_{\chi_{\hat\alpha}}$.
	But as we noted in Remark~\ref{rem:postkreweras}, this is exactly the condition that $\pi^{(2)}$ is the Kreweras complement of $\pi^{(1)}$ (on relabelled indices).
	Thus the claimed equation follows.
\end{proof}

If we restrict the above theorem to the case that each of $I$ and $J$ contains a single element, we have a formula for how cumulants behave under bi-free multiplicative convolution.
By summing over $K_\BNC(\pi)$ instead of $\pi$, we also find that multiplicative convolution is commutative:
$$\mu_1\boxtimes\boxtimes_K\mu_2 = \mu_2\boxtimes\boxtimes_K\mu_1.$$
% 
% 
% 
% 
% 
% 
% With this in hand, we can now describe the bi-free multiplicative convolution $\boxtimes\boxtimes_K$.
% \begin{theorem}
% 	\label{thm:multkrewer}
% 	Let $\paren{(z_\ell^{(\iota)}, z_r^{(\iota)})}_{\iota\in\set{1,2}}$ be a family of bi-free pairs of variables, and set $z_\ell = z_\ell^{(1)}z_\ell^{(2)}$, $z_r = z_r^{(2)}z_r^{(1)}$.
% 	Then for every $\chi : [n] \to \slr$, we have
% 	$$\kappa_\chi(z_{\chi(1)}, \ldots, z_{\chi(n)}) = \sum_{\pi \in \BNC(\chi)} \kappa_\pi(z_{\chi(1)}^{(1)}, \ldots, z_{\chi(n)}^{(1)}) \kappa_{K_{\BNC}(\pi)}(z_{\chi(1)}^{(2)}, \ldots, z_{\chi(n)}^{(2)}).$$
% \end{theorem}
% 
% \begin{proof}
% 	We first notice that we can describe the moments of the pair $(z_\ell, z_r)$ as follows: given $\chi : [n] \to \slr$, let $\beta : [2n] \to \slr$ be such that $\beta(2k-1) = \beta(2k) = \chi(k)$, and $\iota : [2n] \to \set{1,2}$ be so that $\iota(2k-1) = 1$ and $\iota(2k) = 2$ when $\chi(k) = \ell$, and $\iota(2k-1) = 2$ and $\iota(2k) = 1$ when $\chi(k) = r$; then
% 	\begin{align*}
% 		\varphi(z_{\chi(1)}\cdots z_{\chi(n)})
% 		&= \varphi\paren{z_{\beta(1)}^{(\iota(1))}z_{\beta(2)}^{(\iota(2))}\cdots z_{\beta(2n)}^{(\iota(2n))}} \\
% 		&= \sum_{\substack{\pi \in \BNC(\beta) \\ \pi \leq \iota}} \kappa_\pi(z_{\beta(1)}^{(\iota(1))}, \ldots, z_{\beta(2n)}^{(\iota(2n))}) \\
% 		&= \sum_{\pi \in \BNC(\chi)} \kappa_\pi(z_{\chi(1)}^{(1)}, \ldots, z_{\chi(n)}^{(1)}) \sum_{\substack{\sigma\in\BNC(\chi) \\ \sigma \leq K_{\BNC}(\pi)}} \kappa_\sigma(z_{\chi(1)}^{(2)}, \ldots, z_{\chi(n)}^{(2)}) \\
% 		&= \sum_{\pi \in \BNC(\chi)} \kappa_{\pi}(z_{\chi(1)}^{(1)}, \ldots, z_{\chi(n)}^{(1)}) \varphi_{K_{\BNC}(\pi)}(z_{\chi(1)}^{(2)}, \ldots, z_{\chi(n)}^{(2)})
% 	\end{align*}
% 
% 	Now if we let $m_1, k_1 \in IA(\BNC)$ be the multiplicative functions satisfying
% 	$$m_1(0_\chi, 1_\chi) = \varphi(z_{\chi(1)}^{(1)} \cdots z_{\chi(n)}^{(1)}) \qquad\text{and}\qquad
% 	k_1(0_\chi, 1_\chi) = \kappa_\chi(z_{\chi(1)}^{(1)}, \ldots, z_{\chi(n)}^{(1)}),$$
% 	and we define $m_2, k_2, m, k \in IA(\BNC)$ similarly, the above computation has shown us that
% 	$$m = k_1\star m_2,\qquad\text{hence}\qquad
% 	m\star\zeta_{\BNC} = k_1\star m_2 \star \zeta_{\BNC} = k_1\star k_2,$$
% 	which is the formula we claimed.
% \end{proof}
% 
% We also remark that since convolution is commutative on multiplicative functions, we have $\mu_1\boxtimes\boxtimes_K\mu_2 = \mu_2\boxtimes\boxtimes_K\mu_1$.

\section{An operator model for pairs of faces.}
In this section, we will construct an operator model for a two-faced family in a non-commutative probability space.
The model will generalize the operator model of Nica, \cite{Nica1996271}, from the free setting to the bi-free setting.
Our goal is to recognize the variables as formal sums of (densely-defined unbounded) operators on a Fock space, for which the pairings of the form $\ang{e_{i_1}\otimes\cdots\otimes e_{i_n}, Te_{j_1}\otimes\cdots\otimes e_{j_m}}$ make sense and are finite for any $T$ in the algebra generated by the representatives of the variables.

\subsection{Nica's operator model.}
We begin by reviewing Nica's operator model, with the intent of making the description of our model more understandable.
Suppose that we are given a law $\mu : \C\ang{X_1, \ldots, X_n} \to \C$, and let $\kappa$ be the corresponding cumulant functional.
Let $\cH := \cF(\C^n)$, and take $\set{e_1, \ldots, e_n}$ to be the standard orthonormal basis of $\C^n$, and let us use the shorthand $\ell_j = \ell(e_j), \ell_j^* = \ell^*(e_j)$ for the corresponding creation and annihilation operators.
We extend the vacuum state on $\cH$ to the space on which we are working by $\omega(T) = \ang{\Omega, T\Omega}$.

Formally, we define:
\begin{align*}
	\Theta_\mu &:= I + \sum_{k\geq1} \sum_{i_1, \ldots, i_k \in [n]} \kappa_k(X_{i_1}, \ldots, X_{i_k}) \ell_{i_k}\cdots \ell_{i_1} \\
	Z_i &:= \ell_i^*\Theta_\mu = \ell_i^* + \sum_{k\geq 0} \sum_{i_1, \ldots, i_k \in [n]} \kappa_{k+1}(X_{i_1}, \ldots, X_{i_k}, X_i) \ell_{i_k}\cdots \ell_{i_1},
\end{align*}
where we understand an empty product of creation operators as the identity operator.

\begin{proposition}
	The variables $X_1, \ldots, X_n$ with respect to $\varphi$ and $Z_1, \ldots, Z_n$ with respect to $\omega$ have the same law.
\end{proposition}

\begin{proof}
	Fix $i_1, \ldots, i_k \in [n]$.
	On the one hand, from the moment cumulant formula, we have
	$$\varphi(X_{i_1}\cdots X_{i_k}) = \sum_{\pi\in\NC(k)}\kappa(X_{i_1}, \ldots, X_{i_k}).$$

	On the other hand, let us consider the expansion of $Z_{i_1}\cdots Z_{i_k}\Omega$.
	Note that in order for a term to survive when paired with $\Omega$, each creation operator must be paired with an annihilation operator to its left, and each annihilation operator must be paired with a creation operator to its right; the indices of these operators must agree.
	Then we can associate to each surviving term a unique element of $\NC(k)$ by taking the minimal partition which satisfies that $a\sim b$ whenever $Z_{i_a}$ introduces a creation operator paired with the annihilation operator from $Z_{i_b}$.
	This non-crossing partition is unique, and for each non-crossing partition we may find such a term which survives.
	But now the weight of the term corresponding to $\pi$ is precisely the product over the blocks of $B$ of the corresponding cumulants, namely, $\kappa_\pi(X_{i_1}, \ldots, X_{i_k})$.
	Thus
	$$\omega(Z_{i_1}\cdots X_{i_k}) = \sum_{\pi\in\NC(k)}\kappa(X_{i_1}, \ldots, X_{i_k}).$$
\end{proof}

We think of the operator as follows: each term in the sum of $\Theta_\mu$ corresponds to all possible blocks which may occur in a non-crossing partition.
In the product $\ell_{i_1}^*\Theta_\mu\ell_{i_2}^*\Theta_\mu\cdots \ell_{i_k}^*\Theta_\mu$, each annihilation operator ``fills'' a slot in a block which has been introduced by a prior $\Theta_\mu$, while each $\Theta_\mu$ may introduce a new block at the current point in the progress of building the partition.
We introduce many potential blocks by $\Theta_\mu$, but those which are not valid for the product we are currently building cannot be completed by the appropriate annihilation operators and so do not contribute.

\subsection{Skeletons corresponding to bi-non-crossing partitions.}
We now aim to find an appropriate extension of this model to the bi-free setting.
One may be tempted to try to model left variables by left creation and annihilation operators, and right variables by right creation and annihilation operators.
Unfortunately the fact that in general left and right variables from the same pair of faces need not commute with each other means that such a model cannot possibly work; we need more complicated models to capture this behaviour.

\begin{definition}
	Let $\chi : [n] \to \slr$ and let $\pi \in \BNC(\chi)$.
	A \emph{skeleton on $\pi$} is the data $(\alpha, k)$ where $\alpha : [n] \to I\coprod J$ is such that $\chi_\alpha = \chi$, and $0 \leq k \leq n$.
	We will represent this data by drawing the diagram corresponding to $\pi$, labelling the nodes according to $\alpha$, and filling the bottom $k$ nodes while leaving the top $n-k$ empty.
	If $k = n$, the skeleton is said to be \emph{completed}; if $k = 0$, it is said to be \emph{empty}; and if $k = 1$ it is said to be a \emph{starter skeleton}.
\end{definition}

\begin{example}
	Suppose $\chi$ corresponds to the sequence $(\ell, \ell, r, \ell, r)$, fix $\alpha : [5] \to I\coprod J$, and suppose $\pi = \set{\set{1,3}, \set{2,4,5}}$.
	Then the six skeletons corresponding to $\pi$ and $\alpha$ are:
	\[\begin{tikzpicture}[baseline]
		\foreach \asdf in {0,...,5} {
			\pgfmathtruncatemacro{\xshft}{4*Mod(\asdf,3)}
			%\begin{scope}[xshift=4*{Mod(\asdf,3)}*1cm, yshift=floor(\asdf/3)*3cm]
				\begin{scope}[xshift=\xshft cm, yshift=-floor(\asdf/3)*3cm]
					\draw[thick] (-1,0.25) -- (-1, -2.25) -- (1,-2.25) -- (1,0.25);

					\def\sidez{{-1,-1,1,-1,1}}
					\foreach \y in {0,...,4} {
						\pgfmathtruncatemacro{\nodename}{\y+1}
					\pgfmathtruncatemacro{\sd}{\sidez[\y]}
					\ifthenelse{\asdf > \y}
					{\node (ball\nodename) [draw, shade, circle, ball color=white, inner sep=0.07cm] at (\sd, -\y*0.5) {};}
					{\node (ball\nodename) [draw, shade, circle, ball color=black, inner sep=0.07cm] at (\sd, -\y*0.5) {};}
					\ifthenelse{\sd=1}{\node[right]}{\node[left]} at (ball\nodename) {$\alpha(\nodename)$};
					}

					\draw [thick] (ball1) -- (ball1 -| 0.25,0) |- (ball3);
					\draw [thick] (ball2) -- (ball2 -| -0.25,0) |- (ball5);
					\draw [thick] (ball4) -- (ball4 -| -0.25,0);

				\end{scope}
				}
			\end{tikzpicture}\]
	\end{example}



	We will use skeletons to make it easier to keep track of what operators have acted so far, and how new ones could potentially be added.
	Skeletons can be related to Nica's model as well: one thinks of any term $\ell_{i_1}\cdots \ell_{i_k}$ in the sum in $\Theta_\mu$ as corresponding to adding to introducing an empty skeleton consisting of a single block to the current skeleton, with nodes labelled by $i_1, \ldots, i_k$, while each annihilation operator $\ell^*_{j}$ fills in the next empty node if it is labelled by $j$, and otherwise returns $0$.
	A skeleton corresponds to a tensor product of the vectors with labels corresponding to its unfilled nodes; the partition corresponding to the skeleton contributes only to the scalar portion of the vector.
	At the end, only completed skeletons will contribute as only they will correspond to vectors in $\C\Omega$.

	\begin{example}
		Let us consider the product $Z_1Z_2Z_3Z_4$ in Nica's model, and examine how the term corresponding to $\kappa_2(X_1, X_4)\kappa_2(X_2, X_3)$ may be realized.
		This term corresponds to choosing $\kappa_2(X_1, X_4)\ell_4^*\ell_4\ell_1$ from $Z_4$, $\kappa_2(X_2, X_3)\ell_3^*\ell_3\ell_2$ from $Z_3$, $\ell_2^*$ from $Z_2$, and $\ell_1^*$ from $Z_1$.
		\[
			\ell_1^*\ell_2^*\ell_3^*\ell_3\ell_2\ell_4^*\ell_4\ell_1\Omega
		= \ell_1^*\ell_2^*\ell_3^*\ell_3\ell_2
		\begin{tikzpicture}[baseline]
			\draw[thick] (-1, 1.25) -- (-1, -0.75);
			\node (ball1) [draw, shade, circle, ball color=white, inner sep=0.07cm] at (-1, 1) {};
			\node[left] at (ball1) {1};

			\node (ball4) [draw, shade, circle, ball color=black, inner sep=0.07cm] at (-1, -0.5) {};
			\node[left] at (ball4) {4};
			\draw [thick] (ball4) -- ++(1,0) |- (ball1);
		\end{tikzpicture}
		= \ell_1^*\ell_2^*
		\begin{tikzpicture}[baseline]
			\draw[thick] (-1, 1.25) -- (-1, -0.75);
			\node (ball1) [draw, shade, circle, ball color=white, inner sep=0.07cm] at (-1, 1) {};
			\node[left] at (ball1) {1};

			\node (ball2) [draw, shade, circle, ball color=white, inner sep=0.07cm] at (-1, 0.5) {};
			\node[left] at (ball2) {2};

			\node (ball3) [draw, shade, circle, ball color=black, inner sep=0.07cm] at (-1, 0) {};
			\node[left] at (ball3) {3};

			\node (ball4) [draw, shade, circle, ball color=black, inner sep=0.07cm] at (-1, -0.5) {};
			\node[left] at (ball4) {4};

			\draw [thick] (ball4) -- ++(1,0) |- (ball1);
			\draw [thick] (ball3) -- ++(.5,0) |- (ball2);
		\end{tikzpicture}
		= \ell_1^*
		\begin{tikzpicture}[baseline]
			\draw[thick] (-1, 1.25) -- (-1, -0.75);
			\node (ball1) [draw, shade, circle, ball color=white, inner sep=0.07cm] at (-1, 1) {};
			\node[left] at (ball1) {1};

			\node (ball2) [draw, shade, circle, ball color=black, inner sep=0.07cm] at (-1, 0.5) {};
			\node[left] at (ball2) {2};

			\node (ball3) [draw, shade, circle, ball color=black, inner sep=0.07cm] at (-1, 0) {};
			\node[left] at (ball3) {3};

			\node (ball4) [draw, shade, circle, ball color=black, inner sep=0.07cm] at (-1, -0.5) {};
			\node[left] at (ball4) {4};
			\draw [thick] (ball4) -- ++(1,0) |- (ball1);
			\draw [thick] (ball3) -- ++(.5,0) |- (ball2);
		\end{tikzpicture}
		=
		\begin{tikzpicture}[baseline]
			\draw[thick] (-1, 1.25) -- (-1, -0.75);
			\node (ball1) [draw, shade, circle, ball color=black, inner sep=0.07cm] at (-1, 1) {};
			\node[left] at (ball1) {1};

			\node (ball2) [draw, shade, circle, ball color=black, inner sep=0.07cm] at (-1, 0.5) {};
			\node[left] at (ball2) {2};

			\node (ball3) [draw, shade, circle, ball color=black, inner sep=0.07cm] at (-1, 0) {};
			\node[left] at (ball3) {3};

			\node (ball4) [draw, shade, circle, ball color=black, inner sep=0.07cm] at (-1, -0.5) {};
			\node[left] at (ball4) {4};
			\draw [thick] (ball4) -- ++(1,0) |- (ball1);
			\draw [thick] (ball3) -- ++(.5,0) |- (ball2);
		\end{tikzpicture}
	\]
\end{example}


\subsection{A construction.}
\label{constructingtheoperatormodel}

We will now construct our operator model for pairs of faces, motivated by our realization of Nica's operator model.
Above, the model constructed all weighted non-crossing partitions by using creation operators to glue in full non-crossing blocks and annihilation operators to approve or reject non-crossing diagrams.
As the combinatorics of pairs of faces is dictated by bi-non-crossing partitions, we must construct the appropriate creation operators to glue together bi-non-crossing partitions.
However, unlike with non-crossing partitions where there is only one way to glue in a full block at any given point, there may be multiple or no ways to glue one bi-non-crossing skeleton into another.
As such, the description of the appropriate creation operators is more complicated.

Let $z = ((z_i)_{i \in I}, (z_j)_{j \in J})$ be a two-faced family in $(\A, \varphi)$.
We will construct our model as formal sums of products of creation and annihilation operators on the Fock space $\cH := \cF\paren{\C^{I \coprod J}}$, with $\set{e_k : k \in I \coprod J}$ an orthonormal basis.

For $\alpha : [n] \to I \coprod J$, we will define (unbounded) operators $T_\alpha \in \cL(\cH)$ which will play the roles of the terms in the sum in the definition of $\Theta_\mu$ in Nica's model; that is, each will add an appropriate empty skeleton.
We aim to construct an operator $\Theta$ of the form
$$\Theta := I + \sum_{n\geq1}\sum_{\alpha : [n] \to I\coprod J}\kappa_{\chi_\alpha}(z_{\alpha(1)}, \ldots, z_{\alpha(n)})T_\alpha,$$
in such a way that the variables
$$Z_k := \ell_k^*\Theta = \ell_k^* + \sum_{n \geq 0}\sum_{\substack{\alpha : [n+1] \to I\coprod J \\ \alpha(n+1) = k}} \kappa_{\alpha}(z_{\alpha(1)}, \ldots, z_{\alpha(n)}, z_k)T_\alpha$$
give us the correct distribution.

Though we will often speak of actions on skeletons, one can recover the context of $\cH$ by letting a partially completed skeleton correspond to the vector formed by taking the tensor product of the basis elements matching the labels of its open nodes, from bottom to top, and weighting it based on which cumulants have been chosen.
For example, the skeleton
\[\begin{tikzpicture}[baseline]
	\draw[thick] (-1,0.25) -- (-1, -2.25) -- (1,-2.25) -- (1,0.25);

	\def\sidez{{-1,-1,1,1,-1}}
	\foreach \y in {0,...,4} {
		\pgfmathtruncatemacro{\nodename}{\y+1}
		\pgfmathtruncatemacro{\sd}{\sidez[\y]}
		\ifthenelse{\y < 3}
		{\node (ball\nodename) [draw, shade, circle, ball color=white, inner sep=0.07cm] at (\sd, -\y*0.5) {};}
		{\node (ball\nodename) [draw, shade, circle, ball color=black, inner sep=0.07cm] at (\sd, -\y*0.5) {};}
		\ifthenelse{\sd=1}{\node[right]}{\node[left]} at (ball\nodename) {$\alpha(\nodename)$};
	}

	\draw [thick] (ball1) -- (ball1 -| 0,0) |- (ball4);
	\draw [thick] (ball2) -- (ball2 -| -0.5,0) |- (ball5);
\end{tikzpicture}\]
corresponds to the vector $e_{\alpha(3)}\otimes e_{\alpha(2)} \otimes e_{\alpha(1)}$ and will be weighted by the product of cumulants $\kappa(z_{\alpha(2)}, z_{\alpha(5)})\kappa(z_{\alpha(1)}, z_{\alpha(4)})\kappa(z_{\alpha(3)})$.
The key point here is that the only choices of future $Z_k$ which yield a non-zero $\Omega$ component when applied to such a vector have annihilation operators in the correct order.
In the above example, in order for this skeleton to make a contribution to the final term, we must act on it by $Z_{\alpha(3)}$, $Z_{\alpha(2)}$, and $Z_{\alpha(1)}$ in that order (though other variables may occur between them).
Since the closed nodes of the skeleton only effect the resulting quantity in terms of its weight and cannot affect the action of future operators (as indeed they must not, for the vector has forgotten them) we will sometimes truncate diagrams of skeletons to show only the open nodes.
It is implied that there may be significantly more nodes and blocks below the bottom of the diagrams that follow, but their representation is eschewed.
Likewise, in order to ensure that $T_\alpha$ is well-defined, we cannot have behaviour depending on which partial skeletons have been chosen, but only the choice of side and of labels of the open nodes.


For $n = 1$, we define $T_\alpha :=  \ell_{\alpha(1)}$.
In this setting, one may think of $T_\alpha$ as adding an empty skeleton in the lowest possible position with a single open node on the left or on the right depending on whether $\alpha(1)$ is in $I$ or $J$.
For example,
\[
	\begin{tikzpicture}
		\def\sidez{{0,-1,-1,1,-1,-1,1}}
		\def\labelz{{"", "$\beta(1)$", "$\beta(2)$", "$\beta(3)$", "$\beta(4)$", "$\alpha(1)$", "$\alpha(1)$"}}
		\def\clrz{{"", "white", "white", "black", "black", "white", "white"}}
		\begin{scope}[xshift=-6cm]
			\def\ord{{1,2,0,3,4}}
			\bnc[n=5,labelz=\labelz,sidez=\sidez,colourz=\clrz,order=\ord]
			\draw [thick] (ball1) -- (ball1 -| 0,0) |- (ball4);
			\draw [thick] (ball2) -- (ball2 -| -0.5,0) |- (ball5);
			\coordinate (cr1) at (cr);
		\end{scope}

		\begin{scope}[xshift=-1cm]
			\def\ord{{1,2,5,3,4}}
			\bnc[n=5,labelz=\labelz,sidez=\sidez,colourz=\clrz,order=\ord]

			\draw [thick] (ball1) -- (ball1 -| 0,0) |- (ball4);
			\draw [thick] (ball2) -- (ball2 -| -0.5,0) |- (ball5);
			\coordinate (cr2) at (cr);
			\coordinate (cl2) at (cl);
		\end{scope}

		\begin{scope}[xshift=4cm]
			\def\ord{{1,2,6,3,4}}
			\bnc[n=5,labelz=\labelz,sidez=\sidez,colourz=\clrz,order=\ord]

			\draw [thick] (ball1) -- (ball1 -| 0,0) |- (ball4);
			\draw [thick] (ball2) -- (ball2 -| -0.5,0) |- (ball5);
			\coordinate (cl3) at (cl);
			\coordinate (cr3) at (cr);
		\end{scope}

		\draw [thick,->] ($ (cr1) + (1,0) $) -- node [above] {$T_\alpha$} ($ (cl2) - (1,0) $);
		\node at ($ (cr2) ! .5 ! (cl3) $) {or};

		\node[right] at ($ (cr3) + (0.2,0) $) {\phantom{$\alpha(1)$},};
	\end{tikzpicture}
\]
depending on whether $\alpha(1)$ is in $I$ or $J$.
Observe that $T_\alpha$ adds an open node in the lowest valid location (i.e., immediately above all closed nodes); this behaviour will be mimicked by the other $T_\alpha$ as well.
That is, the lowest open node added will always be added directly above the highest closed node.

Let $\Sigma : \cH \oplus \cH \to \cH$ be defined by
\[
	\Sigma\paren{f_1\otimes\cdots\otimes f_n, f_{n+1} \otimes \cdots \otimes f_{n+m}} := \sum_\sigma f_{\sigma(1)}\otimes\cdots\otimes f_{\sigma(n+m)},
\]
where the sum is over all permutations $\sigma \in S_{n+m}$ so that $\sigma|_{[1, n]}$ and $\sigma|_{[n+1, n+m]}$ are increasing; that is, $\sigma$ interleaves the sets $[n]$ and $\set{n+1,\ldots, n+m}$.
Note that $\Sigma(\xi, \Omega) = \xi = \Sigma(\Omega, \xi)$.
As an example,
\begin{align*}
	\Sigma(e_{1}\otimes e_{2}, e_{3}\otimes e_4) =& \  e_1 \otimes e_2 \otimes e_3 \otimes e_4 + e_1 \otimes e_3 \otimes e_2 \otimes e_4 + e_3 \otimes e_1 \otimes e_2 \otimes e_4   \\
	& + e_1 \otimes e_3 \otimes e_4 \otimes e_2 + e_3 \otimes e_1 \otimes e_4\otimes e_2 + e_3 \otimes e_4 \otimes e_1 \otimes e_2.
\end{align*}
We will use $\Sigma$ to account for the fact that nodes on the right may be added with any order to nodes on the left to obtain a valid skeleton.



For $\alpha : [n] \to I \coprod J$ we define
\[
	T_{\alpha}(\Omega) := \ell_{\alpha(n)} \ell_{\alpha(n-1)} \cdots \ell_{\alpha(1)}(\Omega) = e_{\alpha(1)}\otimes\cdots\otimes e_{\alpha(n)}.
\]
Note that this corresponds to taking a completed skeleton (possibly with no nodes), and adding the empty skeleton corresponding to $\alpha$ above it.


We will now define $T_{\alpha}$ for $n\geq 2$ on tensors of basis elements, and extend by linearity to their span (which, together with $\Omega$, is dense in $\cH$).
We consider only the case $\alpha(n) \in I$, as the case when $\alpha(n) \in J$ will be similar.
Let $\eta = e_{\beta(m)}\otimes\cdots\otimes e_{\beta(1)} \in \cH$, where $\beta : \set{1, \ldots, m} \to I \coprod J$.

%Let $k$ be the largest element of \i{1, \ldots, n\}$ such that $\alpha(k) \in J$ (or $k = 0$ if $\alpha$ maps into $I$.
If $\beta^{-1}(I) = \emptyset$, let $k = \max\paren{\set0\cup \alpha^{-1}(J)}$, so that $k$ is the lowest of the nodes to be added by which falls on the right.
We define $T_\alpha(\eta)$ as follows:
\[
	T_\alpha(\eta) := e_{\alpha(n)} \otimes \Sigma\paren{ e_{\alpha(n - 1)} \otimes \cdots \otimes e_{\alpha(k+1)}, e_{\beta(m)} \otimes \cdots \otimes e_{\beta(1)} } \otimes e_{\alpha(k)} \otimes \cdots \otimes e_{\alpha(1)}.
\]
This is mimicking the action of adding a new skeleton to the existing skeleton.
%In order to ensure that no crossings are introduced, all nodes on the right of the old skeleton must be placed before any new nodes can be added to the right, since these new nodes must be connected to the first node on the left; before that, however, the nodes added on the left may be added with any relative order to the existing nodes on the right.
In order to ensure that no crossings are introduced, all new nodes on the right must be placed above all existing nodes on the right; before any new right nodes are added, though, nodes on the left can be added freely.
One should think of this as the sum of all valid partially completed skeletons where the old skeleton is below and to the right of starter skeleton corresponding to $\alpha$, with the node corresponding to $\alpha(n)$ in the lowest possible position.

\begin{example}
	If $\alpha : [4] \to I \coprod J$ satisfies $\alpha^{-1}(I) = \set{1,3,4}$ and $\alpha^{-1}(J) = \set{2}$, and $j \in J$, then
	\[
		T_\alpha(e_{j}) = e_{\alpha(4)}\otimes e_{\alpha(3)} \otimes e_{j} \otimes e_{\alpha(2)} \otimes e_{\alpha(1)} +e_{\alpha(4)}\otimes e_{j} \otimes e_{\alpha(3)} \otimes  e_{\alpha(2)} \otimes e_{\alpha(1)}.
	\]
	This action corresponds to the following diagram:
	\[
		\begin{tikzpicture}
			\begin{scope}[xshift=-6cm]
				\draw[thick] (-1,1.75) -- (-1, -1.25);
				\draw[thick] (1,1.75) -- (1, -1.25);
				\draw[thick,dashed] (-1,-1.25) -- ++(0,-0.5);
				\draw[thick,dashed] (1,-1.25) -- ++(0,-0.5);

				\node (ballj) [draw, shade, circle, ball color=white, inner sep=0.07cm] at (1, 0) {};
				\node [right] at (ballj) {$j$};
				\draw[thick] (ballj) -| (0,-1.25);
				\draw[thick,dashed] (0,-1.25) -- ++(0,-0.5);
			\end{scope}

			\draw [thick,->] (-4.0,0) -- node [above] {$T_\alpha$} (-3.0,0);

			\begin{scope}[xshift=-1cm]
				\draw[thick] (-1,1.75) -- (-1, -1.25);
				\draw[thick] (1,1.75) -- (1,-1.25);
				\draw[thick,dashed] (-1,-1.25) -- ++(0,-0.5);
				\draw[thick,dashed] (1,-1.25) -- ++(0,-0.5);

				\def\sidez{{-1,1,-1,-1}}
				\foreach \nnn in {0,1,2,3} {
					\pgfmathtruncatemacro{\nodename}{\nnn+1}
				\pgfmathtruncatemacro{\sd}{\sidez[\nnn]}
				\ifthenelse{\nnn < 2}{\pgfmathtruncatemacro{\y}{\nnn}}{\pgfmathtruncatemacro{\y}{\nnn+2}}
				\node (ball\nodename) [draw, shade, circle, ball color=white, inner sep=0.07cm] at (\sd, 1.5-\y*0.5) {};
				\ifthenelse{\sd=1}{\node[right]}{\node[left]} at (ball\nodename) {$\alpha(\nodename)$};
				\draw[thick] (ball\nodename) -- (ball\nodename -| -0.5,0);
				}

				\draw [thick] (ball1 -| -0.5,0) -- (ball4 -| -0.5,0);

				\node (ballj) [draw, shade, circle, ball color=white, inner sep=0.07cm] at (1, 0) {};
				\node [right] at (ballj) {$j$};
				\draw[thick] (ballj) -| (0,-1.25);
				\draw[thick,dashed] (0,-1.25) -- ++(0,-0.5);
			\end{scope}

			\node at (1.5,0) {$+$};

			\begin{scope}[xshift=4cm]
				\draw[thick] (-1,1.75) -- (-1, -1.25);
				\draw[thick] (1,1.75) -- (1,-1.25);
				\draw[thick,dashed] (-1,-1.25) -- ++(0,-0.5);
				\draw[thick,dashed] (1,-1.25) -- ++(0,-0.5);

				\def\sidez{{-1,1,-1,-1}}
				\foreach \nnn in {0,1,2,3} {
					\pgfmathtruncatemacro{\nodename}{\nnn+1}
				\pgfmathtruncatemacro{\sd}{\sidez[\nnn]}
				\ifthenelse{\nnn < 3}{\pgfmathtruncatemacro{\y}{\nnn}}{\pgfmathtruncatemacro{\y}{\nnn+2}}
				\node (ball\nodename) [draw, shade, circle, ball color=white, inner sep=0.07cm] at (\sd, 1.5-\y*0.5) {};
				\ifthenelse{\sd=1}{\node[right]}{\node[left]} at (ball\nodename) {$\alpha(\nodename)$};
				\draw[thick] (ball\nodename) -- (ball\nodename -| -0.5,0);
				}

				\draw [thick] (ball1 -| -0.5,0) -- (ball4 -| -0.5,0);

				\node (ballj) [draw, shade, circle, ball color=white, inner sep=0.07cm] at (1, 0) {};
				\node [right] at (ballj) {$j$};
				\draw[thick] (ballj) -| (0,-1.25);
				\draw[thick,dashed] (0,-1.25) -- ++(0,-0.5);
			\end{scope}
		\end{tikzpicture}
\]
The purpose of allowing multiple diagrams is that the cumulant corresponding to a bi-non-crossing diagram for a sequence of operators is equal to the same cumulant for the sequence of operators obtained by interchanging the $k$-th and $(k+1)$-th operators and the $k$-th and $(k+1)$-th nodes in the bi-non-crossing diagram provided $k$ and $k+1$ are in different blocks and on different sides of the diagram.
%Thus multiple skeletons must be created simultaneously in order to facilitate all of the possible bi-non-crossing diagrams obtained by interchanging consecutive nodes on opposite sides of the diagrams from different blocks.
In the end, a sequence of annihilation operators can complete at most one skeleton and will produce the correct completed skeleton for a given sequence of operators.

Note that the node labelled $j$ above must have been introduced by some earlier operator, and so must be connected to something below the diagram.
It is not possible that $j$ is isolated; in bi-non-crossing partitions where $j$ is isolated, it will be added to the diagram by a later operator.


As a further example, suppose $\alpha : [3] \to I$ and $j_1, j_2 \in J$.
Then
\begin{align*}
	T_\alpha(e_{j_2}\otimes e_{j_1})
	&= e_{\alpha(3)} \otimes e_{\alpha(2)} \otimes e_{\alpha(1)} \otimes e_{j_2} \otimes e_{j_1}
	+ e_{\alpha(3)} \otimes e_{\alpha(2)} \otimes e_{j_2} \otimes e_{\alpha(1)} \otimes e_{j_1}\\
	&\quad + e_{\alpha(3)} \otimes e_{\alpha(2)} \otimes e_{j_2} \otimes e_{j_1} \otimes e_{\alpha(1)}
	+ e_{\alpha(3)} \otimes e_{j_2} \otimes e_{\alpha(2)} \otimes e_{\alpha(1)} \otimes e_{j_1} \\
	&\quad + e_{\alpha(3)} \otimes e_{j_2} \otimes e_{\alpha(2)} \otimes e_{j_1} \otimes e_{\alpha(1)}
	+ e_{\alpha(3)} \otimes e_{j_2} \otimes e_{j_1} \otimes e_{\alpha(2)} \otimes e_{\alpha(1)}
\end{align*}
This action corresponds to the following diagram:
\[
	\begin{tikzpicture}
		\begin{scope}[xshift=-6cm]
			\draw[thick] (-1,1.25) -- (-1, -1.25);
			\draw[thick] (1,1.25) -- (1, -1.25);
			\draw[thick,dashed] (-1,-1.25) -- ++(0,-0.5);
			\draw[thick,dashed] (1,-1.25) -- ++(0,-0.5);

			\node (ballj1) [draw, shade, circle, ball color=white, inner sep=0.07cm] at (1, 0.25) {};
			\node [right] at (ballj1) {$j_1$};
			\node (ballj2) [draw, shade, circle, ball color=white, inner sep=0.07cm] at (1, -0.25) {};
			\node [right] at (ballj2) {$j_2$};
			\draw[thick] (ballj1) -| (0.25,-1.25);
			\draw[thick] (ballj2) -- (ballj2 -| 0.25,-1.25);
			\draw[thick,dashed] (0.25,-1.25) -- ++(0,-0.5);
		\end{scope}

		\draw [thick,->] (-4.5,0) -- node [above] {$T_\alpha$} (-3.5,0);

		\def\doit{
			\draw[thick] (-1,1.75) -- (-1, -1.25);
		\draw[thick] (1,1.75) -- (1,-1.25);
		\draw[thick,dashed] (-1,-1.25) -- ++(0,-0.5);
		\draw[thick,dashed] (1,-1.25) -- ++(0,-0.5);

		\def\sidez{{1,1,-1,-1,-1}}
		\def\namez{{"$j_1$", "$j_2$", "$\alpha(1)$", "$\alpha(2)$", "$\alpha(3)$"}}
		\def\nodz{{"j1", "j2", "a1", "a2", "a3"}}
		\foreach \w in {0,...,4} {
			\pgfmathtruncatemacro{\y}{\ordr[\w]}
			\pgfmathtruncatemacro{\sd}{\sidez[\y]}
			\pgfmathparse{\nodz[\y]}
			\edef\nodenum{\pgfmathresult}
			\node (ball\nodenum) [draw, shade, circle, ball color=white, inner sep=0.07cm] at (\sd, 1.5-\w*0.5) {};
			\pgfmathparse{\namez[\y]}
			\ifthenelse{\sd=1}{\node[right]}{\node[left]} at (ball\nodenum) {\pgfmathresult};
		}

		\draw [thick] (balla1) -- ++(0.75,0) |- (balla3);
		\draw [thick] (balla2) -- ++(0.75,0);
		\draw[thick] (ballj1) -| (0.25,-1.25);
		\draw[thick] (ballj2) -- (ballj2 -| 0.25,-1.25);
		\draw[thick,dashed] (0.25,-1.25) -- ++(0,-0.5);
		}

		\begin{scope}[xshift=-1cm]
			\def\ordr{{0,1,2,3,4}}
			\doit
		\end{scope}

		\node at (.75,0) {$+$};

		\begin{scope}[xshift=3cm]
			\def\ordr{{0,2,1,3,4}}
			\doit
		\end{scope}

		\node at (4.75,0) {$+$};

		\begin{scope}[xshift=7cm]
			\def\ordr{{2,0,1,3,4}}
			\doit
		\end{scope}

	\begin{scope}[yshift=-4cm]
		\node at (-3.25,0) {$+$};
		\begin{scope}[xshift=-1cm]
			\def\ordr{{0,2,3,1,4}}
			\doit
		\end{scope}

		\node at (.75,0) {$+$};

		\begin{scope}[xshift=3cm]
			\def\ordr{{2,0,3,1,4}}
			\doit
		\end{scope}

		\node at (4.75,0) {$+$};

		\begin{scope}[xshift=7cm]
			\def\ordr{{2,3,0,1,4}}
			\doit
		\end{scope}
	\end{scope}

\end{tikzpicture}
\]
Note that we could equally well have drawn a diagram where $j_1$ and $j_2$ were not connected; however, these two diagrams correspond to proportional vectors and the difference between them is only on the scalar level.
\end{example}






Now, suppose that $\beta^{-1}(I) \neq \emptyset$, and let $k = \max\paren{\beta^{-1}(I)}$.
This corresponds to a partially completed skeleton with open nodes on both the left and right, where the lowest open node on the left is the $k^\mathrm{th}$ from the top.
We set $T_{\alpha}(\eta) = 0$ if $\alpha(t) \in J$ for some $t$, since the partially completed skeleton has open nodes on the left and right we cannot add the empty skeleton of $\alpha$ without introducing a crossing, since the lowest node of $\alpha$ is on the left.
Otherwise $\alpha(t) \in I$ for all $t \in \set{1,\ldots, n}$, and we set
\[T_{\alpha}(\eta) := e_{\alpha(n)}\otimes \Sigma\paren{ e_{\alpha(n-1)} \otimes \cdots \otimes e_{\alpha(1)}, e_{\beta(m)} \otimes \cdots \otimes e_{\beta(k+1)} } \otimes e_{\beta(k)} \otimes \cdots \otimes e_{\beta(1)}.
\]
One can think of this as the sum of all valid partially completed skeletons where the empty skeleton of $\alpha$ sits below the lowest open node on the left of the old skeleton.

\begin{example}
	If $\alpha : [2] \to I \coprod J$ has $\alpha(2) \in I$, $\alpha(1) \in J$ and $i \in I$, then
	$$T_\alpha(e_i) = 0.$$
	This is because there is no way to glue the empty skeleton corresponding to $\alpha$ into the partially completed skeleton without introducing a crossing while placing the lowest node of $\alpha$ at the bottom of the diagram (directly above the highest closed node):
	\[
		\begin{tikzpicture}[baseline]
			\begin{scope}[yshift=1cm]
				\draw[thick] (-1,0.25) -- (-1, -1.25);
				\draw[thick] (1,0.25) -- (1,-1.25);
				\draw[thick,dashed] (-1,-1.25) -- ++(0,-0.5);
				\draw[thick,dashed] (1,-1.25) -- ++(0,-0.5);

				\def\sidez{{-1,1,-1}}
				\foreach \y in {0,1,2} {
					\pgfmathtruncatemacro{\nodename}{\y+1}
				\pgfmathtruncatemacro{\sd}{\sidez[\y]}
				\node (ball\nodename) [draw, shade, circle, ball color=white, inner sep=0.07cm] at (\sd, -\y*0.5) {};
				}

				\node[left] at (ball1) {$i$};
				\node[right] at (ball2) {$\alpha(1)$};
				\node[left] at (ball3) {$\alpha(2)$};

				\draw[thick] (ball2) -- ++(-0.75,0) |- (ball3);
				\draw[line width=0.2cm,color=white] (-0.25,0) -- (-0.25,-1.2);
				\draw[thick] (ball1) -| (-0.25,-1.25);
				\draw[thick,dashed] (-0.25,-1.25) -- ++(0,-0.5);
			\end{scope}
		\end{tikzpicture}
	.
	\]

	If $\alpha : \set{1,2} \to I$ and $i \in I$, $j, j' \in J$ then
	\[
		T_\alpha \paren{e_j\otimes e_i\otimes e_{j'}} = e_{\alpha(2)} \otimes e_{\alpha(1)} \otimes e_{j}\otimes e_{i} \otimes e_{j'} + e_{\alpha(2)} \otimes e_{j}\otimes e_{\alpha(1)} \otimes  e_{i} \otimes e_{j'}.
	\]
	This action corresponds to the following diagram:
	\[
		\begin{tikzpicture}
			\def\clrs{{"white", "white", "white", "white", "white", "white"}}
			\def\foo{{1,-1,1,-1,-1,0}}
			\def\labelz{{"$j'$", "$i$", "$j$","$\alpha(1)$","$\alpha(2)$",""}}
			\def\nodenames{{"j1", "i", "j2", "1", "2", "0", "0"}}
			\begin{scope}[xshift=-6cm]
				\def\order{{0,1,2,5,5}}
				\bnc[n=5,sidez=\foo,colourz=\clrs,labelz=\labelz,drawbase=0,order=\order,nodez=\nodenames]
				\draw [thick,dashed] (bl) -- ++(0,-0.5)
						(br) -- ++(0,-0.5)
						(bl -| 0,0) -- ++(0, -0.5);
				\draw [thick] (ballj1) -- (ballj1 -| 0,0) |- (ballj2)
					(balli) -| (bl -| 0,0);
			\end{scope}

			\draw [thick,->] (-4.5,-1) -- node [above] {$T_\alpha$} (-3.5,-1);

			\begin{scope}[xshift=-1cm]
				\def\order{{0,1,2,3,4}}
				\bnc[n=5,sidez=\foo,colourz=\clrs,labelz=\labelz,drawbase=0,order=\order,nodez=\nodenames]
				\draw [thick,dashed] (bl) -- ++(0,-0.5)
						(br) -- ++(0,-0.5)
						(bl -| 0,0) -- ++(0, -0.5);
				\draw [thick] (ballj1) -- (ballj1 -| 0,0) |- (ballj2)
					(balli) -| (bl -| 0,0);
				\draw [thick] (ball1) -- ++(0.5,0) |- (ball2);
			\end{scope}

			\node at (.75,-1) {$+$};

			\begin{scope}[xshift=3cm]
				\def\order{{0,1,3,2,4}}
				\bnc[n=5,sidez=\foo,colourz=\clrs,labelz=\labelz,drawbase=0,order=\order,nodez=\nodenames]
				\draw [thick,dashed] (bl) -- ++(0,-0.5)
						(br) -- ++(0,-0.5)
						(bl -| 0,0) -- ++(0, -0.5);
				\draw [thick] (ballj1) -- (ballj1 -| 0,0) |- (ballj2)
					(balli) -| (bl -| 0,0);
				\draw [thick] (ball1) -- ++(0.5,0) |- (ball2);
			\end{scope}
		\end{tikzpicture}.\]
\end{example}

As $T_{\alpha}$ has been defined on an orthonormal basis, we may extend by linearity to obtain a densely defined operator on $\cH$; note that $T_\alpha$ may not be bounded due to the action of $\Sigma$.
On the other hand, if $\alpha : [n] \to I$ then $T_{\alpha}$ acts on the Fock subspace generated by $\set{e_i}_{i\in I}$ as $ \ell_{\alpha(n)} \cdots \ell_{\alpha(1)}$.
Thus, if one considers only left variables, the resulting operators are precisely those of Nica's model.

We define $T_\alpha$ in a similar manner when $\alpha(n) \in J$.

%%%%%%%%%%%%%%%%%%%%%%%%%%%%%%%%%%%%%%%%%%%%%%%%%%%%%%%%%%%%%%%%%%
\subsection{The operator model for pairs of faces.}
With the above construction, the operator model for a pair of faces is at hand.

\begin{theorem}
	\label{operatormodelforapairoffaces}
	Let $z = \paren{ \paren{z_i}_{i \in I}, \paren{z_j}_{j \in J}}$ be a pair of faces in a non-commutative probability space $(\A, \varphi)$.
	With notation as in Subsection~\ref{constructingtheoperatormodel}, consider the formal sum
	\[
		\Theta_z := I + \sum_{n\geq 1}\sum_{\alpha : [n] \to I \coprod J} \kappa_\alpha(z) T_{\alpha},
	\]
	and for $k \in I \coprod J$, set $Z_k := \ell_k^*\Theta_z$.
	If $T \in \mathrm{alg}(\set{Z_k}_{k \in I \coprod J})$ then $\ang{\Omega,T\Omega}$ is well-defined.
	Moreover, with respect to the vacuum state $\omega(T) = \ang{\Omega, T\Omega}$, the joint distribution of $\set{Z_k}_{k \in I \coprod J}$ is the same as the joint distribution of $z$ with respect to $\varphi$.
\end{theorem}

Before we begin the proof, we give the following example.
\begin{example}
	\label{exampletoshowoperatormodelworks}
	In this example, let $I = \set{1}$ and $J = \set{2}$.
	We will examine how the completed skeleton below is constructed for $Z_1Z_2Z_1Z_1Z_2Z_2Z_1Z_2Z_1Z_1$.
	\[\begin{tikzpicture}[baseline]
		\def\sdz{{-1,1,-1,-1,1,1,-1,1,-1,-1}}
		\def\labelz{{1,2,1,1,2,2,1,2,1,1}}
		\bnc[n=10,sidez=\sdz,labelz=\labelz]
		\draw[thick] (ball1) -- (ball1 -| -0.2,0) |- (ball9)
				(ball2) -- (ball2 -| -0.2,0) |- (ball3);
		\draw[thick] (ball4) -- (ball4 -| -0.6,0) |- (ball7);
		\draw[thick] (ball5) -- (ball5 -| 0.2,0) |- (ball10);
		\draw[thick] (ball6) -- (ball6 -| 0.6,0) |- (ball8);
	\end{tikzpicture}\]

	First $\kappa_{(21)}(z) \ell_1^*T_{(21)}$ is applied to $\Omega$ to get the partially completed skeleton
	\[\begin{tikzpicture}
		\def\sdz{{-1,-1,1,1}}
		\def\labelz{{1,1,2,2}}
		\def\colourz{{"black", "white", "black", "white"}}
		\def\order{{3,0}}
		\bnc[n=2,sidez=\sdz,labelz=\labelz,colourz=\colourz,order=\order]
		\draw[thick] (ball1) -- (ball1 -| 0,0) |- (ball2);
	\end{tikzpicture}.\]

	Then $\kappa_{(1211)}(z) \ell_1^*T_{(1211)}$ is applied to obtain the following collection of partially completed skeletons:
	\[\begin{tikzpicture}
		\def\sdz{{-1,-1,1,1}}
		\def\labelz{{1,1,2,2}}
		\def\colourz{{"black", "white", "black", "white"}}
		\begin{scope}[xshift=-2cm]
			\def\order{{1,3,3,1,0,0}}
			\bnc[n=6,sidez=\sdz,labelz=\labelz,colourz=\colourz,order=\order]
			\draw[thick] (ball1) -- (ball1 -| -0.25,0) |- (ball5)
					(ball2) -- (ball2 -| -0.25,0) |- (ball4);
			\draw[thick] (ball3) -- (ball3 -| 0.25,0) |- (ball6);
		\end{scope}
		\begin{scope}[xshift=2cm]
			\def\order{{1,3,1,3,0,0}}
			\bnc[n=6,sidez=\sdz,labelz=\labelz,colourz=\colourz,order=\order]
			\draw[thick] (ball1) -- (ball1 -| -0.25,0) |- (ball5)
					(ball2) -- (ball2 -| -0.25,0) |- (ball3);
			\draw[thick] (ball4) -- (ball4 -| 0.25,0) |- (ball6);
		\end{scope}
	\end{tikzpicture}.\]

	Applying $\kappa_{(22)}\ell_2^*T_{(22)}$ then gives the following collection of partially completed skeletons (where the first two below are from the first above and the third below is from the second above):
	\[\begin{tikzpicture}
		\def\sdz{{-1,-1,1,1}}
		\def\labelz{{1,1,2,2}}
		\def\colourz{{"black", "white", "black", "white"}}
		\begin{scope}[xshift=-3.5cm]
			\def\order{{1,3,3,3,1,2,0,0}}
			\bnc[n=8,sidez=\sdz,labelz=\labelz,colourz=\colourz,order=\order]
			\draw[thick] (ball1) -- (ball1 -| -0.2,0) |- (ball7)
					(ball2) -- (ball2 -| -0.2,0) |- (ball5);
			\draw[thick] (ball3) -- (ball3 -| 0.2,0) |- (ball8);
			\draw[thick] (ball4) -- (ball4 -| 0.6,0) |- (ball6);
		\end{scope}
		\begin{scope}[xshift=0cm]
			\def\order{{1,3,3,1,3,2,0,0}}
			\bnc[n=8,sidez=\sdz,labelz=\labelz,colourz=\colourz,order=\order]
			\draw[thick] (ball1) -- (ball1 -| -0.2,0) |- (ball7)
					(ball2) -- (ball2 -| -0.2,0) |- (ball4);
			\draw[thick] (ball3) -- (ball3 -| 0.2,0) |- (ball8);
			\draw[thick] (ball5) -- (ball5 -| 0.6,0) |- (ball6);
		\end{scope}
		\begin{scope}[xshift=3.5cm]
			\def\order{{1,3,1,3,3,2,0,0}}
			\bnc[n=8,sidez=\sdz,labelz=\labelz,colourz=\colourz,order=\order]
			\draw[thick] (ball1) -- (ball1 -| -0.2,0) |- (ball7)
					(ball2) -- (ball2 -| -0.2,0) |- (ball3);
			\draw[thick] (ball4) -- (ball4 -| 0.2,0) |- (ball8);
			\draw[thick] (ball5) -- (ball5 -| 0.6,0) |- (ball6);
		\end{scope}
	\end{tikzpicture}.\]
	Now applying $\kappa_{(11)}\ell_1^*T_{(11)}$ gives the following collection of partially completed skeletons (where the first below is from the first above, the second and third below are from the second above, and the last three are from the third above):

	\[\begin{tikzpicture}[xscale=0.75]
		\def\sdz{{-1,-1,1,1}}
		\def\labelz{{1,1,2,2}}
		\def\colourz{{"black", "white", "black", "white"}}

		\begin{scope}[xshift=-10cm]
			\def\order{{1,3,3,3,1,1,0,2,0,0}}
			\bnc[n=10,sidez=\sdz,labelz=\labelz,colourz=\colourz,order=\order]
			\draw[thick] (ball1) -- (ball1 -| -0.2,0) |- (ball9)
					(ball2) -- (ball2 -| -0.2,0) |- (ball5);
			\draw[thick] (ball6) -- (ball6 -| -0.6,0) |- (ball7);
			\draw[thick] (ball3) -- (ball3 -| 0.2,0) |- (ball10);
			\draw[thick] (ball4) -- (ball4 -| 0.6,0) |- (ball8);
		\end{scope}

		\begin{scope}[xshift=-6cm]
			\def\order{{1,3,3,1,3,1,0,2,0,0}}
			\bnc[n=10,sidez=\sdz,labelz=\labelz,colourz=\colourz,order=\order]
			\draw[thick] (ball1) -- (ball1 -| -0.2,0) |- (ball9)
					(ball2) -- (ball2 -| -0.2,0) |- (ball4);
			\draw[thick] (ball6) -- (ball6 -| -0.6,0) |- (ball7);
			\draw[thick] (ball3) -- (ball3 -| 0.2,0) |- (ball10);
			\draw[thick] (ball5) -- (ball5 -| 0.6,0) |- (ball8);
		\end{scope}

		\begin{scope}[xshift=-2cm]
			\def\order{{1,3,3,1,1,3,0,2,0,0}}
			\bnc[n=10,sidez=\sdz,labelz=\labelz,colourz=\colourz,order=\order]
			\draw[thick] (ball1) -- (ball1 -| -0.2,0) |- (ball9)
					(ball2) -- (ball2 -| -0.2,0) |- (ball4);
			\draw[thick] (ball5) -- (ball5 -| -0.6,0) |- (ball7);
			\draw[thick] (ball3) -- (ball3 -| 0.2,0) |- (ball10);
			\draw[thick] (ball6) -- (ball6 -| 0.6,0) |- (ball8);
		\end{scope}

		\begin{scope}[xshift=2cm]
			\def\order{{1,3,1,3,3,1,0,2,0,0}}
			\bnc[n=10,sidez=\sdz,labelz=\labelz,colourz=\colourz,order=\order]
			\draw[thick] (ball1) -- (ball1 -| -0.2,0) |- (ball9)
					(ball2) -- (ball2 -| -0.2,0) |- (ball3);
			\draw[thick] (ball6) -- (ball6 -| -0.6,0) |- (ball7);
			\draw[thick] (ball4) -- (ball4 -| 0.2,0) |- (ball10);
			\draw[thick] (ball5) -- (ball5 -| 0.6,0) |- (ball8);
		\end{scope}

		\begin{scope}[xshift=6cm]
			\def\order{{1,3,1,3,1,3,0,2,0,0}}
			\bnc[n=10,sidez=\sdz,labelz=\labelz,colourz=\colourz,order=\order]
			\draw[thick] (ball1) -- (ball1 -| -0.2,0) |- (ball9)
					(ball2) -- (ball2 -| -0.2,0) |- (ball3);
			\draw[thick] (ball5) -- (ball5 -| -0.6,0) |- (ball7);
			\draw[thick] (ball4) -- (ball4 -| 0.2,0) |- (ball10);
			\draw[thick] (ball6) -- (ball6 -| 0.6,0) |- (ball8);
		\end{scope}

		\begin{scope}[xshift=10cm]
			\def\order{{1,3,1,1,3,3,0,2,0,0}}
			\bnc[n=10,sidez=\sdz,labelz=\labelz,colourz=\colourz,order=\order]
			\draw[thick] (ball1) -- (ball1 -| -0.2,0) |- (ball9)
					(ball2) -- (ball2 -| -0.2,0) |- (ball3);
			\draw[thick] (ball4) -- (ball4 -| -0.6,0) |- (ball7);
			\draw[thick] (ball5) -- (ball5 -| 0.2,0) |- (ball10);
			\draw[thick] (ball6) -- (ball6 -| 0.6,0) |- (ball8);
		\end{scope}
	\end{tikzpicture}.\]

	Applying $\ell_2^*$ then gives the following collection of partially completed skeletons (where the first, second, and fourth diagrams above were destroyed):
	\[\begin{tikzpicture}
		\def\sdz{{-1,-1,1,1}}
		\def\labelz{{1,1,2,2}}
		\def\colourz{{"black", "white", "black", "white"}}

		\begin{scope}[xshift=-4cm]
			\def\order{{1,3,3,1,1,2,0,2,0,0}}
			\bnc[n=10,sidez=\sdz,labelz=\labelz,colourz=\colourz,order=\order]
			\draw[thick] (ball1) -- (ball1 -| -0.2,0) |- (ball9)
					(ball2) -- (ball2 -| -0.2,0) |- (ball4);
			\draw[thick] (ball5) -- (ball5 -| -0.6,0) |- (ball7);
			\draw[thick] (ball3) -- (ball3 -| 0.2,0) |- (ball10);
			\draw[thick] (ball6) -- (ball6 -| 0.6,0) |- (ball8);
		\end{scope}

		\begin{scope}[xshift=0cm]
			\def\order{{1,3,1,3,1,2,0,2,0,0}}
			\bnc[n=10,sidez=\sdz,labelz=\labelz,colourz=\colourz,order=\order]
			\draw[thick] (ball1) -- (ball1 -| -0.2,0) |- (ball9)
					(ball2) -- (ball2 -| -0.2,0) |- (ball3);
			\draw[thick] (ball5) -- (ball5 -| -0.6,0) |- (ball7);
			\draw[thick] (ball4) -- (ball4 -| 0.2,0) |- (ball10);
			\draw[thick] (ball6) -- (ball6 -| 0.6,0) |- (ball8);
		\end{scope}

		\begin{scope}[xshift=4cm]
			\def\order{{1,3,1,1,3,2,0,2,0,0}}
			\bnc[n=10,sidez=\sdz,labelz=\labelz,colourz=\colourz,order=\order]
			\draw[thick] (ball1) -- (ball1 -| -0.2,0) |- (ball9)
					(ball2) -- (ball2 -| -0.2,0) |- (ball3);
			\draw[thick] (ball4) -- (ball4 -| -0.6,0) |- (ball7);
			\draw[thick] (ball5) -- (ball5 -| 0.2,0) |- (ball10);
			\draw[thick] (ball6) -- (ball6 -| 0.6,0) |- (ball8);
		\end{scope}
	\end{tikzpicture}.\]

	Next, applying $\ell_2^*$ removes all but the last diagram to give:
	\[\begin{tikzpicture}
		\def\sdz{{-1,-1,1,1}}
		\def\labelz{{1,1,2,2}}
		\def\colourz{{"black", "white", "black", "white"}}
		\def\order{{1,3,1,1,2,2,0,2,0,0}}
		\bnc[n=10,sidez=\sdz,labelz=\labelz,colourz=\colourz,order=\order]
		\draw[thick] (ball1) -- (ball1 -| -0.2,0) |- (ball9)
				(ball2) -- (ball2 -| -0.2,0) |- (ball3);
		\draw[thick] (ball4) -- (ball4 -| -0.6,0) |- (ball7);
		\draw[thick] (ball5) -- (ball5 -| 0.2,0) |- (ball10);
		\draw[thick] (ball6) -- (ball6 -| 0.6,0) |- (ball8);
	\end{tikzpicture}.\]

	Applying $\ell_1^*\ell_2^*\ell_1^*\ell_1^*$ then gives us the desired diagram.
	We also see the diagram was weighted by 
	$$\kappa_{(21)}(z)\kappa_{(1211)}(z)\kappa_{(22)}\kappa_{(11)}(z)$$
	which is the correct product of bi-free cumulants for this bi-non-crossing partition.
\end{example}

\begin{proof}[Proof of Theorem \ref{operatormodelforapairoffaces}]
	Let $\alpha : [n] \to I \coprod J$.
	To see that
	\[
		\omega(Z_{\alpha(1)} \cdots Z_{\alpha(n)}) = \varphi(z_{\alpha(1)} \cdots z_{\alpha(n)}),
	\]
	we must demonstrate that the sum of over all
	\[
		A_{k} \in \set{\ell_{\alpha(k)}^*} \cup \set{\kappa_\beta(z) \ell_{\alpha(k)}^*T_{\beta} \, \mid \, \beta : \{1,\ldots, m} \to I \coprod J\}
	\]
	of
	\[
		\ang{\Omega, A_1\cdots A_n\Omega}
	\]
	is precisely $\varphi(z_{\alpha(1)} \cdots z_{\alpha(n)})$.
	(Note that $\ell_{\alpha(k)}^*T_\beta = 0$ unless $\beta(m) = \alpha(k)$.)
	This suffices as these are precisely the terms that appear in expanding the product $Z_{\alpha(1)}\cdots Z_{\alpha(n)}$.
	By construction $A_{1} \cdots A_{n}$ acting on $\Omega$ corresponds to creating a (sequence of) partially completed skeletons and $\ang{\Omega, A_1\cdots A_n\Omega}$ will be the weight of the skeleton if the skeleton is complete and otherwise will be zero.
	Since
	\[
		\varphi(z_{\alpha(1)} \cdots z_{\alpha(n)}) = \sum_{\pi \in \BNC(\chi_\alpha)}\kappa_{\pi}(z).
	\]
	it suffices to show that there is a bijection between completed skeletons and elements $\pi$ of $\BNC(\chi_\alpha)$, and that the weight of the skeleton is the corresponding cumulant.

	Observe that after $A_k$ is applied, the bottom $n-k+1$ nodes of the partially completed skeleton will be closed, as $A_k$ itself either closed an open node which was already present or added a new starter skeleton containing one closed node and zero or more open nodes.
	In particular, the $(n-k+1)$-th node from the bottom must be on the side corresponding to $\alpha(k)$ since it was closed by $\ell_{\alpha(k)}^*$.
	Thus when we have applied $A_1\cdots A_n$, any completed skeleton surviving has precisely $n$ nodes and structure arising from $\alpha$.

	From a bi-non-crossing partition $\pi \in BNC(\alpha)$, we can recover the choice of $A_1, \ldots, A_n$ which produces it.
	To do so, for each block $V = \set{k_1 < \ldots < k_t}$, we let $A_{k_i} = \ell_{k_i}^*$ for $i \neq t$, and with $\beta_V(i) = \alpha(k_i)$, we set $A_{k_t} = \kappa_{\beta_V}(z)\ell_{k_t}^*T_{\beta_V}$.
	Indeed, the partially created skeletons created by $A_k\cdots A_n$ agree with $\pi$ on the bottom $n-k+1$ nodes.
	Moreover, given any other product $A_1'\cdots A_n'$ which differs from $A_1\cdots A_n$, consider the greatest index $k$ so that $A_k'\neq A_k$.
	Then all partially completed skeletons in $A_k'\cdots A_n'$ and $A_k\cdots A_n$ agree in structure for their bottom $n-k$ nodes, while the next either starts a new block in one case but not the other or starts new blocks of different shapes; thus there is only one sequence of choices of $A_j$ for each bi-non-crossing partition.
	Finally, note that if $\beta_V$ corresponds to the block $V \in \pi$ as above, then $\kappa_{\beta_V}(z) = \kappa_{\pi|_V}(z)$ and so the total weight on the skeleton is precisely $\kappa_\pi(z)$.
\end{proof}


\begin{remark}
	In Theorem 7.4 of \cite{voiculescu2014free}, an operator model for the bi-free central limit distributions was given as sums of creation and annihilation operators on a Fock space as follows.
	Let $C = (C_{k,l})_{k,l\in I\coprod J}$ be a matrix with complex entries, and let $h, h^* : I \coprod J \to \cH$ be maps into a Hilbert space $\cH$ so that $C_{k,l} = \ang{h^*(k), h(l)}$.
	Then if we define
	$$z_i = \ell(h(i)) + \ell^*(h^*(i)) \text{ for } i \in I \qquad\text{ and }\qquad
	z_j = r(h(j)) + r^*(h^*(j)) \text{ for } j \in J,$$
	we find that $z_i$ and $z_j$ are bi-free central limit distributions with covariance $\omega(z_kz_l) = C_{k,l}$ (that is, the only non-vanishing cumulants for $z$ are those of second order, which are given by the matrix $C$).

	It is interesting that the operator model from Theorem \ref{operatormodelforapairoffaces} uses different operators.
	Indeed for $i, i' \in I$ and $j \in J$, one can check that
	\[
		T_{(i,i')} = \sum_{n\geq 0} \sum_{\alpha : [n] \to J} \ell_{i'}\ell_{\alpha(1)} \cdots \ell_{\alpha(n)} \ell_{i} \ell^*_{\alpha(n)} \cdots \ell^*_{\alpha(1)}
	\]
	and
	\[
		T_{(j,i')} = \ell_{i'}r_j P
	\]
	where $P$ is the projection onto the Fock subspace of $\cH$ generated by $\set{e_j}_{j \in J}$, and $r_j$ is the right creation operator corresponding to $e_j$.
	Therefore, if $C_{k_1, k_2} = \varphi(z_{k_1}z_{k_2})$ for $k_1, k_2 \in I \coprod J$ with $z$ a bi-free central limit distribution, Theorem \ref{operatormodelforapairoffaces} produces the operators
	\[
		Z_k = \ell_k^* + \sum_{k' \in I \coprod J} C_{k', k} \ell_{k}^*T_{(k',k)}
	\]
	which are very different from $\ell_k + \ell_k^*$ (if $k \in I$) and $r_k+r_k^*$ (if $k \in J$) proposed in \cite{voiculescu2014free}.
	The main issues with the model involving $\set{\ell_i, \ell^*_i, r_j, r^*_j, :  i \in I, j \in J}$ is that the vectors obtained by applying the algebra generated by these operators to $\Omega$ do not generate the full Fock space: indeed, they only generate vectors of the form
	\[
		e_{i_1} \otimes \cdots \otimes e_{i_n} \otimes e_{j_m} \otimes \cdots \otimes e_{j_1}
	\]
	where $n,m \geq 0$, $i_1, \ldots, i_n \in I$, and $j_1, \ldots, j_m \in J$.
	It is not difficult to see that the vectors obtained by the algebra generated by $\set{L_i^*, L_j^*, T_{(i,i)}, T_{(j,j)} : i \in I, j \in J}$ applied to $\Omega$ generate the full Fock space.
\end{remark}











\section{Additional cases of bi-free independence.}
\label{sec:morebifreeexamples}
We take a moment to describe some useful techniques for producing examples of families which are bi-freely independent.
The techniques covered here apply in the broader context of Chapter~\ref{ch:abfp}, but are more easily stated in this scalar-valued setting, and so we will cover them before proceeding to that greater generality.

\subsection{Conjugation by a Haar pair of unitaries.}
\label{ssec:haarunitary}
Recall that a Haar unitary is a unitary $u \in \A$ such that $\varphi(u^k) = 0$ for non-zero $k \in \Z$.
A concrete example is given by pointwise multiplication by $e^{2\pi i x}$ on $L^2([0, 1], d\lambda)$.
The following result is well-known, and motivates us to search for a similar statement in the bi-free settings.

\begin{proposition}
	Suppose that $(\A, \varphi)$ is a non-commutative probability space, and $B \subset \A$ is free from a Haar unitary $u \in \A$.
	Then $\paren{u^k\A u^{-k}}_{k\in\Z}$ are free, and $u^k\A u^{-k}$ is equal in distribution to $\A$.
\end{proposition}

We will not sketch a proof of this proposition here, though it will follow from our bi-free result at the end of this section.

\begin{definition}
	Suppose $(\A, \varphi)$ is a non-commutative probability space.
	A Haar pair of unitaries is a pair $(u_\ell, u_r)$ of invertible elements of $A$ which agree in distribution with the pair $(u, u^*)$ where $u$ is a Haar unitary; that is, if for every word $f$ in non-commuting indeterminates $X$, $X^*$, $Y$, and $Y^*$, we have $\varphi(f(u_\ell, u_\ell^*, u_r, u_r^*)) = 0$ unless
	$$\deg(X) + \deg(Y^*) = \deg(X^*) + \deg(Y),$$
	and in that case $\varphi(f(u_\ell, u_\ell^*, u_r, u_r^*)) = 1$.
\end{definition}
We remark here that although $u_r$ is distributed as $u_\ell^*$, they are not necessarily equal.
This is necessary because we will want to take things to be bi-free from the pair $(u_\ell, u_r)$, but any left variable bi-free from that pair must commute in distribution with $u_r$ while being free from $u_\ell$.
Indeed, if we were to begin with the pair $(u, u^*)$ and take its bi-free product with a pair $(\A_\ell, \A_r)$ we would find that $u$ is represented as $\lambda(u)$ while $u^*$ is as $\rho(u^*)$ on the free product space, and certainly $\lambda(u)^* = \lambda(u^*) \neq \rho(u^*)$.

\begin{theorem}
	\label{thm:bihaarconj}
	Suppose that $(\A, \varphi)$ is a non-commutative probability space, and $(\A_\ell, \A_r) \subset \A$ is a pair of faces bi-free from a Haar pair of unitaries $(u_\ell, u_r)$ in $\A$.
	Then $\paren{(u_\ell^k\A_\ell u_\ell^{-k}, u_r^k \A_r u_r^{-k})}_{k\in\Z}$ are bi-free, and identically distributed.
\end{theorem}

% \begin{proof}
% 	We will first show that each individual pair of faces has the same distribution as the original.
% 	To that end, fix $k \in \Z\setminus\set0$, $\chi : [n] \to \slr$, and $x_i \in \A_{\chi(i)}$.
% 	Note that if $\chi$ is constant, we have
% 	$$\varphi\paren{u_{\chi(1)}^k x_1 u_{\chi(1)}^{-k} \cdots u_{\chi(n)}^k x_n u_{\chi(n)}^{-k}} = \varphi\paren{u_{\chi(1)}^k x_1\cdots x_n u_{\chi(n)}^{-k}}.$$
% 	Then from freeness, and using the fact that $\varphi(u_{\chi(1)}^k) = 0 = \varphi(u_{\chi(n)}^k)$, we have that this agrees with $\varphi(x_1\cdots x_n)$.
% 	(Note that we do not require $\varphi$ to be tracial for this result, although the proof is even simpler with that additional assumption.)
% 
% 	Therefore let us assume $\chi$ is non-constant.
% 	Let $y = u_{\chi(1)}^k x_1 u_{\chi(1)}^{-k} \cdots u_{\chi(n)}^k x_n u_{\chi(n)}^{-k}$; we wish to show that
% 	$$\varphi\paren{y} = \varphi\paren{x_1\cdots x_n}.$$
% 	Notice first of all that since $u_{\ell}$ commutes with $u_r$ and $\A_r$, we may collect and cancel all except the first and last instance of $u_\ell$'s; we may do the same with the $u_r$'s.
% 	We may then move the remaining $u_\ell$ and $u_r$ terms to the top and bottom of the corresponding bi-non-crossing diagram and we conclude that 
% 	$$\varphi\paren{y} = \varphi\paren{u_{\ell}^ku_r^k x_1x_2\cdots x_n u_\ell^{-k}u_r^{-k}}.$$
% 	%Now let $\lambda_\ell, \lambda_r$ be so that if $x_i' = x_i - \lambda_{\chi(i)}$, then the left and right $\chi$-intervals of $x_1'\cdots x_n'$ are centred.
% 	%From vaccine we have
% 	%$$0 = \varphi\paren{u_{\ell}^ku_r^k x_1'x_2'\cdots x_n' (u_\ell^{-k}-1)(u_r^{-k}+1)};$$
% 	%but since $\varphi(u_\ell^{-k}) = 0 = \varphi(u_r^{k})$, only two of the four terms arising from expanding the last multiplication above survive and we have
% 	%$$\varphi\paren{u_{\ell}^ku_r^k x_1'x_2'\cdots x_n' u_\ell^{-k}u_r^{-k}} = \varphi\paren{u_{\ell}^ku_r^k x_1'x_2'\cdots x_n'}.$$
% 	%But now 
% 	Now, consider the moment-cumulant formula applied to the above situation.
% 	Let $\chi' : [n+4] \to \slr$ be so that $\chi'(x+2) = \chi(x)$ for $x \in [n]$, $\chi'(1) = \chi'(n+3) = \ell$, and $\chi'(2) = \chi'(n+4) = r$.
% 	By bi-freeness, all mixed cumulants vanish; as $\varphi(u_\ell^{\pm k}) = 0 = \varphi(u_r^{\pm k})$, all cumulants in which any of these four terms is isolated also vanish.
% 	%Let us divide the remaining portions of $\BNC(\chi')$ based on their restriction to the nodes $1,2,n+3$, and $n+4$: let $S_=, S_{||}$, and $S_{\mathrm{I}}$ be those elements of $\BNC(\chi')$ containing, respectively, the blocks: $\set{1,2}$ and $\set{n+3,n+4}$; $\set{1,n+3}$ and $\set{2,n+4}$; and $\set{1,2,n+3,n+4}$.
% 	%Then
% 	%$$\varphi\paren{u_{\chi(1)}^k x_1 u_{\chi(1)}^{-k} \cdots u_{\chi(n)}^k x_n u_{\chi(n)}^{-k}}
% 	%= \paren{\sum_{\pi \in S_=} + \sum_{\pi\in S_{||}} + \sum_{\pi\in S_{\mathrm{I}}} } \kappa_\pi\paren{u_\ell^k, u_r^k, x_1, \ldots, x_n, u_\ell^{-k}, u_r^{-k}}.$$
% 	Let $S \subset \BNC(\chi)$ consist of those $\chi$-non-crossing partitions in which no left node connects to any right node.
% 	Given $\pi \in S$, let $\pi_=, \pi_{||},$ and $\pi_{\mathrm{I}} \in \BNC(\chi')$ be the partitions gained by putting the partition $\pi$ on the nodes $\set{3, \ldots, n+2}$, and adding the blocks $\set{1,2}$ and $\set{n+3, n+4}$, $\set{1,n+3}$ and $\set{2, n+4}$, or $\set{1,2,n+3,n+4}$, respectively.
% 	Note that $\pi_= \in \BNC(\chi')$ for any $\pi \in \BNC(\chi)$, not just $\pi \in S$.
% 	We have the following:
% 	\begin{align*}
% 		\varphi\paren{y}
% 		&= \sum_{\pi \in S} (\kappa_{\pi_=} + \kappa_{\pi_{||}} + \kappa_{\pi_{\mathrm{I}}})\paren{u_\ell^k, u_r^k, x_1, \ldots, x_n, u_\ell^{-k}, u_r^{-k}}
% 	+\sum_{\pi \in \BNC(\chi)\setminus S} \kappa_{\pi_=}\paren{u_\ell^k, u_r^k, x_1, \ldots, x_n, u_\ell^{-k}, u_r^{-k}}.\\
% 	\end{align*}
% 	In the above equation, we can factor the terms corresponding to the $u$'s out of the cumulants.
% 	In the first sum, the combine to give $\varphi(u_\ell^k u_r^k u_\ell^{-k} u_r^{-k}) = 1$, while in the second, they combine to give $\varphi(u_\ell^k u_r^k)\varphi(u_\ell^{-k}u_r^{-k}) = 1$; both of these occur because these are the sums of the only non-vanishing cumulants under the respective partitions of $\set{1,2,n+3,n+4}$.
% 	We are then left with
% 	$$\varphi(y) = \sum_{\pi \in \BNC(\chi)} \kappa_\pi(x_1, \ldots, x_n) = \varphi(x_1\cdots x_n).$$
% 
% 
% 
% 	We now wish to show bi-free independence.
% 	Let $n \geq 1$, $\chi : [n] \to \set{\ell, r}$, and $k_1, \ldots, k_n \in \Z$.
% 	Choose $z_1, \ldots, z_n \in \A$ so that $z_i = u_{\chi(i)}^{k_i} x_i u_{\chi(i)}^{-k_i}$ with $x_i \in \A_{\chi(i)}$, and suppose that whenever $\set{i_1 < \cdots < i_k}$ is a maximal monochromatic $\chi$-interval, $\varphi(z_{i_1}\cdots z_{i_k}) = 0$.
% 	By the above, this is equivalent to saying $\varphi(x_{i_1}\cdots x_{i_k}) = 0$.
% 	But now once again using the fact that $u$'s commute with everything on the opposite side of the diagram, we may move them about to assuem that they only occur at the ends of $\chi$-intervals or the bottom of the diagram.
% 	In particular, ....
% \end{proof}

\begin{proof}
	We will first show that each individual pair of faces has the same distribution as the original.
	To that end, fix $k \in \Z\setminus\set0$, $\chi : [n] \to \slr$, and $x_i \in \A_{\chi(i)}$.
	Note that if $\chi$ is constant, we have
	$$\varphi\paren{u_{\chi(1)}^k x_1 u_{\chi(1)}^{-k} \cdots u_{\chi(n)}^k x_n u_{\chi(n)}^{-k}} = \varphi\paren{u_{\chi(1)}^k x_1\cdots x_n u_{\chi(n)}^{-k}}.$$
	Then from freeness, and using the fact that $\varphi(u_{\chi(1)}^k) = 0 = \varphi(u_{\chi(n)}^k)$, we have that this agrees with $\varphi(x_1\cdots x_n)$.
	(Note that we do not require $\varphi$ to be tracial for this result, although the proof is even simpler with that additional assumption.)

	Therefore let us assume $\chi$ is non-constant.
	Let us further assume that our representation is on a free product space with $(u_\ell, u_r)$ the image under $\lambda$ and $\rho$ of a pair $(u, u^*)$, so in particular, $\varphi(Tu_\ell^{k}u_r^{k}) = \varphi(T)$ for any $T \in \A$ (since $\lambda(u^k)\rho\paren{u^{-k}}\xi = \lambda(u^ku^{-k})\xi = \xi$).
	Similarly, we have $\varphi(u_\ell^k r_r^k T) = \varphi(T)$ as is $T\xi = \lambda\xi + \eta$ with $\eta \in \oX$ then $u_\ell^k u_r^k T\xi = \lambda\xi + \lambda(u^k)\rho(u^{-k})\eta$ and the later term remains in $\oX$.
	Then take $y = u_{\chi(1)}^k x_1 u_{\chi(1)}^{-k} \cdots u_{\chi(n)}^k x_n u_{\chi(n)}^{-k}$; we wish to show that
	$$\varphi\paren{y} = \varphi\paren{x_1\cdots x_n}.$$
	Notice that since $u_{\ell}$ commutes with $u_r$ and $\A_r$, we may collect and cancel all except the first and last instance of $u_\ell$'s; we may do the same with the $u_r$'s.
	We may then move the remaining $u_\ell$ and $u_r$ terms to the top and bottom of the corresponding bi-non-crossing diagram and we conclude that 
	$$\varphi\paren{y} = \varphi\paren{u_{\ell}^ku_r^k x_1x_2\cdots x_n u_\ell^{-k}u_r^{-k}} = \varphi(x_1\cdots x_n).$$

	We now wish to show bi-free independence.
	Let $n \geq 1$, $\chi : [n] \to \set{\ell, r}$, and $k_1, \ldots, k_n \in \Z$.
	Choose $z_1, \ldots, z_n \in \A$ so that $z_i = u_{\chi(i)}^{k_i} x_i u_{\chi(i)}^{-k_i}$ with $x_i \in \A_{\chi(i)}$, and suppose that whenever $\set{i_1 < \cdots < i_k}$ is a maximal monochromatic $\chi$-interval, $\varphi(z_{i_1}\cdots z_{i_k}) = 0$.
	By the above, this is equivalent to saying $\varphi(x_{i_1}\cdots x_{i_k}) = 0$.
	Using the same tricks as in the first part of the argument, we may cancel all $u$'s which do not occur at the beginning or end of a $\chi$-interval.
	Then vaccine together with the bi-freeness of $(\A_\ell, \A_r)$ and $(u_\ell, u_r)$ tells us that $\varphi(z_1\cdots z_n) = 0$.
\end{proof}

\subsection{Bi-free independence for bipartite pairs of faces.}
A family of pairs of faces $\fpf$ is said to be \emph{bipartite} if $\sq{\A_\ell^{(i)}, \A_r^{(j)}} = 0$ for every $i, j \in \I$.
In \cite{voiculescu2014free,voiculescu2016free}, Voiculescu made special focus on bipartite families of pairs of faces, as this additional assumption provides much more power for determining their behaviour.
For example, to fully describe the joint distribution of such a family it suffices to specify the two-bands moments: the moments of a product of left variables followed by a product of right variables.
We will show that in this context it can be much simpler to verify bi-free independence.

\begin{theorem}
	\label{thm:bipartitefreetobifree}
	Let $\fpf$ be a bipartite family of pairs of faces in a non-commutative probability space $(\A, \varphi)$, acting on a vector space with specified state vector $(X, \oX, \xi)$.
	Suppose, further, that for every $\iota \in \I$ and every $T \in \A_r^{(\iota)}$ there is $S \in \A_\ell^{(\iota)}$ such that $T\xi = S\xi$.
	Then $\fpf$ are bi-free if and only if $\paren{\A_\ell^{(\iota)}}_{\iota\in\I}$ are free.
\end{theorem}

\begin{proof}
	As we remarked in Subsection~\ref{ss:introbifree} of the introduction, bi-freeness of $\fpf$ implies the freeness of the left faces.
	It is up to us here to demonstrate the converse.
	%We will do this by verifying that the identity from Corollary~\ref{cor:bifreemob} holds, which we will do by induction on the number of right operators.

	Notice that if $T, S$ are as in the statement of the theorem, the condition $T\xi = S\xi$ tells us that for any $Q \in \A$ we have $\varphi(QT) = \varphi(QS)$.
	Now, take $\chi : [n] \to \slr$ and $\iota : [n] \to \I$, and let $z_i \in \A_{\chi(i)}^{(\iota(i))}$ be such that whenever $\set{i_1 < \cdots < i_k}$ is a maximal monochromatic $\chi$-interval, $\varphi(z_{i_1}\cdots z_{i_k}) = 0$.
	As all left variables commute with all right variables, we may assume that there is some $0 \leq a \leq n$ so that $\chi(i) = \ell$ if and only if $i \leq a$.
	For $i > a$, let $y_i$ be so that $z_i\xi = y_i$.
	Then $\varphi(z_{1}\cdots z_{n}) = \varphi(z_1\cdots z_a y_n \cdots y_{a+1})$ by repeatedly changing the rightmost $z$ to its corresponding $y$, and then commuting it past the remaining right-side $z$'s.
	Notice that this also preserves the centredness of the $\chi$-intervals we care about, as each operation of commuting or replacing a right operator by a left does not affect the moment of the product, while the same operators remain in the same intervals throughout.
	Having done this, grouping adjacent operators which come from the same family yields us with an alternating product of centred left variables, which vanishes by ordinary freeness.
	Thus $0 = \varphi(z_1\cdots z_a y_n \cdots y_{a+1}) = \varphi(z_1\cdots z_n)$ and vaccine implies that $\fpf$ are bi-free.
\end{proof}

\begin{corollary}
	Let $\fpf$ be a family of pairs of faces in a non-commutative probability space $(\A, \varphi)$ acting on a vector space with specified state vector $(X, \oX, \xi)$.
	Suppose that for every $\iota\in\I$, $\A^{(\iota)}_\ell \xi = \A^{(\iota)}_r$.
	Then the following three conditions are equivalent:
	\begin{itemize}
		\item the family $\paren{\A_\ell^{(\iota)}}_{\iota\in\I}$ is free;
		\item the family $\paren{\A_r^{(\iota)}}_{\iota\in\I}$ is free; and
		\item the family of pairs of faces $\fpf$ is bi-free.
	\end{itemize}
\end{corollary}

% We close this section with the brief remark that although vaccine made the above proofs easier, it was not essential.
% In the context of Theorem~\ref{thm:bihaarconj}, bi-free indepdence can be shown to arise under conjugation by Haar pairs of unitaries by representing them explicitly as left and right shift operators on $\ell^2(\Z)$, and carefully tracking their actions on a free product representation of $\A$.
% The $x$ portion of each $u_{\chi(i)}^{k_i} x_i u_{\chi(i)}^{-k_i}$ will only be able to interact with another if teh $u$'s exactly cancel, and one can find an isomorphism of vector spaces with specified state vectors to verify that the action of $\paren{u_\ell^{k}\A_\ell u_\ell^{-k}, u_r^k\A_r u_r^{-k}}_{k\in\Z}$ on the state vector in a free product representation corresponding to $\paren{\A_\ell\st\A_r}\st\paren{B(\ell^2(\Z))}$ matches that of $\Z$-many copies of $(\A_\ell, \A_r)$ acting on the state vector in $\st_{k\in\Z} (\A_\ell\st\A_r)$.
% 
% Similarly, Theorem~\ref{thm:bipartitefreetobifree} may be demonstrated by showing via induction on the number of right variables that the condition from Corollary~\ref{cor:bifreemob} holds.


\chapter{Amalgamated bi-free probability.}
\label{ch:abfp}
We now turn to an examination of the amalgamated setting of bi-free probability.
Section~8 of \cite{voiculescu2014free} laid the framework for generalizing $\cB$-valued free probability to the bi-free setting; our goal here is to make use of the combinatorial tools we have developed in Chapter~\ref{ch:bfi} to explore operator-valued bi-free probability in much greater depth.



















%%%%%%%%%%%%%%%%%%%%%%%%%%%%%%%%%%%%%%%%%%%%%%%%%%%%%%%%%%%%%%%%%%%%
%	   Bi-Free Families with Amalgamation		   %
%%%%%%%%%%%%%%%%%%%%%%%%%%%%%%%%%%%%%%%%%%%%%%%%%%%%%%%%%%%%%%%%%%%%
\section{Bi-free families with amalgamation.}
\label{sec:bifreefamilieswithamalgamation}



In this section, we will recall and develop the structures from \cite{voiculescu2014free}*{Section 8} necessary to discuss bi-freeness with amalgamation.
Throughout, $\cB$ will denote a unital algebra over $\C$.



%%%%%%%%%%%%%%%%%%%%%%%%%%%%%%%%%%%%%%%%%%%%%%%

\subsection{Concrete structures for bi-free probability with amalgamation.}


To begin the necessary constructions in the amalgamated setting, we need an analogue of a vector space with a specified vector state.
\begin{definition}
	A \emph{$\cB$-$\cB$-bimodule with a specified $\cB$-vector state} is a triple $(\cX, \ocX, p)$ where $\cX$ is a direct sum of $\cB$-$\cB$-bimodules
	\[
		\cX = \cB \oplus \ocX,
	\]
	and $p : \cX \to \cB$ is the linear map
	\[
		p(b \oplus \eta) = b.
	\]
\end{definition}
Given a $\cB$-$\cB$-bimodule with a specified $\cB$-vector state $(\cX, \ocX, p)$, for $b_1, b_2 \in \cB$ and $\eta \in \cX$ we have
\[
	p(b_1 \cdot \eta \cdot b_2) = b_1 p(\eta) b_2.
\]


\begin{definition}
	Given a $\cB$-$\cB$-bimodule with a specified $\cB$-vector state $(\cX, \ocX, p)$, let $\cL(\cX)$ denote the set of linear operators on $\cX$.
	Given $b \in \cB$, we define two operators $L_b, R_b \in \cL(\cX)$ by
	\[
		L_b(\eta) = b \cdot \eta \qquad \text{ and } \qquad R_b(\eta) = \eta \cdot b \qquad \text{ for } \eta \in \cX.
	\]

	In addition, we define the unital subalgebras $\cL_\ell(\cX)$ and $\cL_r(\cX)$ of $\cL(\cX)$ by
	\begin{align*}
		\cL_\ell(\cX) &:= \set{ T \in \cL(\cX) \, \mid \, TR_b = R_b T \text{ for all }b \in \cB}\\
		\cL_r(\cX) &:= \set{ T \in \cL(\cX) \, \mid \, TL_b = L_b T \text{ for all }b \in \cB}.
	\end{align*}
	We call $\cL_\ell(\cX)$ and $\cL_r(\cX)$ the \emph{left} and \emph{right algebras} of $\cL(\cX)$, respectively.
\end{definition}

Note $\cL_\ell(\cX)$ consists of all operators in $\cL(\cX)$ that are right $\cB$-linear; that is, if $T \in \cL_\ell(\cX)$ then
\[
	T( \eta \cdot b) = T(R_b(\eta)) = R_b(T(\eta)) = T(\eta) \cdot b
\]
for all $b \in \cB$ and $\eta \in \cX$.
%In the usual treatment of bimodules, what we have denoted $\cL_\ell(\cX)$ would instead be $\cL_r(\cX)$ and vice versa, to reflect the fact that its elements are right $\cB$-linear.
This may seem counter-intuitive; however, we take our left (resp. right) face to be a sub-algebra of $\cL_\ell(\cX)$ (resp. $\cL_r(\cX)$), and we would like to think of right multiplication by $\cB$ as a right variable.
One sees from the $\cB$-$\cB$-bimodule structure that $b \mapsto L_b$ is a homomorphism, $b \mapsto R_b$ is an anti-homomorphism, and the ranges of these maps commute.
Hence
\[
	\set{L_b \, \mid \, b \in \cB} \subseteq \cL_\ell(\cX) \qquad \text{and} \qquad \set{R_b \, \mid \, b \in \cB} \subseteq \cL_r(\cX).
\]
Thus, in the context of this paper, $\cL_\ell(\cX)$ consists of `left' operators and $\cL_r(\cX)$ consists of `right' operators.

As we are interested in $\cL(\cX)$ and amalgamating over $\cB$, we will need an ``expectation'' from $\cL(\cX)$ to $\cB$.
\begin{definition}
	\label{defn:expectationofLXontoB}
	Given a $\cB$-$\cB$-bimodule with a specified $\cB$-vector state $(\cX, \ocX, p)$, we define the linear map $E_{\cL(\cX)} : \cL(\cX) \to \cB$ by
	\[
		E_{\cL(\cX)}(T) = p(T(1_\cB \oplus 0))
	\]
	for all $T \in \cL(\cX)$.
	We call $E_{\cL(\cX)}$ the \emph{expectation of $\cL(\cX)$ onto $\cB$}.
\end{definition}
The following important properties justify calling $E_{\cL(\cX)}$ an expectation.

\begin{proposition}
	\label{prop:propertiesofEforLX}
	Let $(\cX, \ocX, p)$ be a $\cB$-$\cB$-bimodule with a specified $\cB$-vector state.
	Then
	\[
		E_{\cL(\cX)}(L_{b_1} R_{b_2} T) = b_1 E_{\cL(\cX)}(T) b_2
	\]
	for all $b_1, b_2 \in \cB$ and $T \in \cL(\cX)$, and
	\[
		E_{\cL(\cX)}(TL_b) = E_{\cL(\cX)}(TR_b)
	\]
	for all $b \in \cB$ and $T \in \cL(\cX)$.
\end{proposition}

\begin{proof}
	If $b_1, b_2 \in \cB$ and $T \in \cL(\cX)$, we see that
	\begin{align*}
		E_{\cL(\cX)}(L_{b_1} R_{b_2} T)
		&= p(L_{b_1} R_{b_2} T(1_\cB \oplus 0))
		= p(L_{b_1} R_{b_2} (E(T) \oplus \eta)) \\
		&\qquad\qquad = p( (b_1 E(T) b_2) \oplus (b_1 \cdot \eta \cdot b_2))
		= b_1 E(T) b_2
	\end{align*}
	for some $\eta \in \ocX$. The second result holds as $L_b(1_\cB\oplus 0) = b = R_b(1_\cB\oplus0)$.
\end{proof}

To complete this section, we recall the construction of the reduced free product of $\cB$-$\cB$-bimodules with specified $\cB$-vector states.
This will be similar in spirit to the construction of a free product of vector spaces with specified state vectors from Subsection~\ref{ssec:freeind}.

\begin{construction}
	\label{cons:freeproductconstruction}
	Let $\set{(\cX_{\iota}, \ocX_{\iota}, p_{\iota})}_{\iota \in \I}$ be $\cB$-$\cB$-bimodules with specified $\cB$-vector states.
	For simplicity, let $E_{\iota}$ denote $E_{\cL(\cX_{\iota})}$.
	The \emph{free product of $\set{(\cX_{\iota}, \ocX_{\iota}, p_{\iota})}_{\iota \in \I}$ with amalgamation over $\cB$} is defined to be the $\cB$-$\cB$-bimodule with specified vector state $(\cX, \ocX, p)$
	where $\cX = \cB \oplus \ocX$ and $\ocX$ is the $\cB$-$\cB$-bimodule
	\begin{align*}
		\ocX=\bigoplus_{n\geq 1}\bigoplus_{k_1\neq k_2\neq\cdots\neq k_n} \ocX_{k_1}\otimes_\cB\cdots\otimes_\cB\ocX_{k_n}
	\end{align*}
	with the left and right actions of $\cB$ on $\ocX$ defined by
	\begin{align*}
		b \cdot (x_1\otimes\cdots\otimes x_n) &= (L_b x_1)\otimes\cdots\otimes x_n\\
		(x_1\otimes\cdots\otimes x_n) \cdot b&= x_1\otimes\cdots\otimes (R_b x_n).
	\end{align*}
	We use $\st_{\iota \in \I} \cX_{\iota}$ to denote $\cX$.

	For each $\iota \in \I$, we define the left representation $\lambda_{\iota} : \cL_\ell(\cX_{\iota}) \to \cL(\cX)$ as follows:
	let
	\[
		W_{\iota} : \cX \to \cX_{\iota} \otimes_\cB \paren{\cB\oplus \bigoplus_{n\geq 1}\bigoplus_{\substack{k_1\neq k_2\neq \cdots \neq k_n\\k_1\neq k}} \ocX_{k_1}\otimes_\cB\cdots\otimes_\cB\ocX_{k_n}}
	\]
	be the $\cB$-$\cB$-bimodule isomorphism defined analogously to $W_\iota$ from Subsection~\ref{ssec:freeind}, and set
	\[
		\lambda_{\iota}(T) = W_{\iota}^{-1}(T \otimes I)W_{\iota}.
	\]
	Note that this is unambiguous precisely because $T \in \cL_\ell(\cX_\iota)$, so we have
	$$(T\otimes I)(x\cdot b)\otimes \xi = (Tx)\cdot b \otimes \xi = (T\otimes I)(x \otimes b\cdot \xi).$$
	We can compute $\lambda_\iota(T)$ explicitly: for $b \in \cB$ and $T \in \cL_\ell(\cX_\iota)$,
	\begin{align*}
		\lambda_{\iota}(T) (b) = E_{k}(T)b + (T-L_{E_{\iota}(T)})b,
	\end{align*}
	while
	\begin{align*}
		\lambda_{\iota}(T) (x_1\otimes\cdots \otimes x_n) =\left\{
			\begin{array}{ll}
				\paren{L_{p_{k}(Tx_1)} x_2\otimes\cdots\otimes x_n } + \paren{
				[(1-p_{k})Tx_1]\otimes\cdots \otimes x_n } & \text{if }x_1\in \ocX_{\iota}\\
				\paren{L_{E_{k}(T)} x_1\otimes\cdots\otimes x_n} + \paren{[(T-L_{E_{k}(T)})1_\cB]\otimes x_1\otimes\cdots\otimes x_n } & \text{if }x_1\not\in\ocX_{\iota}
		\end{array}\right..
	\end{align*}
	Here as usual we interpret a tensor product of length zero as the vector $1_{\cB}$.
	Observe that $\lambda_{\iota}$ is a homomorphism, $\lambda_{\iota}(L_b)=L_b$, and $\lambda_{\iota}(\cL_\ell(\cX_{\iota})) \subseteq \cL_\ell(\cX)$.

	Similarly, for each $\iota \in \I$, we define the map $\rho_{\iota} : \cL_r(\cX_{\iota}) \to \cL(\cX)$ as follows:
	let
	\[
		U_{\iota} : \cX \to \paren{\cB\oplus\bigoplus_{n\geq1}\bigoplus_{\substack{k_1\neq k_2\neq\cdots\neq k_n\\ k_n \neq k}}
		\ocX_{k_1}\otimes_\cB\cdots\otimes_\cB\ocX_{k_n}} \otimes_\cB \cX_{\iota}
	\]
	be the $\cB$-$\cB$-bimodule isomorphism analogous to $U_\iota$ in Subsection~\ref{ssec:freeind}, and define
	\[
		\rho_{\iota}(T) = U_{\iota}^{-1}(I \otimes T)U_{\iota};
	\]
	again this is well-defined precisely because $T$ commutes with the left action of $\cB$.
	As before, we find
	\begin{align*}
		\rho_{\iota}(T)(b) = b E_{\iota}(T) + (T-R_{E_{\iota}(T)}) b,
	\end{align*}
	and
	\begin{align*}
		\rho_{\iota}(T)(x_1\otimes\cdots \otimes x_n)=\left\{
			\begin{array}{cl}
				\paren{x_1\otimes\cdots\otimes R_{p_{k}(T x_n)}x_{n-1}}
				+ \paren{x_1\otimes\cdots \otimes [(1-p_{k})Tx_n]} & \text{if }x_n\in \ocX_{\iota}\\
				\paren{x_1\otimes\cdots\otimes R_{E_{\iota}(T)}x_n}
				+ \paren{x_1\otimes\cdots\otimes x_n\otimes [(T-R_{E_{\iota}(T)})1_\cB]} & \text{if }x_n\not\in\ocX_{\iota}
		\end{array}\right.,
	\end{align*}
	for all $T \in \cL(\cX_{\iota})$.
	Clearly $\rho_{\iota}$ is a homomorphism, $\rho_{\iota}(R_b)=R_b$, and $\rho_{\iota}(\cL_r(\cX_{\iota})) \subseteq \cL_r(\cX)$.




	In addition, note that if $T \in \cL_\ell(\cX_{\iota})$ then
	\begin{align*}
		E_{\cL(\cX)}(\lambda_{\iota}(T))=p(\lambda_{\iota}(T)1_\cB)= p( E_{\iota}(T)+ [T-L_{E_{\iota}(T)}]1_\cB ) = E_{\iota}(T)
	\end{align*}
	and similarly $E_{\cL(\cX)}(\rho_{\iota}(T)) = E_{\iota}(T)$ if $T \in \cL_r(\cX_\iota)$.
	Hence, the above shows that $\cL(\cX)$ contains each $\cL(\cX_{\iota})$ in a left-preserving, right-preserving, state-preserving way.
\end{construction}

With computation, we see that $\lambda_i(T)$ and $\rho_j(S)$ commute when $T \in \cL_{\ell}(\cX_i)$, $S \in \cL_r(\cX_j)$, and $i \neq j$.
Indeed, notice if $b \in \cB$ then
\begin{align*}
	& \lambda_i(T) \rho_j(S) b \\
	&= \lambda_i(T) \paren{b E_j(S) + (S-R_{E_j(S)}) b
	} \\
	&= E_{i}(T)bE_j(S) + (T-L_{E_i(T)})bE_j(S) + L_{E_i(T)}(S-R_{E_j(S)}) b
	+ \paren{[(T-L_{E_{i}(T)})1_\cB] \otimes [(S-R_{E_j(S)}) b]
	},
\end{align*}
whereas
\begin{align*}
	&
	\rho_j(S) \lambda_i(T)b \\
	&= \rho_j(S) \paren{ E_{i}(T)b + (T-L_{E_i(T)})b} \\
	&= E_{i}(T)b E_j(S) + (S-R_{E_j(S)})E_{i}(T)b +
	R_{E_j(S)} (T-L_{E_i(T)})b +
	\paren{[(T-L_{E_{i}(T)})b] \otimes [(S-R_{E_j(S)}) 1_\cB]
	}.
\end{align*}
Since $T \in \cL_{\ell}(\cX_i)$ and $S \in \cL_r(\cX_j)$, one sees that
\begin{align*}
	L_{E_i(T)}(S-R_{E_j(S)}) b = (S-R_{E_j(S)})L_{E_i(T)} b = (S-R_{E_j(S)})E_i(T) b,\\
	R_{E_j(S)} (T-L_{E_i(T)})b = (T-L_{E_i(T)}) R_{E_j(S)} b =
	(T-L_{E_i(T)})bE_j(S),
\end{align*}
and
\begin{align*}
	[(T-L_{E_{i}(T)})b] \otimes [(S-R_{E_j(S)}) 1_\cB] & = [(T-L_{E_{i}(T)})R_b 1_\cB] \otimes [(S-R_{E_j(S)}) 1_\cB] \\
	& = [R_b(T-L_{E_{i}(T)}) 1_\cB] \otimes [(S-R_{E_j(S)}) 1_\cB] \\
	& = [(T-L_{E_{i}(T)}) 1_\cB] \otimes [L_b(S-R_{E_j(S)}) 1_\cB] \\
	& = [(T-L_{E_{i}(T)}) 1_\cB] \otimes [(S-R_{E_j(S)}) L_b1_\cB] \\
	& = [(T-L_{E_{i}(T)})1_\cB] \otimes [(S-R_{E_j(S)}) b] .
\end{align*}
Thus $\lambda_i(T) \rho_j(S) b = \rho_j(S) \lambda_i(T) b$.
Similar computations show $\lambda_i(T)$ and $\rho_j(T)$ commute on $\ocX_i$, $\ocX_j$, and $\ocX_i \otimes \ocX_j$, and it is trivial to see that $\lambda_i(T)$ and $\rho_j(T)$ commute on all other components of $\ocX$.

Note that $\lambda_i(T)$ and $\rho_i(S)$ need not commute, though their commutator will be supported on $\cB \oplus \ocX_i$ and there will be equal to the commutator $[T, S]$.










%%%%%%%%%%%%%%%%%%%%%%%%%%%%%%%%%%%%%%%%%%%%%%%
%	Abstract Structures for Bi-Free Probability with Amalgamation
%%%%%%%%%%%%%%%%%%%%%%%%%%%%%%%%%%%%%%%%%%%%%%%
\subsection{Abstract structures for bi-free probability with amalgamation.}

The purpose of this section is to develop an abstract notion of the pair $(\cL(\cX), E_{\cL(\cX)})$.
Based on the previous section and Proposition \ref{prop:propertiesofEforLX}, we make the following definition.
\begin{definition}
	\label{defn:BBncps}
	A \emph{$\cB$-$\cB$-non-commutative probability space} is a triple $(\A, E_\A, \varepsilon)$ where $\A$ is a unital algebra, $\varepsilon : \cB \otimes \cB^{\mathrm{op}} \to \A$ is a unital homomorphism such that $\varepsilon|_{\cB \otimes 1_\cB}$ and $\varepsilon|_{1_\cB \otimes \cB^{\mathrm{op}}}$ are injective, and $E_\A : \A \to \cB$ is a unital linear map such that
	\[
		E_{\A}(\varepsilon(b_1 \otimes b_2)T) = b_1 E_{\A}(T) b_2
	\]
	for all $b_1, b_2 \in \cB$ and $T \in \A$, and
	\[
		E_{\A}(T\varepsilon(b \otimes 1_\cB)) = E_{\A}(T\varepsilon(1_\cB \otimes b))
	\]
	for all $b \in \cB$ and $T \in \A$.

	In addition, we define the unital subalgebras $\A_\ell$ and $\A_r$ of $\A$ by
	\begin{align*}
		\A_\ell &:= \set{ T \in \A
		\, \mid \, T\varepsilon(1_\cB \otimes b) = \varepsilon(1_\cB \otimes b) T \text{ for all }b \in \cB}\\
		\A_r &:= \set{ T \in \A
		\, \mid \, T\varepsilon(b \otimes 1_\cB) = \varepsilon(b \otimes 1_\cB) T \text{ for all }b \in \cB}.
	\end{align*}
	We call $\A_\ell$ and $\A_r$ the \emph{left} and \emph{right algebras} of $\A$, respectively.
\end{definition}

If $(\cX, \ocX, p)$ is a $\cB$-$\cB$-bimodule with a specified $\cB$-vector state, we see via Proposition \ref{prop:propertiesofEforLX} that $(\cL(\cX), E_{\cL(\cX)}, \varepsilon)$ is a $\cB$-$\cB$-non-commutative probability space where $E_{\cL(\cX)}$ is as in Definition \ref{defn:expectationofLXontoB} and $\varepsilon : \cB \otimes \cB^{\mathrm{op}} \to \cB$ is defined by $\varepsilon(b_1 \otimes b_2) = L_{b_1} R_{b_2}$.
As such, in an arbitrary $\cB$-$\cB$-non-commutative probability space $(\A, E_\A, \varepsilon)$, we will often use
$L_b$ instead of $\varepsilon(b \otimes 1)$ and $R_b$ instead of $\varepsilon(1 \otimes b)$, in which case $L_b \in \A_\ell$ and $R_b \in \A_r$ for all $b \in \cB$.
For $b \in \cB$, we will call $L_b$ a \emph{left $\cB$-operator} and $R_b$ a \emph{right $\cB$-operator}.

It may appear that Definition~\ref{defn:BBncps} is incompatible with the notion of a $\cB$-probability space in free probability: that is, a pair $(\A, \E)$ where $\A$ is a unital algebra containing $\cB$, and $\E : \A \to \cB$ is a linear map such that $\E(b_1Tb_2) = b_1\E(T)b_2$ for all $b_1, b_2 \in \cB$ and $T \in \A$.
However, $\A$ is a $\cB$-$\cB$-bimodule by left and right multiplication by $\cB$, and $\A$ can be made into a $\cB$-$\cB$-bimodule with specified $\cB$-vector state via $p = \E$ and $\ocX = \ker(\E)$.
Hence the above discussion implies $\cL(\A)$ is a $\cB$-$\cB$-non-commutative probability space with
\[
	E_{\cL(\A)}(T) = \E(T)
\]
for all $T \in \cL(\A)$.
In addition, we can view $\A$ as a unital subalgebra of both $\cL_\ell(\A)$ and $\cL_r(\A)$ by left and right multiplication on $\A$ respectively.


Viewing $\A \subseteq \cL_\ell(\A)$, it is clear we can recover the joint $\cB$-moments of elements of $\A$ from $E_{\cL(\A)}$.
Indeed, for $T \in \A \subseteq \cL_\ell(\A)$ we have
\[
	E_{\cL(\A)}(L_{b_1} T L_{b_2}) = E_{\cL(\A)}(L_{b_1} T R_{b_2}) = E_{\cL(\A)}(L_{b_1} R_{b_2} T) = b_1 E_{\cL(\A)}(T) b_2,
\]
which is consistent with the defining property of $\E$.
In particular, the same proof shows $(\A_\ell, E)$ is a $\cB$-non-commutative probability space and $(\A_r, E)$ is a $\cB^{\mathrm{op}}$-non-commutative probability space.

One should note that Definition~\ref{defn:BBncps} differs slightly from \cite{voiculescu2014free}*{Definition 8.3}.
However, given Proposition \ref{prop:propertiesofEforLX} and the following result which demonstrates that a $\cB$-$\cB$-non-commutative probability space embeds into $\cL(\cX)$ for a $\cB$-$\cB$-bimodule with a specified $\cB$-vector state $\cX$, Definition \ref{defn:BBncps} indeed specifies the correct abstract objects to study.

\begin{theorem}
	\label{thm:representingbbncps}
	Let $(\A, E_\A, \varepsilon)$ be a $\cB$-$\cB$-non-commutative probability space.
	Then there exists a $\cB$-$\cB$-bimodule with a specified $\cB$-vector state $(\cX, \ocX, p)$ and a unital homomorphism $\theta : \A \to \cL(\cX)$ such that 
	\[
		\theta(L_{b_1} R_{b_2}) = L_{b_1} R_{b_2}, \quad \theta(\A_\ell) \subseteq \cL_\ell(\cX), \quad
		\theta(\A_r) \subseteq \cL_r(\cX), \quad
		\text{and} \quad
		E_{\cL(\cX)}(\theta(T)) = E_\A(T)
	\]
	for all $b_1, b_2 \in \cB$ and $T \in \A$.
\end{theorem}

\begin{proof}
	Consider the vector space $\cX = \cB \oplus \cY$, where
	\[
		\cY = \ker(E_\A) / \mathrm{span}\set{TL_b - TR_b \, \mid \, T \in \A, b \in \cB}.
	\]
	Note $\cY$ is a well-defined quotient vector space since $E_{\A}(TL_b - TR_b) = 0$ by Definition \ref{defn:BBncps}.
	We will postpone describing the $\cB$-$\cB$-module structure on $\cX$ until later in the proof.

	Let $q : \ker(E_\A) \to \cY$ denote the canonical quotient map.
	Then, for $T, A \in \A$ with $E_\A(A) = 0$ and $b \in \cB$, we define $\theta(T) \in \cL(\cX)$ by
	\[
		\theta(T)(b) = E_\A(TL_b) \oplus q(TL_b - L_{E_\A(TL_b)})
	\]
	and
	\[
		\theta(T)(q(A)) = E_\A(TA) \oplus q(TA - L_{E_\A(TA)}).
	\]
	Note that $\mathrm{span}\set{TL_b - TR_b \, \mid \, T \in \A, b \in \cB}$ is a left-ideal in $\A$, so $q(A) = 0$ implies $q(TA) = 0$ and thus $\theta$ is well-defined.



	We wish to show that $\theta$ is a homomorphism; it is immediate that $\theta$ is linear.
	To see that $\theta$ is multiplicative, fix $T, S \in \A$.
	If $b \in \cB$, then
	\[
		\theta(T)(b) = E_\A(TL_b) \oplus q(TL_b - L_{E(_{\A}TL_b)}).
	\]
	Thus
	\begin{align*}
		\theta(S)(\theta(T)(b))
		= & \,\,E_{\A}(SL_{E_{\A}(TL_b)}) \oplus q\paren{SL_{E_{\A}(TL_b)} - L_{E_{\A}(SL_{E_{\A}(TL_b)})}
		} \\
		& + E_{\A}(S (TL_b - L_{E_{\A}(TL_b)})) \oplus q\paren{S(TL_b - L_{E_{\A}(TL_b)}) - L_{E_{\A}(S(TL_b - L_{E_{\A}(TL_b)}))}
		}
		\\
		= & \,\,E_{\A}(STL_b) \oplus q\paren{STL_b - L_{E_{\A}(STL_b)}}
		\\
		= & \,\,\theta(ST)(b).
	\end{align*}
	Similarly, if $q(A)\in \cY$ then
	\[
		\theta(T)(q(A)) = E_{\A}(TA) \oplus q(TA - L_{E_{\A}(TA)}).
	\]
	Thus
	\begin{align*}
		\theta(S)(\theta(T)(q(A))) = &\,\, E_{\A}\paren{SL_{E_{\A}(TA)}} \oplus q\paren{SL_{E_{\A}(TA)} - L_{E_{\A}(SL_{E_{\A}(TA)})} } \\
		& + E_{\A}(S(TA - L_{E_{\A}(TA)})) \oplus q(S(TA - L_{E_{\A}(TA)}) - L_{E_{\A}(S(TA - L_{E_{\A}(TA)}))} )\\
		= & \,\,E_{\A}(STA) \oplus q\paren{ STA - L_{E_{\A}(STA)}} \\
		= & \,\,\theta(ST)(q(A)).
	\end{align*}
	Hence $\theta$ is a homomorphism.






	To make $\cX$ a $\cB$-$\cB$-bimodule, we define
	\[
		b \cdot \xi = \theta(L_b)(\xi) \qquad \text{and}\qquad \xi \cdot b = \theta(R_b)(\xi)
	\]
	for all $\xi \in \cX$ and $b \in \cB$; thus we automatically have $\theta(L_{b_1}R_{b_2}) = L_{b_1}R_{b_2}$.


	To demonstrate that $\cX$ is indeed a $\cB$-$\cB$-bimodule with a specified vector state, we must show that $\cY$ is invariant under this $\cB$-$\cB$-bimodule structure, and that the $\cB$-$\cB$-bimodule structure when restricted to $\cB \subseteq \cX$ is the canonical one.
	If $b, b' \in \cB$ and $q(A) \in \cY$, then
	\[
		\theta(L_b)(b') = E_{\A}(L_b L_{b'}) \oplus q(L_b L_{b'} - L_{E_{\A}(L_b L_{b'})}) = bb' \oplus q(L_{bb'} - L_{bb'}) = bb' \oplus 0
	\]
	and
	\begin{align*}
		\theta(L_b)(q(A)) &= E_{\A}(L_b A) \oplus q(L_b A - L_{E_{\A}(L_b A)})\\
		&= bE_{\A}(A) \oplus q(L_b A - L_{E_{\A}(L_b A)}) = 0 \oplus q(L_b A - L_{E_{\A}(L_b A)}).
	\end{align*}
	Similarly,
	\[
		\theta(R_b)(b') = E_{\A}(R_b L_{b'}) \oplus q(R_b L_{b'} - L_{E_{\A}(R_b L_{b'})}) = b'b \oplus q(L_{b'}R_b - L_{b'}L_{b}) = b'b \oplus 0
	\]
	and
	\begin{align*}
		\theta(R_b)(q(A)) &= E_{\A}(R_b A) \oplus q(R_b A - L_{E(R_b A)})\\
		&= E_{\A}(A)b \oplus q(R_b A - L_{E_{\A}(R_b A)}) = 0 \oplus q(R_b A - L_{E_{\A}(R_b A)}).
	\end{align*}
	Thus $\cX$ is a $\cB$-$\cB$-bimodule with a specified $\cB$-vector state.

	Since $\theta$ is a homomorphism, it is clear that $\theta(\A_\ell) \subseteq \cL_\ell(\cX)$ and $\theta(\A_r) \subseteq \cL_r(\cX)$ due to the definition of the $\cB$-$\cB$-bimodule structure on $\cX$.
	Finally, if $T \in \A$ then
	\[
		E_{\cL(\cX)}(\theta(T)) = p(\theta(T) (1_\cB \oplus 0)) = p(E_\A(T) \oplus q(T - L_{E_\A(T)})) = E_\A(T).\qedhere
	\]
\end{proof}







%%%%%%%%%%%%%%%%%%%%%%%%%%%%%%%%%%%%%%%%%%%%%%%
%	Abstract Structures for Bi-Free Probability with Amalgamation
%%%%%%%%%%%%%%%%%%%%%%%%%%%%%%%%%%%%%%%%%%%%%%%

\subsection{Bi-free families of pairs of $\cB$-faces.}


With the notion of a $\cB$-$\cB$-non-commutative probability space from Definition \ref{defn:BBncps}, we are now able to define the main concept of this chapter, following \cite{voiculescu2014free}*{Definition 8.5}.
\begin{definition}
	\label{defn:pairofBfaces}
	Let $(\A, E_\A, \varepsilon)$ be a $\cB$-$\cB$-non-commutative probability space.
	A \emph{pair of $\cB$-faces of $\A$} is a pair $(C, D)$ of unital subalgebras of $\A$ such that
	\[
		\varepsilon(\cB \otimes 1_\cB) \subseteq C \subseteq \A_\ell \qquad \mathrm{and}\qquad \varepsilon(1_\cB \otimes \cB^{op}) \subseteq D \subseteq \A_r.
	\]

	A family $\set{(C_\iota, D_\iota)}_{\iota \in \I}$ of pairs of $\cB$-faces of $\A$ is said to be \emph{bi-free with amalgamation over $\cB$} (or simply \emph{bi-free over $\cB$}) if there exist $\cB$-$\cB$-bimodules with specified $\cB$-vector states $\set{(\cX_\iota, \ocX_\iota, p_\iota)}_{\iota \in \I}$ and unital homomorphisms $l_\iota : C_\iota \to \cL_{\ell}(\cX_\iota)$, $r_\iota : D_\iota \to \cL_{r}(\cX_\iota)$ such that the joint distribution of $\set{(C_\iota, D_\iota)}_{\iota \in \I}$ with respect to $E_\A$ is equal to the joint distribution of the images $\set{((\lambda_\iota \circ l_\iota)(C_\iota), (\rho_\iota \circ r_\iota)(D_\iota))}_{\iota \in \I}$ inside $\cL(\st_{\iota \in \I} \cX_\iota)$ with respect to $E_{\cL(\st_{\iota \in \I} \cX_\iota)}$.
\end{definition}

It will be an immediate consequence of Theorem \ref{thm:bifreeequivalenttouniversalpolys} that the selection of representations in Definition \ref{defn:pairofBfaces} does not matter (see \cite{voiculescu2014free}*{Proposition 2.9}).
Note that if $\set{(C_\iota, D_\iota)}_{\iota \in \I}$ is bi-free over $\cB$, then $\set{C_\iota}_{\iota \in \I}$ is free with amalgamation over $\cB$ (as is $\set{D_\iota}_{\iota \in \I}$) and $C_i$ and $D_j$ commute in distributions whenever $i \neq j$.

To conclude this section, we give the following example.
\begin{example}
	Let $(M_1, \tau_1)$ and $(M_2, \tau_2)$ be $\mathrm{II}_1$ factors, and $(N, \tau_N)$ a common von Neumann sub-algebra.
	Then if $M = M_1 \st_N M_2$ is their amalgamated free product as von Neumann algebras, $L^2(M)$ has the structure of an $N$-$N$-bimodule via left and right multiplication.
	If we take $p$ to be the orthogonal projection of $L^2(M)$ onto $L^2(N)$, this makes $(L^2(M), L^2(N)^\perp, p)$ into a an $N$-$N$-bimodule with specified $B$-vector state.
	Then taking $\lambda_i$ and $\rho_i$ to be the left and right representations of $M_i$ on $M$, we find that $(\lambda_1(M_1), \rho_1(M_1))$ and $(\lambda_2(M_2), \rho_2(M_2))$ are bi-free with amalgamation over $N$ in $\cL\paren{L^2(M)}$.
\end{example}





%%%%%%%%%%%%%%%%%%%%%%%%%%%%%%%%%%%%%%%%%%%%%%%%%%%%%%%%%%%%%%%%%%%%
%	    Operator-Valued Bi-Multiplicative Functions	   %
%%%%%%%%%%%%%%%%%%%%%%%%%%%%%%%%%%%%%%%%%%%%%%%%%%%%%%%%%%%%%%%%%%%%
\section{Operator-valued bi-multiplicative functions.}
\label{sec:OperatorValuedBiMultiplicativeFunctions}





In this section, we will develop a notion of $\cB$-valued bi-multiplicative functions in order to study $\cB$-$\cB$-non-commutative probability spaces (compare \cite{nica2002operator}*{Section 2} or \cite{speicher1998combinatorial}*{Section 2}).
Our goal is once again to use this theory to understand operator-valued bi-free cumulants.

%%%%%%%%%%%%%%%%%%%%%%%%%%%%%%%%%%%%%%%%%%%%%%%%%%%%%%%%%%%%%%%%%%%%%%%
\subsection{Definition of bi-multiplicative functions.}
We begin by examining the operator-valued generalization of the multiplicative functions used in Chapter~\ref{ch:bfi}.

\begin{definition}
	\label{defnbimultiplicative}
	Let $(\A, E, \varepsilon)$ be a $\cB$-$\cB$-non-commutative probability space and let 
	\[
		\Phi : \bigcup_{n\geq 1} \bigcup_{\chi : [n] \to \slr} \BNC(\chi) \times \A_{\chi(1)} \times \cdots \times \A_{\chi(n)} \to B
	\]
	be a function that is linear in each $\A_{\chi(k)}$.
	We say that $\Phi$ is \emph{bi-multiplicative} if for every $\chi : [n] \to \slr$, $T_k \in \A_{\chi(k)}$, $b \in \cB$, and $\pi \in \BNC(\chi)$, the following four conditions hold:
	\begin{enumerate}[label=(\roman*)]
		\item\label{def:bimult:i} Let
			\[
				q = \max\set{ k \in [n] \, \mid \, \chi(k) \neq \chi(n)}.
			\]
			If $\chi(n) = \ell$ then
			\[
				\Phi_{1_\chi}(T_1, \ldots, T_{n-1}, T_nL_b) = \left\{
					\begin{array}{ll}
						\Phi_{1_\chi}(T_1, \ldots, T_{q-1}, T_q R_b, T_{q+1}, \ldots, T_n) & \text{if } q \neq -\infty
						\\
						\Phi_{1_\chi}(T_1, \ldots, T_{n-1}, T_n)b & \text{if } q = -\infty
				\end{array} \right. .
			\]
			If $\chi(n) = r$ then 
			\[
				\Phi_{1_\chi}(T_1, \ldots, T_{n-1}, T_nR_b) = \left\{
					\begin{array}{ll}
						\Phi_{1_\chi}(T_1, \ldots, T_{q-1}, T_q L_b, T_{q+1}, \ldots, T_n) & \text{if } q \neq -\infty
						\\
						b\Phi_{1_\chi}(T_1, \ldots, T_{n-1}, T_n) & \text{if } q = -\infty
				\end{array} \right. .
			\]

		\item\label{def:bimult:ii} Let $p \in [n]$, and let
			\[
				r = \max\set{ k \in [n] \, \mid \, \chi(k) = \chi(p), k < p}.
			\]
			If $\chi(p) = \ell$ then
			\[
				\Phi_{1_\chi}(T_1, \ldots, T_{p-1}, L_bT_p, T_{p+1}, \ldots, T_n) = \left\{
					\begin{array}{ll}
						\Phi_{1_\chi}(T_1, \ldots, T_{q-1}, T_qL_b, T_{q+1}, \ldots, T_n) & \text{if } q \neq -\infty
						\\
						b \Phi_{1_\chi}(T_1, T_2, \ldots, T_n) & \text{if } q = -\infty
				\end{array} \right. .
			\]
			If $\chi(p) = r$ then
			\[
				\Phi_{1_\chi}(T_1, \ldots, T_{p-1}, R_bT_p, T_{p+1}, \ldots, T_n) = \left\{
					\begin{array}{ll}
						\Phi_{1_\chi}(T_1, \ldots, T_{q-1}, T_qR_b, T_{q+1}, \ldots, T_n) & \text{if } q \neq -\infty
						\\
						\Phi_{1_\chi}(T_1, T_2, \ldots, T_n) b & \text{if } q = -\infty
				\end{array} \right. .
			\]

		\item\label{def:bimult:iii} 
			Suppose that $V_1, \ldots, V_m$ are $\chi$-intervals which partition $[n]$ so that $\pi \leq \set{V_1, \ldots, V_m}$.
			Further, suppose $V_1 \prec_\chi \ldots \prec_\chi V_m$ (i.e., the relation $\prec_\chi$ holds for every choice of elements from these sets).
			Then
			\[
				\Phi_\pi(T_1, \ldots, T_n) = \Phi_{\pi|_{V_1}}\paren{(T_1, \ldots, T_n)|_{V_1}} \cdots \Phi_{\pi|_{V_m}}\paren{(T_1, \ldots, T_n)|_{V_m}}.
			\]

		\item\label{def:bimult:iv} Suppose that $V$ and $W$ partition $[n]$, $\pi \leq \set{V, W}$, and $V$ is a $\chi$-interval which is inner in $\set{V,W}$ in the sense of Subsection~\ref{ss:condbifree}.
			Let
			\[
				\theta = \max_{\prec_\chi}\paren{\set{k \in W
				\, \mid \, k \prec_\chi \min_{\prec_\chi}(V)}} \qquad\text{ and } \qquad \gamma = \min_{\prec_\chi}\paren{\set{k \in W
				\, \mid \, \max_{\prec_\chi}(V) \prec_\chi k}}.
			\]
			Then
			\begin{align*}
				\Phi_\pi(T_1, \ldots, T_n) &= \left\{
					\begin{array}{ll}
						\Phi_{\pi|_{W}}\paren{\paren{T_1, \ldots, T_{\theta-1}, T_\theta L_{\Phi_{\pi|_{V}}\paren{(T_1,\ldots, T_n)|_{V}}}, T_{\theta+1}, \ldots, T_n}|_{W}}
						& \text{if } \chi(\theta) = \ell \\
						\Phi_{\pi|_{W}}\paren{\paren{T_1, \ldots, T_{\theta-1}, R_{\Phi_{\pi|_{V}}\paren{(T_1,\ldots, T_n)|_{V}}} T_\theta, T_{\theta+1}, \ldots, T_n}|_{W}}
						& \text{if } \chi(\theta) = r 
				\end{array} \right. \\
				&= \left\{
					\begin{array}{ll}
						\Phi_{\pi|_{W}}\paren{\paren{T_1, \ldots, T_{\gamma-1},
						L_{\Phi_{\pi|_{V}}\paren{(T_1,\ldots, T_n)|_{V}}} T_\gamma, T_{\gamma+1}, \ldots, T_n}|_{W}}
						& \text{if } \chi(\gamma) = \ell
						\\
						\Phi_{\pi|_{W}}\paren{\paren{T_1, \ldots, T_{\gamma-1}, T_\gamma R_{\Phi_{\pi|_{V}}\paren{(T_1,\ldots, T_n)|_{V}}}, T_{\gamma+1}, \ldots, T_n}|_{W}} & \text{if } \chi(\gamma) = r
				\end{array} \right. .
			\end{align*}
	\end{enumerate}
\end{definition}







\begin{example}
	Suppose that $\Phi$ is a bi-multiplicative function, and that $\chi : [5] \to \slr$ corresponds to the sequence $(\ell, \ell, r, \ell, r)$.
	Using Properties~\ref{def:bimult:i} and \ref{def:bimult:ii}, we obtain that
	$$\Phi_\pi(T_1, L_{b_1}T_2, R_{b_2}T_3, T_4, T_5 R_{b_3}) = \Phi_\pi(T_1L_{b_1}, T_2, T_3, T_4 L_{b_3}, T_5)b_2.$$
	This can be thought of as allowing us to move elements of $\cB$ between nodes on the diagram of the corresponding bi-non-crossing partition:
	\[
		\begin{tikzpicture}
			\def\sdz{{-1,-1,1,-1,1}}
			\def\labelz{{"$T_1$", "$T_2$", "$T_3$", "$T_4$", "$T_5$"}}
			\foreach \shft in {0,1} {
				\begin{scope}[xshift=6*\shft cm]
					\bnc[n=5,sidez=\sdz,labelz=\labelz]
					\foreach \y in {1, ..., 5} {
						\draw [thick] (ball\y) -| (ball1 -| 0,0);
						\coordinate (b\shft\y) at (ball\y);
					}
					\coordinate (cl\shft) at (cl);
					\coordinate (cr\shft) at (cr);
					\coordinate (tr\shft) at (tr);
					\coordinate (br\shft) at (br);

					\node [left=0.4cm] at ($ (ball2) ! 0.5 ! (ball1) $) {$L_{b_1}$};
				\end{scope}
			}
			\node at ($ (cl1) ! .5 ! (cr0) $) {$=$};

			\node [above, draw, shade, circle, ball color=gray, inner sep=0.07cm] at (b02) {};
			\node [below, draw, shade, circle, ball color=gray, inner sep=0.07cm] at (b11) {};

			\node [above, draw, shade, circle, ball color=gray, inner sep=0.07cm] at (b03) {};
			\node [draw, shade, circle, ball color=gray, inner sep=0.07cm] at (tr1) {};
			\path (b03) |- node[right=0.4cm] {$R_{b_2}$} ($ (b03) ! 0.5 ! (b02) $);
			\node [right=0.4cm] at (tr1) {$R_{b_2}$};

			\node [below, draw, shade, circle, ball color=gray, inner sep=0.07cm] at (b05) {};
			\node [below, draw, shade, circle, ball color=gray, inner sep=0.07cm] at (b14) {};
			\node [right=0.4cm] at (br0) {$R_{b_3}$};
			\path (b14) |- node [left=0.4cm] {$L_{b_3}$} ($ (b14) ! 0.5 ! (b15) $);

		\end{tikzpicture}
	\]
\end{example}

\begin{example}
	Again, take $\Phi$ to be a bi-multiplicative function.
	Suppose $\chi : [8] \to \slr$ corresponds to the sequence $(\ell, r, \ell, \ell, \ell, r, r, \ell)$.
	Let $\pi = \set{\set{1,3,4}, \set{5,7,8}, \set{2,6}}$ and $\sigma = \set{\set{1,2,6}, \set{3,7}, \set{4,5,8}}$.
	Then
	$$\Phi_\pi(T_1, \ldots, T_8) = \Phi_{\pi_1}(T_1, T_3, T_4) \Phi_{\pi_2}(T_5, T_7, T_8) \Phi_{\pi_3}(T_2, T_6),$$
	and
	$$\Phi_\sigma(T_1,\ldots, T_8) = \Phi_{\sigma_1}\paren{T_1L_{\Phi_{\sigma_2}\paren{T_3, T_7R_{\Phi_{\sigma_3}(T_4, T_5, T_6)}}}, T_2, T_6}.$$
	\[
		\begin{tikzpicture}
			\def\sdz{{-1,1,-1,-1,-1,1,1,-1}}
			\def\labelz{{"$T_1$", "$T_2$", "$T_3$", "$T_4$", "$T_5$", "$T_6$", "$T_7$", "$T_8$"}}
			\begin{scope}[xshift=-3cm]
				\bnc[n=8,sidez=\sdz,labelz=\labelz]
				\foreach \y in {1,3,4} {
					\draw [thick] (ball\y) -| (ball1 -| -0.3,0);
				}
				\foreach \y in {5,7,8} {
					\draw [thick] (ball\y) -| (ball5 -| -0.3,0);
				}
				\foreach \y in {2,6} {
					\draw [thick] (ball\y) -| (ball2 -| 0.3,0);
				}
				\node[below] at (bc) {$\pi$};
			\end{scope}
			\begin{scope}[xshift=3cm]
				\bnc[n=8,sidez=\sdz,labelz=\labelz]
				\foreach \y in {1,2,6} {\draw [thick] (ball\y) -| (ball1 -| 0.5,0);}
				\foreach \y in {3,7} {\draw [thick] (ball\y) -| (ball3 -| 0,0);}
				\foreach \y in {4,5,8} {\draw [thick] (ball\y) -| (ball4 -| -0.5,0);}
				\node[below] at (bc) {$\sigma$};
			\end{scope}
		\end{tikzpicture}
	\]
	The underlying idea is this: any interval of blocks in $\pi$ may be evaluated with $\Phi$ and replaced by the corresponding element of $\cB$ at the location in the diagram where the block was removed.
	Only blocks which do not bound other blocks may be reduced in this manner; inner blocks must be reduced first.
\end{example}

Although Definition \ref{defnbimultiplicative} is cumbersome (due to the necessity of specifying cases based on whether certain terms are left or right operators), its properties can be viewed as direct analogues of those of a multiplicative map as described in \cite{nica2002operator}*{Section 2.2}.
Indeed, for $\pi \in \BNC(\chi)$ and a bi-multiplicative map $\Phi$, each expression of $\Phi_\pi(T_1, \ldots, T_n)$ in Definition \ref{defnbimultiplicative} comes from viewing $s_\chi^{-1}
\circ \pi \in NC(n)$, rearranging the $n$-tuple $(T_1, \ldots, T_n)$ to $(T_{s_\chi(1)}, \ldots, T_{s_\chi(n)})$, replacing any occurrences of $L_bT_j$, $T_j L_b$, $R_b T_j$, and $T_j R_b$ with $bT_j$, $T_j b$, $T_j b$, and $bT_j$ respectively, applying one of the properties of a multiplicative map from \cite{nica2002operator}*{Section 2.2}, and reversing the above identifications.
In particular, these properties reduce to those of a multiplicative map when $\chi^{-1}(\set{\ell}) = [n]$.
We use the more complex Definition \ref{defnbimultiplicative} as it will be easier to verify for functions later on.

Since a bi-multiplicative function satisfies all of these properties, it is easy to see that if $\Phi$ is bi-multiplicative, then $\Phi_\pi(T_1, \ldots, T_n)$ is determined by the values 
\[
	\set{\Phi_{1_{\chi'}}(S_1, \ldots, S_m) \, \mid \, m \in \N, \chi' : \set{1,\ldots, m} \to \slr, S_k \in \A_{\chi(k)}}.
\]
There may be multiple ways to reduce $\Phi$ to an expression involving elements from the above set, but Definition \ref{defnbimultiplicative} implies that all such reductions are equal.

Note that Definition \ref{defnbimultiplicative} automatically implies additional properties for bi-multiplicative functions.
Indeed one can either verify the following proposition via Definition \ref{defnbimultiplicative} and casework, or can appeal to the fact that the properties of bi-multiplicative functions can be described via the properties of multiplicative functions as above, and use the fact that multiplicative functions have additional properties (see, e.g., \cite{speicher1998combinatorial}*{Remark 2.1.3}).

\begin{proposition}
	\label{propenhancedproperties}
	Let $(\A, E, \varepsilon)$ be a $\cB$-$\cB$-non-commutative probability space and let
	\[
		\Phi : \bigcup_{n\geq 1} \bigcup_{\chi : [n] \to \slr} \BNC(\chi) \times \A_{\chi(1)} \times \cdots \times \A_{\chi(n)} \to B
	\]
	be a bi-multiplicative function.
	Given any $\chi : [n] \to \slr$, $\pi \in \BNC(\chi)$, and $T_k \in \A_{\chi(k)}$ Properties (i) and (ii) of Definition \ref{defnbimultiplicative} hold when $1_\chi$ is replaced with $\pi$.
\end{proposition}















%%%%%%%%%%%%%%%%%%%%%%%%%%%%%%%%%%%%%%%%%%%%%%%%%%%%%%%%%%%%%%%%%%%%
%	    Bi-Free Operator-Valued Moment Function is Bi-Multiplicative   	   %
%%%%%%%%%%%%%%%%%%%%%%%%%%%%%%%%%%%%%%%%%%%%%%%%%%%%%%%%%%%%%%%%%%%%

\section{Bi-free operator-valued moment function is bi-multiplicative.}
\label{sec:verifyingrecursivedefinitionfromuniversalpolynomialshasdesiredproperties}

In this section, we will define the bi-free operator-valued moment function based on recursively defined functions $E_\pi(T_1,\ldots, T_n)$ that appear via actions on free product spaces.
However, it is not immediate that it is bi-multiplicative.
The proof of this result requires substantial case work, to which this section is dedicated.


%%%%%%%%%%%%%%%%%%%%%%%%%%%%%%%%%%%%%%%%%%%%%%%%%%%%%%%%%%%%%%%%%%%%%%%
\subsection{Definition of the bi-free operator-valued moment function.}

We will begin with the recursive definition of expressions that appear in the operator-valued moment polynomials.
These will arise in the proof of Theorem~\ref{thm:bifreeequivalenttouniversalpolys}, where we will give a characterisation of bi-freeness with amalgamation over $\cB$ akin to that of bi-freeness in Corollary~\ref{cor:bifreemob}.

\begin{definition}
	\label{defn:recursivedefinitionofEpi}
	Let $(\A, E, \varepsilon)$ be a $\cB$-$\cB$-non-commutative probability space.
	For $\chi : [n] \to \slr$, $\pi \in \BNC(\chi)$, and $T_1, \ldots, T_n \in \A$, we define $E_\pi(T_1,\ldots, T_n) \in \cB$ via the following recursive process.
	Let $V$ be the block of $\pi$ that terminates closest to the bottom, so $\min(V)$ is largest among all blocks of $\pi$. Then:
	\begin{itemize}
		\item If $\pi$ contains exactly one block (that is, $\pi = 1_\chi)$, we set $E_{1_\chi}(T_1, \ldots, T_n) = E(T_1 \cdots T_n)$.

		\item If $V = \set{k+1, \ldots, n}$ for some $k < n$, then $\min(V)$ is not adjacent to any spines of $\pi$ and we define
			\[
				E_\pi(T_1, \ldots, T_n) := \left\{
					\begin{array}{ll}
						E_{\pi|_{V^c}}(T_1, \ldots, T_k L_{E_{\pi|_V}(T_{k+1},\ldots, T_n)}) & \text{if } \chi(\min(V)) = \ell
						\\
						E_{\pi|_{V^c}}(T_1, \ldots, T_k R_{E_{\pi|_V}(T_{k+1},\ldots, T_n)}) & \text{if } \chi(\min(V)) = r
				\end{array} \right..
			\]
			In the long run, it will not matter if we choose $L$ or $R$ by the first part of this recursive definition and Definition \ref{defn:BBncps}.

		\item Otherwise, $\min(V)$ is adjacent to a spine. Let $W$ denote the block of $\pi$ corresponding to the spine adjacent to $\min(V)$, and
			let $k$ be the first element of $W$ below where $V$ terminates -- that is, $k$ is the smallest element of $W$ that is larger than $\min(V)$.
			We define
			\[
				E_\pi(T_1, \ldots, T_n) := \left\{
					\begin{array}{l}
						E_{\pi|_{V^c}}((T_1, \ldots, T_{k-1}, L_{E_{\pi|_V}((T_{1},\ldots, T_n)|_V)} T_k, T_{k+1}, \ldots, T_n)|_{V^c}) \qquad
						\\\hfill \text{if } \chi(\min(V)) = \ell \\
						E_{\pi|_{V^c}}((T_1, \ldots, T_{k-1}, R_{E_{\pi|_V}((T_{1},\ldots, T_n)|_V)} T_k, T_{k+1}, \ldots, T_n)|_{V^c}) \qquad
						\\\hfill \text{if } \chi(\min(V)) = r
				\end{array} \right..
			\]
	\end{itemize}
\end{definition}
Notice that if $\cB = \C$ and $E = \varphi$ is a state, then $E_\pi$ in the above sense is precisely $\varphi_\pi$ in the notation from Chapter~\ref{ch:bfi}.

\begin{example}
	Let $\pi$ be the following bi-non-crossing partition.
	\[
		\begin{tikzpicture}
			\def\sdz{{-1,1,-1,-1,1,1,-1,1,1}}
			\bnc[n=9,sidez=\sdz]
			\foreach \y in {1,2} {\draw [thick] (ball\y) -| (ball1 -| 0,0);}
			\foreach \y in {3,5,9} {\draw [thick] (ball\y) -| (ball3 -| 0,0);}
			\foreach \y in {4,7} {\draw [thick] (ball\y) -| (ball4 -| -0.5,0);}
			\foreach \y in {6,8} {\draw [thick] (ball\y) -| (ball6 -| 0.5,0);}
		\end{tikzpicture}
	\]
	Then
	\[
		E_\pi(T_1, \ldots, T_9) = E\paren{T_1T_2 L_{E\paren{T_3 L_{E(T_4T_7)}
			T_5 R_{E(T_6T_8)} T_9 }}
		}
	\]
	via the following sequence of diagrams (where $X = L_{E\paren{T_3 L_{E(T_4T_7)} T_5 R_{E(T_6T_8)} T_9 }}$):
	\[
		\begin{tikzpicture}
			\def\sdz{{0,-1,1,-1,-1,1,1,-1,1,1}}
			\def\labelz{{"", "$T_1$", "$T_2$", "$T_3$", "$T_4$", "$T_5$", "$T_6$", "$T_7$", "$T_8$", "$T_9$"}}

			\begin{scope}[xshift=0cm]
				\def\ord{{1,2,3,4,5,6,7,8,9}}
				\bnc[n=9,sidez=\sdz,order=\ord,labelz=\labelz]
				\foreach \y in {1,2} {\draw [thick] (ball\y) -| (ball1 -| 0,0);}
				\foreach \y in {3,5,9} {\draw [thick] (ball\y) -| (ball3 -| 0,0);}
				\foreach \y in {4,7} {\draw [thick] (ball\y) -| (ball4 -| -0.5,0);}
				\foreach \y in {6,8} {\draw [thick] (ball\y) -| (ball6 -| 0.5,0);}

				\draw [thick,->] ($ (cr) + (0.75,0) $) -- ++(0.5,0);
			\end{scope}

			\begin{scope}[xshift=4cm]
				\def\ord{{1,2,3,4,5,0,7,0,9}}
				\bnc[n=9,sidez=\sdz,order=\ord,labelz=\labelz]
				\foreach \y in {1,2} {\draw [thick] (ball\y) -| (ball1 -| 0,0);}
				\foreach \y in {3,5,9} {\draw [thick] (ball\y) -| (ball3 -| 0,0);}
				\foreach \y in {4,7} {\draw [thick] (ball\y) -| (ball4 -| -0.5,0);}

				\node (a1) [draw, shade, circle, ball color=gray, inner sep=0.07cm] at ($ (ball9) + (0,0.25) $) {};
				\node [right] at (a1) {\scriptsize $R_{E(T_6T_8)}$};

				\draw [thick,->] ($ (cr) + (0.75,0) $) -- ++(0.5,0);
			\end{scope}

			\begin{scope}[xshift=8cm]
				\def\ord{{1,2,3,0,5,0,0,0,9}}
				\bnc[n=9,sidez=\sdz,order=\ord,labelz=\labelz]
				\foreach \y in {1,2} {\draw [thick] (ball\y) -| (ball1 -| 0,0);}
				\foreach \y in {3,5,9} {\draw [thick] (ball\y) -| (ball3 -| 0,0);}

				\node (a1) [draw, shade, circle, ball color=gray, inner sep=0.07cm] at ($ (ball9) + (0,0.25) $) {};
				\node [right] at (a1) {\scriptsize $R_{E(T_6T_8)}$};
				\node (a2) [draw, shade, circle, ball color=gray, inner sep=0.07cm] at ($ (ball5) + (0,0.25) $) {};
				\node [right] at (a2) {\scriptsize $L_{E(T_4T_7)}$};

				\draw [thick,->] ($ (cr) + (0.75,0) $) -- ++(0.5,0);
			\end{scope}

			\begin{scope}[xshift=12cm]
				\def\ord{{1,2,0,0,0,0,0,0,0}}
				\bnc[n=9,sidez=\sdz,order=\ord,labelz=\labelz]
				\foreach \y in {1,2} {\draw [thick] (ball\y) -| (ball1 -| 0,0);}

				\node (a0) [draw, shade, circle, ball color=gray, inner sep=0.07cm] at ($ (ball2) + (0,-0.25) $) {};
				\node [right] at (a0) {\scriptsize $X$};
			\end{scope}

		\end{tikzpicture}
	\]
\end{example}

Note that the definition of $E_\pi(T_1,\ldots, T_n)$ is invariant under $\cB$-$\cB$-non-commutative probability space embeddings, such as those listed in Theorem \ref{thm:representingbbncps}.
Observe that in the context of Definition \ref{defn:recursivedefinitionofEpi}, we ignore the notions of left and right operators.
However, we are ultimately interested in the following.

\begin{definition}
	Let $(\A, E, \varepsilon)$ be a $\cB$-$\cB$-non-commutative probability space.
	The \emph{bi-free operator-valued moment function}
	\[
		\cE : \bigcup_{n\geq 1} \bigcup_{\chi : [n] \to \slr} \BNC(\chi) \times \A_{\chi(1)} \times \cdots \times \A_{\chi(n)} \to B
	\]
	is defined by
	\[
		\cE_\pi(T_1, \ldots, T_n) = E_\pi(T_1, \ldots, T_n)
	\]
	for each $\chi : [n] \to \slr$, $\pi \in \BNC(\chi)$, and $T_k \in \A_{\chi(k)}$.
\end{definition}

Our next goal is the prove the following which is not apparent from Definition \ref{defn:recursivedefinitionofEpi}.

\begin{theorem}
	\label{thm:samantha}
	The operator-valued bi-free moment function $\cE$ on $\A$ is bi-multiplicative.
\end{theorem}

We divide the proof of the above theorem into several lemmata, verifying various of properties from Definition \ref{defnbimultiplicative}.
Properties (i) and (ii) are immediate but, unfortunately, the remaining properties are not as easily verified.

%%%%%%%%%%%%%%%%%%%%%%%%%%%%%%%%%%%%%%%%%%%%%%%%%%%%%%%%%%%%%%%%%%%%%%%
%\subsection{Verification of Properties (i) and (ii) from Definition \ref{defnbimultiplicative} for $\cE$}



\begin{lemma}
	\label{lemeasy}
	The operator-valued bi-free moment function $\cE$ satisfies Properties (i) and (ii) of Definition \ref{defnbimultiplicative}.
\end{lemma}

\begin{proof}
	This follows from the facts that $\cE_{1_\chi}(T_1, \ldots, T_n) = E(T_1\cdots T_n)$, that left (resp. right) variables commute with $R_b$ (resp. $L_b$) for $b \in \cB$, and that $E(T_1\cdots T_nL_b) = E(T_1\cdots T_nR_b)$.
\end{proof}




%%%%%%%%%%%%%%%%%%%%%%%%%%%%%%%%%%%%%%%%%%%%%%%%%%%%%%%%%%%%%%%%%%%%%%%
\subsection{Verification of Property (iii) from Definition~\ref{defnbimultiplicative} for $\cE$.}

\begin{lemma}
	\label{lemproductofdisjointblocksforE}
	The operator-valued bi-free moment function $\cE$ satisfies Property (iii) of Definition \ref{defnbimultiplicative}.
\end{lemma}

\begin{proof}
	We claim it suffices to consider the case when $\sigma = \set{V_1, \ldots, V_m}$ is the finest partition such that each $V_j$ is a $\chi$-interval and $\pi \leq \sigma$; indeed, given any other $\rho$ with $\pi \leq \rho$ having $\chi$-intervals $W_1 \prec_\chi \cdots \prec_\chi W_{m'}$ as blocks, applying the argument for this restricted case to the blocks of $\rho$ yields
	$$\cE_{\pi|_{W_1}}((T_i)|_{W_1}) \cdots \cE_{\pi|_{W_{m'}}}((T_i)|_{W_{m'}})
	= \cE_{\pi|_{V_1}}((T_i)|_{V_1}) \cdots \cE_{\pi|_{V_m}}((T_i)|_{V_m})
	= \cE_\pi(T_1, \ldots, T_n).$$
	Note that for $\sigma$ above, it must be the case that $\min_{\prec_\chi}(V_i) \sim_{\pi} \max_{\prec_\chi}(V_i)$, as otherwise a finer partition is possible.


	%We now proceed by induction on $m$, the number of blocks in $\sigma$, with the case $m = 1$ being immediate.
	%Assume Property (iii) of Definition \ref{defnbimultiplicative} is satisfied for $\cE$ for all smaller values of $m$.
	%Notice that the reductions in the definition of $\cE_\pi$ only depend on the block which is completed nearest the bottom of the diagram, and if it is tangled with another block, that block as well (note that it can be tangled with at most one other block since no block terminates below it).

	We now proceed by induction on $m$, the number of blocks in $\sigma$, with the case $m = 1$ being immediate.
	Assume Property (iii) of Definition \ref{defnbimultiplicative} is satisfied for $\cE$ for all smaller values of $m$.
	Fix $V_1, \ldots, V_m$ and note that either $1 \in V_1$ (i.e. $\chi(1) = \ell$) or $1 \in V_m$ (i.e. $\chi(1) = r$).
	We will treat the case when $1 \in V_1$; for the other case, consult a mirror.
	Let $V'_1 \subseteq V_1$ be the block of $\pi$ containing $1$ and $\max_{\prec_\chi}(V_1)$.
	The proof is divided into three cases.

	\paragraph{Case 1: $\min(V_k) > \max(V_1)$ for all $k \neq 1$.}

	As an example of this case, consider the following diagram where $V_1 = \set{1, 2, 3 }$, $V_2 = \set{4, 6}$, and $V_3 = \set{5, 7, 8, 9}$.
	\[
		\begin{tikzpicture}[baseline]
			\def\sidez{{-1,-1,-1,-1,1,-1,1,1,-1}}
			\def\clrz{{0,0,0,1,2,1,2,2,2}}
			\bnc[n=9,sidez=\sidez,colourzfrompalette=\clrz]
			\foreach \y in {1,3} {\draw [thick, pal0] (ball\y) -| (ball1 -| -0.5,0);}
			\foreach \y in {4,6} {\draw [thick, pal1] (ball\y) -| (ball4 -| -0.5,0);}
			\foreach \y in {5,9} {\draw [thick, pal2] (ball\y) -| (ball5 -| 0,0);}
			\foreach \y in {7,8} {\draw [thick, pal2] (ball\y) -| (ball7 -| 0.5,0);}
		\end{tikzpicture}
	\]

	In this case, drawing a horizontal line directly beneath $\max(V_1)$ will hit no spines in $\pi$ and $V_1 \subseteq \chi^{-1}(\set{\ell})$.
	Let $V'_1 = \set{1 = q_1 < q_2 < \cdots < q_p}$ and $V_0 = \bigcup^m_{k=2} V_k$.
	Repeatedly applying the definition of $E$, we may find $b_1, \ldots, b_{p-1} \in \cB$ depending only on $(T_1, \ldots, T_n)|_{V_1}$ and $\pi|_{V_1}$, so that writing $T'_{q_k} = T_{q_k}L_{b_k}$ we have
	\[
		E_\pi(T_1, \ldots, T_n)
		= E\paren{T'_{q_1} T'_{q_2} \cdots T'_{q_{p-1}}T_{q_p} L_{E_{\pi|_{V_0}}((T_1, \ldots, T_n)|_{V_0})} }
		= E\paren{T'_{q_1} T'_{q_2} \cdots T'_{q_{p-1}}T_{q_p} R_{E_{\pi|_{V_0}}((T_1, \ldots, T_n)|_{V_0})} }.
	\]
	By the assumptions in this case, each $T_k \in \A_\ell$ for all $k \in V'_1$ and since right $\cB$-operators commute with elements of $\A_\ell$, we obtain
	\begin{align*}
		E_\pi(T_1, \ldots, T_n) &= E\paren{T'_{q_1} T'_{q_2} \cdots T'_{q_p}R_{E_{\pi|_{V_0}}((T_1, \ldots, T_n)|_{V_0})} }
		\\
		&= E\paren{R_{E_{\pi|_{V_0}}((T_1, \ldots, T_n)|_{V_0})}T'_{q_1} T'_{q_2} \cdots T'_{q_p} } \\
		&= E(T'_{q_1} T'_{q_2} \cdots T'_{q_p}) E_{\pi|_{V_0}}((T_1, \ldots, T_n)|_{V_0}) \\
		&=
		E_{\pi|_{V_1}}((T_1, \ldots, T_n)|_{V_1})E_{\pi|_{V_0}}((T_1, \ldots, T_n)|_{V_0}) \\
		&= \cE_{\pi|_{V_1}}\paren{(T_1, \ldots, T_n)|_{V_1}} \cE_{\pi|_{V_2}}\paren{(T_1, \ldots, T_n)|_{V_1}} \cdots \cE_{\pi|_{V_m}}\paren{(T_1, \ldots, T_n)|_{V_m}}
	\end{align*}
	with the last step following by the inductive hypothesis.

	If we are not in Case 1, then there exists a $k \neq 1$ such that $\min(V_k) < \max(V_1)$.
	In particular, $V_m$ must terminate on the right above $\max(V_1)$, so $\min(V_m) < \max(V_1)$ and $\chi(\min(V_m)) = r$.
	We thus find that there are two further cases.

	\paragraph{Case 2: $\max(V_1) < \max(V_m)$.}
	As an example of this case, consider the following diagram where $V_1 = \set{1, 3 }$, $V_2 = \set{4, 6, 7, 8, 9}$, and $V_3 = \set{2, 5}$.
	\[
		\begin{tikzpicture}[baseline]
			\def\sidez{{-1,1,-1,-1,1,1,-1,1,-1}}
			\def\clrz{{0,2,0,1,2,1,1,1,1}}
			\bnc[n=9,sidez=\sidez,colourzfrompalette=\clrz]
			\foreach \y in {1,3} {\draw [thick, pal0] (ball\y) -| (ball1 -| 0,0);}
			\foreach \y in {2,5} {\draw [thick, pal2] (ball\y) -| (ball5 -| 0.5,0);}
			\foreach \y in {4,6} {\draw [thick, pal1] (ball\y) -| (ball4 -| 0,0);}
			\foreach \y in {7,8,9} {\draw [thick, pal1] (ball\y) -| (ball7 -| 0,0);}
		\end{tikzpicture}
	\]

	Again $V_1 \subseteq \chi^{-1}(\set{\ell})$, as its lowest element is higher than the lowest element of $V_m$.
	With the same conventions as above, by repeated application of the rules defining $E$, we obtain
	\[
		E_\pi(T_1, \ldots, T_n) = E\paren{T'_{q_1} T'_{q_2} \cdots T'_{q_{p_1}} R_{E_{\pi|_{V_0}}((T_1, \ldots, T_n)|_{V_0})} T'_{q_{p_1+1}} \cdots T'_{q_{p_2}}},
	\]
	where $p_1$ is the smallest element of $V_1'$ greater than $\min(V_m)$.
	Note that all of the other blocks of $\pi$ are consumed into the block $V_m$: blocks are only consumed into either the lowest block or a block they are tangled with, and not block except $V_m$ may be tangled with $V_1$.
	By the assumptions in this case, each $T_k \in \A_\ell$ for all $k \in V'_1$, and since right $\cB$-operators commute with elements of $\A_\ell$, one obtains
	\begin{align*}
		E_\pi(T_1, \ldots, T_n) &=
		E\paren{T'_{q_1} T'_{q_2} \cdots T'_{q_{p_1}} R_{E_{\pi|_{V_0}}((T_1, \ldots, T_n)|_{V_0})} T'_{q_{p_1+1}} \cdots T'_{q_{p_2}}} \\
		&= E\paren{R_{E_{\pi|_{V_0}}((T_1, \ldots, T_n)|_{V_0})}T'_{q_1} T'_{q_2} \cdots T'_{q_{p_2}} } \\
		&= E(T'_{q_1} T'_{q_2} \cdots T'_{q_{p_2}}) E_{\pi|_{V_0}}((T_1, \ldots, T_n)|_{V_0}) \\
		&=
		E_{\pi|_{V_1}}((T_1, \ldots, T_n)|_{V_1})E_{\pi|_{V_0}}((T_1, \ldots, T_n)|_{V_0})\\
		&= \cE_{\pi|_{V_1}}\paren{(T_1, \ldots, T_n)|_{V_1}} \cE_{\pi|_{V_2}}\paren{(T_1, \ldots, T_n)|_{V_1}} \cdots \cE_{\pi|_{V_m}}\paren{(T_1, \ldots, T_n)|_{V_m}},
	\end{align*}
	with the last step following by the inductive hypothesis.

	\paragraph{Case 3: $\max(V_1) > \max(V_m)$.}
	As an example of this case, consider the following diagram where $V_1 = \set{1, 5 }$, $V_2 = \set{ 6, 8, 9}$, $V_3 = \set{4, 7}$, and $V_4 = \set{2, 3}$.
	\[
		\begin{tikzpicture}[baseline]
			\def\sidez{{-1,1,1,1,-1,-1,1,1,-1}}
			\def\clrz{{0,3,3,2,0,1,2,1,1}}
			\bnc[n=9,sidez=\sidez,colourzfrompalette=\clrz]
			\foreach \y in {1,5} {\draw [thick, pal0] (ball\y) -| (ball1 -| -0.2,0);}
			\foreach \y in {2,3} {\draw [thick, pal3] (ball\y) -| (ball3 -| 0.2,0);}
			\foreach \y in {4,7} {\draw [thick, pal2] (ball\y) -| (ball4 -| 0.2,0);}
			\foreach \y in {6,8,9} {\draw [thick, pal1] (ball\y) -| (ball8 -| -0.2,0);}
		\end{tikzpicture}
	\]

	Let $V_0 = \bigcup^{m-1}_{k=1} V_k$.
	Once again appealing to the properties of $E$ given in Definition~\ref{defn:recursivedefinitionofEpi} we may find $T'_{q_1}, \ldots, T'_{q_{p_1}}$ and $S \in \A$ where $T'_k$ differs from $T_k$ by a left multiplication operator, so that
	\[
		E_\pi(T_1, \ldots, T_n) = E\paren{T'_{q_1} T'_{q_2} \cdots T'_{q_{p_1}} R_{E_{\pi|_{V_m}}((T_1, \ldots, T_n)|_{V_m})} S}. 
	\]
	Since right $\cB$-operators commute with elements of $\A_\ell$, one obtains
	\begin{align*}
		E_\pi(T_1, \ldots, T_n) &=
		E\paren{T'_{q_1} T'_{q_2} \cdots T'_{q_{p_1}} R_{E_{\pi|_{V_m}}((T_1, \ldots, T_n)|_{V_m})} S} \\
		&= E\paren{R_{E_{\pi|_{V_m}}((T_1, \ldots, T_n)|_{V_m})}T'_{q_1} T'_{q_2} \cdots T'_{q_{p_1}}S } \\
		&= E(T'_{q_1} T'_{q_2} \cdots T'_{q_{p_1}}S) E_{\pi|_{V_m}}((T_1, \ldots, T_n)|_{V_m}) \\
		&=
		E_{\pi|_{V_0}}((T_1, \ldots, T_n)|_{V_0})E_{\pi|_{V_m}}((T_1, \ldots, T_n)|_{V_m})\\
		&= \cE_{\pi|_{V_1}}\paren{(T_1, \ldots, T_n)|_{V_1}} \cE_{\pi|_{V_2}}\paren{(T_1, \ldots, T_n)|_{V_1}} \cdots \cE_{\pi|_{V_m}}\paren{(T_1, \ldots, T_n)|_{V_m}}
	\end{align*}
	with the last step following by the inductive hypothesis.
\end{proof}




%%%%%%%%%%%%%%%%%%%%%%%%%%%%%%%%%%%%%%%%%%%%%%%%%%%%%%%%%%%%%%%%%%%%%%%
\subsection{Verification of Property~(iv) from Definition~\ref{defnbimultiplicative} for $\cE$.}


We begin with the following intermediate step on the way to verifying that $\cE$ satisfies Property (iv).
Recall that in the context of Definition~\ref{defnbimultiplicative} Property~\ref{def:bimult:iv}, we have an inner $\chi$-interval $V$, $W := [n]\setminus V$, and we have labelled the nodes which are immediately before and after $V$ in the $\prec_\chi$-order as $\theta$ and $\gamma$, respectively.

\begin{lemma}
	\label{lemreductionofbimultiplicativeinsideablockforE}
	The operator-valued bi-free moment function $\cE$ satisfies Property~\ref{def:bimult:iv} of Definition~\ref{defnbimultiplicative} with the additional assumption that there exists a block $W_0 \subseteq W$ of $\pi$ such that 
	\[
		\theta, \gamma, \min_{\prec_\chi}([n]),
		\max_{\prec_\chi}([n]) \in W_0.
	\]
\end{lemma}

\begin{proof}
	We will present only the proof of the case $\chi(\theta) = \ell$ as the other case is similar.

	Let $\set{V_1 \prec_\chi \ldots \prec_\chi V_m}$ be the finest partition of $V$ consisting of $\chi$-intervals which has $\pi|_V$ as a refinement.
	Note that $\theta$ immediately precedes $\min_{\prec_\chi}(V_1)$ and $\gamma$ immediately follows $\max_{\prec_\chi}(V_m)$.

	The proof is now divided into three cases.
	In the case $\chi(n) = \ell$, we have $\chi \equiv \ell$ since $q = -\infty$.

	\paragraph{Case 1: $\chi(\gamma) = \ell$.}
	As an example of this case, consider the following diagram where $W = W_0 = \set{1, 5, 9 }$, $V_1 = \set{2, 3 }$, $V_2 = \set{4, 6, 7, 8}$, $\theta = 1$, and $\gamma = 9$.
	\[
		\begin{tikzpicture}[baseline]
			\def\sidez{{-1,-1,-1,-1,1,-1,-1,-1,-1}}
			\def\clrz{{0,1,1,1,0,1,1,1,0}}
			\bnc[n=9,sidez=\sidez,colourzfrompalette=\clrz]
			\foreach \y in {1,5,9} {\draw [thick, pal0] (ball\y) -| (ball1 -| 0.2,0);}
			\foreach \y in {2,3} {\draw [thick, pal1] (ball\y) -| (ball3 -| -0.2,0);}
			\foreach \y in {4,6,8} {\draw [thick, pal1] (ball\y) -| (ball4 -| -0.2,0);}
		\end{tikzpicture}
	\]

	In this case $V \subseteq \chi^{-1}(\set{\ell})$.
	Write $X_k = L_{E_{\pi|_{V_k}}((T_1, \ldots, T_n)|_{V_k})}$ and $W_0 = \set{q_1 < q_2 < \cdots < q_{k_{m+1}}}$.
	Then
	\[
		E_\pi(T_1, \ldots, T_n) = E\paren{ T'_{q_1} T'_{q_2} \cdots T'_{q_{k_1}} X_1 T'_{q_{k_1+1}} \cdots T'_{q_{k_m}} X_m
		T'_{q_{k_m+1}} \cdots T'_{q_{k_{m+1}}}},
	\]
	where $T_k'$ is $T_k$, potentially multiplied on the left and/or right by appropriate $L_b$ and $R_b$.
	Here $T_\theta$ appears left of $X_1$, $\gamma = q_{k_{m+1}}$, and every operator between $T_\theta$ and $T_\gamma$ is either some $X_k$ or a right operator.
	Hence, by the commutation of left $\cB$-operators with elements of $\A_r$, we obtain
	\[
		E_\pi(T_1, \ldots, T_n) = E\paren{ T'_{q_1} \cdots T'_{q_{j-1}}
			R_{b}L_{b'}\paren{T_{\theta} X_1 X_2 \cdots X_m} R_{b''} T'_{q_{j+1}} \cdots T'_{q_{k_{m+1}}}
		}
	\]
	for some $b,b',b'' \in \cB$.
	Since
	\[
		\cE_{\pi|_{V_1}}((T_1, \ldots, T_n)|_{V_1}) \cdots \cE_{\pi|_{V_m}}((T_1, \ldots, T_n)|_{V_m}) = \cE_{\pi|_{V}}((T_1, \ldots, T_n)|_{V}),
	\]
	by Lemma~\ref{lemproductofdisjointblocksforE}, we have
	\begin{align*}
		\cE_\pi(T_1, \ldots, T_n)
		&= E\paren{T'_{q_1} \cdots T'_{q_{j-1}}
		R_{b}L_{b'}\paren{T_{\theta} L_{\cE_{\pi|_{V}}((T_1, \ldots, T_n)|_{V})}} R_{b''} T'_{q_{j+1}} \cdots T'_{q_{k_{m+1}}} } \\
		&= \cE_{\pi|_{W}}\paren{\left.\paren{T_1, \ldots, T_{\theta-1}, T_\theta L_{\cE_{\pi|_{V}}\paren{(T_1,\ldots, T_n)|_{V}}}, T_{\theta+1}, \ldots, T_n}\right|_{W}}
		\end{align*}
		where the last step follows as $\cE_{\pi |_W}$ ignores arguments corresponding to $V$.

		\paragraph{Case 2: $\chi(\gamma) = r$ and $\theta < \gamma$.}
		As an example of this case, consider the following diagram where $W = W_0 = \set{1, 3, 6 }$, $V_1 = \set{2, 4 }$, $V_2 = \set{5,
		8}$, $V_3 = \set{7, 9}$, $\theta = 1$, and $\gamma = 6$.
		\[
			\begin{tikzpicture}[baseline]
				\def\sidez{{-1,-1,1,-1,-1,1,1,-1,-1}}
				\def\clrz{{0,1,0,1,1,0,1,1,1}}
				\bnc[n=9,sidez=\sidez,colourzfrompalette=\clrz]
				\foreach \y in {1,3,6} {\draw [thick, pal0] (ball\y) -| (ball1 -| 0.2,0);}
				\foreach \y in {2,4} {\draw [thick, pal1] (ball\y) -| (ball2 -| -0.2,0);}
				\foreach \y in {5,8} {\draw [thick, pal1] (ball\y) -| (ball5 -| -0.2,0);}
				\foreach \y in {7,9} {\draw [thick, pal1] (ball\y) -| (ball7 -| 0.2,0);}
			\end{tikzpicture}
		\]

		Let $p$ be the index of the last sub-interval of $V$ to begin above $\gamma$, if such exists, and $0$ otherwise.
		That is, let $p$ be such that $\min(V_k) > \gamma$ if and only if $k > p$.
		Note that if $k < p$ we have $V_k \subset \chi^{-1}(\ell)$, and if $p > 0$, $\chi(\min(V_p)) = \ell$.

		Considering the process which reductions are performed in evaluating $E$, we find that every block of $\pi$ contained in $V_k$ with $k>p$ will wind up either attached to a node in $V_p$ or multiplied on the right of $T_\gamma$, depending on whether $V_p$ terminates above or below $\gamma$.
		Letting $W_0 = \set{q_1 < \cdots < q_t}$, we find that there are $T'_k$ which differ from $T_k$ by multiplication by $L_b$ and/or $R_b$ (these corresponding to the reduction of other blocks in $W$) so that $E_\pi(T_1, \ldots, T_n)$ is of the form
		$$E\paren{T_{q_1}' \cdots T_\theta' Z T_\gamma' L_Y},$$
		where: $Z$ is a product of $T_{q_j}'$ having only right pieces (with $\theta < q_j < q_{j+1} < \gamma$), $L_{E_{\pi|_{V_k}}}((T_i)|_{V_k}$ with $k \leq p$, and possibly one term of the form $L_{E_{\pi|_{V_{\geq p}}}((T_i)|_{V_{\geq p}})}$; and $Y$ is either $L_{E_{\pi|_{V_{> p}}}((T_i)|_{V_{>p}})}$ or $1$.
			The point, though, is that we can commute all the terms arising from $V$ next to each other next to $T_\theta$, and use Lemma~\ref{lemproductofdisjointblocksforE} to replace their product by $L_{E_{\pi|_V}((T_i)|_V)}$:
			\begin{align*}
				E_\pi(T_1,\ldots, T_n)
				&= E(T_{q_1}' \cdots T_{q_{j-1}}' T_\theta' L_{E_{\pi|_V}((T_i)|_V)} T_{q_{j+1}}' \cdots T_\gamma') \\
				&= E_{\pi|_W}\paren{\paren{T_1, \ldots, T_{\theta-1}, T_\theta L_{\cE_{\pi|_V}\paren{(T_i)|_V}}, T_{\theta+1},\ldots, T_n}|_W},
			\end{align*}
			where the second equality follows by reversing the sequence of reductions which compressed the blocks of $W$ and created the $T_{q_i}'$, as these reductions could not be influenced by the presence or absences of $V$, being separated from it by $W_0$.





			\paragraph{Case 3: $\chi(\gamma) = r$ and $\theta > \gamma$.}
			The argument in this case is essentially the same as the above, except one finds that terms from $V$ are collected as right multiplication operators rather than as left ones, and always occurring after $T_\gamma$ in the expansion of the product.
			All things arising from $W$ after $T_\gamma$ must be left operators, though, and so they commute with the terms coming from the reduction of $V$ which can the be collected as a right multiplication operator after $T_\theta$.
			Since $T_\theta$ must be the last operator in $W$, we are able to replace this right multiplication coming from $V$ with a left one, as required by Definition~\ref{defnbimultiplicative}:
			\begin{align*}	
				E_\pi(T_1, \ldots, T_n)
				= E(T_{q_1}' \cdots T_\theta' R_{\cE_{\pi|_V}((T_i)|_V)})
				= E(T_{q_1}' \cdots T_\theta' L_{\cE_{\pi|_V}((T_i)|_V)}).
				&\qedhere
			\end{align*}
		\end{proof}

		In order to establish that Property~\ref{def:bimult:iv} holds without our special assumption above, it will be useful to prove the following stronger versions of Properties~\ref{def:bimult:i} and \ref{def:bimult:ii}.

		\begin{lemma}
			\label{lemenhancedpropertyiforE}
	%The operator-valued bi-free moment function $\cE$ satisfies the $q = -\infty$ case of Property (i) of Definition \ref{defnbimultiplicative} when $1_\chi$ is replaced with an arbitrary $\pi \in \BNC(\chi)$.
			The operator-valued bi-free moment function $\cE$ satisfies Property~\ref{def:bimult:i} of Definition~\ref{defnbimultiplicative} when $\chi$ is constant and $1_\chi$ is replaced by an arbitrary $\pi \in \BNC(\chi)$.
		\end{lemma}


		\begin{proof}
			We will demonstrate the case $\chi \equiv \ell$, as the other case follows mutatis mutandis.
			Notice that $\chi$ being constant means that $q := \max\set{k\in[n] | \chi(k) \neq \chi(n)} = -\infty$.
			Let $\set{V_1 \prec_\chi \ldots \prec_\chi V_m}$ be the finest partition of $V$ consisting of $\chi$-intervals which has $\pi|_V$ as a refinement, and let $V_i'$ be the outer block of $\pi$ contained in $V_i$.
			By Lemma \ref{lemproductofdisjointblocksforE}, we may assume $m = 1$.


			Writing $V_1' = \set{1 = q_1 < q_2 < \cdots < q_{p+1} = n}$, for some $b_j \in \cB$ depending only on $(T_1, \ldots, T_n)|_{(V'_1)^c}$ and on $\pi$,
			\[
				E_\pi(T_1, \ldots, T_n) = E\paren{ T_{q_1} L_{b_1} T_{q_2} L_{b_2} \cdots T_{q_p} L_{b_p} T_{q_{p+1}}}.
			\]
			Hence, by the commutation of right $\cB$-operators with elements of $\A_\ell$, we obtain
			\begin{align*}
				\cE_\pi(T_1, \ldots, T_n)b & = E\paren{ T_{q_1} L_{b_1} T_{q_2} L_{b_2} \cdots T_{q_p} L_{b_p} T_{q_{p+1}}}b\\
				& = E\paren{ R_b T_{q_1} L_{b_1} T_{q_2} L_{b_2} \cdots T_{q_p} L_{b_p} T_{q_{p+1}}}\\
				& = E\paren{ T_{q_1} L_{b_1} T_{q_2} L_{b_2} \cdots T_{q_p} L_{b_p} T_{q_{p+1}}R_b }\\
				& = E\paren{ T_{q_1} L_{b_1} T_{q_2} L_{b_2} \cdots T_{q_p} L_{b_p} T_{q_{p+1}}L_b }\\
				& = \cE_\pi(T_1, \ldots, T_n L_b).\qedhere
			\end{align*}
		\end{proof}


		\begin{lemma}
			\label{lemenhancedpropertyiiforE}
			The operator-valued bi-free moment function $\cE$ satisfies Property~\ref{def:bimult:ii} of Definition~\ref{defnbimultiplicative} when $q$ in the context of that definition is $-\infty$, and $1_\chi$ is replaced by an arbitrary $\pi \in \BNC(\chi)$.
		\end{lemma}


		\begin{proof}
			We will assume $\chi(p) = \ell$ as the case where $\chi(p) = r$ is once again similar.
			Let $\set{V_1 \prec_\chi \ldots \prec_\chi V_m}$ be the finest partition of $V$ consisting of $\chi$-intervals which has $\pi|_V$ as a refinement, and let $V_i'$ be the outer block of $\pi$ contained in $V_i$.
			Notice that since $p$ is the first node on the left side of the partition, we necessarily have $p \in V'_1$.
			Thus Lemma \ref{lemproductofdisjointblocksforE} implies we may reduce to the case where $m = 1$.



			Notice that any block which will be contracted on the left in the evaluation of $\cE_\pi$ must be below $p$; then writing $V_1' = \set{q_1 < q_2 < \cdots < q_{k}}$, for some $b_j \in \cB$ depending only on $(T_1, \ldots, T_n)|_{(V'_1)^c}$ and on $\pi$, for some $S \in \A$, and for some $z < k$,
			\[
				\cE_\pi(T_1, \ldots, T_n) = E\paren{ T_{q_1} R_{b_1}
				\cdots T_{q_z} R_{b_z} T_p S}.
			\]

			Hence, by the commutation of left $\cB$-operators with elements of $\A_r$, we obtain
			\begin{align*}
				b\cE_\pi(T_1, \ldots, T_n) & =bE\paren{ T_{q_1} R_{b_1}
				\cdots T_{q_z} R_{b'_z} T_p S}\\
				& =E\paren{L_bT_{q_1} R_{b_1}
				\cdots T_{q_z} R_{b_z} T_p S}\\
				& = E\paren{ T_{q_1} R_{b_1}
				\cdots T_{q_z} R_{b_z} L_bT_p S}\\
				& = \cE_\pi(T_1, \ldots, T_{p-1}, L_b T_p, T_{p+1}, \ldots, T_n).
				\qedhere
			\end{align*}
		\end{proof}

		\begin{lemma}
			\label{lemfullreductionofproductsinvolvingblocksinsideblocks}
			The operator-valued bi-free moment function $\cE$ satisfies Property~\ref{def:bimult:iv} of Definition \ref{defnbimultiplicative}.
		\end{lemma}

		\begin{proof}
			Again, only the proof of the first case where $\chi(\theta) = \ell$ will be presented.
			We proceed by induction on the number of blocks $U\in\pi$ with
			\[
				U\subseteq W, \qquad
				\min_{\prec_\chi}(U) \prec_{\chi} \min_{\prec_\chi}(V), \qquad
				\text{and} \qquad
				\max_{\prec_\chi}(V)
				\prec_{\chi} \max_{\prec_\chi}(U),
			\]
			which we will denote by $m$.
			Such blocks are the ones which enclose $V$.

			We will first treat the case $m=0$.
			Let
			\[
				W_1 = \set{k \in [n] \, \mid \, k \preceq_\chi \theta} \qquad
				\text{and} \qquad
				W_2 = \set{k \in [n] \, \mid \, \gamma \preceq_\chi k}.
			\]
			Now both $W_1$ and $W_2$ are $\chi$-intervals that are unions of blocks of $\pi$ such that $W = W_1 \sqcup W_2$, and $W_1 \subseteq \chi^{-1}(\ell)$.
			Therefore by Lemmata \ref{lemproductofdisjointblocksforE} and \ref{lemenhancedpropertyiforE},
			\begin{align*}
				\cE_\pi(T_1,\ldots, T_n) &=\cE_{\pi|_{W_1}}((T_1,\ldots, T_n)|_{W_1})\cE_{\pi|_{V}}((T_1,\ldots, T_n)|_{V})\cE_{\pi|_{W_2}}((T_1,\ldots, T_n)|_{W_2}) \\
				& = \cE_{\pi|_{W_1}}((T_1,\ldots, T_{\theta - 1}, T_\theta L_{\cE_{\pi|_{V}}((T_1,\ldots, T_n)|_{V})})|_{W_1})\cE_{\pi|_{W_2}}((T_1,\ldots, T_n)|_{W_2}) \\
				& =\cE_{\pi|_{W}}((T_1,\ldots, T_{\theta - 1}, T_\theta L_{\cE_{\pi|_{V}}((T_1,\ldots, T_n)|_{V})}, T_{\theta + 1}, \ldots, T_n)|_{W}).
			\end{align*}
			Note that we would either invoke Lemma \ref{lemenhancedpropertyiiforE} instead of \ref{lemenhancedpropertyiforE} in the case $\chi(\theta) = r$, or else bundle $\cE_{\pi|_V}((T_i)|_V)$ into the expectation corresponding to $W_2$.

			We must also establish the case $m = 1$ before we can begin the inductive step.
			Let $W_0$ be the corresponding block counted by $m$, and let
			\begin{align*}
				\alpha_1 &= \min_{\prec_\chi}(W_0),
				& & \alpha_2 = \max_{\prec_\chi}(\set{ k \in W_0 \, \mid \, k \preceq_{\chi} \theta
				}), \\
				\beta_1& = \max_{\prec_\chi}(W_0), & \text{and} \qquad \qquad \qquad &
				\beta_2 = \min_{\prec_\chi}(\set{ k \in W_0 \, \mid \, \gamma \preceq_\chi k
				}).
			\end{align*}
			Furthermore, let
			\begin{align*}
				W'_1 &= \set{k \in [n] \, \mid \, k \prec_\chi \alpha_1 }, & &
				W'_2 = \set{k \in [n] \, \mid \,
				\beta_1 \prec_\chi k}, \\
				W''_1 &= \set{k \in [n] \, \mid \, \alpha_2 \prec k \preceq_\chi \theta},
				& \text{and} \qquad \quad &
				W''_2 = \set{k \in [n] \, \mid \, \gamma \preceq_\chi k \prec_\chi \beta_2}.
			\end{align*}
			Representing things graphically,
			\[
				\begin{tikzpicture}
					\def\sdz{{0,-1,1,1,-1, -1, 1}}
					\def\labelz{{"", "$\alpha_1$", "$\beta_1$", "$\beta_2$", "$\alpha_2$", "$\theta$", "$\gamma$"}}
					\def\ord{{0,0,1,2,0,3,4,0,5,6,0,0}}
					\bnc[n=12,order=\ord,sidez=\sdz,labelz=\labelz]
					\draw [thick] ($ (ball3) + (-0.25, 0.25) $) -- ++(-0.25,0) |- node[pos=.25,left]{$W_1'$} ($ (tl) + (-0.25,0) $);
					\draw [thick] ($ (ball4) + (0.25, 0.25) $) -- ++(0.25,0) |- node[pos=.25,right]{$W_2'$} ($ (tr) + (0.25,0) $);
					\draw [thick] ($ (ball6) + (0.25, -0.25) $) -- ++(0.25,0) |- node[pos=.25,right]{$W_2''$} ($ (ball10) + (0.25,-0.25) $);
					\draw [thick] ($ (ball7) + (-0.25, -0.25) $) -- ++(-0.25,0) |- node[pos=.25,left]{$W_1''$} ($ (ball9) + (-0.25,-0.25) $);
					\draw [thick] ($ (ball10) + (0,-0.5) $) -- coordinate (hi) ($ (ball9) + (0,-0.5) $);
					\node at ($ (hi) ! .5 ! (bc) $) {$V$};

					\node at ($ ($ (ball3) ! .5 ! (ball7) $) + (0.5,0) $) {$\vdots$};
					\node at ($ ($ (ball4) ! .5 ! (ball6) $) + (-0.5,0) $) {$\vdots$};

					\foreach \y in {3,4,6,7} {\draw [thick] (ball\y) -| (ball7 -| 0,0);}

				\end{tikzpicture}
			\]

			Therefore, if
			\begin{align*}
				X'_1 &= \cE_{\pi|_{W'_1}}((T_1,\ldots, T_n)|_{W'_1}), & & X'_2 = \cE_{\pi|_{W'_2}}((T_1,\ldots, T_n)|_{W'_2}), \\
				X''_1 &= \cE_{\pi|_{W''_1}}((T_1,\ldots, T_n)|_{W''_1}), & \text{and} \qquad \qquad \qquad & X''_2 = \cE_{\pi|_{W''_2}}((T_1,\ldots, T_n)|_{W''_2}),
			\end{align*} 
			then by Lemmata \ref{lemproductofdisjointblocksforE} and \ref{lemreductionofbimultiplicativeinsideablockforE},
			\begin{align*}
				\cE_\pi(T_1,\ldots, T_n)
				&= X'_1 \cE_{\pi|_{W_0 \cup W''_1 \cup W''_2 \cup V}}((T_1,\ldots, T_n)|_{W_0 \cup W''_1 \cup W''_2 \cup V})X'_2 \\
				&= X'_1 \cE_{\pi|_{W_0 }}\paren{\paren{T_1,\ldots, T_{\alpha_2 - 1}, T_{\alpha_2}L_{\cE_{\pi|_{W''_1 \cup W''_2 \cup V}}((T_1, \ldots, T_n)|_ {W''_1 \cup W''_2 \cup V})}, T_{\alpha_2+1},\ldots, T_n}|_{W_0}}X'_2 \\
				&= X'_1\cE_{\pi|_{W_0 }}\paren{\paren{T_1,\ldots, T_{\alpha_2 - 1}, T_{\alpha_2} L_{X''_1 \cE_{\pi|_{V}}((T_1, \ldots, T_n)|_ {V})
				X''_2}, T_{\alpha_2+1},\ldots, T_n}|_{W_0}}X'_2.
			\end{align*}

			If $W''_1$ is empty, then $\alpha_2 = \theta$ and 
			\[
				T_{\alpha_2} L_{X''_1
					\cE_{\pi|_{V}}((T_1, \ldots, T_n)|_ {V})
				X''_2} = T_{\theta} L_{\cE_{\pi|_{V}}((T_1, \ldots, T_n)|_ {V})} L_{X''_2}.
			\]
			On the other hand, if $W''_1$ is non-empty, then Lemma \ref{lemenhancedpropertyiforE} implies that
			\[
				X''_1\cE_{\pi|_{V}}((T_1, \ldots, T_n)|_ {V}) = \cE_{\pi|_{W''_1}}((T_1, \ldots, T_{\theta - 1}, T_\theta L_{\cE_{\pi|_{V}}((T_1, \ldots, T_n)|_ {V})}, T_{\theta + 1}, \ldots, T_n)|_ {W''_1}).
			\] 
			The result follows now from Lemmata \ref{lemproductofdisjointblocksforE} and \ref{lemreductionofbimultiplicativeinsideablockforE} in the direction opposite the above.




			Inductively, suppose that the result holds when the number of enclosing blocks is at most $m$.
			Suppose $W$ contains blocks $W_0, \ldots, W_m$ of $\pi$ which satisfy the above enclosing inequalities, ordered so that
			\[
				\min_{\prec_\chi}(W_0) \prec_\chi \cdots \prec_\chi \min_{\prec_\chi}(W_m).
			\]
			Note that as $\pi$ is bi-non-crossing, this implies
			\[
				\max_{\prec_\chi}(W_m) \prec_\chi \cdots \prec_\chi \max_{\prec_\chi}(W_0).
			\]
			Let $\alpha_1, \alpha_2, \beta_1, \beta_2$, $W'_1, W'_2, X'_1,$ and $X'_2$ be as above.
			Hence applying Lemmata \ref{lemproductofdisjointblocksforE} and \ref{lemreductionofbimultiplicativeinsideablockforE} once again gives
			\begin{align*}
				\cE_\pi&(T_1,\ldots, T_n) \\
				&= X'_1\cE_{\pi|_{(W'_1 \cup W'_2)^c}}((T_1,\ldots, T_n)|_{(W'_1 \cup W'_2)^c})X'_2\\
				&= X'_1\cE_{\pi|_{W_0 }}\paren{\paren{T_1,\ldots, T_{\alpha_2 - 1}, T_{\alpha_2}L_{\cE_{\pi|_{(W_0 \cup W'_1 \cup W'_2)^c}}((T_1, \ldots, T_n)|_ {(W_0 \cup W'_1 \cup W'_2)^c})}, T_{\alpha_2+1},\ldots, T_n}|_{W_0}}X'_2.
			\end{align*}
			Now, by the inductive hypothesis, we see that
			\begin{align*}
				\cE_{\pi|_{(W_0 \cup W'_1 \cup W'_2)^c}}&((T_1, \ldots, T_n)|_ {(W_0 \cup W'_1 \cup W'_2)^c}) \\
				&
				= \cE_{\pi|_{(W_0 \cup W'_1 \cup W'_2)^c \setminus V}}((T_1, \ldots, T_{\theta - 1}, T_\theta L_{\cE_{\pi|_V}((T_1,\ldots, T_n)|_{V})}, T_{\theta + 1}, \ldots, T_n)|_ {(W_0 \cup W'_1 \cup W'_2)^c\setminus V}).
			\end{align*}
			Hence, by substituting this expression into the above computation and applying Lemmata \ref{lemproductofdisjointblocksforE} and \ref{lemreductionofbimultiplicativeinsideablockforE} in the opposite order, the inductive step is complete.
		\end{proof}



		With this result, the proof of Theorem \ref{thm:samantha} is complete.
























%%%%%%%%%%%%%%%%%%%%%%%%%%%%%%%%%%%%%%%%%%%%%%%%%%%%%%%%%%%%%%%%%%%%
%	    Operator-Valued Bi-Free Cumulants   	   %
%%%%%%%%%%%%%%%%%%%%%%%%%%%%%%%%%%%%%%%%%%%%%%%%%%%%%%%%%%%%%%%%%%%%

		\section{Operator-valued bi-free cumulants.}
		\label{sec:operatorvaluedbifreecumulants}






%%%%%%%%%%%%%%%%%%%%%%%%%%%%%%%%%%%%%%%%%%%%%%%%%%%%%%%%%%%%%%%%%%%%%%%
		\subsection{Cumulants via convolution.}
		Following \cite{speicher1998combinatorial}*{Definition 2.1.6}, we begin with a definition of operator-valued convolution.

		\begin{definition}
			Let $(\A, E, \varepsilon)$ be a $\cB$-$\cB$-non-commutative probability space, let
			\[
				\Phi : \bigcup_{n\geq 1} \bigcup_{\chi : [n] \to \slr} \BNC(\chi) \times \A_{\chi(1)} \times \cdots \times \A_{\chi(n)} \to \cB,
			\]
			and let $f \in IA(\BNC)$.
			We define \emph{the convolution of $\Phi$ and $f$}, denoted $\Phi \st f$, by
			\[
				(\Phi \st f)_\pi(T_1, \ldots, T_n) := \sum_{\substack{\sigma \in \BNC(\chi)\\ \sigma \leq \pi}} \Phi_\sigma(T_1,\ldots, T_n) f(\sigma, \pi)
			\]
			for all $\chi : [n] \to \slr$, $\pi \in \BNC(\chi)$, and $T_k \in \A_{\chi(k)}$.
		\end{definition}

		One can check that if $\Phi$ is as above and $f,g \in IA(\BNC)$ then $(\Phi \st f) \st g = \Phi \st (f \st g)$; this comes down to swapping the order of two finite sums.

		\begin{definition}
			\label{def:kappa}
			Let $(\A, E, \varepsilon)$ be a $\cB$-$\cB$-non-commutative probability space and let $\cE$ be the operator-valued bi-free moment function on $\A$.
			The \emph{operator-valued bi-free cumulant function} is the function
			\[
				\kappa : \bigcup_{n\geq 1} \bigcup_{\chi : [n] \to \slr} \BNC(\chi) \times \A_{\chi(1)} \times \cdots \times \A_{\chi(n)} \to B
			\]
			defined by
			\[
				\kappa := \cE \st \mu_{\BNC}.
			\]
		\end{definition}

		Note for $\chi : [n] \to \slr$, $\pi \in \BNC(\chi)$, and $T_k \in \A_{\chi(k)}$ that
		\[
			\kappa_\pi(T_1,\ldots, T_n) = \sum_{\sigma \leq \pi} \cE_\sigma(T_1, \ldots, T_n) \mu_{\BNC}(\sigma, \pi)
			\qquad \text{ and }\qquad
			\cE_\pi(T_1, \ldots, T_n) = \sum_{\sigma \leq \pi} \kappa_\sigma(T_1, \ldots, T_n).
		\]
		Also observe that if $\cB = \C$ is the complex numbers, the operator-valued bi-free cumulant function $\kappa$ is precisely the usual bi-free cumulant functional.





%%%%%%%%%%%%%%%%%%%%%%%%%%%%%%%%%%%%%%%%%%%%%%%%%%%%%%%%%%%%%%%%%%%%%%%
		\subsection{Convolution preserves bi-multiplicativity.}
		It is now straightforward to demonstrate the operator-valued bi-free cumulant function is bi-multiplicative, as a corollary to the following theorem:

		\begin{theorem}
			\label{thmbimulticonvoledwithmultiisbimult}
			Let $(\A, E, \varepsilon)$ be a $\cB$-$\cB$-non-commutative probability space, let
			\[
				\Phi : \bigcup_{n\geq 1} \bigcup_{\chi : [n] \to \slr} \BNC(\chi) \times \A_{\chi(1)} \times \cdots \times \A_{\chi(n)} \to \cB, 
			\]
			and let $f \in IA(\BNC)$.
			If $\Phi$ is bi-multiplicative and $f$ is multiplicative, then $\Phi \st f$ is bi-multiplicative.
		\end{theorem}

		\begin{proof}
			Clearly $(\Phi \st f)_\pi$ is linear in each entry.
			Furthermore, Proposition \ref{propenhancedproperties} establishes that $\Phi \st f$ satisfies Properties (i) and (ii) of Definition \ref{defnbimultiplicative}.
			Thus it remains to verify Properties (iii) and (iv).

			Suppose the hypotheses of Property (iii).
			We see that
			\begin{align*}
				(\Phi \st f)_\pi&(T_1,\ldots, T_n) \\
				&= \sum_{\sigma \leq \pi} \Phi_\sigma(T_1, \ldots, T_n) f(\sigma, \pi)
				\\
				&= \sum_{\sigma \leq \pi} \Phi_{\sigma|_{V_1}}((T_1,\ldots, T_n)|_{V_1})
				\cdots \Phi_{\sigma|_{V_m}}((T_1,\ldots, T_n)|_{V_m})
				f(\sigma|_{V_1}, \pi|_{V_1}) \cdots f(\sigma|_{V_m}, \pi|_{V_m})
				\\
				&= (\Phi \st f)_{\pi|_{V_1}}((T_1,\ldots, T_n)|_{V_1})
				\cdots (\Phi \st f)_{\pi|_{V_m}}((T_1,\ldots, T_n)|_{V_m}),
			\end{align*}
			using the bi-multiplicativity of $\Phi$ and the multiplicativity of $f$.

			To see Property (iv) holds, note under the hypotheses of its initial case,
			\begin{align*}
				(\Phi \st f)_\pi&(T_1,\ldots, T_n) \\
				&= \sum_{\sigma \leq \pi} \Phi_\sigma(T_1, \ldots, T_n) f(\sigma, \pi)
				\\
				&= \sum_{\sigma \leq \pi} \Phi_{\sigma|_{W}}((T_1,\ldots, T_{\theta-1}, T_\theta L_{\Phi_{\sigma|_{V}}((T_1, \ldots, T_n)|_V)}, T_{\theta + 1}, \ldots, T_n)|_{W})
				f(\sigma|_{V}, \pi|_{V}) f(\sigma|_{W}, \pi|_{W})
				\\
				&= \sum_{\sigma \leq \pi} \Phi_{\sigma|_{W}}((T_1,\ldots, T_{\theta-1}, T_\theta L_{\Phi_{\sigma|_{V}}((T_1, \ldots, T_n)|_V)f(\sigma|_{V}, \pi|_{V})}, T_{\theta + 1}, \ldots, T_n)|_{W})
				f(\sigma|_{W}, \pi|_{W})
				\\
				&= (\Phi \st f)_{\pi|_{W}}((T_1,\ldots, T_{\theta-1}, T_\theta L_{\Phi_{\sigma|_{V}}((T_1, \ldots, T_n)|_V)}, T_{\theta + 1}, \ldots, T_n)|_{W}),
			\end{align*}
			again by the corresponding properties of $\Phi$ and $f$.
			The proof of the remaining three statements in Property (iv) is identical.
		\end{proof}

		\begin{corollary}
			\label{cor:cumulantsarebimultiplicative}
			The operator-valued bi-free cumulant function is bi-multiplicative.
		\end{corollary}


















%%%%%%%%%%%%%%%%%%%%%%%%%%%%%%%%%%%%%%%%%%%%%%%%%%%%%%%%%%%%%%%%%%%%%%%
		\subsection{Bi-moment and bi-cumulant functions.}
		\label{subsec:bimomentandbicumulantfunctions}



		Inspired by \cite{speicher1998combinatorial}*{Section 3.2}, we define the formal classes of bi-moment and bi-cumulant functions and give an important relation between them.
		It follows readily that the operator-valued bi-free moment and cumulant functions on a $\cB$-$\cB$-non-commutative probability space are examples of these types of functions, respectively.



		We begin with the following useful notation.
		\begin{notation}
			Let $\chi : [n] \to \slr$, $\pi \in \BNC(\chi)$, and $q \in [n]$.
			We denote by $\chi|_{\setminus q}$ the restriction of $\chi$ to the set $[n] \setminus \set{q}$.
			In addition, if $q \neq n$ and $\chi(q) = \chi(q+1)$, we define $\pi|_{q = q+1} \in \BNC(\chi|_{\setminus q})$ to be the bi-non-crossing partition which results from identifying $q$ and $q+1$ in $\pi$ (i.e. if $q$ and $q+1$ are in the same block as $\pi$ then $\pi|_{q=q+1}$ is obtained from $\pi$ by just removing $q$ from the block in which $q$ occurs, while if $q$ and $q+1$ are in different blocks, $\pi|_{q=q+1}$ is obtained from $\pi$ by merging the two blocks and then removing $q$).
		\end{notation}

		\begin{definition}
			\label{defnbimomentandbicumulantfunctions}
			Let $(\A, E, \varepsilon)$ be a $\cB$-$\cB$-non-commutative probability space and let
			\[
				\Phi : \bigcup_{n\geq 1} \bigcup_{\chi : [n] \to \slr} \BNC(\chi) \times \A_{\chi(1)} \times \cdots \times \A_{\chi(n)} \to B
			\]
			be bi-multiplicative.
			We say that $\Phi$ is a \emph{bi-moment function} if whenever $\chi : [n] \to \slr$ is such that there exists a $q \in \set{1,\ldots, n-1}$ with $\chi(q) = \chi(q+1)$, then
			\[
				\Phi_{1_\chi}(T_1, \ldots, T_n) = \Phi_{1_{\chi|_{\setminus q}}}(T_1, \ldots, T_{q-1}, T_qT_{q+1}, T_{q+2},\ldots, T_n)
			\]
			for all $T_k \in \A_{\chi(k)}$.
			Similarly, we say that $\Phi$ is a \emph{bi-cumulant function} if whenever $\chi : [n] \to \slr$ and $\pi \in \BNC(\chi)$ are such that there exists a $q \in \set{1,\ldots, n-1}$ with $\chi(q) = \chi(q+1)$, then
			\[
				\Phi_{1_{\chi|_{\setminus q}}}(T_1, \ldots, T_{q-1}, T_qT_{q+1}, T_{q+2},\ldots, T_n) = \Phi_{1_\chi}(T_1, \ldots, T_n) + \sum_{\substack{\pi \in \BNC(\chi)\\ |\pi| =2, q\nsim_\pi q+1}} \Phi_\pi(T_1, \ldots, T_n)
			\]
			for all $T_k \in \A_{\chi(k)}$.
		\end{definition}

		Notice that any operator-valued bi-free moment function $\cE$ is a bi-moment function.

		Before relating the notions of bi-moment and bi-cumulant functions, we note the following alternate formulations.
		\begin{lemma}
			\label{lemequivalentnotionsofbimomentandbicumulant}
			Let $(\A, E, \varepsilon)$ be a $\cB$-$\cB$-non-commutative probability space and let
			\[
				\Phi : \bigcup_{n\geq 1} \bigcup_{\chi : [n] \to \slr} \BNC(\chi) \times \A_{\chi(1)} \times \cdots \times \A_{\chi(n)} \to B
			\]
			be bi-multiplicative.
			Then $\Phi$ is a bi-moment function if and only if whenever $\chi : [n] \to \slr$ and $\pi \in \BNC(\chi)$ are such that there exists a $q \in \set{1,\ldots, n-1}$ with $\chi(q) = \chi(q+1)$ and $q \sim_\pi q+1$, then
			\[
				\Phi_\pi(T_1, \ldots, T_n) = \Phi_{\pi|_{q = q+1}}(T_1, \ldots, T_{q-1}, T_qT_{q+1}, T_{q+2}, \ldots T_n)
			\]
			for all $T_k \in \A_{\chi(k)}$.
			Similarly, $\Phi$ is a bi-cumulant function if and only if whenever $\chi : [n] \to \slr$ is such that there exists a $q \in \set{1,\ldots, n-1}$ with $\chi(q) = \chi(q+1)$, we have
			\[
				\Phi_{\pi}(T_1, \ldots, T_{q-1}, T_qT_{q+1}, T_{q+2}, \ldots T_n) = \sum_{\substack{\sigma \in \BNC(\chi) \\ \sigma|_{q = q+1} = \pi}} \Phi_\sigma(T_1, \ldots, T_n)
			\]
			for all $\pi \in \BNC(\chi|_{\setminus q})$.
		\end{lemma}

		To establish the lemma, one uses bi-multiplicativity to reduce to the case of full partitions and then applies Definition \ref{defnbimomentandbicumulantfunctions}.


		\begin{theorem}
			\label{thmrelationbetweenbimomentandbicumulantfunctions}
			Let $(\A, E, \varepsilon)$ be a $\cB$-$\cB$-non-commutative probability space and let
			\[
				\Phi , \Psi: \bigcup_{n\geq 1} \bigcup_{\chi : [n] \to \slr} \BNC(\chi) \times \A_{\chi(1)} \times \cdots \times \A_{\chi(n)} \to B
			\]
			be related by the formulae
			\[
				\Phi = \Psi \st \zeta_{\BNC}, \qquad \text{or equivalently,} \qquad \Psi = \Phi \st \mu_{\BNC}.
			\]
			Then $\Phi$ is a bi-moment function if and only if $\Psi$ is a bi-cumulant function. 
		\end{theorem}

		\begin{proof}
			To begin, note $\Phi$ is bi-multiplicative if and only if $\Psi$ is bi-multiplicative by Theorem~\ref{thmbimulticonvoledwithmultiisbimult}.


			Suppose $\Psi$ is a bi-cumulant function.
			If $\chi : [n] \to \slr$ is such that there exists a $q \in \set{1,\ldots, n-1}$ with $\chi(q) = \chi(q+1)$, then for all $T_k \in \A_{\chi(k)}$
			\begin{align*}
				\Phi_{1_{\chi|_{\setminus q}}}(T_1,\ldots, T_{q-1}, T_q T_{q+1}, T_{q+2}, \ldots, T_n) &= \sum_{\pi \in \BNC(\chi|_{\setminus q})} \Psi_{\pi}(T_1,\ldots, T_{q-1}, T_q T_{q+1}, T_{q+2}, \ldots, T_n)
				\\
				&=
				\sum_{\pi \in \BNC(\chi|_{\setminus q})} \sum_{\substack{\sigma \in \BNC(\chi) \\ \sigma|_{q = q+1} = \pi}} \Psi_{\sigma}(T_1,\ldots, T_n) \\
				&=
				\sum_{\sigma \in \BNC(\chi)} \Psi_{\sigma}(T_1,\ldots, T_n) \\
				&=
				\Phi_{1_\chi}(T_1,\ldots, T_n).
			\end{align*}
			Thus $\Phi$ is a bi-moment function.

			For the other direction, suppose $\Phi$ is a bi-moment function.
			Let $\chi : [n] \to \slr$.
			We will proceed by induction on $n$.
			If $n = 1$, there is nothing to check.
			If $n = 2$, then
			\[
				\Psi_{1_{\chi|_{1=2}}}(T_1T_2) = \Phi_{1_{\chi|_{1=2}}}(T_1T_2) = \Phi_{1_\chi}(T_1, T_2) = \Psi_{1_\chi}(T_1, T_2) + \Psi_{0_\chi}(T_1, T_2)
			\]
			as required.
			\par 
			Suppose that the formula from Definition \ref{defnbimomentandbicumulantfunctions} holds for $n-1$.
			Then using the induction hypothesis and bi-multiplicativity of $\Psi$, we see for all $\pi \in \BNC(\chi|_{\setminus q}) \setminus \set{1_{\chi|_{\setminus q}}}$ that
			\[
				\Psi_{\pi}(T_1, \ldots, T_{q-1}, T_qT_{q+1}, T_{q+2}, \ldots, T_n) = \sum_{\substack{\sigma \in \BNC(\chi) \\ \sigma|_{q = q+1} = \pi}} \Psi_\sigma(T_1, \ldots, T_n).
			\]
			Hence
			\begin{align*}
				\Psi_{1_{\chi|_{\setminus q}}}&(T_1, \ldots, T_{q-1}, T_qT_{q+1}, T_{q+2}, \ldots, T_n) \\
				&= \Phi_{1_{\chi|_{\setminus q}}}(T_1, \ldots, T_{q-1}, T_qT_{q+1}, T_{q+2}, \ldots, T_n) - \sum_{\substack{\pi \in \BNC(\chi|_{\setminus q}) \\
				\pi \neq 1_{\chi|_{\setminus q}}}}
				\Psi_{\pi}(T_1, \ldots, T_{q-1}, T_qT_{q+1}, T_{q+2}, \ldots T_n) \\
				&= \Phi_{1_{\chi}}(T_1, \ldots, T_n) - \sum_{\substack{\pi \in \BNC(\chi|_{\setminus q}) \\
				\pi \neq 1_{\chi|_{\setminus q}}}} \sum_{\substack{\sigma \in \BNC(\chi) \\ \sigma|_{q = q+1} = \pi}} \Psi_\sigma(T_1, \ldots, T_n)
				\\
				&= \sum_{\sigma\in \BNC(\chi)}\Psi_{\sigma}(T_1, \ldots, T_n) - \sum_{\substack{\sigma \in \BNC(\chi) \\ \sigma|_{q = q+1} \neq 1_{\chi|_{\setminus q}}}} \Psi_\sigma(T_1, \ldots, T_n)
				\\
				&= \sum_{\substack{\sigma \in \BNC(\chi) \\ \sigma|_{q = q+1} = 1_{\chi|_{\setminus q}}}} \Psi_\sigma(T_1, \ldots, T_n). \qedhere
			\end{align*}
		\end{proof}

		\begin{corollary}
			\label{corcumulantfunctionisabicumulantfunction}
			The operator-valued bi-free cumulant function $\kappa$ is a bi-cumulant function.
		\end{corollary}







%%%%%%%%%%%%%%%%%%%%%%%%%%%%%%%%%%%%%%%%%%%%%%%%%%%%%%%%%%%%%%%%%%%%%%%
		\subsection{Vanishing of operator-valued bi-free cumulants.}

		The following demonstrates, like with classical and free cumulants, that operator-valued bi-free cumulants of order at least two vanish provided at least one entry is an element of $\cB$.
		\begin{proposition}
			\label{prop:inputaL_borR_bandyougetzerocumulants}
			Let $(\A, E, \varepsilon)$ be a $\cB$-$\cB$-non-commutative probability space, $\chi : [n] \to \slr$ with $n \geq 2$, and $T_k \in \A_{\chi(k)}$.
			If there exist $q \in [n]$ and $b \in \cB$ such that $T_q = L_b$ if $\chi(q) = \ell$ or $T_q = R_b$ if $\chi(q) = r$, then
			\[
				\kappa(T_1, \ldots, T_n) = 0.
			\]
		\end{proposition}

		\begin{proof}
			We will proceed by induction on $n$.
			The base case can be readily established by computing directly the cumulants of order two.

			For the inductive step, suppose the result holds for all $\chi : \set{1, \ldots, k} \to \slr$ with $k < n$.
			Fix $\chi : [n] \to \slr$ and $T_k \in \A_{\chi(k)}$. 
			Suppose that for some $q \in [n]$ we have $\chi(q) = \ell$ and $T_q = L_b$ with $b \in \cB$, as the argument for the right side is similar.

			Let
			\[
				p = \max\set{k \in [n] \, \mid \, \chi(k) = \ell, k < q}.
			\]
			The proof is now divided into two cases.

			\paragraph{Case 1: $p \neq -\infty$.}
			In this case we notice that
			\begin{align*}
				\kappa_{1_\chi}(T_1, \ldots, T_n)
				&= \cE_{1_\chi}(T_1, \ldots, T_n) - \sum_{\substack{\pi \in \BNC(\chi)\\ \pi \neq 1_\chi}}\kappa_{\pi}(T_1, \ldots, T_n)
				\\
				&= \cE_{1_\chi}(T_1, \ldots, T_n) - \sum_{\substack{\pi \in \BNC(\chi)\\ \set{q} \in \pi}}\kappa_{\pi}(T_1, \ldots, T_{p-1}, T_p, T_{p+1}, \ldots, T_{q-1}, L_b, T_{q+1}, \ldots, T_n)
				\\
				&= \cE_{1_\chi}(T_1, \ldots, T_n) - \sum_{\sigma \in \BNC(\chi|_{\setminus q})}\kappa_{\sigma}(T_1, \ldots, T_{p-1}, T_p L_b, T_{p+1}, \ldots, T_{q-1}, T_{q+1}, \ldots, T_n),
			\end{align*}
			by induction and Proposition \ref{propenhancedproperties}.
			Since
			\begin{align*}
				\cE_{1_\chi}(T_1, \ldots, T_n) &= E(T_1\cdots T_n)
				\\
				&= E(T_1 \cdots T_{q-1} L_b T_{q+1} \cdots T_n) \\
				&= E(T_1 \cdots T_p L_b T_{p+1} \cdots T_{q-1}T_{1+1}\cdots T_n) \\
				&=
				\sum_{\sigma \in \BNC(\chi|_{\setminus q})}\kappa_{\sigma}(T_1, \ldots, T_{p-1}, T_p L_b, T_{p+1}, \ldots, T_{q-1}, T_{q+1}, \ldots, T_n),
			\end{align*}
			the proof is complete in this case.



			\paragraph{Case 2: $p = -\infty$.}
			In this case, notice that
			\begin{align*}
				\kappa_{1_\chi}(T_1, \ldots, T_n)
				&= \cE_{1_\chi}(T_1, \ldots, T_n) - \sum_{\substack{\pi \in \BNC(\chi)\\ \pi \neq 1_\chi}}\kappa_{\pi}(T_1, \ldots, T_n)
				\\
				&= \cE_{1_\chi}(T_1, \ldots, T_n) - \sum_{\substack{\pi \in \BNC(\chi)\\ \set{q} \in \pi}}\kappa_{\pi}(T_1, \ldots, T_{q-1}, L_b, T_{q+1}, \ldots, T_n)
				\\
				&= \cE_{1_\chi}(T_1, \ldots, T_n) - \sum_{\sigma \in \BNC(\chi|_{\setminus q})} b\kappa_{\sigma}(T_1, \ldots, T_{q-1}, T_{q+1}, \ldots, T_n),
			\end{align*}
			by induction and Proposition \ref{propenhancedproperties}.
			Since
			\begin{align*}
				\cE_{1_\chi}(T_1, \ldots, T_n) &= E(T_1\cdots T_n)
				\\
				&= E(T_1 \cdots T_{q-1} L_b T_{q+1} \cdots T_n) \\
				&=
				E(L_bT_1 \cdots T_{q-1} T_{q+1}\cdots T_n) \\
				&=
				bE(T_1 \cdots T_{q-1} T_{q+1}\cdots T_n) \\
				&=
				\sum_{\sigma \in \BNC(\chi|_{\setminus q})} b\kappa_{\sigma}(T_1, \ldots, T_{q-1}, T_{q+1}, \ldots, T_n),
			\end{align*}
			the proof is complete in this case as well.
		\end{proof}














%%%%%%%%%%%%%%%%%%%%%%%%%%%%%%%%%%%%%%%%%%%%%%%%%%%%%%%%%%%%%%%%%%%%
%	   Universal Moment Polynomials for Bi-Free Families with Amalgamation	   %
%%%%%%%%%%%%%%%%%%%%%%%%%%%%%%%%%%%%%%%%%%%%%%%%%%%%%%%%%%%%%%%%%%%%
		\section{Universal moment polynomials for bi-free families with amalgamation.}
		\label{sec:universalmomentpolysforbifreewithamalgamation}

		In this section, we will generalize Corollary~\ref{cor:bifreelats} to demonstrate that algebras will be bi-free with amalgamation over $\cB$ if and only if certain moment expressions hold; our goal is to use the same technology to eventually show that bi-freeness with amalgamation can be characterised by the vanishing of mixed cumulants.
		To begin, we will need to extend the definition of $E_\pi(T_1,\ldots, T_n)$ to certain diagrams in the style of Subsection~\ref{ss:shadedlrdiagrams} of Chapter~\ref{ch:bfi}.

%	Equivalence of Bi-Free with Amalgamation and Universal Polynomials
%%%%%%%%%%%%%%%%%%%%%%%%%%%%%%%%%%%%%%%%%%%%%%%

		\subsection{Bi-freeness with amalgamation through universal moment polynomials.}

		\begin{definition}
			Let $\chi : [n] \to \slr$ and let $\iota : [n] \to \I$.
			Let $LR^\lat_k(\chi, \iota)$ denote the closure of $LR_k(\chi, \iota)$ under lateral refinement.
			Observe that every diagram in $LR^\lat_k(\chi, \iota)$ still has $k$ strings reaching its top, as lateral refinements may only introduce cuts between ribs.
			We denote
			\[
				LR^\lat(\chi, \iota) := \bigcup_{k\geq 0} LR^\lat_k(\chi, \iota).
			\]
		\end{definition}









		\begin{definition}
			\label{defn:ELRrecursiveproof}
			Let $\set{(\cX_\iota, \ocX_\iota, p_\iota)}_{\iota \in \I}$ be $\cB$-$\cB$-bimodules with specified $\cB$-vector states, let $\lambda_\iota$ and $\rho_\iota$ be as defined in Construction \ref{cons:freeproductconstruction}, and let $\cX = \st_{\iota \in \I} \cX_\iota$.
			Let $\chi : [n] \to \slr$, $\iota : [n] \to \I$, $D \in LR^\lat(\chi, \iota)$, and $T_\iota \in \cL_{\chi(k)}(\cX_{\iota(k)})$.
			Define $\mu_k(T_k) = \lambda_{\iota(k)}(T_k)$ if $\chi(k) = \ell$ and $\mu_k(T_k) = \rho_{\iota(k)}(T_k)$ if $\chi(k) = r$.
			We define $E_D(\mu_1(T_1),\ldots, \mu_n(T_n))$ as follows: first, apply the same recursive process as in Definition \ref{defn:recursivedefinitionofEpi} until every block of $\pi$ has a spine reaching the top.
			If every block of $D$ has a spine reaching the top, enumerate them from left to right according to their spines as $V_1,\ldots, V_m$ with $V_j=\set{k_{j,1}<\cdots <k_{j,q_j}}$, and set
			\[
				E_D(\mu_1(T_1),\ldots,\mu_n(T_n))=[(1-p_{\iota(k_{1,1})})T_{k_{1,1}} \cdots T_{k_{1,q_1}} 1_\cB] \otimes \cdots \otimes [(1-p_{\iota(k_{m,1})})T_{k_{m,1}} \cdots T_{k_{m,q_m}} 1_\cB].
			\]
		\end{definition}








		\begin{lemma}
			\label{lem:actingonFreeproductspace}
			With the notation as in Definition \ref{defn:ELRrecursiveproof},
			\[
				\mu_1(T_1) \cdots \mu_n(T_n)1_\cB = \sum_{m=0}^n \sum_{D \in LR^\lat_m(\chi, \iota)} \left[ \sum_{\substack{ D' \in LR_m(\chi, \iota) \\ D' \geq_{\mathrm{lat}} D}} (-1)^{|D| - |D'|}
				\right] E_{D}(\mu_1(T_1),\ldots, \mu_n(T_n)),
			\]
			where $|D|$ and $|D'|$ denote the number of blocks of $D$ and $D'$ respectively.
			In particular,
			\[
				E_{\cL(\cX)}(\mu_1(T_1) \mu_2(T_2) \cdots \mu_n(T_n)) = \sum_{\pi \in \BNC(\chi)} \left[ \sum_{\substack{\sigma\in \BNC(\chi)\\\pi\leq\sigma\leq\e}}\mu_{\BNC}(\pi, \sigma) \right] \cE_{\pi}(\mu_1(T_1),\ldots, \mu_n(T_n)).
			\]
		\end{lemma}





		\begin{proof}
			To begin, note that the second claim follows from the first by Definition~\ref{defn:ELRrecursiveproof} and by Proposition~\ref{prop:lattomob}.
			To prove the main claim we will proceed by induction on the number of operators $n$.
			The case $n = 1$ is immediate.

			For the inductive step, we will assume that $\chi(1) = \ell$ as the proof in the case $\chi(1) = r$ will follow by similar arguments. Let $\chi_0 = \chi|_{\set{2,\ldots, n}}$ and $\iota_0 = \iota|_{\set{2,\ldots, n}}$.
			By induction, we obtain that
			\[
				\mu_2(T_2) \cdots \mu_n(T_n)1_\cB = \sum_{m=0}^{n-1} \sum_{D_0 \in LR^\lat_m(\chi_0, \iota_0)} \left[ \sum_{\substack{ D'_0 \in LR_m(\chi_0, \iota_0) \\ D'_0 \geq_{\mathrm{lat}} D_0}} (-1)^{|D_0| - |D'_0|}
				\right] E_{D_0}(\mu_2(T_2),\ldots, \mu_n(T_n)).
			\]
			The result will follow by applying $\lambda_1(T_1)$ to each $E_{D_0}(\mu_2(T_2),\ldots, \mu_n(T_n))$, checking the correct terms appear, collecting the same terms, and verifying the coefficients are correct.


			Fix $D_0 \in LR^\lat_m(\chi_0, \iota_0)$.
			We can write
			\[
				E_{D_0}(\mu_2(T_2),\ldots, \mu_n(T_n)) = [(1-p_{\iota(k_{1})}) S_1 1_\cB] \otimes \cdots \otimes [(1-p_{\iota(k_{m})}) S_m1_\cB]
			\]
			for some operators $S_p \in \mathrm{alg}(\lambda_{k_p}(\cL_\ell(\cX_{k_p})), \rho_{k_p}(\cL_r(\cX_{k_p})))$.
			To demonstrate the correct terms appear, we divide the analysis into three cases.













			\paragraph{Case 1: $m = 0$.}
			In this case $E_{D_0}(\mu_2(T_2),\ldots, \mu_n(T_n)) = b \in \cB$.
			As such, we see that
			\begin{align*}
				\lambda_{\iota(1)}(T_1)E_{D_0}(\mu_2(T_2),\ldots, \mu_n(T_n)) & = E(T_1)b \oplus [(1-p_{\iota(1)})T_1 b].
			\end{align*}
			If $D_1$ is the $LR$-diagram obtained from $D_0$ by placing a node shaded $\iota(1)$ at the top left and $D_2$ is the $LR$-diagram obtained from $D_0$ by placing a node $\iota(1)$ at the top left and drawing a spine from this node to the top, then since 
			\[
				E(\mu_1(T_1) L_b) = E(\mu_1(T_1) R_b) = E(R_b \mu_1(T_1)) = E(T_1) b
			\]
			and
			\[
				T_1 b = T_1 R_b 1_\cB = T_1 L_b 1_B
			\]
			one easily sees that 
			\[
				E(T_1)b = E_{D_1}(\mu_1(T_1), \mu_2(T_2),\ldots, \mu_n(T_n)) \quad \text{and} \quad (1-p_{\iota(1)})T_1 b = E_{D_2}(\mu_1(T_1), \mu_2(T_2),\ldots, \mu_n(T_n)).
			\]






			\paragraph{Case 2: $m \neq 0$ and $\iota(1) \neq \iota(k_1)$.}
			In this case, $(1-p_{\iota(k_{1})}) S_1 1_\cB$ is in a space orthogonal to $\ocX_{\iota(1)}$.
			Thus
			\begin{align*}
				\lambda_{\iota(1)}(T_1)E_{D_0}(\mu_2(T_2),\ldots, \mu_n(T_n))
				=& \paren{[L_{E(T_1)}(1-p_{\iota(k_{1})}) S_1 1_\cB] \otimes \cdots \otimes [(1-p_{\iota(k_{m})}) S_m1_\cB]} \\
				&
				\oplus \paren{[(1-p_{\iota(1)})T_1 1_\cB]
				\otimes [(1-p_{\iota(k_{1})}) S_1 1_\cB] \otimes \cdots \otimes [(1-p_{\iota(k_{m})}) S_m1_\cB]}.
			\end{align*}
			If $D_1$ is the $LR$-diagram obtained from $D_0$ by placing a node shaded $\iota(1)$ at the top and $D_2$ is the $LR$-diagram obtained from $D_0$ by placing a node $\iota(1)$ at the top and drawing a spine from this node to the top, then since 
			\[
				L_{E(T_1)}(1-p_{\iota(k_{1})}) S_1 1_\cB = (1-p_{\iota(k_{1})}) L_{E(T_1)}S_1 1_\cB,
			\]
			one easily sees that 
			\begin{align*}
				[L_{E(T_1)}(1-p_{\iota(k_{1})}) S_1 1_\cB] \otimes \cdots \otimes [(1-p_{\iota(k_{m})}) S_m1_\cB] = E_{D_1}(\mu_1(T_1), \mu_2(T_2),\ldots, \mu_n(T_n)) \\
				[(1-p_{\iota(1)})T_1 1_\cB]
				\otimes [(1-p_{\iota(k_{1})}) S_1 1_\cB] \otimes \cdots \otimes [(1-p_{\iota(k_{m})}) S_m1_\cB] = E_{D_2}(\mu_1(T_1), \mu_2(T_2),\ldots, \mu_n(T_n)).
			\end{align*}








			\paragraph{Case 3: $m \neq 0$ and $\iota(1) = \iota(k_1)$.}
			In this case, there is a spine in $D$ that reaches the top and is the same shading as $T_1$.
			Thus $(1-p_{\iota(k_{1})}) S_1 1_\cB \in\ocX_{\iota(1)}$, so
			\begin{align*}
				\lambda_{\iota(1)}(T_1)E_{D_0}(\mu_2(T_2),\ldots, \mu_n(T_n))
				=& \paren{[L_{p_{\iota(1)}\paren{T_1(1-p_{\iota(1)}) S_1 1_\cB}}(1-p_{\iota(k_{2})}) S_2 1_\cB] \otimes \cdots \otimes [(1-p_{\iota(k_{m})}) S_m1_\cB]} \\
				&
				\oplus \paren{[(1-p_{\iota(1)})T_1(1-p_{\iota(1)}) S_1 1_\cB] \otimes \cdots \otimes [(1-p_{\iota(k_{m})}) S_m1_\cB]}.
			\end{align*}
			Expanding $T_1(1-p_{\iota(1)}) S_11_\cB = T_1S_11_\cB - T_1E(S_1)$, the above becomes
			\begin{align*}
				\lambda_{\iota(1)}(T_1)E_{D_0}(\mu_2(T_2),\ldots, \mu_n(T_n))
				=& \paren{[L_{E(T_1S_1)}(1-p_{\iota(k_{2})}) S_2 1_\cB] \otimes \cdots \otimes [(1-p_{\iota(k_{m})}) S_m1_\cB]} \\
				& \oplus (-1) \paren{[L_{p_{\iota(1)}\paren{T_1E(S_1)}}(1-p_{\iota(k_{2})}) S_2 1_\cB] \otimes \cdots \otimes [(1-p_{\iota(k_{m})}) S_m1_\cB]} \\
				&
				\oplus \paren{[(1-p_{\iota(1)})T_1S_1 1_\cB] \otimes \cdots \otimes [(1-p_{\iota(k_{m})}) S_m1_\cB]}\\
				&
				\oplus (-1)\paren{[(1-p_{\iota(1)})T_1E(S_1)] \otimes \cdots \otimes [(1-p_{\iota(k_{m})}) S_m1_\cB]}.
			\end{align*}
			Let $D_1$ be the $LR$-diagram obtained from $D_0$ by placing a node shaded $\iota(1)$ at the top and terminating the left-most spine at this node, $D_2$ be the $LR$-diagram obtained by laterally refining $D_1$ by cutting the spine attached to the top node directly beneath the top node, $D_3$ be the $LR$-diagram obtained from $D_0$ by placing a node shaded $\iota(1)$ at the top and connecting this node to the left-most spine, and $D_4$ be the $LR$-diagram obtained by laterally refining $D_3$ by cutting the spine attached to the top node directly beneath the top node.
			As in the previous case, we see (by applying Lemma \ref{lemproductofdisjointblocksforE} if $m = 1$) that
			\[
				[L_{E(T_1S_1)}(1-p_{\iota(k_{2})}) S_2 1_\cB] \otimes \cdots \otimes [(1-p_{\iota(k_{m})}) S_m1_\cB] = E_{D_1}(\mu_1(T_1), \mu_2(T_2),\ldots, \mu_n(T_n))
			\]
			and 
			\[
				[(1-p_{\iota(1)})T_1S_1 1_\cB] \otimes \cdots \otimes [(1-p_{\iota(k_{m})}) S_m1_\cB] = E_{D_3}(\mu_1(T_1), \mu_2(T_2),\ldots, \mu_n(T_n)).
			\]

			We will demonstrate that
			\[
				[L_{p_{\iota(1)}\paren{T_1E(S_1)}}(1-p_{\iota(k_{2})}) S_2 1_\cB] \otimes \cdots \otimes [(1-p_{\iota(k_{m})}) S_m1_\cB] = E_{D_2}(\mu_1(T_1), \mu_2(T_2),\ldots, \mu_n(T_n))
			\]
			and forgo the similar proof that
			\[
				[(1-p_{\iota(1)})T_1E(S_1)] \otimes \cdots \otimes [(1-p_{\iota(k_{m})}) S_m1_\cB] = E_{D_4}(\mu_1(T_1), \mu_2(T_2),\ldots, \mu_n(T_n)).
			\]
			Notice that 
			\[
				L_{p_{\iota(1)}\paren{T_1E(S_1)}} = L_{p_{\iota(1)}\paren{T_1R_{E(S_1)}1_\cB}} = L_{p_{\iota(1)}\paren{T_11_\cB} E(S_1)} = L_{p_{\iota(1)}\paren{T_11_\cB}}L_{E(S_1)},
			\]
			and so
			\[
				L_{p_{\iota(1)}\paren{T_1E(S_1)}}(1-p_{\iota(k_{2})}) S_2 1_\cB = (1-p_{\iota(k_{2})}) L_{p_{\iota(1)}\paren{T_11_\cB}}L_{E(S_1)}S_2 1_\cB.
			\]
			Thus, unless $m = 1$, $L_{p_{\iota(1)}\paren{T_11_\cB}}$ appears as it should in the definition of $E_{D_2}$ although the $E(S_1)$ term may not be as it should.
			To obtain the desired result, we make the following corrections.

			Recall that $S_1$ corresponds to the left-most spine of $D_0$ reaching the top.
			Let $W \subseteq \set{2,\ldots, n}$ be the set of $k$ for which $T_k$ appears in the expression for $S_1$.
			Note that $S_1$ will be of the form $C_{E(S'_1)} C_{E(S'_2)} \cdots C_{E(S'_{p-1})} S'_p$, where each $E(S'_k)$ is the moment of a disjoint $\chi$-interval $W_k$ composed of blocks of $D_2$ with the property that $\min_{\prec_\chi}(W_k)$ and $\max_{\prec_\chi}(W_k)$ lie in the same block ($C$ denotes either $L$ or $R$, as appropriate).
			Observe that $W = \bigcup^p_{k=1} W_k$.
			Therefore, by Proposition \ref{prop:propertiesofEforLX}
			\[
				E(S_1) = E(C_{E(S'_1)} C_{E(S'_2)} \cdots C_{E(S'_{p-1})} S'_p)= E(S''_1) \cdots E(S''_p),
			\]
			where the $S''_1,\ldots, S''_p$ are $S'_1,\ldots, S'_p$ up to reordering by $\prec_\chi$. Let $W'_1,\ldots, W'_k$ be $W_1,\ldots, W_k$ under this new ordering. 

			Through Lemma \ref{lemproductofdisjointblocksforE} in the case $m=1$ and computations similar to the reverse of those used in cases 1 and 2 of Lemma \ref{lemreductionofbimultiplicativeinsideablockforE} one can verify these terms can be moved into the correct positions.



			Finally, it is clear that the coefficients of each $E_D(\mu_1(T_1),\ldots, \mu_n(T_n))$ are correct for each $D \in LR^\lat(\chi, \iota)$.
			Alternatively, one can check the coefficients in the second claim by noting that the coefficients did not depend on the algebra $\cB$, setting $\cB=\C$, and using Corollary~\ref{cor:bifreemob}.
		\end{proof}

		\begin{theorem}
			\label{thm:bifreeequivalenttouniversalpolys}
			Let $(\A, E_\A, \varepsilon)$ be a $\cB$-$\cB$-non-commutative probability space and let $\fpf$ be a family of pairs of $\cB$-faces of $\A$.
			Then $\fpf$ are bi-free with amalgamation over $\cB$ if and only if for all $\chi : [n] \to \slr$, $\iota : [n] \to \I$, and $T_k \in \A_{\chi(k)}^{(\iota(k))}$, the following formula holds:
			\[
				E_{\A}(T_1 \cdots T_n) = \sum_{\pi \in \BNC(\chi)} \sq{ \sum_{\substack{\sigma\in \BNC(\chi)\\\pi\leq\sigma\leq\e}}\mu_{\BNC}(\pi, \sigma) } \cE_{\pi}(T_1,\ldots, T_n).
			\]
		\end{theorem}

		\begin{proof}
			If $\fpf$ are bi-free with amalgamation over $\cB$, then Lemma~\ref{lem:actingonFreeproductspace} implies the desired formula holds.

			Conversely, suppose that the formula holds.
			By Theorem \ref{thm:representingbbncps} there exists a $\cB$-$\cB$-bimodule with a specified $\cB$-vector state $(\cX, \ocX, p)$ and a unital homomorphism $\theta : \A \to \cL(\cX)$ such that 
			\[
				\theta(\varepsilon(b_1 \otimes b_2)) = L_{b_1} R_{b_2}, \quad \theta(\A_\ell) \subseteq \cL_\ell(\cX),\quad \theta(\A_r) \subseteq \cL_r(\cX), \quad \mathrm{and}\quad E_{\cL(\cX)}(\theta(T)) = E_\A(T)
			\]
			for all $b_1, b_2 \in \cB$ and $T \in \A$.
			For each $\iota \in \I$, let $(\cX_\iota, \ocX_\iota, p_\iota)$ be a copy of $(\cX, \ocX, p)$ and $l_\iota$ and $r_\iota$ be copies of $\theta\colon \A\to \cL(\cX_\iota)$.
			Since the formula holds for $E_{\cL(\cX)}\circ\theta$ as well by Lemma~\ref{lem:actingonFreeproductspace}, $\fpf$ are bi-free over $\cB$.
		\end{proof}

%%%%%%%%%%%%%%%%%%%%%%%%%%%%%%%%%%%%%%%%%%%%%%%%%%%%%%%%%%%%%%%%%%%%
%	     Additivity of Operator-Valued Bi-Free Cumulants        	   %
%%%%%%%%%%%%%%%%%%%%%%%%%%%%%%%%%%%%%%%%%%%%%%%%%%%%%%%%%%%%%%%%%%%%

		\section{Additivity of operator-valued bi-free cumulants.}
		\label{sec:additivity}

%%%%%%%%%%%%%%%%%%%%%%%%%%%%%%%%%%%%%%%%%%%%%%%%%%%%%%%%%%%%%%%%%%%%%%%
		\subsection{Equivalence of bi-freeness with vanishing of mixed cumulants.}
		We now state the operator-valued analogue of the main result of Section~\ref{sec:unibifree} from Chapter~\ref{ch:bfi}, namely, that bi-freeness of families of pairs of $\cB$-faces is equivalent to the vanishing of their mixed cumulants.

		\begin{theorem}
			\label{thmequivalenceofbifreeandcombintoriallybifree}
			Let $(\A, E, \varepsilon)$ be a $\cB$-$\cB$-non-commutative probability space and let $\fpf$ be a family of pairs of $\cB$-faces from $\A$.
			Then $\fpf$ are bi-free with amalgamation over $\cB$ if and only if for all $\chi : [n] \to \slr$, $\iota : [n] \to \I$ non-constant, and $T_k \in \A_{\chi(k)}^{(\iota(k))}$, we have
			\[\kappa_{1_\chi}(T_1, \ldots, T_n) = 0.\]
		\end{theorem}

		\begin{proof}
			Suppose $\fpf$ are bi-free over $\cB$.
			Fix a shading $\iota : [n] \to \I$ and let $\chi : [n] \to \slr$.
			If $T_1, \ldots, T_n$ are operators as above, Theorem \ref{thm:bifreeequivalenttouniversalpolys} implies
			\[
				\cE_{1_\chi}\paren{ T_1, \ldots, T_n}
				= \sum_{{\pi \in \BNC(\alpha)}} \left[
				\sum_{\substack{\sigma \in \BNC(\chi) \\ \pi \leq \sigma \leq \iota}} \mu_{\BNC}(\pi, \sigma) \right] \cE_{\pi}\paren{ T_1, \ldots, T_n}.
			\]
			Therefore
			\[
				\cE_{1_\chi}\paren{ T_1, \ldots, T_n } = \sum_{\substack{\sigma \in \BNC(\chi) \\ \sigma \leq \iota}} \kappa_\sigma\paren{T_1, \ldots, T_n}
			\]
			by Definition \ref{def:kappa}.
			Using the above formula, we will proceed inductively to show that $\kappa_\sigma\paren{T_1, \ldots, T_n} = 0$ if $\sigma \in \BNC(\chi)$ and $\sigma \nleq \iota$.
			The base case where $n = 1$ holds vacuously.



			For the inductive case, suppose the result holds for every $q < n$.
			Suppose $\iota$ is not constant and note $1_\chi \nleq \iota$.
			Then
			\[
				\sum_{\sigma \in \BNC(\chi)} \kappa_\sigma\paren{ T_1, \ldots, T_n }
				= \cE_{1_\chi}\paren{ T_1, \ldots, T_n }
				= \sum_{\substack{\sigma \in \BNC(\chi) \\ \sigma \leq \iota}} \kappa_\sigma\paren{T_1, \ldots, T_n}.
			\]
			On the other hand, by induction and the bi-multiplicativity of $\kappa$, $\kappa_\sigma\paren{T_1, \ldots, T_n} = 0$ provided $\sigma \in \BNC(\chi)\setminus \set{1_\chi}$ and $\sigma \nleq \iota$.
			Consequently,
			\[
				\sum_{\sigma \in \BNC(\chi)} \kappa_\sigma\paren{ T_1, \ldots, T_n } = \kappa_{1_\chi}\paren{ T_1, \ldots, T_n } + \sum_{\substack{\sigma \in \BNC(\chi) \\ \sigma \leq \iota}} \kappa_\sigma\paren{ T_1, \ldots, T_n }.
			\]
			Combining these two equations gives $\kappa_{1_\chi}\paren{ T_1, \ldots, T_n } = 0$, completing the inductive step.

			Conversely, suppose all mixed cumulants vanish.
			Then we have
			\begin{align*}
				\cE_{1_\chi}\paren{ T_1, \ldots, T_n }
				&= \sum_{\sigma \in \BNC(\chi)} \kappa_\sigma\paren{ T_1, \ldots, T_n }\\
				&=
				\sum_{\substack{\sigma \in \BNC(\chi) \\ \sigma \leq \iota}} \kappa_\sigma\paren{ T_1, \ldots, T_n } \\
				&=
				\sum_{\substack{\sigma \in \BNC(\chi) \\ \sigma \leq \iota}} \sum_{\substack{\pi \in \BNC(\chi)\\ \pi \leq \sigma}}\cE_\pi\paren{ T_1, \ldots, T_n } \mu_{\BNC}(\pi, \sigma) \\
				&=
				\sum_{{\pi\in \BNC(\chi) }} \left[\sum_{\substack{\sigma \in \BNC(\chi)\\ \pi \leq \sigma \leq \iota}} \mu_{\BNC}(\pi, \sigma) \right] \cE_\pi\paren{ T_1, \ldots, T_n }.
			\end{align*}
			Hence Theorem \ref{thm:bifreeequivalenttouniversalpolys} implies $\fpf$ are bi-free over $\cB$.
		\end{proof}



%%%%%%%%%%%%%%%%%%%%%%%%%%%%%%%%%%%%%%%%%%%%%%%%%%%%%%%%%%%%%%%%%%%%%%%
		\subsection{Moment and cumulant series.}

		In this section, we will begin the study of pairs of $\cB$-faces generated by operators.

		Let $(\A, E, \varepsilon)$ be a $\cB$-$\cB$-non-commutative probability space and let $(C, D)$ be a pair of $\cB$-faces such that
		\[
		C = \mathrm{alg}(\set{L_b \, \mid \, b \in \cB} \cup \set{z_i}_{i \in I}\})
		\qquad
		\text{and}
		\qquad
	D = \mathrm{alg}(\set{R_b \, \mid \, b \in \cB} \cup \set{z_j}_{j \in J}\}).
\]
We desire to compute the joint distribution of $(C, D)$ under $E$.
To do so, it suffices to compute
\[
	E((C_{b_1} z_{\alpha(1)} C_{b'_1}) \cdots (C_{b_n} z_{\alpha(n)} C_{b'_n})) = \cE_{1_{\chi_\alpha}}(C_{b_1'} z_{\alpha(1)} C_{b_1}, \ldots, C_{b_n'} z_{\alpha(n)} C_{b_n})
\]
where $\alpha : [n] \to I \sqcup J$, $b_1, b'_1, \ldots, b_n, b'_n \in \cB$, and $C_{b_k} = L_{b_k}$, $C_{b'_k} = L_{b'_k}$ if $\alpha(k) \in I$ and $C_{b_k} = R_{b_k}$, $C_{b'_k} = R_{b'_k}$ if $\alpha(k) \in J$.
However, because $\cE$ is bi-multiplicative it suffices to treat the case where $b_k' = 1$ and so $C_{b_k'} = I$; we may even assume $b_n = 1$.
Similarly, to compute all possible cumulants, it suffices to compute
\[
	\kappa_{1_{\chi_\alpha}}(z_{\alpha(1)} C_{b_1},
		z_{\alpha(2)} C_{b_2}, \ldots,
		z_{\alpha(n-1)} C_{b_{n-1}},
	z_{\alpha(n)}).
\]
As such we make the following definition.

\begin{definition}
	\label{defnmomentandcumulantseries}
	Let $(\A, E, \varepsilon)$ be a $\cB$-$\cB$-non-commutative probability space and let $(C, D)$ be a pair of $\cB$-faces such that
	\[
	C = \mathrm{alg}(\set{L_b \, \mid \, b \in \cB} \cup \set{z_i}_{i \in I}\})
	\qquad
	\text{and}
	\qquad
D = \mathrm{alg}(\set{R_b \, \mid \, b \in \cB} \cup \set{z_j}_{j \in J}\}).
		  \]
		  The \emph{moment series} of $z = ((z_i)_{i \in I}, (z_j)_{j \in J})$ is the collection of maps
		  \[
			  \set{\mu^z_\alpha : \cB^{n-1} \to \cB \, \mid \, n \in \N, \alpha :[n] \to I \sqcup J}
		  \]
		  given by
		  \[
			  \mu^z_\alpha(b_1, \ldots, b_{n-1}) = \cE_{1_{\chi_\alpha}}(z_{\alpha(1)} C_{b_1},
				  z_{\alpha(2)} C_{b_2}, \ldots,
				  z_{\alpha(n-1)} C_{b_{n-1}},
			  z_{\alpha(n)}),
		  \]
		  where $C_{b_k} = L_{b_k}$ if $\alpha(k) \in I$ and $C_{b_k} = R_{b_k}$ otherwise.

		  Similarly, the \emph{cumulant series} of $z$ is the collection of maps
		  \[
			  \set{\kappa^z_\alpha : \cB^{n-1} \to \cB \, \mid \, n \in \N, \alpha :[n] \to I \sqcup J}
		  \]
		  given by
		  \[
			  \kappa^z_\alpha(b_1, \ldots, b_{n-1}) = \kappa_{1_{\chi_\alpha}}(z_{\alpha(1)} C_{b_1},
				  z_{\alpha(2)} C_{b_2}, \ldots,
				  z_{\alpha(n-1)} C_{b_{n-1}},
			  z_{\alpha(n)}).
		  \]

		  Note that if $n = 1$, we have $\mu_\alpha^z = E(z_{\alpha(1)}) = \kappa_\alpha^z$.
	  \end{definition}







	  \begin{proposition}
		  Let $(\A, E)$ be a $\cB$-$\cB$-non-commutative probability space, and for $\iota\in \set{',''}$ let $\set{z_i^\iota}_{i\in I}\subset \A_\ell$ and $\set{z_j^\iota}_{j\in J}\subset \A_r$. If 
		  \[
			  C^\iota = \alg\paren{\set{L_b : b \in \cB} \cup \set{z_i^\iota}_{i\in I}}\qquad\text{and}\qquad
			  D^\iota = \alg\paren{\set{R_b : b \in \cB} \cup \set{z_j^\iota}_{j\in J}}
		  \]
		  are such that $(C', D')$ and $(C'', D'')$ are bi-free, then
		  \[
			  \kappa_\alpha^{z'+z''} = \kappa_\alpha^{z'}+\kappa_\alpha^{z''}.
		  \]
	  \end{proposition}

	  \begin{proof}
		  This follows directly from Theorem \ref{thmequivalenceofbifreeandcombintoriallybifree}.
	  \end{proof}




















%%%%%%%%%%%%%%%%%%%%%%%%%%%%%%%%%%%%%%%%%%%%%%%%%%%%%%%%%%%%%%%%%%%%
%	    Multiplicative Convolution for Families of Pairs of $\cB$-Faces   	   %
%%%%%%%%%%%%%%%%%%%%%%%%%%%%%%%%%%%%%%%%%%%%%%%%%%%%%%%%%%%%%%%%%%%%



% \section{Multiplicative convolution for families of pairs of $\cB$-faces.}
% \label{sec:MultiplicativeConvolution}
% 
% In this section, we will demonstrate how operator-valued bi-free cumulants involving products of operators may be computed.
% The main theorem of this section, Theorem \ref{thm:advancedcumulantreduction}, also gives rise to an extension of Theorem~\ref{thm:multkrewer} in the case $\cB = \C$.
% 
% 
% %%%%%%%%%%%%%%%%%%%%%%%%%%%%%%%%%%%%%%%%%%%%%%%%%%%%%%%%%%%%%%%%%%%%%%%
% \subsection{Operator-valued bi-free cumulants of products.}
% 
% \begin{notation}
% 	\label{nota:expandingnodesonthesameside}
% 	Let $m,n \in \N$ with $m < n$, and fix a sequence of integers 
% 	\[
% 		k(0) = 0 < k(1) < \cdots < k(m) = n
% 	\]
% 	For $\chi : \set{1,\ldots, m} \to \slr$, we define $\widehat{\chi} : [n] \to \slr$ via
% 	\[
% 		\widehat{\chi}(q) = \chi(p_q)
% 	\]
% 	where $p_q$ is the unique element of $\set{1,\ldots, m}$ such that $k(p_q-1) < q \leq k(p_q)$.
% 
% \end{notation}
% 
% There exists an embedding of $\BNC(\chi)$ into $\BNC(\widehat{\chi})$ via $\pi \mapsto \widehat{\pi}$ where the $p^{\mathrm{th}}$ node of $\pi$ is replaced by the block $(k(p-1)+1, \ldots, k(p))$.
% Observe that all of these nodes appear on the side $p$ was on originally.
% Alternatively, this map can be viewed as an analogue of the map on non-crossing partitions from \cite{nica2006lectures}*{Notation 11.9} after applying $s^{-1}_\chi$.
% 
% It is easy to see that $\widehat{1_\chi} = 1_{\widehat{\chi}}$, $\widehat{0_\chi}$ is the partition with blocks $\set{(k(p-1)+1, \ldots, k(p))
% }_{p=1}^m$, and $\pi \mapsto \widehat{\pi}$ is injective and preserves the partial ordering on $\BNC$.
% Furthermore the image of $\BNC(\chi)$ under this map is
% \[
% 	\widehat{\BNC}(\chi) = \left[\widehat{0_\chi}, \widehat{1_\chi}\right] = \left[\widehat{0_\chi}, 1_{\widehat{\chi}}\right] \subseteq \BNC(\widehat{\chi}).
% \]
% Finally, since the lattice structure is preserved by this map, we see that $\mu_{\BNC}(\sigma, \pi) = \mu_{\BNC}(\widehat{\sigma}, \widehat{\pi})$.
% 
% 
% 
% Recall that since $\mu_{\BNC}$ is the M\"obius function on the lattice of bi-non-crossing partitions, we have for each $\sigma,\pi \in \BNC(\chi)$ with $\sigma \leq \pi$ that
% \[
% 	\sum_{\substack{ \tau \in \BNC(\chi) \\ \sigma \leq \tau \leq \pi
% }} \mu_{\BNC}(\tau, \pi) =
% \left\{
% 	\begin{array}{ll}
% 		1 & \text{if } \sigma = \pi
% 		\\
% 			0 & \text{otherwise }
% 	\end{array} \right. .
% \]
% Therefore, it is easy to see that the partial M\"{o}bius inversion from \cite{nica2006lectures}*{Proposition 10.11} holds in our setting: that is, if $f, g : \BNC(\chi) \to \cB$ are such that 
% \[
% 	f(\pi) = \sum_{\substack{\sigma \in \BNC(\chi) \\ \sigma \leq \pi}} g(\sigma)
% \]
% for all $\pi \in \BNC(\chi)$, then for all $\pi, \sigma \in \BNC(\chi)$ with $\sigma \leq \pi$, we have the relation
% \[
% 	\sum_{\substack{\tau \in \BNC(\chi) \\ \sigma \leq \tau \leq \pi }} f(\tau) \mu_{\BNC}(\tau, \pi) = \sum_{\substack{\omega \in \BNC(\chi) \\ \omega \vee \sigma = \pi }} g(\omega).
% \]
% 
% 
% 
% We now describe the operator-valued bi-free cumulants involving products of operators in terms of the above notation, following the spirit of \cite{nica2006lectures}*{Theorem 11.12}.
% \begin{theorem}
% 	\label{thm:advancedcumulantreduction}
% 	Let $(\A, E, \varepsilon)$ be a $\cB$-$\cB$-non-commutative probability space, $m, n \in \N$ with $m < n$, $\chi : [m] \to \slr$, and
% 	\[
% 		k(0) = 0 < k(1) < \cdots < k(m) = n.
% 	\]
% 	If $\pi \in \BNC(\chi)$ and $T_k \in \A_{\widehat{\chi}(k)}$ for all $k \in [n]$, then
% 	\[
% 		\kappa_\pi\paren{T_1 \cdots T_{k(1)}, T_{k(1)+1} \cdots T_{k(2)}, \ldots, T_{k(m-1)+1} \cdots T_{k(m)} } =
% 	\sum_{\substack{\sigma \in \BNC(\widehat{\chi})\\ \sigma \vee \widehat{0_\chi} = \widehat{\pi}}} \kappa_\sigma(T_1, \ldots, T_n).
% 	\]
% 	In particular, for $\pi = 1_\chi$, we have
% 	\[
% 		\kappa_{1_\chi}\paren{T_1 \cdots T_{k(1)}, T_{k(1)+1} \cdots T_{k(2)}, \ldots, T_{k(m-1)+1} \cdots T_{k(m)}} = \sum_{\substack{\sigma \in \BNC(\widehat{\chi})\\ \sigma \vee \widehat{0_\chi} = 1_{\widehat{\chi}}}} \kappa_\sigma(T_1, \ldots, T_n).
% 	\]
% \end{theorem}
% 
% \begin{proof}
% 	For $j \in \set{1,\ldots, m}$, let $S_j = T_{k(j-1)+1} \cdots T_{k(j)}$.
% 	Then, by Definition \ref{defn:recursivedefinitionofEpi},
% 	\begin{align*}
% 		\kappa_\pi(S_1, \ldots, S_m) &= \sum_{\substack{\tau \in \BNC(\chi) \\ \tau \leq \pi}} \cE_\tau(S_1, \ldots, S_m) \mu_{\BNC}(\tau, \pi) \\
% 		& = \sum_{\substack{\tau \in \BNC(\chi)\\
% 		\tau \leq \pi}} \cE_{\widehat{\tau}}(T_1, \ldots, T_n) \mu_{\BNC}(\widehat{\tau}, \widehat{\pi}) \\
% 		& = \sum_{\substack{\sigma \in \BNC(\widehat{\chi}) \\ \widehat{0_\chi} \leq \sigma \leq \widehat{\pi}}} \cE_{\sigma}(T_1, \ldots, T_n) \mu_{\BNC}(\sigma, \widehat{\pi}) \\
% 		& = \sum_{\substack{\sigma \in \BNC(\widehat{\chi})\\ \sigma \vee \widehat{0_\chi} = \widehat{\pi}}} \kappa_\sigma(T_1, \ldots, T_n),
% 	\end{align*}
% 	with the last line following from our comments before the statement of the theorem.
% \end{proof}
% 
% 
% 
% %%%%%%%%%%%%%%%%%%%%%%%%%%%%%%%%%%%%%%%%%%%%%%%%%%%%%%%%%%%%%%%%%%%%%%%
% \subsection{Multiplicative convolution of bi-free two-faced families.}
% 
% Recall the definition of the Kreweras complement $K_{\BNC}$ from Section~\ref{sec:bimultconv} of Chapter~\ref{ch:bfi}.
% Using Theorem \ref{thm:advancedcumulantreduction}, we can extend Theorem~\ref{thm:multkrewer} as follows, using ideas from the proof of \cite{nica2006lectures}*{Theorem 14.4}.
% 
% 
% \begin{proposition}
% 	\label{prop:multiplcative-convolution}
% 	Let $(\A, \varphi)$ be a non-commutative probability space.
% 	Let $z'=( (z'_i)_{i \in I}, (z'_j)_{j \in J})$ and $z''=( (z''_i)_{i \in I}, (z''_j)_{j \in J})$ be bi-free two-faced families in $\A$, and set $z_i = z'_iz''_i$, $z_j = z''_jz'_j$ for $i \in I$ and $j \in J$.
% 	Then for $\alpha : [n] \to I \sqcup J$, we have
% 	\[
% 		\kappa_{\chi_\alpha}(z_{\alpha(1)}, \ldots, z_{\alpha(n)}) = \sum_{\pi \in \BNC(\chi_\alpha)} \kappa_\pi(z'_{\alpha(1)}, \ldots, z'_{\alpha(n)}) \cdot \kappa_{K_{\BNC}(\pi)}(z''_{\alpha(1)}, \ldots, z''_{\alpha(n)}).
% 	\]
% \end{proposition}
% 
% \begin{proof}
% 	Define $\widehat{\alpha} : \set{1,\ldots, 2n} \to I \sqcup J$ by $\widehat{\alpha}(2k-1) = \widehat{\alpha}(2k) = \alpha(k)$ for $k \in [n]$, and define $\iota : \set{1,\ldots, 2n} \to \set{', ''}$ by
% 	\[
% 		\iota(2k-1)
% 	= \left\{
% 		\begin{array}{ll}
% 			' & \text{if } \alpha(k) \in I
% 			\\
% 				'' & \text{if } \alpha(k) \in J
% 		\end{array} \right.
% 	\qquad \text{ and }\qquad \iota(2k) = 
% 	\left\{
% 		\begin{array}{ll}
% 			'' & \text{if } \alpha(k) \in I
% 			\\
% 			' & \text{if } \alpha(k) \in J
% 		\end{array} \right..
% 	\]
% 	Using Theorem \ref{thm:advancedcumulantreduction}, we easily obtain
% 	\[
% 		\kappa_\chi(z_{\alpha(1)}, \ldots, z_{\alpha(n)}) = \sum_{\substack{\pi \in \BNC(\chi_{\widehat{\alpha}}) \\ \pi \vee \sigma = 1_{\chi_{\widehat{\alpha}}}}} \kappa_\pi\paren{z^{\iota(1)}_{\alpha(1)}, z^{\iota(2)}_{\alpha(1)}, \ldots, z^{\iota(2n-1)}_{\alpha(n)}, z^{\iota(2n)}_{\alpha(n)}}
% 	\]
% 	where $\sigma = \set{(1,2), (3,4), \ldots, (2n-1, 2_n)}$.
% 	Since $z'$ and $z''$ are bi-free, Theorem \ref{thmequivalenceofbifreeandcombintoriallybifree} (or simply Theorem~\ref{thm:biequiv}) implies mixed bi-free cumulants vanish and thus only $\pi$ of the form $\pi = \pi' \cup \pi''$ with $\pi' \in \BNC\paren{ \chi_{\widehat{\alpha}}|_{\set{k \, \mid \, \iota(k) = '}}}$ and $\pi'' \in \BNC\paren{ \chi_{\widehat{\alpha}}|_{\set{k \, \mid \, \iota(k) = ''}}}$ will provide a non-zero contribution.
% 	However, for an arbitrary $\pi' \in \BNC\paren{ \chi_{\widehat{\alpha}}|_{\set{k \, \mid \, \iota(k) = '}}}$, it is elementary to see (for example, by the relations between the Kreweras complements for bi-non-crossing and non-crossing partitions) that there exists a unique $\pi'' \in \BNC\paren{ \chi_{\widehat{\alpha}}|_{\set{k \, \mid \, \iota(k) = ''}}}$ such that $(\pi' \cup \pi'') \vee \sigma = 1_{\chi_{\widehat{\alpha}}}$; namely, $\pi'' = K_{\BNC}(\pi')$.
% 	Therefore, since we are in the scalar setting and $\kappa_\tau(T_1,\ldots, T_n) = \kappa_{\tau|_{V}}((T_1,\ldots, T_n)|_V) \kappa_{\tau|_{V^c}}((T_1,\ldots, T_n)|_{V^c})$ whenever $\tau \in \BNC(\chi')$ and $V$ is a block of $\pi$, we obtain the desired equation.
% \end{proof}
% 






	  \section{Amalgamated versions of results from Chapter~\ref{ch:bfi}.}
	  In this section we remark that several of our results from Chapter~\ref{ch:bfi} have immediate or almost-immediate extensions to the operator-valued setting.
	  We will eschew many of the details as they are for the most part restatements of the proofs in the scalar-valued case with much more tedious bookkeeping.

	  \subsection{Amalgamated vaccine.}
	  We first turn our attention to the vaccine condition, which extends to the bi-free setting by replacing every instance of $\varphi$ with $\cE$.
	  Lemma~\ref{lem:bifreeimpliesvaccine} (that bi-free families exhibit vaccine) still holds in this situation, its proof being entirely combinatorial.
	  One must take some care when reducing cumulants into products based on their blocks since $\cB$ is potentially non-commutative; however, the key point of the argument is that if an interval is not connected to any nodes outside of it, its moment is multiplied into the product, and this still holds.
	  All the terms corresponding to the isolated interval can be collected in one place in the correct order, and then replaced by one of $0$, $L_0$, or $R_0$.

% We are not aware of a proof of an analogue of Lemma~\ref{lem:vaccineimpliesbifree} (that vaccine suffices for bi-free independence), however, essentially because $\cB$ is very likely to not be algebraically closed.
% Thus we do not know that we can necessarily apply the same trick to reduce any moment to a polynomial of moments of smaller order.
% We do arrive at the following much weaker statement:
% \begin{lemma}
% 	\label{lem:amalgavaccineimpliesbifree}
% 	Let $\fpf$ be a family of pairs of $\cB$-faces in a $\cB$-$\cB$-non-commutative probability space $(\A, E, \varepsilon)$.
% 	Suppose further that for each single $\iota \in \I$ and every $\chi : [n] \to \slr$ and $z_i \in \A^{(\iota)}_{\chi(i)}$, there are $b_i \in \varepsilon(\cB\otimes\cB)$ so that
% 	$$E\paren{(z_1 - b_1) \cdots (z_n - b_n)} = 0.$$
% 	Then $\fpf$ are bi-free with amalgamation over $\cB$ if they exhibit vaccine.
% \end{lemma}
% With the stronger assumption, the proof of Lemma~\ref{lem:vaccineimpliesbifree} carries over unchanged.
% 
% We do point out that the assumptions of Lemma~\ref{lem:amalgavaccineimpliesbifree} are satisfied in more than just the scalar case.
% For example, suppose that $\cB \subset \cZ(\A)$ is a $W^*$-algebra (so isomorphic to the space of continuous functions on some compact set), and $\varepsilon(f\otimes g) = fg$.
% Then given an equation of the form of the one in Lemma~\ref{lem:amalgavaccineimpliesbifree}, we may reduce this to a polynomial equation in $\cB$ by expanding, commuting all of the $b$ terms to the front of the monomials they lie in, and then using multiplicativity to pull them outside of $E$; we are left trying to solve an $n$-variable equation in $\cB$, where no variable has degree greater than $1$ and the coefficients are elements of $\cB$.
% But we also know that the coefficient of the term $b_1\cdots b_n$ is $\pm1$; in particular, if we collect the terms in which $b_1$ appears and let $b_2 = \cdots = b_n = \lambda$, for sufficiently large $\lambda$ the $\C[b_2, \ldots, b_n]$-coefficient of $b_1$ becomes invertible.
% In particular, the equation has a solution of the form $(b, \lambda, \ldots, \lambda)$ for some $b \in \cB$ and $\lambda \in \R$ large.

	  The analogue of Lemma~\ref{lem:vaccineimpliesbifree} requires a bit more care, however, essentially due to the fact that $\cB$ is probably not algebraically closed, and even if it were, terms of the form $E\paren{(z_1-b_1)\cdots(z_n-b_n)}$ (with $z_i \in \A$ and $b_i \in \varepsilon(\cB\otimes\cB)$) do not directly reduce to polynomials with coefficients in $\cB$, as the $b$'s cannot be assumed to commute past the $z$'s and so can't be pulled out of the $E$ without further argument.
	  However, we do have the following Lemma which will suffice for our purposes:
	  \begin{lemma}
		  Suppose $(\A, E, \varepsilon)$ is a $\cB$-$\cB$-non-commutative probability space, with $\cB$ a Banach algebra.
		  Then for every $\chi : [n] \to \slr$ and $z_i \in \A_{\chi(i)}$ there exist $\hat b_i \in \cB$ so that, with $b_i = \varepsilon(\hat b_i\otimes1) = L_{\hat b_i}$ if $\chi(i) = \ell$ and $b_i = \varepsilon(1\otimes \hat b_i) = R_{\hat b_i}$ if $\chi(i) = r$, we have
		  $$
		  E\paren{(z_1-b_1)\cdots (z_n-b_n)} = 0.
		  $$
	  \end{lemma}

	  \begin{proof}
		  Let $j = \min_{\prec_\chi}([n])$.
		  Notice that we can write
		  \begin{align*}
			  E\paren{(z_1-b_1)\cdots (z_n-b_n)}
			  &= E\paren{(z_1-b_1)\cdots(z_{j-1}-b_{j-1})z_j(z_{j+1}-b_{j+1})\cdots(z_n-b_n)} \\
			  &\qquad- \hat b_jE\paren{(z_1-b_1)\cdots(z_{j-1}-b_{j-1})(z_{j+1}-b_{j+1})\cdots(z_n-b_n)}.
		  \end{align*}
		  Indeed, this is immediate if $\chi(j) = \ell$, while if $\chi(j) = r$ we must be in the case $j = n$, so we can replace $R_{\hat b_n}$ by $L_{\hat b_n}$ and then pull $\hat b_n$ out of the left.
		  Now, if we take $b_i = \lambda \in \C$ for $i \neq j$, we find that $b_i$ commutes with every $z_k$, and
		  $$E\paren{(z_1-\lambda)\cdots(z_{j-1}-\lambda)(z_{j+1}-\lambda)\cdots(z_n-\lambda)} = (-\lambda)^n + \bigO{\lambda^{n-1}}$$
		  becomes a polynomial in $\lambda$ with coefficients in $\cB$ and leading term $(-\lambda)^n$.
		  In particular, for $\lambda$ sufficiently large it is invertible in $\cB$.
		  Then once $\lambda$ is large enough, we may take
		  \begin{align*}
			  \hat b_j
			  &= {E\paren{(z_1-\lambda)\cdots(z_{j-1}-\lambda)z_j(z_{j+1}-\lambda)\cdots(z_n-\lambda)}} \\
			  &\qquad \cdot{E\paren{(z_1-\lambda)\cdots(z_{j-1}-\lambda)(z_{j+1}-\lambda)\cdots(z_n-\lambda)}}^{-1}
		  \end{align*}
		  producing a solution to our equation.
	  \end{proof}
	  Of course, we needed something slightly weaker than $\cB$ being a Banach algebra: we only require that monic polynomials with coefficients in $\cB$ have spectrum which is not all of $\C$.

	  With this lemma in hand, we can reprove Lemma~\ref{lem:vaccineimpliesbifree} in the amalgamated setting; the only difference is that for each maximal $\chi$-interval $I$ we must choose a solution to an equation with $\abs{I}$ unknowns, rather than only one.
	  It is of course important to note that if we have $L_b$ or $R_b$ terms occurring in the expansion of this equation, they can be multiplied into adjacent $z$'s to still reduce the total number of variables in the moment being considered.
	  We therefore have the following theorem:
	  \begin{theorem}
		  Suppose that $\cB$ is a Banach algebra, and let $\fpf$ be a family of pairs of $\cB$-faces in a $\cB$-$\cB$-non-commutative probability space $(\A, E, \varepsilon)$.
		  Then the family has vaccine if and only if the pairs of $\cB$-faces are bi-free with amalgamation over $\cB$.
	  \end{theorem}

	  With vaccine established for the amalgamated setting, the results in Section~\ref{sec:morebifreeexamples} follow readily, using essentially the same proofs.
	  In fact if one is more careful about bookkeeping, and carefully studies the action of variables in their standard representation on a free product space, one can establish Theorems~\ref{thm:bihaarconj} and \ref{thm:bipartitefreetobifree} when $\cB$ is merely an algebra with no assumptions of closure; the details may be found in \cite{Charlesworth:2015aa}.

	  \subsection{Amalgamated multiplicative convolution.}
	  In the realm of multiplicative convolution things do not work out as nicely.
	  We do still receive a direct analogue of Proposition~\ref{prop:multicumulantplicationthingywhateverihatelabelsnowsosueme}: the combinatorial argument we employed goes through without difficulty.
	  However we do not arrive at an equation like the one in Theorem~\ref{thm:multiconv}: while it is true in the scalar setting that we can decompose a cumulant $\kappa_{\pi^{(1)}\cup\pi^{(2)}}$ into a product $\kappa_{\pi^{(1)}}\cdot\kappa_{\pi^{(2)}}$, this does not hold in the $\cB$-valued case: the terms corresponding to the second partition may be dispersed through the first cumulant and impossible to collect.


\chapter{An investigation into regularity and free entropy.}
In this section we collect some results dealing with the regularity of non-commutative random variables.
We also show how some of the ideas from the study of free entropy, such as those in \cite{Voiculescu1999101}, and ideas from the study of free unitary Brownian motion coming from \cite{biane1997free} may be used to gain further insight into bi-free probability.

\section{Regularity results.}
We begin by recalling several useful theorems from the literature, which we will use as a starting point for our results.

\subsection{Useful results from the literature.}
This first result is paraphrased to suit our purposes.
\begin{theorem}[\cite{anderson2008law}*{Theorem 2.9}]
	Suppose that $\mu$ is a non-atomic probability measure with algebraic Cauchy transform.
	Then $\mu$ has density $f$ with respect to Lebesgue measure which fails to exist at only finitely many points, and for some $d > 0$ and every $a \in \R$ satisfies
	$$\lim_{x\to a}(x-a)^{1-d} f(x) < \infty.$$
	In particular, if $1 < p < (1-d)^{-1}$, we have $f \in L^p(\R)$.
	\label{thm:alginlp}
\end{theorem}

\begin{theorem}[\cite{2014arXiv1406.6664A}*{Theorem 1}]
	\label{thm:algebraicandfree}
	Let $(\A, \varphi)$ be a non-commutative probability space.
	Let $x_1, \ldots, x_n \in \A$ be freely independent self-adjoint non-commutative random variables.
	Let $$X = X^* \in M_N(\C)\otimes\C\ang{x_1, \ldots, x_n} \subset M_N(\C)\otimes\A$$ be a self-adjoint matrix polynomial.
	If the laws of $x_1, \ldots, x_n$ are algebraic, then so is the law of $X$.
\end{theorem}

\begin{theorem}[\cite{Charlesworth2016}*{Theorem 3}]
	\label{thm:noatoms}
	Let $(M, \tau)$ be a finite tracial von Neumann algebra, and $y_1, \ldots, y_n \in M$.
	Suppose that Voiculescu's free entropy dimension $\delta^*(y_1, \ldots, y_n) = n$.
	Then for any self-adjoint non-constant non-commutative polynomial $P$, the spectral measure of $y = P(y_1, \ldots, y_n)$ has no atoms.
\end{theorem}
We will not define Voiculescu's free entropy dimension, but instead state that the condition in the above theorem is satisfied if $\chi(y_1, \ldots, y_n) > -\infty$ or $\Phi^*(y_1, \ldots, y_n) < \infty$.

\section{Algebraicity and finite entropy.}
We the help of the following proposition, we are able to combine the above results to show that finite entropy is preserved under polynomial convolutions for sufficiently smooth variables.


\begin{proposition}
	\label{prop:lpentropy}
	Suppose that $y = y^*$ is a self-adjoint variable in a finite tracial von Neumann algebra $(M, \tau)$.
	Further suppose that the spectral distribution of $y$ is Lebesgue absolutely continuous, with density $f$.
	If $f \in L^p(\R)$ for some $p > 1$, then $\chi(y) > -\infty$.
\end{proposition}
\begin{proof}
	Using interpolation and the fact that $\norm{f}_1 = 1$, we may assume that $p < 2$.
	From Theorem~\ref{thm:singlevariableentropy}, we know that the free entropy of $y$ is given by
	$$\chi(y) = \iint_{\R^2}\log\abs{s-t} \,d\mu_y(s)\,d\mu_y(t) + \frac34 + \frac12\log2\pi,$$
	so it suffices to bound the integral above.
	Let $L(t) := 1_{(-4M, 4M)}(t)\log\abs{t}$, where $M = \norm{y}$ (so that the support of $\mu_y$ is contained in $[-M, M]$).
	Now for $s \in \R$, we have
	$$f(s) \int f(t) \log\abs{s-t}\,dt = f(s)\int f(t)L(s-t)\,dt = f(s)(f\star L)(s).$$
	We compute:
	\begin{align*}
		\abs{\iint \log\abs{t-s}\,d\mu(t)\,d\mu(s)}
		&= \abs{\int f(s) \int f(t)\log\abs{s-t}\,dt\,ds}\\
		&= \abs{\int f(s)(f\star L)(s)\,ds}\\
		&\leq \norm{f\cdot(f\star L)}_1\\
		&\leq \norm{f}_p\norm{f\star L}_q,
	\end{align*}
	where $1 = \frac1p+\frac1q$, by H\"older's inequality; note that $q > 2$ since $p < 2$.
	It suffices, now, to show that $f\star L \in L^q(\R)$; for this, we appeal to Young's inequality.
	Indeed, if $s = \frac q2 > 1$, then $1 + \frac 1q = \frac1p+\frac1s$ and so we have
	$$\norm{f \star L}_q \leq \norm{f}_p\norm{L}_s.$$
	Yet $\norm{L}_s < \infty$ for any $1 \leq s < \infty$, so we conclude $\abs{\chi(x)} < \infty$.
\end{proof}


\begin{corollary}
	Suppose that $y_1, \ldots, y_n$ are freely independent, self-adjoint, and algebraic, with $\chi(y_i) > -\infty$.
	Then if $P$ is a non-constant polynomial and $y = P(y_1, \ldots, y_n)$, we have $\chi(y) > -\infty$.
\end{corollary}

\begin{proof}
	From Theorem~\ref{thm:algebraicandfree}, we know that $y$ is algebraic.
	As $y_1, \ldots, y_n$ are free, we have $\chi(y_1, \ldots, y_n) = \chi(y_1) + \ldots + \chi(y_n) > -\infty$, and so Theorem~\ref{thm:noatoms} informs us that the spectral measure of $y$ has no atoms.
	Then Theorem~\ref{thm:alginlp} tells us that Proposition~\ref{prop:lpentropy} applies, and we conclude that $\chi(y) > -\infty$.
\end{proof}

It is tempting to believe that the above corollary is true far more generally, such as when $y_1, \ldots, y_n$ is merely a system of variables with $\chi(y_1, \ldots, y_n) > -\infty$.
Indeed, one expects that polynomial convolution is an operation which should be well-behaved.
However, we are not aware of a proof that works in this full generality.


\subsection{Properties of spectral measures.}
We will show in this section that certain strong regularity properties on the generators of an algebra do ensure weaker regularity of elements of that algebra.
We begin by establishing several useful lemmata which will allow us to gain some control over operators based on their spectral measures.
The first result may be found in \cite{voiculescu1979some}*{Section 4}, though our narrower version of it essentially goes back to Kato~\cite{kato1966perturbation}.
We sketch a proof here for completeness.
\begin{lemma}
	\label{lem:singularcontractioncommutation}
	Let $x = x^* \in B(\cH)$ be a self-adjoint operator, and assume that the spectral measure of $x$ is not Lebesgue absolutely continuous.
	Then there exists a sequence $T_n$ of finite-rank operators which satisfy the following properties:
	\begin{itemize}
		\item $0 \leq T_n \leq 1$;
		\item $T_n \to p$ weakly, where $p$ is the spectral projection of $x$ corresponding to the support of the Lebesgue-singular part of its spectral measure; and
		\item $\norm{[T_n, x]}_1 \to 0$.
	\end{itemize}
\end{lemma}


\begin{proof}
	Replacing $x$ by $pxp$, we may assume that $x$ has singular spectral measure.
	Let us further assume that the spectrum of $x$ is contained in $[0,1]$ (renormalizing and shifting $x$ if necessary) and that $\cH = L^2([0,1], \mu_x)$.
	Fix $n > 0$ and let $E_1, \ldots, E_k$ be a disjoint collection of Borel subsets of $[0,1]$ such that
	$$\sum_{i=1}^k\operatorname{diam}(E_i) < \frac1n \qquad\text{but}\qquad \mu_x\paren{\bigcup_{i=1}^k E_i} > 1-\frac1n.$$
	Now let $Q_i$ be the rank $1$ projection onto the function $1_{E_i}$, and $P_n = Q_1 + \ldots + Q_k$ a rank $k$ projection.
	Plainly $P_n \to 1$ weakly, by our choice of $E_i$.
	On the other hand, 
	$$[P_n, x] = \sum_{i=1}^k [Q_i, x].$$
	Now $[Q_i, x]$ has rank at most two, while $\norm{[Q_i, x]} \leq \operatorname{diam}(E_i) < \frac1n$; hence 
	$$\norm{[P_n, x]}_1 \leq \sum_{i=1}^k \norm{[Q_i, x]}_1 \leq 2\operatorname{diam}(E_i) \leq \frac2n \to 0.$$
\end{proof}

The next two lemmata are similar to each other in spirit; each allows us to conclude that the vanishing of certain derivative-like quantities implies that a variable must be constant.
First, though, we establish some convenient notation.
\begin{notation}
	In what follows, given $x, y \in \A$, we will denote $y\otimes x =: (x\otimes y)^\sigma$, and extend this map linearly to all of $\A\otimes\A$.
	Recall also the notation we established in Section~\ref{ss:freeentropyintro}: if $X$ is an $\A$-$\A$-bimodule, $\xi\in X$, and $x, y \in \A$, then $(x\otimes y)\#\xi = x\cdot\xi\cdot y$.
\end{notation}

\begin{lemma}
	\label{lem:zero-der}
	Let $(\A, \tau)$ be a non-commutative probability space with faithful tracial state $\tau$, generated by algebraically free self-adjoint elements $y_1, \ldots, y_n$.
	Take $y \in \A$ a polynomial.
	Let $\partial_i : \A\otimes\A \to \A$ be the free difference quotients, as in Definition~\ref{defn:freedifferencequotients}.
	Then $y \in \operatorname{Alg}\paren{y_1, \ldots, \hat{y}_i, \ldots, y_n}$ if and only if
	$$\paren{\partial_i y}^\sigma\#y^* = 0.$$
	Consequently, if the above equation holds for each $i$, $y \in \C$.
\end{lemma}

\begin{proof}
	One direction is immediate as $\operatorname{Alg}(y_1, \ldots, \hat y_i, \ldots, y_n) \subseteq \ker\partial_i$.

	Let $\mathcal{N}_i : \A \to \A$ be the number operator associated to $y_i$, the linear map which multiplies each monomial by its $y_i$-degree.
	Observe that
	$$(\partial_i y)\# y_i = \mathcal{N}_i(y),$$
	as each monomial $m$ in $y$ contributes $\sum_{m = ay_ib} (a\otimes b)\# y_i = \mathcal{N}_i(m)$.

	Suppose, then, that $(\partial_i y)^\sigma\#y^* = 0$, so $y_i(\partial_i y)^\sigma\#y^* = 0$ as well.
	Let $\varphi_\lambda : A \to A$ be the algebra homomorphism given by $\varphi_\lambda(y_i) = \lambda y_i$, $\varphi_\lambda(y_j) = y_j$ for $j \neq i$, which exists as $y_1, \ldots, y_n$ are algebraically free.
	We compute the following:
	$$
	0
	= \tau\circ\varphi_\lambda(0)
	= \tau\circ\varphi_\lambda\left(y_i(\partial_i y)^\sigma\#y^*\right)
	= \tau\circ\varphi_\lambda\left(y^*(\partial_i y)\# y_i\right)
	= \tau\circ\varphi_\lambda\left(y^* \mathcal{N}_i(y)\right).
	$$

	Now, suppose $\deg_{y_i}(y) = d$, and take $x, z \in A$ so that $x$ is $y_i$-homogeneous of degree $d$, $\deg_{y_i}(z) < d$, and $y = x + z$.
	Then:
	$$
	0
	= \tau\circ\varphi_\lambda\left( y^*\mathcal{N}_i(y) \right)
	= \tau\circ\varphi_\lambda(dx^*x) + \tau\circ\varphi_\lambda(z^*\mathcal{N}_i(y) + x^*\mathcal{N}_i(z))
	= d\lambda^{2d}\tau(x^*x) + \bigO{\lambda^{2d-1}}.
	$$
	Thus $d\lambda^{2d}\tau(x^*x) = 0$, and as $\tau$ is faithful, either $x = 0$ (in which case $y = 0$) or $d = 0$ (in which case $\deg_{y_i}(y) = 0$).
	Either way, $y \in \operatorname{Alg}(y_1, \ldots, \hat y_i, \ldots, y_n)$.

	Repeated application of the above yields the final claim.
\end{proof}

\begin{lemma}
	\label{lem:zero-der-again}
	Let $y_1, \ldots, y_n$ be algebraically free self-adjoint elements which generate a finite tracial von Neumann algebra $(M, \tau)$, and $\A$ be the algebra they generate.
	Suppose further that $\delta^*(y_1, \ldots, y_n) = n$.
	Once again, let $\partial_i : \A\otimes\A \to \A$ be the free difference quotients.
	Suppose that $N: \A \to \A$ is a linear combination of number operators $\cN_i$, so for $x \in \A$,
	$$N(x) = \sum_{i=1}^n a_i(\partial_ix)\# y_i.$$
	Let $y = y^* \in \A$ be an eigenvector of $N$.
	Then if for some spectral projection $p \in \cP(W^*(y))$ and for each $1 \leq i \leq n$ with $a_i \neq 0$ we have
	$$(\partial_i y)^\sigma\#(py) = 0,$$
	it follows that $N(y) = 0$.
\end{lemma}

\begin{proof}
	As in the proof of Lemma \ref{lem:zero-der}, we compute:
	$$
	0 = \sum_{i=1}^n a_i\tau\left(y_i(\partial_iy)^\sigma\#(py)\right)
	= \tau\left( py \sum_{i=1}^n a_i(\partial_i y)\#y_i \right)
	= \tau\left( py N(y) \right)
	= \lambda \tau\left( pyyp \right).
	$$
	If $\lambda \neq 0$, it follows that $pyyp = 0$, hence $py = 0$.
	But then by Theorem~\ref{thm:noatoms} we have that $y$ is constant and each $\mathcal{N}_i(y) = 0$.
	In either case, $N(y) = 0$.
\end{proof}

We now ready to state and prove the main result of this section.
\begin{theorem}
	Let $y_1, \ldots, y_n$ be algebraically free self-adjoint elements which generate a finite tracial von Neumann algebra $(M, \tau)$, and $\A$ be the algebra they generate.
	Suppose further that $y_1, \ldots, y_n$ admit a dual system $R_1, \ldots, R_n$ as in Definition~\ref{defn:dualsystem}.
	Take $y = y^* \in \A$ be a non-constant polynomial evaluated at $(y_1, \ldots, y_n)$.
	Then the spectral measure of $y$ is not singular with respect to Lebesgue measure.
	Moreover, if there is $N$ is in the positive linear span of the number operators $\cN_i$ such that each $\cN_i$ has a non-zero coefficient and $y$ is an eigenvector of $N$, then the spectral measure of $y$ is Lebesgue absolutely continuous.
\end{theorem}

\begin{proof}
	Assume to the contrary that the spectral measure of $y$ is not absolutely continuous with respect to Lebesgue measure.
	By Lemma~\ref{lem:singularcontractioncommutation}, we may choose $0\leq T_n \leq 1$ finite rank operators with $T_n\to p$ weakly and $\norm{[T_n,y]}_1 \to 0$, where $p$ is the spectral projection onto the singular part of $y$.
	Let $J : L^2(M)\to L^2(M)$ be Tomita's conjugation operator, defined on $x \in M$ by $J(x) = x^*$, and extended by continuity to $L^2(M)$.

	Fix $x \in M$.
	Note that by definition of $R_j$, $[R_j,y_k]=\partial_j (y_k) \# P_1$; since both $[R_j,\cdot]$ and $\partial_j(\cdot)\#P_1$ are derivations, it follows that $[R_j,y]=\partial_j(y)\#P_1$.
	We now compute as follows:
	\begin{align*}
		0
		&= \lim_{n\to\infty} \norm{ Jx^*J R_i y }_\infty \norm{ [T_n, y] }_1 \\
		&\geq \lim_{n\to\infty} \abs{ \operatorname{Tr}( Jx^*J R_i y [T_n, y]) } \\
		&= \lim_{n\to\infty} \abs{ \operatorname{Tr}(Jx^*J [y, R_i] yT_n) } \\
		&= \lim_{n\to\infty} \abs{ \operatorname{Tr}(Jx^*J ((\partial_i y)\# P_1) yT_n) } \\
		&= \abs{ \operatorname{Tr}( Jx^*J P_1 ((\partial_i y)^\sigma \# (yp)) ) } \\
		&= \abs{ \tau(((\partial_i y)^\sigma\#(yp))x) }
	\end{align*}
	Here we used: the inequality $\norm{ Tw }_1 \leq \norm{ T}_1 \norm{ w}_\infty$, the trace property and commutation between $Jx^*J$ and $y$, the equality $[y,R_i] =-\partial_i(y)\#P_1$, the fact that $P_1$ is finite rank (so that we can pass to the limit $T_n\to p$) and finally the equality $\tau(z) = \langle z1, 1\rangle =\operatorname{Tr} (zP_1)$.

	We conclude that $(\partial_i y)^\sigma\#(yp) = 0$ as $\tau$ is faithful and as $x$ was arbitrary.
	Then if $p = 1$, Lemma~\ref{lem:zero-der} implies that $y$ is constant, which is absurd; thus the spectral measure of $y$ cannot be singular with respect to Lebesgue measure.

	Further, if we have $N$ as in the statement of the theorem, then Lemma~\ref{lem:zero-der-again} implies that $N(y) = 0$.
	It follows that for any non-zero monomial $m$ in $y$, $N(m) = 0$.
	As $m$ is an eigenvector of each $\cN_i$ and each has a non-negative coefficient in $N$, we learn that $\cN_i(m) = 0$.
	Thus $m$ has zero degree for each $y_i$, and so $m \in \C$.
	We conclude that $y \in \C$, a contradiction.
\end{proof}




\section{Bi-free unitary Brownian motion.}
Our aim in this section is to study multiplicative bi-free Brownian motion, as an analogue to the free unitary Brownian motion introduced by Biane \cite{biane1997free}.
Many related results in the free case were obtained in the context of a tracial von Neumann algebra, allowing the arguments to be simplified; unfortunately that luxury is not available to us in the context of bi-free probability as we are not aware of an appropriate analogue of traciality.
As the following example demonstrates, simply asking that the state on the non-commutative probability space be tracial is too restrictive.

\begin{example}
	Suppose $\paren{\A_\ell^{(\iota)}, \A_r^{(\iota)}}_{\iota\in\set{\makeaball{0},\makeaball{1}}}$ are bi-free pairs of faces in a non-commutative probability space $(\A, \varphi)$.
	Then we have for $x \in \A_\ell^{(\makeaball{0})}$, $w \in \A_r^{(\makeaball{0})}$, $y \in \A_\ell^{(\makeaball{1})}$, and $z \in \A_r^{(\makeaball{1})}$ that
	\[
		\begin{tikzpicture}[baseline]
			\def\colours{{0, 0, 1, 1}}
			\def\sidez{{1, -1,-1,1,1}}
			\def\opnames{{"$w$", "$x$", "$y$", "$z$"}}

			\begin{scope}[shift={(-\textwidth*0.25,0)}]
				\def\ordr{{1,2,3,0}}
				\bnc [n=4,colourzfrompalette=\colours,sidez=\sidez,order=\ordr,labelz=\opnames] 
				\node [below, scale=0.7] (vp1) at (bc) {$\varphi(xyzw) = \varphi(xw)\varphi(y)\varphi(z) + \varphi(x)\varphi(w)\varphi(yz) - \varphi(x)\varphi(w)\varphi(y)\varphi(z)$};
			\end{scope}
			\begin{scope}[shift={(\textwidth*0.25,0)}]
				\def\ordr{{0,1,2,3}}
				\bnc [n=4,colourzfrompalette=\colours,sidez=\sidez,order=\ordr,labelz=\opnames] 
				\node [below, scale=0.7] (vp2) at (bc) {$\varphi(wxyz) = \varphi(wx)\varphi(yz)$.};
			\end{scope}
			\path (vp1) -- node [scale=0.7] {while} (vp2);
		\end{tikzpicture}
	\]
	Note that these two terms fail to be equal even when $(x, w)$, $(y, z)$ are a bi-free standard semicircular system with $\varphi(wx) = \varphi(yz) = 1$, as the left expression vanishes while the right equals $1$.
\end{example}

\subsection{Free Brownian motion.}
We take some time to review the concept of free Brownian motion, which is the free analogue of the Gaussian process acting on a Hilbert space.
A more complete description of free Brownian motion and free stochastic calculus may be found in \cite{voiculescu1992free}.

\begin{definition}
	A \emph{free Brownian motion} in a non-commutative probability space $(\A, \varphi)$ is a non-commutative stochastic process $(S(t))_{t\geq0}$ such that:
	\begin{itemize}
		\item the increments of $S(t)$ are free: for $0\leq t_1 < \cdots < t_k$, the collection $S(t_2)-S(t_1), \ldots, S(t_k)-S(t_{k-1})$ are freely independent; and
		\item the process is stationary, with semicircular increments: for $0 \leq s < t$, $S(t)-S(s)$ is semicircular with variance $t-s$.
	\end{itemize}
\end{definition}

Free Brownian motion can be modelled on a Fock space \cite{voiculescu1992free}. Indeed, suppose
$$\mathcal{F}\paren{L^2(\R_{\geq0})} := \C\Omega \oplus \bigoplus_{n\geq1} L^2(\R_{\geq0})^{\otimes n}.$$
Let $\xi_t = 1_{[0, t]}$, and define $S(t) = l(\xi_t) + l^*(\xi_t)$.
Then $(S(t))_t$ is a free Brownian motion.

Free unitary Brownian motion was initially introduced by Biane in \cite{biane1997free} as a multiplicative analogue of the (additive) free Brownian motion above.
It's definition makes reference to a certain family of measures $(\nu_t)_{t\geq0}$ supported on $\mathbb{T}$, introduced by Bercovici and Voiculescu in \cite{bercovici1992levy}.
In particular, $\nu_t$ has the property that for $t, s \geq 0$, $\nu_t\boxtimes\nu_s = \nu_{t+s}$.
We do not require the particular details of its introduction and so will eschew them.

\begin{definition}
	A \emph{free unitary Brownian motion} in a non-commutative probability space $(\A, \varphi)$ is a non-commutative stochastic process $(U(t))_{t\geq0}$ such that:
	\begin{itemize}
		\item the (left) multiplicative increments of $U(t)$ are free: for $0 \leq t_1 < \cdots < t_k$, the increments given by $U^*(t_1)U(t_2), U^*(t_2)U(t_3), \ldots, U^*(t_{n-1})U(t_n)$ are freely independent; and
		\item the process is stationary with increments prescribed by $\nu_\cdot$: the distribution of $U^*(t)U(s)$ depends only on $s-t$, and is in fact $\nu_{s-t}$.
	\end{itemize}
\end{definition}

It was shown in \cite{biane1997free} that if $S(t)$ is a Fock space realization of a free additive Brownian motion and $U(t)$ the solution to the free stochastic differential equation
$$dU(t) = iU(t)\,dS(t) - \frac12 U(t)\,dt$$
with $U(0) = 1$, then $U(t)$ is a free unitary Brownian motion.
Moreover, the moments of a free unitary Brownian motion were computed: for $n > 0$,
$$\varphi(U(t)^n) = \sum_{k=0}^{n-1}(-1)^k\frac{t^k}{k!}n^{k-1}{n \choose k+1}e^{-nt/2}.$$
A consequence is that free unitary Brownian motion converges in distribution as $t\to\infty$ to a Haar unitary, i.e., a unitary $u_\infty$ with $\varphi(u_\infty^{k}) = \delta_{k=0}$ for $k \in \Z$.
Another important results from \cite{biane1997free} is the following bound: for some $K > 0$ and any $t > 0$,
$$\norm{U(t)-e^{-t/2}} \leq K\sqrt{t}.$$
We have already identified the bi-free analogue of Haar unitaries in Subsection~\ref{ssec:haarunitary}, which we will use to motivate our approach to a bi-free version of Brownian motion: we want a process which tends to the distribution of a Haar pair of unitaries, so that conjugating by the process asymptotically creates bi-freeness and can therefore be seen as a sort of liberation.





\subsection{The free liberation derivation.}
Suppose that $A, B$ are algebraically free unital sub-algebras generating a tracial non-commutative probability space $(\A, \tau)$.
In \cite{Voiculescu1999101}, Voiculescu defined the derivation $\delta_{A:B} : \A \to \A\otimes\A$ to be a linear map satisfying the Leibniz rule such that $\delta_{A:B}(a) = a\otimes1-1\otimes a$ for $a \in A$ and $\delta_{A:B}(b) = 0$ for $b \in B$.
It was shown that $A$ and $B$ are freely independent if and only if $(\tau\otimes\tau)\circ\delta_{A:B} \equiv 0$.
Moreover, the derivation $\delta_{A:B}$ relates to how the joint distribution of $A$ and $B$ changes as $A$ is perturbed by unitary free Brownian motion.
\begin{proposition}[\cite{Voiculescu1999101}*{Proposition 5.6}]
	Let $A, B$ be two unital $*$-subalgebras in $(\A, \tau)$ and let $\paren{U(t)}_{t\geq0}$ be a unitary free Brownian motion, which is freely independent of $A\vee B$.
	If $a_j \in A$ and $b_j \in B$ for $1\leq j \leq n$, then
	\begin{align*}
		\tau\paren{U(\epsilon)a_1U(\epsilon)^*b_1\cdots U(\epsilon)a_nU(\epsilon)^*b_n} 
		&= 
		\frac\epsilon2(\tau\otimes\tau)\left(\delta_{A:B}\left(\sum_{k=1}^n a_kb_k\cdots a_nb_na_1b_1\cdots a_{k-1}b_{k-1} \right.\right.\\
		&\qquad\qquad\qquad- \left.\left.\sum_{k=1}^n b_ka_{k+1}b_{k+1}\cdots a_nb_na_1b_1\cdots b_{k-1}a_k\right)\right) \\
		&\qquad+\tau(a_1b_1\cdots a_nb_n) + \bigO{\epsilon^2}.
	\end{align*}
\end{proposition}

Important to the proof of the above proposition, and of use to us here also, is the following approximation result.
\begin{proposition}[\cite{Voiculescu1999101}*{Proposition 1.4}]
	\label{prop:approxubm}
	Let $A$ be a $W^*$-subalgebra, $(U(t))_t$ a unitary free Brownian motion, and $S$ a $(0,1)$-semicircular element in $(M, \tau)$ so that $A$ and $(U(t))_t$ are $*$-free and $A$ and $S$ are also free. If $a_j \in A$ and $\alpha_j \in \set{1,-1}$, then we have
	$$\tau\paren{\prod_{1\leq j \leq n}^{\rightarrow} a_j U(t)^{\alpha_j}}
	= \tau\paren{\prod_{1\leq j \leq n}^{\rightarrow} a_j\paren{\paren{1-\frac{t}{2}}+i\alpha_j\sqrt{t}S}} + \mathcal{O}\paren{t^2},$$
	where the products place the terms in order from left to right.
\end{proposition}
Although the proposition was stated in terms of a tracial $W^*$-probability space, traciality was not needed in the proof.

\subsection{A bi-free analogue to the liberation derivation.}
\renewcommand{\e}{\epsilon} %this is gonna be painful to fix... TODO
For the remainder of this section, we will always be working in the context of a family of pairs of faces $\paren{(\A_\ell^{(\iota)}, \A_r^{(\iota)})}_{\iota\in\I}$ generating a non-commutative probability space $(\A, \varphi)$.
We will further denote by $\A_\ell$ and $\A_r$ the algebras generated by $\set{\A_\ell^{(\iota)} : \iota \in \I}$ and $\set{\A_r^{(\iota)} : \iota \in \I}$ respectively, and by $\A^{(\iota)}$ the algebra generated by $\A_\ell^{(\iota)}$ and $\A_r^{(\iota)}$.
Moreover, we assume that there are no algebraic relations between $\A^{(i)}$ and $\A^{(j)}$ other than $[\A^{(i)}_\ell, \A^{(j)}_r] = 0$ when $i \neq j$, and possibly $[\A^{(i)}_\ell, \A^{(i)}_r] = 0$.
In particular, we want to ensure that given a product $z_1\cdots z_n$ we can determine the $\chi$-order of the variables.

Suppose $\chi : [n] \to \set{\ell, r}$ and let $1 \leq i, j \leq n$ with $i \preceq_\chi j$.
We denote $[i, j]_\chi := \set{k : i \preceq_\chi k \preceq_\chi j}$ the $\chi$-interval between $i$ and $j$, and define analogously $[i, j)_\chi$, $(i, j]_\chi$, and $(i, j)_\chi$.
Likewise we define $[i, \infty)_\chi := \set{k : 1 \leq k \leq n, i \preceq k}$ and analogously the other rays.

We will also change our conventions on the use of $\iota$ from earlier.
For the remainder of this section, $\iota$ will always be an element of $\I$, never a map; we will use $\gem$ to denote a map $\gem: [n] \to \I$.

\begin{definition}
	Fix $\iota \in \I$.
	We define a map
	$$\taur_{\A^{(\iota)} : \bigvee_{j \in \I\setminus\set{\iota}} \A^{(j)}} : \A \to \A \otimes \A$$
	as follows.
	Given $\chi : [n] \to \slr$, $\gem: [n] \to \I$, and $z_i \in \A^{(\gem(i))}_{\chi(i)}$,
	%\begin{dmath*}
	$$
	\taur_{\A^{(\iota)} : \bigvee_{j \in \I\setminus\set{\iota}} \A^{(j)}}(z_1\cdots z_n)
	= \sum_{i \in \e^{-1}(\iota)} \sum_{\substack{j\in\e^{-1}(\iota)\\i \preceq_\chi j}}
	z_{[i,j]_\chi^c} \otimes z_{[i,j]_\chi}
	- z_{[i,j)_\chi^c}\otimes z_{[i,j)_\chi}
	- z_{(i,j]_\chi^c}\otimes z_{(i,j]_\chi}
	+ z_{(i,j)_\chi^c}\otimes z_{(i,j)_\chi}.
	$$
	%\end{dmath*}
	We now extend this definition by linearity to all of $\A$.
	When context makes our intent clear, we will sometimes write $\taur_\iota$ for $\taur_{\A^{(\iota)} : \bigvee_{j \in \I\setminus\set{\iota}} \A^{(j)}}$.

	The subscript $A : B$ is meant to mimic that in the free situation, and the basic properties present there still hold: $B \subset \ker\taur_{A:B}$ and for $a \in A$, $\taur_{A:B}(a) = 1\otimes a-a\otimes 1$.
	However, $\taur_{A:B}$ is not a derivation, even when restricted to the left or right faces of $A$ and $B$.
\end{definition}
\begin{lemma}
	\label{lem:taur}
	$\taur_\iota$ is well-defined.
	Moreover, the only terms which do not cancel in the sum defining $\taur_\iota$ are those in which no $\gem$-monochromatic $\chi$-interval is split across a tensor sign.
\end{lemma}
\begin{proof}
	Our assumptions about the lack of algebraic relations in $\A$ mean that the only ambiguity in writing a product $z_1\cdots z_n$ comes from grouping or failing to group adjacent terms, and commuting left and right terms; the latter has no impact on $\taur_\iota$ because it does not change the $\chi$-ordering of the variables.
	Notice that if $i \prec_\chi i_+$ are consecutive under the $\chi$-ordering and both contribute to the sum, then all intervals with $i$ as an open left endpoint are intervals with $i_+$ as a closed left endpoint and have opposite sign in their contributions to the two terms; likewise, all intervals with $i$ as a closed right endpoint are intervals with $i_+$ as an open right endpoint and again cancel.
	Hence the value of $\taur_\iota$ does not change if a product is written differently, and the only terms which do not cancel are those with the tensor sign falling between two $\e$-monochromatic $\chi$-intervals (or one such interval and the edge of the product), exactly one of which is $\iota$-coloured.
\end{proof}

In essence, $\taur_\iota$ acts by adding one term for each $\chi$-interval with endpoints either before or after terms coming from $\A^{(\iota)}$, consisting of the product of the terms not in that interval tensored with the product of the terms in the interval.
The sign is chosen so that if the division comes before both chosen nodes or after both chosen nodes the term counts negatively, and otherwise counts positively.
Notice that when $i = j$, the terms corresponding to $[i,i)_\chi$ and $(i,i)_\chi$ cancel and only one term contributing $-z_1\cdots z_n \otimes 1$ survives.

The liberation gradient $\delta_{A^{(\iota)}:B}$ can be expressed in a similar manner:
$$\delta_{A:B}(z_1\cdots z_n) = \sum_{i \in \gem^{-1}(\iota)} -z_{(-\infty, i)}\otimes z_{(-\infty, i)^c} + z_{(-\infty, i]} \otimes z_{(-\infty, i]^c}.$$

\begin{example}
	Let $\chi:[n]\to\slr$ and $\gem:[n]\to\I$ be as in Example~\ref{ex:vaccine}.
	Then $\taur_{\makeaball{1}}(z_1\cdots z_{10})$ is a sum of the following eight terms:
	\[\begin{tikzpicture}[baseline]
			\def\colours{{0, 0, 0, 1, 0, 1, 1, 0, 0, 0}}
			\def\sidez{{1,-1,1,1,-1,-1,-1,1,-1,1}}

			\begin{scope}[shift={(-\textwidth*0.375,0)},scale=1]
				\draw[thick] (-1,0.125) -- (-1, -2.375) --
				node[below,scale=2/3] {$-z_1\cdots z_{10}\otimes 1$}
				(1,-2.375) -- (1,0.125);
				\foreach \y in {0,...,9} {
					\pgfmathtruncatemacro{\nodename}{\y+1}
					\pgfmathtruncatemacro{\sd}{\sidez[\y]}
					\pgfmathparse{\palette[\colours[\y]]}
					\def\clr{\pgfmathresult}
					\node (ball\nodename) [draw, shade, circle, ball color=\clr, inner sep=0.07cm*2/3] at (\sd, -\y*1/4) {};
					\ifthenelse{\sd=1}{\node[scale=2/3,right] at (\sd, -\y*0.25) {\nodename}}
					{\node[scale=2/3,left] at (\sd, -\y*0.25) {\nodename}};
				}
				\draw (-1.1, -1.125) -- (-.8, -1.125);
			\end{scope}
			\begin{scope}[shift={(-\textwidth*0.125,0)},scale=1]
				\draw[thick] (-1,0.125) -- (-1, -2.375) --
				node[below,scale=2/3] {$z_1\cdots z_5 z_8z_9z_{10}\otimes z_6z_7$}
				(1,-2.375) -- (1,0.125);
				\foreach \y in {0,...,9} {
					\pgfmathtruncatemacro{\nodename}{\y+1}
					\pgfmathtruncatemacro{\sd}{\sidez[\y]}
					\pgfmathparse{\palette[\colours[\y]]}
					\def\clr{\pgfmathresult}
					\node (ball\nodename) [draw, shade, circle, ball color=\clr, inner sep=0.07cm*2/3] at (\sd, -\y*1/4) {};
					\ifthenelse{\sd=1}{\node[scale=2/3,right] at (\sd, -\y*0.25) {\nodename}}
					{\node[scale=2/3,left] at (\sd, -\y*0.25) {\nodename}};
				}
				\draw (-1.1, -1.125) -- (-1, -1.125) to [in=90,out=0] (-.8, -1.375) to [in=0,out=270] (-1, -1.625) -- (-1.1, -1.625);
			\end{scope}
			\begin{scope}[shift={(\textwidth*0.125,0)},scale=1]
				\draw[thick] (-1,0.125) -- (-1, -2.375) --
				node[below,scale=2/3] {$-z_1\cdots z_5\otimes z_6\cdots z_{10}$}
				(1,-2.375) -- (1,0.125);
				\foreach \y in {0,...,9} {
					\pgfmathtruncatemacro{\nodename}{\y+1}
					\pgfmathtruncatemacro{\sd}{\sidez[\y]}
					\pgfmathparse{\palette[\colours[\y]]}
					\def\clr{\pgfmathresult}
					\node (ball\nodename) [draw, shade, circle, ball color=\clr, inner sep=0.07cm*2/3] at (\sd, -\y*1/4) {};
					\ifthenelse{\sd=1}{\node[scale=2/3,right] at (\sd, -\y*0.25) {\nodename}}
					{\node[scale=2/3,left] at (\sd, -\y*0.25) {\nodename}};
				}
			%\draw plot [smooth] coordinates {(-1.1, -1.125) (-0.5, -1.125) (0, -1) (0.5, -0.875) (1.1, -0.875)};
				\draw (-1.1, -1.125) -- ++(0.6,0) .. controls +(.5,0) and +(-.5, 0) .. (0.5, -0.875) -- ++(0.6,0);
			\end{scope}
			\begin{scope}[shift={(\textwidth*0.375,0)},scale=1]
				\draw[thick] (-1,0.125) -- (-1, -2.375) --
				node[below,scale=2/3] {$z_1z_2z_3z_5\otimes z_4z_6\cdots z_{10}$}
				(1,-2.375) -- (1,0.125);
				\foreach \y in {0,...,9} {
					\pgfmathtruncatemacro{\nodename}{\y+1}
					\pgfmathtruncatemacro{\sd}{\sidez[\y]}
					\pgfmathparse{\palette[\colours[\y]]}
					\def\clr{\pgfmathresult}
					\node (ball\nodename) [draw, shade, circle, ball color=\clr, inner sep=0.07cm*2/3] at (\sd, -\y*1/4) {};
					\ifthenelse{\sd=1}{\node[scale=2/3,right] at (\sd, -\y*0.25) {\nodename}}
					{\node[scale=2/3,left] at (\sd, -\y*0.25) {\nodename}};
				}
			%\draw plot [smooth] coordinates {(-1.1, -1.125) (-0.5, -1.125) (0, -.875) (0.5, -0.625) (1.1, -0.625)};
				\draw (-1.1, -1.125) -- ++(0.6,0) .. controls +(.5,0) and +(-.5, 0) .. (0.5, -0.625) -- ++(0.6,0);
			\end{scope}
			\begin{scope}[yshift=-3cm]
				\begin{scope}[shift={(-\textwidth*0.375,0)},scale=1]
					\draw[thick] (-1,0.125) -- (-1, -2.375) --
					node[below,scale=2/3] {$z_1\cdots z_7\otimes z_8z_9z_{10}$}
					(1,-2.375) -- (1,0.125);
					\foreach \y in {0,...,9} {
						\pgfmathtruncatemacro{\nodename}{\y+1}
						\pgfmathtruncatemacro{\sd}{\sidez[\y]}
						\pgfmathparse{\palette[\colours[\y]]}
						\def\clr{\pgfmathresult}
						\node (ball\nodename) [draw, shade, circle, ball color=\clr, inner sep=0.07cm*2/3] at (\sd, -\y*1/4) {};
						\ifthenelse{\sd=1}{\node[scale=2/3,right] at (\sd, -\y*0.25) {\nodename}}
						{\node[scale=2/3,left] at (\sd, -\y*0.25) {\nodename}};
					}
			%\draw plot [smooth] coordinates {(-1.1, -1.625) (-0.5, -1.625) (0, -1.25) (0.5, -0.875) (1.1, -0.875)};
					\draw (-1.1, -1.625) -- ++(0.6,0) .. controls +(.5,0) and +(-.5, 0) .. (0.5, -0.875) -- ++(0.6,0);
				\end{scope}
				\begin{scope}[shift={(-\textwidth*0.125,0)},scale=1]
					\draw[thick] (-1,0.125) -- (-1, -2.375) --
					node[below,scale=2/3] {$-z_1z_2z_3z_5z_6z_7\otimes z_4z_8z_9z_{10}$}
					(1,-2.375) -- (1,0.125);
					\foreach \y in {0,...,9} {
						\pgfmathtruncatemacro{\nodename}{\y+1}
						\pgfmathtruncatemacro{\sd}{\sidez[\y]}
						\pgfmathparse{\palette[\colours[\y]]}
						\def\clr{\pgfmathresult}
						\node (ball\nodename) [draw, shade, circle, ball color=\clr, inner sep=0.07cm*2/3] at (\sd, -\y*1/4) {};
						\ifthenelse{\sd=1}{\node[scale=2/3,right] at (\sd, -\y*0.25) {\nodename}}
						{\node[scale=2/3,left] at (\sd, -\y*0.25) {\nodename}};
					}
			%\draw plot [smooth] coordinates {(-1.1, -1.625) (-0.5, -1.625) (0, -1.125) (0.5, -0.625) (1.1, -0.625)};
					\draw (-1.1, -1.625) -- ++(0.6,0) .. controls +(.5,0) and +(-.5, 0) .. (0.5, -0.625) -- ++(0.6,0);
				\end{scope}
				\begin{scope}[shift={(\textwidth*0.125,0)},scale=1]
					\draw[thick] (-1,0.125) -- (-1, -2.375) --
					node[below,scale=2/3] {$z_1z_2z_3z_5\cdots z_{10}\otimes z_4$}
					(1,-2.375) -- (1,0.125);
					\foreach \y in {0,...,9} {
						\pgfmathtruncatemacro{\nodename}{\y+1}
						\pgfmathtruncatemacro{\sd}{\sidez[\y]}
						\pgfmathparse{\palette[\colours[\y]]}
						\def\clr{\pgfmathresult}
						\node (ball\nodename) [draw, shade, circle, ball color=\clr, inner sep=0.07cm*2/3] at (\sd, -\y*1/4) {};
						\ifthenelse{\sd=1}{\node[scale=2/3,right] at (\sd, -\y*0.25) {\nodename}}
						{\node[scale=2/3,left] at (\sd, -\y*0.25) {\nodename}};
					}
					\draw (1.1, -0.875) -- (1, -0.875) to [in=270,out=180] (.8, -0.75) to [in=180,out=90] (1, -0.625) -- (1.1, -0.625);
				\end{scope}
				\begin{scope}[shift={(\textwidth*0.375,0)},scale=1]
					\draw[thick] (-1,0.125) -- (-1, -2.375) --
					node[below,scale=2/3] {$-z_1\cdots z_{10}\otimes 1$}
					(1,-2.375) -- (1,0.125);
					\foreach \y in {0,...,9} {
						\pgfmathtruncatemacro{\nodename}{\y+1}
						\pgfmathtruncatemacro{\sd}{\sidez[\y]}
						\pgfmathparse{\palette[\colours[\y]]}
						\def\clr{\pgfmathresult}
						\node (ball\nodename) [draw, shade, circle, ball color=\clr, inner sep=0.07cm*2/3] at (\sd, -\y*1/4) {};
						\ifthenelse{\sd=1}{\node[scale=2/3,right] at (\sd, -\y*0.25) {\nodename}}
						{\node[scale=2/3,left] at (\sd, -\y*0.25) {\nodename}};
					}
					\draw (1.1, -0.875) -- (0.8, -0.875);
				\end{scope}
			\end{scope}
	\end{tikzpicture}\]
\end{example}

\begin{theorem}
	Let the notation be as above, and suppose $\I = \set{1,2}$.
	Then $(\A_\ell^{(1)}, \A_r^{(1)})$ and $(\A_\ell^{(2)}, \A_r^{(2)})$ are bi-free if and only if $(\varphi\otimes\varphi)\circ\taur_{1} \equiv 0$.
\end{theorem}

\begin{proof}
	Suppose first that bi-freeness holds.
	Note that for $\lambda \in \C$, $\taur_1(\lambda) = 0$, so it suffices to check the condition on products $z_1\cdots z_n$ with $z_i \in \A_{\chi(i)}^{(\gem(i))}$ and each maximal $\gem$-monochromatic $\chi$-interval centred, since an arbitrary term may be written as a sum of such terms.
	However, by Lemma~\ref{lem:taur} we know that each term in $\taur_1(z_1\cdots z_n)$ is a tensor product with zero or more centred $\chi$-intervals occurring on each side of the tensor. The vaccine condition from bi-freeness then tells us that $\varphi\otimes\varphi$ of such a term is $0$, and so $(\varphi\otimes\varphi)\circ\taur_1(z_1\cdots z_n) = 0$.

	Now, suppose that bi-freeness fails, and let $z_1, \ldots, z_n$ be an example of the failure of vaccine with a minimum number of terms.
	Then the only terms which possibly fail to vanish under $\varphi\otimes\varphi$ from $\taur_1$ are those of the form $z_1\cdots z_n\otimes 1$ or $1\otimes z_1\cdots z_n$ (the rest being ones to which vaccine should apply, which are of shorter length and so not counterexamples by minimality).
	The term $z_1\cdots z_n\otimes1$ occurs once per $1$-coloured $\chi$-interval with negative sign, while $1\otimes z_1\cdots z_n$ occurs once with positive sign if the $\chi$-first and $\chi$-last variables are both in $\A^{(1)}$, and not at all otherwise; let $k$ be the number of $1$-coloured $\chi$-intervals, and $d = 1$ if the $\chi$-first and $\chi$-last variables are in $\A^{(1)}$, with $d = 0$ otherwise.
	Hence $\varphi\otimes\varphi(\taur_1(z_1\cdots z_n)) = (d-k)\varphi(z_1\cdots z_n) \neq 0$ unless $k = d$; but if $k = d$ either there is one $1$-interval which is $[n]$, or there are no $1$-intervals; in either case, $\gem$ is constant and so $z_1\cdots z_n$ cannot actually be a counterexample of vaccine.
\end{proof}



\subsection{Bi-free unitary Brownian motion.}
We are now ready to introduce a notion of bi-free unitary Brownian motion.

\begin{definition}
	A pair of free stochastic processes $(U_\ell(t), U_r(t))_{t\geq0}$ is a \emph{bi-freely liberating unitary Brownian motion} (or \emph{bi-flu Brownian motion}) if:
	\begin{itemize}
		\item the multiplicative increments are bi-free: if $0\leq t_1 < \cdots < t_n$, then the family of pairs of faces $((U_\ell^*(t_\iota)U_\ell(t_{\iota+1}), U_r(t_{\iota+1})U_r^*(t_\iota))_{\iota=1}^{n-1}$ is bi-free;
		\item $(U_\ell(t))_{t\geq0}$ and $(U^*_r(t))_{t\geq0}$ are each free unitary Brownian motions, and for all $t > 0$ the $*$-distribution of the pair $(U_\ell(t), U_r(t))$ matches that of $(U_\ell(t), U_\ell^*(t))$; and
		\item the distribution is stationary: the moments of $(U_\ell^*(s)U_\ell(t), U_r(t)U_r^*(s))$ depend only on $t-s$.
	\end{itemize}
\end{definition}
We have qualified this as a liberating unitary Brownian motion because it asymptotically introduces bi-free independence without modifying the distributions of the faces being liberated; we reserve the possibility of using ``bi-free unitary Brownian motion'' more broadly, such as for processes with different covariance between the left and right faces.

We will show that once again a bi-flu Brownian motion may be realized from an additive Brownian motion.
\begin{lemma}
	Suppose that $(S_\ell(t))_{t\geq0}$ is a free Brownian motion in a tracial von Neumann algebra $(M, \tau)$, and $J : L^2(M) \to L^2(M)$ is the Tomita operator defined on $M$ by $J(x) = x^*$ and extended continuously to $L^2(M)$.
	Let $S_r(t) = JS_\ell(t)J \in M'$.
	Then if $(U_\ell(t), U_r(t))$ are solutions to the stochastic differential equations
	$$dU_\ell(t) = iU_\ell(t)\,dS_\ell(t) - \frac12 U_\ell(t)\,dt \qquad\text{and}\qquad dU_r(t) = -iU_r(t)\,dS_r(t) - \frac12 U_r(t)\,dt,$$
	with initial conditions $U_\ell(0) = 1 = U_r(0)$, the pair $(U_\ell(t), U_r(t))$ is a bi-flu Brownian motion.
	Moreover, $(U_\ell(t), U_r(t))$ converges in distribution as $t \to \infty$ to a Haar pair of unitaries.
\end{lemma}

\begin{proof}
	We find immediately that $U_\ell(t)$ is a unitary free Brownian motion.
	Note that integrating a stochastic process $\omega_t\sharp dX_t$ comes down to finding a limit in $L^2(\A)$ of approximations of the form $\sum \theta_{t_k}(x_{t_k}-x_{t_k-1})\phi_{t_k}$, where $\sum \theta_{t_k}\otimes \phi_{t_k}$ approximates $\omega_t$.
	It follows that $d(JX_t^*J) = J(dX_t)^*J$, and in particular, $JdS_r(t)J = dS_\ell(t)$.
	Conjugating the equation for $dU_r(t)$ above, we find
	$$
	d(JU_r(t)J)
	= i\paren{JU_r(t)J}JdS_r(t)J - \frac12\paren{JU_r(t)J}\,dt 
	= i\paren{JU_r(t)J}dS_\ell(t) - \frac12\paren{JU_r(t)J}\,dt.
	$$
	Thus $JU_r(t)J$ satisfies the same differential equation as $U_\ell$, whence the two are equal.
	We conclude that $U_r(t)$ corresponds to right multiplication in the standard representation on $L^2(M)$ by $U_\ell^*(t)$.
	The remaining properties of bi-flu Brownian motion now follow readily from the free properties possessed by $(U_\ell(t))_{t\geq0}$, using, essentially, the techniques of Theorem~\ref{thm:bipartitefreetobifree}.
\end{proof}

We find that conjugating by bi-flu Brownian motion leads to bi-freeness as $t\to\infty$, much like in the free case, and this allows to think of this as a sort of bi-free liberation process.
A strange consequence is the following: suppose that $X, Y \in L^\infty(\Omega, \mu) \subset \A$ are classical random variables, and $((U_\ell(t), U_r(t))_{t\geq 0}$ a bi-flu Brownian motion in $\A$, bi-free from $(X, Y)$.
Then $X$ commutes in distribution with $Y$, $U_r(t)$, and $U_r^*(t)$, so in particular, $X$ and $U_r(t)YU_r^*(t)$ become independent as $t\to\infty$ while always generating a commutative probability space.
One finds that
\begin{align*}
	\varphi(f(X)U_r(t)g(Y)U_r^*(t))
	&= \varphi(f(X)g(Y))\varphi(U_r(t))\varphi(U_r^*(t)) \\
	&\qquad+ \varphi(f(X))\varphi(g(Y))\paren{1-\varphi(U_r(t))\varphi(U_r^*(t))} \\
	&= \varphi(f(X)g(Y))e^{-t} + \varphi(f(X))\varphi(g(Y))\paren{1-e^{-t}}.
\end{align*}

We will demonstrate a connection between liberation and the map $\taur$, but first we need a bi-free version of Proposition~\ref{prop:approxubm}.

\begin{lemma}
	\label{lem:ubmest}
	Suppose $(\A_\ell, \A_r)$ is a pair of faces in $\A$ and $\paren{U_\ell(t), U_r(t)}$ is a bi-flu Brownian motion, bi-free from $(\A_\ell, \A_r)$.
	Suppose further that $(S_\ell, S_r)$ is a pair of semicircular variables with covariance matrix containing a $1$ in every entry, also bi-free from $(\A_\ell, \A_r)$.
	Let $\chi : [n] \to \set{\ell, r}$, and for $1 \leq j \leq n$, take $a_j \in \A_{\chi_j}$ and $\alpha_j \in \set{1, 0, -1}$.
	Define $\psi : [n] \to \set{1, -1}$ by $\psi(j) = \alpha_j$ if $\chi(j) = \ell$, and $\psi(j) = -\alpha_j$ otherwise.
	Then we have
	$$\varphi\paren{\prod_{1\leq j \leq n}^{\rightarrow} a_jU_{\chi(j)}(t)^{\alpha_j}}
	= \varphi\paren{\prod_{1\leq j \leq n}^{\rightarrow} a_j\paren{\paren{1-\abs{\alpha_j}\frac{t}{2}}+i\psi(j)\sqrt{t}S_{\chi(j)}}} + \mathcal{O}\paren{t^2},$$
\end{lemma}

Essentially, this lemma tells us that the pair $(U_\ell(t), U_r(t))$ behaves in $*$-distribution to order $t$ the same as the pair $\paren{1-\frac{t}2 + i\sqrt{t}S_\ell, 1-\frac{t}2 - i\sqrt{t}S_r}$.

\begin{proof}
	We proceed along the same lines as in the proof of Proposition~\ref{prop:approxubm}.
	Let $I = \set{j : \alpha_j \neq 0}$, and write $m := \abs{I}$.
	Since the $*$-distribution of $(U_\ell(t), U_r(t))$ is the same as that of $(U_\ell(t), U_\ell^*(t))$, one can check that for any sequence $j_1 < \ldots < j_k$ of terms in $I$,
	$$\varphi\paren{(U_{\chi(j_1)}(t)^{\alpha_{j_1}}-e^{-t/2})\cdots(U_{\chi(j_k)}(t)^{\alpha_{j_k}}-e^{-t/2})}
	= -\delta_{k=2}\psi(j_1)\psi(j_2)t + \bigO{t^2}.$$
	This follows from the fact that the same is true in the free case, which was used in the original proof of Proposition~\ref{prop:approxubm} (\emph{cf.} \cite{Voiculescu1999101}).

	Now for each $j \in I$, we rewrite $U_{\chi(j)}(t)^{\alpha_j}$ as $\paren{U_{\chi(j)}(t)^{\alpha_j} - e^{-t/2}} + e^{-t/2}$, and expand the product.
	As we have the estimate $\norm{U_{\chi(j)}(t)^{\alpha_j} - e^{-t/2}} \leq K\sqrt{t}$, we find that only terms where at most three of these are chosen will contribute more than $\bigO{t^2}$.
	But by the above argument, terms with one or three such differences are $\bigO{t^2}$ under $\varphi$;
	then only terms which contribute are those where precisely zero or two $\paren{U_{\chi(j)}(t)^{\alpha_j} - e^{-t/2}}$ terms are chosen.
	Hence, if we abbreviate
	$$Z(t) := \paren{\sum_{\substack{1 \preceq_\chi p \prec_\chi q \preceq_\chi n\\p, q \in I}} \varphi\paren{a_1\cdots a_p (U_{\chi(p)}^{\alpha_p}-e^{-t/2}) a_{p+1}\cdots a_q (U_{\chi(q)}^{\alpha_q} - e^{-t/2}) a_{q+1}\cdots a_n}},$$
	we have
	\begin{align*}
		\varphi\paren{\prod_{1\leq j \leq n}^{\rightarrow} a_jU_{\chi(j)}(t)^{\alpha_j}}
		&= \varphi(a_1\cdots a_n)e^{-mt/2} + \bigO{t^2} - e^{-(m-2)t/2}Z(t) \\
		&= \varphi(a_1\cdots a_n)e^{-mt/2} + \bigO{t^2} \\
		&\qquad- t e^{-(m-2)t/2} \paren{\sum_{\substack{1 \preceq_\chi p \prec_\chi q \preceq_\chi n\\p, q \in I}} \varphi(a_{(p, q]_\chi})\varphi(a_{(p, q]_\chi^c})\psi(p)\psi(q)} \\
		&= \varphi(a_1\cdots a_n)\paren{1 - m\frac{t}2} + \bigO{t^2} \\
		&\qquad- t\paren{\sum_{\substack{1 \preceq_\chi p \prec_\chi q \preceq_\chi n\\p, q \in I}} \varphi(a_{(p, q]_\chi})\varphi(a_{(p, q]_\chi^c})\psi(p)\psi(q)}.
	\end{align*}
	Here the second equality may require some justification.
	One can verify that it is correct by considering the expansion in terms of cumulants; the terms corresponding to partitions with blocks of mixed colour or partitions that do not connect the $U$ terms both vanish, and we are left with all the bi-non crossing partitions which have the two joined. Summing over these, in turn, produces the product of the two moments claimed.

	Next we turn our attention to the right hand side of the equation.
	Notice that the pair $(S_\ell, S_r)$ has the same distribution as $(-S_\ell, -S_r)$ while both are bi-free from $(\A_\ell, \A_r)$, so replacing $\sqrt{t}$ by $-\sqrt{t}$ does not change the value and thus we are in fact working with a power series in $t$ rather than $\sqrt{t}$.
	Since the constant term is clearly correct, we need only establish that the $t$ term agrees.
	Contributions to the linear term come either from selecting a single $\frac{t}{2}$ in the product (together these contribute $-m\frac{t}{2}\varphi(a_1\cdots a_n)$) or from selecting a pair indices to include the semicircular terms from.
	But now
	$$\varphi(a_1\cdots a_{p}(i\psi(p)\sqrt{t})S_{\chi(p)}a_{p+1}\cdots a_{q} (i\psi(q)\sqrt{t})S_{\chi(q)} a_{q+1}\cdots a_n)
	= -t\psi(p)\psi(q)\varphi(a_{(p, q]_\chi})\varphi(a_{(p, q]^c_\chi}).
	$$
	Summing over the terms from which semi-circular elements may be selected, which is to say those with indices coming from $I$, we see the two sides of the claimed equation agree at order $t$, also. 
\end{proof}

\begin{theorem}
	Suppose $(\A^{(\iota)}_\ell, \A^{(\iota)}_r)_{\iota \in \set{\makeaball{0}, \makeaball{1}}}$ are algebraically-free pairs of faces in a non-commutative probability space $(\A, \varphi)$, which is bi-free from the bi-flu Brownian motion $(U_\ell(t), U_r(t))$.
	Given $\chi : [n]$, $\gem : [n]\to\set{\makeaball{0}, \makeaball{1}}$, and $x_i \in \A_{\chi(i)}$, set
	$$z_i{(t)} = \left\{\begin{array}{l@{\emph{ if }}l} % emph to prevent italics...
			x_i & \e(i) = \makeaball{0}\\
			U_{\chi(i)}(t) x_i U_{\chi(i)}^*(t) & \e(i) = \makeaball{1}.
	\end{array}\right.$$
	Then we have the following estimate:
	$$
	\varphi(z_1{(t)}\cdots z_n{(t)}) = \varphi(x_1\cdots x_n) + t\varphi\otimes\varphi\paren{\taur_{\makeaball{1}}(x_1\cdots x_n)} + \bigO{t^2}.
	$$
\end{theorem}

\begin{proof}
	We first apply Lemma~\ref{lem:ubmest} to replace $U_\ell(t)^{\pm1}$ by $1-\frac{t}{2} \pm i\sqrt{t}S_\ell$ and $U_r(t)^{\pm1}$ by $1-\frac{t}{2}\mp i\sqrt{t}S_r$, for some $(S_\ell, S_r)$ bi-free from $(\A_\ell, \A_r)$ as in Lemma~\ref{lem:ubmest}.
	Again, as the distribution of $(S_\ell, S_r)$ matches that of $(-S_\ell, -S_r)$, we find that we are dealing with a power series in $t$; further, it is evident that the constant term is correct.
	We therefore consider contributions to the linear term.

	However, note that these precisely correspond to the terms in the definition of $\taur_{\makeaball{1}}$.
	Indeed, we notice that when $i \prec_\chi j$ with $\e(i) = \e(j) = \makeaball{1}$, selecting the $S$ terms on either side of $x_i$ and $x_j$ contribute a total of
	$$t\paren{\varphi(x_{[i,j]_\chi^c})\varphi(x_{[i,j]_\chi}) - \varphi(x_{[i,j)_\chi^c})\varphi(x_{[i,j)_\chi})
	- \varphi(x_{(i,j]_\chi^c})\varphi(x_{(i,j]_\chi}) + \varphi(x_{(i,j)_\chi^c})\varphi(x_{(i,j)_\chi})}.$$
	The signs occur because the signs of $S$'s $\chi$-before their respective elements, or $\chi$-after, always match.
	This accounts for all the contributions coming from selecting two semicircular variables when expanding the product; what's left are the terms corresponding to selecting a $-\frac{t}{2}$ term, so each $x_i$ coming from \makeaball{1} winds up contributing $-t\varphi(x_1\cdots x_n)$ in total.
	Yet this precisely matches the contribution to $\taur_{\makeaball{1}}$ corresponding to selecting the empty terms with $i = j$.
	We conclude that the linear term in $\varphi(z_1(t)\cdots z_n(t))$ is precisely $t\varphi\otimes\varphi\paren{\taur_{\makeaball{1}}(x_1\cdots x_n)}$.
\end{proof}

In \cite{Voiculescu1999101}, Voiculescu used the free liberation process to define the liberation gradient and a mutual non-microstates free entropy.
Thus our approach here may be seen as taking the first steps towards a non-microstates theory of bi-free entropy.
\renewcommand{\e}{\iota}


\bibliography {/Users/ilc/Documents/Papers/biblio}
\bibliographystyle {thesis}

\end {document}

