\documentclass[11pt]{article}
\usepackage[margin=1in]{geometry}

\usepackage{fancyhdr}
\pagestyle{fancy}

%\usepackage{mathrsfs}

\usepackage{setspace}

\doublespacing
\rhead{Anton Bobkov}

\lhead{Personal Statement of Career Goals}




\begin{document}
\section*{Personal statement}

P-adic numbers are a simple, yet a very deep construction. They were only discovered a hundred years ago, but could have been studied in classical mathematics when number theory was just forming. Their construction is simple enough to explain at the undergraduate level, yet has a very rich number theoretic structure. Normally the real numbers are constructed by first taking rational numbers in decimal form and allowing infinite decimal sequences after the decimal point. 
Letting decimals be infinite before the decimal point yields a well behaved mathematical object as well, but with a drastically different behavior from real numbers, now depending on the base in which the decimals were written. 
When the base is a prime number p, this constructs p-adic numbers. These were first studied exclusively within number theory, but later
			 found applications in other areas of math, 
			physics, and computer science. My research 
			will allow 
			for a finer understanding of the finite structure 
		of polynomially definable sets in p-adic numbers. 
			In my career as an educator I hope to increase exposure to this elegant and rich construction for students both inside and outside of mathematics.
		
	
My research lies in the area of model theory, a branch of formal logic. Model theory began with G\"odel and Malcev in the 1930s, but first matured as a subject in the work of Abraham Robinson, Tarski, Vaught, and others in the 1950s. Model theory studies sets definable by first order formulas in a variety of mathematical objects. Restricting to subsets definable by simple formulas gives access to an array of powerful techniques such as indiscernible sequences and nonstandard extensions. These allow insights not otherwise accessible by classical methods. Nonstandard real numbers, for example, formalize the notion of infinitesimals. Model theory is an extremely flexible field with applications in many areas of mathematics including algebra, analysis, geometry, number theory, and combinatorics as well as some applications to computer science and quantum mechanics. In my career as a mathematician I hope to expose researchers in other fields to model theoretic methods allowing them to explore alternative approaches to classical mathematical objects.

My research concentrates on the concept of VC-density, a recent notion of rank in NIP theories. The study of a structure in model theory usually starts with quantifier elimination, followed by a finer analysis of definable functions and interpretability. The study of VC-density goes one step further, looking at a structure of the asymptotic growth of finite definable families. In the most geometric examples, VC-density coincides with the natural notion of dimension. However, no geometric structure is required for the definition of VC-density, thus we can get some notion of geometric dimension for families of sets given without any geometric context! In my career as a researcher I hope to further explore this notion and introduce other model theorists to its applications.

To summarize, I intend to follow a career path in academia, balancing my teaching with my research. An important part of being a mathematician is communicating and disseminating mathematical knowledge. I have enjoyed my work as a teaching assistant, and look forward to working with students at all stages of their mathematical education. Another equally important part is developing and progressing mathematical knowledge. My work in model theory has been a great motivation for me, and I plan to stay an active researcher for the rest of my mathematical career.
\end{document}

