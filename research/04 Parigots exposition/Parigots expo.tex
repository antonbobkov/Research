\documentclass{amsart}

\usepackage{../AMC_style}	
\usepackage{../Research}

\usepackage{tikz}

\DeclareMathOperator{\TT}{\boldface T}
\DeclareMathOperator{\A}{\boldface A}
\DeclareMathOperator{\B}{\boldface B}
\DeclareMathOperator{\PR}{P}

\begin{document}

\title{NIP and stable trees}
\author{Anton Bobkov}
\email{bobkov@math.ucla.edu}
%more info

\maketitle

\section{Definitions}

\begin{Definition}\ 
	\begin{enumerate}
		\item Given a set $A$ denote its \emph{initial closure} 
		\begin{align*}
			I(A) = \{b \mid \exists a \in A \ b \leq a\}
		\end{align*}
		\item \emph{Initial segements} are sets that are closed under initial closure.
		\item Given set $A$ define its \emph{initial part} (which is an initial segment)
		\begin{align*}
			S(A) = \{b \mid b < A\} = \{b \mid \forall a \in A \ b < a\}
		\end{align*}
		\item Given initial segment $S$ define its \emph{final part}
		\begin{align*}
			F(S) = \{b \mid b > A\} = \{b \mid \forall a \in A \ b > a\}
		\end{align*}
		\item Given initial segment $S$ define its \emph{trunk part}
		\begin{align*}
			T(S) = A - F(S)
		\end{align*}
	\end{enumerate}
\end{Definition}

\end{document}