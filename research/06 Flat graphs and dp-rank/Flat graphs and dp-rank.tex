\documentclass{amsart}

\usepackage{../AMC_style}	
\usepackage{../Research}


\begin{document}

\title{Superflat graphs are dp-minimal}
\author{Anton Bobkov}
\email{bobkov@math.ucla.edu}

\begin{abstract}
	We show that the theory of superflat graphs is dp-minimal.
\end{abstract}

\maketitle

\section{Preliminaries}

Superflat graphs in model theoretic context were first introduced in \cite{stable_graphs} as a natural class of stable graphs.

We work with an infinite graph $G$ and a subset of vertices $V \subset V(G)$. Say that $V$ is $n$-connected if there aren't a set of $n-1$ vertices removing which disconnects every pair of vertices in $V$. Connectivity of $V$ is the smallest $n$ such that $V$ are $n$-connected.

\begin{Definition}
	Suppose $V \subset V(G)$ has finite connectivity $n+1$. Let connectivity hull of $V$ to be union of all $n$-point sets that disconnect it.
\end{Definition}

\begin{Note}
	If $V$ has finite connectivity $n+1$, then having finite connectivity hull is the same as having finitely many $n$-point sets that disconnect it.
\end{Note}

Connectivity of graphs is well described by Megner's Theorem. We will need an infinite version of that theorem which we can find in \cite{infinite_megner}

\begin{Theorem}
	Let $V$ be a subset of a graph $G$ with connectivity $n$. Then there exists a set of $n$ disjoint paths from $V$ into itself.
\end{Theorem}

\begin{Corollary}
	With assumptions as above, connectivity hull of $V$ is finite.
\end{Corollary}

\begin{proof}
	All the disconnecting sets have to have exactly one vertex in each of those paths. 
\end{proof}

\section{Application to indiscernible sequences}

In this section we work in a flat graph. It is stable so all the indiscernible sequences are totally indiscernible. Also note that by indiscernibility all pairwise distances between points are the same.

Superflat graphs are the graphs that for every $m \in \N$ there is $n \in \N$ such that avoid $K^m_n$ is avoided, where $K^m_n$ is $m$-subdvision of complete graph on $n$ vertices. Equivalent characterization given in \cite{stable_graphs}, Theorem 2

\begin{Theorem}
	In a flat Graphs for every $n \in \N$ and an infinite set $A$ of vertices, there exists a finite $B$ and infinite $A' \subseteq A$ such that for all $x,y \in A'$ we have $d_{B}(x, y) > n$.
\end{Theorem}

Roughly, the intuition is that from every infinite set we can extract a sparse infinite subset (after throwing away finitely many nodes).

We will need a refined notion of connectivity for the following argument to work. Suppose we have two points $a,b$ distance $n$ apart. Denote $P(a,b)$ union of all paths of length $n$ going from $a$ to $b$. If we have a collection of vertices $V$ such that every two have distance $n$ between them, denote

Let $V \subset V(G)$. Denote $P_n(V)$ a union of all paths of length $\leq n$ between points of $V$. It is a subgraph of $G$.

\begin{align*}
	P(V) = \bigcup_{a \neq b \in V} P(a,b)
\end{align*}

\begin{Lemma}
	Let $(a_i)_{i \in I}$ be a countable indiscernible sequence over $A$. Fix $n \in \N$. There exists a finite set $B$ such that
	\begin{align*}
		\forall i \neq j \ d_B(a_i, a_j) > n
	\end{align*}
\end{Lemma}

\begin{proof}
	By a flatness result we can find an infinite $J \subset I$ and a finite set $B'$ such that each pair from $(a_j)_{j \in J}$ have distance $>n$ over $B'$. Using total indiscernibility we have an automorphism sending $(a_j)_{j \in J}$ to $(a_i)_{i \in I}$. Image of $B'$ under this automorphism is the required set $B$.
\end{proof}

In other words, $B$ disconnects $P_n(\{a_i\})$. This shows that $\{a_i\}$ has finite connectivity in $P(\{a_i\})$. Applying lemma from last section we obtain that connectivity hull of $\{a_i\}$ in $P_n(\{a_i\})$ is finite.

\begin{Lemma}
	Connectivity hull of $\{a_i\}$ in $P_n(\{a_i\})$ is $\{a_i\}$-definable as a subset of $G$.
\end{Lemma}

\begin{Definition}
	Given a graph $G$ and $V \subset V(G)$ define $H(G, V)$ to be connectivity hull of $V$ in $G$.
\end{Definition}

\begin{Note}
	Given a finite $V$ we have $H(P_n(V), V)$ is definable.
\end{Note}

\begin{proof}
	Consider finite parts of the sequence $I_i = \{a_1, a_2, \ldots, a_i\}$. We study $H_i = H(P_n(I_i), I_i)$ as a function of $i$ as approximations of the hull in question. We have the following properties
	\begin{align*}
		 \forall i \ H(P_n(I_i), I_i) &\subseteq H(P_n(I), I_i) \\
		 \forall i \leq j \ H(P_n(I), I_i) &\supseteq H(P_n(I), I_j)
	\end{align*}
		Eventually $H(P_n(I_i))$ contains $n$ disjoint paths for the whole graph, thus stabilizes at $H(P_n(I), I)$. This shows that for large enough $i>N$ we have $H_i = H_{i+m}$. By symmetry of indiscernible sequence we have that any subset of size $N$ defines the connectivity hull.
\end{proof}

\begin{Lemma}
	$I$ is indiscernible over the $A \cup H(P_n(I), I)$.
\end{Lemma}

\begin{proof}
	Denote the hull by $H$. Fix an $A$-formula $\phi(x,y)$. Consider a collection of traces $\phi(\vec a, H^\{|y|\})$ for $\vec a \in I^{|x|}$. As $H$ is $I$ definable those are either all distinct or all the same. Finiteness of $H$ forces latter. This shows indiscernability.
\end{proof}

\begin{Corollary}
	Let $(a_i)_{i \in I}$ be a countable indiscernible sequence over $A$. Then there is a countable $B$ such that  $(a_i)$ is indiscernible over $A \cup B$ and
	\begin{align*}
		\forall i \neq j \ d_B(a_i, a_j) = \infty
	\end{align*}
\end{Corollary}

\begin{proof}
	Let $B_i = H(P_i(I), I)$. Successive applications of previous lemma yield the appropriate set $B = \bigcup B_i$.
\end{proof}

That is every indiscernible sequence can be upgraded to have infinite distance over its parameter set.

\section{Superflat graphs are dp-minimal}

\begin{Lemma}
	Suppose $x \equiv_A y$ and $d_A(x, c) = d_A(y, c) = \infty$. Then $x \equiv_{Ac} y$
\end{Lemma}

\begin{proof}
	Define an equivalence relation $G - A$. Two points $p, q$ are equivalent if $d_A(p,q)$ is finite. There is an automorphism $f$ of $G$ fixing $A$ sending $x$ to $y$. Denote by $X$ and $Y$ equivalence classes of $x$ and $y$ respectively. It's easy to see that $f(X) = Y$. Define the following function
	\begin{align*}
		&g = f \text { on } X \\
		&g = f^{-1} \text { on } Y \\
		&\text{identity otherwise}
	\end{align*}
	It is easy to see that $g$ is an automorphism fixing $Ac$ that sends $x$ to $y$.
\end{proof}

\begin{Theorem}
	Let $G$ be a flat graph with $(a_i)_{i\in\Q}$ indiscernible over $A$ and $b \in G$. There exists $c \in \Q$ such that all $(a_i)_{i\in\{\Q - c\}}$ have the same type over $Ab$.
\end{Theorem}

\begin{proof}
	Find $B \supseteq A$ such that $(a_i)$ is indiscernible over $B$ and has infinite distance over $B$. All the elements of the indiscernible sequence fall into distinct equivalence classes. $b$ can be in at most one of them. Exclude that element from the sequence. Remaining sequence elements are all infinitely far away from $b$. By previous lemma we have that elements of indiscernible sequence all have the same type over $Bb$.
\end{proof}

But this is exactly what it means to be dp-minimal, as given, say, in \cite{simon_dp_minimal} Lemma 1.4.4

\begin{Corollary}
	Flat graphs are dp-minimal.
\end{Corollary}


\begin{thebibliography}{9}

\bibitem{diestel}
	Reinhard Diestel. \textit{Graph Theory}, volume 173 of \textit{Grad. Texts in Math.} Springer, 2005.

\bibitem{infinite_megner}
	Aharoni, Ron and Berger, Eli (2009). "Menger's Theorem for infinite graphs". \textit{Inventiones Mathematicae} 176: 1–62

\bibitem{stable_graphs}
	Klaus-Peter Podewski and Martin Ziegler. Stable graphs. \textit{Fund. Math.}, 100:101-107, 1978.
	
\bibitem{simon_dp_minimal}
	P. Simon, \textit{On dp-minimal ordered structures}, J. Symbolic Logic 76 (2011), no. 2, 448–460.


\end{thebibliography}

\end{document}
