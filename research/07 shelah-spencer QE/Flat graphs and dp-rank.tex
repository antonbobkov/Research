\documentclass{amsart}

\usepackage{../AMC_style}	
\usepackage{../Research}

\DeclareMathOperator{\A}{\mathcal A}
\DeclareMathOperator{\B}{\mathcal B}
\renewcommand{\C}{\mathcal C}
\DeclareMathOperator{\D}{\mathcal D}

\DeclareMathOperator{\K}{\boldface K_\alpha}

\DeclareMathOperator{\acl}{acl}

\begin{document}

\title{Quantifier elimination in Shelah-Spencer graphs}
\author{Anton Bobkov}
\email{bobkov@math.ucla.edu}

\begin{abstract}
	We simplify \cite{Laskowski} proof of quantifier elimination in Shelah-Spencer graphs.
\end{abstract}

\maketitle

\section{Introduction}

Laskowski's paper \cite{Laskowski} provides a combinatorial proof of quantifier elimination in Shelah-Spencer graphs. Here we provide a simplification of the proof using only maximal chains and avoiding technical lemmas of sections 3 and 4.

\section{Proof}

\begin{Definition}
	Fix $\C \in \K$, $\Phi, \Gamma$ finite subsets of $\K$, and $m \in \omega$ such that for each $\B \in \Phi$ or $\B \in \Gamma$ we have $\A \subseteq \B$ and $|B \backslash A| < m$.
\end{Definition}

\begin{thebibliography}{9}

\bibitem{stable_graphs}
	Klaus-Peter Podewski and Martin Ziegler. Stable graphs. \textit{Fund. Math.}, 100:101-107, 1978.

\bibitem{infinite_megner}
	Aharoni, Ron and Berger, Eli (2009). "Menger's Theorem for infinite graphs". \textit{Inventiones Mathematicae} 176: 1–62
	
\bibitem{simon_dp_minimal}
	P. Simon, \textit{On dp-minimal ordered structures}, J. Symbolic Logic 76 (2011), no. 2, 448–460.

%\bibitem{diestel}
%	Reinhard Diestel. \textit{Graph Theory}, volume 173 of \textit{Grad. Texts in Math.} Springer, 2005.

\end{thebibliography}

\end{document}
