\section{VC-dimension and VC-density}
Throughout this section we work with a collection $\F$ of subsets of an infinite set $X$.
We call the pair $(X, \F)$ a \defn{set system}.

\begin{Definition} \ 
  \begin{itemize} 
  \item Given a subset $A$ of $X$, we define the set system $(A, A \cap \F)$
    where
    \begin{align*}
      A \cap \F = \curly{A \cap F \mid F\in \F}.      
    \end{align*}
  \item For $A \subseteq X$ we say that $\F$ \defn{shatters} $A$ if $A \cap \F = \PP(A)$ (the power set of $A$).
  \end{itemize}    
\end{Definition}  

\begin{Definition}
  We say $(X, \F)$ has \defn{VC-dimension} $n$ if the largest subset of $X$ shattered by $\F$ is of size $n$.
  If $\F$ shatters arbitrarily large subsets of $X$, we say that $(X, \F)$ has infinite VC-dimension.
  We denote the VC-dimension of $(X, \F)$ by $\VC(X, \F)$.
\end{Definition}  

\begin{Note}
  We may drop $X$ from the notation $\VC(X, \F)$, as the VC-dimension doesn't depend on the base set and is determined by $(\bigcup \F, \F)$.
\end{Note}
Set systems of finite VC-dimension tend to have good combinatorial properties,
and we consider set systems with infinite VC-dimension to be poorly behaved.

Another natural combinatorial notion is that of the dual system of a set system:
\begin{Definition}
  For $a \in X$ define $X_a = \curly{F \in \F \mid a \in F}$.
  Let $\F^* = \curly{X_a \mid a \in X}$.
  We call $(\F, \F^*)$ the \defn{dual system} of $(X, \F)$.
  The VC-dimension of the dual system of $(X, \F)$ is referred to as the \defn{dual VC-dimension} of $(X, \F)$ and denoted by $\VC^*(\F)$.
  (As before, this notion doesn't depend on $X$.)
\end{Definition}  

\begin{Lemma} [see 2.13b in \cite{ash7}]
  A set system $(X, \F)$ has finite VC-dimension if and only if its dual system has finite VC-dimension.
  More precisely
  \begin{align*}
    \VC^*(\F) \leq 2^{1+\VC(\F)}.
  \end{align*}
\end{Lemma}

For a more refined notion of complexity of $(X, \F)$ we look at the traces of our family on finite sets:
\begin{Definition}
  Define the \defn{shatter function} $\pi_\F \colon \N \arr \N$ of $\F$ and the \defn{dual shatter function} $\pi^*_\F \colon \N \arr \N$ of $\F$ by 
  \begin{align*}
    \pi_\F(n) &= \max \curly{|A \cap \F| \mid A \subseteq X \text{ and } |A| = n} \\
    \pi^*_\F(n) &= \max \curly{\text{atoms($B$)} \mid B \subseteq \F, |B| = n}
  \end{align*}
  where atoms($B$) is the number of atoms in the boolean algebra of sets generated by $B$.
  Note that the dual shatter function is precisely the shatter function of the dual system: $\pi^*_\F = \pi_{\F^*}$.
\end{Definition}  

A simple upper bound is $\pi_\F(n) \leq 2^n$ (same for the dual).
If the VC-dimension of $\F$ is infinite then clearly $\pi_\F(n) = 2^n$ for all $n$. Conversely we have the following remarkable fact:
\begin{Theorem} [Sauer-Shelah '72, see \cite{sauer}, \cite{shelah}]
  If the set system $(X, \F)$ has finite VC-dimension $d$ then $\pi_\F(n) \leq {n \choose \leq d}$ for all $n$, where
  ${n \choose \leq d} = {n \choose d} + {n \choose d - 1} + \ldots + {n \choose 1}$.    
\end{Theorem}

Thus the systems with a finite VC-dimension are precisely the systems where the shatter function grows polynomially.
The VC-density of $\F$ quantifies the growth of the shatter function of $\F$: 
\begin{Definition}
  Define the \defn{VC-density} and \defn{dual VC-density} of $\F$ as
  \begin{align*}
    \vc(\F) &= \limsup_{n \to \infty}\frac{\log \pi_\F(n)}{\log n} \in \R^{\geq 0} \cup \curly{+\infty},\\
    \vc^*(\F) &= \limsup_{n \to \infty}\frac{\log \pi^*_\F(n)}{\log n}\in \R^{\geq 0} \cup \curly{+\infty}.
  \end{align*}
\end{Definition}

Generally speaking a shatter function that is bounded by a polynomial doesn't itself have to be a polynomial.
Proposition 4.12 in \cite{density} gives an example of a shatter function that grows like $n \log n$ (so it has VC-density $1$).

So far the notions that we have defined are purely combinatorial.
We now adapt VC-dimension and VC-density to the model theoretic context.

\begin{Definition}
  Work in a first-order structure $\MM$.
  Fix a finite collection of formulas $\Phi(x, y)$ in the language $\LL(M)$ of $\MM$.

  \begin{itemize}
  \item For $\phi(x, y) \in \LL(M)$ and $b \in M^{|y|}$ let 
    \begin{align*}
      \phi(M^{|x|}, b) = \{a \in M^{|x|} \mid \phi(a, b)\} \subseteq M^{|x|}.
    \end{align*}
  \item Let $\Phi(M^{|x|}, M^{|y|})= \{\phi(M^{|x|}, b) \mid \phi \in \Phi, b \in M^{|y|}\} \subseteq \PP(M^{|x|})$.
  \item Let $\F_\Phi = \Phi(M^{|x|}, M^{|y|})$, giving rise to a set system $(M^{|x|}, \F_\Phi)$.
  \item Define the \defn{VC-dimension} $\VC(\Phi)$ of $\Phi$ to be the VC-dimension of $(M^{|x|}, \F_\Phi)$, similarly for the dual.
  \item Define the \defn{VC-density} $\vc(\Phi)$ of $\Phi$ to be the VC-density of $(M^{|x|}, \F_\Phi)$, similarly for the dual.
  \end{itemize}

  We will also refer to the VC-density and VC-dimension of a single formula $\phi$
  viewing it as a one element collection $\Phi = \curly{\phi}$.
\end{Definition}

Counting atoms of a boolean algebra in a model theoretic setting corresponds to counting types,
so it is instructive to rewrite the shatter function in terms of types.

\begin{Definition} 
  \begin{align*}
    \pi^*_\Phi(n) &= \max \curly{\text{number of $\Phi$-types over $B$} \mid B \subseteq M, |B| = n}.
  \end{align*}
  Here a $\Phi$-type over $B$ is a maximal consistent collection of formulas of the form $\phi(x, b)$ or $\neg\phi(x, b)$
  where $\phi \in \Phi$ and $b \in B$.
\end{Definition}

The functions $\pi^*_{\Phi}$ and $\pi^*_{\F_\Phi}$ do not have to agree,
as one fixes the number of generators of a boolean algebra of sets and the other fixes the size of the parameter set.
However, as the following lemma demonstrates, they both give the same asymptotic definition of dual VC-density.

\begin{Lemma} \label{count_types}
  \begin{align*}
    \vc^*(\Phi) &= \text{degree of polynomial growth of $\pi^*_\Phi(n)$}  = \limsup_{n \to \infty}\frac{\log \pi^*_\Phi(n)}{\log n}.
  \end{align*}  
\end{Lemma}

\begin{proof}
  With a parameter set $B$ of size $n$, we get at most $|\Phi|n$ sets $\phi(M^{|x|}, b)$ with $\phi \in \Phi, b \in B$.
  We check that asymptotically it doesn't matter whether we look at growth of boolean algebra of sets generated by
  $n$ or by $|\Phi|n$ many sets.
  We have:
  \begin{align*}
    \pi^*_{\F_\Phi}\paren{n} \leq \pi^*_\Phi(n) \leq \pi^*_{\F_\Phi}\paren{|\Phi|n}.
  \end{align*}
  Hence:
  \begin{align*}
    &\vc^*(\Phi) \leq \limsup_{n \to \infty}\frac{\log \pi^*_\Phi(n)}{\log n} \leq \limsup_{n \to \infty}\frac{\log \pi^*_{\F_\Phi}\paren{|\Phi|n}}{\log n} = \\
    & = \limsup_{n \to \infty}\frac{\log \pi^*_{\F_\Phi}\paren{|\Phi|n}}{\log |\Phi|n} \frac{\log |\Phi|n}{\log n} =
      \limsup_{n \to \infty}\frac{\log \pi^*_{\F_\Phi}\paren{|\Phi|n}}{\log |\Phi|n} \leq \\
    &\leq \limsup_{n \to \infty}\frac{\log \pi^*_{\F_\Phi}\paren{n}}{\log n} = \vc^*(\Phi).
  \end{align*}
\end{proof} 

One can check that the shatter function and hence VC-dimension and VC-density of a formula are elementary notions,
so they only depend on the first-order theory of the structure $\MM$.

NIP theories are a natural context for studying VC-density.
In fact we can take the following as the definition of NIP:
\begin{Definition}
  Define $\phi$ to be NIP if it has finite VC-dimension in a theory $T$.
  A theory $T$ is NIP if all the formulas in $T$ are NIP.
\end{Definition}

In a general combinatorial context (for arbitrary set systems),
VC-density can be any real number in $0 \cup [1, \infty)$ (see \cite{ash8}).
Less is known if we restrict our attention to NIP theories.
Proposition 4.6 in \cite{density} gives examples of formulas that have non-integer rational VC-density in an NIP theory,
however it is open whether one can get an irrational VC-density in this model-theoretic setting.

Instead of working with a theory formula by formula, we can look for a uniform bound for all formulas:
\begin{Definition} \label{vc_fn_def}
  For a given NIP structure $\MM$, define the \defn{VC-density function}
  \begin{align*}
    \vc^\MM(n) &= \sup \{\vc^*(\phi(x, y)) \mid \phi \in \LL(M), |x| = n\} \\
             &= \sup \{\vc(\phi(x, y)) \mid \phi \in \LL(M), |y| = n\} \in \R^{\geq 0} \cup \curly{+\infty}.
  \end{align*}
\end{Definition}

As before this definition is elementary, so it only depends on the theory of $\MM$.
We omit the superscript $\MM$ if it is understood from the context.
One can easily check the following bounds:
\begin{Lemma} [Lemma 3.22 in \cite{density}] \label{vcone}
  We have $\vc(1) \geq 1$ and $\vc(n) \geq n\vc(1)$.  
\end{Lemma}

However, it is not known whether the second inequality can be strict or even just whether $\vc(1) < \infty$ implies $\vc(n) < \infty$.
