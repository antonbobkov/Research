\documentclass{amsart}

\usepackage{../AMC_style}	
\usepackage{../Research}
\usepackage{../Thm}

\usepackage{mathrsfs}
\usepackage{pgfpages} 
\usepackage{setspace}

\doublespacing
%% \usepackage[margin=.75in]{geometry}
%% \pgfpagesuselayout{2 on 1}

\renewcommand{\AA}{\mathscr A}
\newcommand{\BB}{\mathscr B}
\newcommand{\DD}{\mathscr D}
\newcommand{\II}{\mathscr I}
\newcommand{\GG}{\mathbb G}

\newcommand{\F}{\mathcal F}
\renewcommand{\LL}{\mathcal L}

\newcommand{\defn}{\underline}

\DeclareMathOperator{\diag}{diag}

\newcommand{\DB}{\mathbb D}
\newcommand{\ppp}{\partial}

\newcommand{\A}{A}
\newcommand{\B}{B}
\renewcommand{\C}{\mathcal C}
\newcommand{\D}{\mathcal D}
\renewcommand{\H}{\mathcal H}
\newcommand{\G}{\mathcal G}
\newcommand{\M}{\mathcal M}
\newcommand{\U}{\mathcal U}	
\newcommand{\X}{X}
\newcommand{\Y}{Y}

\newcommand{\K}{\boldface K_\alpha}
\renewcommand{\S}{S_\alpha}

\newcommand{\curly}[1]{\left\{#1\right\}}
\newcommand{\paren}[1]{\left(#1\right)}
\newcommand{\abs}[1]{\left|#1\right|}
\newcommand{\agl}[1]{\left\langle #1 \right\rangle}

\providecommand{\floor}[1]{\left \lfloor #1 \right \rfloor }

% \DeclareMathOperator{\dim}{dim}

\title{Some vc-density computations in Shelah-Spencer graphs}
\author{Anton Bobkov}
\email{bobkov@math.ucla.edu}

\begin{document}

\begin{abstract}
  We investigate vc-density in Shelah-Spencer graphs.
  We provide an upper bound on formula-by-formula basis and show that there isn't a uniform lower bound forcing the vc-function to be infinite.
\end{abstract}

\maketitle

%%%%%%%%%%%%%%%%%%%%%%%%%%%%%%%%%%%%%%%%%%%%%%%%%%%%%%%%%%%%%%%%%%%%%%%%%%%%%%%%%%%%%%%%%%%%%%%%%%%%%%%%%%%%%%%%% 

VC-density was studied in \cite{density} by Aschenbrenner, Dolich, Haskell, MacPherson, and Starchenko as a natural notion of dimension for NIP theories.
In a complete NIP theory $T$ we can define the vc-function

\begin{align*}
  \vc^T = \vc : \N \arr \R \cup \curly{\infty}
\end{align*}

where $\vc(n)$ measures the worst-case complexity of families of definable sets in an $n$-fold Cartesian power of the underlying set of a model of $T$
(see \ref{vc_fn_def} below for a precise definition of $\vc^T$).
The simplest possible behavior is $\vc(n) = n$ for all $n$. Theories with the property that $\vc(1) = 1$ are known to be dp-minimal, i.e., having the smallest possible dp-rank. It is not known whether there can be a dp-minimal theory which doesn't satisfy $\vc(n)=n$
(see \cite{density}, diagram on pg. 41).

In this paper, we investigate vc-density of definable sets in Shelah-Spencer graphs.
In our description of Shelah-Spencer graphs we follow closely the treatment in \cite{laskowski}.
A Shelah-Spencer graph is a limit of random structures $G(n, n^{-\alpha})$ for an irrational $\alpha \in (0,1)$.
$G(n, n^{-\alpha})$ is a random graph on $n$ vertices with edge probability $n^{-\alpha}$.

Our first result is that in Shelah-Spencer graphs
\begin{align*}
  \vc(n) = \infty
\end{align*}
which implies that they are not dp-minimal.
Our second result is providing an upper bound on a vc-density for a formula $\phi$
\begin{align*}
  \vc(\phi) \leq K(\phi) \frac{Y(\phi)}{\epsilon(\phi)}    
\end{align*}
where $K(\phi), Y(\phi), \epsilon(\phi)$ are paramters easily computable from the quantifier free form of $\phi$.

Chapter 1 introduces basic facts about VC-dimension and vc-density.
More can be found in \cite{density}.
Chapter 2 summarizes notation and basic facts concerning Shelah-Spencer graphs.
We direct the reader to \cite{laskowski} for a more in-depth treatment.
In chapter 3 we introduce some measure of dimension for quantifier free formulas as well as proving some elementary facts about it.
Chapter 4 computes a lower bound for vc-density to demonstrate that $\vc(n) = \infty$.
Chapter 5 computes an upper bound for vc-density on a formula-by-formula basis.

% This struc is axiomatized by $S_\alpha$.
% Our ambient model is $\GG$.
% Notations we use are $\dim(\A), \dim(\A/\B), \A \leq \B$ as well as notions of $N$-strong substructure, minimal extension, chain minimal extension, minimal pair, and $N$-strong closure.

\section{VC-dimension and VC-density}
Throughout this section we work with a collection $\F$ of subsets of an infinite set $X$.
We call the pair $(X, \F)$ a \defn{set system}.

\begin{Definition} \ 
  \begin{itemize} 
  \item Given a subset $A$ of $X$, we define the set system $(A, A \cap \F)$
    where $A \cap \F = \curly{A \cap F \mid F\in \F}$.
  \item For $A \subseteq X$ we say that $\F$ \defn{shatters} $A$ if $A \cap \F = \PP(A)$ (the power set of $A$).
  \end{itemize}    
\end{Definition}  

\begin{Definition}
  We say $(X, \F)$ has \defn{VC-dimension} $n$ if the largest subset of $X$ shattered by $\F$ is of size $n$.
  If $\F$ shatters arbitrarily large subsets of $X$, we say that $(X, \F)$ has infinite VC-dimension.
  We denote the VC-dimension of $(X, \F)$ by $\VC(X, \F)$.
\end{Definition}  

\begin{Note}
  We may drop $X$ from the notation $\VC(X, \F)$, as the VC-dimension doesn't depend on the base set and is determined by $(\bigcup \F, \F)$.
\end{Note}
Set systems of finite VC-dimension tend to have good combinatorial properties,
and we consider set systems with infinite VC-dimension to be poorly behaved.

Another natural combinatorial notion is that of the dual system of a set system:
\begin{Definition}
  For $a \in X$ define $X_a = \curly{F \in \F \mid a \in F}$.
  Let $\F^* = \curly{X_a \mid a \in X}$.
  We call $(\F, \F^*)$ the \defn{dual system} of $(X, \F)$.
  The VC-dimension of the dual system of $(X, \F)$ is referred to as the \defn{dual VC-dimension} of $(X, \F)$ and denoted by $\VC^*(\F)$.
  (As before, this notion doesn't depend on $X$.)
\end{Definition}  

\begin{Lemma} [see 2.13b in \cite{ash7}]
  A set system $(X, \F)$ has finite VC-dimension if and only if its dual system has finite VC-dimension.
  More precisely
  \begin{align*}
    \VC^*(\F) \leq 2^{1+\VC(\F)}.
  \end{align*}
\end{Lemma}

For a more refined notion of complexity of $(X, \F)$ we look at the traces of our family on finite sets:
\begin{Definition}
  Define the \defn{shatter function} $\pi_\F \colon \N \arr \N$ of $\F$ and the \defn{dual shatter function} $\pi^*_\F \colon \N \arr \N$ of $\F$ by 
  \begin{align*}
    \pi_\F(n) &= \max \curly{|A \cap \F| \mid A \subseteq X \text{ and } |A| = n} \\
    \pi^*_\F(n) &= \max \curly{\text{atoms($B$)} \mid B \subseteq \F, |B| = n}
  \end{align*}
  where atoms($B$) is the number of atoms in the boolean algebra of sets generated by $B$.
  Note that the dual shatter function is precisely the shatter function of the dual system: $\pi^*_\F = \pi_{\F^*}$.
\end{Definition}  

A simple upper bound is $\pi_\F(n) \leq 2^n$ (same for the dual).
If the VC-dimension of $\F$ is infinite then clearly $\pi_\F(n) = 2^n$ for all $n$. Conversely we have the following remarkable fact:
\begin{Theorem} [Sauer-Shelah '72, see \cite{sauer}, \cite{shelah}]
  If the set system $(X, \F)$ has finite VC-dimension $d$ then $\pi_\F(n) \leq {n \choose \leq d}$ for all $n$, where
  ${n \choose \leq d} = {n \choose d} + {n \choose d - 1} + \ldots + {n \choose 1}$.    
\end{Theorem}

Thus the systems with a finite VC-dimension are precisely the systems where the shatter function grows polynomially.
The VC-density of $\F$ quantifies the growth of the shatter function of $\F$: 
\begin{Definition}
  Define the \defn{VC-density} and \defn{dual VC-density} of $\F$ as
  \begin{align*}
    \vc(\F) &= \limsup_{n \to \infty}\frac{\log \pi_\F(n)}{\log n} \in \R^{\geq 0} \cup \curly{+\infty},\\
    \vc^*(\F) &= \limsup_{n \to \infty}\frac{\log \pi^*_\F(n)}{\log n}\in \R^{\geq 0} \cup \curly{+\infty}.
  \end{align*}
\end{Definition}

Generally speaking a shatter function that is bounded by a polynomial doesn't itself have to be a polynomial.
Proposition 4.12 in \cite{density} gives an example of a shatter function that grows like $n \log n$ (so it has VC-density $1$).

So far the notions that we have defined are purely combinatorial.
We now adapt VC-dimension and VC-density to the model theoretic context.

\begin{Definition}
  Work in a first-order structure $\MM$.
  Fix a finite collection of formulas $\Phi(x, y)$ in the language $\LL(M)$ of $\MM$.

  \begin{itemize}
  \item For $\phi(x, y) \in \LL(M)$ and $b \in M^{|y|}$ let 
    \begin{align*}
      \phi(M^{|x|}, b) = \{a \in M^{|x|} \mid \phi(a, b)\} \subseteq M^{|x|}.
    \end{align*}
  \item Let $\Phi(M^{|x|}, M^{|y|})= \{\phi(M^{|x|}, b) \mid \phi \in \Phi, b \in M^{|y|}\} \subseteq \PP(M^{|x|})$.
  \item Let $\F_\Phi = \Phi(M^{|x|}, M^{|y|})$, giving rise to a set system $(M^{|x|}, \F_\Phi)$.
  \item Define the \defn{VC-dimension} $\VC(\Phi)$ of $\Phi$ to be the VC-dimension of $(M^{|x|}, \F_\Phi)$, similarly for the dual.
  \item Define the \defn{VC-density} $\vc(\Phi)$ of $\Phi$ to be the VC-density of $(M^{|x|}, \F_\Phi)$, similarly for the dual.
  \end{itemize}

  We will also refer to the VC-density and VC-dimension of a single formula $\phi$
  viewing it as a one element collection $\Phi = \curly{\phi}$.
\end{Definition}

Counting atoms of a boolean algebra in a model theoretic setting corresponds to counting types,
so it is instructive to rewrite the shatter function in terms of types.

\begin{Definition} 
  \begin{align*}
    \pi^*_\Phi(n) &= \max \curly{\text{number of $\Phi$-types over $B$} \mid B \subseteq M, |B| = n}.
  \end{align*}
  Here a $\Phi$-type over $B$ is a maximal consistent collection of formulas of the form $\phi(x, b)$ or $\neg\phi(x, b)$
  where $\phi \in \Phi$ and $b \in B$.
\end{Definition}

The functions $\pi^*_{\Phi}$ and $\pi^*_{\F_\Phi}$ do not have to agree,
as one fixes the number of generators of a boolean algebra of sets and the other fixes the size of the parameter set.
However, as the following lemma demonstrates, they both give the same asymptotic definition of dual VC-density.

\begin{Lemma} \label{count_types}
  \begin{align*}
    \vc^*(\Phi) &= \text{degree of polynomial growth of $\pi^*_\Phi(n)$}  = \limsup_{n \to \infty}\frac{\log \pi^*_\Phi(n)}{\log n}.
  \end{align*}  
\end{Lemma}

\begin{proof}
  With a parameter set $B$ of size $n$, we get at most $|\Phi|n$ sets $\phi(M^{|x|}, b)$ with $\phi \in \Phi, b \in B$.
  We check that asymptotically it doesn't matter whether we look at growth of boolean algebra of sets generated by
  $n$ or by $|\Phi|n$ many sets.
  We have:
  \begin{align*}
    \pi^*_{\F_\Phi}\paren{n} \leq \pi^*_\Phi(n) \leq \pi^*_{\F_\Phi}\paren{|\Phi|n}.
  \end{align*}
  Hence:
  \begin{align*}
    &\vc^*(\Phi) \leq \limsup_{n \to \infty}\frac{\log \pi^*_\Phi(n)}{\log n} \leq \limsup_{n \to \infty}\frac{\log \pi^*_{\F_\Phi}\paren{|\Phi|n}}{\log n} = \\
    & = \limsup_{n \to \infty}\frac{\log \pi^*_{\F_\Phi}\paren{|\Phi|n}}{\log |\Phi|n} \frac{\log |\Phi|n}{\log n} =
      \limsup_{n \to \infty}\frac{\log \pi^*_{\F_\Phi}\paren{|\Phi|n}}{\log |\Phi|n} \leq \\
    &\leq \limsup_{n \to \infty}\frac{\log \pi^*_{\F_\Phi}\paren{n}}{\log n} = \vc^*(\Phi).
  \end{align*}
\end{proof} 

One can check that the shatter function and hence VC-dimension and VC-density of a formula are elementary notions,
so they only depend on the first-order theory of the structure $\MM$.

NIP theories are a natural context for studying VC-density.
In fact we can take the following as the definition of NIP:
\begin{Definition}
  Define $\phi$ to be NIP if it has finite VC-dimension in a theory $T$.
  A theory $T$ is NIP if all the formulas in $T$ are NIP.
\end{Definition}

In a general combinatorial context (for arbitrary set systems),
VC-density can be any real number in $0 \cup [1, \infty)$ (see \cite{ash8}).
Less is known if we restrict our attention to NIP theories.
Proposition 4.6 in \cite{density} gives examples of formulas that have non-integer rational VC-density in an NIP theory,
however it is open whether one can get an irrational VC-density in this model-theoretic setting.

Instead of working with a theory formula by formula, we can look for a uniform bound for all formulas:
\begin{Definition} \label{vc_fn_def}
  For a given NIP structure $\MM$, define the \defn{VC-density function}
  \begin{align*}
    \vc^\MM(n) &= \sup \{\vc^*(\phi(x, y)) \mid \phi \in \LL(M), |x| = n\} \\
             &= \sup \{\vc(\phi(x, y)) \mid \phi \in \LL(M), |y| = n\} \in \R^{\geq 0} \cup \curly{+\infty}.
  \end{align*}
\end{Definition}

As before this definition is elementary, so it only depends on the theory of $\MM$.
We omit the superscript $\MM$ if it is understood from the context.
One can easily check the following bounds:
\begin{Lemma} [Lemma 3.22 in \cite{density}] \label{vcone}
  We have $\vc(1) \geq 1$ and $\vc(n) \geq n\vc(1)$.  
\end{Lemma}

However, it is not known whether the second inequality can be strict or even just whether $\vc(1) < \infty$ implies $\vc(n) < \infty$.


%%%%%%%%%%%%%%%%%%%%%%%%%%%%%%%%%%%%%%%%%%%%%%%%%%%%%%%%%%%%%%%%%%%%%%%%%%%%%%%%%%%%%%%%%%%%%%%%%%%%%%%%%%%%%%%%% 
\section{Graph Combinatorics}

Throughout this paper $A, B, C, M$ will denote finite graphs, and $\DB$ will be used to denote potentially infinite graphs.
For a graph $\A$ the set of its vertices is denoted by $v(\A)$, and the set of its edges by $e(\A)$.
Number of vertices of $\A$ will be denoted as $|\A|$.
Subgraph always means induced subgraph and $A \subseteq B$ means that $A$ is a subgraph of $B$.
For two subgraphs $\A, \B$ of a larger graph, the union $\A \cup \B$ denotes the graph induced by $v(\A) \cup v(\B)$.
Similarly, $A - B$ means a subgraph of $A$ induced by the vertices of $v(A) - v(B)$.
For $A \subseteq B \subseteq D$ and $A \subseteq C \subseteq D$,
graphs $B,C$ are said to be \defn{disjoint over $A$} if $v(B) - v(A)$ is disjoint from $v(C) - v(A)$
and there are no edges from $v(B) - v(A)$ to $v(C) - v(A)$ in $D$.

For the remainder of the paper fix $\alpha \in (0,1)$, irrational.
\begin{Definition} \ 
  \begin{itemize}
  \item For a graph $\A$ let $\dim(\A) = |\A| - \alpha |e(\A)|$.
  \item For $\A,\B$ with $\A \subseteq \B$ define $\dim(\B/\A) = \dim(\B) - \dim(\A)$.
  \item We say that $\A \leq \B$ if $\A \subseteq \B$ and $\dim(\A'/A) > 0$ for all $\A \subsetneq \A' \subseteq \B$.
  \item Define $\A$ to be \defn{positive} if for all $\A' \subseteq \A$ we have $\dim(\A') \geq 0$.
  \item We work in theory $S_\alpha$ in the language of graphs axiomatized by:
    \begin{itemize}
    \item Every finite substructure is positive.
    \item Given a model $\GG$ and graphs $\A \leq \B$, every embedding $f : \A \arr \GG$ extends to an embedding $g: \B \arr \GG$.
    \end{itemize}
    (Here an embedding maps edges to edges and nonedges to nonedges.)
    This theory is complete and stable (see 5.7 and 7.1 in \cite{laskowski}).
    From now on fix an ambient model $\GG \models S_\alpha$.
    This will be the only infinite graph we work with.
  \item For $\A, \B$ positive, $(\A, \B)$ is called a \defn{minimal pair} if
    $\A \subseteq \B$, $\dim(\B/\A) < 0$ but $\dim(\A'/\A) \geq 0$ for all proper $\A \subseteq \A' \subsetneq \B$.
    We call $B$ a \defn{minimal extension} of $A$.
    The dimension of a minimal pair is defined as $|\dim(B/A)|$.
  \item A sequence $\agl{M_i}_{0 \leq i \leq n}$ is called a \defn{minimal chain} if $(M_i, M_{i+1})$ is a minimal pair for all $0 \leq i < n$.
  \item For a graph $\A$ with the tuple of vertices $x$ let $\diag_\A(x)$ be the atomic diagram of $\A$,
    i.e. the first-order formula recording whether there is an edge between every pair of vertices.
  \item Given $\A \subseteq \B$ let 
    \begin{align*}
      \phi_{\A,\B}(x) = \diag_\A(x) \wedge \exists z \; \diag_\B(x, z).
    \end{align*}
    Any graph isomorphic to $\B$ is called a \defn{witness} of $\phi_{A,B}$.
  \item A formula $\phi_{A,B}$ is called a \defn{basic formula}
    if there is a minimal chain $\agl{M_i}_{0 \leq i \leq n}$
    such that $A = M_0$ and $B = M_n$.
  \end{itemize}
\end{Definition}

\begin{Theorem} [Quantifier elimination, 5.6 in \cite{laskowski}]
  In theory $S_\alpha$ every formula is equivalent to a boolean combination of basic formulas.
\end{Theorem}

\begin{Definition}
  A graph $S \subseteq \DB$ is called \defn{$N$-strong} if for any $S \subseteq T \subseteq D$ with $|T| - |S| \leq N$ we have $S \leq T$.
\end{Definition}

%%%%%%%%%%%%%%%%%%%%%%%%%%%%%%%%%%%%%%%%%%%%%%%%%%%%%%%%%%%%%%%%%%%%%%%%%%%%%%%%%%%%%%%%%%%%%%%%%%%%%%%%%%%%%%%%% 
\section{Basic Definitions and Lemmas}

\begin{Definition} \label{def_basic}
  Suppose $\phi(x, y)$ is a basic formula.
  Define $\X$ to be the graph on vertices $x$ with edges defined by $\phi$.
  Similarly define $\Y$.
  Note that $\X$, $\Y$ are positive.
  Additionally, let $\Y'$ be a subgraph of $\Y$ induced by vertices of $\Y$ that are connected to $W - (X \cup Y)$, where $W$ is a witness of $\phi$.
\end{Definition}

We will require the following lemmas from \cite{laskowski}:

\begin{Lemma} \label{diamond} [See 2.3 in \cite{laskowski}]
  Let $A, B \subseteq \DB$.
  Then
  \begin{align*}
    \dim(A \cup B / A) \leq \dim(\B / A \cap B).
  \end{align*}
  Moreover, 
  \begin{align*}
    \dim(A \cup B / A) = \dim(\B / A \cap B) - \alpha E,
  \end{align*}
  where $E$ is the number of edges connecting the vertices of $A \cup B - A$ to the vertices of $A - A \cap B$.
\end{Lemma}

\begin{Lemma} \label{las_min} [See 4.1 in \cite{laskowski}]
  Suppose $A$ is a positive graph, with at least $1/\alpha + 2$ vertices.
  Then for any $\epsilon > 0$ there exists a graph $B$ such that $(A, B)$ is a minimal pair with dimension $\leq \epsilon$.
  Moreover, every vertex in $A$ is connected to a vertex in $B - A$.
\end{Lemma}

\begin{Lemma} \label{las_str} [See 4.4 in \cite{laskowski}]
  Suppose $A$ is a positive graph, and $\G$ a model of $S_\alpha$.
  Then for any integer $S$ there exists an embedding $f \colon A \arr \G$ such that $f(A)$ is $S$-strong in $\G$.
\end{Lemma}
    
\begin{Lemma} \label{las_closure} [See 3.8 in \cite{laskowski}]
  For all $S > 0$ there exists $M = M(S, \alpha) \in \N$ with the following property.
  Suppose $A \subseteq \G$ where $\G$ is a model of $S_\alpha$.
  Then there exists $B$ with $A \subseteq B \subseteq \G$ such that $B$ is $S$-strong in $\GG$ and $|B| \leq M|A|$.
\end{Lemma}

We conclude this section by stating a couple of technical lemmas that will be useful in our proofs later.

\begin{Lemma} \label{minimal_over_set}
  Work in an ambient graph $\DB$.
  Suppose we have a set $B$ and a minimal pair $(A, M)$ with $A \subseteq B$ and $\dim(M/A) = -\epsilon$.
  Then either $M \subseteq B$ or $\dim(M \cup B/B) < -\epsilon$.
\end{Lemma}

\begin{proof}
  By Lemma \ref{diamond}
  \begin{align*}
    \dim(M \cup B/B) \leq \dim(M / M \cap B),
  \end{align*}
  and as $A \subseteq M \cap B \subseteq M$
  \begin{align*}
    \dim (M/A) = \dim(M / M \cap B) + \dim(M \cap B / A).
  \end{align*}
  In addition we are given $\dim (M/A) = -\epsilon$.
  If $M \not\subseteq B$ then $A \subseteq M \cap B \subsetneq M$ and by minimality $\dim(M \cap B / A) > 0$.
  Combining the inequalities above we obtain the desired result:
  \begin{align*}
    \dim(M \cup B/B) \leq \dim(M / M \cap B) = \dim (M/A) - \dim(M \cap B / A) < -\epsilon.
  \end{align*}
\end{proof}

\begin{Lemma}	\label{chain_lemma}
  Work in an ambient graph $\DB$.
  Suppose we have a set $B$ and a minimal chain  $\agl{M_i}_{0 \leq i \leq n}$ with dimensions
  \begin{align*}
    \dim(M_{i+1}/M_i) = -\epsilon_i.
  \end{align*}
  Let $\epsilon = \min_{0 \leq i \leq n} \epsilon_i$.
  Then either $M_n \subseteq B$ or $\dim((M_n \cup B)/B) < -\epsilon$.
\end{Lemma}

\begin{proof}
  Let $\bar M_i = M_i \cup B$. Then:
  \begin{align*}
    \dim(\bar M_n/B) = \dim(\bar M_n/\bar M_{n-1}) + \ldots + \dim(\bar M_2/\bar M_1) + \dim(\bar M_1/B).
  \end{align*}
  Either $M_n \subseteq B$ or at least one of the summands above is nonzero.
  Apply previous lemma.
\end{proof}

\begin{Lemma} \label{minimal_subset}
  Suppose we have a minimal pair $(A, M)$ with dimension $\epsilon$.
  Suppose we have some $B \subseteq M$.
  Then $\dim B / (A \cap B) \geq -\epsilon$.
  Moreover if $B \cup A \neq M$ then $\dim B / (A \cap B) \geq 0$.
\end{Lemma}

\begin{proof}
  We have $\dim (B \cup A / A) \leq \dim B / (A \cap B)$ by Lemma \ref{diamond}.
  As $A \subseteq B \cup A \subseteq M$ we have $\dim (B \cup A / A) \geq -\epsilon$ by minimality.
  Moreover, minimality implies that it is positive if $B \cup A \neq M$.
\end{proof}

\begin{Lemma} \label{chain_intersect}
  Suppose we have a minimal chain  $\agl{M_i}_{0 \leq i \leq n}$ with dimensions
  \begin{align*}
    \dim(M_{i+1}/M_i) = -\epsilon_i.
  \end{align*}
  Let $\epsilon$ be the sum of all $\epsilon_i$.
  Suppose we have a graph $B$ with $B \subseteq M_n$.
  Then $\dim B / (M_0 \cap B) \geq -\epsilon$.
\end{Lemma}

\begin{proof}
  Let $B_i = B \cap M_i$.
  We have $\dim B_{i+1}/B_i \geq \dim M_{i+1}/M_i$ by the previous lemma.
  Thus
  \begin{align*}
    \dim B / (M_0 \cap B) = \dim B_n / B_0 = \sum \dim B_{i+1}/B_i \geq -\epsilon.
  \end{align*}
\end{proof}

%%%%%%%%%%%%%%%%%%%%%%%%%%%%%%%%%%%%%%%%%%%%%%%%%%%%%%%%%%%%%%%%%%%%%%%%%%%%%%%%%%%%%%%%%%%%%%%%%%%%%%%%%%%%%%%%% 
\section{Lower bound}

In this section we restrict our attention to the following family of basic formulas $\phi(x,y)$:
\begin{itemize}
%\item Graphs defined by $x,y$ are $\X, \Y$.
\item All formulas have $\Y' = \Y$ (see Definition \ref{def_basic}).
\item All formulas define no edges between $X$ and $Y$.
\item Minimal chain of $\phi(x,y)$ consists of one step, that is we only have one minimal extension as opposed to a chain of minimal extensions.
\item The dimension of that minimal extension is smaller than $\alpha$.
\end{itemize}

We obtain a lower bound for the formulas that are boolean combinations of basic formulas written in the disjunctive-conjunctive form.
First, define $\epsilon_L(\phi)$.

\begin{Definition} 
  For a basic formula $\phi = \phi_{\agl{M_i}_{0 \leq i \leq n}}(x, y)$ let
  \begin{itemize}
  \item $\epsilon_i(\phi) = -\dim \paren{M_i/M_{i-1}}$.
  \item $\epsilon_L(\phi) = \sum_1^{n} \epsilon_i(\phi)$.
  \end{itemize}
\end{Definition}

\begin{Definition}[Negation]
  If $\phi$ is a basic formula, then define
  \begin{align*}
    \epsilon_L(\neg \phi) &= \epsilon_L(\phi).
  \end{align*}
\end{Definition}

\begin{Definition}[Conjunction]
  Take a collection of formulas $\phi_i(x, y)$ where each $\phi_i$ is a positive or a negative basic formula.
  If both positive and negative formulas are present then $\epsilon_L(\phi) = \infty$.
  We don't have a lower bound for that case.
  If different formulas define $\X$ or $\Y$ differently then $\epsilon_L(\phi) = \infty$.
  In the case of conflicting definitions the formula would have no realizations.
  Otherwise let
  \begin{align*}
    \epsilon_L\paren{\bigwedge \phi_i} &= \sum \epsilon_L(\phi_i).
  \end{align*}
\end{Definition}

\begin{Definition} [Disjunction]
  Take a collection of formulas $\psi_i$ where each instance is a conjunction as above all agreing on $\X$ and $\Y$.
  Then
  \begin{align*}
    \epsilon_L\paren{\bigvee \psi_i} &= \min \epsilon_L(\psi_i).
  \end{align*}
\end{Definition}
\begin{Theorem}
  For a formula $\psi$ as above we have
  \begin{align*}
    \vc \psi \geq \floor{\frac{Y(\psi)}{\epsilon_L(\psi)}},
  \end{align*}
  where $Y(\psi)$ is $\dim(Y)$ (as all basic componenets agree on $\Y$).
\end{Theorem}
\begin{proof}
  First, work with a formula that is a conjunction of positive basic formulas $\psi = \bigwedge_{i \in I} \phi_i$.
  Then as we have defined above
  \begin{align*}
    \epsilon_L(\psi) = \sum_{i \in I} \epsilon_L(\phi_i).
  \end{align*}
  If $W_i$ is a witness of $\phi_i$, let $S_i = |W_i|$.
  Let $n_1$ be the largest natural number such that
  \begin{align*}
    n_1 \epsilon_L(\psi) < Y(\psi).
  \end{align*}
  Let $\epsilon'$ be the smallest value among $\epsilon_L(\phi_i)$.
  Suppose it corresponds to the formula $\phi'$.
  Let $n_2$ be the largest natural number such that
  \begin{align*}
    n_1 \epsilon_L(\psi) + n_2 \epsilon' < Y(\psi).
  \end{align*}

  Fix some $N > n_1 + n_2$.
  Let 
  \begin{align*}
    J = \curly{0 \leq j < N} \subseteq \N.
  \end{align*}
  Let $a_j$ be a graph isomorphic to $\X$ for each $j \in J$, pairwise disjoint.
  Let $A = \bigcup_{1 \leq j \leq N} a_j$.
  Let 
  \begin{align*}
    S = |Y| + (n_1 + n_2 + 1) \sum_{i \in I} S_i.
  \end{align*}

  By Lemma \ref{las_str} the graph $A$ can be embedded into $\GG$ as an $S$-strong graph. 
  Abusing notation, we identify $A$ with this embedding.
  Thus we have $A \subseteq \GG$, $S$-strong. 

  Let $J_1, J_2$ be disjoint subsets of $J$, of sizes $n_1, n_2$ respectively.
  Let $b$ be a graph isomorphic to $\Y$.
  For each $i \in I, j \in J_1$ let $W_{ij}$ be a witness of $\phi_i(a_j, b)$.
  (Note that then $(a_j \cup b, W_{ij})$ is a minimal pair.)
  For each $j \in J_1$ let $W_j$ be a union of $\curly{W_{ij}}_{i \in I}$ disjoint over $a_j \cup b$.
  For each $j \in J_2$ let $W_{j}$ be a witness of $\phi'(a_j, b)$.
  Let $W'$ be a union of $\curly{W_j}_{j \in J_1 \cup J_2}$ disjoint over $b$.
  Let $W$ be a union of $W'$ and $A$ disjoint over $\curly{a_j}_{j \in J_1 \cup J_2}$.
  \begin{Claim}
    We have $A \leq W$.
  \end{Claim}
  \begin{proof}
    Consider some $A \subsetneq B \subseteq W$.
    We need to show $\dim (B/A) > 0$.
    Let $\bar A = A \cup b$.
    We have
    \begin{align*}
      \dim(B/A) = \dim(B/ B \cap \bar A) + \dim(B \cap \bar A / A).
    \end{align*}
    Let $B_{ij} = B \cap W_{ij}$.
    Let $B_{j} = B \cap W_{j}$.
    To unify indices, relabel all the graphs above as $\curly{B_k}_{k \in K}$ for some index set $K$.
    By the construction of $W$ we have
    \begin{align*}
      \dim(B/ B \cap \bar A) = \sum_{k \in K} \dim(B_k/ B_k \cap \bar A).
    \end{align*}
    Fix $k$.
    We have $B_k \subseteq W_k$, where $W_k$ is a minimal extension of $M^k_0 = a \cup b$ for some $a \in A$.
    Let $\epsilon_k$ be the dimension of this minimal extension.
    We have $\dim(B_k / B_k \cap \bar A) = \dim(B_k / a \cup (B \cap b))$.

    Case 1: $B \cap b = b$.
    Then $M_0^k \subseteq B_k \subseteq W_k$ and
    \begin{align*}
      \dim(B_k / a \cup (B \cap b)) = \dim (B_k/M_0^k).
    \end{align*}
    By minimality of $(M_0^k, B_k)$ we have $\dim (B_k/M_0^k) \geq -\epsilon_k$.
    Thus
    \begin{align*}
      \dim(B/ B \cap \bar A) \geq - \sum_{k \in K} \epsilon_k = -\paren{n_1 \epsilon_L(\psi) + n_2 \epsilon'}.
    \end{align*}
    In addition
    \begin{align*}
      \dim(B \cap \bar A / A) = \dim (b) = Y(\psi).
    \end{align*}
    Combining the two, we get
    \begin{align*}
      \dim(B/A) \geq Y(\psi) - \paren{n_1 \epsilon_L(\psi) + n_2 \epsilon'},
    \end{align*}
    which is positive by the construction of $n_1, n_2$ as needed.
    
    Case 2: $B \cap b \subsetneq b$.
    \begin{Claim} We have $\dim(B_k / B_k \cap \bar A) > 0$.
    \end{Claim}
    \begin{proof}
      Recall that $\dim(B_k / B_k \cap \bar A) = \dim(B_k / a \cup (B \cap b))$.
      First, suppose that $B_k \cup M_0^k \neq W_k$.
      Then by Lemma \ref{minimal_subset} we get the required inequality.
      Thus we may assume that $B_k \cup M_0^k = W_k$.
      By Lemma \ref{diamond} we have
      \begin{align*}
        \dim(B_k \cup M_0^k / M_0^k) = \dim(B_k / B_k \cap M_0^k) - \alpha E,
      \end{align*}
      where $E$ is the number of edges connecting the vertices of $B_k \cup M_0^k - M_0^k$ to the vertices of $M_0^k - B_k \cap M_0^k$.
      Noting that $B_k \cup M_0^k = W_k$, $\dim{W_k / M_0^k} = -\epsilon_k$, and $B_k \cap M_0^k = a \cup (B \cap b)$
      we may rewrite the equality above as
      \begin{align*}
        \dim(B_k / a \cup (B \cap b)) = \alpha E - \epsilon,
      \end{align*}
      and $E$ is the number of edges connecting the vertices of $W_k - M_0^k$ to the vertices of $M_0^k - a \cup (B \cap b)$.
      As $\Y = \Y'$ and $B \cap b \subsetneq b$ we must have $E \geq 1$.
      But then as $\alpha > \epsilon$ we have $\dim(B_k / a \cup (B \cap b)) > 0$ as needed.
    \end{proof}
    Now, recall that
    \begin{align*}
      \dim(B/A) = \dim(B \cap \bar A / A) + \sum_{k \in K} \dim(B_k/ B_k \cap \bar A).
    \end{align*}
    By the claim above each of $\dim(B_k/ B_k \cap \bar A) > 0$, thus
    \begin{align*}
      \dim(B/A) > \dim(B \cap \bar A / A).
    \end{align*}
    In addition
    \begin{align*}
      \dim(B \cap \bar A / A) = \dim (B \cap b) \geq 0,
    \end{align*}
    as $b$ is postive.
    Thus $\dim (B/A) > 0$ as needed.
  \end{proof}

  As $A \leq W$ and $A \subseteq \GG$, we can embed $W$ into $\GG$ over $A$.
  Abusing notation again, we identify $W$ with its embedding $A \leq W \subseteq \GG$.
  In particular, now we have $b \in \GG$.
  Also note that
  \begin{align*}
    \dim(W/A) &= Y(\psi) - \paren{n_1 \epsilon_L(\psi) + n_2 \epsilon'}, \\
    |W| - |A| &\leq |b| + (n_1 + n_2) \sum_{i \in I} S_i.
  \end{align*}

  \begin{Lemma} We have
    \begin{align*}
      \curly{a_j}_{j \in J_1} \subseteq \psi(A, b) \subseteq \curly{a_j}_{j \in J_1 \cup J_2}.
    \end{align*}
  \end{Lemma}
  \begin{proof}
    First inclusion $\curly{a_j}_{j \in J_1} \subseteq \psi(A, b)$ is immediate from the construction of $W$,
    as $W_{ij}$ witnesses that $\phi_i(a_j, b)$ holds.
    For the second inclusion, suppose that there is $a \in A - \curly{a_j}_{j \in J_1 \cup J_2}$ such that $\psi(a,b)$ holds.
    Let $W' \subseteq \GG$ be a witness of $\phi_1(a,b)$.
    First, note that the case $W' \subseteq W$ is impossible
    as there are no edges between $a$ and $W - a$, but there are edges between $a$ and $W' - a$.
    Thus assume $W' \not\subseteq W$.
    As $(a \cup b, W')$ is minimal, by Lemma \ref{minimal_over_set} we have $\dim (W' \cup W / W) < -\epsilon_1$.
    Therefore
    \begin{align*}
      \dim(W' \cup W / A) = \dim (W' \cup W / W) + \dim(W/A) < Y(\psi) - \paren{n_1 \epsilon_L(\psi) + n_2 \epsilon'} - \epsilon_1,
    \end{align*}
    which is negative by the construction of $n_1, n_2$.
    Thus $A \not\leq W \cup W'$, as then it would have a positive dimension.
    Additionally,
    \begin{align*}
      |W' \cup W| - |A| \leq |W' - W| + |W| - |A| \leq S_1 + |b| + (n_1 + n_2) \sum_{i \in I} S_i \leq S,
    \end{align*}
    but then this contradicts that $A$ is $S$-strong, as then we would have $A \leq W \cup W'$.
  \end{proof}

  In the construction of $W$ we have chosen indices $J_1, J_2$ arbitrarily.
  In particular, suppose we let $J_2$ to be the last $n_2$ indices of $J$ and
  $J_1$ an arbitrary $n_1$-element subset of the first $N - n_2$ elements of $J$.
  Each of those choices would then yield a different trace $\psi(A, b)$ by the lemma above.
  Thus $\psi(A, M^{|y|}) \geq {N - n_2 \choose n_1}$ and therefore $\vc(\psi) \geq n_1$.
  By the definition of $n_1$ we have $n_1 = \floor{\frac{Y(\psi)}{\epsilon_L(\psi)}}$, so this proves the theorem for $\psi$.
 
  Now consider a formula which is a conjunction consisting of negative basic formulas $\psi = \bigwedge_{i \in I} \neg \phi_i$.
  Let $\bar \psi = \bigwedge_{i \in I} \phi_i$.
  Do the construction above for $\bar \psi$ and suppose its trace is $X \subseteq A$ for some $b$.
  Then over $b$ the same construction gives trace $(A - X)$ for $\psi$. Thus we get as many traces as above, and the same bound.
  
  Finally consider a formula which is a disjunction of formulas considered above $\theta = \bigvee_{k \in K} \psi_k$.
  Choose the one with the smallest $\epsilon_L$, say $\psi_k$, and repeat the construction above for $\psi_k$.
  Any trace we obtain is automatically a trace for $\theta$, and thus we get as many traces as above, and the same bound.
\end{proof}

\begin{Corollary}
  VC-function is infinite in Shelah-Spencer random graphs:
  \begin{align*}
    \vc(n) = \infty.
  \end{align*}
\end{Corollary}

\begin{proof}
  Let $A$ be a graph consisting of $1/\alpha + 2 + n$ disconnected vertices.
  Fix $\epsilon > 0$.
  By Lemma \ref{las_min}, there exists $B$ such that $(A, B)$ is minimal with dimension $\leq \epsilon$.
  Consider a basic formula $\psi_{A, B}(x, y)$ where $|x| = 1/\alpha + 2$ and $|y| = n$.
  Then by the theorem above $\vc(n) \geq \vc (\psi_{A,B}) \geq \frac{n}{\epsilon}$.
  As $\epsilon$ was arbitrary, this number can be made arbitrarily large, giving $vc(n) = \infty$ as needed.
\end{proof}



%%%%%%%%%%%%%%%%%%%%%%%%%%%%%%%%%%%%%%%%%%%%%%%%%%%%%%%%%%%%%%%%%%%%%%%%%%%%%%%%%%%%%%%%%%%%%%%%%%%%%%%%%%%%%%%%% 
\section{Upper bound}

We bound the number of types of some finite collection of formulas $\Psi(\vec x, \vec y) = \curly{\phi_i(\vec x, \vec y)}_{i\in I}$ over a parameter set $B$ of size $N$,
where $\phi_i$ is a basic formula.

Fix a formula $\phi$ from our collection.
Suppose it defines a minimal chain extension over $\{x, y\}$. 
Record the size of that extension as $K(\phi)$ and its total dimension $\epsilon(\phi) = \epsilon_U(\phi)$.
Define dimension of that formula $D(\phi) = |\vec y| \frac{K(\phi)}{\epsilon(\phi)}$
Define dimension of the entire collection as $D(\Psi) = \max_{i \in I} D(\phi_i)$

Fix $S = ??$.
Suppose we have a finite parameter set $A_0 \subseteq \GG^{|x|}$ with $|A_0| = N_0$.
We would like to bound $\phi(A_0, \GG^{|y|})$.
Let $A_1 \subseteq \GG$ consist of the components of the elements of $A_0$.
Then $|A_1| \leq |x| N_0$.
Using Lemma \ref{las_closure} let $A$ be a graph $A_0 \subseteq A \subseteq \GG$, $S$-strong in $\GG$.
Let $N = |A|$.
We have $N \leq |x| N_0 M$ (where $M$ is the constant from the Lemma \ref{las_closure}).
As $A_0 \subseteq A^{|x|}$ we have
\begin{align*}
  \abs{{\phi(A_0, \GG^{|y|})} \leq \abs{\phi(A^{|x|}, \GG^{|y|})}.
\end{align*}
Therefore it suffices to bound $\abs{\phi(A^{|x|}, \GG^{|y|})}$.
Let
\begin{align*}
  \GG_1 = \curly{b \in \GG^{|y|} \mid b \subseteq A}
\end{align*}
and $\GG_2 = \GG^{|y|} - \GG_1$.
Then
\begin{align*}
  \abs{\phi(A^{|x|}, \GG^{|y|})}
  \leq \abs{\phi(A^{|x|}, \GG_1)} + \abs{\phi(A^{|x|}, \GG_2)} =
  \abs{\GG_1} + \abs{\phi(A^{|x|}, \GG_2)} \leq N^{|y|} + \abs{\phi(A^{|x|}, \GG_2)}.
\end{align*}
Our goal now is to bound $\abs{\phi(A^{|x|}, \GG_2)}$.

\begin{Definition}
  \begin{itemize}
  \item For all $a \in A^{|x|}, b \in \GG_2$ if $\phi(a, b)$ holds fix $W_{a,b} \subseteq \GG$, a witness of this formula.
  \item For $b \in \GG_2$ let 
    \begin{align*}
      W_b = \bigcup {W_{a,b} \mid a \in A^{|x|}, \GG \models \phi(a,b)}.
    \end{align*}
  \end{itemize}
\end{Definition}

\begin{Definition}
  For sets $C, B \subset \GG$ define the boundary of $C$ over $B$
  \begin{align*}
    \partial(C, B) = \curly{b \in B \mid \text{there is an edge between $b$ and a vertex in $C - B$}}
  \end{align*}
\end{Definition}

\begin{Definition}
  \begin{itemize}
  \item For $b \in \GG_2$ let $\partial_b$ to be the boundary $\partial(W_b, A)$.
  \item For $b \in \GG_2$ let $\bar W_b = (W_b - A) \cup \ppp_b$.
  \item For $b_1, b_2 \in \GG_2$ we say that $b_1 \sim b_2$ if $\ppp_{b_1} = \ppp_{b_2}$
    and there exists a graph isomorphism from $\bar W_{b_1}$ to $\bar W_{b_2}$ that fixes $\ppp_{b_1}$ and
    maps $b_1$ to $b_2$.
    One easily checks that this defines an equivalence relation.
  \item For $b \in \GG_2$ define $\II_b$ to be the $\sim$-equivalence class of $b$.
  \end{itemize}
\end{Definition}

\begin{Lemma} \label {bound_trace}
  For $b_1, b_2 \in \GG_2$ if $b_1 \sim b_2$ then $\phi(A^{|x|}, b_1) = \phi(A^{|x|}, b_2)$.
\end{Lemma}

\begin{proof}
  Fix the graph isomorphism $\bar f \colon \bar W_{b_1} \arr \bar W_{b_2}$.
  Extend it to an isomorphism $f \colon W_{b_1} \cup A \arr W_{b_2} \cup A$ by being an identity map on the new vertices.
  Suppose $\GG \models phi(a, b_1)$ for some $a \in A^{|x|}$.
  Then $f(W_{a, b_1})$ is a witness for  $\phi(a, b_2)$ (though not necessarily equal to $W_{a, b_2}$)
  and thus $\GG \models phi(a, b_2)$.
  As $a$ was arbitrary, this proves the equality of the traces.
\end{proof}

Thus to bound the number of traces it is sufficient to bound the number of possibilities for $\II_a$.

\begin{Theorem} \label{main_bound}
  \begin{align*}
    |\partial_a| &\leq D(\Psi) \\ 
    |\bar M_b - \bar A| &\leq D(\Psi)
  \end{align*}
\end{Theorem}

\begin{Corollary}
  \begin{align*}
    \vc(\phi) \leq K(\phi) \frac{Y(\phi)}{\epsilon(\phi)}
  \end{align*}
\end{Corollary}

\begin{proof}
  We count possible $\partial$-isomorphism classes $\II_b$.
  Let $W = K(\phi) \frac{Y(\phi)}{\epsilon(\phi)}$.
  If the parameter set $A$ is of size $N$ then there are $N \choose W$ choices for boundary $\partial_b$.
  On top of the boundary there are at most $W$ extra vertices and $(2W)^2$ extra edges.
  Thus there are at most
  \begin{align*}
    W \cdot 2^{(2W)^2}
  \end{align*}
  configurations up to a graph isomorphism.
  In total this gives us 
  \begin{align*}
    {N \choose W} \cdot W \cdot 2^{(2W)^2} = O(N^W)
  \end{align*}
  options for $\partial$-isomorphism classes.
  By Lemma \ref{bound_trace} there are at most $O(N^W)$ many traces, giving the required bound.
\end{proof}

\begin{proof} \textit{(of Theorem \ref{main_bound})}
  Fix some $b$-trace $A_b$. Enumerate $A_b = \{a_1, \ldots, a_I\}$.

  Let $M_i / \{a_i, b\}$ be a witness of $\phi(a_i, b)$ for each $i \leq I$.
  Let $\bar M_i = \bigcup_{j < i} M_j$.
  Let $\bar M = \bigcup M_i$, a witness of $A_b$
  
  \begin{Claim}
    \begin{align*}
      &\abs{\partial(M_i M, \bar A) - \partial(M, \bar A)} \leq |M_i| = K(\phi)\\
      &\dim(M_i M \bar A / M \bar A) > -\epsilon(\phi)
    \end{align*}
  \end{Claim}
  
  \begin{Definition}
    $(j-1, j)$ is called a \defn{jump} if some of the following conditions happen
    \begin{itemize}
    \item New vertices are added outside of $\bar A$ i.e.
      \begin{align*}
        \bar M_j - \bar A \neq \bar M_{j-1} - \bar A
      \end{align*}
    \item New vertices are added to the boundary, i.e.
      \begin{align*}
        \partial(\bar M_j, \bar A) \neq \partial(\bar M_{j-1}, \bar A)
      \end{align*}
    \end{itemize}
  \end{Definition}

  \begin{Definition}
    We now let $m_i$ count all jumps below $i$
    % Let $d_i = \dim(\bar M_i/A)$.
    \begin{align*}
      m_i = \abs{\curly{j < i \mid (j-1, j) \text{ is a jump}}}
    \end{align*}
  \end{Definition}

  \begin{Lemma} \label{ub_lemma}
    \begin{align*}
      \dim(\bar M_i / \bar A) &\leq -m_i \cdot \epsilon(\phi) \\
      |\partial(\bar M_i, \bar A)| &\leq m_i \cdot K(\phi) \\
      |\bar M_j - \bar A| &\leq m_i \cdot K(\phi)
    \end{align*}
  \end{Lemma}

  \begin{proof} \textit{(of Lemma \ref{ub_lemma})}
    Proceed by induction.
    Second and third propositions are clear.
    For the first proposition base case is clear.
    
    Induction step.
    Suppose $\bar M_j \cap (A \cup b) = \bar M_{j+1}$ and $\partial(\bar M_j, A) = \partial(\bar M_{j+1}, A)$.
    Then $m_i = m_{i+1}$ and the quantities don't change.
    Thus assume at least one of these equalities fails.
    
    Apply Lemma \ref{chain_lemma} to $\bar M_j \cup (A \cup b)$ and $(M_{j+1}, a_{j+1}b)$.
    There are two options
    
    \begin{itemize}
    \item $\dim(\bar M_{j+1} \cup (A \cup b) / \bar M_i \cup (A \cup b)) \leq -\epsilon_U$.
      This implies the proposition.
    \item $M_{j+1} \subseteq \bar M_j \cup (A \cup b)$.
      Then by our assumption it has to be $\partial(\bar M_j, A) \neq \partial(\bar M_{j+1}, A)$.
      There are edges between $M_{j+1} \cap (\partial(\bar M_{j+1}, A) - \partial(\bar M_j, A))$ so they contribute some negative dimension $\leq \epsilon_U$.
    \end{itemize}
    This ends the proof for Lemma \ref{ub_lemma}.
  \end{proof}
  \textit{(Proof of Theorem \ref{main_bound} continued)}
  First part of lemma \ref{ub_lemma} implies that we have $\dim(\bar M / \bar A) \leq -m_I \cdot \epsilon(\phi)$.
  The requirement of $A$ to be $S$-strong forces
  \begin{align*}
    m_I \cdot \epsilon(\phi) &< Y(\phi) \\
    m_I  &< \frac{Y(\phi)}{\epsilon(\phi)} \\
  \end{align*}
  % Let $W = \frac{K(\phi)Y(\phi)}{\epsilon(\phi)}$
  Applying the rest of \ref{ub_lemma} gives us
  \begin{align*}
    |\partial(\bar M, A)| &\leq m_I \cdot K(\phi) \leq \frac{K(\phi)Y(\phi)}{\epsilon(\phi)} \\
    |\bar M \cap A| &\leq m_I \cdot K(\phi) \leq \frac{K(\phi)Y(\phi)}{\epsilon(\phi)}
  \end{align*}
  as needed.
  This ends the proof for Theorem \ref{main_bound}.
\end{proof}

So far we have computed an upper bound for a single basic formula $\phi$.

To bound an arbitrary formula, write it as a boolean combination of basic formulas $\phi_i$ (via quantifier elimination)
It suffices to bound vc-density for collection of formulas $\{\phi_i\}$ to obtain a bound for the original formula.

In general work with a collection of basic formulas $\{\phi_i\}_{i \in I}$.
The proof generalizes in a straightforward manner.
Instead of $A^{|x|}$ we now work with $A^{|x|} \times I$ separating traces of different formulas.
Formula with the largest quantity $Y(\phi)\frac{K(\phi)}{\epsilon(\phi)}$ contributes the most to the vc-density.
Thus we have
\begin{align*}
  \Phi &= \{\phi_i\}_{i \in I} \\
  \vc(\Phi) &\leq   \max_{i \in I} Y(\phi_i) \frac{K(\phi_i)}{\epsilon_{\phi_i}}
\end{align*}

%%%%%%%%%%%%%%%%%%%%%%%%%%%%%%%%%%%%%%%%%%%%%%%%%%%%%%%%%%%%%%%%%%%%%%%%%%%%%%%%%%%%%%%%%%%%%%%%%%%%%%%%%%%%%%%%% 

\begin{thebibliography}{9}

\bibitem{density}
  M. Aschenbrenner, A. Dolich, D. Haskell, D. Macpherson, S. Starchenko,
  \textit{Vapnik-Chervonenkis density in some theories without the independence property}, I,
  Trans. Amer. Math. Soc. 368 (2016), 5889-5949
  
\bibitem{laskowski}
  Michael C. Laskowski, \textit{A simpler axiomatization of the Shelah-Spencer almost sure theories},
  Israel J. Math. \textbf{161} (2007), 157–186. MR MR2350161	

\bibitem{ash7}
  P. Assouad, \textit{Densit´e et dimension}, Ann. Inst. Fourier (Grenoble) 33 (1983), no. 3, 233-282.
\bibitem{ash8}
  P. Assouad, \textit{Observations sur les classes de Vapnik-Cervonenkis et la dimension combinatoire de Blei},
  in: Seminaire d’Analyse Harmonique, 1983-1984, pp. 92-112, Publications Math´ematiques
  d’Orsay, vol. 85-2, Universit´e de Paris-Sud, D´epartement de Math´ematiques, Orsay, 1985.
\bibitem{sauer}
  N. Sauer, \textit{On the density of families of sets}, J. Combinatorial Theory Ser. A 13 (1972), 145-147.
\bibitem{shelah}
  S. Shelah, \textit{A combinatorial problem; stability and order for models and theories in infinitary languages},
  Pacific J. Math. 41 (1972), 247-261.

\end{thebibliography}

\end{document}

