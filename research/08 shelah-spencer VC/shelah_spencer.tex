\documentclass{amsart}

\usepackage{../AMC_style}	
\usepackage{../Research}
\usepackage{../Thm}

\usepackage{mathrsfs}
\usepackage{pgfpages} 
\usepackage{setspace}

\doublespacing
%% \usepackage[margin=.75in]{geometry}
%% \pgfpagesuselayout{2 on 1}

\renewcommand{\AA}{\mathscr A}
\newcommand{\BB}{\mathscr B}
\newcommand{\DD}{\mathscr D}
\newcommand{\II}{\mathscr I}
\newcommand{\GG}{\mathcal G}

\newcommand{\F}{\mathcal F}
\renewcommand{\LL}{\mathcal L}

\newcommand{\defn}{\underline}



\newcommand{\A}{\mathcal A}
\newcommand{\B}{\mathcal B}
\renewcommand{\C}{\mathcal C}
\newcommand{\D}{\mathcal D}
\renewcommand{\H}{\mathcal H}
\newcommand{\G}{\mathcal G}
\newcommand{\M}{\mathcal M}
\newcommand{\U}{\mathcal U}	
\newcommand{\X}{\mathcal X}
\newcommand{\Y}{\mathcal Y}

\newcommand{\K}{\boldface K_\alpha}
\renewcommand{\S}{S_\alpha}

\newcommand{\curly}[1]{\left\{#1\right\}}
\newcommand{\paren}[1]{\left(#1\right)}
\newcommand{\abs}[1]{\left|#1\right|}
\newcommand{\agl}[1]{\left\langle #1 \right\rangle}

\providecommand{\floor}[1]{\left \lfloor #1 \right \rfloor }

% \DeclareMathOperator{\dim}{dim}

\title{Some vc-density computations in Shelah-Spencer graphs}
\author{Anton Bobkov}
\email{bobkov@math.ucla.edu}

\begin{document}

\begin{abstract}
  We investigate vc-density in Shelah-Spencer graphs.
  We provide an upper bound on formula-by-formula basis and show that there isn't a uniform lower bound forcing the vc-function to be infinite.
\end{abstract}

\maketitle

%%%%%%%%%%%%%%%%%%%%%%%%%%%%%%%%%%%%%%%%%%%%%%%%%%%%%%%%%%%%%%%%%%%%%%%%%%%%%%%%%%%%%%%%%%%%%%%%%%%%%%%%%%%%%%%%% 

VC-density was studied in \cite{density} by Aschenbrenner, Dolich, Haskell, MacPherson, and Starchenko as a natural notion of dimension for NIP theories.
In a complete NIP theory $T$ we can define the vc-function

\begin{align*}
  \vc^T = \vc : \N \arr \R \cup \curly{\infty}
\end{align*}

where $\vc(n)$ measures the worst-case complexity of families of definable sets in an $n$-fold Cartesian power of the underlying set of a model of $T$
(see \ref{vc_fn_def} below for a precise definition of $\vc^T$).
The simplest possible behavior is $\vc(n) = n$ for all $n$. Theories with the property that $\vc(1) = 1$ are known to be dp-minimal, i.e., having the smallest possible dp-rank. It is not known whether there can be a dp-minimal theory which doesn't satisfy $\vc(n)=n$
(see \cite{density}, diagram on pg. 41).

In this paper, we investigate vc-density of definable sets in Shelah-Spencer graphs.
In our description of Shelah-Spencer graphs we follow closely the treatment in \cite{laskowski}.
A Shelah-Spencer graph is a limit of random structures $G(n, n^{-\alpha})$ for an irrational $\alpha \in (0,1)$.
$G(n, n^{-\alpha})$ is a random graph on $n$ vertices with edge probability $n^{-\alpha}$.

Our first result is that in Shelah-Spencer graphs
\begin{align*}
  \vc(n) = \infty
\end{align*}
which implies that they are not dp-minimal.
Our second result is providing an upper bound on a vc-density for a formula $\phi$
\begin{align*}
  \vc(\phi) \leq K(\phi) \frac{Y(\phi)}{\epsilon(\phi)}    
\end{align*}
where $K(\phi), Y(\phi), \epsilon(\phi)$ are paramters easily computable from the quantifier free form of $\phi$.

Chapter 1 introduces basic facts about VC-dimension and vc-density.
More can be found in \cite{density}.
Chapter 2 summarizes notation and basic facts concerning Shelah-Spencer graphs.
We direct the reader to \cite{laskowski} for a more in-depth treatment.
In chapter 3 we introduce some measure of dimension for quantifier free formulas as well as proving some elementary facts about it.
Chapter 4 computes a lower bound for vc-density to demonstrate that $\vc(n) = \infty$.
Chapter 5 computes an upper bound for vc-density on a formula-by-formula basis.

% This struc is axiomatized by $S_\alpha$.
% Our ambient model is $\GG$.
% Notations we use are $\dim(\A), \dim(\A/\B), \A \leq \B$ as well as notions of $N$-strong substructure, minimal extension, chain minimal extension, minimal pair, and $N$-strong closure.


%%%%%%%%%%%%%%%%%%%%%%%%%%%%%%%% 

\section{VC-dimension and vc-density}

%%%%%%%%%%%%%%%%%%%%%%%%%%%%%%%% 



Throughout this section we work with a collection $\F$ of subsets of a set $X$.
We call the pair $(X, \F)$ a \defn{set system}.

\begin{Definition} \ 
  \begin{itemize} 
  \item Given a subset $A$ of $X$, we define the set system $(A, A \cap \F)$
    where $A \cap \F = \curly{A \cap F \mid F\in \F}$.
  \item For $A \subset X$ we say that $\F$ \defn{shatters} $A$ if $A \cap \F = \PP(A)$ (the power set of $A$).
  \end{itemize}    
\end{Definition}  

\begin{Definition}
  We say $(X, \F)$ has \defn{VC-dimension} $n$ if the largest subset of $X$ shattered by $\F$ is of size $n$.
  If $\F$ shatters arbitrarily large subsets of $X$, we say that $(X, \F)$ has infinite VC-dimension.
  We denote the VC-dimension of $(X, \F)$ by $\VC(X, \F)$.
\end{Definition}  

\begin{Note}
  We may drop $X$ from the notation $\VC(X, \F)$, as the VC-dimension doesn't depend on the base set and is determined by $(\bigcup \F, \F)$.
\end{Note}
Set systems of finite VC-dimension tend to have good combinatorial properties,
and we consider set systems with infinite VC-dimension to be poorly behaved.

Another natural combinatorial notion is that of a dual system:
\begin{Definition}
  For $a \in X$ define $X_a = \curly{F \in \F \mid a \in F}$.
  Let $\F^* = \curly{X_a \mid a \in X}$.
  We call $(\F, \F^*)$ the \defn{dual system} of $(X, \F)$.
  The VC-dimension of the dual system of $(X, \F)$ is referred to as the \defn{dual VC-dimension} of $(X, \F)$ and denoted by $\VC^*(\F)$.
  (As before, this notion doesn't depend on $X$.)
\end{Definition}  

\begin{Lemma} [see 2.13b in \cite{ash7}]
  A set system $(X, \F)$ has finite VC-dimension if and only if its dual system has finite VC-dimension.
  More precisely
  \begin{align*}
    \VC^*(\F) \leq 2^{1+\VC(\F)}.
  \end{align*}
\end{Lemma}

For a more refined notion of complexity of $(X, \F)$ we look at the traces of our family on finite sets:
\begin{Definition}
  Define the \defn{shatter function} $\pi_\F \colon \N \arr \N$ of $\F$ and the \defn{dual shatter function} $\pi^*_\F \colon \N \arr \N$ of $\F$ by 
  \begin{align*}
    \pi_\F(n) &= \max \curly{|A \cap \F| \mid A \subset X \text{ and } |A| = n} \\
    \pi^*_\F(n) &= \max \curly{\text{atoms($B$)} \mid B \subset \F, |B| = n}
  \end{align*}
  where atoms($B$) = number of atoms in the boolean algebra of sets generated by $B$.
  Note that the dual shatter function is precisely the shatter function of the dual system: $\pi^*_\F = \pi_{\F^*}$.
\end{Definition}  

A simple upper bound is $\pi_\F(n) \leq 2^n$ (same for the dual).
If the VC-dimension of $\F$ is infinite then clearly $\pi_\F(n) = 2^n$ for all $n$. Conversely we have the following remarkable fact:
\begin{Theorem} [Sauer-Shelah '72, see \cite{sauer}, \cite{shelah}]
  If the set system $(X, \F)$ has finite VC-dimension $d$ then $\pi_\F(n) \leq {n \choose \leq d}$ for all $n$, where
  ${n \choose \leq d} = {n \choose d} + {n \choose d - 1} + \ldots + {n \choose 1}$.    
\end{Theorem}

Thus the systems with a finite VC-dimension are precisely the systems where the shatter function grows polynomially.
Define the vc-density of $\F$ to quantify the growth of the shatter function of $\F$: 
\begin{Definition}
  Define the \defn{vc-density} and \defn{dual vc-density} of $\F$ as
  \begin{align*}
    \vc(\F) &= \limsup_{n \to \infty}\frac{\log \pi_\F(n)}{\log n} \in \R^{\geq 0} \cup \curly{+\infty},\\
    \vc^*(\F) &= \limsup_{n \to \infty}\frac{\log \pi^*_\F(n)}{\log n}\in \R^{\geq 0} \cup \curly{+\infty}.
  \end{align*}
\end{Definition}

Generally speaking a shatter function that is bounded by a polynomial doesn't itself have to be a polynomial.
Proposition 4.12 in \cite{density} gives an example of a shatter function that grows like $n \log n$ (so it has vc-density $1$).

So far the notions that we have defined are purely combinatorial.
We now adapt VC-dimension and vc-density to the model theoretic context.

\begin{Definition}
  Work in a first-order structure $M$.
  Fix a finite collection of formulas $\Phi(x, y)$.

  \begin{itemize}
  \item For $\phi(x, y) \in \LL(M)$ and $b \in M^{|y|}$ let 
    \begin{align*}
      \phi(M^{|x|}, b) = \{a \in M^{|x|} \mid \phi(a, b)\} \subseteq M^{|x|}.
    \end{align*}
  \item Let $\Phi(M^{|x|}, M^{|y|})= \{\phi(M^{|x|}, b) \mid \phi_i \in \Phi, b \in M^{|y|}\} \subseteq \PP(M^{|x|})$.
  \item Let $\F_\Phi = \Phi(M^{|x|}, M^{|y|})$, giving rise to a set system $(M^{|x|}, \F_\Phi)$.
  \item Define the \defn{VC-dimension} $\VC(\Phi)$ of $\Phi$, to be the VC-dimension of $(M^{|x|}, \F_\Phi)$, similarly for the dual.
  \item Define the \defn{vc-density} $\vc(\Phi)$ of $\Phi$, to be the vc-density of $(M^{|x|}, \F_\Phi)$, similarly for the dual.
  \end{itemize}

  We will also refer to the vc-density and VC-dimension of a single formula $\phi$
  viewing it as a one element collection $\Phi = \curly{\phi}$.
\end{Definition}

Counting atoms of a boolean algebra in a model theoretic setting corresponds to counting types,
so it is instructive to rewrite the shatter function in terms of types.

\begin{Definition} 
  \begin{align*}
    \pi^*_\Phi(n) &= \max \curly{\text{number of $\Phi$-types over $B$} \mid B \subset M, |B| = n}
  \end{align*}
  Here a $\Phi$-type over $B$ is a maximal consistent collection of formulas of the form $\phi(x, b)$ or $\neg\phi(x, b)$
  where $\phi \in \Phi$ and $b \in B$.
\end{Definition}

Functions $\pi^*_{\Phi}$ and $\pi^*_{\F_\Phi}$ are not equal, as one fixes the size of boolean algebra and another fixes the size of the parameter set.
However, as the following lemma demonstrates, they both give the same asymptotic definition of dual $\vc$-density.

\begin{Lemma} \label{count_types}
  \begin{align*}
    \vc^*(\Phi) &= \text{degree of polynomial growth of $\pi^*_\Phi(n)$}  = \limsup_{n \to \infty}\frac{\log \pi^*_\Phi(n)}{\log n}
  \end{align*}  
\end{Lemma}

\begin{proof}
  With parameter set of size $n$, we get $|\Phi|n$ elements in the boolean algebra.
  We check that asymptotically it doesn't matter whether we look at growth of boolean algebra of size $n$ or size $|\Phi|n$.
  \begin{align*}
    &\pi^*_{\F_\Phi}\paren{n} \leq \pi^*_\Phi(n) \leq \pi^*_{\F_\Phi}\paren{|\Phi|n} \\
    &\vc^*(\Phi) \leq \limsup_{n \to \infty}\frac{\log \pi^*_\Phi(n)}{\log n} \leq \limsup_{n \to \infty}\frac{\log \pi^*_{\F_\Phi}\paren{|\Phi|n}}{\log n} = \\
    & = \limsup_{n \to \infty}\frac{\log \pi^*_{\F_\Phi}\paren{|\Phi|n}}{\log |\Phi|n} \frac{\log |\Phi|n}{\log n} =
      \limsup_{n \to \infty}\frac{\log \pi^*_{\F_\Phi}\paren{|\Phi|n}}{\log |\Phi|n} \leq \\
    &\leq \limsup_{n \to \infty}\frac{\log \pi^*_{\F_\Phi}\paren{n}}{\log n} = \vc^*(\Phi)
  \end{align*}
\end{proof} 

One can check that the shatter function and hence VC-dimension and vc-density of a formula are elementary notions,
so they only depend on the first-order theory of the structure $M$.

NIP theories are a natural context for studying vc-density.
In fact we can take the following as the definition of NIP:
\begin{Definition}
  Define $\phi$ to be NIP if it has finite VC-dimension in a theory $T$.
  A theory $T$ is NIP if all the formulas in $T$ are NIP.
\end{Definition}

In a general combinatorial context for arbitrary set systems,
vc-density can be any real number in $0 \cup [1, \infty)$ (see \cite{ash8}).
Less is known if we restrict our attention to NIP theories.
Proposition 4.6 in \cite{density} gives examples of formulas that have non-integer rational vc-density in an NIP theory,
however it is open whether one can get an irrational vc-density in this model-theoretic setting.

Instead of working with a theory formula by formula, we can look for a uniform bound for all formulas:
\begin{Definition} \label{vc_fn_def}
  For a given NIP structure $M$, define the \defn{vc-function}
  \begin{align*}
    \vc^M(n) &= \sup \{\vc^*(\phi(x, y)) \mid \phi \in \LL(M), |x| = n\} \\
             &= \sup \{\vc(\phi(x, y)) \mid \phi \in \LL(M), |y| = n\} \in \R^{\geq 0} \cup \curly{+\infty}
  \end{align*}
\end{Definition}

As before this definition is elementary, so it only depends on the theory of $M$.
We omit the superscript $M$ if it is understood from the context.
One can easily check the following bounds:
\begin{Lemma} [Lemma 3.22 in \cite{density}] We have $\vc(1) \geq 1$ and $\vc(n) \geq n\vc(1)$.
  
\end{Lemma}

However, it is not known whether the second inequality can be strict or even whether $\vc(1) < \infty$ implies $\vc(n) < \infty$.


%%%%%%%%%%%%%%%%%%%%%%%%%%%%%%%%%%%%%%%%%%%%%%%%%%%%%%%%%%%%%%%%%%%%%%%%%%%%%%%%%%%%%%%%%%%%%%%%%%%%%%%%%%%%%%%%% 
\section{Graph Combinatorics}

We denote a graph by $\A$, the set of its vertices by $v(\A)$, and the set of its edges by $e(\A)$.
Number of vertices of $\A$ will be denoted as $|\A|$.
For two subgraphs $\A, \B$ of a larger graph, the union $\A \cup \B$ denotes the graph induced on $v(\A) \cup v(\B)$.
\begin{Definition}
  Fix $\alpha \in (0,1)$, irrational.
  \begin{itemize}
  \item For a finite graph $\A$ let $\dim(\A) = |\A| - \alpha |e(\A)|$.
  \item For finite $\A,\B$ with $\A \subseteq \B$ define $\dim(\B/\A) = \dim(\B) - \dim(\A)$.
  \item We say that $\A \leq \B$ if $\A \subseteq \B$ and $\dim(\A'/\B) > 0$ for all $\A \subseteq \A' \subsetneq \B$.
  \item We say that finite $\A$ is \defn{positive} if for all $\A' \subseteq \A$ we have $\dim(\A') \geq 0$.
  \item We work in theory $S_\alpha$ axiomatized by
    \begin{itemize}
    \item Every finite substructure is positive.
    \item For a model $\GG$ given $\A \leq \B$ every embedding $f : \A \arr \GG$ extends to $g: \B \arr \GG$.
    \end{itemize}
  \item For $\A, \B$ positive, $(\A, \B)$ is called a \defn{minimal pair} if
    $\A \subseteq \B$, $\dim(\B/\A) < 0$ but $\dim(\A'/\A) \geq 0$ for all proper $\A \subseteq \A' \subsetneq \B$.
  \item $\agl{\A_i}_{i \leq m}$ is called a \defn{minimal chain} if $(\A_i, \A_i+1)$ is a minimal pair (for all $i < m$).
  \item For a positive $\A$ let $\dim_\A(\bar x)$ be the atomic diagram of $\A$. For positive $\A \subseteq \B$ let 
    \begin{align*}
      \Psi_{\A,\B}(\bar x) = \dim_\A(\bar x) \wedge \exists \bar y \; \dim_\B(\bar x, \bar y).
    \end{align*}
    Such formula is called a \defn{chain-minimal extension formula} if in addition we have that there is a minimal chain starting at 
    $\A$ and ending in $\B$.
    Denote such formulas as $\Psi_{\agl{\M_i}}$.
  \end{itemize}
\end{Definition}

\begin{Theorem} [5.6 in \cite{laskowski}]
  $S_\alpha$ admits quantifier elimination down to boolean combination of chain-minimal extension formulas.
\end{Theorem}

Fix $\GG$, an ambient structure satisfying $S_\alpha$.

\begin{Definition}
  A graph $S \subseteq \GG$ is called \defn{$N$-strong} if for any $S \subseteq T \subseteq \GG$ with $|T| - |S| \leq N$ we have $S \leq T$.
\end{Definition}

%%%%%%%%%%%%%%%%%%%%%%%%%%%%%%%%%%%%%%%%%%%%%%%%%%%%%%%%%%%%%%%%%%%%%%%%%%%%%%%%%%%%%%%%%%%%%%%%%%%%%%%%%%%%%%%%% 
\section{Basic Definitions and Lemmas}

Fix tuples $x = (x_1, \ldots x_n), y = (y_1, \ldots, y_m)$.
We refer to chain-minimal extension formulas as basic formulas.
Let $\phi_{\agl{\M_i}}(x, y)$ be a basic formula.

\begin{Definition}
  Define $\X$ to be the graph on vertices $\{x_i\}$ with edges as defined by $\phi_{\agl{\M_i}}$.
  Similarly define $\Y$.
  We define those abstractly, i.e. on a new set of vertices disjoint from $\GG$.
\end{Definition}

Note that $\X$, $\Y$ are positive as they are subgraphs of $\M_0$.
As usual $X, Y$ will refer to vertices of those graphs.

We restrict our attention to formulas that define no edges between $X$ and $Y$.

\begin{Note} \label{note_edges}
  We can handle edges between $x$ and $y$ as separate elements of the minimal chain extension.
\end{Note}

\begin{Definition} \label{def_basic}
  For a basic formula $\phi = \phi_{\agl{\M_i}_{i \leq k}}(x, y)$ let
  \begin{itemize}
  \item $\epsilon_i(\phi) = -\dim \paren{M_i/M_{i-1}}$.
  \item $\epsilon_L(\phi) = \sum_{[1..k]} \epsilon_i(\phi)$.
  \item $\epsilon_U(\phi) = \min_{[1..k]} \epsilon_i(\phi)$.
  \item Let $\Y'$ be a subgraph of $\Y$ induced by vertices of $\Y$ that are connected to $M_k - (X \cup Y)$.
  \item Let $Y(\phi) = \dim (\Y')$.
    In particular if $\Y = \Y'$ and $\Y$ is disconnected then $Y(\phi)$ is just the arity of the tuple $y$.
  \end{itemize}
\end{Definition}

We will require the following lemmas from \cite{laskowski}:

\begin{Lemma} \label{diamond} [See 2.3 in \cite{laskowski}]
  Let $A, B \subseteq D$.
  Then
  \begin{align*}
    \dim{A \cup B / A} \leq \dim{\B / A \cap B}.
  \end{align*}
  Moreover, 
  \begin{align*}
    \dim{A \cup B / A} = \dim{\B / A \cap B} - \alpha E,
  \end{align*}
  where $E$ is the number of edges connecting vertices of $A \cup B - A$ to vertices of $A - A \cap B$.
\end{Lemma}

\begin{Lemma} \label{las_min} [See 4.1 in \cite{laskowski}]
  Suppose $A$ is a positive graph, with at least $1/\alpha + 2$ vertices.
  Then for any $\epsilon > 0$ there exists a graph $B$ such that $(A, B)$ is a minimal pair with dimension $\leq \epsilon$.
  Moreover every vertex in $A$ is connected to a vertex in $B - A$.
\end{Lemma}

\begin{Lemma} \label{las_str} [See 4.4 in \cite{laskowski}]
  Suppose $A$ is a positive graph, and $G$ a model of $S_\alpha$.
  Then for any integer $S$ there exists an embedding $f \colon A \arr G$ such that $f(A)$ is $S$-strong in $G$.
\end{Lemma}
    
We conclude this section by stating a couple of technical lemmas that will be useful in our proofs later.

\begin{Lemma} \ref{minimal_over_set}
  Work in $\GG$.
  Suppose we have a set $B$ and a minimal pair $(A, M)$ with $A \subseteq B$ and $\dim(M/A) = -\epsilon$.
  Then either $M \subseteq B$ or $\dim((M \cup B)/B) < -\epsilon$.
\end{Lemma}

\begin{proof}
  By Lemma \ref{diamond}
  \begin{align*}
    \dim((M \cup B)/B) \leq \dim(M / (M \cap B))
  \end{align*}
  and as $A \subseteq M \cap B \subseteq M$
  \begin{align*}
    \dim (M/A) = \dim(M / (M \cap B)) + \dim((M \cap B) / A).
  \end{align*}
  In addition we are given $\dim (M/A) = -\epsilon$.
  If $M \not\subseteq B$ then $A \subseteq M \cap B \subsetneq M$ and by minimality $\dim((M \cap B) / A) > 0$.
  Combining the inequalities above we obtain the desired result:
  \begin{align*}
    \dim((M \cup B)/B) \leq \dim(M / (M \cap B)) = \dim (M/A) - \dim((M \cap B) / A) < -\epsilon.
  \end{align*}
\end{proof}



\begin{Lemma}	\label{chain_lemma}
  Suppose we have a set $B$ and a minimal chain $M_n$ with $M_0 \subseteq B$ and dimensions $-\epsilon_i$.
  Let $\epsilon$ be the minimal of $\epsilon_i$.
  Then either $M_n \subseteq B$ or $\dim((M_n \cup B)/B) < -\epsilon$.
\end{Lemma}


\begin{proof}
  Let $\bar M_i = M_i \cup B$. Then:
  \begin{align*}
    \dim(\bar M_n/B) = \dim(\bar M_n/\bar M_{n-1}) + \ldots + \dim(\bar M_2/\bar M_1) + \dim(\bar M_1/B).
  \end{align*}
  Either $M_n \subseteq B$ or at least one of the summands above is nonzero.
  Apply previous lemma.
\end{proof}

\begin{Lemma} \label{minimal_subset}
  Suppose we have a minimal pair $(A, M)$ with dimension $-\epsilon$.
  Suppose we have some $B \subseteq M$.
  Then $\dim B / (A \cap B) \geq -\epsilon$.
  Moreover if $B \cup A \neq M$ then $\dim B / (A \cap B) \geq 0$
\end{Lemma}

\begin{proof}
  We have $\dim (B \cup A / A) \leq \dim B / (A \cap B)$ by Lemma \ref{diamond}.
  As $A \subseteq B \cup A \subseteq M$ we have $\dim (B \cup A / A) \geq -\epsilon$ by minimality.
  Moreover, minimality implies that it is positive if $B \cup A \neq M$.
\end{proof}

\begin{Lemma} \label{chain_intersect}
  Suppose we have a minimal chain $M_n$ with dimensions $-\epsilon_i$.
  Let $\epsilon$ be the sum of all $\epsilon_i$.
  Suppose we have some $B$ with $B \subseteq M_n$.
  Then $\dim B / (M_0 \cap B) \geq -\epsilon$.
\end{Lemma}

\begin{proof}
  Let $B_i = B \cap M_i$.
  We have $\dim B_{i+1}/B_i \geq \dim M_{i+1}/M_i$ by the previous lemma.
  $\dim B / (M_0 \cap B) = \dim B_n / B_0 = \sum \dim B_{i+1}/B_i \geq -\epsilon$.
\end{proof}

%%%%%%%%%%%%%%%%%%%%%%%%%%%%%%%%%%%%%%%%%%%%%%%%%%%%%%%%%%%%%%%%%%%%%%%%%%%%%%%%%%%%%%%%%%%%%%%%%%%%%%%%%%%%%%%%% 
\section{Lower bound}

In this section restrict our attention to the following family of the basic formulas $\phi(x,y)$:
\begin{itemize}
%\item Graphs defined by $x,y$ are $\X, \Y$.
\item All formulas have $\Y' = \Y$ (see Definition \ref{def_basic}).
\item All formulas define no edges between $X$ and $Y$.
\item Minimal chain of $\phi(x,y)$ consists of one step, that is we only have minimal extension as opposed to a chain of minimal extensions.
\item Dimension of that minimal extension is smaller than $\alpha$.
\end{itemize}

We obtain a lower bound for the formulas that are boolean combinations of basic formulas written in disjunctive-conjunctive form.
First, extend our definition of $\epsilon$.

\begin{Definition}[Negation]
  If $\phi$ is a basic formula, then define
  \begin{align*}
    \epsilon_L(\neg \phi) &= \epsilon_L(\phi)
  \end{align*}
\end{Definition}

\begin{Definition}[Conjunction]
  Take a collection of formulas $\phi_i(x, y)$ where each $\phi_i$ is positive or negative basic formula.
  If both positive and negative formulas are present then $\epsilon_L(\phi) = \infty$.
  We don't have a lower bound for that case.
  If different formulas define $\X$ or $\Y$ differently then $\epsilon_L(\phi) = \infty$.
  In that case of the conflicting definitions would make the formula have no realizations.
  Otherwise
  \begin{align*}
    \epsilon_L(\bigwedge \phi_i) &= \sum \epsilon_L(\phi_i)
  \end{align*}
\end{Definition}

\begin{Definition} [Disjunction]
  Take a collection of formulas $\psi_i$ where each instance is a conjunction of positive and negative instances of basic formulas that agree on $\X$ and $\Y$.
  \begin{align*}
    \epsilon_L(\bigvee \psi_i) &= \min \epsilon_L(\psi_i).
  \end{align*}
\end{Definition}

\begin{Theorem}
  For a formula $\phi$ as above
  \begin{align*}
    \vc \phi \geq \floor{\frac{Y(\phi)}{\epsilon_L(\phi)}}
  \end{align*}
  where $Y(\phi)$ is $Y(\psi)$ for $\psi$ one the basic components of $\phi$ (all basic componenets agree on $\Y$).
\end{Theorem}

\begin{proof}
  First, work with a formula that is a conjunction of positive basic formulas $\psi = \bigwedge_{i \in I} \phi_i$.
  Then as we defined above
  \begin{align*}
    \epsilon_L(\psi) = \sum \epsilon_L(\phi_i)
  \end{align*}
  %Let $\phi$ be one of the basic formulas in $\psi$ with a chain $\agl{M_i}_{i \leq k}$.
  %Let $K_\phi = |M_k|$ i.e. the size of the extension.
  %Let $K$ be the largest such size among all $\phi_i$.
  Let $n_1$ be the largest natural number such that
  \begin{align*}
    n_1 \epsilon_L(\psi) < Y.
  \end{align*}
  Let $\epsilon'$ be the smallest value among $\epsilon_L(\phi_i)$ corresponding to the formula $\phi'$.
  Let $n_2$ be the largest natural number such that
  \begin{align*}
    n_1 \epsilon_L(\psi) + n_2 \epsilon' < Y.
  \end{align*}

  Fix some $N > n$.
  Let $a_j$ be a graph isomorphic to $\X$ for each $1 \leq j \leq N$.
  Let $A = \bigsqcup_{1 \leq j \leq N} a_j$.
  Let $S = ??$.

  By Lemma \ref{las_str} $A$ can be embedded into $\GG$ as a $S$-strong graph. 
  Abusing notation, we identify $A$ with this embedding.
  Thus we have $A \subseteq \GG$, $S$-strong. 

  Let $J_1$ be the index set enumerating first $n_1$ natural numbers,
  $J_2$ enumerating the following $n_2$ numbers.%, $J_3$ -- the following $n_3$ numbers.
  Let $b$ be a graph isomorphic to $\Y$.
  For each $i \in I, j \in J_1$ let $W_{ij}$ be a witness of $\phi_i(a_j, b)$.
  For each $j \in J_1$ let $W_j$ be a union of $\curly{W_{ij}}_{i \in I}$ disjoint over $a_j, b$.
  For each $j \in J_2$ let $W_{j}$ be a witness of $\phi'(a_j, b)$.
  Let $W_1$ be a union of
  \begin{align*}
    \curly{W_j}_{j \in J_1 \cup J_2}
  \end{align*}
  disjoint over $b$.
  Let $W$ be a union of $W_1$ and $A$ disjoint over $\curly{a_j}_{j \in J_1 \cup J_2}$.
  %Additionally, modify $W$ to include one edge from a vertex in $a_j$ to a vertex in $b$ for each $j \in J_3$. 

  \begin{Claim}
    $A \leq W$.
  \end{Claim}
  \begin{proof}
    Consider some $A \subsetneq B \subseteq W$.
    We need to show $\dim (B/A) > 0$
    Let $\bar A = A \cup b$.
    We have
    \begin{align*}
      \dim(B/A) = \dim(B/ B \cap \bar A) + \dim(B \cap \bar A / A).
    \end{align*}
    Let $B_{ij} = B \cap W_{ij} \subseteq W_{ij}$.
    Let $B_{j} = B \cap W_{j} \subseteq W_{j}$.
    To unify indices, relabel all the graphs above as $\curly{B_k}_{k \in K}$.
    %We have $B_k$'s are disjoint over $\bar A$ and their union is all of $B$.
    By construction of $W$ we have
    \begin{align*}
      \dim(B/ B \cap \bar A) = \sum_{k \in K} \dim(B_k/ B_k \cap \bar A)
    \end{align*}
    Fix $k$.
    We have $B_k \subset W_k$, where $W_k$ is a minimal extension over $M^k_0 = a \cup b$ for some $a \in A$.
    We have $\dim(B_k / B_k \cap \bar A) = \dim(B_k / a \cup (B \cap b))$.
    Let $\epsilon_k$ be the dimension of this minimal extension.

    Case 1: $B \cap b = b$.
    Then $M_0^k \subseteq B_k \subseteq W_k$ and $\dim(B_k / a \cup (B \cap b)) = \dim (B_k/M_0^k)$.
    By minimality of $(M_0^k, B_k)$ we have $\dim (B_k/M_0^k) \geq -\epsilon_k$.
    Thus
    \begin{align*}
      \dim(B/ B \cap \bar A) \geq - \sum_{k \in K} \epsilon_k = -\paren{n_1 \epsilon_L(\psi) + n_2 \epsilon'}.
    \end{align*}
    In addition
    \begin{align*}
      \dim(B \cap \bar A / A) = \dim (b) = Y(\psi).
    \end{align*}
    Combining the two, we get
    \begin{align*}
      \dim(B/A) \geq Y(\psi) - \paren{n_1 \epsilon_L(\psi) + n_2 \epsilon'},
    \end{align*}
    which is positive by construction of $n_1, n_2$ as needed.
    
    Case 2: $B \cap b \subsetneq b$.
    \begin{Claim}
    \begin{align*}
      \dim(B_k / B_k \cap \bar A) > 0
    \end{align*}
    \end{Claim}
    \begin{proof}
      Recall that $\dim(B_k / B_k \cap \bar A) = \dim(B_k / a \cup (B \cap b))$.
      First, suppose that $B_k \cup M_0^k \neq W_k$.
      Then by Lemma \ref{minimal_subset} we get the required inequality.
      Thus we may assume that $B_k \cup M_0^k = W_k$.
      By Lemma \ref{diamond} we have
      \begin{align*}
        \dim{B_k \cup M_0^k / M_0^k} = \dim{B_k / B_k \cap M_0^k} - \alpha E,
      \end{align*}
      where $E$ is the number of edges connecting vertices of $B_k \cup M_0^k - M_0^k$ to vertices of $M_0^k - B_k \cap M_0^k$.
      Noting that $B_k \cup M_0^k = W_k$, $\dim{W_k / M_0^k} = -\epsilon_k$, and $B_k \cap M_0^k = a \cup (B \cap b)$
      we may rewrite the equality above as
      \begin{align*}
        \dim{B_k / a \cup (B \cap b)} = \alpha E - \epsilon,
      \end{align*}
      and $E$ is the number of edges connecting vertices of $W_k - M_0^k$ to vertices of $M_0^k - a \cup (B \cap b)$.
      as $\Y = \Y'$ and $B \cap b \subsetneq b$ we must have $E > 0$.
      But then as $\alpha > \epsilon$ we have $\dim{B_k / a \cup (B \cap b)} > 0$ as needed.
    \end{proof}
    Now, recall that
    \begin{align*}
      \dim(B/A) = \dim(B \cap \bar A / A) + \sum_{k \in K} \dim(B_k/ B_k \cap \bar A)
    \end{align*}
    By the claim above each of $\dim(B_k/ B_k \cap \bar A) > 0$, thus
    \begin{align*}
      \dim(B/A) > \dim(B \cap \bar A / A)
    \end{align*}
    In addition
    \begin{align*}
      \dim(B \cap \bar A / A) = \dim (b \cap B) \geq 0,
    \end{align*}
    as $b$ is postive.
    Thus $\dim (B/A) > 0$ as needed.
  \end{proof}

  As $A \leq W$ and $A \subseteq \GG$, we can embed $W$ into $\GG$ over $A$.
  Abusing notation again, we identify $W$ with its embedding $A \leq W \subseteq \GG$.
  In particular, now we have $b \in \GG$.
  Also note that
  \begin{align*}
    \dim(W/A) &= Y(\psi) - \paren{n_1 \epsilon_L(\psi) + n_2 \epsilon'} \\
    |W| - |A| &\leq |b| + (n_1 + n_2) \sum_{i \in I} S_i
  \end{align*}

  \begin{Lemma}
    \begin{align*}
      \curly{a_j}_{j \in J_1} \subseteq \psi(A, b) \subseteq \curly{a_j}_{j \in J_1 \cup J_2}
    \end{align*}
  \end{Lemma}
  \begin{proof}
    First inclusion $\curly{a_j}_{j \in J_1} \subseteq \psi(A, b)$ is immediate from construction of $W$,
    as $W_{ij}$ witnesses that $\phi_i(a_j, b)$ holds.
    For the second inclusion, suppose that there is $a \in A - \curly{a_j}_{j \in J_1 \cup J_2}$ such that $\psi(a,b)$ holds.
    Let $W' \subseteq \GG$ be a witness of $\phi_1(a,b)$.
    First, note that the case $W' \subseteq W$ is impossible
    as there are no edges between $a$ and $W - a$, but there are edges between $a$ and $W' - a$.
    Thus assume $W' \not\subset W$.
    As $(a \cup b, W')$ is minimal, by Lemma \ref{minimal_over_set} we have $\dim (W' \cup W / W) < -\epsilon_1$.
    \begin{align*}
      \dim(W' \cup W / A) = \dim (W' \cup W / W) + \dim(W/A) < Y(\psi) - \paren{n_1 \epsilon_L(\psi) + n_2 \epsilon'} - \epsilon_1,
    \end{align*}
    which is negative by construction of $n_1, n_2$.
    Thus $A \not\leq W \cup W'$, as then it would have a positive dimension.
    Additionally,
    \begin{align*}
      |W' \cup W| - |A| \leq |W' - W| + |W| - |A| \leq S_1 + |b| + (n_1 + n_2) \sum_{i \in I} S_i \leq S,
    \end{align*}
    but then this contradicts that $A$ is $S$-strong, as then we would have $A \leq W \cup W'$.
  \end{proof}

  In the construction of $W$ we could have chosen indices $J_1, J_2$ arbitrarily, instead of at the beginning of $A$.
  In particular, say we let $J_2$ to be the last $n_2$ indices of $J$ and
  $J_1$ an arbitrary $n_1$-element subset of the first $N$ elements of $J$.
  Each of those choices would then yield a different trace $\psi(A, b)$ by the lemma above.
  Thus $\psi(A, M^{|y|}) \geq {N \choose n_1}$ and therefore $\vc(\psi) \geq n_1$.
  By definition of $n_1$ we have $n_1 = \floor{\frac{Y(\psi)}{\epsilon_L(\psi)}}$, so this proves the theorem for $\psi$.
 
  Now consider a formula which is a conjunction consisting of negative basic formulas
  \begin{align*}
    \psi = \bigwedge_{i \in I} \neg \phi_i
  \end{align*}
  Let
  \begin{align*}
    \bar \psi = \bigwedge_{i \in I} \phi_i
  \end{align*}

  Do the construction above for $\bar \psi$ and suppose its trace is $X \subseteq A$ for some $b$.
  Then over $b$ the same construction gives trace $(A - X)$ for $\psi$. Thus we get as many traces as above, and the same bound.
  
  Finally consider a formula which is a disjunction of formulas considered above.
  \begin{align*}
    \theta = \bigvee{k \in K} \psi_k
  \end{align*}
  Choose the one with the smallest $\epsilon_L$, say $\psi_k$, and repeat the construction above for $\psi_k$.
  Any trace we obtain is automatically a trace for $\theta$, and thus we get as many traces as above, and the same bound.
\end{proof}

\begin{Corollary}
  VC-function is infinite in Shelah-Spencer random graphs:
  \begin{align*}
    \vc(n) = \infty.
  \end{align*}
\end{Corollary}

\begin{proof}
  Let $A$ be a graph consisting of $1/\alpha + 2 + n$ disconnected vertices.
  Fix $\epsilon > 0$.
  By Lemma \ref{las_min}, there exists $B$ such that $(A, B)$ is minimal with dimension $\leq \epsilon$.
  Consider a basic formula $\psi_{A, B}(x, y)$ where $|x| = 1/\alpha + 2$ and $|y| = n$.
  Then by the theorem above $\vc (\psi_{A,B}) \geq \frac{n}{\epsilon}$.
  As $\epsilon$ was arbitrary, this finishes the proof.
\end{proof}



%%%%%%%%%%%%%%%%%%%%%%%%%%%%%%%%%%%%%%%%%%%%%%%%%%%%%%%%%%%%%%%%%%%%%%%%%%%%%%%%%%%%%%%%%%%%%%%%%%%%%%%%%%%%%%%%% 
\section{Upper bound}

We bound the number of types of some finite collection of formulas $\Psi(\vec x, \vec y) = \curly{\phi_i(\vec x, \vec y)}_{i\in I}$ over a parameter set $B$ of size $N$,
where $\phi_i$ is a basic formula.

Fix a formula $\phi$ from our collection.
Suppose it defines a minimal chain extension over $\{x, y\}$. 
Record the size of that extension as $K(\phi)$ and its total dimension $\epsilon(\phi) = \epsilon_U(\phi)$.
Define dimension of that formula $D(\phi) = |\vec y| \frac{K(\phi)}{\epsilon(\phi)}$
Define dimension of the entire collection as $D(\Psi) = \max_{i \in I} D(\phi_i)$

In general we have parameter set $B \subseteq \GG^{|y|}$, however without loss of generality we may work with
a parameter set $B^{|y|}$, with $B \subseteq \GG$.

Let $S = \floor{D(\Psi)}$.

For our proof to work we also need $B$ to be $S$-strong.
We can achieve this by taking (the unique) $S$-strong closure of $B$.
If size of $B$ is $N$ then the size of its closure is $O(N)$.	%elaborate
So without loss of generality we can assume that $B$ is $S$-strong.

\begin{Definition}
  A \defn{witness} of $a$ is a union of realizations of the existential formulas $\phi_i(a, b)$ for all $i, b$ so that the formula holds.
\end{Definition}

\begin{Definition}
  For sets $C, B$ define the boundary of $C$ over $B$
  \begin{align*}
    \partial(C, B) = \curly{b \in B \mid \text{there is an edge between $b$ and element of $C - B$}}
  \end{align*}
\end{Definition}

\begin{Definition}
  For each $a$ pick some $\bar M_a$ to be its witness.
  Define two quantities
  \begin{itemize}
  \item $\partial_a$ is the boundary $\partial(\bar M_a, B \cup a)$
  \item Suppose $G_1, G_2$ are some subgraphs of our model and $a_1 \subseteq G_1, a_2 \subseteq G_2$ finite tuples of vertices.
    Call $f \colon (G_1, a_1) \arr (G_2, a_2)$ a $\partial$-isomorphism if it is a graph isomorphism,
    $f$ and $f^{-1}$ are constant on $B$, and
    $f(a_1) = a_2$.
  \item Define $\II_a$ as the $\partial$-isomorphism class of $(\bar M_a, a)$.
  \end{itemize}
\end{Definition}

\begin{Lemma} \label {bound_trace}
  If $\II_{a_1} = \II_{a_2}$ then $a_1, a_2$ have the same $\Psi$-type over $B$.
\end{Lemma}

\begin{proof}
  Fix a $\partial$-isomorphism $f \colon (\bar M_{a_1}, a_1) \arr (\bar M_{a_1}, a_2)$.
  Suppose we have $\phi(a_1, b)$ for some $b \in B$.
  Pick witness of this existential formula $M_1 \subseteq \bar M_{a_1}$.
  Then $f(M_1)$ is a witness for  $\phi(a_2, b)$.
\end{proof}

Thus to bound the number of traces it is sufficient to bound the number of possibilities for $\II_a$.

\begin{Theorem} \label{main_bound}
  \begin{align*}
    |\partial_a| &\leq D(\Psi) \\ 
    |\bar M_b - \bar A| &\leq D(\Psi)
  \end{align*}
\end{Theorem}

\begin{Corollary}
  \begin{align*}
    \vc(\phi) \leq K(\phi) \frac{Y(\phi)}{\epsilon(\phi)}
  \end{align*}
\end{Corollary}

\begin{proof}
  We count possible $\partial$-isomorphism classes $\II_b$.
  Let $W = K(\phi) \frac{Y(\phi)}{\epsilon(\phi)}$.
  If the parameter set $A$ is of size $N$ then there are $N \choose W$ choices for boundary $\partial_b$.
  On top of the boundary there are at most $W$ extra vertices and $(2W)^2$ extra edges.
  Thus there are at most
  \begin{align*}
    W \cdot 2^{(2W)^2}
  \end{align*}
  configurations up to a graph isomorphism.
  In total this gives us 
  \begin{align*}
    {N \choose W} \cdot W \cdot 2^{(2W)^2} = O(N^W)
  \end{align*}
  options for $\partial$-isomorphism classes.
  By Lemma \ref{bound_trace} there are at most $O(N^W)$ many traces, giving the required bound.
\end{proof}

\begin{proof} \textit{(of Theorem \ref{main_bound})}
  Fix some $b$-trace $A_b$. Enumerate $A_b = \{a_1, \ldots, a_I\}$.

  Let $M_i / \{a_i, b\}$ be a witness of $\phi(a_i, b)$ for each $i \leq I$.
  Let $\bar M_i = \bigcup_{j < i} M_j$.
  Let $\bar M = \bigcup M_i$, a witness of $A_b$
  
  \begin{Claim}
    \begin{align*}
      &\abs{\partial(M_i M, \bar A) - \partial(M, \bar A)} \leq |M_i| = K(\phi)\\
      &\dim(M_i M \bar A / M \bar A) > -\epsilon(\phi)
    \end{align*}
  \end{Claim}
  
  \begin{Definition}
    $(j-1, j)$ is called a \defn{jump} if some of the following conditions happen
    \begin{itemize}
    \item New vertices are added outside of $\bar A$ i.e.
      \begin{align*}
        \bar M_j - \bar A \neq \bar M_{j-1} - \bar A
      \end{align*}
    \item New vertices are added to the boundary, i.e.
      \begin{align*}
        \partial(\bar M_j, \bar A) \neq \partial(\bar M_{j-1}, \bar A)
      \end{align*}
    \end{itemize}
  \end{Definition}

  \begin{Definition}
    We now let $m_i$ count all jumps below $i$
    % Let $d_i = \dim(\bar M_i/A)$.
    \begin{align*}
      m_i = \abs{\curly{j < i \mid (j-1, j) \text{ is a jump}}}
    \end{align*}
  \end{Definition}

  \begin{Lemma} \label{ub_lemma}
    \begin{align*}
      \dim(\bar M_i / \bar A) &\leq -m_i \cdot \epsilon(\phi) \\
      |\partial(\bar M_i, \bar A)| &\leq m_i \cdot K(\phi) \\
      |\bar M_j - \bar A| &\leq m_i \cdot K(\phi)
    \end{align*}
  \end{Lemma}

  \begin{proof} \textit{(of Lemma \ref{ub_lemma})}
    Proceed by induction.
    Second and third propositions are clear.
    For the first proposition base case is clear.
    
    Induction step.
    Suppose $\bar M_j \cap (A \cup b) = \bar M_{j+1}$ and $\partial(\bar M_j, A) = \partial(\bar M_{j+1}, A)$.
    Then $m_i = m_{i+1}$ and the quantities don't change.
    Thus assume at least one of these equalities fails.
    
    Apply Lemma \ref{chain_lemma} to $\bar M_j \cup (A \cup b)$ and $(M_{j+1}, a_{j+1}b)$.
    There are two options
    
    \begin{itemize}
    \item $\dim(\bar M_{j+1} \cup (A \cup b) / \bar M_i \cup (A \cup b)) \leq -\epsilon_U$.
      This implies the proposition.
    \item $M_{j+1} \subseteq \bar M_j \cup (A \cup b)$.
      Then by our assumption it has to be $\partial(\bar M_j, A) \neq \partial(\bar M_{j+1}, A)$.
      There are edges between $M_{j+1} \cap (\partial(\bar M_{j+1}, A) - \partial(\bar M_j, A))$ so they contribute some negative dimension $\leq \epsilon_U$.
    \end{itemize}
    This ends the proof for Lemma \ref{ub_lemma}.
  \end{proof}
  \textit{(Proof of Theorem \ref{main_bound} continued)}
  First part of lemma \ref{ub_lemma} implies that we have $\dim(\bar M / \bar A) \leq -m_I \cdot \epsilon(\phi)$.
  The requirement of $A$ to be $S$-strong forces
  \begin{align*}
    m_I \cdot \epsilon(\phi) &< Y(\phi) \\
    m_I  &< \frac{Y(\phi)}{\epsilon(\phi)} \\
  \end{align*}
  % Let $W = \frac{K(\phi)Y(\phi)}{\epsilon(\phi)}$
  Applying the rest of \ref{ub_lemma} gives us
  \begin{align*}
    |\partial(\bar M, A)| &\leq m_I \cdot K(\phi) \leq \frac{K(\phi)Y(\phi)}{\epsilon(\phi)} \\
    |\bar M \cap A| &\leq m_I \cdot K(\phi) \leq \frac{K(\phi)Y(\phi)}{\epsilon(\phi)}
  \end{align*}
  as needed.
  This ends the proof for Theorem \ref{main_bound}.
\end{proof}

So far we have computed an upper bound for a single basic formula $\phi$.

To bound an arbitrary formula, write it as a boolean combination of basic formulas $\phi_i$ (via quantifier elimination)
It suffices to bound vc-density for collection of formulas $\{\phi_i\}$ to obtain a bound for the original formula.

In general work with a collection of basic formulas $\{\phi_i\}_{i \in I}$.
The proof generalizes in a straightforward manner.
Instead of $A^{|x|}$ we now work with $A^{|x|} \times I$ separating traces of different formulas.
Formula with the largest quantity $Y(\phi)\frac{K(\phi)}{\epsilon(\phi)}$ contributes the most to the vc-density.
Thus we have
\begin{align*}
  \Phi &= \{\phi_i\}_{i \in I} \\
  \vc(\Phi) &\leq   \max_{i \in I} Y(\phi_i) \frac{K(\phi_i)}{\epsilon_{\phi_i}}
\end{align*}

%%%%%%%%%%%%%%%%%%%%%%%%%%%%%%%%%%%%%%%%%%%%%%%%%%%%%%%%%%%%%%%%%%%%%%%%%%%%%%%%%%%%%%%%%%%%%%%%%%%%%%%%%%%%%%%%% 

\begin{thebibliography}{9}

\bibitem{density}
  M. Aschenbrenner, A. Dolich, D. Haskell, D. Macpherson, S. Starchenko,
  \textit{Vapnik-Chervonenkis density in some theories without the independence property}, I,
  Trans. Amer. Math. Soc. 368 (2016), 5889-5949
  
\bibitem{laskowski}
  Michael C. Laskowski, \textit{A simpler axiomatization of the Shelah-Spencer almost sure theories},
  Israel J. Math. \textbf{161} (2007), 157–186. MR MR2350161	

\bibitem{ash7}
  P. Assouad, \textit{Densit´e et dimension}, Ann. Inst. Fourier (Grenoble) 33 (1983), no. 3, 233-282.
\bibitem{ash8}
  P. Assouad, \textit{Observations sur les classes de Vapnik-Cervonenkis et la dimension combinatoire de Blei},
  in: Seminaire d’Analyse Harmonique, 1983-1984, pp. 92-112, Publications Math´ematiques
  d’Orsay, vol. 85-2, Universit´e de Paris-Sud, D´epartement de Math´ematiques, Orsay, 1985.
\bibitem{sauer}
  N. Sauer, \textit{On the density of families of sets}, J. Combinatorial Theory Ser. A 13 (1972), 145-147.
\bibitem{shelah}
  S. Shelah, \textit{A combinatorial problem; stability and order for models and theories in infinitary languages},
  Pacific J. Math. 41 (1972), 247-261.

\end{thebibliography}

\end{document}

