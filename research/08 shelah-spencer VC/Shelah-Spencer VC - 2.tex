\documentclass{amsart}

\usepackage{../AMC_style}	
\usepackage{../Research}

  \newcommand{\A}{\mathcal A}
  \newcommand{\B}{\mathcal B}
\renewcommand{\C}{\mathcal C}
  \newcommand{\D}{\mathcal D}
\renewcommand{\H}{\mathcal H}
  \newcommand{\G}{\mathcal G}
  \newcommand{\M}{\mathcal M}

  \newcommand{\K}{\boldface K_\alpha}
\renewcommand{\S}{S_\alpha}

\newcommand{\curly}[1]{\left\{#1\right\}}
\newcommand{\paren}[1]{\left(#1\right)}
\newcommand{\abs}[1]{\left|#1\right|}

%\DeclareMathOperator{\dim}{dim}

\begin{document}

\title{Some vc-density computations in Shelah-Spencer graphs}
\author{Anton Bobkov}
\email{bobkov@math.ucla.edu}

Fix a formula $\phi(x, y)$ that is a minimal extension $M/\curly{x,y}$. 
\begin{itemize}
	\item $\dim \paren{M/\curly{x,y}} = -\epsilon$
	\item there are no edges between $x$ and $y$.
	\item there are no edges between $x$.
\end{itemize}

Let $Y = \dim (y)$

Let $n$ be such that $n\epsilon < Y$ but $(n+1)\epsilon > Y$.
Fix a parameter set $A$, strongly embedded and disconnected (thus indiscernible).

\section{Lower bound}

Pick a finite $B \subset A^{|x|}$.

Consider the graph $x \cup y$.
If $y/x$ is not a proper extension, then $\phi$ has no realizations over $B$.
If it is, abstractly make a realization of $y$, label it by $b$.

Fix arbitrary elements of $B$, label them $a_i$ for $i=[0..n]$, with each $|a_i| = |x|$.
Abstractly adjoin $M_i/\curly{a_i, b} = M/\curly{x,y}$ for each $i$.
Let $\bar M = \bigcup M_i$.

Claim: $A \leq \bar M$.
It's total dimension is $Y - n\epsilon > 0$ and all subextensions are positive as well.

Thus a copy of $\bar M$ can be embedded over $A$ into our ambient model.
Choice of elements of $B$ was arbitrary, thus showing that any $n$ elements can be traced out.
Thus we have $O(|B|^n)$ many traces showing vc-density of $n$.

\section{Upper bound}

Pick a trace of $\phi(x,y)$ on $A^{|x|}$ by a parameter $b$.

\begin{align*}
	B = \curly{a \in A^{|x|} \mid \phi(a, b)}
\end{align*}

Pick $B' \subset B$, ordered $B' = \{a_1, \ldots\}$ such that
\begin{align*}
	a_i \cap \bigcup_{j \neq i} a_j \neq \emptyset
\end{align*}
This is always possible by starting with $B$ and taking away elements one by one.
Call such a set a \emph{generating set} of $B$.

Let $M_i / \{a_i, b\}$ be a witness of $\phi(a_i, b)$.
Let $\bar M = \bigcup M_i$.
Consider $\bar M / A$.

Claim: $\dim(\bar M / A)$ is minimized when all $M_i$ are disjoint.
Suppose not.
Suppose there is $j$ such that
\begin{align*}
	M_j \cap \bigcup_{i \neq j} M_i \neq \emptyset
\end{align*}

Apply the key lemma to see that making it disjoint would reduce dimension contradicting minimality.

Thus as $A$ is strong we need $|B'| \epsilon < Y$.
This gives us $|B'| \leq n$.
Finally we need to relate $|B'|$ to $|B|$.

Suppose we have $C \subset A^{|x|}$, finite with $|C| = N$.
A generating set for a trace has to have size $\leq n$.
Thus there are ${N \choose n} \leq N^n$ choices for a generating set.
A set generated from set of size $n$ can have at most $(x|n|)^{|x|}$ elements.
Thus a given set of size $n$ can generate at most
\begin{align*}
	2^{(x|n|)^{|x|}}
\end{align*}
sets.
Thus the number of possible traces on $C$ is bounded above by
\begin{align*}
  2^{(x|n|)^{|x|}} \cdot N^n = O(N^n)
\end{align*}
This bounds the vc-density by $n$.

Lemma

Suppose we have a set $B$ and a minimal pair $(M, A)$ with $A \subset M$ and $\dim(M/A) = -\epsilon$.
Then either $M \subseteq B$ or $\dim((M \cup B)/B) \leq -\epsilon$.

Proof

By diamond construction

\begin{align*}
	\dim((M \cup B)/B) \leq \dim(M / (M \cap B))
\end{align*}

and 

\begin{align*}
	\dim(M / (M \cap B)) &= \dim (M/A) - \dim(M / (M \cap B)) \\
	\dim (M/A) &= -\epsilon \\
	\dim(M / (M \cap B)) &> 0
\end{align*}


Lemma

Suppose we have a set $B$ and a minimal chain $M_n$ with $M_0 \subset B$ and dimensions $-\epsilon_i$.
Let $\epsilon$ be the minimal of $\epsilon_i$.
Then either $M_n \subseteq B$ or $\dim((M_n \cup B)/B) \leq -\epsilon$.

Proof

Apply previous lemma.
\end{document}
