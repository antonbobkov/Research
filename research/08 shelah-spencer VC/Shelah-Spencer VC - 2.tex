\documentclass{amsart}

\usepackage{../AMC_style}	
\usepackage{../Research}
\usepackage{../Thm}

\usepackage{mathrsfs}



\renewcommand{\AA}{\mathscr A}
  \newcommand{\II}{\mathscr I}
  \newcommand{\MM}{\mathscr M}

  \newcommand{\A}{\mathcal A}
  \newcommand{\B}{\mathcal B}
\renewcommand{\C}{\mathcal C}
  \newcommand{\D}{\mathcal D}
\renewcommand{\H}{\mathcal H}
  \newcommand{\G}{\mathcal G}
  \newcommand{\M}{\mathcal M}

  \newcommand{\U}{\mathcal U}	

  \newcommand{\K}{\boldface K_\alpha}
\renewcommand{\S}{S_\alpha}

\newcommand{\curly}[1]{\left\{#1\right\}}
\newcommand{\paren}[1]{\left(#1\right)}
\newcommand{\abs}[1]{\left|#1\right|}

\providecommand{\floor}[1]{\left \lfloor #1 \right \rfloor }

%\DeclareMathOperator{\dim}{dim}

\title{Some vc-density computations in Shelah-Spencer graphs}
\author{Anton Bobkov}
\email{bobkov@math.ucla.edu}

\begin{document}

\maketitle

%%%%%%%%%%%%%%%%%%%%%%%%%%%%%%%%%%%%%%%%%%%%%%%%%%%%%%%%%%%%%%%%%%%%%%%%%%%%%%%%%%%%%%%%%%%%%%%%%%%%%%%%%%%%%%%%%
\section{Preliminaries}

VC density was introduced in \cite{vc_density} by Aschenbrenner, Dolich, Haskell, MacPherson, and Starchenko as a natural notion of dimension for NIP theories. In a NIP theory we can define a VC function

\begin{align*}
	\vc : \N \arr \N
\end{align*}

Where $vc(n)$ measures complexity of definable sets in an $n$-dimensional space. Simplest possible behavior is $\vc(n) = n$ for all $n$. Theories with that property are known to be dp-minimal, i.e. having the smallest possible dp-rank. In general, it is not known whether there can be a dp-minimal theory which doesn't satisfy $\vc(n)=n$.

In this paper, we investigate vc-density of definable sets in Shelah-Spencer structures.
We follow notations in \cite{laskowski}.
In this paper we work with limit of random structure $G(n, n^{-\alpha})$ for $\alpha \in (0,1)$, irrational.
This structure is axiomatized by $S_\alpha$.
Our ambient model is $\MM$.
Notations we use are $\delta(\A), \delta(\A/\B), \A \leq \B$ as well as notions of strong substructure, minimal extension, chain minimal extension.


%%%%%%%%%%%%%%%%%%%%%%%%%%%%%%%%%%%%%%%%%%%%%%%%%%%%%%%%%%%%%%%%%%%%%%%%%%%%%%%%%%%%%%%%%%%%%%%%%%%%%%%%%%%%%%%%%
\section{Definitions}

\begin{Definition}
	Let $x = (x_1, \ldots x_n), y = (y_1, \ldots, y_m)$ be variable tuples.
	We call a formula $\phi(x, y)$ \emph{basic} when
	\begin{itemize}
		\item $\phi(x, y)$ is a minimal chain extension, denoted by $\curly{M_i}_{i \in [0..k]}$ with $M_0 = \{x, y\}$.
		\item $\phi(x, y)$ determines edges and non-edges on its variables $\{x_1, \ldots x_n\} \cup \{y_1, \ldots y_m\}$.
		\item there is no edge between $x_i$ and $y_j$ for all $i,j$. (see note \ref{note_edges})
		\item Define $\mathbf x$ to be the graph on vertices $\{x_i\}$ with edges as defined by $\phi$.
		Similarly define $\mathbf y$.
		We require $\mathbf x$ and $\mathbf y$ to be positive. (see note \ref{note_positive})\
		%\item all elements of $y$ that are connected to $M_k - \{x,y\}$. (see note \ref{note_special})
	\end{itemize}
\end{Definition}

% issue: multiple possible chain decompositions?

% address strength

% address edges between x and y

\begin{Note} \label{note_edges}
	We handle edges between $x$ and $y$ as separate elements of the minimal chain extension.
\end{Note}

\begin{Note} \label{note_positive}
	If either graph $\mathbf x$ or $\mathbf y$ is negative, then the formula would have no realizations as negative graphs cannot be embedded into our ambient model.
\end{Note}

%\begin{Note} \label{note_special}
	%We add the final condition to simplify our analysis. Similar techniques can be used to acquire bounds on formulas not subject to that condition.
%\end{Note}

\begin{Definition} \label{def_basic}
	For a basic formula $\phi(x, y)$ let
	\begin{itemize}
		\item $\epsilon_i(\phi) = -\dim \paren{M_i/M_{i-1}}$.
		%\item if $x$ or $y$ aren't positive, then $\epsilon_L(\phi) = \epsilon_U(\phi) = \infty$. Otherwise
		\item $\epsilon_L(\phi) = \sum_{[1..k]} \epsilon_i(\phi)$.
		\item $\epsilon_U(\phi) = \min_{[1..k]} \epsilon_i(\phi)$.
		\item Let $\mathbf y'$ be a subgraph of $\mathbf y$ induced by elements of $\{y_i\}$ that are connected to $M_k - (\{x_i\} \cup \{y_j\})$.
		\item Let $Y(\phi) = \dim (\mathbf y')$.
		In particular if $\mathbf y = \mathbf y'$ and $\mathbf y$ is disconnected then $Y$ is arity of $y$.
	\end{itemize}
\end{Definition}

%\begin{Definition}
%\begin{align*}
	%\epsilon_L(\neg \phi) &= \epsilon_L(\phi) \\
	%\epsilon_U(\neg \phi) &= \epsilon_U(\phi) \\
	%\epsilon_L(\phi \wedge \psi) &= \epsilon_L(\phi) + \epsilon_L(\psi) \\
	%\epsilon_U(\phi \wedge \psi) &= \min(\epsilon_U(\phi), \epsilon_U(\psi)) \\
	%\epsilon_L(\phi \vee \psi) &= \min(\epsilon_L(\phi), \epsilon_L(\psi)) \\
	%\epsilon_U(\phi \vee \psi) &= \min(\epsilon_U(\phi), \epsilon_U(\psi)) 
%\end{align*}
%\end{Definition}

%%%%%%%%%%%%%%%%%%%%%%%%%%%%%%%%%%%%%%%%%%%%%%%%%%%%%%%%%%%%%%%%%%%%%%%%%%%%%%%%%%%%%%%%%%%%%%%%%%%%%%%%%%%%%%%%%
\section{Lower bound}

We work with formulas that are boolean combinations of basic formulas written in disjunctive-conjunctive form.
Define dimensions for those inductively.

%For simplicity we assume that for all basic formulas in our collection all elements of $y$ that are connected to $M_k - \{x,y\}$. This simplifies our analysis.
As a simplification for our lower bound computation we assume that all the basic formulas involved we have $\mathbf y' = \mathbf y$ (see Definition \ref{def_basic}).

\begin{Definition}[Negation]
	If $\phi$ is a basic formula, then
	\begin{align*}
		\epsilon_L(\neg \phi) &= \epsilon_L(\phi)
	\end{align*}
\end{Definition}

\begin{Definition}[Conjunction]
	Take a collection of formulas $\phi_i(x, y)$ where each $\phi_i$ is positive or negative basic formula.
	If both positive and negative formulas are present then $\epsilon_L(\phi) = \infty$.
	We don't have a lower bound for that case.
	If different formulas define graphs for $x$ or $y$ differently then $\epsilon_L(\phi) = \infty$.
	In that case the conflicting definitions would make the formula have no realizations.
	Otherwise
	\begin{align*}
		\epsilon_L(\bigwedge \phi_i) &= \sum \epsilon_L(\phi_i)
	\end{align*}
\end{Definition}

\begin{Definition} [Disjunction]
	Take a collection of formulas $\psi_i$ where each instance is a conjunction of positive and negative instances of basic formulas.
	\begin{align*}
		\epsilon_L(\bigvee \psi_i) &= \min \epsilon_L(\psi_i)
	\end{align*}
\end{Definition}

\begin{Theorem}
	For a formula $\phi$ as above
	\begin{align*}
		\vc \phi \geq \floor{\frac{Y(\phi)}{\epsilon_L(\phi)}}
	\end{align*}
\end{Theorem}

%First we discuss some cases with $\epsilon_L$ infinite.

%If $x$ or $y$ are not positive, then the formula doesn't have any realizations.

%Consider a formula in conjunctive-disjunctive form. Look at individual conjunctive components. If there are positive and negative components, then the formula has no realizations over sufficiently large parameter sets, as will be shown later. So we only consider only components where all the formulas are positive or all the formulas are negative.

%Each basic formula has to describe graph of $x$.
%If there are two formulas in the conjunct that disagree on that, then there are no realizations of the conjunction.

\begin{proof}
	First work with a formula that is a conjunction of positive basic formulas.

	\begin{align*}
		\psi = \bigwedge \phi_j
	\end{align*}
	Then as we defined above
	\begin{align*}
		\epsilon_L(\psi) = \sum \epsilon_L(\phi_j)
	\end{align*}

	Let $n$ be the integer such that $n \epsilon_L(\psi) < Y$ and $(n+1) \epsilon_L(\psi) > Y$.

	Take an abstract realization of $y$ as dictated by $\psi$, and label it by $b$.

	Pick parameter set 

	\begin{align*}
		A = \bigcup_{i<N} b_i
	\end{align*}

	a disjoint union where each $b_i$ is an ordered tuple of size $|x|$ connected according to $\psi$.
	We also require $A$ to be strong.

	Fix $n$ arbitrary elements out of $b_i$, label them $a_i$.

	Fix an individual formula $\phi \in \Phi$, with minimal sequence $M_i$.

	Abstractly adjoin $M_i/\curly{a_i, b} = M/\curly{x,y}$ for each $i$.
	Let $\bar M_\phi = \bigcup M_i$ (disjointly).

	We can join those for all $\phi \in \Phi$.
	Let $\bar M = \bigcup M_\phi$ (disjointly).

	\begin{Claim}
		$(A \cap \bar M) \leq \bar M$.
	\end{Claim}
	\begin{proof}
		Its total dimension is $Y - n\epsilon_L(\psi) > 0$ and all its subextensions are positive as well.
	\end{proof}

	Thus a copy of $\bar M$ can be embedded over $A$ into our ambient model.
	Our choice of $b_i$'s was arbitrary, so we get ${N \choose n}$ choices out of $N|x|$ many elements.
	Thus we have $O(|A|^n)$ many traces.

	\begin{Lemma}
		There are arbitrarily large sets with properties of $A$.
	\end{Lemma}

	This shows

	\begin{align*}
		\vc \psi \geq n = \floor{\frac{Y}{\epsilon_L}}
	\end{align*}

	Now consider the formula which is a conjunction consists of negative basic formulas
	\begin{align*}
		\psi = \bigwedge \neg \phi_i
	\end{align*}
	Let
	\begin{align*}
		\bar \psi = \bigwedge \phi_i
	\end{align*}

	Do the construction above for $\bar \psi$ and suppose its trace is $X \subset A$ for some $b$.
	Then over $b$ the same construction gives trace $(A - X)$ for $\psi$. Thus we get as many traces.
	
	Finally consider a formula which is a disjunction of formulas considered above.
	Choose the one with the smallest $\epsilon_L$, this yields the lower bound for the entire formula. %explain!
\end{proof}

\begin{Claim}
	We can find a minimal extension $M / \{x, y\}$ with arbitrarily small dimension.
\end{Claim}

This shows that vc function is infinite in Shelah-Spencer random graphs.

\begin{align*}
	\vc(n) = \infty
\end{align*}

%%%%%%%%%%%%%%%%%%%%%%%%%%%%%%%%%%%%%%%%%%%%%%%%%%%%%%%%%%%%%%%%%%%%%%%%%%%%%%%%%%%%%%%%%%%%%%%%%%%%%%%%%%%%%%%%%
\section{Upper bound}


%\begin{Definition}
	%Let $\phi$ be a basic formula with $M_i$ a minimal chain,
	%$\epsilon_i$ its corresponding dimensions,
	%and $M$ its total size.
	%\begin{align*}
		%U_\phi = \frac{M}{\min \epsilon_i}
	%\end{align*}
%\end{Definition}



%\begin{Definition} [Negation]
	%Let $\phi$ be basic
	%\begin{align*}
		%U_{\neg \phi} = U_{\phi}
	%\end{align*}
%\end{Definition}
%
%\begin{Definition} [Conjunction and Disjunction]
	%Let $\phi_{ij}$ be basic or a negation of a basic formula.
	%\begin{align*}
		%\psi = \bigvee \bigwedge \phi_{ij}
	%\end{align*}
	%\begin{align*}
		%U_\psi = \max U_{\phi_{ij}}
	%\end{align*}
%\end{Definition}

%Let $n$ be the integer such that $n \epsilon_U < Y$ and $(n+1) \epsilon_U > Y$.

%Consider a formula which is a conjunction of positive basic formulas.

Consider a case of a single basic formula $\phi(\vec x, \vec y)$.

Suppose it defines a minimal chain extension $\{C_i\}$ over $\{x, y\}$. 
Record the size of that extension as $K(\phi)$ and its total dimension $\epsilon(\phi) = \epsilon_U(\phi)$.

In general we have parameter set $A \subset \MM^{|x|}$, however without loss of generality we may work with
a parameter set $A^{|x|}$, with $A \subset \MM$.

Let $S = \floor{\frac{K(\phi)Y(\phi)}{\epsilon(\phi)}}$ (dependent only on phi).

For our proof to work we also need $A$ to be $n$-strong.
We can achieve this by taking (the unique) $n$-strong closure of $A$.
If size of $A$ is $N$ then the size of its closure is $O(N)$.
So without loss of generality we can assume that $A$ is $n$-strong.

\begin{Definition}
	Define a $b$-trace of $\phi$ on $A$
	\begin{align*}
		A_b = \phi(A, b) = \curly{a \in A^{|x|} \mid \phi(a, b)}
	\end{align*}
\end{Definition}

Let $\bar A = A \cup b$. % what about A \cap b???

\begin{Definition}
	For a set $C$ define the boundary of $C$ over $\bar A$
	\begin{align*}
		\partial(C, \bar A) = \curly{a \in \bar A \mid \text{there is an edge between $a$ and element of $C - \bar A$}}
	\end{align*}
\end{Definition}

\begin{Definition}
	A \emph{witness} of $\phi(a, b)$ is a realization of the existential formula together with ${a_i, b}$.
\end{Definition}

\begin{Definition}
	For a trace $A_b = \{a_1, \ldots, a_I\}$ for each $\phi(a_i, b)$ pick a witness and then take a union of all those witnesses. Call this a witness of the trace $A_b$.	%uniqueness
\end{Definition}

For each $b$ we pick $\bar M_b$ to be a witness of $A_b$. We look at two quantities
\begin{itemize}
	\item $\II_b$ Isomorphism class of $\bar M_b - \bar A$
	\item $\partial_b$ Boundary $\partial(\bar M_b, \bar A)$
\end{itemize}

\begin{Lemma} \label {bound_trace}
	If $\II_{b_1} = \II_{b_2}$ and $\partial_{b_1} = \partial_{b_2}$ then $A_{b_1} = A_{b_2}$.
\end{Lemma}

Thus to bound the number of traces it is sufficient to bound the number of possibilities for $I_b$ and $\partial_b$.

\begin{Theorem} \label{main_bound}
	\begin{align*}
		|\II_b| &\leq K(\phi) \frac{Y(\phi)}{\epsilon_\phi}\\
		|\partial_b| &\leq K(\phi) \frac{Y(\phi)}{\epsilon_\phi}
	\end{align*}
\end{Theorem}

\begin{Corollary}
	\begin{align*}
		\vc(\phi) \leq K(\phi) \frac{Y(\phi)}{\epsilon_\phi}
	\end{align*}
\end{Corollary}

\begin{proof}
		Let $W = K(\phi) \frac{Y(\phi)}{\epsilon_\phi}$.
		Then the number of possible isomorphism classes of size $\leq W$ is $\leq 2^{W^2}$.
		If the parameter set $A$ is of size $N$ then there are $N \choose W$ choices for boundary.
		Using lemma \ref{bound_trace} this bounds the number of traces by $O(N^W)$.
		Thus the vc-density of $\phi$ is $W$ as needed.
\end{proof}

\begin{proof} \textit{(of Theorem \ref{main_bound})}
	Fix some $b$-trace $A_b$. Enumerate $A_b = \{a_1, \ldots, a_I\}$.

	Let $M_i / \{a_i, b\}$ be a witness of $\phi(a_i, b)$ for each $i \leq I$.
	Let $\bar M_i = \bigcup_{j < i} M_j$.
	Let $\bar M = \bigcup M_i$, a witness of $A_b$
	
	\begin{Claim}
		\begin{align*}
			&\abs{\partial(M_i M, \bar A) - \partial(M, \bar A)} \leq |M_i| = K(\phi)\\
			&\dim(M_i M \bar A / M \bar A) > -\epsilon_\phi
		\end{align*}
	\end{Claim}
	
	\begin{Definition}
		$(j-1, j)$ is called a \emph{jump} if some of the following conditions happen
		\begin{itemize}
			\item New vertices are added outside of $A$ i.e.
				\begin{align*}
					\bar M_j - A \neq \bar M_{j-1} - A
				\end{align*}
			\item New vertices are added to the boundary, i.e.
				\begin{align*}
					\partial(\bar M_j, A) \neq \partial(\bar M_{j-1}, A)
				\end{align*}
		\end{itemize}
	\end{Definition}

	\begin{Definition}
		We now let $m_i$ count all jumps below $i$
		%Let $d_i = \dim(\bar M_i/A)$.
		\begin{align*}
			m_i = \abs{\curly{j < i \mid (j-1, j) \text{ is a jump}}}
		\end{align*}
	\end{Definition}

	\begin{Lemma} \label{ub_lemma}
		\begin{align*}
			\dim(\bar M_i / \bar A) &\leq -m_i \cdot \epsilon_\phi \\
			|\partial(\bar M_i, \bar A)| &\leq m_i \cdot K(\phi) \\
			|\bar M_j - \bar A| &\leq m_i \cdot K(\phi)
		\end{align*}
	\end{Lemma}

	\begin{proof} \textit{(of Lemma \ref{ub_lemma})}
		Proceed by induction.
		Second and third propositions are clear.
		For the first proposition base case is clear.
		
		Induction step.
		Suppose $\bar M_j \cap (A \cup b) = \bar M_{j+1}$ and $\partial(\bar M_j, A) = \partial(\bar M_{j+1}, A)$.
		Then $m_i = m_{i+1}$ and the quantities don't change.
		Thus assume at least one of these equalities fails.
		
		Apply Lemma \ref{chain_lemma} to $\bar M_j \cup (A \cup b)$ and $(M_{j+1}, a_{j+1}b)$.
		There are two options
		
		\begin{itemize}
			\item $\dim(\bar M_{j+1} \cup (A \cup b) / \bar M_i \cup (A \cup b)) \leq -\epsilon_U$.
			This implies the proposition.
			\item $M_{j+1} \subset \bar M_j \cup (A \cup b)$.
			Then by our assumption it has to be $\partial(\bar M_j, A) \neq \partial(\bar M_{j+1}, A)$.
			There are edges between $M_{j+1} \cap (\partial(\bar M_{j+1}, A) - \partial(\bar M_j, A))$ so they contribute some negative dimension $\leq \epsilon_U$.
		\end{itemize}
		This ends the proof for Lemma \ref{ub_lemma}.
	\end{proof}
	\textit{(Proof of Theorem \ref{main_bound} continued)}
	Let $m = m_I$.
	Thus we have $\dim(\bar M / \bar A) = \leq -m \cdot \epsilon_U $.
	Thus as $A$ is strong we need $m_I \cdot \epsilon_U < Y$.
	Let $W = \frac{|M|Y}{\epsilon_U}$.
	\begin{align*}
		|\partial(\bar M, A)| &\leq m \cdot |M| \leq W \\
		|\bar M \cap A| &\leq m \cdot |M| \leq W
	\end{align*}
	as needed.
	This ends the proof for Theorem \ref{main_bound}.
\end{proof}

So far we have computed an upper bound for a single basic formula $\phi$.

To bound an arbitrary formula, write it as a boolean combination of basic formulas $\phi_i$ (via quantifier elimination)
It suffices to bound vc-density for collection of formulas $\{\phi_i\}$ to obtain a bound for the original formula.

In general work with a collection of basic formulas $\{\phi_i\}_{i \in I}$.
The proof generalizes in a straightforward manner.
Instead of $A^{|x|}$ we now work with $A^{|x|} \times I$ separating traces of different formulas.
Formula with the largest quantity $Y(\phi)\frac{K(\phi)}{\epsilon_\phi}$ contributes the most to the vc-density.
Thus we have
\begin{align*}
	\Phi &= \{\phi_i\}_{i \in I} \\
	\vc(\Phi) &=  \max_{i \in I} Y(\phi_i) \frac{K(\phi_i)}{\epsilon_{\phi_i}}
\end{align*}


%\begin{Definition}
	%\begin{align*}
		%d &= \dim \bar M / \bar A \\
		%s &= |\bar M - \bar A| \\
		%b &= |\partial(\bar M, \bar A)|
	%\end{align*}
%\end{Definition}


%Thus as we consider $\bar M$ as an increasing union of witnesses to chain-minimal extensions, we see the extension with the largest ratio can contribute most to the boundary.
%Thus is our upper bound for the boundary.






%Now, classify every trace by the isomorphism class of $\bar M - A \cup \partial(\bar M, A)$ and by $\partial(\bar M, A)$.
%\begin{Lemma}
	%Suppose we have traces $b_1, b_2$ with the same components as above.
	%Then $A_{b_1} = A_{b_2}$.
%\end{Lemma}
%
%Consider $\bar M - A \cup \partial(\bar M, A)$.
%Number of vertices is $\leq (2W)^2$.
%Thus number of isomorphism classes $\leq 2^{(2W)^2}$.
%
%Consider $\partial(\bar M, A)$.
%Let $N = |A|$.
%Order matters, so the total number of choices for it is
%\begin{align*}
	%N \cdot (N-1) \cdot \ldots \cdot (N - W + 1) = \frac{N!}{(N-W)!}
%\end{align*}
%
%Thus the number of possible different traces is bounded by 
%\begin{align*}
	%2^{(2W)^2} \cdot \frac{N!}{(N-W)!} = O(N^W)
%\end{align*}
%
%Since choice of $A$ was arbitrary, this gives
%\begin{align*}
	%\vc{\phi} \leq W = \frac{|M|Y}{\epsilon_U}
%\end{align*}

%%%%%%%%%%%%%%%%%%%%%%%%%%%%%%%%%%%%%%%%%%%%%%%%%%%%%%%%%%%%%%%%%%%%%%%%%%%%%%%%%%%%%%%%%%%%%%%%%%%%%%%%%%%%%%%%%
\section{Technical Lemmas}

\begin{Lemma}
	Suppose we have a set $B$ and a minimal pair $(M, A)$ with $A \subset B$ and $\dim(M/A) = -\epsilon$.
Then either $M \subseteq B$ or $\dim((M \cup B)/B) < -\epsilon$.
\end{Lemma}

\begin{proof}
	By diamond construction

	\begin{align*}
		\dim((M \cup B)/B) \leq \dim(M / (M \cap B))
	\end{align*}

	and 

	\begin{align*}
		\dim(M / (M \cap B)) &= \dim (M/A) - \dim(M / (M \cap B)) \\
		\dim (M/A) &= -\epsilon \\
		\dim(M / (M \cap B)) &> 0
	\end{align*}
\end{proof}



\begin{Lemma}	\label{chain_lemma}
	Suppose we have a set $B$ and a minimal chain $M_n$ with $M_0 \subset B$ and dimensions $-\epsilon_i$.
Let $\epsilon$ be the minimal of $\epsilon_i$.
Then either $M_n \subseteq B$ or $\dim((M_n \cup B)/B) < -\epsilon$.
\end{Lemma}


\begin{proof}
	Let $\bar M_i = M_i \cup B$

	\begin{align*}
		\dim(\bar M_n/B) = \dim(\bar M_n/\bar M_{n-1}) + \ldots + \dim(\bar M_2/\bar M_1) + \dim(\bar M_1/B)
	\end{align*}

	Either $M_n \subseteq B$ or one of the summands above is nonzero.
	Apply previous lemma.
\end{proof}

%%%%%%%%%%%%%%%%%%%%%%%%%%%%%%%%%%%%%%%%%%%%%%%%%%%%%%%%%%%%%%%%%%%%%%%%%%%%%%%%%%%%%%%%%%%%%%%%%%%%%%%%%%%%%%%%%
\section{Counterexamples}

% Add complete graph counterexample
% where we have a bunch of minimal extensions intersecting in a tiny way
% or something similar?

% \AA is indiscernible
% example of indiscernible sequence that is not strong
% example with non-strong embedding on every n-tuple of vertices?

% 2 chain minimal extension that is larger than the lower bound
%%%%%%%%%%%%%%%%%%%%%%%%%%%%%%%%%%%%%%%%%%%%%%%%%%%%%%%%%%%%%%%%%%%%%%%%%%%%%%%%%%%%%%%%%%%%%%%%%%%%%%%%%%%%%%%%%
\section{Upper bound on $\AA$}

\begin{Definition}
	\begin{align*}
		\AA = \curly{A \subset \U^{y} \mid \text{finite, disconnected, strongly embedded}}
	\end{align*}
\end{Definition}

Let $n$ be the integer such that $n \epsilon_U < Y$ and $(n+1) \epsilon_U > Y$.

Pick a trace of $\phi(x,y)$ on $A^{|x|}$ by a parameter $b$.

\begin{align*}
	B = \curly{a \in A^{|x|} \mid \phi(a, b)}
\end{align*}

Pick $B' \subset B$, ordered $B' = \{a_i\}_{i \in I}$ such that
\begin{align*}
	%a_i \cap \bigcup_{j \neq i} a_j \neq \emptyset
	a_i \cap \bigcup_{j < i} a_j \neq \emptyset
\end{align*}
This is always possible by starting with $B$ and taking away elements one by one.
Call such a set a \emph{generating set} of $B$.

Let $M_i / \{a_i, b\}$ be a witness of $\phi(a_i, b)$ for each $i \in I$.
Let $\bar M = \bigcup M_i$.
Consider $\bar M / A$.

Pick $\bar M$ such that $\dim(\bar M / A)$ is maximized.

$\bar M \cap A \leq \bar M$ as $A$ is strong. (Make sure $M$ is not too big!)
Let $\bar A = A - \curly{a_i}_{i \in I}$.
Suppose $\bar A \cap \bar M \neq \emptyset$.
Then we can abstractly reembed $\M$ over $A$ such that $\bar A \cap \bar M = \emptyset$.
This would increase the dimension, contradicting maximality.
Thus we can assume $A \cap \bar M = \{a_i\}_{i \in I}$

Let $\bar M_j = \bigcup_{i < j} M_i$.

\begin{Lemma}
	$\dim(\bar M_j / A) \leq j \cdot \epsilon_U$
\end{Lemma}
\begin{proof}
	Proceed by induction.
	Base case is clear.

	For induction case apply lemma to $\bar M_j \cup \{a_j\}$ and $M_j / \{a_j, b\}$.
	There are two cases
	\begin{enumerate}
		\item $M_j \subset \bar M' \cup \{a_j\}$.
		In this case there are edges between $\{a_j\}$ and $M_j$ that contribute to dimension less than $-\epsilon_U$.
		\item Otherwise $M_j$ adds extra dimension less than $-\epsilon_U$
	\end{enumerate}
\end{proof}

Thus we have $\dim(\bar M / A) = \dim(\bar M_n / A) \leq -\epsilon_U n$.

Thus as $A$ is strong we need $|B'| \epsilon_U < Y$.
This gives us $|B'| \leq n$.
Finally we need to relate $|B'|$ to $|B|$.

Suppose we have $C \subset A^{|x|}$, finite with $|C| = N$.
A generating set for a trace has to have size $\leq n$.
Thus there are ${N \choose n} \leq N^n$ choices for a generating set.
A set generated from set of size $n$ can have at most $(x|n|)^{|x|}$ elements.
Thus a given set of size $n$ can generate at most
\begin{align*}
	2^{(x|n|)^{|x|}}
\end{align*}
sets.
Thus the number of possible traces on $C$ is bounded above by
\begin{align*}
  2^{(x|n|)^{|x|}} \cdot N^n = O(N^n)
\end{align*}
This bounds the vc-density by $n$.

\begin{align*}
	\vc_\AA(\phi) \geq \floor{\frac{Y}{\epsilon_U}}
\end{align*}


\begin{thebibliography}{9}

\bibitem{vc_density}
	M. Aschenbrenner, A. Dolich, D. Haskell, D. Macpherson, S. Starchenko,
	\textit{Vapnik-Chervonenkis density in some theories without the independence property}, I, preprint (2011)
	
\bibitem{laskowski}
  Michael C. Laskowski, \textit{A simpler axiomatization of the Shelah-Spencer almost sure theories},
  Israel J. Math. \textbf{161} (2007), 157–186. MR MR2350161	

\end{thebibliography}
	
\end{document}

% Include edges between y as a chain minimal extension