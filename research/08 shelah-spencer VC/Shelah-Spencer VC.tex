\documentclass{amsart}

\usepackage{../AMC_style}	
\usepackage{../Research}

\usepackage{diagrams}

  \newcommand{\A}{\mathcal A}
  \newcommand{\B}{\mathcal B}
\renewcommand{\C}{\mathcal C}
  \newcommand{\D}{\mathcal D}
\renewcommand{\H}{\mathcal H}
  \newcommand{\G}{\mathcal G}
  \newcommand{\M}{\mathcal M}

  \newcommand{\K}{\boldface K_\alpha}
\renewcommand{\S}{S_\alpha}

\begin{document}

\title{Some vc-density computations in Shelah-Spencer graphs}
\author{Anton Bobkov}
\email{bobkov@math.ucla.edu}

\begin{abstract}
	We compute vc-densities of minimal extension formulas in Shelah-Spencer random graphs.
\end{abstract}

\maketitle

We fix the density of the graph $\alpha$.

\begin{Lemma}
	For any $\A \in \K$ and $\epsilon > 0$ there exists an $\B$ such that $(\A, \B)$ is minimal and $\delta(\B/\A) < \epsilon$.
\end{Lemma}

\begin{proof}
	Let $m$ be an integer such that $m\alpha < 1 < (m+1)\alpha$. Suppose $\A$ has less than $m+1$ vertices. Make a construction $\A_0 = \A$ and $\A_{i+1}$ is $\A_i$ with one extra vertex connected to every single vertex of $A_i$. Stop when the total number of vertices is $m+1$. Proceed as in \cite{Laskowski} 4.1. Resulting construction is still minimal.
\end{proof}

\begin{Lemma}
	Let $\A_1 \subset \B_1$ and $\A_2 \subset \B_2$ be $\K$ structures with $(\A_2, \B_2)$ a minimal pair with $\epsilon = \delta (\B_2/\A_2)$. Let $M$ be some ambient structure. Fix embeddings of $\A_1, \B_1, \A_2$ into $M$. Assume that it is not that case that $\A_2 \subset \B_2$ and $\A_1$ is disjoint from $\A_2$ (No!). Now consider all possible embeddings $f \colon \B_2 \to M$ over $\A_1$.  Let $\A = \A_1 \cup \A_2$ and $\B_f = \B_1 \cup f(\B_2)$ with $\delta_f = \delta(\B_f/\A)$. Then $\delta_f$ is at most $\delta(\B_1 \cup \A/\A) + \epsilon$
\end{Lemma}

Fix an embedding $f$. It induces the following substructure diagram in $M$. Denote 
\begin{align*}
	\A &= \A_1 \cup \A_2 \\
	\B_f^* &= \B_1 \cup f(\B_2) \\
	\B_1^* &= \B_1 \cup \A \\
	\B_2^* &= f(\B_2) \cup \A \\
	\B^* &= \B_1^* \cap \B_2^*
\end{align*}

\begin{diagram}
								&							&\B_f		\\
								&\ruLine    	&										&\luLine	\\
	\B_1^*      	&           	&										&					&\B_2^* \\
								&\luLine			&										&\ruLine	\\
								&							&\B^* \\
								&							&\uLine \\
								&							&\A\\
\end{diagram}

From the diagram we see that
\begin{align*}
	\delta(\B_f/\A) \leq \delta(\B_1^*/\A) + \delta(\B_2^*/\B^*)
\end{align*}
Thus all we need to do is to verify that
\begin{align*}
	\delta(\B_2^*/\B^*) \leq \epsilon
\end{align*}
Let $\B'$ denote graph induced on all the vertices in $(f(B_2) / B_1) \cup A_2$.
Then $\B'$ is a substructure of $\B_2$ over $\A_2$. By minimality we get that $\delta(\B'/\A_2) \leq \epsilon$.
We need to show $\delta(\B_2^*/\B^*) \leq \delta(\B'/\A_2)$.
Do the vertex computation
\begin{align*}
	B_2^* - B^* &= \\
	f(B_2) - (B_1 \cap f(B_2)) - A &= \\
	f(B_2) - B_1 - A &= \\
	f(B_2) - B_1 - A_2
\end{align*}
and
\begin{align*}
	B' - A_2 &=	
	f(B_2) - B_1 - A_2
\end{align*}

So the sets of the extra vertices in the extension are the same. The base $\B_2^*/\B^*$ is larger so we can introduce some extra edges but no new vertices. This means that $\delta(\B_2^*/\B^*) \leq \delta(\B'/\A_2)$ giving us the original statement.


Let $\phi(x,y)$ be a formula in a random graph with $|x|=|y|=1$ saying that there exists $\D$ over $\C = \{x,y\}$ such that $(\D, \C)$ is minimal with relative dimension $\epsilon$. Let $N$ be such that $N\epsilon < 1 < (N+1)\epsilon$. Then we argue that $vc(\phi) = N$.

Fix a $m$-strong (for any $m > |D|$) set of non-connected vertices $A$. Fix some $a*$. We invesitgate the trace of $\phi(x, a*)$ on $A$. Suppose we have $a_1, \ldots, a_k$ satisfying $\phi(a_i, a^*)$ as witnessed by $D_i / \{a_i, a*\}$. Let $\D^* = \bigcap \D_i$ and $\C^*$

Call $\M$ $n$-composite embedding if there are distinct vertices $a_1, \ldots a_n$ and $a*$ in $M$ and there are an embeddings $\D \arr \M$ with $\C$ going to $\{a_i, a^*\}$. Image of $i$-th embedding is denoted $\D_i$. Note that images of embeddings can intersect each other or $a_j$'s. Consider $\D^* = \bigcap \D_i$ and $\C^* = \{a_1, \ldots a_n, a^*\}$. Dimension of $M$ is $\delta(\D^*/\C^*)$.

Lemma: Dimension of $n$-composite embedding is at most $-n\epsilon$.

Note: if $\D_i$ are disjoint over $\C^*$ then the dimension is exactly $-n\epsilon$.

Take $n$-composite embedding with maximal dimension. Suppose it is larger than $-n\epsilon$.
Without loss of generality we may assume $\D_n$ intersects with $\D_1 \cup \ldots \cup \D_{n-1}$ over $\C^*$.
Consider two cases.
First, suppose that there is some element in $\D_n$ outside of $\D_1 \cup \ldots \cup \D_{n-1}$.
Let $\B_1 = \D_1 \cup \ldots \cup \D_{n-1}$.
Let $\A_1 = \{a_1, \ldots a_{n-1}\} \cup \{a*\}$.
Let $\B_2 = \D_n$.
Let $\A_2 = \{a_n, a*\}$.

Lemma applies to the above. Above dimension is minimized when $\D_n$ is disjoint. Contradiction.

Second, suppose that $\D_n \subseteq \B_1$. In particular $a_n \in \B_1$. Consider

Consider sets $\B_1 \ldots \B_n$ with 
\begin{enumerate}
	\item $a_i \in \B_i$
	\item $a_i \in A$
	\item $a_i \neq a_j$
	\item $a* \in \bigcap \B_i$
\end{enumerate}
 and  s.t. $\B_i / \{a*, a_i\}$ is isomorphic to $\B/\A$. We look at all the possible embeddings with those properties. We argue that a disjoint configuration minimizes total dimension of the whole construction.

We argue by induction on $n$. Fix an embedding $\B_1, \ldots \B_n$ and consider possible choices for $\B_{n+1}, a_{n+1}$. We can pick $a_n$ to be an element of $A$ not used so far and embed $\B_{n+1}$ over $\{a*, a_i\}$ disjoint from the entire construction. On the other hand suppose it is embedded such that there is an intersection. We set up to apply the previous lemma. Let
\begin{align*}
	\B_1 &= \bigcup_{1..n} \B_i \\
	\A_1 &= \{a_1, \ldots a_n\} \\
	\B_2 &= \B_{n+1} \\
	\A_2 &= \{a^*, a_{n+1}\}
\end{align*}
Applying the lemma say that the extra dimension cannot be larger then $\epsilon$.

\begin{thebibliography}{9}

\bibitem{Laskowski}
	Michael C. Laskowski, \textsl{A simpler axiomatization of the Shelah-Spencer almost sure theories,}
	Israel J. Math. \textbf{161} (2007), 157-186. MR MR2350161

\end{thebibliography}

\end{document}
