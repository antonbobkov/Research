\documentclass{amsart}

\usepackage{../AMC_style}	
\usepackage{../Research}
\usepackage{../Thm}

\usepackage{mathrsfs}
\usepackage{pgfpages} 
\usepackage{setspace}

\doublespacing
%% \usepackage[margin=.75in]{geometry}
%% \pgfpagesuselayout{2 on 1}

\renewcommand{\AA}{\mathscr A}
\newcommand{\BB}{\mathscr B}
\newcommand{\DD}{\mathscr D}
\newcommand{\II}{\mathscr I}

\newcommand{\GG}{\mathbb G}
\newcommand{\GGY}{\GG^{|y|}}
\newcommand{\AX}{A^{|x|}}
\newcommand{\BA}{\bar A}

\newcommand{\F}{\mathcal F}
\renewcommand{\LL}{\mathcal L}

\newcommand{\defn}{\underline}

\DeclareMathOperator{\diag}{diag}

\newcommand{\DB}{\mathbb D}
\newcommand{\ppp}{\partial}
\newcommand{\BM}{\bar M_{j-1}}
\newcommand{\E}{\mathscr E}
%\DeclarePairedDelimiter{\ceil}{\lceil}{\rceil}

\newcommand{\A}{A}
\newcommand{\B}{B}
\renewcommand{\C}{\mathcal C}
\newcommand{\D}{\mathcal D}
\renewcommand{\H}{\mathcal H}
\newcommand{\G}{\mathcal G}
\newcommand{\M}{\mathcal M}
\newcommand{\U}{\mathcal U}	
\newcommand{\X}{X}
\newcommand{\Y}{Y}

\newcommand{\K}{\boldface K_\alpha}
\renewcommand{\S}{S_\alpha}

\newcommand{\curly}[1]{\left\{#1\right\}}
\newcommand{\paren}[1]{\left(#1\right)}
\newcommand{\abs}[1]{\left|#1\right|}
\newcommand{\agl}[1]{\left\langle #1 \right\rangle}

\providecommand{\floor}[1]{\left \lfloor #1 \right \rfloor }

% \DeclareMathOperator{\dim}{dim}

\title{Some vc-density computations in Shelah-Spencer graphs}
\author{Anton Bobkov}
\email{bobkov@math.ucla.edu}

\begin{document}

%%%%%%%%%%%%%%%%%%%%%%%%%%%%%%%%%%%%%%%%%%%%%%%%%%%%%%%%%%%%%%%%%%%%%%%%%%%%%%%%%%%%%%%%%%%%%%%%%%%%%%%%%%%%%%%%% 
\section{Graph Combinatorics}

Throughout this paper $A, B, C, M$ will denote finite graphs, and $\DB$ will be used to denote potentially infinite graphs.
For a graph $\A$ the set of its vertices is denoted by $v(\A)$, and the set of its edges by $e(\A)$.
Number of vertices of $\A$ will be denoted as $|\A|$.
Subgraph always means induced subgraph and $A \subseteq B$ means that $A$ is a subgraph of $B$.
For two subgraphs $\A, \B$ of a larger graph, the union $\A \cup \B$ denotes the graph induced by $v(\A) \cup v(\B)$.
Similarly, $A - B$ means a subgraph of $A$ induced by the vertices of $v(A) - v(B)$.
For $A \subseteq B \subseteq D$ and $A \subseteq C \subseteq D$,
graphs $B,C$ are said to be \defn{disjoint over $A$} if $v(B) - v(A)$ is disjoint from $v(C) - v(A)$
and there are no edges from $v(B) - v(A)$ to $v(C) - v(A)$ in $D$.

For the remainder of the paper fix $\alpha \in (0,1)$, irrational.
\begin{Definition} \ 
  \begin{itemize}
  \item For a graph $\A$ let $\dim(\A) = |\A| - \alpha |e(\A)|$.
  \item For $\A,\B$ with $\A \subseteq \B$ define $\dim(\B/\A) = \dim(\B) - \dim(\A)$.
  \item We say that $\A \leq \B$ if $\A \subseteq \B$ and $\dim(\A'/A) > 0$ for all $\A \subsetneq \A' \subseteq \B$.
  \item Define $\A$ to be \defn{positive} if for all $\A' \subseteq \A$ we have $\dim(\A') \geq 0$.
  \item We work in theory $S_\alpha$ in the language of graphs axiomatized by:
    \begin{itemize}
    \item Every finite substructure is positive.
    \item Given a model $\GG$ and graphs $\A \leq \B$, every embedding $f : \A \arr \GG$ extends to an embedding $g: \B \arr \GG$.
    \end{itemize}
    (Here an embedding maps edges to edges and nonedges to nonedges.)
    This theory is complete and stable (see 5.7 and 7.1 in \cite{laskowski}).
    From now on fix an ambient model $\GG \models S_\alpha$.
    This will be the only infinite graph we work with.
  \item For $\A, \B$ positive, $(\A, \B)$ is called a \defn{minimal pair} if
    $\A \subseteq \B$, $\dim(\B/\A) < 0$ but $\dim(\A'/\A) \geq 0$ for all proper $\A \subseteq \A' \subsetneq \B$.
    We call $B$ a \defn{minimal extension} of $A$.
    The dimension of a minimal pair is defined as $|\dim(B/A)|$.
  \item A sequence $\agl{M_i}_{0 \leq i \leq n}$ is called a \defn{minimal chain} if $(M_i, M_{i+1})$ is a minimal pair for all $0 \leq i < n$.
  \item For a graph $\A$ with the tuple of vertices $x$ let $\diag_\A(x)$ be the atomic diagram of $\A$,
    i.e. the first-order formula recording whether there is an edge between every pair of vertices.
  \item Given $\A \subseteq \B$ let 
    \begin{align*}
      \phi_{\A,\B}(x) = \diag_\A(x) \wedge \exists z \; \diag_\B(x, z).
    \end{align*}
    Any graph isomorphic to $\B$ is called a \defn{witness} of $\phi_{A,B}$.
  \item A formula $\phi_{A,B}$ is called a \defn{basic formula}
    if there is a minimal chain $\agl{M_i}_{0 \leq i \leq n}$
    such that $A = M_0$ and $B = M_n$.
  \end{itemize}
\end{Definition}

\begin{Theorem} [Quantifier elimination, 5.6 in \cite{laskowski}]
  In theory $S_\alpha$ every formula is equivalent to a boolean combination of basic formulas.
\end{Theorem}

\begin{Definition}
  A graph $S \subseteq \DB$ is called \defn{$N$-strong} if for any $S \subseteq T \subseteq D$ with $|T| - |S| \leq N$ we have $S \leq T$.
\end{Definition}

%%%%%%%%%%%%%%%%%%%%%%%%%%%%%%%%%%%%%%%%%%%%%%%%%%%%%%%%%%%%%%%%%%%%%%%%%%%%%%%%%%%%%%%%%%%%%%%%%%%%%%%%%%%%%%%%% 
\section{Basic Definitions and Lemmas}

\begin{Definition} \label{def_basic}
  Suppose $\phi(x, y)$ is a basic formula.
  Define $\X$ to be the graph on vertices $x$ with edges defined by $\phi$.
  Similarly define $\Y$.
  Note that $\X$, $\Y$ are positive.
  Additionally, let $\Y'$ be a subgraph of $\Y$ induced by vertices of $\Y$ that are connected to $W - (X \cup Y)$, where $W$ is a witness of $\phi$.
\end{Definition}

\begin{Definition} \label{def_e}
  Suppose $A, B$ are subgraphs of $\D$ such that $v(A), v(B)$ are disjoint.
  Then define $\E(A, B)$ to be the number of edges between the vertices in $v(A)$ and the vertices in $v(B)$.
\end{Definition}

We will require the following lemmas from \cite{laskowski}:

\begin{Lemma} \label{diamond} [See 2.3 in \cite{laskowski}]
  Let $A, B \subseteq \DB$.
  Then
  \begin{align*}
    \dim(A \cup B / A) \leq \dim(\B / A \cap B).
  \end{align*}
  Moreover, 
  \begin{align*}
    \dim(A \cup B / A) = \dim(\B / A \cap B) - \alpha E,
  \end{align*}
  % where $E$ is the number of edges connecting the vertices of $A \cup B - A$ to the vertices of $A - A \cap B$.
  where $E$ is the number of edges connecting the vertices of $B - A$ to the vertices of $A - B$.
\end{Lemma}

\begin{Lemma} \label{las_min} [See 4.1 in \cite{laskowski}]
  Suppose $A$ is a positive graph, with at least $1/\alpha + 2$ vertices.
  Then for any $\epsilon > 0$ there exists a graph $B$ such that $(A, B)$ is a minimal pair with dimension $\leq \epsilon$.
  Moreover, every vertex in $A$ is connected to a vertex in $B - A$.
\end{Lemma}

\begin{Lemma} \label{las_str} [See 4.4 in \cite{laskowski}]
  Suppose $A$ is a positive graph, and $\G$ a model of $S_\alpha$.
  Then for any integer $S$ there exists an embedding $f \colon A \arr \G$ such that $f(A)$ is $S$-strong in $\G$.
\end{Lemma}
    
\begin{Lemma} \label{las_closure} [See 3.8 in \cite{laskowski}]
  For all $S > 0$ there exists $M = M(S, \alpha) \in \N$ with the following property.
  Suppose $A \subseteq \G$ where $\G$ is a model of $S_\alpha$.
  Then there exists $B$ with $A \subseteq B \subseteq \G$ such that $B$ is $S$-strong in $\GG$ and $|B| \leq M|A|$.
\end{Lemma}

We conclude this section by stating a couple of technical lemmas that will be useful in our proofs later.

\begin{Lemma} \label{minimal_over_set}
  Work in an ambient graph $\DB$.
  Suppose we have a set $B$ and a minimal pair $(A, M)$ with $A \subseteq B$ and $\dim(M/A) = -\epsilon$.
  Then either $M \subseteq B$ or $\dim(M \cup B/B) < -\epsilon$.
\end{Lemma}

\begin{proof}
  By Lemma \ref{diamond}
  \begin{align*}
    \dim(M \cup B/B) \leq \dim(M / M \cap B),
  \end{align*}
  and as $A \subseteq M \cap B \subseteq M$
  \begin{align*}
    \dim (M/A) = \dim(M / M \cap B) + \dim(M \cap B / A).
  \end{align*}
  In addition we are given $\dim (M/A) = -\epsilon$.
  If $M \not\subseteq B$ then $A \subseteq M \cap B \subsetneq M$ and by minimality $\dim(M \cap B / A) > 0$.
  Combining the inequalities above we obtain the desired result:
  \begin{align*}
    \dim(M \cup B/B) \leq \dim(M / M \cap B) = \dim (M/A) - \dim(M \cap B / A) < -\epsilon.
  \end{align*}
\end{proof}

\begin{Lemma}	\label{chain_lemma}
  Work in an ambient graph $\DB$.
  Suppose we have a set $B$ and a minimal chain  $\agl{M_i}_{0 \leq i \leq n}$ with dimensions
  \begin{align*}
    \dim(M_{i+1}/M_i) = -\epsilon_i
  \end{align*}
  and $M_0 \subseteq B$.
  Let $\epsilon = \min_{0 \leq i \leq n} \epsilon_i$.
  Then either $M_n \subseteq B$ or $\dim((M_n \cup B)/B) < -\epsilon$.
\end{Lemma}

\begin{proof}
  Let $\bar M_i = M_i \cup B$. Then:
  \begin{align*}
    \dim(\bar M_n/B) = \dim(\bar M_n/\bar M_{n-1}) + \ldots + \dim(\bar M_2/\bar M_1) + \dim(\bar M_1/B).
  \end{align*}
  Either $M_n \subseteq B$ or at least one of the summands above is nonzero.
  Apply previous lemma.
\end{proof}

\begin{Lemma} \label{minimal_subset}
  Suppose we have a minimal pair $(A, M)$ with dimension $\epsilon$.
  Suppose we have some $B \subseteq M$.
  Then $\dim B / (A \cap B) \geq -\epsilon$.
  Moreover if $B \cup A \neq M$ then $\dim B / (A \cap B) \geq 0$.
\end{Lemma}

\begin{proof}
  We have $\dim (B \cup A / A) \leq \dim B / (A \cap B)$ by Lemma \ref{diamond}.
  As $A \subseteq B \cup A \subseteq M$ we have $\dim (B \cup A / A) \geq -\epsilon$ by minimality.
  Moreover, minimality implies that it is positive if $B \cup A \neq M$.
\end{proof}

\begin{Lemma} \label{chain_intersect}
  Suppose we have a minimal chain  $\agl{M_i}_{0 \leq i \leq n}$ with dimensions
  \begin{align*}
    \dim(M_{i+1}/M_i) = -\epsilon_i.
  \end{align*}
  Let $\epsilon$ be the sum of all $\epsilon_i$.
  Suppose we have a graph $B$ with $B \subseteq M_n$.
  Then $\dim B / (M_0 \cap B) \geq -\epsilon$.
\end{Lemma}

\begin{proof}
  Let $B_i = B \cap M_i$.
  We have $\dim B_{i+1}/B_i \geq \dim M_{i+1}/M_i$ by the previous lemma.
  Thus
  \begin{align*}
    \dim B / (M_0 \cap B) = \dim B_n / B_0 = \sum \dim B_{i+1}/B_i \geq -\epsilon.
  \end{align*}
\end{proof}

%%%%%%%%%%%%%%%%%%%%%%%%%%%%%%%%%%%%%%%%%%%%%%%%%%%%%%%%%%%%%%%%%%%%%%%%%%%%%%%%%%%%%%%%%%%%%%%%%%%%%%%%%%%%%%%%% 

%%%%%%%%%%%%%%%%%%%%%%%%%%%%%%%%%%%%%%%%%%%%%%%%%%%%%%%%%%%%%%%%%%%%%%%%%%%%%%%%%%%%%%%%%%%%%%%%%%%%%%%%%%%%%%%%% 
\section{Upper bound}

Consider a basic formula $\phi(x,y)$ with a minimal chain  $\agl{M_i}_{0 \leq i \leq n_{\phi}}$ with dimensions  $\dim(M_{i+1}/M_i) = -\epsilon_i$.
Define
\begin{align*}
  \epsilon(\phi) &= \min \curly{\epsilon_i}_{0 \leq i \leq n_\phi}\\
  K(\phi) &= |M_{n_\phi}|.
\end{align*}
Now consider a finite collection of basic formulas
\begin{align*}
  \Phi = \Phi(\vec x, \vec y) = \curly{\phi_i(\vec x, \vec y)}_{i\in I}.
\end{align*}
Define
\begin{align*}
  \epsilon(\Phi) &= \min \curly{\epsilon(\phi_i)}_{i \in I} \cup \curly{\alpha}, \\
  K(\Phi) &= \max \curly{K(\phi_i)}_{i \in I},\\
  D_1(\Phi) &= \frac{K(\Phi)}{\epsilon(\Phi)}, \\
  D(\Phi) &= |y| D_1(\Phi).\\
\end{align*}
We claim that
\begin{Theorem} \label{upper}
  If $\phi$ is a boolean combination of formulas from $\Phi$, then $\vc(\phi) \leq D(\Phi)$.
\end{Theorem}
Let
\begin{align*}
  S = \left\lceil{\paren{\frac{|y|}{\epsilon(\phi)} + 1} K(\phi)}\right\rceil.
\end{align*}
Suppose we have a finite parameter set $A_0 \subseteq \GG^{|x|}$ with $|A_0| = N_0$.
We would like to bound $\phi(A_0, \GGY)$.
Let $A_1 \subseteq \GG$ consist of the components of the elements of $A_0$.
Then $|A_1| \leq |x| N_0$.
Using Lemma \ref{las_closure} let $A$ be a graph $A_0 \subseteq A \subseteq \GG$, $S$-strong in $\GG$.
Let $N = |A|$.
We have $N \leq |x| N_0 M$ (where $M$ is the constant from the Lemma \ref{las_closure}).
As $A_0 \subseteq \AX$ we have
\begin{align*}
  \abs{\phi(A_0, \GGY)} \leq \abs{\phi(\AX, \GGY)}.
\end{align*}
Therefore it suffices to bound $\abs{\phi(\AX, \GGY)}$.

\begin{Definition}
  For $A \subseteq \GG^{|x|}, B \subseteq \GG^{|y|}$ define
  \begin{align*}
    \Phi(A, B) = \curly{(a, i) \in A \times I \mid \GG \models \phi_i(a, b)} \subseteq A \times I
  \end{align*}  
\end{Definition}

\begin{Lemma}
  For $A \subseteq \GG^{|x|}, B \subseteq \GG^{|y|}$
  if $\phi$ is a boolean combination of formulas from $\Phi$ then
  \begin{align*}
    \abs{\phi(A, B)} \leq \abs{\Phi(A, B)}
  \end{align*}  
\end{Lemma}
\begin{proof}
  Clear, as for all $a \in A, b \in B$ the set $\Phi(a,b)$ determines the truth value of $\phi(a,b)$.
\end{proof}

Thus it suffices to bound  $\abs{\Phi(\AX, \GGY)}$.

\begin{Definition} \;
  \begin{itemize}
  \item For all $i \in I, a \in \AX, b \in \GGY$ if $\phi_i(a, b)$ holds fix $W^i_{a,b} \subseteq \GG$, a witness of this formula.
  \item For $b \in \GGY$ let 
    \begin{align*}
      W_b = \bigcup \curly{W^i_{a,b} \mid a \in \AX, i \in I, \GG \models \phi_i(a,b)}.
    \end{align*}
  \item For sets $C, B \subset \GG$ define the \defn{boundary} of $C$ over $B$
    \begin{align*}
      \partial(C, B) = \curly{b \in B \mid \E(b, C - B) \neq \emptyset}
    \end{align*}
    (see Definition \ref{def_e} for $\E$).
  \item For $b \in \GGY$ let $\partial_b \subseteq A$ be the boundary $\partial(W_b, A)$.
  \item For $b \in \GGY$ let $\bar W_b = (W_b - A) \cup \ppp_b$.
  \item For $b_1, b_2 \in \GGY$ we say that $b_1 \sim b_2$ if $\ppp_{b_1} = \ppp_{b_2}$,
    $b_1 \cap A = b_2 \cap A$,
    and there exists a graph isomorphism from $\bar W_{b_1} \cup b_1$ to $\bar W_{b_2} \cup b_2$ that fixes $\ppp_{b_1}$ and
    maps $b_1$ to $b_2$.
    One easily checks that this defines an equivalence relation.
  \item For $b \in \GGY$ define $\II_b$ to be the $\sim$-equivalence class of $b$.
  \end{itemize}
\end{Definition}

\begin{Lemma} \label {bound_trace}
  For $b_1, b_2 \in \GGY$ if $b_1 \sim b_2$ then $\Phi(\AX, b_1) = \Phi(\AX, b_2)$.
\end{Lemma}

\begin{proof}
  Fix a graph isomorphism $\bar f \colon \bar W_{b_1} \cup b_1 \arr \bar W_{b_2} \cup b_2$.
  Extend it to an isomorphism $f \colon W_{b_1} \cup A \arr W_{b_2} \cup A$ by being an identity map on the new vertices.
  Suppose $\GG \models \phi_i(a, b_1)$ for some $a \in \AX$.
  Then $f(W^i_{a, b_1})$ is a witness for  $\phi_i(a, b_2)$ (though not necessarily equal to $W^i_{a, b_2}$)
  and thus $\GG \models \phi_i(a, b_2)$.
  As $a$ was arbitrary, this proves the equality of the traces.
\end{proof}

Thus to bound the number of traces it is sufficient to bound the number of possibilities for $\II_b$.

\begin{Theorem} \label{main_bound}
  Suppose we have $b \in \GGY$.
  Let $Y = \abs{b - A}$.
  Then
  \begin{align*}
    |\partial_b| &\leq Y D_1(\phi) \\ 
    |\bar W_b| &\leq 3 Y D_1(\phi)
  \end{align*}
\end{Theorem}

From this theorem we get the desired result:
\begin{Corollary} (Theorem \ref{upper})
  If $\phi$ is a boolean combination of formulas from $\Phi$, then $\vc(\phi) \leq D(\Phi)$.
\end{Corollary}

\begin{proof}
  We count possible equivalence classes of $\sim$.
  This essentially boils down to bounding possibilities for $\partial_b$, $b \cap A$, and number of isomorphism classes of $W_b$.
  Fix $b \in \GGY$ and let $Y = \abs{b - A}$.
  Let $D = Y D_1(\Phi)$.
  By the first part of Theorem \ref{main_bound} there are $N \choose D$ choices for boundary $\partial_b$.
  By the second part of Theorem \ref{main_bound} there are at most $3D$ vertices in $\bar W_b$.
  Thus to determine the graph $\bar W_b$ we need to choose how many vertices it has and then decide where edges go.
  This amounts to at most $3D 2^{(3D)^2}$ choices.
  Additionally there are $N \choose |y| - Y$ choices for $b \cap A$.
  In total this gives us at most
  \begin{align*}
    &{N \choose D} \cdot {N \choose |y| - Y} \cdot 3D 2^{(3D)^2} = O(N^{D + |y| - Y}) = \\
    &= O(N^{Y D_1(\Phi) + |y| - Y}) = O(N^{|y| D_1(\Phi)}) = O(N^{D(\Phi)})
  \end{align*}
  choices (second to last inequality uses $D_1(\Phi) \geq 1$).
  By Lemma \ref{bound_trace} we have $\abs{\Phi(\AX, \GGY)} = O(N^{D(\Phi)})$.
  Recall that 
  \begin{align*}
    \abs{\phi(A_0, \GGY)} \leq \abs{\Phi(\AX, \GGY)}.    
  \end{align*}
  Therefore we have
  \begin{align*}
    \abs{\phi(A_0, \GGY)} &= O(N^{D(\Phi)}) = O(\paren{|x| N_0 M}^{D(\Phi)}) = \\
    &= O(\paren{|x| M}^{D(\Phi)} N_0^{D(\Phi)}) = O(N_0^{D(\Phi)}).
  \end{align*}
  As $A_0$ was arbitrary, this shows that $\vc(\phi) \leq D(\Phi)$ as needed.
  (Note that throughout this proof we are using the fact that quantities $D_1(\Phi), D(\Phi), M$ are completely determined by $\Phi$,
  thus independent from $A_0$.)
\end{proof}

\begin{proof} \textit{(of Theorem \ref{main_bound})}

  The graph $W_b$ is a union of witnesses of the from $W_{a,b}$ for some $a \in \AX, b \in \GGY$.
  Enumerate all of them as $\curly{W_j}_{1 \leq j \leq J}$.
  Define $M_j = \bigcup_1^j W_{j'}$ for $1 \leq j \leq J$ and let $M_0 = b$.
  Let $\BA = A \cup b$.
  \begin{Definition}
    For $0 \leq j \leq J$ define:
    \begin{itemize}
    \item Let $v_j = 1$ if new vertices are added to $M_j$ outside of $\BA$, that is if
      $M_j - \BA \neq M_{j-1} - \B$,
      and let it be $0$ otherwise.
    \item Let %$E_j = \curly{a \in A - W_j \mid  \E(a, M_j - A) \neq \emptyset}$.
      $E_j = \partial(A - W_j, M_j - A)$.
    \item Let
      \begin{align*}
        m_j = \sum_{j' = 0}^j (v_j + |E_j|).
      \end{align*}
    \end{itemize}
    (Here assume $M_{-1} = \emptyset$.)
  \end{Definition}

  \begin{Lemma} \label{ubd_lemma}
    For $0 \leq j \leq J$ we have
    \begin{align*}
      |\partial(M_j, A)| \leq |E_0| + m_j K(\Phi) 
    \end{align*}
  \end{Lemma}

  \begin{proof} %\textit{(of Lemma \ref{ub_lemma})}
    Proceed by induction. The base case $j = 0$ is clear.
    For an induction step suppose that
    \begin{align*}
      |\partial(M_{j-1}, A)| \leq m_{j-1}  K(\Phi)
    \end{align*}
    holds.
    Let
    \begin{align*}
      \delta_1 &= \partial(M_j, A) - \partial(M_{j-1}, A) = \\
               &=\curly{a \in A \mid  \E(a, M_j - A) \neq \emptyset \text{ and } \E(a, M_{j-1} - A) = \emptyset}.
    \end{align*}
    If $M_j - A = M_{j-1} - A$ then $\delta_1 = \emptyset$ and we are done as $m_j$ is increasing.
    Suppose not.
    We have $|\delta_1| = |\delta_1 \cap W_j| + |\delta_1 - W_j|$, and
    \begin{align*}
      \delta_1 - W_j = \curly{a \in A - W_j \mid \E(a, M_j - A) \neq \emptyset \text{ and } \E(a, M_{j-1} - A) = \emptyset}.
    \end{align*}
    But then it's clear that $\delta_1 - W_j \subseteq E_j$ as
    \begin{align*}
      &W_j - M_{j-1} - A \subseteq M_j - A, \\
      &(W_j - M_{j-1} - A) \cap (M_{j-1} - A) = \emptyset.
    \end{align*}
    As $b \in M_{j-1}$ and $M_j - A \neq M_{j-1} - A$, then $M_j - \BA \neq M_{j-1} - \BA$, and thus $v_j = 1$. 
    Therefore we have
    \begin{align*}
      |\delta_1| &= |\delta_1 \cap W_j| + |\delta_1 - W_j| \leq |W_j| + |E_j| \leq \\
      &\leq K(\Phi) + |E_j|
      \leq (v_j + |E_j|) K(\Phi)  \leq (m_j - m_{j-1}) K(\Phi),
    \end{align*}
    as needed.
  \end{proof}

  \begin{Lemma} \label{ub_lemma}
    For $0 \leq j \leq J$ we have
    \begin{align*}
      |M_j - \BA| \leq \sum_{j'=0}^j v_{j'} K(\Phi)
    \end{align*}
  \end{Lemma}

  \begin{proof} %\textit{(of Lemma \ref{ub_lemma})}
    Proceed by induction. The base case $j = 0$ is clear.
    For an induction step suppose that
    \begin{align*}
      |M_{j-1} - \BA| \leq \sum_{j'=0}^{j-1} v_{j'} K(\Phi)
    \end{align*}
    holds.
    If $M_j - \BA = M_{j-1} - \BA$ then the inequality is immediate as $v_j \geq 0$.
    Therefore assume this is not the case, so $v_j = 1$ and $|M_j - A| - |M_{j-1} - A| \leq |W_j| \leq v_j K(\Phi)$, and so we get the required inequality.
    \begin{align*}
    \end{align*}
  \end{proof}
  
  \begin{Lemma} \label{ubdim_lemma}
    For $0 \leq j \leq J$ we have
    \begin{align*}
      \dim(M_j \cup \BA / \BA) \leq -m_j  \epsilon(\Phi),
    \end{align*}
  \end{Lemma}
  \begin{proof}
    Proceed by induction. Base case $j = 0$ is clear.    
    For an induction step suppose that
    \begin{align*}
      \dim(M_{j-1} \cup \BA / \BA) \leq  - m_{j-1}  \epsilon(\Phi)
    \end{align*}
    holds.
    We have
    \begin{align*}
      \dim(M_j \cup \BA / \BA) &= \dim(M_j \cup \BA / M_{j-1} \cup \BA) + \dim(M_{j-1} \cup \BA / \BA) \leq \\
      &\leq \dim(M_j \cup \BA / M_{j-1} \cup \BA) - m_{j-1}  \epsilon(\Phi).
    \end{align*}
    Let $\BM = M_{j-1} \cup \BA$.
    By Lemma \ref{diamond}
    \begin{align*}
      \dim(M_j \cup \BA / M_{j-1} \cup \BA) = \dim(W_j \cup \BM / \BM) = \dim(W_j / W_j \cap \BM) - e \alpha
    \end{align*}
    where $e$ is the number of edges connecting the vertices of $\BM - W_j$ to the vertices of $W_j - \BM$.
    Recall that       $E_j = \partial(A - W_j, M_j - A)$.
    We have $A - W_j \subseteq \BM - W_j$ (as $A \subseteq \BM$) and $W_j - M_{j-1} - A = W_j - \BM$ (as for $j > 1$, we have $b \subseteq M_{j-1}$).
    Thus $|E_j| \leq e$, and we get 
    \begin{align*}
      \dim(M_j \cup \BA / M_{j-1} \cup \BA) \leq \dim(W_j / W_j \cap \BM) - |E_j| \alpha.
    \end{align*}
    If $W_j \subseteq \BM$ then $\dim(W_j / W_j \cap \BM) = 0$.
    If not, then by Lemma \ref{chain_lemma} we have $\dim(W_j / W_j \cap \BM) \leq - \epsilon(\Phi)$.
    Either way, we have $\dim(W_j / W_j \cap \BM) \leq - v_j \epsilon(\Phi)$.
    Using this and the fact that $\epsilon(\Phi) \leq \alpha$, we obtain
    \begin{align*}
      \dim(M_j \cup \BA / M_{j-1} \cup \BA) \leq - v_j \epsilon(\Phi) - |E_j| \epsilon(\Phi) = -(m_j - m_{j-1})\epsilon(\Phi).
    \end{align*}
    Finally,
    \begin{align*}
      \dim(M_j \cup \BA / \BA) &\leq \dim(M_j \cup \BA / M_{j-1} \cup \BA) - m_{j-1}  \epsilon(\Phi) \leq \\
      &\leq  -(m_j - m_{j-1})\epsilon(\Phi) - m_{j-1}  \epsilon(\Phi) =  - m_j  \epsilon(\Phi),
    \end{align*}
    as needed.
  \end{proof}
  \textit{(Proof of Theorem \ref{main_bound} continued)}
  For any $0 \leq j \leq J$ we have
  \begin{align*}
    \dim(M_j \cup A / A) &= \dim(\BA / A) + \dim(M_j \cup \BA / \BA) \\
    &\leq Y - |E_0|\alpha + \dim(M_j \cup \BA / \BA).
  \end{align*}
  Lemma \ref{ubdim_lemma} gives us
  \begin{align*}
    \dim(M_j \cup \BA / \BA) \leq -m_j  \epsilon(\Phi).
  \end{align*}
  Thus
  \begin{align*}
    \dim(M_j \cup A / A) \leq Y - |E_0| \alpha - m_j  \epsilon(\Phi).
  \end{align*}
  Suppose $j$ is an index such that
  \begin{align*}
    &Y - |E_0| \alpha - m_j  \epsilon(\Phi) \geq 0, \\
    &Y - |E_0| \alpha - m_{j+1}  \epsilon(\Phi) < 0
  \end{align*}
  if one exists.
  Then 
  \begin{align*}
    m_j \leq \frac{Y - |E_0| \alpha}{\epsilon(\Phi)}.
  \end{align*}
  By Lemma \ref{ub_lemma} we have
  \begin{align*}
    \abs{M_{j+1} - A} &\leq \paren{\sum_{j'=1}^{j+1} v_{j'}} K(\Phi) \leq (m_j + 1) K(\Phi) \\
                     &\leq \paren{\frac{Y - |E_0| \alpha}{\epsilon(\Phi)} + 1} K(\Phi) \leq S.
  \end{align*}
  This is a contradiction, as $A$ is $S$-strong and $\dim(M_{j+1} \cup A / A)$ is negative.
  Thus $Y - |E_0| \alpha - m_j  \epsilon(\Phi) \geq 0$ for all $j \leq J$.
  In particular $Y - |E_0| \alpha - m_J  \epsilon(\Phi) \geq 0$, so $m_J \leq \frac{Y - |E_0| \alpha}{\epsilon(\Phi)}$.
  Noting that $M_J = W_b$, Lemma \ref{ubd_lemma} gives us 
  \begin{align*}
      |\ppp_b| = |\partial(W_b, A)| \leq |E_0| + m_J  K(\Phi) \leq |E_0| + K(\Phi) \frac{Y - |E_0| \alpha}{\epsilon(\Phi)}.
  \end{align*}
  As $K(\Phi) \geq 1$ and $\epsilon(\Phi) \geq \alpha$, we get
  \begin{align*}
      |\ppp_b| \leq K(\Phi) \frac{Y}{\epsilon(\Phi)} = Y D_1(\Phi).
  \end{align*}
  But this is precisely the first inequality we need to prove.
  For the second inequality, Lemma \ref{ub_lemma} gives us
  \begin{align*}
    \abs{W_b - \BA} &\leq Y + \paren{\sum_{j'=0}^J v_{j'}} K(\Phi) \leq Y + m_J K(\Phi) \leq \\
    &\leq Y + K(\Phi) \frac{Y}{\epsilon(\Phi)} \leq 2 Y D_1(\Phi).
  \end{align*}
  Thus we have
  \begin{align*}
      |\bar W_b| \leq \abs{W_b - A} + \abs{\ppp_b} \leq 3 Y D_1(\Phi),
  \end{align*}
  as needed.
  This ends the proof for Theorem \ref{main_bound}.
\end{proof}


%%%%%%%%%%%%%%%%%%%%%%%%%%%%%%%%%%%%%%%%%%%%%%%%%%%%%%%%%%%%%%%%%%%%%%%%%%%%%%%%%%%%%%%%%%%%%%%%%%%%%%%%%%%%%%%%% 

\begin{thebibliography}{9}

\bibitem{density}
  M. Aschenbrenner, A. Dolich, D. Haskell, D. Macpherson, S. Starchenko,
  \textit{Vapnik-Chervonenkis density in some theories without the independence property}, I,
  Trans. Amer. Math. Soc. 368 (2016), 5889-5949
  
\bibitem{laskowski}
  Michael C. Laskowski, \textit{A simpler axiomatization of the Shelah-Spencer almost sure theories},
  Israel J. Math. \textbf{161} (2007), 157–186. MR MR2350161	

\bibitem{ash7}
  P. Assouad, \textit{Densit´e et dimension}, Ann. Inst. Fourier (Grenoble) 33 (1983), no. 3, 233-282.
\bibitem{ash8}
  P. Assouad, \textit{Observations sur les classes de Vapnik-Cervonenkis et la dimension combinatoire de Blei},
  in: Seminaire d’Analyse Harmonique, 1983-1984, pp. 92-112, Publications Math´ematiques
  d’Orsay, vol. 85-2, Universit´e de Paris-Sud, D´epartement de Math´ematiques, Orsay, 1985.
\bibitem{sauer}
  N. Sauer, \textit{On the density of families of sets}, J. Combinatorial Theory Ser. A 13 (1972), 145-147.
\bibitem{shelah}
  S. Shelah, \textit{A combinatorial problem; stability and order for models and theories in infinitary languages},
  Pacific J. Math. 41 (1972), 247-261.

\end{thebibliography}

\end{document}

