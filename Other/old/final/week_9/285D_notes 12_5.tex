\documentclass[12pt]{article}
\RequirePackage[left=1.5in,right=1.5in,top=1.5in,bottom=1.5in]{geometry}   %margins
\usepackage{amsmath}   %general math symbols

\usepackage{fourier}   %Font
\usepackage{amssymb}  %\textbf 
%\usepackage{mathrsfs}  %\mathscr
\usepackage{dsfont}     %\mathds
\usepackage{bussproofs}
%\usepackage{latexsym}
\usepackage{multicol}

% This is the "centered" symbol
\def\fCenter{{\mbox{\Large{$\rightarrow$}}}}

% Optional to turn on the short abbreviations
\EnableBpAbbreviations

\usepackage{tikz}      %commutative diagrams
\usepackage{enumerate}    %lists

\usepackage{amsthm}    
\usepackage{bm}


\swapnumbers
\theoremstyle{theorem}            %bold title, italicized font
\newtheorem{theorem}{Theorem}[section]

\theoremstyle{definition}           %bold title, regular font
\newtheorem{example}[theorem]{Example}
\newtheorem{definition}[theorem]{Definition}
\newtheorem{proposition}[theorem]{Proposition}
\newtheorem{lemma}[theorem]{Lemma}
\newtheorem{corollary}[theorem]{Corollary}
\newtheorem{exercise}[theorem]{Exercise}
\newtheorem*{problem}{Problem}
\newtheorem{warning}[theorem]{Warning}
\newtheorem*{solution}{Solution}
\newtheorem*{remark}{Remark}

\theoremstyle{empty}
\newtheorem{namedtheorem}{}

\newcommand{\customtheorem}[3]{\theoremstyle{theorem} \newtheorem{theorem#1}[theorem]{#1} \begin{theorem#1}[#2]#3 \end{theorem#1}}

\newcommand{\customdefinition}[2]{\theoremstyle{definition} \newtheorem{definition#1}[theorem]{#1} \begin{definition#1}#2 \end{definition#1}}


\newcommand{\bigslant}[2]{{\raisebox{.2em}{$#1$}\left/\raisebox{-.2em}{$#2$}\right.}}
\def\dotminus{\mathbin{\ooalign{\hss\raise1ex\hbox{.}\hss\cr\mathsurround=0pt$-$}}}

\renewcommand{\restriction}{\mathord{\upharpoonright}}

\begin{document}
\tikzset{node distance=2cm, auto}

\begin{center} \begin{Large} Math 285D Notes: 12/5 \end{Large}\\
\text{} \\
\begin{large} Tyler Arant  \end{large}
\end{center}

\textbf{Correction for proof of Weierstrass Preparation.} We had $f, g\in \mathds{C}\{X, T\}$, $f$ regular of order $d$, and
$$F=u^{-1}\sum_{i<d}f_iT^i, \quad u=f_d + f_{d+1}T+ \cdots \in \mathds{C}\{X, T\}^\times.$$
Choose $(r', r_{m+1})\in (\mathds{R}^{>0})^{m+1}$ such that
$$\|g\|_r, \|u^{-1}\|_r, \|f_0\|_{r'}, \dots, \|f_{d+1}\|_{r'}<\infty.$$
Then, we can achieve
$$\|F\|_r\leq \|u^{-1}\|\cdot \sum_{i<d}\|f_i\|_{r'}r^i_{m+1}<r^d_{m+1},$$
since the $f_i$ vanish at $0$ we can make the norms as small as we want by choosing $r'$ small enough.\\

Let $R\subset S$ be an extension of commutative rings.

\begin{definition} $S$ is \textit{flat over} $R$ if each solution in $S$ to an equation
\begin{equation} r_1x_1+\cdots + r_n x_n=0 \qquad (r_i\in R) \tag{$*$}\end{equation}
is an $S$-linear combination of solutions in $R$. \end{definition}

\begin{lemma} If $S$ free as an $R$-module, then $S$ is flat over $R$. \end{lemma}

\begin{proof} Let $s=(s_1, \dots, s_n)\in S^n$ be a solution to $(*)$.  Take $R$-linearly independent $e_1, \dots, e_k\in S$ such that
$$s_i=\sum_j w_{ij} e_j \qquad (w_{ij}\in R).$$
Put $w_j=(w_{1j}, \dots, w_{nj})$.  Then $w_j$ is a solution to $(*)$ and $s=\sum_j e_jw_j.$
\end{proof}

We give some examples
\begin{enumerate}[$\bullet$]
\item If $R$ is a field, then each $S$ is flat over $R$.  
\item $S=R[X_1, \dots, X_n]$ flat over $R$.
\end{enumerate}

\begin{lemma} Suppose $S$ is flat over $R$.  Then each solution in $S$ to a system
$$r_{i1}x_1 + \cdots + r_{in}x_n=0 \qquad (i=1, \dots, m ; r_{ij}\in R)$$
is an $S$-linear combination of solutions in $R$. \end{lemma}

\begin{proof} By induction on $m$.  \end{proof}

\begin{definition} We say that $S$ is \textit{faithfully flat} over $R$ if
\begin{enumerate}[$\bullet$]
\item $S$ is flat over $R$.
\item Each equation 
$$r_1x_1+\cdots +r_nx_n=1 \qquad (r_i\in R)$$
that has a solution in $S$ has a solution in $R$. \end{enumerate}
\end{definition}

\begin{lemma} Suppose $S$ is flat over $R$.  The following are equivalent.
\begin{enumerate}[(1)]
\item $S$ is faithfully flat over $R$.
\item For each maximal ideal $\mathfrak{m}$ of $R$, we have $\mathfrak{m}S\neq S$.
\item Each system 
\begin{equation} \sum_{j=1}^n r_{ij}x_j = t_i \qquad (i=1, \dots, m ; r_{ij}, t_i\in R) \tag{$*$} \end{equation}
that has a solution in $S$ has a solution in $R$. 
\end{enumerate} \end{lemma}

$(1)\implies(2)$: Suppose, by means of contradiction, that $\mathfrak{m}S=S$.  Then, there are $r_i\in \mathfrak{m}$ and $s_i\in S$ such that
$$r_1s_1+\cdots + r_n s_n = 1,$$
i.e., $s=(s_1, \dots, s_n)$ is a solution to the equation
$$r_1x_1 + \cdots + r_n x_n = 1.$$
Since $S$ is faithfully flat over $R$, there is a solution $w=(w_1, \dots, w_n)$ so that
$$1= r_1w_1+\cdots + r_n w_n \in \mathfrak{m},$$
which is a contradiction. 

$(2)\implies (1)$:  Suppose $S$ is not faithfully flat over $R$; then, there are $r_i\in R$ such that
$$r_1x_1+\cdots + r_n x_n=1$$
has a solution $s$ in $S$ but not a solution in $R$.  Then the ideal $\mathfrak{a}=(r_1, \dots, r_n)$ in $R$ is proper.  Let $\mathfrak{m}$ be an ideal in $R$ that contains $\mathfrak{a}$. Then, $\mathfrak{m}S=S$ since
$$1=r_1s_1+\cdots r_n s_n \in \mathfrak{m}S.$$

$(3)\implies (1)$ trivially.

$(1)\implies (3)$: Suppose $(*)$ has a solution $s=(s_1, \dots, s_n)\in S^n$.  Then, $(1, s)= (1, s_1, \dots, s_n)$ is a solution to the homogenous system
\begin{align*} -t_1x_0+\sum_{j=1}^n r_{1j}x_j &=  0  \\
				&\vdots     \tag{$**$} \\
		-t_mx_0 + \sum_{j=1}^n r_{mj}x_j  & =  0 \end{align*}
Since $S$ is flat over $R$, $(1, s)$ is an $S$-linear combination of solutions $(u_1, v_1), \dots, (u_k, v_k)$ in $R^{1+n}$. So,
$$(1, s) = w_1(u_1, v_1) + \cdots + w_k(u_k, v_k) \qquad (w_i\in S),$$
hence
$$1= w_1 u_1 + \cdots + w_k u_k.$$
By $(1)$, there exists $\omega_1, \dots, \omega_k\in R$ such that
$$1= \omega_1u_1 + \cdots + \omega_k u_k.$$
Then, $\omega_1v_1+\cdots + \omega_k v_k\in R^n$ solves $(*)$. 



\begin{theorem} $\mathds{C}[[X]]$ is faithfully flat over $\mathds{C}\{X\}$. \end{theorem}

\begin{proof}[Proof sketch] We proceed by induction on the number of variable.  Consider
\begin{equation}f_1y_1+\cdots + f_n y_n = 0 \qquad (f_i\in \mathds{C}\{X, T\}. \tag{$*$} \end{equation}
We may assume that all $f_i$, if nonzero, are regular in $T$.  Then, apply Weierstass Preparation in $\mathds{C}\{X, T\}$ to the $f_i\neq 0$.  Then, we can assume that the $f_i\neq0$ are Weierstrass polynomials: $f_i\in \mathds{C}\{X\}[T]$ monic of some degree $d_i$.  Set
$$z_2 = \left [ \begin{array}{c} f_2 \\ -f_1 \\ 0 \\ \vdots \\ 0 \end{array} \right ], z_3= \left [ \begin{array}{c} f_3 \\ 0 \\ -f_1 \\ 0 \\ \vdots \\ 0 \end{array} \right ], \cdots, z_n= \left [ \begin{array}{c} f_n \\ 0 \\ \vdots \\ 0 \\ -f_1\end{array} \right ],$$
which together is a solution of $(*)$.  We have $y_i=q_if_1+r_i$, where $q_i\in \mathds{C}[[X, T]]$, and $r_i\in \mathds{C}[[X]][T]$ is of degree $<d_1$.  Then consider
$$y+q_2z_2= \left [ \begin{array}{c} * \\ r_2 \\ y_3 \\ \vdots \\ y_n \end{array} \right ], \dots , y+q_2z_2+\cdots f_ny_n= \left [ \begin{array}{c} * \\ r_2\\ r_3 \\ \vdots \\ r_n \end{array} \right ],$$
and conclude that we can assume $y_2, \dots, y_n\in \mathds{C}[[X]][T]$.   We have
$$g:= f_1y_1 = -(f_2y_2+\cdots + f_ny_n)\in \mathds{C}\{X\}[T].$$
We can find $h, r$ with $g=f_1h+r$ with $h, r\in \mathds{C}[[X]][T]$, $\deg r<d_1$.  So, $g=f_1y_1+0$ in $\mathds{C}[[X, T]]$, hence $r=0$ and $y_1=h\in \mathds{C}[[X]][T]$.  This reduces the proof to showing: $R\subset S$ flat $\implies R[T]\subset S[T]$ flat.
\end{proof}

\section{Restricted Analytic Functions}

\begin{lemma}[Taylor expansion] Suppose $f\in \mathds{C}\{X\}_s$ and $b\in D_s(0)$, $j\in \mathds{N}^m$.  Then,
\begin{enumerate}[(1)]
\item $\partial^j f :=\left (\frac{\partial}{\partial X_1} \right )^{j_1} \cdots \left (\frac{\partial}{\partial X_m} \right )^{j_m}f\in \mathds{C}\{X\}_r$ for all $r<s$.
\item $(\partial^jf)(b):= \sum_{i\geq j} f_i \frac{i!}{(i-j)!}b^{i-j}$ converges absolutely.  
\item $\sum_j \frac{1}{j!} (\partial^jf)(b)X^j\in \mathds{C}\{X\}_{s(b)}$, where
$$s(b)=(s_1-|b_1|, \dots, s_m-|b_m|),$$
and
$$f(x+b) = \sum_j\frac{1}{j!}(\partial^jf)(b)x^j \qquad (x\in \overline{D_{s(b)}}(0)).$$
\end{enumerate}
\end{lemma}

\begin{proof} \begin{enumerate}[(1)]
\item By induction on $|j|$.  The case $\frac{\partial}{\partial x+k}$ follows from Abel's Lemma: a finite bound on $|f_i|s^i$ gives a finite bound on $i_k|f_i|r^{i-e_k}$ (for given $r<s$, with $e_k$ is the $k^th$ standard basis vector).  
\item Follows easily from $(1)$.
\item Left as an exercise using $(2)$ and multivariate binomial theorem. 
\end{enumerate}\end{proof}



\end{document}


