\WikiLevelTwo{ Week 2 }

(Notes by John Lensmire)

\WikiLevelThree{ Monday 10-13-2014 }

Let $a,b\in \mathbf{No}$. Recall that $a + b = \{a_L + b, a + b_L | a_R + b, a + b_R \}$.

==== Theorem 2.5 ====

$(\mathbf{No},+,<)$ is an ordered abelian group with $0 = () = \{\emptyset, \emptyset \}$ and $-a$ is obtained by reversing all signs in $a$.

'''Proof:'''

We have already proven that $\leq$ is translation invariant.

Commutativity is clear from the symmetric nature of the definition.

We show by induction on $l(a)$ that $a+0 = a$. The base case is clear, and
\begin{align*}
a + 0 &= \{a_L + 0, a + 0_L | a_R + 0, a + 0_R \} \\
&= \{a_L + 0 | a_R + 0\} \ (\text{as } 0_L = 0_R = \emptyset) \\
&= \{a_L | a_R\} \ (\text{by induction}) \\
&= a
\end{align*}

We next show the associative law by induction on $l(a)\oplus l(b)\oplus l(c)$.
We have
\begin{align*}
(a+b)+c &= \{(a+b)_L + c, (a+b) + c_L | (a+b)_R + c, (a+b) + c_R \} \\
&= \{(a_L+b) + c, (a+b_L) + c, (a+b) + c_L | (a_R+b) + c, (a+b_R) + c, (a+b) + c_R\}
\end{align*}
where the second equality holds because of uniformity.
An identical calculation shows:
\[
a+(b+c) = \{a_L+ (b + c), a+ (b_L + c), a+ (b + c_L) | a_R+ (b + c), a+ (b_R + c), a+ (b + c_R)\}
\]
and hence $(a+b)+c = a+(b+c)$ holds by induction.

To show $a + (-a) = 0$ first note:
\begin{enumerate}
  \item  $b <_s a \Rightarrow -b <_s -a$
  \item  $b < a \Rightarrow -b > -a$
\end{enumerate}
Hence, $-a = \{-a_R | -a_L\}$. Thus,
\[
a + (-a) = \{a_L + (-a), a + (-a_R) | a_R + (-a), a + (-a_L) \}
\]
By the induction hypothesis and the fact that $+$ is increasing we have the following:
\begin{enumerate}
  \item  $a_L + (-a) < a_L + (-a_L) = 0$
  \item  $a + (-a_R) < a_R + (-a_R) = 0$
  \item  $a_R + (-a) > a_R + (-a_R) = 0$
  \item  $a + (-a_L) > a_L + (-a_L) = 0$
\end{enumerate}

==== Definition 2.6 ====

For $a,b\in \mathbf{No}$ set
\[
a\cdot b = \{a_L\cdot b + a\cdot b_L - a_L\cdot b_L, a_R\cdot b + a\cdot b_R - a_R\cdot b_R | a_L\cdot b + a\cdot b_R - a_L\cdot b_R, a_R\cdot b + a\cdot b_L - a_R\cdot b_L \}
\]
As motivation for this definition, note that in any ordered field: if $a'<a,b'<b$ then $(a-a')(b-b')>0$ so in particular $a'b + ab' - a'b' < ab$.

==== Lemma 2.7 ====

Suppose for all $a,b\in \mathbf{No}$ with $l(a)\oplus l(b) < \gamma$ we have defined $a\cdot b$ so that (2.6) holds, and for all $a,b,c,d\in \mathbf{No}$ with the natural sum of the lengths of each factor is $<\gamma$ $(*)$ holds, where
$(*): a>b, c>d \Rightarrow ac-bc > ad-bd.$
Then: (2.6) holds for all $a,b\in \mathbf{No}$ with $l(a)\oplus l(b) \leq \gamma$ and $(*)$ holds for all $a,b,c,d\in \mathbf{No}$ with the natural sum of the lengths of each factor is $\leq \gamma$.

'''Proof:'''

Let $P(a,b,c,d)\Leftrightarrow ac - bc > ad - bd.$ Because the surreal numbers form an ordered abelian group, $P$ is  "transitive in the last two variables" (i.e. $P(a,b,c,d) \ \&\ P(a,b,d,e) \Rightarrow P(a,b,c,e)$) and similarly in the first two variables.

Fix $a,b\in \mathbf{No}$. For $a' <_s a, b' <_s b$ we define $f(a',b') = a'b + ab' - a'b'$.

Claim:
\begin{enumerate}
	\item  $a' < a \Rightarrow b' \mapsto f(a',b')$ is an increasing function
	\item  $a' > a \Rightarrow b' \mapsto f(a',b')$ is a decreasing function
	\item  $b' < b \Rightarrow a' \mapsto f(a',b')$ is an increasing function
	\item  $b' > b \Rightarrow a' \mapsto f(a',b')$ is a decreasing function
\end{enumerate}

We prove 1 (the rest are left as an exercise). Let $b'_1,b'_2 <_s b$ and $b'_1 < b'_2$. Then
\begin{align*}
f(a',b'_2) > f(a',b'_1) &\Leftrightarrow (a'b + ab'_2 - a'b'_2) > (a'b + ab'_1 - a'b'_1)  \\
&\Leftrightarrow (ab'_2 - a'b'_2) > (ab'_1 - a'b'_1) \\
&\Leftrightarrow P(a,a',b'_2,b'_1)
\end{align*}
and $P(a,a',b'_2,b'_1)$ holds by induction, proving 1.

1-4 in the claim give us respectively:
\begin{enumerate}
  \item  $f(a_L, b_L) < f(a_L, b_R)$
  \item  $f(a_R, b_R) < f(a_R, b_L)$
  \item  $f(a_L, b_L) < f(a_R, b_R)$
  \item  $f(a_R, b_R) < f(a_L, b_L)$
\end{enumerate}
These facts exactly give us that $a\cdot b$ is well-defined.

We are left to show that $(*)$ continues to hold. We'll continue this on Wednesday.

\WikiLevelThree{ Wednesday 10-15-2014 }

Recall the definition of multiplication from last time:
\[
a\cdot b = \{a_L\cdot b + a\cdot b_L - a_L\cdot b_L, a_R\cdot b + a\cdot b_R - a_R\cdot b_R | a_L\cdot b + a\cdot b_R - a_L\cdot b_R, a_R\cdot b + a\cdot b_L - a_R\cdot b_L \}
\]
and the statement $(*): a>b, c>d \Rightarrow ac-bc > ad-bd$. We'll also continue write $P(a,b,c,d)\Leftrightarrow ac - bc > ad - bd$.

Note we can rephrase the defining inequalities for $a\cdot b$ as
\[
(\Delta): P(a,a_L,b,b_L), P(a_R,a,b_R,b), P(a,a_L,b_R,b), P(a_R,a,b,b_L)
\]

To finish the proof of Lemma 2.7, suppose $a>b>c>d$ (of suitable lengths). We want to show $P(a,b,c,d)$.

Case 1:  Suppose in each pair $\{a,b\}, \{c,d\}$ one of the elements is an initial segment of the other. Then note we are done by $(\Delta)$.

Case 2:  Suppose $a \not<_s b, b\not<_s a$ but $c <_s d$ or $d <_s c$. Then we have that $a > a\wedge b > b$ and by Case 1, $P(a,a\wedge b, c, d)$ and $P(a\wedge b, b, c, d)$. This implies (by transitivity of the first two variables) $P(a,b,c,d)$.

Case 3:  Suppose $c \not<_s d, d\not<_s c$. Then by Case 1 and 2, $P(a,b,c,c\wedge d)$ and $P(a,b,c\wedge d, d)$, so (transitivity of the last two variables) $P(a,b,c,d)$ as needed. This completes the proof of Lemma 2.7.

==== Lemma 2.8 ====

The uniformity property holds for multiplication.

'''Proof:'''

Fix $a,b\in \mathbf{No}$. For any $a',b'\in \mathbf{No}$ we define (as last time) $f(a',b') = a'b + ab' - a'b'$. Using Lemma 2.7, we can extend the Claim from last time to hold in general:

Claim:
\begin{enumerate}
\item $a' < a \Rightarrow f(a',-)$ is an increasing function
\item $a' > a \Rightarrow f(a',-)$ is a decreasing function
\item $b' < b \Rightarrow f(-,b')$ is an increasing function
\item $b' > b \Rightarrow f(-,b')$ is a decreasing function
\end{enumerate}

Let $a = \{L|R\}, b = \{L'|R'\}$ be any representations of $a,b$. We want to verify the hypothesis of the Cofinality Theorem 1.10.

Let $a_l, b_l$ range over $L,L'$. As an example, note
\[
f(a_l, b_l) < ab \Leftrightarrow 0 < ab - (a_lb + ab_l - a_lb_l) = (ab - a_lb) - (ab_l - a_lb_l)
\Leftrightarrow P(a,a_l,b,b_l)
\]
which holds as $a_l<a, b_l<b$. Checking the other inequalities in a similar manner gives us the first assumption of Theorem 1.10.

To get the second assumption (the cofinality hypothesis), let e.g. $f(a_L,b_L)$ in the left side of the representation of $ab$. By inverse cofinality, we get $a_l\in L, b_l\in L'$ with $a_l \geq a_L, b_l\geq b_L$.
Then (using (3) and (1) respectively from the above claim): $f(a_l,b_l) \geq f(a_L,b_l)\geq f(a_L,b_L)$ as needed. Again, the other cases are similar.

Therefore, both assumptions of Theorem 1.10 hold, giving us the uniformity property for multiplication as needed.

==== Proposition 2.9 ====

$(\mathbf{No},+,\cdot,\leq)$ is an ordered commutative ring, with multiplicative identity $1 = (+) = \{0 | \emptyset \}$.

'''Proof:'''

Commutativity is clear by the symmetry in the definition.

We show the distributative law by induction on $l(a) \oplus l(b) \oplus l(c)$ that $(a+b)\cdot c = a\cdot c + b\cdot c$.

In general, the typical element in the cut for $(a+b)\cdot c$ is $(a+b)_* c + (a+b)_* c_* - (a+b)_*c_*$ (where $(a+b)_*$ is either $(a+b)_L$ or $(a+b)_R$ and similar for $c_*$). This element is less than $(a+b)c$ if and only if there are $0$ or $4$ $*'s$ equal to $R$.

By uniformity, we can replace $(a+b)_*$ with $a_*+b,a+b_*$, so the typical terms become (using induction):
\[
(a_*+b)c + (a+b)c_* - (a_*+b)c_* = a_*c + bc + ac_* - a_*c_*
\]
or similarly
\[
(a+b_*)c + (a+b)c_* - (a+b_*)c_* = ac + b_*c + bc_* - b_*c_*
\]
On the other hand, the typical elements of $ac + bc$ are
\[
(ac)_* + bc = a_*c + ac_* - a_*c_* + bc \text{ or } ac + (bc)_* = ac + b_*c + bc_* - b_*c_*
\]
(note the same parity rule for $*$'s applies here.) A quick check shows that this matches a typical element of $(a+b)\cdot c$ we have $(a+b)c = ac+bc$ as needed.

Associativity is proven using a very similar argument (and is left as an exercise).

We are left to check the identity element. Note $a\cdot 0 = a\cdot (0+0) = a\cdot 0 + a\cdot 0$ which implies $a\cdot 0 = 0$.

By definition:
\[
a\cdot 1 = \{a_L\cdot 1 + a\cdot 1_L - a_L\cdot 1_L, a_R\cdot 1 + a\cdot 1_R - a_R\cdot 1_R | a_L\cdot 1 + a \cdot 1_R - a_L\cdot 1_R, a_R \cdot 1 + a\cdot 1_L - a_R\cdot 1_L\}
\]
Using the fact that $1_R = \emptyset$ and multiplication by $0 = 1_L$ is zero, we have
\[
a\cdot 1 = \{a_L\cdot 1 | a_R \cdot 1\} = \{a_L | a_R\} = a
\]
by induction. This completes the proof.

Our next goal is to define $1/a$ for $a>0$, i.e. find a solution to $a\cdot x = 1$.
Note, the naive idea is to set $x = \{1/a_R | 1/a_L, (a_L\neq 0)\}$ but this does not work in general.

\WikiLevelThree{ Friday 10-17-2014 }

==== Definition of Inverses ====

Let $a\in \mathbf{No}$ with $a>0$. Our aim is to define $1/a$. Let $a = \{L|R\}$ the canonical representation of $a$. Observe that $a'\geq 0$ for all $a'\in L$ (as $a' <_s a$).

For every finite sequence $(a_1,\ldots,a_n)\in (L\cup R)\setminus \{0\}$ we define $\langle a_1,\ldots, a_n\rangle \in \mathbf{No}$ by induction on $n$. Set $\langle \ \rangle = 0$ and inductively set $\langle a_1,\ldots, a_n, a_{n+1}\rangle = \langle a_1,\ldots, a_n \rangle \circ a_{n+1}$.
Here, for arbitrary $b\in \mathbf{No}$ and $a'\in (L\cup R)\setminus \{0\}$ let $b\circ a'$ be the unique solution to $(a-a')b + a'x = 1$, i.e.
\[
b\circ a' = [1-(a-a')b]/a'.
\]
This works as inductively we'll have already defined $1/a'$.

For example, $\langle a_1 \rangle = \langle \ \rangle \circ a_1 = 0 \circ a_1 = 1/a_1$.

Now set (as candidates for defining $1/a$)
\begin{align*}
L^{-1} &= \{ \langle a_1,\ldots, a_n \rangle | \text{ the number of } a_i \text{ in }L \text{ is even} \} \\
R^{-1} &= \{ \langle a_1,\ldots, a_n \rangle | \text{ the number of } a_i \text{ in }L \text{ is odd} \}
\end{align*}
Note that this definition is an expansion of the naive idea presented at the end of last lecture.

We first show
\[
(*) x \in L^{-1} \Rightarrow ax < 1 \text{ and } x\in R^{-1} \Rightarrow ax > 1.
\]
by induction on $n$. In particular, this yields that $L^{-1} < R^{-1}$.

The base case is clear, as $\langle \ \rangle = 0\in L^{-1}$ and $a\cdot 0 = 0 < 1$.

For the induction, suppose $b\in L^{-1}\cup R^{-1}$ satisfying $(*)$ and $0\neq a' <_s a$. We show that $x = b\circ a'$ also satisfies $(*)$.

Claim:
\begin{enumerate}
\item $x > b \Leftrightarrow 1 > ab$
\item $ax = 1 + (a-a')(x-b)$.
\end{enumerate}
By definition, $x$ is the solution to $(a-a')b + a'x = 1$ and $ab = ab - a'b + a'b = (a-a')b + a'b$. These two equations yield $ab = 1 + a'(b-x)$. Both parts of the claim follow.

Now suppose, $b\in L^{-1}, a'\in L$. Then $x\in R^{-1}$ so want to check $ax > 1$.
$b\in L^{-1}$ and $(*)$ implies that $ab < 1$ and hence Claim 1 tells us that $x > b$. Now by Claim 2, $ax = 1 + (a-a')(x-b)$ hence $ax > 1$ (because $a>a', x>b$) as needed.

The other cases are similar, so $(*)$ holds in general.

Thus, we can set $c = \{L^{-1} | R^{-1} \}$. We claim that $1/a = c$, that is, $ac = 1 = \{0 | \emptyset\}$.

The typical element used to define $ac$ is $a'c + ac' - a'c'$ with $a'\in L\cup R, c'\in L^{-1}\cup R^{-1}$.

We first show $\{\text{ lower elements for }ac\} < 1 < \{(\text{ upper elements for }ac \}$.

Suppose that $a' = 0$, then we get an element $a'c + ac' - a'c' = ac'$ which is an upper element for $ac$ if and only if $c'\in R^{-1}$ by definition. However, by $(*)$ if $c'\in R^{-1}$ then $ac' = a'c + ac' - a'c' > 1$ (as needed since an upper element), else $c'\in L^{-1}$ and then $ac' = a'c + ac' - a'c' < 1$ (as needed since a lower element).

If $a'\neq 0$, then $x = c'\circ a'$ is defined, lies in $L^{-1}\cup R^{-1}$, and obeys $(\Delta): (a-a')c' + a'x = 1$. Hence,
\begin{align*}
a'c + ac' - a'c' \text{ is a lower element for } ac &\Leftrightarrow a' \ \&\ c' \text{ are on the same side of } a \ \&\ \text{(respectively) } c \\
&\Leftrightarrow x\in R^{-1} \Leftrightarrow x > c \\
&\Leftrightarrow \text{a typical element } a'c + ac' - a'c' = (a-a')c' + a'c < 1
\end{align*}
where the last equivalence holds by $(\Delta)$ and $a'>0$.

Since $1$ satisfies the cut for $ac$ (and $0$ does not) it is the minimial realization, so $ac = 1$ as needed.

We have thus shown:

==== Theorem 2.10 (Conway) ====

$(\mathbf{No}, +, \cdot, \leq)$ is an ordered field.

We'll now begin to focus on how to view real numbers and ordinals as surreal numbers.

We have $0 = ()$ the additive identity of $\mathbf{No}$ and $1 = (+)$ the multiplicative identity of $\mathbf{No}$.

We also have an embedding of ordered rings $\mathbb{Z} \hookrightarrow \mathbf{No}$ where $k\mapsto k\cdot 1$ (where $k\cdot 1$ is $k$ additions of $+1$ or $-1$).

==== Lemma 3.1 ====

For $n\in \mathbb{N}$, $n\cdot 1 = (+\cdots +)$ (i.e. $n$ $+$'s).
