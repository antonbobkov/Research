\documentclass{article}

\usepackage{amsmath}
\usepackage{amssymb}
\usepackage{amsthm}
\usepackage{enumerate}
\usepackage{colonequals}
\usepackage{fullpage}

\usepackage{dsfont}

\newtheorem{theorem}{Theorem}
\newtheorem{defn}{Definition}
\newtheorem{cor}{Corollary}
\newtheorem{claim}{Claim}
\newtheorem{lem}{Lemma}

\newtheorem{lemma}{Lemma}

\newcommand{\R}{\mathbb{R}}
%\newcommand{\concat}{\mathbin{\raisebox{1ex}{\scalebox{.7}{$\frown$}}}} %sequence concatenation
\newcommand{\concat}{\frown} %sequence concatenation
\newcommand{\dom}[1]{\operatorname{dom}\paren{#1}}
\newcommand{\ZFC}{\mathsf{ZFC}}
\newcommand{\NBG}{\mathsf{NBG}}
\newcommand{\coloneq}{\colonequals}
\newcommand{\N}{\mathbb{N}}

\newcommand{\No}{\mathbf{No}}
\newcommand{\On}{\mathbf{On}}
\newcommand{\paren}[1]{\left( #1 \right)}
\newcommand{\brac}[1]{\left[ #1 \right]}
\newcommand{\curly}[1]{\left\{ #1 \right\}}
\newcommand{\abs}[1]{\left| #1 \right|}
\newcommand{\rar}{\rightarrow}
\newcommand{\arr}{\rightarrow}

\DeclareMathOperator{\supp}{supp}
\DeclareMathOperator{\lt}{lt}

\newcommand{\w}{\omega}
\newcommand{\midr}[1]{\restriction_{#1}}

% This is the "centered" symbol
\def\fCenter{{\mbox{\Large{$\rightarrow$}}}}

\newcommand{\bigslant}[2]{{\raisebox{.2em}{$#1$}\left/\raisebox{-.2em}{$#2$}\right.}}
\def\dotminus{\mathbin{\ooalign{\hss\raise1ex\hbox{.}\hss\cr\mathsurround=0pt$-$}}}

\swapnumbers
\theoremstyle{theorem}            %bold title, italicized font

\theoremstyle{definition}           %bold title, regular font
\newtheorem{example}[theorem]{Example}
\newtheorem{definition}[theorem]{Definition}
\newtheorem{proposition}[theorem]{Proposition}
\newtheorem{corollary}[theorem]{Corollary}
\newtheorem{exercise}[theorem]{Exercise}
\newtheorem*{problem}{Problem}
\newtheorem{warning}[theorem]{Warning}
\newtheorem*{solution}{Solution}
\newtheorem*{remark}{Remark}

\theoremstyle{empty}
\newtheorem{namedtheorem}{}

\newcommand{\customtheorem}[3]{\theoremstyle{theorem} \newtheorem{theorem#1}[theorem]{#1} \begin{theorem#1}[#2]#3 \end{theorem#1}}

\newcommand{\customdefinition}[2]{\theoremstyle{definition} \newtheorem{definition#1}[theorem]{#1} \begin{definition#1}#2 \end{definition#1}}


\renewcommand{\restriction}{\mathord{\upharpoonright}}


\title{Notes on Surreal Numbers \\ Math 285: Fall 2014}
\author{Class Taught by Prof. Aschenbrenner}
\date{\today}
\begin{document}
\maketitle{}

\NotesBy{Notes by John Suice}

\Day{Day 1: Friday October 3, 2014}
Surreal numbers were discovered by John Conway. 
The class of all surreal numbers is denoted $\No$ and 
this class comes equipped with a natural linear ordering and 
arithmetic operations making $\No$ a real closed ordered field. 

For example, $1/ \omega, \omega - \pi, \sqrt{\omega} \in \No$, 
where $\omega$ denotes the first infinite ordinal. 

\begin{theorem}[Kruskal, 1980s]
	There is an exponential function $\exp \colon \No \rar \No$
	exteding the usual exponential $x \mapsto e^x$ on $\R$. 
	\label{}
\end{theorem}

\begin{theorem}[van den Dries-Ehrlich, c. 2000]
	$(\R, 0, 1, +, \cdot, \leq, e^x) \preccurlyeq 
	(\No, 0, 1, +, \cdot, \leq, \exp)$. 	
	\label{}
\end{theorem}

\section{Basic Definitions and Existence Theorem}
Throughout this class, we will work in von Neumann-Bernays-G\"odel 
set theory with global choice ($\NBG$). This is conservative over 
$\ZFC$ (see Ehrlich, \emph{Absolutely Saturated Models}). 

An example of a surreal number is the following: 
\begin{align*}
	f \colon \curly{0, 1, 2} &\longrightarrow \curly{+, -} \\
	0 &\longmapsto + \\
	1 &\longmapsto - \\
	2 &\longmapsto +
\end{align*}
This may be depicted in tree form as follows:
%------------------------Beautiful Tree Diagram-------------------------------------
%------------------------DO NOT ALTER IN ANY WAY------------------------------------
%----------------------Violators WILL be prosecuted---------------------------------
%----The above is not meant to exclude the possibility of extrajudical punishment--- 
\tikzset{every tree node/.style={minimum width=2em,draw,circle},
         blank/.style={draw=none},
         edge from parent/.style=
         {draw, edge from parent path={(\tikzparentnode) -- (\tikzchildnode)}},
         level distance=1.5cm}
\begin{align*}
\begin{tikzpicture}[sibling distance=40pt]
\Tree
[.{} 
	\edge[dashed]; 
	[.\node[dashed]{-};
	\edge[dashed];
	[.\node[dashed]{-}; 
	\edge[dashed]; \node[dashed]{-}; 
	\edge[dashed]; \node[dashed]{+};]
	%[.\node[dashed]{+}; \node[dashed]{-}; \node[dashed]{+};]	
	\edge[dashed]; [.\node[dashed]{+};
	\edge[dashed]; \node[dashed]{-};
	\edge[dashed]; \node[dashed]{+};]
	]
	\edge[thick];    
    [.\node[thick]{+};  
    \edge[thick]; [.\node[thick]{-};
             \edge[dashed]; \node[dashed]{-};
             \edge[thick]; \node[thick]{+};
         ]
	\edge[dashed]; [.\node[dashed]{+};
	\edge[dashed]; \node[dashed]{-};
	\edge[dashed]; \node[dashed]{+};
	]
    ]
]
\end{tikzpicture}
\end{align*}
%---------------------------------------------------------------------
%---------------------------------------------------------------------
We will denote such a surreal number by $f=(+-+)$
Another example is: 
\begin{align*}
	f \colon \omega + \omega &\longrightarrow \curly{+, -} \\
	n &\longmapsto + \\
	\omega + n &\longmapsto -
\end{align*}
We write $\No$ for the class of surreal numbers. We often view 
$f \in \No$ as a function $f \colon \On \rar \curly{+, -, 0}$ by 
setting $f(\alpha) = 0$ for $\alpha \notin \dom{f}$. 

\begin{defn}
	Let $a, b \in \No$. 
	\begin{enumerate}
		\item We say that $a$ is an \emph{initial segment} of 
			$b$ if $l(a) \leq l(b)$ and $b \restriction 
			\dom{a} = a$. We denote this by $a \leq_s b$
			(read: ``$a$ is simpler than $b$''). 
		\item We say that $a$ is a \emph{proper initial segment}
			of $b$ if $a \leq_s b$ and $a \neq b$. We denote 
			this by $a <_s b$. 
		\item If $a \leq_s b$, then the \emph{tail} of $a$ in 
			$b$ is the surreal number $c$ of length 
			$l(b) - l(a)$ satisfying $c(\alpha) = 
			a(l(b) + \alpha)$ for all $\alpha$. 
		\item We define $a \concat b$ to be the surreal number 
			satisfying: 
			\begin{align*}
				(a \concat b)(\alpha) = 
				\begin{cases}
					a(\alpha) & \alpha < l(a) \\
					b(\alpha - l(a)) & \alpha \geq l(a)
				\end{cases}
			\end{align*}
			(so in particular if $a \leq_s b$ and $c$ is the tail 
			of $a$ in $b$, then $b = a \concat c$). 
		\item Suppose $a \neq b$. Then the \emph{common initial 
			segment} of $a$ and $b$ is the element 
			$c \in \No$ with minimal length such that 
			$a(l(c)) \neq b(l(c))$ and $c \restriction l(c) 
			= a \restriction 
			l(c) = b \restriction l(c)$. We write 
			$c = a \wedge b$, and also set $a \wedge a = a$. 
	\end{enumerate}
\end{defn}
Note that 
\begin{align*}
	a \leq_s b \iff a \wedge b = a
\end{align*}

\Day{Day 2: Monday October 6, 2014}
\begin{defn}
	We order $\left\{ +, -, 0 \right\}$ by setting
	$- < 0 < +$ and for $a, b \in \No$ we define
	\begin{align*}
		a < b &\iff a < b \text{ lexicographically} \\
		&\iff a \neq b \land a(\alpha_0) < b(\alpha_0) 
		\text{ where } \alpha_0 = l(a \wedge b)
	\end{align*}
	As usual we also set $a \leq b \iff a < b \lor a = b$. 
\end{defn}
Clearly $\leq$ is a linear ordering on $\No$. 

Examples: 
\begin{align*}
	(+-+) < (+++ \cdots --- \cdots) \\
	(-+) < () < (+-) < (+) < (++)
\end{align*}
Remark: if $a \leq_s b$ then $a \wedge b = a$ and if 
$b \leq_s a$ then $a \wedge b = b$. Suppose that neither 
$a \leq_s b$ or $b \leq_s a$. Put: 
\begin{align*}
	\alpha_0 = \min{ \curly{\alpha \colon a(\alpha) \neq b(\alpha)}}
\end{align*}
Then either $a(\alpha_0) = +$ and $b(\alpha_0) = -$, in which 
case $b < (a \wedge b) < a$, or $a(\alpha_0) = -$ and $b(\alpha_0) = +$, 
in which case $a < (a \wedge b) < b$. In either case: 
\begin{align*}
	a \wedge b \in \brac{ \min{ \curly{a, b}, \max{ \curly{a, b}}}}
\end{align*}

\begin{defn}
	Let $L, R$ be subsets (or subclasses) of $\No$. We say 
	$L < R$ if $l < r$ for all $l \in L$ and $r \in R$. We define 
	$A < c$ for $A \cup \curly{c} \subseteq \No$ in the obvious manner. 
\end{defn}
Note that $\emptyset < A$ and $A < \emptyset$ for all $A \subseteq \No$ by 
vacuous satisfaction. 

\begin{theorem}[Existence Theorem]
	Let $L, R$ be sub\emph{sets} of $\No$ with $L < R$. 
	Then there exists a unique $c \in \No$ of minimal length 
	such that $L < c < R$. 
	\label{}
\end{theorem}
\begin{proof}
%--------------Redundant Section (Covered at beginning of next day)------------------
%	First assume that there exists $c \in \No$ with $L < c < R$. 
%	By minimizing over the lengths of all such $c$ (using the fact that 
%	the ordinals are well-ordered), we may assume without loss of 
%	generality that $c$ has minimal length. But then it is immediate 
%	that $c$ is unique; for if $\tilde{c} \neq c$ satisfied 
%	$L < \tilde{c} < R$ and $l(c) = l(\tilde{c})$, then by 
%	the note at the beginning of this section we would have: 
%	\begin{align*}
%		L < \min{ \curly{c, \tilde{c}}}
%		< (c \land \tilde{c}) < \max{ \curly{c, 
%			\tilde{c}}} < R	
%	\end{align*}
%	contradicting minimality of $l(c)$. 
%
%	Now for existence: let 
%------------------------------------------------------------------------------------
	We first prove existence. Let 
	\begin{align*}
		\gamma = \sup_{a \in L \cup R}{(l(a) + 1)}
	\end{align*}
	be the least strict upper bound of lengths of elements of 
	$L \cup R$ (it is here that we use that $L$ and $R$ are sets 
	rather than proper classes). For each ordinal $\alpha$, 
	denote by $L\restriction \alpha$ the set  $\curly{l \restriction \alpha 
	\colon l \in L}$, and similarly for $R$. Note that 
	$L \restriction \gamma = L$ and $R \restriction \gamma = R$. 	
	We construct $c$ of length $\gamma$ by defining the 
	values $c(\alpha)$ by induction on 
	$\alpha \leq \gamma$ as follows: 
	\begin{align*}
		c(\alpha) = 
		\begin{cases}
			- & \text{ if } 
			(c \restriction \alpha \concat (-) ) \geq 
			L \restriction (\alpha + 1) \\
			+ & \text{ otherwise}
		\end{cases}
	\end{align*}
	\begin{claim}
		$c \geq L$
	\end{claim}
	\begin{proof}[Proof of Claim]
		Otherwise there is $l \in L$ such that 
		$c < l$. This means $c(\alpha_0) < l(\alpha_0)$
		where $\alpha_0 = l(c \wedge l)$. Since 
		$c(\alpha_0)$ must be in $\left\{ -, + \right\}$ (i.e. 
		is nonzero) this implies $c(\alpha_0) = -$ even though 
		$(c \restriction \alpha_0 \concat (-)) \not \geq 
		l \restriction (\alpha_0 + 1)$, a contradiction. 
	\end{proof}
	\begin{claim}
		$c \leq R$	
	\end{claim}
	\begin{proof}[Proof of Claim]
		Otherwise there exists $r \in R$ such that 
		$r < c$. This means $r(\alpha_0) < c(\alpha_0)$ 
		where $\alpha_0 = l(r \land c)$. 
		%We may assume 
		%that $\alpha_0$ is least possible, i.e. that 
		%$c \restriction \alpha_0 \leq r' \restriction \alpha_0$
		%for all $r' \in R$. 
		Since $c(\alpha_0) > r(\alpha_0)$, 
		we must be in the ``$c(\alpha_0) = +$'' case, and so 
		there is some $l \in L$ such that 
		$l \restriction (\alpha_0 + 1) > (c \restriction \alpha_0) 
		\concat (-) = (r \restriction \alpha_0) \concat (-)$. 
		In particular $l(\alpha_0) \in \curly{0, +}$. 
		So if $r(\alpha_0) = -$ then $r < l$, and if 
		$r(\alpha_0) = 0$ then $r \leq l$, in either 
		case contradicting $L < R$. 
	\end{proof}
	At this point we have shown $L \leq c \leq R$. 
	But by construction $c$ has length $\gamma$, and so 
	in particular cannot be an element of $L \cup R$. 
	Thus 
	\begin{align*}
		L < c < R
	\end{align*}
	as desired. 
\end{proof}

\Day{Day 3: Wednesday October 8, 2014}
Last time we showed that there is $c \in \No$ with $L < c < R$. 
The well-ordering principle on $\On$ gives us such a $c$ of minimal 
length. Let now $d \in \No$ satisfy $L < d < R$. Then 
$L < (c \wedge d) < R$. By minimality of $l(c)$ and since 
$(c \wedge d) \leq_s c$ we have $l(c \wedge d) = l(c)$. 
Therefore $(c \wedge d) = c$, or in other words $c \leq_s d$. 

Notation: $\left\{ L \vert R \right\}$ denotes the $c \in \No$
of minimal length with $L < c < R$. Some remarks: 
\begin{enumerate}[(1)]
	\item $\left\{ L \vert \emptyset \right\}$ consists only of 
		$+$'s. 
	\item $\left\{ \emptyset \vert R \right\}$ consists only of 
		$-$'s. 
\end{enumerate}
\begin{lem}
	If $L < R$ are subsets of $\No$, then 
	\begin{align*}
		l( \curly{L \vert R}) \leq 
		\min{ \curly{\alpha \colon l(b) < \alpha \text{ for all 
		$b \in L \cup R$} }}
	\end{align*}
	Conversely, every $a \in \No$ is of the form 
	$a = \curly{L \vert R}$ where $L < R$ are subsets of 
	$\No$ such that $l(b) < l(a)$ for all $b \in L \cup R$. 
	\label{lemma_on_length_of_cuts}
\end{lem}
\begin{proof}
	Suppose that $\alpha$ satisfies $l(\left\{ L \vert R \right\}) > 
	\alpha > l(b)$ for all $b \in L \cup R$. Then 
	$c \coloneq \curly{L \vert R} \restriction \alpha$ also 
	satsfies $L < c < R$, contradicting the minimality of 
	$l(\left\{ L \vert R \right\})$. For the second part, let 
	$a \in \No$ and set $\alpha \coloneq l(a)$. Put: 
	\begin{align*}
		L &\coloneq \curly{b \in \No \colon b < a 
			\text{ and } l(b) < \alpha} \\
			R &\coloneq \curly{b \in \No \colon 
				b > a \text{ and } l(b) < \alpha}
	\end{align*}
	Then $L < a < R$ and $L \cup R$ contains all surreals of 
	length $< \alpha = l(a)$. So $a = \curly{L \vert R}$. 
\end{proof}
\begin{defn}
	Let $L, L', R, R'$ be subsets of $\No$. We say that 
	$(L', R')$ is \emph{cofinal} in $(L, R)$ if: 
	\begin{itemize}
		\item $(\forall a \in L)(\exists a' \in L')$ 
		such that $a \leq a'$, and 
		\item $(\forall b \in R)(\exists b' \in R')$
		such that $b \geq b'$.
	\end{itemize}
\end{defn}
Some trivial observations: 
\begin{itemize}
	\item If $L' \supseteq L$ and $R' \supseteq R$, then 
		$(L', R')$ is cofinal in $(L, R)$ and in 
		particular $(L, R)$ is cofinal in $(L, R)$. 
	\item Cofinality is transitive. 
	\item If $(L', R')$ is cofinal in $(L, R)$ and 
		$L' < R'$, then $L < R$. 
	\item If $(L', R')$ is cofinal in $(L, R)$ and 
		$L' < a < R'$, then $L < a < R$. 
\end{itemize}
\begin{theorem}[The ``Cofinality Theorem'']
	Let $L, L', R, R'$ be subsets of $\No$ with 
	$L < R$. Suppose $L' < \curly{L \vert R} < R'$ and 
	$(L', R')$ is cofinal in $(L, R)$. Then $\left\{ L \vert 
	R\right\} = \curly{L' \vert R'}$. 
	\label{cofinality_theorem}
\end{theorem}
\begin{proof}
	Suppose that $L' < a < R'$. Then $L < a < R$ since 
	$(L', R')$ is cofinal in $(L, R)$. Hence 
	$l(a) \geq l( \curly{L \vert R})$. Thus 
	$\left\{ L \vert R \right\} = \curly{L \vert R'}$. 
\end{proof}
\begin{cor}[Canonical Representation]
	Let $a \in \No$ and set 
	\begin{align*}
		L' &= \curly{b \colon b < a \text{ and } b <_s a} \\
		R' &= \curly{b \colon b > a \text{ and } b <_s a}
	\end{align*}
	Then $a = \curly{L' \vert R'}$. 
\end{cor}
\begin{proof}
	By Lemma \ref{lemma_on_length_of_cuts} take 
	$L < R$ such that $a = \curly{L \vert R}$ and 
	$l(b) < l(a)$ for all $b \in L \cup R$. Then 
	$L' \subseteq L$ and $R' \subseteq R$, so $(L, R)$ is 
	cofinal in $(L', R')$. By Theorem \ref{cofinality_theorem}
	it remains to show that $(L', R')$ is cofinal in 
	$(L, R)$. 

	For this let $b \in L$ be arbitrary. Then 
	$l(a \wedge b) \leq l(b) < l(a)$ and 
	thus $b \leq (a \wedge b) < a$. Therefore 
	$a \wedge b \in L'$. Similarly for $R$. 
\end{proof}
Exercise: let $a = \curly{L' \vert R'}$ be the canonical 
representation of $a \in \No$. Then 
\begin{align*}
	L' &= \curly{a \restriction \beta \colon a(\beta) = +} \\
	R' &= \curly{a \restriction \beta \colon a(\beta) = -}
\end{align*}

Exercise: Let $a = \curly{L' \vert R'}$ be the canonical representation 
of $a \in \No$. Then 
\begin{align*}
	L' &= \curly{a \restriction \beta \colon a(\beta) = +} \\
	R' &= \curly{a \restriction \beta \colon a(\beta) = 1}
\end{align*}
For example, if $a = (++-+--+)$, then $L' = \{(), (+), (++-), (++-+--)\}$
and $R' = \{(++), (++-+), (++-+-)\}$. Note that the elements of 
$L'$ decrease in the ordering as their length increases, whereas those 
of $R'$ do the opposite. Also note that the canonical representation 
is not minimal, as $a$ may also be realized as the cut 
$a = \curly{(++-+--) \vert (++-+-)}$. 
\begin{cor}[``Inverse Cofinality Theorem'']
	Let $a = \curly{L \vert R}$ be the canonical representation 
	of $a$ and let $a = \curly{L' \vert R'}$ be an arbitrary 
	representation. Then $(L', R')$ is cofinal in $(L, R)$. 
	\label{inverse_cofinality_theorem}
\end{cor}
\begin{proof}
	Let $b \in L$ and suppose that for a contradiction that 
	$L' < b$. Then $L' < b < a < R'$, and $l(b) < l(a)$, 
	contradicting $a = \curly{L' \vert R'}$. 
\end{proof}
\section{Arithmetic Operators}
We will define addition and multiplication on $\No$ and we will 
show that they, together with the ordering, make $\No$ into 
an ordered field. 
\Day{Day 4: Friday, October 10, 2014}
We begin by recalling some facts about ordinal arithmetic: 
\begin{theorem}[Cantor's Normal Form Theorem]
	Every ordinal $\alpha$ can be uniquely represented as
	\begin{align*}
		\alpha = \omega^{\alpha_1} a_1 + \omega^{\alpha_2}
		a_2 + \cdots + \omega^{\alpha_n} a_n
	\end{align*}
	where $\alpha_1 > \cdots > \alpha_n$ are ordinals and 
	$a_1, \cdots, a_n \in \N \setminus \curly{0}$. 
	\label{}
\end{theorem}
\begin{defn}
	The (Hessenberg) \emph{natural sum} $\alpha \oplus \beta$ of 
	two ordinals
	\begin{align*}
		\alpha &= \omega^{\gamma_1} a_1 + \cdots \omega^{\gamma_n}
		a_n \\
		\beta &= \omega^{\gamma_1} b_1 + \cdots \omega^{\gamma_n} 
		b_n
	\end{align*}
	where $\gamma_1 > \cdots > \gamma_n$ are ordinals and 
	$a_i, b_j \in \N$, is defined by: 
	\begin{align*}
		a \oplus \beta = \omega^{\gamma_1}(a_1 + b_1) + \cdots 
		+ \omega^{\gamma_n}(a_n + b_n)
	\end{align*}
\end{defn}
The operation $\oplus$ is associative, commutative, and strictly increasing 
in each argument, i.e. $\alpha < \beta \implies a \oplus \gamma < \beta \oplus 
\gamma$ for all $\alpha, \beta, \gamma \in \On$. Hence 
$\oplus$ is \emph{cancellative}: $\alpha \oplus \gamma = \beta \oplus 
\gamma \implies \alpha = \beta$. There is also a notion of 
\emph{natural product} of ordinals: 
\begin{defn}
	For $\alpha, \beta$ as above, set 
	\begin{align*}
		\alpha \otimes \beta \coloneq 
		\bigoplus_{i, j}{\omega^{\gamma_i \oplus \gamma_j}a_i 
	b_j}
	\end{align*}
\end{defn}
The natural product is also associative, commutative, and strictly 
increasing in each argument. The distributive law also holds for 
$\oplus$, $\otimes$: 
\begin{align*}
	\alpha \otimes (\beta \oplus \gamma) = (\alpha \otimes \beta) 
	\oplus (\alpha \otimes \gamma)
\end{align*}
In general $\alpha \oplus \beta \geq \alpha + \beta$. Moreover 
strict inequality may occur: $1 \oplus \omega = \omega + 1 > \omega = 
1 + \omega$. 

%In the following, if $a = \curly{L \vert R}$ is the canonical 
%representation of $a \in \No$ then we let $a_L$ range over 
%$L$ and $a_R$ range over $R$ (so in particular $a_L < a < a_R$). 
In the following, if $a = \curly{L \vert R}$ is the canonical 
representation of $a \in \No$, we set $L(a) = L$ and 
$R(a) = R$. We will use the shorthand $X + a = 
\left\{ x + a \colon x \in X \right\}$ (and its obvious 
variations) for $X$ a subset of 
$\No$ and $a \in \No$. 

\begin{defn}
	Let $a, b \in \No$. Set
	\begin{align}
		a + b \coloneq 
		\left\{ (L(a) + b) \cup (L(b) + a) \vert 
		(R(a) + b) \cup (R(b) + a) \right\}
		\label{defn_of_surreal_sum}
	\end{align}
\end{defn}
Some remarks: 
\begin{enumerate}[(1)]
	\item This is an inductive definition on $l(a) \oplus l(b)$. 
		There is no special treatment needed for the base 
		case: $\left\{ \emptyset \vert \emptyset \right\} = 
		+ \curly{\emptyset \vert \emptyset} = 
		\left\{ \emptyset \vert \emptyset \right\}$. 
	\item To justify the definition we need to check that 
		the sets $L, R$ used in defining $a + b = 
		\left\{ L \vert R \right\}$ satisfy $L < R$. 
\end{enumerate}
\begin{lem}
	Suppose that for all $a, b \in \No$ with $l(a) \oplus 
	l(b) < \gamma$ we have defined $a + b$ so that 
	Equation \ref{defn_of_surreal_sum} holds and 
	\begin{align*}
		b > c \implies a + b > a + c 
		\text{ and } b + a > c + a
		\tag{$*$}
	\end{align*}
	holds for all $a, b, c \in \No$ with $l(a) \oplus 
	l(b) < \gamma$ and $l(a) \oplus l(c) < \gamma$. Then 
	for all $a, b \in \No$ with $l(a) \oplus l(b) \leq \gamma$ we have 
	\begin{align*}
		(L(a) + b) \cup (L(b) + a) < 
		(R(a) + b) \cup (R(b) + a)
	\end{align*}
	and defining $a + b$ as in Equation \ref{defn_of_surreal_sum}, 
	$(*)$ holds for all $a, b, c \in \No$ with $l(a) \oplus 
	l(b) \leq \gamma$ and $l(a) \oplus l(c) \leq \gamma$. 
\end{lem}
\begin{proof}
	The first part is immediate from $(*)$ in conjunction with the 
	fact that $l(a_L), l(a_R) < l(a)$, $l(b_L), l(b_R) < l(b)$
	for all $a_L \in L(a), a_R \in R(a)$, $b_L \in L(b)$, and 
	$b_R \in R(b)$. 
Define $a + b$ for $a, b \in \No$ with $l(a) \oplus l(b) \leq 
\gamma$ as in Equation \ref{defn_of_surreal_sum}. Suppose 
$a, b, c \in \No$ with $l(a) \oplus l(b), l(a) \oplus l(c) \leq 
\gamma$, and $b > c$. Then by definition we have 
\begin{align*}
	a + b_L < \;& a + b \\
	& a + c < a + c_R
\end{align*}
for all $b_L \in L(b)$ and $c_R \in R(c)$. If $c <_s b$ then 
we can take $b_L = c$ and get $a + b > a + c$. Similarly, if 
$b <_s c$, then we can take $c_R = b$ and also get $a + b > a + c$. 
Suppose neither $c <_s b$ nor $b <_s c$ and put 
$d \coloneq b \wedge c$. Then $l(d) < l(b), l(c)$ and 
$b > d > c$. Hence by $(*)$, $a + b > a + d > a + c$. 

We may show $b + a > c + a$ similarly. 
\end{proof}
\begin{lem}[``Uniformity'' of the Definition of $a$ and $b$]
	Let $a = \curly{L \vert R}$ and $a' = \curly{L' \vert R'}$. 
	Then
	\begin{align*}
		a + a' = 
		\left\{ (L + a') \cup (a' + L) \vert 
		(R + a') \cup (a + R') \right\}
	\end{align*}
\end{lem}
\begin{proof}
	Let $a = \curly{L_a \vert R_a}$ be the canonical 
	representation. By Corollary \ref{inverse_cofinality_theorem}
	$(L, R)$ is cofinal in $(L_a, R_a)$ and $(L', R')$ is 
	cofinal in $(L_{a'}, R_{a'})$. Hence 
	\begin{align*}
		\paren{(L + a') \cup (a + L'), (R+a') \cup (a + R')}
	\end{align*}
	is cofinal in 
	\begin{align*}
		\paren{(L_a + a') \cup (a + L_{a'}), (R_a + a') \cup 
		(a + R_{a'})}
	\end{align*}
	Moreover, 
	\begin{align*}
		(L + a') \cup (a + L') < a + a' < 
		(R + a') \cup (a + R')
	\end{align*}
	Now use Theorem \ref{cofinality_theorem} to conclude the 
	proof. 
\end{proof}
\section{Week 2}

Lemma The map

\begin{align*}
	K &\arr \No \\
	f(x) &\mapsto f(\w)
\end{align*}

is onto.


%\documentclass{article}
%\usepackage{amsmath}
%\usepackage{amssymb}
%\usepackage{amsthm}
%
%\newcommand{\N}{\mathbb{N}}
%\newcommand{\Z}{\mathbb{Z}}
%\newcommand{\Q}{\mathbb{Q}}
%\newcommand{\R}{\mathbb{R}}
%\newcommand{\concat}{\mathrel{\hat{\ }}}
%
%\theoremstyle{definition}
%\newtheorem{cor}{Corollary}
%\newtheorem{example}{Example}
%\newtheorem{proposition}{Proposition}
%\newtheorem{lemma}{Lemma}
%\newtheorem{defn}{Definition}
%\newtheorem{theorem}{Theorem}
%
%\begin{document}

Notes by Zach.

\section*{October 20th}

\begin{lemma}
For $n\in\N$, $n\cdot 1 = (\underbrace{++\cdots+}_{n\text{ times}})$.
\label{3.1}
\end{lemma}

\begin{proof} (of 3.1)
By induction on $n$. The cases $n=0,1$ are obvious.
Suppose that $n\ge 1$. We have
\begin{align*}
(n+1)\cdot 1 &= n\cdot 1 + 1 \\
&= (++\cdots +)+(+) \\
&= \{ (\underbrace{+\cdots+}_{n-1}) \,|\, \emptyset \} + \{ 0 \,|\, \emptyset \} \\
&= \{ (n-1)\cdot1 \,|\, \emptyset \} + \{ 0 \,|\, \emptyset \} \\
&= \{(n-1)\cdot 1 + 1,\, n\cdot 1 + 0 \,|\, \emptyset \} \\
&= \{ n\cdot 1 \,|\, \emptyset \} \\
&= \{ (\underbrace{++\cdots + }_{n}) \,|\, \emptyset \} \\
&= (\underbrace{++\cdots +}_{n+1}).
\end{align*}
\end{proof}

\begin{cor}
For $n\in\N$, $-n\cdot 1 = (\underbrace{- - \cdots -}_{n\text{ times}})$.
\end{cor}

An ordered field $k$ is {\em archimedean} if for all $a,b>0$ in $k$ there is $n\in\N$ such that $na>b$.

Note: $\mathbf{No}$ is not archimedean, since $\omega := (++\cdots)$ (with $\omega$ many $+$s) satisfies $\omega>(\underbrace{+\cdots + }_{n})=n\cdot 1$.

From now on we identify $\Z$ as a subring of $\mathbf{No}$.

Q: How to identify $\Q\subseteq \mathbf{No}$?

Idea: $0 = () < (+-) < (+) = 1$. Is $(+-)=\tfrac12$?

Finite sequences of $+$s and $-$s correspond to dyadic rationals, i.e., rationals of the form $\frac{a}{2^s}$ ($a\in\Z$, $s\in\N$). We might conjecture that $\Q$ corresponds to finite sequences in $\mathbf{No}$.

\begin{lemma}
Suppose $a+b = \{2a \,|\, 2b \}$. Then $\frac{a+b}2 = \{a \,|\, b \}$.
\end{lemma}

\begin{proof}
Put $c := \{a \,|\, b\}$. Then $2\cdot c = c+ c = \{a+c \,|\, b+c \}$, and we show that this equals $a+b$:
\[ a+c < a+b < b+c, \text{ since } a< c < b, \text{ and } \]
\[ 2a < a+c, \; 2b > b+c \text{ because } a<c < b. \]
Now the result follows by the uniformity of the definition of $+$: since we assumed $a+b = \{2a \,|\, 2b \}$, we get the claim by cofinality.
\end{proof}

\begin{example}
$1 = \{ 0 \,|\, \emptyset \} = \{ 0 \,|\, 2 \}$. Apply the lemma with $a=0$, $b=1$.
So $\tfrac12 = \{0 \,|\, 1 \} = (+-)$. Taking $a=0$, $b=\tfrac12$, we get $\tfrac14 = (+--)$.
\end{example}

\begin{cor} (3.4)
Suppose $a + b = \{ 2a \,|\, 2b \}$. Then 
\[ \frac{a+b}{2^{s+1}} = \left\{ \frac{a}{2^s} \,\bigg|\, \frac{b}{2^s} \right\}  \]
for all $s\in\N$.
\label{3.4}
\end{cor}

\begin{cor} (3.5)
For all $c\in\N$, $\frac{c}{2^s} + \frac{1}{2^{s+1}} = \left\{ \frac{c}{2^s} \,|\, \frac{c+1}{2^s} \right\}$.
\label{3.5}
\end{cor}

\begin{proof} (Proof of 3.5)
Take $a=c$, $b=c+1$. We have
\begin{align*}
a+b &= 2c+1 \\
&= \{ 2c \,|\, 2c+2 \} \\
&= \{ 2a \,|\, 2b \}.
\end{align*}
Apply Corollary \ref{3.4}.
\end{proof}

\begin{proposition} (3.6)
Surreal numbers of finite length correspond to dyadic rationals.
\label{3.6}
\end{proposition}

\begin{proof} (Proof of 3.6)
Let $d\in\mathbf{No}$ have length $m+n$, where $d(0) = d(1) = \cdots = d(m-1) \ne d(m)$. We'll show that 
$d\in\frac1{2^n}\Z$. Suppose $d(0) = d(m-1) = {+}$. (Similar if $d(0) = {-}$.) 
\begin{description}
\item $n=0$: follows by \eqref{3.1}.
\item $n=1$: Then $d = (\underbrace{++\cdots+}_{m\ge1}-)$. By \eqref{3.5} with $c = m-1$, $s=0$, $m-\frac12 = \frac{m-1}{2^0} + \frac{1}{2^1} = \{m-1 \,|\, m\}$. Clearly $\{m-1 \,|\, m\} = d$.
\end{description}
Now suppose we've shown the claim for all $n\ge r$, and let $n = r+1$; suppose $r\ge 1$. Let $d' = d\restriction (m+r)$.
Either $d = d'\concat (+)$ or $d=d'\concat (-)$; suppose wlog that $d = d'\concat (+)$. So $d' = (\underbrace{++\cdots+}_{m}-\cdots)$. Let $d = \{L \,|\, R\}$ be the canonical representation.
Taking $x=\max L$, $y = \min R$, we have $d = \{x \,|\, y\}$ by cofinality. Note $x = d'$. We don't know much about $y$ 
except that
\[ d = (\underbrace{++\cdots +}_{m}-\cdots -)\le (\underbrace{++\cdots+}_{m}) = m, \]
so $y\le m$. By inductive hypothesis, $x,y\in \frac1{2^r}\Z$. So $d' = x = \frac{c}{2^r}$ for some $c\in\N$. 
If we can show $y = \frac{c+1}{2^r}$, then by \eqref{3.5} 
\[ d = \{x \,|\, y \} = \frac{c}{2^r} + \frac{1}{2^{r+1}} \in \frac1{2^{r+1}}\Z, \]
as required.

Note 
\[m-1<x = d' = \frac{c}{2^r} < \frac{c+1}{2^r} \le y \le m. \]
Put $H := \{ h\in\mathbf{No} \,:\, \ell(h)\le m+ r \,\&\, h\restriction(m+1) = (++\cdots+-)\}$.
We have $|H| = 1+2+2^2+\cdots + 2^{r-1} = 2^r - 1$. By inductive hypothesis every $h\in H$ belongs to $\frac1{2^r}\N$,
and $m-1<h<m$. But there are exactly $2^r - 1$ many dyadic rationals satisfying both of these conditions. 
In particular, $\frac{c+1}{2^r}\in H$ or $\frac{c+1}{2^r}=m$. Either way, $\ell(\frac{c+1}{2^r})\le m+r$.
Since $\frac{c+1}{2^r} > d'$, this implies $\frac{c+1}{2^r} > d$. We have $\frac{c+1}{2^r} <_s d$:
otherwise $e := \frac{c+1}{2^r} \wedge d$ will satisfy $d < e < \frac{e+1}{2^r} \le y$, contradiction to choice of $y$.
Hence $\frac{c+1}{2^r}\in R$, so $\frac{c+1}{2^r}\ge y$.
\end{proof}

\section*{October 22nd}

\textbf{Remark} (to \eqref{3.6}). Suppose $d\in\mathbf{No}$ has length $m+n$, where
\begin{itemize}
\item $d(0) = \cdots = d(m-1)$, and
\item $d(m-1)\ne d(m)$.
\end{itemize}
Define
\[ b(i) := \begin{cases} \pm 1 & \text{if } i< m, \quad d(i)=\pm \\
\pm \frac{1}{2^{i-m+1}} & \text{if } i \ge m, \quad d(i) = \pm.
\end{cases} \]
Then $d = b(0) + b(1) + \cdots + b(m+n-1)$. (Exercise.) Also, every dyadic rational arises in this way. (Exercise.)
We now let $\mathbb{D}$ be the set of dyadic rationals $\Z[\frac12]\subseteq\Q$.

\begin{defn}
A surreal is called {\em real} if it is either of finite length or has length $\omega$ and is not ultimately constant.
\end{defn}

Recall: An ordered field $k$ is {\em dedekind-complete} if every nonempty subset of $k$ that is bounded from above
has a supremum. The ordered field $\R$ is up to (unique) isomorphism the only dedekind-complete ordered field.

\begin{theorem}[Conway] (3.8)
The real surreals form a dedekind-complete ordered subfield of $\mathbf{No}$. Let $a\in\mathbf{No}$, $\ell(a)=\omega$,
with canonical representation $a = \{L \,|\, R\}$ then
\[ a\text{ is real } \iff \begin{cases} L,R\neq\emptyset \\
L\text{ has no max} \\
R\text{ has no min}
\end{cases}.\]
\end{theorem}

The direction ``$\Rightarrow$'' is an exercise. Now we prove the other direction.

\begin{lemma} (3.9)
Let $L,R\subseteq\mathbb{D}$ be such that $L<R$ and $L$ has no max and $R$ has no min. Then $a = \{ L \,|\, R\}$ is real.
\end{lemma}

\begin{proof}
By \eqref{1.9} we have $\ell(a)\le\omega$. Suppose $\ell(a) = \omega$ and $a(n) = +$ eventually. Note that there is some $n_0$ with $a(n_0) = {-}$, since $a<R\ne\emptyset$. We may assume that $a(n) = {+}$ for all $n>n_0$. Let $b = a\restriction n_0$. Then $b>a$ and $b<_s a$. Since $(L,R)$ is cofinal in the canonical representation of $a$, there is $d\le b$, $d\in R$. It follows that $d\in R$, since $R$ has no least element. We can choose $m$ such that $d\le b - \frac1{2^m}$. (Possible since both $b$ \& $d$ are dyadics.) We get $a<d\le b - \frac1{2^m}$. Next let $n>n_0$, $c = a\restriction n$. Then
$c<a$ and $c <_s a$, so by choosing $n$ sufficiently large we can achieve $c>b-\frac1{2^m}$. But then $a>c>b-\frac{1}{2^m}$, oops.

(Why can we do this? Write
\[ c = k + \sum_{i= l}^{n-1} \pm\frac{1}{2^{i-l+1}}\]
and
\[ b = k + \sum_{i=l}^{n_0-1} \pm\frac{1}{2^{i-l+1}}. \]
Then we get
\[ c-b = -\frac{1}{2^{n_0-l+1}} + \frac{1}{2^{n_0-l+2}} + \cdots + \frac{1}{2^{n-l}} = -\frac{1}{2^{n_0-l+1}} + \left( \frac{1}{2^{n_0-l+1}} - \frac{1}{2^{n-l}} \right) = \frac{-1}{2^{n-l}}, \]
so just choose $n>l+m$.)
\end{proof}

\begin{lemma}
Let $a = \{L \,|\, R\}$. Suppose
\begin{enumerate}
\item $x\in L \Rightarrow \exists r\in \mathbb{D}^{>0}$, $y\in L$, $x+r\le y$. [``$L-\mathbb{D}^{>0}$ is cofinal in $L$'']
\item $x\in R \Rightarrow \exists r\in \mathbb{D}^{>0}$, $y\in R$, $y\le x-r$. [``$R+\mathbb{D}^{>0}$ is coinitial in $R$.'']

and also $L' < a < R'$ such that
\item $(\forall r\in \mathbb{D}^{>0})(\exists x'\in L')(\exists y'\in R') y'-x'\le r$. [``$R'-L'$ is coinitial in $\mathbb{D}^{>0}$'']
\end{enumerate}
Then $a = \{ L' \,|\, R' \}$. 
\end{lemma}

\begin{proof}
Check that $(L',R')$ is cofinal in $(L,R)$ and use the cofinality theorem.
\end{proof}

\begin{lemma}
$\mathbb{D}$ is dense in the ordered set of real surreals.
\end{lemma}

\begin{proof}
Let $a<b$ be reals. If neither $a <_s b$ nor $b <_s a$ then $a < a\wedge b < b$. (Note $a\wedge b$ is finite, so it's in $\mathbb{D}$.) So suppose that $a <_s b$. Then $a\in \mathbb{D}$. If also $b\in\mathbb{D}$, then we're done:
$a < \frac{a+b}2 < b$. So suppose $b\notin \mathbb{D}$, with canonical representation $b = \{ L \,|\, R\}$. Then $a\in L$,
and $L$ has no maximum, so we can find some dyadic element of $(a,b)$. Similar if $b <_s a$.
\end{proof}

\begin{lemma}
Let $a = \{ L \,|\, R \}$ be the canonical representation of a real $a\notin \mathbb{D}$. Then for all $r\in \mathbb{D}^{>0}$ there are $a_L, a_R$ with $a_R-a_L\le r$. 
\end{lemma}

\begin{proof}
For each $n$ there are $a_L,a_R$ with $a_L\restriction n = a_R\restriction n$, and $a_R-a_L$ is bounded from above by some expression $\frac1{2^s} + \frac{1}{2^{s+1}}+\cdots$, and this can be made as small as necessary. (Exercise.)
\end{proof}

\section*{October 24th}

\begin{proof}[Proof of Theorem \ref{3.8}]
Clearly $0,1$ are real. Let $a,b\in\mathbf{No}$ be real; we check that $a+b$, $a\cdot b$, $\frac1a$ (if $a\ne0$) are also real. (Note that $-a$ is obviously real.)

\subsubsection*{$a+b$} Suppose $a\in\mathbb{D}$, $b\notin\mathbb{D}$. Let $a = \{ L \,|\, R\}$ be the canonical representation. So $a + b = \{ a_L + b,a+b_L \,|\, a_R+b,a+b_R\}$. We claim that $a+b = \{a+b_L \,|\, a+b_R \}$. (By \eqref{3.9} this then gives $a+b$ real, since $b_L$, $b_R$ dyadic.)
We have $a,a_L\in\mathbb{D}$, so $a-a_L\in\mathbb{D}^{>0}$; hence by \eqref{3.12} there are $b_L,b_R$ such that $b_R-b_L\le a-a_L$. It follows that $a+b_L\ge a_L+b_R \ge a_L+b$. Now use cofinality. (Similar argument for the other side.)

Now suppose that $a,b\notin\mathbb{D}$. Then in the representation of $a+b$ the LHS has no max and the RHS has no min. So (1) \& (2) in \eqref{3.10} are satisfied for this cut. Let $L' := \{a_L+b_L\}$, $R' := \{a_R+b_R\}$. Then
$L' < a+b < R'$ and by \eqref{3.12} $(L',R')$ satisfies (3) in \eqref{3.10}. This means that $a+b = \{L' \,|\, R' \}$ by \eqref{3.10}, and this is real by \eqref{3.9}.

\subsubsection*{$a\cdot b$} Suppose $a\notin\mathbb{D}$, $a,b>0$. The typical element in the representation of 
$a\cdot b$ is $ab-(a-a_*)(b-b_*)$. Show that (1), (2) in \eqref{3.10} are satisfied by this cut. For example,
\[ x = ab - (a-a_L)(b-b_L).\]
Take $a_L'$ with $a_L < a_L' < a$ and set $x' = ab - (a-a_L')(b-b_L)$. Then by \eqref{3.11} $x'-x = (a_L' - a_L)(b-b_L)$ is greater than some element of $\mathbb{D}^{>0}$. This verifies (1) of \eqref{3.10}. In the same way verify (2). Now set
\begin{align*}
L' &:= \{ a'b' \,:\, a',b'\in\mathbb{D},\, 0\le a' < a, \, 0\le b' < b\}, \\
R' &:= \{ a''b'' \,:\, a'',b''\in\mathbb{D}, \, a'' > a, \, b'' > b \}.
\end{align*}
Then $L' < a\cdot b < R'$. We check that \eqref{3.10}(3) holds. Let $r\in \mathbb{D}^{>0}$ be given. Then the same 
argument as proving the limit law for multiplication in calculus gives elements $a',b',a'',b''$ such that $a''b''-a'b'<r$, using \eqref{3.12}. Hence by \eqref{3.10} $a\cdot b = \{ L' \,|\, R' \}$, so $ab$ is real by \eqref{3.9}.

\subsubsection*{$1/a$:} We may assume that $a>0$. Put
\begin{align*}
L &:= \{ d\in \mathbb{D} \,:\, d\concat a < 1 \} \\
R &:= \{ d\in \mathbb{D} \,:\, d\concat a > 1 \}.
\end{align*}
Then $L<R$, $0\in L$, and $R\ne\emptyset$ because: by \eqref{3.11} take $m$ such that $a>\frac1{2^m}$, so that $2^ma>1$. So $2^m\in R$.

\noindent{\bf Claim 1.}
$L$ has no max; $R$ has no min.

\begin{proof}[Proof of Claim 1.]
Let $d\in L$. Then $1-da>0$ is real. So there is some $m$ such that $1-da> \frac1{2^m}$. Also can take $n$ such that $a< 2^n$. Then $\frac{1}{2^{m+n}}a < \frac1{2^m} < 1-da$, so $a(\frac{1}{2^{m+n}} + d) < 1$.
\end{proof}

Hence by \eqref{3.9} $b:= \{L \,|\, R\}$ is real. We are going to show that $|ba-1|<r$ for every $r\in\mathbb{D}^{>0}$. Since $ba-1$ is real, we get $ba-1=0$ by \eqref{3.11}.

\noindent{\bf Claim 2.}
For each $n$ there is $c\in R$ such that $ca\le 1 + \frac1{2^n}$, and there is $c'\in L$ such that $c'a\ge 1  - \frac{1}{2^n}$. 

(Obviously this suffices.)

\begin{proof}[Proof of Claim 2.]
Choose $m$ such that $2^m>a$. Put $S := \{ r\in\N \,:\, (\frac{r}{2^{m+n}})a>1 \}$. Then $S\ne\emptyset$; take $s = \min S$. Then $\frac{s}{2^{m+n}}\in R$, but $(\frac{s-1}{2^{m+n}})a\le 1$. So
\[ \frac{s}{2^{m+n}} a \le 1 + \frac{a}{2^m2^n} \le 1 + \frac1{2^n}.\qedhere \]
\end{proof}
This completes the proof.
\end{proof}

%\end{document}
\include{week_4/week_4}
\include{week_5/week_5}
\include{week_6/week_6}
\NotesBy{Notes by Tyler Arant}
\Week{Week 7}
\Day{11/17, 11/19, 11/21}

\begin{lemma}[Associativity] \label{6.7} Let $\alpha, \beta \in \textbf{On}$, $(a_i)_{i<\alpha+\beta}$ be a strictly decreasing sequence in $\textbf{No}$ and $f_i\in \mathds{R}$ for $i<\alpha+\beta$.  Then,
$$\sum_{i<\alpha+\beta} f_i\omega^{a_i}=\sum_{i<\alpha} f_i\omega^{a_i} + \sum_{j<\beta}f_{\alpha+j}\omega^{a_{\alpha+j}}.$$
\end{lemma}

\begin{proof} We proceed by induction on $\beta$.  In the case that $\beta=\gamma+1$ is a successor ordinal, we have
\begin{align*} \sum_{i<\alpha+(\gamma+1)} f_i\omega^{a_i}&= \sum_{i<\alpha+\gamma} f_i\omega^{a_i} + f_{\alpha+\gamma}\omega^{a_{\alpha+\gamma}} \\
		&= \sum_{i<\alpha} f_i\omega^{a_i}+ \sum_{j<\gamma} f_{\alpha+j} \omega^{a_{\alpha+j}}+ f_{\alpha+\gamma}\omega^{a_{\alpha+\gamma}}\\
		& = \sum_{i<\alpha} f_i\omega^{a_i} + \sum_{j<\gamma+1}f_{\alpha+j}\omega^{a_{\alpha+j}}, \end{align*}
where the first and third equality use the definition of $\sum$ and the second equality uses the induction hypothesis.  

In the case where $\beta$ is a limit ordinal, we let 
$$\{L | R\} = \sum_{j<\beta}f_{\alpha+j}\omega^{a_{\alpha+j}}.$$
Using the definition of addition between surreal numbers and a simple cofinality argument, we obtain
$$\sum_{i<\alpha}f_i\omega^{a_i} + \sum_{j<\beta}f_{\alpha+j}\omega^{a_{\alpha+j}} = \left \{\sum_{i<\alpha}f_i\omega^{a_i} + L \biggl | \sum_{i<\alpha}f_i\omega^{a_i} + R \right \}.$$
A typical element of this cut is 
$$\sum_{i<\alpha}f_i\omega^{a_i} + \sum_{j\leq \gamma} f_{\alpha+j}\omega^{a_{\alpha+j}}-\varepsilon \omega^{a_{\alpha+\gamma}}  \qquad (\gamma<\beta, \varepsilon \in \mathds{R}^{>0}).$$
By inductive hypothesis, this equals
$$\sum_{i<\alpha+\gamma}f_i \omega^{a_i}- \varepsilon \omega^{a_{\alpha+\gamma}}.$$
But these elements are cofinal in the cut defining $\sum_{i<\alpha+\beta} f_i\omega^{a_i}$; hence, the claim follows by cofinality. 
\end{proof}

\begin{proposition}  Let $\alpha \in \textbf{On}$, $(a_i)_{i<\alpha}$ be a strictly decreasing sequence in $\textbf{No}$ and $f_i, g_i\in \mathds{R}$ for $i<\alpha$. Then, 
$$\sum_{i<\alpha}f_i\omega^{a_i} + \sum_{i<\alpha} g_i \omega^{a_i} = \sum_{i<\alpha}(f_i+g_i)\omega^{a_i}.$$
\end{proposition}

\begin{proof} We proceed by induction on $\alpha$.  If $\alpha=\beta+1$ is a successor, then
\begin{align*} \sum_{i<\beta+1}f_i\omega^{a_i} + \sum_{i<\beta+1} g_i \omega^{a_i} & = \left ( \sum_{i<\beta} f_i\omega^{a_i} + f_\beta \omega^{a_\beta} \right ) + \left (  \sum_{i<\beta} g_i \omega^{a_i} + g_\beta \omega^{a_\beta} \right ) \\
	& = \left ( \sum_{i<\beta} f_i\omega^{a_i} + \sum_{i<\beta} g_i \omega^{a_i} \right ) + ( f_\beta \omega^{a_\beta} + g_\beta \omega^{a_\beta}) \\
	& = \sum_{i<\beta} (f_i+g_i) \omega^{a_i} + (f_\beta+g_\beta)\omega^{a_\beta} \\
	& = \sum_{i<\beta+1}(f_i+g_i)\omega^{a_i}, \end{align*}
where the third equality uses the induction hypothesis.

Now suppose $\alpha$ is a limit.  One type of element from the lef-hand-side of the cut defining   $\sum_{i<\alpha}f_i\omega^{a_i} + \sum_{i<\alpha} g_i \omega^{a_i}$ is of the form
$$\sum_{i\leq \beta} f_i \omega^{a_i}-\varepsilon  \omega^{a_\beta} +\sum_{i<\alpha} g_i \omega^{a_i}$$
or of the form
$$\sum_{i<\alpha} f_i \omega^{a_i} +\sum_{i\leq \beta} g_i \omega^{a_i} -\varepsilon  \omega^{a_\beta}.$$
We have
\begin{align*} \sum_{i\leq \beta} f_i \omega^{a_i}-\varepsilon  \omega^{a_\beta} +\sum_{i<\alpha} g_i \omega^{a_i} 
 & = \sum_{i\leq \beta} f_i \omega^{a_i} + \sum_{i\leq \beta} g_i \omega^{a_i} + \sum_{\beta< i<\alpha} g_i \omega^{a_i} -\varepsilon  \omega^{a_\beta} \\
  & = \sum_{i\leq \beta}(f_i+g_i)\omega^{a_i} + \sum_{\beta< i<\alpha} g_i \omega^{a_i} -\varepsilon  \omega^{a_\beta}, \end{align*}
  where the first equality follows from $(\ref{6.7})$ and the second equality uses the inductive hypothesis.  But this is mutually cofinal with 
 $$\sum_{i\leq \beta}(f_i+g_i)\omega^{a_i} - \varepsilon \omega^{a_\beta}.$$
 Similarly if we star with $\sum_{i<\alpha} f_i \omega^{a_i} +\sum_{i\leq \beta} g_i \omega^{a_i} -\varepsilon  \omega^{a_\beta}.$
\end{proof}


\begin{lemma} \label{6.8} Let $\alpha \in \textbf{On}$, $(a_i)_{i<\alpha}$ be a strictly decreasing sequence in $\textbf{No}$, $b\in \textbf{No}$, and $f_i\in \mathds{R}$ for $i<\alpha$. Then,
$$\left ( \sum_{i<\alpha} f_i\omega^{a_i} \right ) \omega^b = \sum_{i<\alpha}f_i\omega^{a_i+b}.$$
Note that the sequence $(a_i+b)_i$ is also strictly decreasing.\end{lemma}

\begin{proof} We proceed by induction on $\alpha$.  If $\alpha=\beta +1$, then
\begin{align*}\left ( \sum_{i<\beta + 1} f_i\omega^{a_i} \right ) \omega^b 
	&= \left ( \sum_{i<\beta} f_i\omega^{a_i} + f_\beta \omega^{a_\beta} \right )\omega^b \\
	& = \left ( \sum_{i<\beta} f_i\omega^{a_i}\right ) \omega^b + f_\beta \omega^{a_\beta}\cdot \omega^b \\
	& = \sum_{i<\beta}f_i\omega^{a_i+b} + f_\beta\omega^{a_\beta+b} \\
	& = \sum_{i<\beta+1}f_i\omega^{a_i+b}, \end{align*}
where the third equality uses the inductive hypothesis.  

Now suppose $\alpha$ is a limit.  Recall that, by their respective definitions,
$$\omega^b=\{0, s\omega^{b'} \ | \ t\omega^{b''}\}$$
and
$$\sum_{i<\alpha}f_i\omega^{a_i} = \left \{ \sum_{i\leq \beta} f_i\omega^{a_i} - \varepsilon \omega^{a_\beta} \ : \ \beta<\alpha, \varepsilon\in \mathds{R}^{>0} \ \biggl | \  \sum_{i\leq \beta} f_i\omega^{a_i} + \varepsilon \omega^{a_\beta} \ : \ \beta<\alpha, \varepsilon\in \mathds{R}^{>0}\right \}.$$
Set $d:=\sum_{i<\alpha}f_i\omega^{a_i}$ and let $d', d''$ be elements from the left and right-hand sides, respectively, of the defining cut determined by the same choice of $\varepsilon$.  Note that
$$d-d' = \varepsilon \omega^{a_\beta} + c', \quad \text{where} \ c'\ll \omega^{a_\beta},$$
and
$$d''-d = \varepsilon \omega^{a_\beta} + c'', \quad \text{where} \ c''\ll \omega^{a_\beta}.$$	
It follows that
\begin{equation}\varepsilon_1\omega^{a_\beta}<d-d', d''-d<\varepsilon_2\omega^{a_\beta}, \quad \text{for all $\varepsilon_1<\varepsilon<\varepsilon_2$ in $\mathds{R}$}, \tag{$*$}\end{equation}
where $\varepsilon$ is given by the choice of $d', d''$.  Now,
\begin{align*} d\omega^b & = \{d' \ | \ d''\} \cdot \{0, s\omega^{b'} \ | \ t\omega^{b''}\} \\
			& = \{d'\omega^b, d' \omega^b+(d-d')s\omega^{b'}, \underline{d''\omega^b-(d''-d)t\omega^{b''}} \ | \\
			& \qquad  d''\omega^b, \underline{d'\omega^b+(d-d')t\omega^{b''}}, d''\omega^b-(d''-d)s\omega^{b'}\},\end{align*}
and we claim that the underlined terms are superfluous; in particular, 
\begin{enumerate}[(1)]
\item $d''\omega^\beta - (d''-d)t\omega^{b''} \leq d'\omega^b + (d-d')s\omega^{b'};$
\item  $d''\omega^b- (d''-d)s\omega^{b'}\leq d'\omega^b + (d-d')t\omega^{b''}$.
\end{enumerate}
To show (1), note that $\omega^{b''}\gg \omega^b\gg \omega^{b'}$ implies
$$(d''-d)t\omega^{b''}+(d-d')s\omega^{b'}\geq \varepsilon_1\omega^{a_\beta}t\omega^{b''} > 2\varepsilon_2 \omega^{a_\beta}\omega^b\geq (d''-d)\omega^b.$$
The verification for (2) is similar.  So, by (1), (2) and confinality, 
$$ d\omega^b =  \{d'\omega^b, d' \omega^b+(d-d')s\omega^{b'}  \ | \
		 d''\omega^b,  d''\omega^b-(d''-d)s\omega^{b'}\}.$$
We claim that we can further simplify this to 
$$d\omega^b=\{d'\omega^b \ | \ d''\omega^b\},$$
then we are done by inductive hypothesis.  Let now $\varepsilon_{1, 2}\in \mathds{R}^{>0}$ with $\varepsilon_1<\varepsilon<\varepsilon_2$ and
$$d_1'= \sum_{i\geq \beta}f_i\omega^{a_i}-\varepsilon_1\omega^{a_\beta}, \quad d_1''=\sum_{i\geq \beta}f_i\omega^{a_i}+\varepsilon_1\omega^{a_\beta}.$$
We claim that 
$$d_1'\omega^b>d'\omega^b+(d-d')s\omega^{b'}, \quad d_1''\omega^b<d''\omega^b-(d''-d)s\omega^{b'}.$$
Notice that the first claim holds if and only if $(d_1'-d')\omega^b>(d-d')s\omega^{b'}$.  But this inequality holds since
$$(d_1'-d)\omega^b= (\varepsilon-\varepsilon_2)\omega^{a_\beta}\omega^b> \varepsilon_2s\omega^{a_\beta}\omega^{b'} \geq (d-d')s\omega^{b'},$$
where the first inequality holds since $\omega^b\gg \omega^{b'}$ and the second inequality holds by $(*)$.  The second part of the claim is proved similarly.
\end{proof}

\begin{proposition} Let $\alpha, \beta \in \textbf{On}$, $(a_i)_{i<\alpha}$, $(b_j)_{j<\beta}$ be strictly decreasing sequences in $\textbf{No}$, and $f_i, g_i\in \mathds{R}$ for $i<\alpha$. Then,
$$\left (\sum_{i<\alpha}f_i\omega^{a_i} \right ) \left ( \sum_{j<\beta}g_j\omega^{b_j} \right ) = \sum_{i<\alpha, j<\beta} f_ig_j \omega^{a_i+b_j}.$$
\end{proposition}

\begin{proof} If either $\alpha$ or $\beta$ are successor ordinals, we verify the proposition by using the inductive hypothesis and lemma $(\ref{6.8})$.  Thus, we only need to consider the case where $\alpha$ and $\beta$ are both limits.  Put 
$$f=\sum_{i<\alpha}f_i X^{a_i}, \quad g=\sum_{j<\beta}g_jX^{a_j}\in K.$$
Recall that the typical element in the cut of $f(\omega)\cdot g(\omega)$ is 
\begin{equation} f(\omega)g(\omega)_{**} + f(\omega)_{*}g(\omega)-f(\omega)_*g(\omega)_{**}, \tag{$\dagger$} \end{equation}
where $*, **$ are either $L$ or $R$.  Moreover, this element is $<f(\omega)g(\omega)$ if and only if $(*, **)=(L, L)$ or $(R, R)$.  Take $f_*, g_{**}\in K$ such that $f_*(\omega)= f(\omega)_*$ and $g_{**}(\omega)=g(\omega)_{**}$.  Then, by inductive hypothesis, $\dagger$ equals
$$(f\cdot g)(\omega) -((f-f_*)(g-g_{**}))(\omega).$$
For example, 
$$f_*=\sum_{i<\gamma}f_iX^{a_i} + (f_\gamma\pm \varepsilon_1)X^{a_\gamma}, \quad \gamma<\alpha$$
implies $f-f_*= \pm \varepsilon_1X^{a_\gamma}+h_1$, where all the terms in $h_1$ have degree $>\gamma$. Similarly, $g-g_{**}= \pm \varepsilon_2X^{b_\delta} + h_2$, where $\delta <\beta$ and all the terms in $h_2$ have degree $>\delta$.  Thus,
$$(f-f_*)(g-g_{**})= \pm \varepsilon_1\varepsilon_2 X^{a_\gamma + b_\delta} + \text{higher order terms},$$
and
$$[(f-f_*)(g-g_{**})](\omega) = \pm \varepsilon_1\varepsilon_2\omega^{\alpha_\gamma+b_\delta} + h_3(\omega),$$
where $h_3(\omega)\ll\omega^{a_\gamma+b_\delta}$.  So by cofinality,
\begin{align*}f(\omega)g(\omega) &=\{ (f\cdot g)(\omega)-\varepsilon\omega^{a_\gamma+b_\delta} \ : \ \gamma<\alpha, \delta < \beta, \varepsilon\in \mathds{R}^{>0} \ | \\  
 &\qquad (f\cdot g)(\omega)+\varepsilon\omega^{a_\gamma+b_\delta} \ : \ \gamma<\alpha, \delta < \beta, \varepsilon\in \mathds{R}^{>0} \}. \end{align*}
 Now,
\begin{align*}(f\cdot g)(\omega) &=\{ (f\cdot g)(\omega)-\varepsilon\omega^{a_\gamma+b_\delta} \ : \ \gamma<\alpha, \delta < \beta \ \text{s.t} \ a_\alpha+b_\delta \in \text{supp}(f\cdot g), \varepsilon\in \mathds{R}^{>0} \ | \\  
 &\qquad f\cdot g(\omega)+\varepsilon\omega^{a_\gamma+b_\delta} \ : \ \gamma<\alpha, \delta < \beta \ \text{s.t} \ a_\alpha+b_\delta \in \text{supp}(f\cdot g),\varepsilon\in \mathds{R}^{>0} \}. \end{align*}
Thus, $(f\cdot g)(\omega)$ satisfies the cut for $f(\omega)\cdot g(\omega)$ and the claim follows by cofinality.  

\end{proof}

All together, this completes the proof of the following theorem.

\begin{theorem} The map
$$\mathds{R}((t^{\bf No})) \xrightarrow{\sim} {\bf No}, \quad \sum_{i<\alpha}f_iX^{a_i} \mapsto \sum_{i<\alpha} f_i\omega^{a_i},$$
is an ordered field isomorphism. \end{theorem}



\section{The Surreals as a Real Closed Field}

Let $K$ be a field. We call $K$ \textit{orderable} if some ordering on $K$ makes it an ordered field.  If $K$ is orderable, then $\text{char}(K)=0$ and $K$ is not algebraically closed. \footnote{To prove that $K$ is not algebraically closed: suppose $K$ is an algebraically closed ordered field and derive a contradiction using $i$, the square root of $-1$.}  We call $K$ \textit{euclidean} if $x^2+y^2\neq -1$ for all $x, y \in K$ and $K=\{\pm x^2 \ : \ x\in K\}$.  If $K$ is euclidean, then $K$ isa an ordered field for a unique ordering---namely, $a\geq 0 \iff \exists x\in K. x^2=a$.  

\begin{theorem}[Artin $\&$ Schreier, 1927] \label {7.1} For a field $K$, the following are equivalent. 
\begin{enumerate}[(1)]
\item $K$ is orderable, but has no proper orderable algebraic field extension. 
\item $K$ is euclidean and every polynomial $p\in K[X]$ of odd degree has a zero in $K$. 
\item $K$ is not algebraically closed, but $K(i)$, $i^2=-1$, is algebraically closed. 
\item $K$ is not algebraically closed, but has an algebraically closed field extension $L$ with $[L:K]<\infty$. 
\end{enumerate} 
We call $K$ \textit{real closed} if it satisfies one of these equivalent conditions.\footnote{See Lange's \textit{Algebra} for partial proof.}
\end{theorem}

\begin{corollary}\label{7.2} Let $K'$ be a subfield of a real closed field $K$.  Then $K'$ is real closed if and only if $K'$ is algebraically closed in $K$. \end{corollary}

\begin{proof} Suppose $K'$ is not algebraically closed in $K$.  Fix $a\in K\setminus K'$ that is algebraic over $K'$.  Then, $K'(a)$ is an proper orderable algebraic field extension of $K'$.  Thus, $K'$ is not real closed by $(1)$ of theorem $(\ref{7.1})$.

Conversely, suppose $K'$ is algebraically closed in $K$.  We verify that condition (2) of theorem $(\ref{7.1})$ holds for $K'$.  Since $K'$ is algebraically closed in $K$, any zero of a polynomial of the form $X^2-a$ or $-X^2-a$, where $a\in K'$, must be in $K'$.  This along with the fact that $K$ is euclidean implies that $K'$ is euclidean.  Moreover, if $p\in K'[X]$ has odd degree, then since $K$ satisfies (2), $p$ has a zero $a\in K$.  But, $a\in K'$ since $K'$ is algebraically closed in $K$.  Thus, $K$ is real closed.  
\end{proof}

The archetypical example of a real closed field is $\mathds{R}$.  By corollary $(\ref{7.2})$, the algebraic closure of $\mathds{Q}$ in $\mathds{R}$ is also real closed.  In fact, the algebraic closure of $\mathds{Q}$ in $\mathds{R}$ can be embedded into any real closed field.

\begin{proposition} Suppose $K$ is real closed and $p\in K[X]$.  Then,
\begin{enumerate}[(1)]
\item $p$ is monic and irreducible if and only if $p=X-a$ for some $a\in K$ or $p=(X-a)^2+b^2$ for some $a, b\in K$, $b\neq 0$. 
\item The map $x\mapsto p(x): K \rightarrow K$ has the intermediate value theorem.  \end{enumerate}\end{proposition}

\begin{theorem}[Tarksi] The theory of real closed ordered fields in the language $\mathcal{L}=\{0, 1, +, -,  \cdot, \leq\}$ of ordered rings admits quantifier elimination.  Hence, for any real closed field $K$, $\mathds{R}\equiv K$ and, if $\mathds{R}$ is a subfield of $K$, then $\mathds{R}\preceq K$.\end{theorem}

\begin{theorem} Let $\Gamma$ be a divisible ordered abelian group and let $k$ be a real closed field.  Then, $K=k((t^\Gamma))$ is real closed. \end{theorem}

We have $K[i]\cong k[i]((t^\Gamma))$, so it's enough to show the following theorem.

\begin{theorem} Let $\Gamma$ be a divisible ordered abelian group and let $k$ be an algebraically closed field of characteristic $0$.  Then, $K=k((t^\Gamma))$ is algebraically closed. \end{theorem}

\begin{remark} This theorem is still true if we drop the characteristic $0$ assumption, but it would require a different proof than the one given below. \end{remark}

\begin{proof} Let $P\in K[X]$ be monic and irreducible, and write
$$P=X^n+a_{n-1}X^{n-1}+\cdots +a_0 \quad (a_i\in K, n\geq n).$$
By replacing $P(X)$ by $P(X-a_{n-1})$, we get
$$P\left(X-\frac{a_{n-1}}{n}\right) = X^n + \text{terms of degree $<n-1$}.$$
Thus, we may assume $a_{n-1}=0$.  Put $\gamma_i:=va_i\in \Gamma\cup\{\infty\}$ (recall that $vf:= \min \text{supp} f$ for $f\in K$) and put
$$\gamma:= \min \left \{\frac{1}{n-i}\gamma_i \ : \ i=0, \dots, n-2 \right \} \in \Gamma.$$
Then,
$$t^{-n\gamma}P(t^\gamma X)= X^n + \sum_{i=0}^{n-2}a_it^{(i-n)\gamma}X^i,$$
where $v(a_it^{(i-n)\gamma}) = \gamma_i + (i-n)\gamma\geq 0$, with equality holding for some $i$.  Thus, we may assume $va_i\geq 0$ for all $i$, and $va_i=0$ for some $i$.  

Let $\mathcal{O}:= \{f\in K \ : \ vf\geq 0\}$.  It is readily verified that this is a subring of $K$ which contains $k$.  We have a ring morphism $\mathcal{O}\rightarrow k$ define by
$$f= \sum_{\gamma\geq 0} f_\gamma t^\gamma \mapsto f_0=: \overline{f}.$$

\begin{lemma} Let $P\in \mathcal{O}[X]$ be monic and $\overline{P}=Q_0R_0$, where $Q_0, R_0\in k[X]$ are monic and relatively prime.  Then there are monic $Q, R\in \mathcal{O}[X]$ with $P=QR$ and $\overline{Q}=Q_0$, $\overline{R}=R_0$.  \end{lemma}

The lemma applies to our $P$.  Since $P$ is assumed irreducible, the lemma implies $\overline{P}=(X-a)^n$ for some $a\in k$, i.e., 
$$\overline{P}= X^n-naX^{n-1}+ \text{lower degree terms}.$$
Since $a_{n-1}=0$, we have $na=0$; hence, $a=0$ since $k$ has characteristic $0$.  Thus, $\overline{P}=X^n$.  But, $va_i=0$ for some $i$, so we have a contradiction.  

We now prove the lemma.  Write $P=\sum_{i<\alpha}P_i(X)t^{a_i}\in k[X]((t^\Gamma))$, where $a_i$ is strictly increasing in $\Gamma$, $a_0=0$, $P_i(X)\in k[X]$ are of degree $<n$ for $i>0$, and $P_0=\overline{P}$.  Suppose we have a strictly increasing sequence $(b_i)_{i<\beta}$ in $\Gamma$ and sequences $(Q_i)_{i<\beta}$, $(R_i)_{i<\beta}$ of polynomials in $k[X]$ of degree $<\deg Q_0$ and $<\deg R_0$, respectively, such that for
$$Q_{<\beta}:= \sum_{i<\beta}Q_it^{b_i}, \quad R_{<\beta}:= \sum_{i<\beta}R_it^{b_i}$$
we have 
$$P\equiv Q_{<\beta}R_{<\beta} \mod{(t^b\mathcal{O})}$$
for all $b\in \Gamma$ with $b\leq b_i$ for some $i$.  Suppose $P\neq Q_{<\beta}R_{<\beta}$; we are going to find $b_\beta\in \Gamma$ and $Q_\beta, R_\beta\in k[X]$ of degrees $< \deg Q_0$ and $< \deg R_0$, respectively, such that
\begin{enumerate}%[$\bullet$]
\item $b_\beta >b_i$ for all $i<\beta$.
\item $P\equiv (Q_{<\beta}+Q_\beta t^{b_\beta})(R_{<\beta} + R_\beta t^{b_\beta}) \mod{(t^b\mathcal{O})}$ for all $b\leq b_\beta$.  
\end{enumerate}
To this end, let $\gamma:= v(P-R_{<\beta}Q_{<\beta})\in \Gamma$.  Then, $b_\beta:= \gamma>b_i$ for all $i<\beta$.  Consider any $G, H\in k[X]$; then
$$P\equiv (Q_{<\beta}+Q_\beta t^{b_\beta})(R_{<\beta} + R_\beta t^{b_\beta}) \mod{(t^b\mathcal{O})}$$
for all $b\leq b_\beta$.  To get this congruence to hold also for $b=b_\beta$, we need $G, H$ to satisfy an equation
$$S=Q_0H+R_0G,$$
where $S\in k[X]$ has degree $<0$.  But we can find such $G, H$ since $Q_0, R_0$ are relatively prime.  Then, take $Q_\beta=G$ and $R_\beta= G$ for such $G, H$.  

\end{proof}
\include{week_8/week_8}
\include{week_9/week_9}
\include{week_10/week_10}
%\include{week_6/week_6}
== Siddharth's extra lectures ==

===Part 1===
''notes by Bill Chen''

A crucial concept for these lectures will be ''games''. These games where two players Left and Right alternate moves, and a player loses if she has no moves. The information of who goes first is not encoded into the game. Formally, a game $G$ consists of two sets of games, $G=\{G_L|G_R\}$, where the left side consists of the valid games which Left can move to, and similarly for the right side. (We use the "typical element" notation for sets, which carries over from the notation for surreal numbers.)

====Example====
* $\{\emptyset|\emptyset\}$ is called the zero game (abbreviated 0). There are no legal moves for either player, and the first player to move loses.
* $\{\emptyset|0\}$ is the game 1. In this game, Left has no legal moves, and Right can move to the 0 game, so Right has a winning strategy no matter who moves first.
* $\{0|\emptyset\}$. Here Left has a winning strategy.
* $\{0|0\}$ First player to move wins. This is a valid game which is not a surreal number as we defined in Week 2.

====Definition 1====
* $G>0$ if Left has a winning strategy.
* $G<0$ if Right has a winning strategy.
* $G\sim 0$ if the second player has a winning strategy. ($G$ is ''similar'' to $0$.)
* $G\parallel 0$ if the first player has a winning strategy. ($G$ is ''fuzzy''.)
* $G\ge 0$ means $G>0$ or $G\sim 0$.

====Lemma 2 (Determinacy)====
For any game $G$, one of $G>0$, $G<0$, $G\sim 0$, or $G\parallel 0$ holds.

'''Proof:'''

Let $A$ be the assertion that there is a $G_L$ with $G_L\ge 0$, and $B$ be the assertion that there is a $G_R$ with $G_R\le 0$.

Then one can check that $G>0$ iff $A\& \neg B$, $G<0$ iff $\neg A \& B$, $G\sim 0$ iff $\neg A \& \neg B$, and $G\parallel 0$ iff $A\& B$. For example, if $A \& B$ holds, then the first player can move to a game that is positive or similar $0$. In the first case, the first player clearly wins. In the second case, the first player becomes the second player of the new game similar to $0$, and hence wins.

====Definition 3====
If $G,H$ are games, the ''disjunctive sum'' $G+H$ is the game in which $G$ and $H$ are "played in parallel." Formally,
$$G+H=\{G_L+H,G+H_L|G_R+H, G+H_R\}.$$

'''Remark:'''
By induction, can prove that $+$ is associative and commutative. 

====Definition 4====
If $G$ is a game, the ''negation'' $-G$ is the game obtained by switching the roles of Left and Right. Formally,
$$-G=\{-G_R|-G_L\}.$$

Notice that these are the same definitions as for surreal numbers.

====Lemma 5 (Basic properties of $+$ and $-$)====
* $-(G+H)=-G+-H$.
* $--G=G$.
* $G\sim 0$ iff $-G\sim 0$.
* $G>0$ iff $-G<0$.
* $G\parallel 0$ iff $-G\parallel 0$.

We won't prove this lemma, but it is not difficult.

====Lemma 6====
Let $H\sim 0$. Then:
* If $G\sim 0$, then $G+H\sim 0$.
* If $G>0$, then $G+H>0$.
* If $G\parallel 0$, then $G+H\parallel 0$.
* If $G+H\sim 0$, then $G\sim 0$.
* If $G+H>0$, then $G>0$.
* If $G+H\parallel 0$, then $G\parallel 0$.

'''Proof:'''

Formally, this is proved by induction. We just describe the strategies in words.

For (1), if the second player has a winning strategy in $G$ and $H$, then the second player can use the winning strategy corresponding to the game in which the first player plays in.

For (2), the proof splits into cases. If Right moves first, either Right moves in $H$, so Left can play according to the second player's strategy in the $H$ game, or Right moves in $G$, so Left can play according to his strategy in $G$. If Left moves first, he plays according to his strategy in $G$ and then according to the previous sentence against the subsequent moves of Right.

(3), is a similar analysis.

The next three follow from the first three by using cases based on determinacy.

====Lemma 7====
* $G+ -G\sim 0$.
* If $G>0$ and $H>0$ then $G+H>0$.

'''Proof:'''

The first assertion follows from the strategy of "playing Go on two boards against the same person."


====Definition 8====
* $G\sim H$ if $G-H\sim 0$.
* $G>H$ if $G-H>0$.


\end{document}