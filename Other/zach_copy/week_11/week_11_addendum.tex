\begin{definition}
\[ G \sim H \iff G - H = 0.\]
\[ G > H \sim G - H > 0.\]
\end{definition}

\begin{lemma}
${\sim}$ is an equivalence relation.
\end{lemma}
\begin{proof}
\begin{enumerate}
\item \[ G-H\sim 0 \iff -(G-H)\sim0 \iff -G+H\sim 0 \iff H-G\sim 0,\]
so $G\sim H \iff H\sim G$.
\item $G\sim G$ because $G-G\sim 0$.
\item If $G\sim G'$ and $G'\sim H$ then $G\sim H$:
\[ G\sim G'\sim 0 \text{ and } G'-H\sim 0 \Rightarrow G-G'+G'-H\sim 0 \Rightarrow G-H\sim 0.\]
\end{enumerate}
\end{proof}

\begin{lemma}
${<}$ is an order (i.e. a transitive antisymmetric relation).
\end{lemma}
\begin{proof}
Similar to previous lemma.
\end{proof}

We want to show that the class of games modulo ${\sim}$ forms a partially ordered abelian group with identity $0$ (the ${\sim}$-class of $\{\emptyset | \emptyset\}$).
That is, we want to show that if $G\sim G'$ and $H\sim H'$ then
\begin{itemize}
\item $G+H \sim G'+H'$
\item $-G\sim -G'$
\item $G>H \iff G'>H'$.
\end{itemize}

\begin{theorem}
$\No$ forms an ordered abelian subgroup of the partially ordered abelian group $G$ of games.
\end{theorem}

\begin{proof}
We define a map $\pi\colon\No\to G$ and prove that it's an injective group homomorphism:
\begin{align*}
\pi \colon& 0\mapsto 0 \\
\pi\colon& \{L | R\} \mapsto \{\pi(L) | \pi(R) \}
\end{align*}
We leave it as an exercise to verify that $\pi$ is well-defined.

It's clear from the definitions of $+$ for surreal numbers and games that $\pi$ is morphism of abelian groups.

We claim that $a=0 \iff \pi(a)=0$ and $a<b\iff \pi(a)<\pi(b)$. The proof is by induction on $a$.

Suppose that $\pi(a) = 0$, so $\{ \pi(a_L) | \pi(a_R) \} = 0$. So $\pi(a_L) < 0 < \pi(a_R)$ for all $a_L$ and $a_R$, and by
induction $a_L < 0 < a_R$. So $a = \{ a_L | a_R \} = 0$.

Suppose $\pi(a)<\pi(b)$, i.e., $\pi(b)-\pi(a)>0$. Recall
\[ \pi(a)-\pi(b) = \{ \pi(b)_L - \pi(a), \, \pi(b) - \pi(a)_L \,|\, \pi(b)_R - \pi(a),\, \pi(b) - \pi(a)_R \}, \]
using $\pi(b)_L = \pi(b_L)$.
\[ \pi(a) - \pi(b) = \{ \pi(b_L-a),\pi(b-a_R) \,|\, \pi(b_R-a),\pi(b-a_L)\}>0.\]
For all $b_R,a_L$, $b_R-a>0$ and $b-a_L>0$. Since $a_L<b$ and $a<b_R$, we conclude that $a<b$.
\end{proof}