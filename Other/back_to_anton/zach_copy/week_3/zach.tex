%\documentclass{article}
%\usepackage{amsmath}
%\usepackage{amssymb}
%\usepackage{amsthm}
%
%\newcommand{\N}{\mathbb{N}}
%\newcommand{\Z}{\mathbb{Z}}
%\newcommand{\Q}{\mathbb{Q}}
%\newcommand{\R}{\mathbb{R}}
%\newcommand{\concat}{\mathrel{\hat{\ }}}
%
%\theoremstyle{definition}
%\newtheorem{cor}{Corollary}
%\newtheorem{example}{Example}
%\newtheorem{proposition}{Proposition}
%\newtheorem{lemma}{Lemma}
%\newtheorem{defn}{Definition}
%\newtheorem{theorem}{Theorem}
%
%\begin{document}

Notes by Zach.

\section*{October 20th}

\begin{lemma}
For $n\in\N$, $n\cdot 1 = (\underbrace{++\cdots+}_{n\text{ times}})$.
\label{3.1}
\end{lemma}

\begin{proof} (of 3.1)
By induction on $n$. The cases $n=0,1$ are obvious.
Suppose that $n\ge 1$. We have
\begin{align*}
(n+1)\cdot 1 &= n\cdot 1 + 1 \\
&= (++\cdots +)+(+) \\
&= \{ (\underbrace{+\cdots+}_{n-1}) \,|\, \emptyset \} + \{ 0 \,|\, \emptyset \} \\
&= \{ (n-1)\cdot1 \,|\, \emptyset \} + \{ 0 \,|\, \emptyset \} \\
&= \{(n-1)\cdot 1 + 1,\, n\cdot 1 + 0 \,|\, \emptyset \} \\
&= \{ n\cdot 1 \,|\, \emptyset \} \\
&= \{ (\underbrace{++\cdots + }_{n}) \,|\, \emptyset \} \\
&= (\underbrace{++\cdots +}_{n+1}).
\end{align*}
\end{proof}

\begin{cor}
For $n\in\N$, $-n\cdot 1 = (\underbrace{- - \cdots -}_{n\text{ times}})$.
\end{cor}

An ordered field $k$ is {\em archimedean} if for all $a,b>0$ in $k$ there is $n\in\N$ such that $na>b$.

Note: $\mathbf{No}$ is not archimedean, since $\omega := (++\cdots)$ (with $\omega$ many $+$s) satisfies $\omega>(\underbrace{+\cdots + }_{n})=n\cdot 1$.

From now on we identify $\Z$ as a subring of $\mathbf{No}$.

Q: How to identify $\Q\subseteq \mathbf{No}$?

Idea: $0 = () < (+-) < (+) = 1$. Is $(+-)=\tfrac12$?

Finite sequences of $+$s and $-$s correspond to dyadic rationals, i.e., rationals of the form $\frac{a}{2^s}$ ($a\in\Z$, $s\in\N$). We might conjecture that $\Q$ corresponds to finite sequences in $\mathbf{No}$.

\begin{lemma}
Suppose $a+b = \{2a \,|\, 2b \}$. Then $\frac{a+b}2 = \{a \,|\, b \}$.
\end{lemma}

\begin{proof}
Put $c := \{a \,|\, b\}$. Then $2\cdot c = c+ c = \{a+c \,|\, b+c \}$, and we show that this equals $a+b$:
\[ a+c < a+b < b+c, \text{ since } a< c < b, \text{ and } \]
\[ 2a < a+c, \; 2b > b+c \text{ because } a<c < b. \]
Now the result follows by the uniformity of the definition of $+$: since we assumed $a+b = \{2a \,|\, 2b \}$, we get the claim by cofinality.
\end{proof}

\begin{example}
$1 = \{ 0 \,|\, \emptyset \} = \{ 0 \,|\, 2 \}$. Apply the lemma with $a=0$, $b=1$.
So $\tfrac12 = \{0 \,|\, 1 \} = (+-)$. Taking $a=0$, $b=\tfrac12$, we get $\tfrac14 = (+--)$.
\end{example}

\begin{cor} (3.4)
Suppose $a + b = \{ 2a \,|\, 2b \}$. Then 
\[ \frac{a+b}{2^{s+1}} = \left\{ \frac{a}{2^s} \,\bigg|\, \frac{b}{2^s} \right\}  \]
for all $s\in\N$.
\label{3.4}
\end{cor}

\begin{cor} (3.5)
For all $c\in\N$, $\frac{c}{2^s} + \frac{1}{2^{s+1}} = \left\{ \frac{c}{2^s} \,|\, \frac{c+1}{2^s} \right\}$.
\label{3.5}
\end{cor}

\begin{proof} (Proof of 3.5)
Take $a=c$, $b=c+1$. We have
\begin{align*}
a+b &= 2c+1 \\
&= \{ 2c \,|\, 2c+2 \} \\
&= \{ 2a \,|\, 2b \}.
\end{align*}
Apply Corollary \ref{3.4}.
\end{proof}

\begin{proposition} (3.6)
Surreal numbers of finite length correspond to dyadic rationals.
\label{3.6}
\end{proposition}

\begin{proof} (Proof of 3.6)
Let $d\in\mathbf{No}$ have length $m+n$, where $d(0) = d(1) = \cdots = d(m-1) \ne d(m)$. We'll show that 
$d\in\frac1{2^n}\Z$. Suppose $d(0) = d(m-1) = {+}$. (Similar if $d(0) = {-}$.) 
\begin{description}
\item $n=0$: follows by \eqref{3.1}.
\item $n=1$: Then $d = (\underbrace{++\cdots+}_{m\ge1}-)$. By \eqref{3.5} with $c = m-1$, $s=0$, $m-\frac12 = \frac{m-1}{2^0} + \frac{1}{2^1} = \{m-1 \,|\, m\}$. Clearly $\{m-1 \,|\, m\} = d$.
\end{description}
Now suppose we've shown the claim for all $n\ge r$, and let $n = r+1$; suppose $r\ge 1$. Let $d' = d\restriction (m+r)$.
Either $d = d'\concat (+)$ or $d=d'\concat (-)$; suppose wlog that $d = d'\concat (+)$. So $d' = (\underbrace{++\cdots+}_{m}-\cdots)$. Let $d = \{L \,|\, R\}$ be the canonical representation.
Taking $x=\max L$, $y = \min R$, we have $d = \{x \,|\, y\}$ by cofinality. Note $x = d'$. We don't know much about $y$ 
except that
\[ d = (\underbrace{++\cdots +}_{m}-\cdots -)\le (\underbrace{++\cdots+}_{m}) = m, \]
so $y\le m$. By inductive hypothesis, $x,y\in \frac1{2^r}\Z$. So $d' = x = \frac{c}{2^r}$ for some $c\in\N$. 
If we can show $y = \frac{c+1}{2^r}$, then by \eqref{3.5} 
\[ d = \{x \,|\, y \} = \frac{c}{2^r} + \frac{1}{2^{r+1}} \in \frac1{2^{r+1}}\Z, \]
as required.

Note 
\[m-1<x = d' = \frac{c}{2^r} < \frac{c+1}{2^r} \le y \le m. \]
Put $H := \{ h\in\mathbf{No} \,:\, \ell(h)\le m+ r \,\&\, h\restriction(m+1) = (++\cdots+-)\}$.
We have $|H| = 1+2+2^2+\cdots + 2^{r-1} = 2^r - 1$. By inductive hypothesis every $h\in H$ belongs to $\frac1{2^r}\N$,
and $m-1<h<m$. But there are exactly $2^r - 1$ many dyadic rationals satisfying both of these conditions. 
In particular, $\frac{c+1}{2^r}\in H$ or $\frac{c+1}{2^r}=m$. Either way, $\ell(\frac{c+1}{2^r})\le m+r$.
Since $\frac{c+1}{2^r} > d'$, this implies $\frac{c+1}{2^r} > d$. We have $\frac{c+1}{2^r} <_s d$:
otherwise $e := \frac{c+1}{2^r} \wedge d$ will satisfy $d < e < \frac{e+1}{2^r} \le y$, contradiction to choice of $y$.
Hence $\frac{c+1}{2^r}\in R$, so $\frac{c+1}{2^r}\ge y$.
\end{proof}

\section*{October 22nd}

\textbf{Remark} (to \eqref{3.6}). Suppose $d\in\mathbf{No}$ has length $m+n$, where
\begin{itemize}
\item $d(0) = \cdots = d(m-1)$, and
\item $d(m-1)\ne d(m)$.
\end{itemize}
Define
\[ b(i) := \begin{cases} \pm 1 & \text{if } i< m, \quad d(i)=\pm \\
\pm \frac{1}{2^{i-m+1}} & \text{if } i \ge m, \quad d(i) = \pm.
\end{cases} \]
Then $d = b(0) + b(1) + \cdots + b(m+n-1)$. (Exercise.) Also, every dyadic rational arises in this way. (Exercise.)
We now let $\mathbb{D}$ be the set of dyadic rationals $\Z[\frac12]\subseteq\Q$.

\begin{defn}
A surreal is called {\em real} if it is either of finite length or has length $\omega$ and is not ultimately constant.
\end{defn}

Recall: An ordered field $k$ is {\em dedekind-complete} if every nonempty subset of $k$ that is bounded from above
has a supremum. The ordered field $\R$ is up to (unique) isomorphism the only dedekind-complete ordered field.

\begin{theorem}[Conway] (3.8)
The real surreals form a dedekind-complete ordered subfield of $\mathbf{No}$. Let $a\in\mathbf{No}$, $\ell(a)=\omega$,
with canonical representation $a = \{L \,|\, R\}$ then
\[ a\text{ is real } \iff \begin{cases} L,R\neq\emptyset \\
L\text{ has no max} \\
R\text{ has no min}
\end{cases}.\]
\end{theorem}

The direction ``$\Rightarrow$'' is an exercise. Now we prove the other direction.

\begin{lemma} (3.9)
Let $L,R\subseteq\mathbb{D}$ be such that $L<R$ and $L$ has no max and $R$ has no min. Then $a = \{ L \,|\, R\}$ is real.
\end{lemma}

\begin{proof}
By \eqref{1.9} we have $\ell(a)\le\omega$. Suppose $\ell(a) = \omega$ and $a(n) = +$ eventually. Note that there is some $n_0$ with $a(n_0) = {-}$, since $a<R\ne\emptyset$. We may assume that $a(n) = {+}$ for all $n>n_0$. Let $b = a\restriction n_0$. Then $b>a$ and $b<_s a$. Since $(L,R)$ is cofinal in the canonical representation of $a$, there is $d\le b$, $d\in R$. It follows that $d\in R$, since $R$ has no least element. We can choose $m$ such that $d\le b - \frac1{2^m}$. (Possible since both $b$ \& $d$ are dyadics.) We get $a<d\le b - \frac1{2^m}$. Next let $n>n_0$, $c = a\restriction n$. Then
$c<a$ and $c <_s a$, so by choosing $n$ sufficiently large we can achieve $c>b-\frac1{2^m}$. But then $a>c>b-\frac{1}{2^m}$, oops.

(Why can we do this? Write
\[ c = k + \sum_{i= l}^{n-1} \pm\frac{1}{2^{i-l+1}}\]
and
\[ b = k + \sum_{i=l}^{n_0-1} \pm\frac{1}{2^{i-l+1}}. \]
Then we get
\[ c-b = -\frac{1}{2^{n_0-l+1}} + \frac{1}{2^{n_0-l+2}} + \cdots + \frac{1}{2^{n-l}} = -\frac{1}{2^{n_0-l+1}} + \left( \frac{1}{2^{n_0-l+1}} - \frac{1}{2^{n-l}} \right) = \frac{-1}{2^{n-l}}, \]
so just choose $n>l+m$.)
\end{proof}

\begin{lemma}
Let $a = \{L \,|\, R\}$. Suppose
\begin{enumerate}
\item $x\in L \Rightarrow \exists r\in \mathbb{D}^{>0}$, $y\in L$, $x+r\le y$. [``$L-\mathbb{D}^{>0}$ is cofinal in $L$'']
\item $x\in R \Rightarrow \exists r\in \mathbb{D}^{>0}$, $y\in R$, $y\le x-r$. [``$R+\mathbb{D}^{>0}$ is coinitial in $R$.'']

and also $L' < a < R'$ such that
\item $(\forall r\in \mathbb{D}^{>0})(\exists x'\in L')(\exists y'\in R') y'-x'\le r$. [``$R'-L'$ is coinitial in $\mathbb{D}^{>0}$'']
\end{enumerate}
Then $a = \{ L' \,|\, R' \}$. 
\end{lemma}

\begin{proof}
Check that $(L',R')$ is cofinal in $(L,R)$ and use the cofinality theorem.
\end{proof}

\begin{lemma}
$\mathbb{D}$ is dense in the ordered set of real surreals.
\end{lemma}

\begin{proof}
Let $a<b$ be reals. If neither $a <_s b$ nor $b <_s a$ then $a < a\wedge b < b$. (Note $a\wedge b$ is finite, so it's in $\mathbb{D}$.) So suppose that $a <_s b$. Then $a\in \mathbb{D}$. If also $b\in\mathbb{D}$, then we're done:
$a < \frac{a+b}2 < b$. So suppose $b\notin \mathbb{D}$, with canonical representation $b = \{ L \,|\, R\}$. Then $a\in L$,
and $L$ has no maximum, so we can find some dyadic element of $(a,b)$. Similar if $b <_s a$.
\end{proof}

\begin{lemma}
Let $a = \{ L \,|\, R \}$ be the canonical representation of a real $a\notin \mathbb{D}$. Then for all $r\in \mathbb{D}^{>0}$ there are $a_L, a_R$ with $a_R-a_L\le r$. 
\end{lemma}

\begin{proof}
For each $n$ there are $a_L,a_R$ with $a_L\restriction n = a_R\restriction n$, and $a_R-a_L$ is bounded from above by some expression $\frac1{2^s} + \frac{1}{2^{s+1}}+\cdots$, and this can be made as small as necessary. (Exercise.)
\end{proof}

\section*{October 24th}

\begin{proof}[Proof of Theorem \ref{3.8}]
Clearly $0,1$ are real. Let $a,b\in\mathbf{No}$ be real; we check that $a+b$, $a\cdot b$, $\frac1a$ (if $a\ne0$) are also real. (Note that $-a$ is obviously real.)

\subsubsection*{$a+b$} Suppose $a\in\mathbb{D}$, $b\notin\mathbb{D}$. Let $a = \{ L \,|\, R\}$ be the canonical representation. So $a + b = \{ a_L + b,a+b_L \,|\, a_R+b,a+b_R\}$. We claim that $a+b = \{a+b_L \,|\, a+b_R \}$. (By \eqref{3.9} this then gives $a+b$ real, since $b_L$, $b_R$ dyadic.)
We have $a,a_L\in\mathbb{D}$, so $a-a_L\in\mathbb{D}^{>0}$; hence by \eqref{3.12} there are $b_L,b_R$ such that $b_R-b_L\le a-a_L$. It follows that $a+b_L\ge a_L+b_R \ge a_L+b$. Now use cofinality. (Similar argument for the other side.)

Now suppose that $a,b\notin\mathbb{D}$. Then in the representation of $a+b$ the LHS has no max and the RHS has no min. So (1) \& (2) in \eqref{3.10} are satisfied for this cut. Let $L' := \{a_L+b_L\}$, $R' := \{a_R+b_R\}$. Then
$L' < a+b < R'$ and by \eqref{3.12} $(L',R')$ satisfies (3) in \eqref{3.10}. This means that $a+b = \{L' \,|\, R' \}$ by \eqref{3.10}, and this is real by \eqref{3.9}.

\subsubsection*{$a\cdot b$} Suppose $a\notin\mathbb{D}$, $a,b>0$. The typical element in the representation of 
$a\cdot b$ is $ab-(a-a_*)(b-b_*)$. Show that (1), (2) in \eqref{3.10} are satisfied by this cut. For example,
\[ x = ab - (a-a_L)(b-b_L).\]
Take $a_L'$ with $a_L < a_L' < a$ and set $x' = ab - (a-a_L')(b-b_L)$. Then by \eqref{3.11} $x'-x = (a_L' - a_L)(b-b_L)$ is greater than some element of $\mathbb{D}^{>0}$. This verifies (1) of \eqref{3.10}. In the same way verify (2). Now set
\begin{align*}
L' &:= \{ a'b' \,:\, a',b'\in\mathbb{D},\, 0\le a' < a, \, 0\le b' < b\}, \\
R' &:= \{ a''b'' \,:\, a'',b''\in\mathbb{D}, \, a'' > a, \, b'' > b \}.
\end{align*}
Then $L' < a\cdot b < R'$. We check that \eqref{3.10}(3) holds. Let $r\in \mathbb{D}^{>0}$ be given. Then the same 
argument as proving the limit law for multiplication in calculus gives elements $a',b',a'',b''$ such that $a''b''-a'b'<r$, using \eqref{3.12}. Hence by \eqref{3.10} $a\cdot b = \{ L' \,|\, R' \}$, so $ab$ is real by \eqref{3.9}.

\subsubsection*{$1/a$:} We may assume that $a>0$. Put
\begin{align*}
L &:= \{ d\in \mathbb{D} \,:\, d\concat a < 1 \} \\
R &:= \{ d\in \mathbb{D} \,:\, d\concat a > 1 \}.
\end{align*}
Then $L<R$, $0\in L$, and $R\ne\emptyset$ because: by \eqref{3.11} take $m$ such that $a>\frac1{2^m}$, so that $2^ma>1$. So $2^m\in R$.

\noindent{\bf Claim 1.}
$L$ has no max; $R$ has no min.

\begin{proof}[Proof of Claim 1.]
Let $d\in L$. Then $1-da>0$ is real. So there is some $m$ such that $1-da> \frac1{2^m}$. Also can take $n$ such that $a< 2^n$. Then $\frac{1}{2^{m+n}}a < \frac1{2^m} < 1-da$, so $a(\frac{1}{2^{m+n}} + d) < 1$.
\end{proof}

Hence by \eqref{3.9} $b:= \{L \,|\, R\}$ is real. We are going to show that $|ba-1|<r$ for every $r\in\mathbb{D}^{>0}$. Since $ba-1$ is real, we get $ba-1=0$ by \eqref{3.11}.

\noindent{\bf Claim 2.}
For each $n$ there is $c\in R$ such that $ca\le 1 + \frac1{2^n}$, and there is $c'\in L$ such that $c'a\ge 1  - \frac{1}{2^n}$. 

(Obviously this suffices.)

\begin{proof}[Proof of Claim 2.]
Choose $m$ such that $2^m>a$. Put $S := \{ r\in\N \,:\, (\frac{r}{2^{m+n}})a>1 \}$. Then $S\ne\emptyset$; take $s = \min S$. Then $\frac{s}{2^{m+n}}\in R$, but $(\frac{s-1}{2^{m+n}})a\le 1$. So
\[ \frac{s}{2^{m+n}} a \le 1 + \frac{a}{2^m2^n} \le 1 + \frac1{2^n}.\qedhere \]
\end{proof}
This completes the proof.
\end{proof}

%\end{document}