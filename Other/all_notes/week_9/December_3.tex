%% LyX 2.0.4 created this file.  For more info, see http://www.lyx.org/.
%% Do not edit unless you really know what you are doing.
\documentclass[12pt,oneside,english]{amsart}
\usepackage[T1]{fontenc}
\usepackage[latin9]{inputenc}
\usepackage{geometry}
\geometry{verbose,tmargin=3cm,bmargin=3cm,lmargin=2cm,rmargin=2cm}
\setlength{\parskip}{\medskipamount}
\setlength{\parindent}{0pt}
\usepackage{units}
\usepackage{amsthm}
\usepackage{amstext}
\usepackage{amssymb}
\usepackage{xargs}[2008/03/08]

\makeatletter
%%%%%%%%%%%%%%%%%%%%%%%%%%%%%% Textclass specific LaTeX commands.
\numberwithin{equation}{section}
\numberwithin{figure}{section}
 \theoremstyle{definition}
 \newtheorem*{defn*}{\protect\definitionname}
  \theoremstyle{definition}
  \newtheorem*{xca*}{\protect\exercisename}
  \theoremstyle{plain}
  \newtheorem*{lem*}{\protect\lemmaname}
  \theoremstyle{plain}
  \newtheorem*{cor*}{\protect\corollaryname}
  \theoremstyle{plain}
  \newtheorem*{thm*}{\protect\theoremname}

%%%%%%%%%%%%%%%%%%%%%%%%%%%%%% User specified LaTeX commands.
 \usepackage{mathptmx}	 
 \usepackage[scaled=0.92]{helvet}
 \usepackage{courier}

\makeatother

\usepackage{babel}
  \providecommand{\corollaryname}{Corollary}
  \providecommand{\definitionname}{Definition}
  \providecommand{\exercisename}{Exercise}
  \providecommand{\lemmaname}{Lemma}
  \providecommand{\theoremname}{Theorem}

\begin{document}
\global\long\def\N{\omega^{\omega}}
\global\long\def\Z{\mathbb{Z}}
\global\long\def\Q{\mathbb{Q}}
\global\long\def\R{\mathbb{R}}
\global\long\def\lto{\longrightarrow}
\global\long\def\es{\emptyset}
\global\long\def\F{\mathcal{F}}
\global\long\def\force{\Vdash}
\global\long\def\dom{\textrm{dom}}
\global\long\def\em{\prec}
\global\long\def\cf{\textrm{cf}}
\newcommandx\cof[1][usedefault, addprefix=\global, 1=]{\mathrm{cof}\left(#1\right)}
\global\long\def\model{\vDash}
\global\long\def\crit{\mathrm{crit}}
\global\long\def\ult{\mathrm{Ult}}
\global\long\def\inj{\hookrightarrow}
\global\long\def\u{\mathcal{U}}
\global\long\def\dprime{\prime\prime}
\global\long\def\C{\mathbb{C}}
\global\long\def\v{\mathcal{V}}
\global\long\def\w{\mathcal{W}}
\global\long\def\i{\imath}
\global\long\def\P{\mathbb{P}}
\newcommandx\norm[1][usedefault, addprefix=\global, 1=]{\left\Vert #1\right\Vert }

\begin{defn*}
(10.4) $\C\left\{ X\right\} =\bigcup_{r}\C\left\{ X\right\} _{r}$.\\
$\C\left\{ X\right\} $ is a subring of $\C\left[\left[X\right]\right]$
containing $\C\left[X\right]$, called the ring of convergent power
series in $X$.\end{defn*}
\begin{xca*}
(10.5) Let $\left\langle f_{j}\right\rangle _{j\in J}$ be a family
in $\C\left\{ X\right\} _{r}$ such that $\mathrm{ord}\left(f_{j}\right)\lto\infty$,
so $\sum_{j\in J}\in\C\left[\left[X\right]\right]$ exists. Then $\left\Vert \sum_{j}f_{j}\right\Vert _{r}\leq\sum_{j}\left\Vert f_{j}\right\Vert _{r}$.
Assume $\sum_{j}\left\Vert f_{j}\right\Vert <\infty$, then $\sum_{j}f_{j}\in\C\left\{ X\right\} _{r}$
and\end{xca*}
\begin{enumerate}
\item $\forall\epsilon>0\exists\textrm{finite }I_{\epsilon}\subset J$ s.t
$\forall\textrm{finite }I_{\epsilon}\subset I\subset J$, $\left\Vert \sum_{j\in J}f_{j}-\sum_{j\in I}f_{j}\right\Vert _{r}<\epsilon$.
\item $\forall x\in D_{r}\left(0\right)$, $\left(\sum_{j\in J}f_{j}\right)\left(x\right)=\sum_{j\in J}f_{j}\left(x\right)$.\end{enumerate}
\begin{lem*}
(10.6) (Abel) Let $f\in\C\left[\left[X\right]\right]$, $s\in\left(\R^{>0}\right)^{m}$,
$L\in\R^{>0}$, s.t $\left|f_{i}\right|s^{i}\leq L$ for all $i$.\\
Then $f\in\C\left\{ X\right\} _{r}$ for all $r<s$.\end{lem*}
\begin{proof}
\[
\sum_{i}\left|f_{i}\right|r^{i}=\sum_{i}\left|f_{i}\right|s^{i}\prod_{k<m}\left(\nicefrac{r_{k}}{s_{k}}\right)^{i_{k}}\leq L\sum_{i}\prod_{k<m}\left(\nicefrac{r_{k}}{s_{k}}\right)^{i_{k}}=L\prod_{k<m}\left(\underbrace{1+\frac{r_{k}}{s_{k}}+\left(\frac{r_{k}}{s_{k}}\right)^{2}+...}_{=1-\frac{r_{k}}{s_{k}}}\right)<\infty.
\]
 \end{proof}
\begin{cor*}
(10.7) $f\in\C\left\{ X\right\} $ if and only if $\exists s,L$ s.t
$\forall i\left(\left|f_{i}\right|s^{i}\leq L\right)$.\end{cor*}
\begin{lem*}
(10.8) Let $f\in\C\left\{ X\right\} $, f$\left(0\right)=0$ . Then
$\left\Vert f\right\Vert _{r}\overset{r\lto0}{\lto}0$.\end{lem*}
\begin{proof}
Take $s$ s.t $\left\Vert f\right\Vert _{s}<\infty$ (exists since
$f\in\C\left\{ X\right\} $). For $r<s$, 
\[
\left\Vert f\right\Vert _{r}=\sum_{i\neq0}\left|f_{i}\right|r^{i}=\underbrace{\sum\left|f_{i}\right|s^{i}}_{=\left\Vert f\right\Vert _{s}}\underbrace{\left(\frac{r}{s}\right)^{i}}_{\underset{r\lto0}{\lto}0}\overset{r\lto0}{\lto}0.
\]
\end{proof}
\begin{cor*}
(10.9) $\C\left\{ X\right\} ^{\times}=\left\{ f\in\C\left\{ X\right\} ;\, f\left(0\right)\neq0\right\} $.\end{cor*}
\begin{proof}
$"\subset"$ is clear. For $"\supset"$, let $f\in\C\left\{ X\right\} $
with $f\left(0\right)\neq0$. Write $f=f\left(0\right)\left(1-g\right)$
where $g\in\C\left\{ X\right\} $, $g\left(0\right)=0$. Then $h=1+g+g^{2}+...$.
We have already seen that $h\in\C\left[\left[X\right]\right]$ and
$h=\left(1-g\right)^{-1}$. So we only need to show that $h\in\C\left\{ X\right\} $.
By (10.8), for small enough $r$, $\norm[g]_{r}<1$. So by (10.5),
$\norm[h]_{r}\leq\sum_{k\neq0}^{\infty}\norm[g]_{r}^{k}<\infty$,
hence $h\in\C\left\{ X\right\} _{r}$.\end{proof}
\begin{lem*}
(10.10) Let $g_{1},...,g_{n}\in\C\left\{ Y\right\} $ have constant
term zero. Then for each $f\in\C\left\{ X\right\} $ we have $f\left(g_{1},...,g_{m}\right)\in\C\left\{ Y\right\} $.
Moreover, for $r,s$ small enough, $f\in\C\left\{ X\right\} _{r}$,
$g_{j}\in\C\left\{ Y\right\} _{s}$, for any $y\in\bar{D}_{s}\left(0\right)$,
$g_{j}\left(y\right)\in\bar{D}_{r}\left(0\right)$ and
\[
f\left(g_{1},...,g_{m}\right)\left(y\right)=f\left(g_{1}\left(y\right),...,g_{m}\left(y\right)\right).
\]
\end{lem*}
\begin{proof}
Exercise.\end{proof}
\begin{thm*}
(10.11) Let $f,g\in\C\left\{ X,T\right\} $ with $f$ regular of order
$d$. Then $g=Qf+R$ with $Q\in\C\left\{ X,T\right\} $ and $R\in\C\left\{ X\right\} \left[T\right]$
of degree $<d$.\end{thm*}
\begin{proof}
We repeat the Weierstrass division theorem, making sure that everything
converges.\\
Recall, $f=\sum_{i\geq0}f_{i}T^{i}$ where $f_{0}\left(0\right)=...=f_{d-1}\left(0\right)=0$
and $f_{d}\left(0\right)\neq0$. We defined $u=\sum_{i\geq d}f_{i}T^{i}$,
$F=u^{-1}\sum_{i<d}f_{i}T^{i}$ (note that by (10.9) $u\in\C\left\{ X,T\right\} $).
So $u^{-1}f=T^{d}+F$, and we work with $T^{d}+F$ instead. i.e. assume
$f=T^{d}+F$.\\
Take $r'=\left(r_{1},...,r_{m}\right)$ and $r_{m+1}\in\R^{>0}$ small
enough such that for $r=\left(r_{1},...,r_{m+1}\right)$, $\norm[g]_{r}$,
$\norm[u^{-1}]_{r}$, \\
$\norm[f_{0}]_{r'}$,...,$\norm[f_{d-1}]_{r'}<\infty$. Thus 
\[
\norm[F]_{r}\leq\norm[u^{-1}]_{r}\sum_{i<d}\norm[f_{i}]_{r'}r_{m+1}^{i}.
\]
By (10.8), for each $i$, $\norm[f_{i}]_{r'}\lto0$ as $r'\lto0$.
Also, $\norm[u^{-1}]_{r}$ can only decrease by decreasing $r'$.
Thus by decreasing $r'$ we can assume $\norm[F]_{r}\leq r_{m+1}^{d}$.
Take $\epsilon=\frac{\norm[F]_{r}}{r_{m+1}^{d}}\in\left[0,1\right)$.
\\
Recall that for 
\[
\tau\left(g\right)=\sum_{i\geq d}g_{i}T^{i-d},\quad\alpha\left(g\right)=\sum_{i<d}g_{i}T^{i},\quad\lambda\left(g\right)=-\tau\left(g\right)F,
\]
we have $g=\tau\left(g\right)f+\alpha\left(g\right)+\lambda\left(g\right)$.
\begin{eqnarray*}
\norm[\tau\left(g\right)]_{r} & \leq & \sum_{i\geq d}\norm[g_{i}]_{r}r_{m+1}^{i-d}<r_{m+1}^{-d}\norm[g]_{r};\\
\norm[\alpha\left(g\right)]_{r} & \leq & \norm[g]_{r};\\
\norm[\lambda\left(g\right)]_{r} & \leq & \norm[\tau\left(g\right)]_{r}\norm[F]_{r}\leq r_{m+1}^{-d}\norm[g]_{r}\norm[F]_{r}\leq\epsilon\norm[g]_{r}.
\end{eqnarray*}
Inductively define $\lambda^{n}\left(g\right)$ according to 
\[
\left(\ast\right)\quad\lambda^{n}\left(g\right)=\tau\left(\lambda^{n}\left(g\right)\right)f+\alpha\left(\lambda^{n}\left(g\right)\right)+\lambda^{n+1}\left(g\right).
\]
Inductively, $\norm[\lambda^{n}\left(g\right)]_{r}\leq\epsilon^{n}\norm[g]_{r}$,
hence $\norm[\tau\left(\lambda^{n}\left(g\right)\right)]\leq\norm[g]\epsilon^{n}r_{m+1}^{-d}$,
$\norm[\alpha\left(\lambda^{n}\left(g\right)\right)]_{r}\leq\norm[g]\epsilon^{n}$.\\
Summing $\left(\ast\right)$ for $n\geq0$, we get $f=Qf+R$, for
$Q=\sum_{n}\tau\left(\lambda^{n}\left(g\right)\right)$, $R=\sum_{n}\alpha\left(\lambda^{n}\left(g\right)\right)$.
Furthermore $\norm[Q]_{r}\leq\sum_{n}\norm[\tau\left(\lambda^{n}\left(g\right)\right)]_{r}\leq\norm[g]_{r}r_{m+1}^{-d}\sum_{n}\epsilon^{n}<\infty$,
$\norm[R]\leq\sum_{n}\norm[\alpha\left(\lambda^{n}\left(g\right)\right)]\leq\norm[g]\sum\epsilon^{n}<\infty$.
\\
Thus $Q,R\in\C\left\{ X,T\right\} $. Note that by definition $R\in\C\left[\left[X\right]\right]\left[T\right]$
is of degree $d$, hence $R\in\C\left\{ X\right\} \left[T\right]$.\end{proof}
\begin{cor*}
(10.12) (Weierstrass preparation)

Let $f\in\C\left\{ X,T\right\} $ be regular in $T$. Then $f=uW$
where $u\in\C\left\{ X,T\right\} ^{\times}$ and $W\in\C\left\{ X\right\} \left[T\right]$
of degree $d$.\end{cor*}
\begin{proof}
Same as (9.4), using (10.11) instead of (9.3), so that everything
converges.\end{proof}

\end{document}
