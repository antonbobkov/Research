\documentclass[english]{article}

\usepackage{amsmath}
\usepackage{amssymb}
\usepackage{amsthm}
\usepackage{enumerate}
\usepackage{colonequals}
\usepackage{fullpage}
\usepackage{tikz-qtree}

\usepackage{dsfont}

%\usepackage{hyperref}
%\hypersetup{colorlinks,citecolor=blue,filecolor=blue,linkcolor=blue,urlcolor=blue}

\newcommand{\R}{\mathbb{R}}
%\newcommand{\concat}{\mathbin{\raisebox{1ex}{\scalebox{.7}{$\frown$}}}} %sequence concatenation
\newcommand{\concat}{\frown} %sequence concatenation
\newcommand{\dom}[1]{\operatorname{dom}\paren{#1}}
\newcommand{\ZFC}{\mathsf{ZFC}}
\newcommand{\NBG}{\mathsf{NBG}}
\newcommand{\coloneq}{\colonequals}
\newcommand{\N}{\mathbb{N}}
\newcommand{\Z}{\mathbb{Z}}
\newcommand{\Q}{\mathbb{Q}}

\newcommand{\No}{\mathbf{No}}
\newcommand{\On}{\mathbf{On}}
\newcommand{\paren}[1]{\left( #1 \right)}
\newcommand{\brac}[1]{\left[ #1 \right]}
\newcommand{\curly}[1]{\left\{ #1 \right\}}
\newcommand{\abs}[1]{\left| #1 \right|}
\newcommand{\rar}{\rightarrow}
\newcommand{\arr}{\rightarrow}

\DeclareMathOperator{\supp}{supp}
\DeclareMathOperator{\lt}{lt}

\newcommand{\w}{\omega}
\newcommand{\midr}[1]{\restriction_{#1}}

% This is the "centered" symbol
\def\fCenter{{\mbox{\Large{$\rightarrow$}}}}

\newcommand{\bigslant}[2]{{\raisebox{.2em}{$#1$}\left/\raisebox{-.2em}{$#2$}\right.}}
\def\dotminus{\mathbin{\ooalign{\hss\raise1ex\hbox{.}\hss\cr\mathsurround=0pt$-$}}}

% \swapnumbers
\theoremstyle{theorem}            %bold title, italicized font
\newtheorem{theorem}{Theorem}[section]

\newtheorem{claim}[theorem]{Claim}

\newtheorem{lem}[theorem]{Lemma}
\newtheorem{lemma}[theorem]{Lemma}

\theoremstyle{definition}           %bold title, regular font
\newtheorem{example}[theorem]{Example}

\newtheorem{defn}[theorem]{Definition}
\newtheorem{definition}[theorem]{Definition}

\newtheorem{proposition}[theorem]{Proposition}

\newtheorem{corollary}[theorem]{Corollary}
\newtheorem{cor}[theorem]{Corollary}

\newtheorem{exercise}[theorem]{Exercise}
\newtheorem*{problem}{Problem}
\newtheorem{warning}[theorem]{Warning}
\newtheorem*{solution}{Solution}
\newtheorem*{remark}{Remark}

\theoremstyle{empty}
\newtheorem{namedtheorem}{}

\newcommand{\customtheorem}[3]{\theoremstyle{theorem} \newtheorem{theorem#1}[theorem]{#1} \begin{theorem#1}[#2]#3 \end{theorem#1}}

\newcommand{\customdefinition}[2]{\theoremstyle{definition} \newtheorem{definition#1}[theorem]{#1} \begin{definition#1}#2 \end{definition#1}}


\renewcommand{\restriction}{\mathord{\upharpoonright}}
 
%\newcommand{\WikiLevelTwo}[1]{\section*{#1}}
%\newcommand{\WikiLevelTwo}[1]{\begin{center} \begin{Large} #1 \end{Large} \end{center}}
%\newcommand{\WikiLevelThree}[1]{\begin{center} \textbf{#1} \end{center}}
%\newcommand{\WikiLevelFour}[1]{\begin{center} \textit{#1} \end{center}}

\newcommand{\Week}[1]{\begin{center} \begin{Large} #1 \end{Large} \end{center}}
\newcommand{\Day}[1]{\begin{center} \textbf{#1} \end{center}}
\newcommand{\NotesBy}[1]{\begin{flushright} \textit{#1} \end{flushright}}

\newcommand{\WikiSigleStar}{}

\newcommand{\WikiItalic}[1]{\textbf{#1}}
\newcommand{\WikiBold}[1]{\textbf{#1}}
\newcommand{\WikiBoldItalic}[1]{\emph{#1}}

\usepackage[T1]{fontenc}
\usepackage[latin9]{inputenc}
\usepackage{geometry}
\geometry{verbose,tmargin=3cm,bmargin=3cm,lmargin=2cm,rmargin=2cm}
\setlength{\parskip}{\medskipamount}
\setlength{\parindent}{0pt}
\usepackage{units}
\usepackage{xargs}[2008/03/08]

\makeatletter
%%%%%%%%%%%%%%%%%%%%%%%%%%%%%% Textclass specific LaTeX commands.
\numberwithin{equation}{section}
\numberwithin{figure}{section}
  \theoremstyle{plain}
  \newtheorem*{prop*}{\protect\propositionname}
  \theoremstyle{plain}
  \newtheorem*{cor*}{\protect\corollaryname}
 \theoremstyle{definition}
 \newtheorem*{defn*}{\protect\definitionname}
  \theoremstyle{remark}
  \newtheorem*{claim*}{\protect\claimname}
  \theoremstyle{plain}
  \newtheorem*{thm*}{\protect\theoremname}
  \theoremstyle{plain}
  \newtheorem*{lem*}{\protect\lemmaname}

\makeatother

\usepackage{babel}
  \providecommand{\corollaryname}{Corollary}
  \providecommand{\definitionname}{Definition}
  \providecommand{\exercisename}{Exercise}
  \providecommand{\lemmaname}{Lemma}
  \providecommand{\theoremname}{Theorem}

\includeonly{week_10/December_8_12,week_9/December_3}

\title{Notes on Surreal Numbers \\ Math 285: Fall 2014}
\author{Class Taught by Prof. Aschenbrenner}
\begin{document} 
\maketitle{}

\tableofcontents

\textit{Notes by John Suice}

\section*{Day 1: Friday October 3, 2014}
Surreal numbers were discovered by John Conway. 
The class of all surreal numbers is denoted $\No$ and 
this class comes equipped with a natural linear ordering and 
arithmetic operations making $\No$ a real closed ordered field. 

For example, $1/ \omega, \omega - \pi, \sqrt{\omega} \in \No$, 
where $\omega$ denotes the first infinite ordinal. 

\begin{theorem}[Kruskal, 1980s]
	There is an exponential function $\exp \colon \No \rar \No$
	exteding the usual exponential $x \mapsto e^x$ on $\R$. 
	\label{}
\end{theorem}

\begin{theorem}[van den Dries-Ehrlich, c. 2000]
	$(\R, 0, 1, +, \cdot, \leq, e^x) \preccurlyeq 
	(\No, 0, 1, +, \cdot, \leq, \exp)$. 	
	\label{}
\end{theorem}

\subsection*{Basic Definitions and Existence Theorem}
Throughout this class, we will work in von Neumann-Bernays-G\"odel 
set theory with global choice ($\NBG$). This is conservative over 
$\ZFC$ (see Ehrlich, \emph{Absolutely Saturated Models}). 

An example of a surreal number is the following: 
\begin{align*}
	f \colon \curly{0, 1, 2} &\longrightarrow \curly{+, -} \\
	0 &\longmapsto + \\
	1 &\longmapsto - \\
	2 &\longmapsto +
\end{align*}
This may be depicted in tree form as follows:
%------------------------Beautiful Tree Diagram-------------------------------------
%------------------------DO NOT ALTER IN ANY WAY------------------------------------
%----------------------Violators WILL be prosecuted---------------------------------
%----The above is not meant to exclude the possibility of extrajudical punishment--- 
%---------------------------------------------------------------------
We will denote such a surreal number by $f=(+-+)$
Another example is: 
\begin{align*}
	f \colon \omega + \omega &\longrightarrow \curly{+, -} \\
	n &\longmapsto + \\
	\omega + n &\longmapsto -
\end{align*}
We write $\No$ for the class of surreal numbers. We often view 
$f \in \No$ as a function $f \colon \On \rar \curly{+, -, 0}$ by 
setting $f(\alpha) = 0$ for $\alpha \notin \dom{f}$. 

\begin{defn}
	Let $a, b \in \No$. 
	\begin{enumerate}
		\item We say that $a$ is an \emph{initial segment} of 
			$b$ if $l(a) \leq l(b)$ and $b \restriction 
			\dom{a} = a$. We denote this by $a \leq_s b$
			(read: ``$a$ is simpler than $b$''). 
		\item We say that $a$ is a \emph{proper initial segment}
			of $b$ if $a \leq_s b$ and $a \neq b$. We denote 
			this by $a <_s b$. 
		\item If $a \leq_s b$, then the \emph{tail} of $a$ in 
			$b$ is the surreal number $c$ of length 
			$l(b) - l(a)$ satisfying $c(\alpha) = 
			a(l(b) + \alpha)$ for all $\alpha$. 
		\item We define $a \concat b$ to be the surreal number 
			satisfying: 
			\begin{align*}
				(a \concat b)(\alpha) = 
				\begin{cases}
					a(\alpha) & \alpha < l(a) \\
					b(\alpha - l(a)) & \alpha \geq l(a)
				\end{cases}
			\end{align*}
			(so in particular if $a \leq_s b$ and $c$ is the tail 
			of $a$ in $b$, then $b = a \concat c$). 
		\item Suppose $a \neq b$. Then the \emph{common initial 
			segment} of $a$ and $b$ is the element 
			$c \in \No$ with minimal length such that 
			$a(l(c)) \neq b(l(c))$ and $c \restriction l(c) 
			= a \restriction 
			l(c) = b \restriction l(c)$. We write 
			$c = a \wedge b$, and also set $a \wedge a = a$. 
	\end{enumerate}
\end{defn}
Note that 
\begin{align*}
	a \leq_s b \iff a \wedge b = a
\end{align*}

\section*{Day 2: Monday October 6, 2014}
\begin{defn}
	We order $\left\{ +, -, 0 \right\}$ by setting
	$- < 0 < +$ and for $a, b \in \No$ we define
	\begin{align*}
		a < b &\iff a < b \text{ lexicographically} \\
		&\iff a \neq b \land a(\alpha_0) < b(\alpha_0) 
		\text{ where } \alpha_0 = l(a \wedge b)
	\end{align*}
	As usual we also set $a \leq b \iff a < b \lor a = b$. 
\end{defn}
Clearly $\leq$ is a linear ordering on $\No$. 

Examples: 
\begin{align*}
	(+-+) < (+++ \cdots --- \cdots) \\
	(-+) < () < (+-) < (+) < (++)
\end{align*}
Remark: if $a \leq_s b$ then $a \wedge b = a$ and if 
$b \leq_s a$ then $a \wedge b = b$. Suppose that neither 
$a \leq_s b$ or $b \leq_s a$. Put: 
\begin{align*}
	\alpha_0 = \min{ \curly{\alpha \colon a(\alpha) \neq b(\alpha)}}
\end{align*}
Then either $a(\alpha_0) = +$ and $b(\alpha_0) = -$, in which 
case $b < (a \wedge b) < a$, or $a(\alpha_0) = -$ and $b(\alpha_0) = +$, 
in which case $a < (a \wedge b) < b$. In either case: 
\begin{align*}
	a \wedge b \in \brac{ \min{ \curly{a, b}, \max{ \curly{a, b}}}}
\end{align*}

\begin{defn}
	Let $L, R$ be subsets (or subclasses) of $\No$. We say 
	$L < R$ if $l < r$ for all $l \in L$ and $r \in R$. We define 
	$A < c$ for $A \cup \curly{c} \subseteq \No$ in the obvious manner. 
\end{defn}
Note that $\emptyset < A$ and $A < \emptyset$ for all $A \subseteq \No$ by 
vacuous satisfaction. 

\begin{theorem}[Existence Theorem]
	Let $L, R$ be sub\emph{sets} of $\No$ with $L < R$. 
	Then there exists a unique $c \in \No$ of minimal length 
	such that $L < c < R$. 
	\label{}
\end{theorem}
\begin{proof}
%--------------Redundant Section (Covered at beginning of next day)------------------
%	First assume that there exists $c \in \No$ with $L < c < R$. 
%	By minimizing over the lengths of all such $c$ (using the fact that 
%	the ordinals are well-ordered), we may assume without loss of 
%	generality that $c$ has minimal length. But then it is immediate 
%	that $c$ is unique; for if $\tilde{c} \neq c$ satisfied 
%	$L < \tilde{c} < R$ and $l(c) = l(\tilde{c})$, then by 
%	the note at the beginning of this section we would have: 
%	\begin{align*}
%		L < \min{ \curly{c, \tilde{c}}}
%		< (c \land \tilde{c}) < \max{ \curly{c, 
%			\tilde{c}}} < R	
%	\end{align*}
%	contradicting minimality of $l(c)$. 
%
%	Now for existence: let 
%------------------------------------------------------------------------------------
	We first prove existence. Let 
	\begin{align*}
		\gamma = \sup_{a \in L \cup R}{(l(a) + 1)}
	\end{align*}
	be the least strict upper bound of lengths of elements of 
	$L \cup R$ (it is here that we use that $L$ and $R$ are sets 
	rather than proper classes). For each ordinal $\alpha$, 
	denote by $L\restriction \alpha$ the set  $\curly{l \restriction \alpha 
	\colon l \in L}$, and similarly for $R$. Note that 
	$L \restriction \gamma = L$ and $R \restriction \gamma = R$. 	
	We construct $c$ of length $\gamma$ by defining the 
	values $c(\alpha)$ by induction on 
	$\alpha \leq \gamma$ as follows: 
	\begin{align*}
		c(\alpha) = 
		\begin{cases}
			- & \text{ if } 
			(c \restriction \alpha \concat (-) ) \geq 
			L \restriction (\alpha + 1) \\
			+ & \text{ otherwise}
		\end{cases}
	\end{align*}
	\begin{claim}
		$c \geq L$
	\end{claim}
	\begin{proof}[Proof of Claim]
		Otherwise there is $l \in L$ such that 
		$c < l$. This means $c(\alpha_0) < l(\alpha_0)$
		where $\alpha_0 = l(c \wedge l)$. Since 
		$c(\alpha_0)$ must be in $\left\{ -, + \right\}$ (i.e. 
		is nonzero) this implies $c(\alpha_0) = -$ even though 
		$(c \restriction \alpha_0 \concat (-)) \not \geq 
		l \restriction (\alpha_0 + 1)$, a contradiction. 
	\end{proof}
	\begin{claim}
		$c \leq R$	
	\end{claim}
	\begin{proof}[Proof of Claim]
		Otherwise there exists $r \in R$ such that 
		$r < c$. This means $r(\alpha_0) < c(\alpha_0)$ 
		where $\alpha_0 = l(r \land c)$. 
		%We may assume 
		%that $\alpha_0$ is least possible, i.e. that 
		%$c \restriction \alpha_0 \leq r' \restriction \alpha_0$
		%for all $r' \in R$. 
		Since $c(\alpha_0) > r(\alpha_0)$, 
		we must be in the ``$c(\alpha_0) = +$'' case, and so 
		there is some $l \in L$ such that 
		$l \restriction (\alpha_0 + 1) > (c \restriction \alpha_0) 
		\concat (-) = (r \restriction \alpha_0) \concat (-)$. 
		In particular $l(\alpha_0) \in \curly{0, +}$. 
		So if $r(\alpha_0) = -$ then $r < l$, and if 
		$r(\alpha_0) = 0$ then $r \leq l$, in either 
		case contradicting $L < R$. 
	\end{proof}
	At this point we have shown $L \leq c \leq R$. 
	But by construction $c$ has length $\gamma$, and so 
	in particular cannot be an element of $L \cup R$. 
	Thus 
	\begin{align*}
		L < c < R
	\end{align*}
	as desired. 
\end{proof}

\section*{Day 3: Wednesday October 8, 2014}
Last time we showed that there is $c \in \No$ with $L < c < R$. 
The well-ordering principle on $\On$ gives us such a $c$ of minimal 
length. Let now $d \in \No$ satisfy $L < d < R$. Then 
$L < (c \wedge d) < R$. By minimality of $l(c)$ and since 
$(c \wedge d) \leq_s c$ we have $l(c \wedge d) = l(c)$. 
Therefore $(c \wedge d) = c$, or in other words $c \leq_s d$. 

Notation: $\left\{ L \vert R \right\}$ denotes the $c \in \No$
of minimal length with $L < c < R$. Some remarks: 
\begin{enumerate}[(1)]
	\item $\left\{ L \vert \emptyset \right\}$ consists only of 
		$+$'s. 
	\item $\left\{ \emptyset \vert R \right\}$ consists only of 
		$-$'s. 
\end{enumerate}
\begin{lem}
	If $L < R$ are subsets of $\No$, then 
	\begin{align*}
		l( \curly{L \vert R}) \leq 
		\min{ \curly{\alpha \colon l(b) < \alpha \text{ for all 
		$b \in L \cup R$} }}
	\end{align*}
	Conversely, every $a \in \No$ is of the form 
	$a = \curly{L \vert R}$ where $L < R$ are subsets of 
	$\No$ such that $l(b) < l(a)$ for all $b \in L \cup R$. 
	\label{lemma_on_length_of_cuts}
\end{lem}
\begin{proof}
	Suppose that $\alpha$ satisfies $l(\left\{ L \vert R \right\}) > 
	\alpha > l(b)$ for all $b \in L \cup R$. Then 
	$c \coloneq \curly{L \vert R} \restriction \alpha$ also 
	satsfies $L < c < R$, contradicting the minimality of 
	$l(\left\{ L \vert R \right\})$. For the second part, let 
	$a \in \No$ and set $\alpha \coloneq l(a)$. Put: 
	\begin{align*}
		L &\coloneq \curly{b \in \No \colon b < a 
			\text{ and } l(b) < \alpha} \\
			R &\coloneq \curly{b \in \No \colon 
				b > a \text{ and } l(b) < \alpha}
	\end{align*}
	Then $L < a < R$ and $L \cup R$ contains all surreals of 
	length $< \alpha = l(a)$. So $a = \curly{L \vert R}$. 
\end{proof}
\begin{defn}
	Let $L, L', R, R'$ be subsets of $\No$. We say that 
	$(L', R')$ is \emph{cofinal} in $(L, R)$ if: 
	\begin{itemize}
		\item $(\forall a \in L)(\exists a' \in L')$ 
		such that $a \leq a'$, and 
		\item $(\forall b \in R)(\exists b' \in R')$
		such that $b \geq b'$.
	\end{itemize}
\end{defn}
Some trivial observations: 
\begin{itemize}
	\item If $L' \supseteq L$ and $R' \supseteq R$, then 
		$(L', R')$ is cofinal in $(L, R)$ and in 
		particular $(L, R)$ is cofinal in $(L, R)$. 
	\item Cofinality is transitive. 
	\item If $(L', R')$ is cofinal in $(L, R)$ and 
		$L' < R'$, then $L < R$. 
	\item If $(L', R')$ is cofinal in $(L, R)$ and 
		$L' < a < R'$, then $L < a < R$. 
\end{itemize}
\begin{theorem}[The ``Cofinality Theorem'']
	Let $L, L', R, R'$ be subsets of $\No$ with 
	$L < R$. Suppose $L' < \curly{L \vert R} < R'$ and 
	$(L', R')$ is cofinal in $(L, R)$. Then $\left\{ L \vert 
	R\right\} = \curly{L' \vert R'}$. 
	\label{cofinality_theorem}
\end{theorem}
\begin{proof}
	Suppose that $L' < a < R'$. Then $L < a < R$ since 
	$(L', R')$ is cofinal in $(L, R)$. Hence 
	$l(a) \geq l( \curly{L \vert R})$. Thus 
	$\left\{ L \vert R \right\} = \curly{L \vert R'}$. 
\end{proof}
\begin{cor}[Canonical Representation]
	Let $a \in \No$ and set 
	\begin{align*}
		L' &= \curly{b \colon b < a \text{ and } b <_s a} \\
		R' &= \curly{b \colon b > a \text{ and } b <_s a}
	\end{align*}
	Then $a = \curly{L' \vert R'}$. 
\end{cor}
\begin{proof}
	By Lemma \ref{lemma_on_length_of_cuts} take 
	$L < R$ such that $a = \curly{L \vert R}$ and 
	$l(b) < l(a)$ for all $b \in L \cup R$. Then 
	$L' \subseteq L$ and $R' \subseteq R$, so $(L, R)$ is 
	cofinal in $(L', R')$. By Theorem \ref{cofinality_theorem}
	it remains to show that $(L', R')$ is cofinal in 
	$(L, R)$. 

	For this let $b \in L$ be arbitrary. Then 
	$l(a \wedge b) \leq l(b) < l(a)$ and 
	thus $b \leq (a \wedge b) < a$. Therefore 
	$a \wedge b \in L'$. Similarly for $R$. 
\end{proof}
Exercise: let $a = \curly{L' \vert R'}$ be the canonical 
representation of $a \in \No$. Then 
\begin{align*}
	L' &= \curly{a \restriction \beta \colon a(\beta) = +} \\
	R' &= \curly{a \restriction \beta \colon a(\beta) = -}
\end{align*}

Exercise: Let $a = \curly{L' \vert R'}$ be the canonical representation 
of $a \in \No$. Then 
\begin{align*}
	L' &= \curly{a \restriction \beta \colon a(\beta) = +} \\
	R' &= \curly{a \restriction \beta \colon a(\beta) = 1}
\end{align*}
For example, if $a = (++-+--+)$, then $L' = \{(), (+), (++-), (++-+--)\}$
and $R' = \{(++), (++-+), (++-+-)\}$. Note that the elements of 
$L'$ decrease in the ordering as their length increases, whereas those 
of $R'$ do the opposite. Also note that the canonical representation 
is not minimal, as $a$ may also be realized as the cut 
$a = \curly{(++-+--) \vert (++-+-)}$. 
\begin{cor}[``Inverse Cofinality Theorem'']
	Let $a = \curly{L \vert R}$ be the canonical representation 
	of $a$ and let $a = \curly{L' \vert R'}$ be an arbitrary 
	representation. Then $(L', R')$ is cofinal in $(L, R)$. 
	\label{inverse_cofinality_theorem}
\end{cor}
\begin{proof}
	Let $b \in L$ and suppose that for a contradiction that 
	$L' < b$. Then $L' < b < a < R'$, and $l(b) < l(a)$, 
	contradicting $a = \curly{L' \vert R'}$. 
\end{proof}
\subsection*{Arithmetic Operators}
We will define addition and multiplication on $\No$ and we will 
show that they, together with the ordering, make $\No$ into 
an ordered field. 
\section*{Day 4: Friday, October 10, 2014}
We begin by recalling some facts about ordinal arithmetic: 
\begin{theorem}[Cantor's Normal Form Theorem]
	Every ordinal $\alpha$ can be uniquely represented as
	\begin{align*}
		\alpha = \omega^{\alpha_1} a_1 + \omega^{\alpha_2}
		a_2 + \cdots + \omega^{\alpha_n} a_n
	\end{align*}
	where $\alpha_1 > \cdots > \alpha_n$ are ordinals and 
	$a_1, \cdots, a_n \in \N \setminus \curly{0}$. 
	\label{}
\end{theorem}
\begin{defn}
	The (Hessenberg) \emph{natural sum} $\alpha \oplus \beta$ of 
	two ordinals
	\begin{align*}
		\alpha &= \omega^{\gamma_1} a_1 + \cdots \omega^{\gamma_n}
		a_n \\
		\beta &= \omega^{\gamma_1} b_1 + \cdots \omega^{\gamma_n} 
		b_n
	\end{align*}
	where $\gamma_1 > \cdots > \gamma_n$ are ordinals and 
	$a_i, b_j \in \N$, is defined by: 
	\begin{align*}
		a \oplus \beta = \omega^{\gamma_1}(a_1 + b_1) + \cdots 
		+ \omega^{\gamma_n}(a_n + b_n)
	\end{align*}
\end{defn}
The operation $\oplus$ is associative, commutative, and strictly increasing 
in each argument, i.e. $\alpha < \beta \implies a \oplus \gamma < \beta \oplus 
\gamma$ for all $\alpha, \beta, \gamma \in \On$. Hence 
$\oplus$ is \emph{cancellative}: $\alpha \oplus \gamma = \beta \oplus 
\gamma \implies \alpha = \beta$. There is also a notion of 
\emph{natural product} of ordinals: 
\begin{defn}
	For $\alpha, \beta$ as above, set 
	\begin{align*}
		\alpha \otimes \beta \coloneq 
		\bigoplus_{i, j}{\omega^{\gamma_i \oplus \gamma_j}a_i 
	b_j}
	\end{align*}
\end{defn}
The natural product is also associative, commutative, and strictly 
increasing in each argument. The distributive law also holds for 
$\oplus$, $\otimes$: 
\begin{align*}
	\alpha \otimes (\beta \oplus \gamma) = (\alpha \otimes \beta) 
	\oplus (\alpha \otimes \gamma)
\end{align*}
In general $\alpha \oplus \beta \geq \alpha + \beta$. Moreover 
strict inequality may occur: $1 \oplus \omega = \omega + 1 > \omega = 
1 + \omega$. 

%In the following, if $a = \curly{L \vert R}$ is the canonical 
%representation of $a \in \No$ then we let $a_L$ range over 
%$L$ and $a_R$ range over $R$ (so in particular $a_L < a < a_R$). 
In the following, if $a = \curly{L \vert R}$ is the canonical 
representation of $a \in \No$, we set $L(a) = L$ and 
$R(a) = R$. We will use the shorthand $X + a = 
\left\{ x + a \colon x \in X \right\}$ (and its obvious 
variations) for $X$ a subset of 
$\No$ and $a \in \No$. 

\begin{defn}
	Let $a, b \in \No$. Set
	\begin{align}
		a + b \coloneq 
		\left\{ (L(a) + b) \cup (L(b) + a) \vert 
		(R(a) + b) \cup (R(b) + a) \right\}
		\label{defn_of_surreal_sum}
	\end{align}
\end{defn}
Some remarks: 
\begin{enumerate}[(1)]
	\item This is an inductive definition on $l(a) \oplus l(b)$. 
		There is no special treatment needed for the base 
		case: $\left\{ \emptyset \vert \emptyset \right\} = 
		+ \curly{\emptyset \vert \emptyset} = 
		\left\{ \emptyset \vert \emptyset \right\}$. 
	\item To justify the definition we need to check that 
		the sets $L, R$ used in defining $a + b = 
		\left\{ L \vert R \right\}$ satisfy $L < R$. 
\end{enumerate}
\begin{lem}
	Suppose that for all $a, b \in \No$ with $l(a) \oplus 
	l(b) < \gamma$ we have defined $a + b$ so that 
	Equation \ref{defn_of_surreal_sum} holds and 
	\begin{align*}
		b > c \implies a + b > a + c 
		\text{ and } b + a > c + a
		\tag{$*$}
	\end{align*}
	holds for all $a, b, c \in \No$ with $l(a) \oplus 
	l(b) < \gamma$ and $l(a) \oplus l(c) < \gamma$. Then 
	for all $a, b \in \No$ with $l(a) \oplus l(b) \leq \gamma$ we have 
	\begin{align*}
		(L(a) + b) \cup (L(b) + a) < 
		(R(a) + b) \cup (R(b) + a)
	\end{align*}
	and defining $a + b$ as in Equation \ref{defn_of_surreal_sum}, 
	$(*)$ holds for all $a, b, c \in \No$ with $l(a) \oplus 
	l(b) \leq \gamma$ and $l(a) \oplus l(c) \leq \gamma$. 
\end{lem}
\begin{proof}
	The first part is immediate from $(*)$ in conjunction with the 
	fact that $l(a_L), l(a_R) < l(a)$, $l(b_L), l(b_R) < l(b)$
	for all $a_L \in L(a), a_R \in R(a)$, $b_L \in L(b)$, and 
	$b_R \in R(b)$. 
Define $a + b$ for $a, b \in \No$ with $l(a) \oplus l(b) \leq 
\gamma$ as in Equation \ref{defn_of_surreal_sum}. Suppose 
$a, b, c \in \No$ with $l(a) \oplus l(b), l(a) \oplus l(c) \leq 
\gamma$, and $b > c$. Then by definition we have 
\begin{align*}
	a + b_L < \;& a + b \\
	& a + c < a + c_R
\end{align*}
for all $b_L \in L(b)$ and $c_R \in R(c)$. If $c <_s b$ then 
we can take $b_L = c$ and get $a + b > a + c$. Similarly, if 
$b <_s c$, then we can take $c_R = b$ and also get $a + b > a + c$. 
Suppose neither $c <_s b$ nor $b <_s c$ and put 
$d \coloneq b \wedge c$. Then $l(d) < l(b), l(c)$ and 
$b > d > c$. Hence by $(*)$, $a + b > a + d > a + c$. 

We may show $b + a > c + a$ similarly. 
\end{proof}
\begin{lem}[``Uniformity'' of the Definition of $a$ and $b$]
	Let $a = \curly{L \vert R}$ and $a' = \curly{L' \vert R'}$. 
	Then
	\begin{align*}
		a + a' = 
		\left\{ (L + a') \cup (a' + L) \vert 
		(R + a') \cup (a + R') \right\}
	\end{align*}
\end{lem}
\begin{proof}
	Let $a = \curly{L_a \vert R_a}$ be the canonical 
	representation. By Corollary \ref{inverse_cofinality_theorem}
	$(L, R)$ is cofinal in $(L_a, R_a)$ and $(L', R')$ is 
	cofinal in $(L_{a'}, R_{a'})$. Hence 
	\begin{align*}
		\paren{(L + a') \cup (a + L'), (R+a') \cup (a + R')}
	\end{align*}
	is cofinal in 
	\begin{align*}
		\paren{(L_a + a') \cup (a + L_{a'}), (R_a + a') \cup 
		(a + R_{a'})}
	\end{align*}
	Moreover, 
	\begin{align*}
		(L + a') \cup (a + L') < a + a' < 
		(R + a') \cup (a + R')
	\end{align*}
	Now use Theorem \ref{cofinality_theorem} to conclude the 
	proof. 
\end{proof}
\section{ Week 2 }

(Notes by John Lensmire)

\subsection{ Monday 10-13-2014 }

Let $a,b\in \mathbf{No}$. Recall that $a + b = \{a_L + b, a + b_L | a_R + b, a + b_R \}$.

==== Theorem 2.5 ====

$(\mathbf{No},+,<)$ is an ordered abelian group with $0 = () = \{\emptyset, \emptyset \}$ and $-a$ is obtained by reversing all signs in $a$.

'''Proof:'''

We have already proven that $\leq$ is translation invariant.

Commutativity is clear from the symmetric nature of the definition.

We show by induction on $l(a)$ that $a+0 = a$. The base case is clear, and
\begin{align*}
a + 0 &= \{a_L + 0, a + 0_L | a_R + 0, a + 0_R \} \\
&= \{a_L + 0 | a_R + 0\} \ (\text{as } 0_L = 0_R = \emptyset) \\
&= \{a_L | a_R\} \ (\text{by induction}) \\
&= a
\end{align*}

We next show the associative law by induction on $l(a)\oplus l(b)\oplus l(c)$.
We have
\begin{align*}
(a+b)+c &= \{(a+b)_L + c, (a+b) + c_L | (a+b)_R + c, (a+b) + c_R \} \\
&= \{(a_L+b) + c, (a+b_L) + c, (a+b) + c_L | (a_R+b) + c, (a+b_R) + c, (a+b) + c_R\}
\end{align*}
where the second equality holds because of uniformity.
An identical calculation shows:
\[
a+(b+c) = \{a_L+ (b + c), a+ (b_L + c), a+ (b + c_L) | a_R+ (b + c), a+ (b_R + c), a+ (b + c_R)\}
\]
and hence $(a+b)+c = a+(b+c)$ holds by induction.

To show $a + (-a) = 0$ first note:
* $b <_s a \Rightarrow -b <_s -a$
* $b < a \Rightarrow -b > -a$

Hence, $-a = \{-a_R | -a_L\}$. Thus,
\[
a + (-a) = \{a_L + (-a), a + (-a_R) | a_R + (-a), a + (-a_L) \}
\]
By the induction hypothesis and the fact that $+$ is increasing we have the following:
* $a_L + (-a) < a_L + (-a_L) = 0$
* $a + (-a_R) < a_R + (-a_R) = 0$
* $a_R + (-a) > a_R + (-a_R) = 0$
* $a + (-a_L) > a_L + (-a_L) = 0$
Thus, $0$ is a realization of the cut. Since $0$ has minimal length, we have $a+(-a) = 0$ as needed.

==== Definition 2.6 ====

For $a,b\in \mathbf{No}$ set
\[
a\cdot b = \{a_L\cdot b + a\cdot b_L - a_L\cdot b_L, a_R\cdot b + a\cdot b_R - a_R\cdot b_R | a_L\cdot b + a\cdot b_R - a_L\cdot b_R, a_R\cdot b + a\cdot b_L - a_R\cdot b_L \}
\]
As motivation for this definition, note that in any ordered field: if $a'<a,b'<b$ then $(a-a')(b-b')>0$ so in particular $a'b + ab' - a'b' < ab$.

==== Lemma 2.7 ====

Suppose for all $a,b\in \mathbf{No}$ with $l(a)\oplus l(b) < \gamma$ we have defined $a\cdot b$ so that (2.6) holds, and for all $a,b,c,d\in \mathbf{No}$ with the natural sum of the lengths of each factor is $<\gamma$ $(*)$ holds, where
$(*): a>b, c>d \Rightarrow ac-bc > ad-bd.$
Then: (2.6) holds for all $a,b\in \mathbf{No}$ with $l(a)\oplus l(b) \leq \gamma$ and $(*)$ holds for all $a,b,c,d\in \mathbf{No}$ with the natural sum of the lengths of each factor is $\leq \gamma$.

'''Proof:'''

Let $P(a,b,c,d)\Leftrightarrow ac - bc > ad - bd.$ Because the surreal numbers form an ordered abelian group, $P$ is  "transitive in the last two variables" (i.e. $P(a,b,c,d) \ \&\ P(a,b,d,e) \Rightarrow P(a,b,c,e)$) and similarly in the first two variables.

Fix $a,b\in \mathbf{No}$. For $a' <_s a, b' <_s b$ we define $f(a',b') = a'b + ab' - a'b'$.

Claim:
\begin{enumerate}
	\item  $a' < a \Rightarrow b' \mapsto f(a',b')$ is an increasing function
	\item  $a' > a \Rightarrow b' \mapsto f(a',b')$ is a decreasing function
	\item  $b' < b \Rightarrow a' \mapsto f(a',b')$ is an increasing function
	\item  $b' > b \Rightarrow a' \mapsto f(a',b')$ is a decreasing function
\end{enumerate}

We prove 1 (the rest are left as an exercise). Let $b'_1,b'_2 <_s b$ and $b'_1 < b'_2$. Then
\begin{align*}
f(a',b'_2) > f(a',b'_1) &\Leftrightarrow (a'b + ab'_2 - a'b'_2) > (a'b + ab'_1 - a'b'_1)  \\
&\Leftrightarrow (ab'_2 - a'b'_2) > (ab'_1 - a'b'_1) \\
&\Leftrightarrow P(a,a',b'_2,b'_1)
\end{align*}
and $P(a,a',b'_2,b'_1)$ holds by induction, proving 1.

1-4 in the claim give us respectively:
* $f(a_L, b_L) < f(a_L, b_R)$
* $f(a_R, b_R) < f(a_R, b_L)$
* $f(a_L, b_L) < f(a_R, b_R)$
* $f(a_R, b_R) < f(a_L, b_L)$

These facts exactly give us that $a\cdot b$ is well-defined.

We are left to show that $(*)$ continues to hold. We'll continue this on Wednesday.

\subsection{ Wednesday 10-15-2014 }

Recall the definition of multiplication from last time:
\[
a\cdot b = \{a_L\cdot b + a\cdot b_L - a_L\cdot b_L, a_R\cdot b + a\cdot b_R - a_R\cdot b_R | a_L\cdot b + a\cdot b_R - a_L\cdot b_R, a_R\cdot b + a\cdot b_L - a_R\cdot b_L \}
\]
and the statement $(*): a>b, c>d \Rightarrow ac-bc > ad-bd$. We'll also continue write $P(a,b,c,d)\Leftrightarrow ac - bc > ad - bd$.

Note we can rephrase the defining inequalities for $a\cdot b$ as
\[
(\Delta): P(a,a_L,b,b_L), P(a_R,a,b_R,b), P(a,a_L,b_R,b), P(a_R,a,b,b_L)
\]

To finish the proof of Lemma 2.7, suppose $a>b>c>d$ (of suitable lengths). We want to show $P(a,b,c,d)$.

Case 1:  Suppose in each pair $\{a,b\}, \{c,d\}$ one of the elements is an initial segment of the other. Then note we are done by $(\Delta)$.

Case 2:  Suppose $a \not<_s b, b\not<_s a$ but $c <_s d$ or $d <_s c$. Then we have that $a > a\wedge b > b$ and by Case 1, $P(a,a\wedge b, c, d)$ and $P(a\wedge b, b, c, d)$. This implies (by transitivity of the first two variables) $P(a,b,c,d)$.

Case 3:  Suppose $c \not<_s d, d\not<_s c$. Then by Case 1 and 2, $P(a,b,c,c\wedge d)$ and $P(a,b,c\wedge d, d)$, so (transitivity of the last two variables) $P(a,b,c,d)$ as needed. This completes the proof of Lemma 2.7.

==== Lemma 2.8 ====

The uniformity property holds for multiplication.

'''Proof:'''

Fix $a,b\in \mathbf{No}$. For any $a',b'\in \mathbf{No}$ we define (as last time) $f(a',b') = a'b + ab' - a'b'$. Using Lemma 2.7, we can extend the Claim from last time to hold in general:

Claim:
\begin{enumerate}
\item $a' < a \Rightarrow f(a',-)$ is an increasing function
\item $a' > a \Rightarrow f(a',-)$ is a decreasing function
\item $b' < b \Rightarrow f(-,b')$ is an increasing function
\item $b' > b \Rightarrow f(-,b')$ is a decreasing function
\end{enumerate}

Let $a = \{L|R\}, b = \{L'|R'\}$ be any representations of $a,b$. We want to verify the hypothesis of the Cofinality Theorem 1.10.

Let $a_l, b_l$ range over $L,L'$. As an example, note
\[
f(a_l, b_l) < ab \Leftrightarrow 0 < ab - (a_lb + ab_l - a_lb_l) = (ab - a_lb) - (ab_l - a_lb_l)
\Leftrightarrow P(a,a_l,b,b_l)
\]
which holds as $a_l<a, b_l<b$. Checking the other inequalities in a similar manner gives us the first assumption of Theorem 1.10.

To get the second assumption (the cofinality hypothesis), let e.g. $f(a_L,b_L)$ in the left side of the representation of $ab$. By inverse cofinality, we get $a_l\in L, b_l\in L'$ with $a_l \geq a_L, b_l\geq b_L$.
Then (using (3) and (1) respectively from the above claim): $f(a_l,b_l) \geq f(a_L,b_l)\geq f(a_L,b_L)$ as needed. Again, the other cases are similar.

Therefore, both assumptions of Theorem 1.10 hold, giving us the uniformity property for multiplication as needed.

==== Proposition 2.9 ====

$(\mathbf{No},+,\cdot,\leq)$ is an ordered commutative ring, with multiplicative identity $1 = (+) = \{0 | \emptyset \}$.

'''Proof:'''

Commutativity is clear by the symmetry in the definition.

We show the distributative law by induction on $l(a) \oplus l(b) \oplus l(c)$ that $(a+b)\cdot c = a\cdot c + b\cdot c$.

In general, the typical element in the cut for $(a+b)\cdot c$ is $(a+b)_* c + (a+b)_* c_* - (a+b)_*c_*$ (where $(a+b)_*$ is either $(a+b)_L$ or $(a+b)_R$ and similar for $c_*$). This element is less than $(a+b)c$ if and only if there are $0$ or $4$ $*'s$ equal to $R$.

By uniformity, we can replace $(a+b)_*$ with $a_*+b,a+b_*$, so the typical terms become (using induction):
\[
(a_*+b)c + (a+b)c_* - (a_*+b)c_* = a_*c + bc + ac_* - a_*c_*
\]
or similarly
\[
(a+b_*)c + (a+b)c_* - (a+b_*)c_* = ac + b_*c + bc_* - b_*c_*
\]
On the other hand, the typical elements of $ac + bc$ are
\[
(ac)_* + bc = a_*c + ac_* - a_*c_* + bc \text{ or } ac + (bc)_* = ac + b_*c + bc_* - b_*c_*
\]
(note the same parity rule for $*$'s applies here.) A quick check shows that this matches a typical element of $(a+b)\cdot c$ we have $(a+b)c = ac+bc$ as needed.

Associativity is proven using a very similar argument (and is left as an exercise).

We are left to check the identity element. Note $a\cdot 0 = a\cdot (0+0) = a\cdot 0 + a\cdot 0$ which implies $a\cdot 0 = 0$.

By definition:
\[
a\cdot 1 = \{a_L\cdot 1 + a\cdot 1_L - a_L\cdot 1_L, a_R\cdot 1 + a\cdot 1_R - a_R\cdot 1_R | a_L\cdot 1 + a \cdot 1_R - a_L\cdot 1_R, a_R \cdot 1 + a\cdot 1_L - a_R\cdot 1_L\}
\]
Using the fact that $1_R = \emptyset$ and multiplication by $0 = 1_L$ is zero, we have
\[
a\cdot 1 = \{a_L\cdot 1 | a_R \cdot 1\} = \{a_L | a_R\} = a
\]
by induction. This completes the proof.

Our next goal is to define $1/a$ for $a>0$, i.e. find a solution to $a\cdot x = 1$.
Note, the naive idea is to set $x = \{1/a_R | 1/a_L, (a_L\neq 0)\}$ but this does not work in general.

\subsection{ Friday 10-17-2014 }

==== Definition of Inverses ====

Let $a\in \mathbf{No}$ with $a>0$. Our aim is to define $1/a$. Let $a = \{L|R\}$ the canonical representation of $a$. Observe that $a'\geq 0$ for all $a'\in L$ (as $a' <_s a$).

For every finite sequence $(a_1,\ldots,a_n)\in (L\cup R)\setminus \{0\}$ we define $\langle a_1,\ldots, a_n\rangle \in \mathbf{No}$ by induction on $n$. Set $\langle \ \rangle = 0$ and inductively set $\langle a_1,\ldots, a_n, a_{n+1}\rangle = \langle a_1,\ldots, a_n \rangle \circ a_{n+1}$.
Here, for arbitrary $b\in \mathbf{No}$ and $a'\in (L\cup R)\setminus \{0\}$ let $b\circ a'$ be the unique solution to $(a-a')b + a'x = 1$, i.e.
\[
b\circ a' = [1-(a-a')b]/a'.
\]
This works as inductively we'll have already defined $1/a'$.

For example, $\langle a_1 \rangle = \langle \ \rangle \circ a_1 = 0 \circ a_1 = 1/a_1$.

Now set (as candidates for defining $1/a$)
\begin{align*}
L^{-1} &= \{ \langle a_1,\ldots, a_n \rangle | \text{ the number of } a_i \text{ in }L \text{ is even} \} \\
R^{-1} &= \{ \langle a_1,\ldots, a_n \rangle | \text{ the number of } a_i \text{ in }L \text{ is odd} \}
\end{align*}
Note that this definition is an expansion of the naive idea presented at the end of last lecture.

We first show
\[
(*) x \in L^{-1} \Rightarrow ax < 1 \text{ and } x\in R^{-1} \Rightarrow ax > 1.
\]
by induction on $n$. In particular, this yields that $L^{-1} < R^{-1}$.

The base case is clear, as $\langle \ \rangle = 0\in L^{-1}$ and $a\cdot 0 = 0 < 1$.

For the induction, suppose $b\in L^{-1}\cup R^{-1}$ satisfying $(*)$ and $0\neq a' <_s a$. We show that $x = b\circ a'$ also satisfies $(*)$.

Claim:
\begin{enumerate}
\item $x > b \Leftrightarrow 1 > ab$
\item $ax = 1 + (a-a')(x-b)$.
\end{enumerate}
By definition, $x$ is the solution to $(a-a')b + a'x = 1$ and $ab = ab - a'b + a'b = (a-a')b + a'b$. These two equations yield $ab = 1 + a'(b-x)$. Both parts of the claim follow.

Now suppose, $b\in L^{-1}, a'\in L$. Then $x\in R^{-1}$ so want to check $ax > 1$.
$b\in L^{-1}$ and $(*)$ implies that $ab < 1$ and hence Claim 1 tells us that $x > b$. Now by Claim 2, $ax = 1 + (a-a')(x-b)$ hence $ax > 1$ (because $a>a', x>b$) as needed.

The other cases are similar, so $(*)$ holds in general.

Thus, we can set $c = \{L^{-1} | R^{-1} \}$. We claim that $1/a = c$, that is, $ac = 1 = \{0 | \emptyset\}$.

The typical element used to define $ac$ is $a'c + ac' - a'c'$ with $a'\in L\cup R, c'\in L^{-1}\cup R^{-1}$.

We first show $\{\text{ lower elements for }ac\} < 1 < \{(\text{ upper elements for }ac \}$.

Suppose that $a' = 0$, then we get an element $a'c + ac' - a'c' = ac'$ which is an upper element for $ac$ if and only if $c'\in R^{-1}$ by definition. However, by $(*)$ if $c'\in R^{-1}$ then $ac' = a'c + ac' - a'c' > 1$ (as needed since an upper element), else $c'\in L^{-1}$ and then $ac' = a'c + ac' - a'c' < 1$ (as needed since a lower element).

If $a'\neq 0$, then $x = c'\circ a'$ is defined, lies in $L^{-1}\cup R^{-1}$, and obeys $(\Delta): (a-a')c' + a'x = 1$. Hence,
\begin{align*}
a'c + ac' - a'c' \text{ is a lower element for } ac &\Leftrightarrow a' \ \&\ c' \text{ are on the same side of } a \ \&\ \text{(respectively) } c \\
&\Leftrightarrow x\in R^{-1} \Leftrightarrow x > c \\
&\Leftrightarrow \text{a typical element } a'c + ac' - a'c' = (a-a')c' + a'c < 1
\end{align*}
where the last equivalence holds by $(\Delta)$ and $a'>0$.

Since $1$ satisfies the cut for $ac$ (and $0$ does not) it is the minimial realization, so $ac = 1$ as needed.

We have thus shown:

==== Theorem 2.10 (Conway) ====

$(\mathbf{No}, +, \cdot, \leq)$ is an ordered field.

We'll now begin to focus on how to view real numbers and ordinals as surreal numbers.

We have $0 = ()$ the additive identity of $\mathbf{No}$ and $1 = (+)$ the multiplicative identity of $\mathbf{No}$.

We also have an embedding of ordered rings $\mathbb{Z} \hookrightarrow \mathbf{No}$ where $k\mapsto k\cdot 1$ (where $k\cdot 1$ is $k$ additions of $+1$ or $-1$).

==== Lemma 3.1 ====

For $n\in \mathbb{N}$, $n\cdot 1 = (+\cdots +)$ (i.e. $n$ $+$'s).

%\documentclass{article}
%\usepackage{amsmath}
%\usepackage{amssymb}
%\usepackage{amsthm}
%
%\newcommand{\N}{\mathbb{N}}
%\newcommand{\Z}{\mathbb{Z}}
%\newcommand{\Q}{\mathbb{Q}}
%\newcommand{\R}{\mathbb{R}}
%\newcommand{\concat}{\mathrel{\hat{\ }}}
%
%\theoremstyle{definition}
%\newtheorem{cor}{Corollary}
%\newtheorem{example}{Example}
%\newtheorem{proposition}{Proposition}
%\newtheorem{lemma}{Lemma}
%\newtheorem{defn}{Definition}
%\newtheorem{theorem}{Theorem}
%
%\begin{document}

Notes by Zach.

\section*{October 20th}

\begin{lemma}
For $n\in\N$, $n\cdot 1 = (\underbrace{++\cdots+}_{n\text{ times}})$.
\label{3.1}
\end{lemma}

\begin{proof} (of 3.1)
By induction on $n$. The cases $n=0,1$ are obvious.
Suppose that $n\ge 1$. We have
\begin{align*}
(n+1)\cdot 1 &= n\cdot 1 + 1 \\
&= (++\cdots +)+(+) \\
&= \{ (\underbrace{+\cdots+}_{n-1}) \,|\, \emptyset \} + \{ 0 \,|\, \emptyset \} \\
&= \{ (n-1)\cdot1 \,|\, \emptyset \} + \{ 0 \,|\, \emptyset \} \\
&= \{(n-1)\cdot 1 + 1,\, n\cdot 1 + 0 \,|\, \emptyset \} \\
&= \{ n\cdot 1 \,|\, \emptyset \} \\
&= \{ (\underbrace{++\cdots + }_{n}) \,|\, \emptyset \} \\
&= (\underbrace{++\cdots +}_{n+1}).
\end{align*}
\end{proof}

\begin{cor}
For $n\in\N$, $-n\cdot 1 = (\underbrace{- - \cdots -}_{n\text{ times}})$.
\end{cor}

An ordered field $k$ is {\em archimedean} if for all $a,b>0$ in $k$ there is $n\in\N$ such that $na>b$.

Note: $\mathbf{No}$ is not archimedean, since $\omega := (++\cdots)$ (with $\omega$ many $+$s) satisfies $\omega>(\underbrace{+\cdots + }_{n})=n\cdot 1$.

From now on we identify $\Z$ as a subring of $\mathbf{No}$.

Q: How to identify $\Q\subseteq \mathbf{No}$?

Idea: $0 = () < (+-) < (+) = 1$. Is $(+-)=\tfrac12$?

Finite sequences of $+$s and $-$s correspond to dyadic rationals, i.e., rationals of the form $\frac{a}{2^s}$ ($a\in\Z$, $s\in\N$). We might conjecture that $\Q$ corresponds to finite sequences in $\mathbf{No}$.

\begin{lemma}
Suppose $a+b = \{2a \,|\, 2b \}$. Then $\frac{a+b}2 = \{a \,|\, b \}$.
\end{lemma}

\begin{proof}
Put $c := \{a \,|\, b\}$. Then $2\cdot c = c+ c = \{a+c \,|\, b+c \}$, and we show that this equals $a+b$:
\[ a+c < a+b < b+c, \text{ since } a< c < b, \text{ and } \]
\[ 2a < a+c, \; 2b > b+c \text{ because } a<c < b. \]
Now the result follows by the uniformity of the definition of $+$: since we assumed $a+b = \{2a \,|\, 2b \}$, we get the claim by cofinality.
\end{proof}

\begin{example}
$1 = \{ 0 \,|\, \emptyset \} = \{ 0 \,|\, 2 \}$. Apply the lemma with $a=0$, $b=1$.
So $\tfrac12 = \{0 \,|\, 1 \} = (+-)$. Taking $a=0$, $b=\tfrac12$, we get $\tfrac14 = (+--)$.
\end{example}

\begin{cor} (3.4)
Suppose $a + b = \{ 2a \,|\, 2b \}$. Then 
\[ \frac{a+b}{2^{s+1}} = \left\{ \frac{a}{2^s} \,\bigg|\, \frac{b}{2^s} \right\}  \]
for all $s\in\N$.
\label{3.4}
\end{cor}

\begin{cor} (3.5)
For all $c\in\N$, $\frac{c}{2^s} + \frac{1}{2^{s+1}} = \left\{ \frac{c}{2^s} \,|\, \frac{c+1}{2^s} \right\}$.
\label{3.5}
\end{cor}

\begin{proof} (Proof of 3.5)
Take $a=c$, $b=c+1$. We have
\begin{align*}
a+b &= 2c+1 \\
&= \{ 2c \,|\, 2c+2 \} \\
&= \{ 2a \,|\, 2b \}.
\end{align*}
Apply Corollary \ref{3.4}.
\end{proof}

\begin{proposition} (3.6)
Surreal numbers of finite length correspond to dyadic rationals.
\label{3.6}
\end{proposition}

\begin{proof} (Proof of 3.6)
Let $d\in\mathbf{No}$ have length $m+n$, where $d(0) = d(1) = \cdots = d(m-1) \ne d(m)$. We'll show that 
$d\in\frac1{2^n}\Z$. Suppose $d(0) = d(m-1) = {+}$. (Similar if $d(0) = {-}$.) 
\begin{description}
\item $n=0$: follows by \eqref{3.1}.
\item $n=1$: Then $d = (\underbrace{++\cdots+}_{m\ge1}-)$. By \eqref{3.5} with $c = m-1$, $s=0$, $m-\frac12 = \frac{m-1}{2^0} + \frac{1}{2^1} = \{m-1 \,|\, m\}$. Clearly $\{m-1 \,|\, m\} = d$.
\end{description}
Now suppose we've shown the claim for all $n\ge r$, and let $n = r+1$; suppose $r\ge 1$. Let $d' = d\restriction (m+r)$.
Either $d = d'\concat (+)$ or $d=d'\concat (-)$; suppose wlog that $d = d'\concat (+)$. So $d' = (\underbrace{++\cdots+}_{m}-\cdots)$. Let $d = \{L \,|\, R\}$ be the canonical representation.
Taking $x=\max L$, $y = \min R$, we have $d = \{x \,|\, y\}$ by cofinality. Note $x = d'$. We don't know much about $y$ 
except that
\[ d = (\underbrace{++\cdots +}_{m}-\cdots -)\le (\underbrace{++\cdots+}_{m}) = m, \]
so $y\le m$. By inductive hypothesis, $x,y\in \frac1{2^r}\Z$. So $d' = x = \frac{c}{2^r}$ for some $c\in\N$. 
If we can show $y = \frac{c+1}{2^r}$, then by \eqref{3.5} 
\[ d = \{x \,|\, y \} = \frac{c}{2^r} + \frac{1}{2^{r+1}} \in \frac1{2^{r+1}}\Z, \]
as required.

Note 
\[m-1<x = d' = \frac{c}{2^r} < \frac{c+1}{2^r} \le y \le m. \]
Put $H := \{ h\in\mathbf{No} \,:\, \ell(h)\le m+ r \,\&\, h\restriction(m+1) = (++\cdots+-)\}$.
We have $|H| = 1+2+2^2+\cdots + 2^{r-1} = 2^r - 1$. By inductive hypothesis every $h\in H$ belongs to $\frac1{2^r}\N$,
and $m-1<h<m$. But there are exactly $2^r - 1$ many dyadic rationals satisfying both of these conditions. 
In particular, $\frac{c+1}{2^r}\in H$ or $\frac{c+1}{2^r}=m$. Either way, $\ell(\frac{c+1}{2^r})\le m+r$.
Since $\frac{c+1}{2^r} > d'$, this implies $\frac{c+1}{2^r} > d$. We have $\frac{c+1}{2^r} <_s d$:
otherwise $e := \frac{c+1}{2^r} \wedge d$ will satisfy $d < e < \frac{e+1}{2^r} \le y$, contradiction to choice of $y$.
Hence $\frac{c+1}{2^r}\in R$, so $\frac{c+1}{2^r}\ge y$.
\end{proof}

\section*{October 22nd}

\textbf{Remark} (to \eqref{3.6}). Suppose $d\in\mathbf{No}$ has length $m+n$, where
\begin{itemize}
\item $d(0) = \cdots = d(m-1)$, and
\item $d(m-1)\ne d(m)$.
\end{itemize}
Define
\[ b(i) := \begin{cases} \pm 1 & \text{if } i< m, \quad d(i)=\pm \\
\pm \frac{1}{2^{i-m+1}} & \text{if } i \ge m, \quad d(i) = \pm.
\end{cases} \]
Then $d = b(0) + b(1) + \cdots + b(m+n-1)$. (Exercise.) Also, every dyadic rational arises in this way. (Exercise.)
We now let $\mathbb{D}$ be the set of dyadic rationals $\Z[\frac12]\subseteq\Q$.

\begin{defn}
A surreal is called {\em real} if it is either of finite length or has length $\omega$ and is not ultimately constant.
\end{defn}

Recall: An ordered field $k$ is {\em dedekind-complete} if every nonempty subset of $k$ that is bounded from above
has a supremum. The ordered field $\R$ is up to (unique) isomorphism the only dedekind-complete ordered field.

\begin{theorem}[Conway] (3.8)
The real surreals form a dedekind-complete ordered subfield of $\mathbf{No}$. Let $a\in\mathbf{No}$, $\ell(a)=\omega$,
with canonical representation $a = \{L \,|\, R\}$ then
\[ a\text{ is real } \iff \begin{cases} L,R\neq\emptyset \\
L\text{ has no max} \\
R\text{ has no min}
\end{cases}.\]
\end{theorem}

The direction ``$\Rightarrow$'' is an exercise. Now we prove the other direction.

\begin{lemma} (3.9)
Let $L,R\subseteq\mathbb{D}$ be such that $L<R$ and $L$ has no max and $R$ has no min. Then $a = \{ L \,|\, R\}$ is real.
\end{lemma}

\begin{proof}
By \eqref{1.9} we have $\ell(a)\le\omega$. Suppose $\ell(a) = \omega$ and $a(n) = +$ eventually. Note that there is some $n_0$ with $a(n_0) = {-}$, since $a<R\ne\emptyset$. We may assume that $a(n) = {+}$ for all $n>n_0$. Let $b = a\restriction n_0$. Then $b>a$ and $b<_s a$. Since $(L,R)$ is cofinal in the canonical representation of $a$, there is $d\le b$, $d\in R$. It follows that $d\in R$, since $R$ has no least element. We can choose $m$ such that $d\le b - \frac1{2^m}$. (Possible since both $b$ \& $d$ are dyadics.) We get $a<d\le b - \frac1{2^m}$. Next let $n>n_0$, $c = a\restriction n$. Then
$c<a$ and $c <_s a$, so by choosing $n$ sufficiently large we can achieve $c>b-\frac1{2^m}$. But then $a>c>b-\frac{1}{2^m}$, oops.

(Why can we do this? Write
\[ c = k + \sum_{i= l}^{n-1} \pm\frac{1}{2^{i-l+1}}\]
and
\[ b = k + \sum_{i=l}^{n_0-1} \pm\frac{1}{2^{i-l+1}}. \]
Then we get
\[ c-b = -\frac{1}{2^{n_0-l+1}} + \frac{1}{2^{n_0-l+2}} + \cdots + \frac{1}{2^{n-l}} = -\frac{1}{2^{n_0-l+1}} + \left( \frac{1}{2^{n_0-l+1}} - \frac{1}{2^{n-l}} \right) = \frac{-1}{2^{n-l}}, \]
so just choose $n>l+m$.)
\end{proof}

\begin{lemma}
Let $a = \{L \,|\, R\}$. Suppose
\begin{enumerate}
\item $x\in L \Rightarrow \exists r\in \mathbb{D}^{>0}$, $y\in L$, $x+r\le y$. [``$L-\mathbb{D}^{>0}$ is cofinal in $L$'']
\item $x\in R \Rightarrow \exists r\in \mathbb{D}^{>0}$, $y\in R$, $y\le x-r$. [``$R+\mathbb{D}^{>0}$ is coinitial in $R$.'']

and also $L' < a < R'$ such that
\item $(\forall r\in \mathbb{D}^{>0})(\exists x'\in L')(\exists y'\in R') y'-x'\le r$. [``$R'-L'$ is coinitial in $\mathbb{D}^{>0}$'']
\end{enumerate}
Then $a = \{ L' \,|\, R' \}$. 
\end{lemma}

\begin{proof}
Check that $(L',R')$ is cofinal in $(L,R)$ and use the cofinality theorem.
\end{proof}

\begin{lemma}
$\mathbb{D}$ is dense in the ordered set of real surreals.
\end{lemma}

\begin{proof}
Let $a<b$ be reals. If neither $a <_s b$ nor $b <_s a$ then $a < a\wedge b < b$. (Note $a\wedge b$ is finite, so it's in $\mathbb{D}$.) So suppose that $a <_s b$. Then $a\in \mathbb{D}$. If also $b\in\mathbb{D}$, then we're done:
$a < \frac{a+b}2 < b$. So suppose $b\notin \mathbb{D}$, with canonical representation $b = \{ L \,|\, R\}$. Then $a\in L$,
and $L$ has no maximum, so we can find some dyadic element of $(a,b)$. Similar if $b <_s a$.
\end{proof}

\begin{lemma}
Let $a = \{ L \,|\, R \}$ be the canonical representation of a real $a\notin \mathbb{D}$. Then for all $r\in \mathbb{D}^{>0}$ there are $a_L, a_R$ with $a_R-a_L\le r$. 
\end{lemma}

\begin{proof}
For each $n$ there are $a_L,a_R$ with $a_L\restriction n = a_R\restriction n$, and $a_R-a_L$ is bounded from above by some expression $\frac1{2^s} + \frac{1}{2^{s+1}}+\cdots$, and this can be made as small as necessary. (Exercise.)
\end{proof}

\section*{October 24th}

\begin{proof}[Proof of Theorem \ref{3.8}]
Clearly $0,1$ are real. Let $a,b\in\mathbf{No}$ be real; we check that $a+b$, $a\cdot b$, $\frac1a$ (if $a\ne0$) are also real. (Note that $-a$ is obviously real.)

\subsubsection*{$a+b$} Suppose $a\in\mathbb{D}$, $b\notin\mathbb{D}$. Let $a = \{ L \,|\, R\}$ be the canonical representation. So $a + b = \{ a_L + b,a+b_L \,|\, a_R+b,a+b_R\}$. We claim that $a+b = \{a+b_L \,|\, a+b_R \}$. (By \eqref{3.9} this then gives $a+b$ real, since $b_L$, $b_R$ dyadic.)
We have $a,a_L\in\mathbb{D}$, so $a-a_L\in\mathbb{D}^{>0}$; hence by \eqref{3.12} there are $b_L,b_R$ such that $b_R-b_L\le a-a_L$. It follows that $a+b_L\ge a_L+b_R \ge a_L+b$. Now use cofinality. (Similar argument for the other side.)

Now suppose that $a,b\notin\mathbb{D}$. Then in the representation of $a+b$ the LHS has no max and the RHS has no min. So (1) \& (2) in \eqref{3.10} are satisfied for this cut. Let $L' := \{a_L+b_L\}$, $R' := \{a_R+b_R\}$. Then
$L' < a+b < R'$ and by \eqref{3.12} $(L',R')$ satisfies (3) in \eqref{3.10}. This means that $a+b = \{L' \,|\, R' \}$ by \eqref{3.10}, and this is real by \eqref{3.9}.

\subsubsection*{$a\cdot b$} Suppose $a\notin\mathbb{D}$, $a,b>0$. The typical element in the representation of 
$a\cdot b$ is $ab-(a-a_*)(b-b_*)$. Show that (1), (2) in \eqref{3.10} are satisfied by this cut. For example,
\[ x = ab - (a-a_L)(b-b_L).\]
Take $a_L'$ with $a_L < a_L' < a$ and set $x' = ab - (a-a_L')(b-b_L)$. Then by \eqref{3.11} $x'-x = (a_L' - a_L)(b-b_L)$ is greater than some element of $\mathbb{D}^{>0}$. This verifies (1) of \eqref{3.10}. In the same way verify (2). Now set
\begin{align*}
L' &:= \{ a'b' \,:\, a',b'\in\mathbb{D},\, 0\le a' < a, \, 0\le b' < b\}, \\
R' &:= \{ a''b'' \,:\, a'',b''\in\mathbb{D}, \, a'' > a, \, b'' > b \}.
\end{align*}
Then $L' < a\cdot b < R'$. We check that \eqref{3.10}(3) holds. Let $r\in \mathbb{D}^{>0}$ be given. Then the same 
argument as proving the limit law for multiplication in calculus gives elements $a',b',a'',b''$ such that $a''b''-a'b'<r$, using \eqref{3.12}. Hence by \eqref{3.10} $a\cdot b = \{ L' \,|\, R' \}$, so $ab$ is real by \eqref{3.9}.

\subsubsection*{$1/a$:} We may assume that $a>0$. Put
\begin{align*}
L &:= \{ d\in \mathbb{D} \,:\, d\concat a < 1 \} \\
R &:= \{ d\in \mathbb{D} \,:\, d\concat a > 1 \}.
\end{align*}
Then $L<R$, $0\in L$, and $R\ne\emptyset$ because: by \eqref{3.11} take $m$ such that $a>\frac1{2^m}$, so that $2^ma>1$. So $2^m\in R$.

\noindent{\bf Claim 1.}
$L$ has no max; $R$ has no min.

\begin{proof}[Proof of Claim 1.]
Let $d\in L$. Then $1-da>0$ is real. So there is some $m$ such that $1-da> \frac1{2^m}$. Also can take $n$ such that $a< 2^n$. Then $\frac{1}{2^{m+n}}a < \frac1{2^m} < 1-da$, so $a(\frac{1}{2^{m+n}} + d) < 1$.
\end{proof}

Hence by \eqref{3.9} $b:= \{L \,|\, R\}$ is real. We are going to show that $|ba-1|<r$ for every $r\in\mathbb{D}^{>0}$. Since $ba-1$ is real, we get $ba-1=0$ by \eqref{3.11}.

\noindent{\bf Claim 2.}
For each $n$ there is $c\in R$ such that $ca\le 1 + \frac1{2^n}$, and there is $c'\in L$ such that $c'a\ge 1  - \frac{1}{2^n}$. 

(Obviously this suffices.)

\begin{proof}[Proof of Claim 2.]
Choose $m$ such that $2^m>a$. Put $S := \{ r\in\N \,:\, (\frac{r}{2^{m+n}})a>1 \}$. Then $S\ne\emptyset$; take $s = \min S$. Then $\frac{s}{2^{m+n}}\in R$, but $(\frac{s-1}{2^{m+n}})a\le 1$. So
\[ \frac{s}{2^{m+n}} a \le 1 + \frac{a}{2^m2^n} \le 1 + \frac1{2^n}.\qedhere \]
\end{proof}
This completes the proof.
\end{proof}

%\end{document}
\NotesBy{Notes by Madeline Barnicle}
\Week{ Week 4 }
\Day{Monday, October 27, 2014}

We need to show that the ordered field of reals in $\mathbf{No}$ is Dedekind-complete.

Let $S \neq \emptyset$ be a set of reals which is bounded from above. We need to show sup $S$ exists. We can assume $S$ has no maximum. Put $R=\{a \in \mathbb{D}: a>S\}, L=\mathbb{D} \setminus R$. Using (3.11), the density of $\mathbb{D}$, $L$ has no maximum. If $R$ has a minimum, then this value is the sup of $S$ and we are done. So suppose $R$ has a minimum, then $r=\{L|R\}$ is real.

Claim: $r=$ sup $S$. Suppose $s \in S$ satisfies $s>r$, take $d \in \mathbb{D}$ with $s>d>r$. Then $d \in L$, contradiction since $d>r$.

So $r$ is an upper bound. If $L<r'<r$, $r'$ also real, take dyadic $d$ with $r'<d<r$. Then $d \in R$, contradiction.

\begin{corollary} % ====Corollary 3.13==== 
Corollary to \eqref{3.8} and \eqref{3.12}: Let $a \in \mathbb{R} \setminus \mathbb{D}$. Then $\lim_{n \to +\infty} a \restriction n =a$ in $\mathbb{R}$.
 \end{corollary}

\textbf{Ordinals}

We identify each $n \in \mathbf{On}$ wih $a_\alpha \in \mathbf{No}$, where $l(a_\alpha)=\alpha$ and $a_{\alpha}(\beta)=+$ (for all $\beta \leq \alpha$?) For all $\alpha, \beta \in \mathbf{On}, \alpha \leq \beta \rightarrow a_\alpha <\leq a_\beta$.

The canonical representation of $a_\alpha$ is $a_\alpha = \{a_{\beta}, \beta \leq \alpha | \emptyset \}$.

If $H$ is any nonempty set of ordinals and $\alpha =$ sup$ H$, then $a_\alpha = \{a_{\beta}, \beta \in H | \emptyset \}$.

\begin{proposition} % ====Proposition 3.14==== 
For all $\alpha, \beta \in \mathbf{On}, a_\alpha + a_\beta = a_{\alpha \oplus \beta}, a_\alpha * a_\beta = a_{\alpha \otimes \beta}$.
 \end{proposition}


\begin{proof} %\WikiBold{Proof:} 

By induction, $a_\alpha + a_\beta =$ sup $\{a_\alpha + a_{\beta'}, a_{\alpha'} + a_\beta | \alpha' \leq \alpha, \beta' \leq \beta \}  =$ by induction, sup $\{a_{\alpha' \oplus \beta}, a_{\alpha \oplus \beta'} | \alpha' \leq \alpha, \beta' \leq \beta \} = a_{\alpha \oplus \beta}$, stated earlier.

For multiplication, using the result for addition, and the distributive laws for $+, \centerdot,$ on $\mathbf{No}$ and $\oplus, \otimes$ on $\mathbf{On}$, we reduce to the case $\alpha = \omega^r, \beta= \omega^s, r, s \in \mathbf{On}$. By definition, $a_\alpha * a_\beta = \{a_{\omega^r} * a_\delta + a_{\gamma}*a_{\omega^s} - a_{\delta}*a_{\gamma} : \delta \leq \omega^s, \gamma \leq \omega^r | \emptyset \}.$ In particular, $a_\alpha * a_\beta \in \mathbf{On}$ by (1.8). By inductive hypothesis and the first part, $a_{\omega^r} * a_\delta + a_{\gamma}*a_{\omega^s} - a_{\delta}*a_{\gamma} \leq a_{(\omega^r \otimes \delta) \oplus (\delta \otimes \omega^s)} < a_{\omega^{r \oplus s}}.$

And $l(a_\alpha * a_\beta) \leq l(\omega^{r \oplus s}) = \omega^{r \oplus s}$.

Now we show the $\geq$ direction.

Let $r'<r$ and $n \in \mathbb{N}$, then $a_\alpha * a_\beta > a_{\omega^{r'}_n} * a_{\beta}=a_{\omega^{r'}_n} \otimes \beta$ by hypothesis $=a_{\omega^{r' \oplus s}_n}$. Similarly if $s'<s, n \in \mathbb{N}$, then $a_\alpha * a_\beta > a_{\omega^{r \oplus s'}_n}$. Therefore, $a_\alpha * a_\beta \geq$ sup $\{a_{\omega^{r' \oplus s}_n}, a_{\omega^{r \oplus s'}_n}  : r' <r, s' <s, n \in \mathbb{N} \} = a_{\omega^{r+s}}$. (See the definition of natural $\oplus$.)
 \end{proof}

====Examples====

\begin{enumerate}
  \item  $\omega -1 = \omega + (-1) = \{n|\emptyset \} + \{\emptyset|0\}=\{n-1|\omega + 0\}= \{n|\omega\} = (+++...-). (\omega$ +s).
  \item  $\omega - (m+1)$? By hypothesis, assume $w-m = (+++...---...-)$ ($\omega$ +s and $m$ -s). $\omega - (m+1)= \{n|\emptyset \} + \{\emptyset|-m\} = \{n-m-1|\omega -m\} = \{n|\omega-m\} = (+++...---...--)$ ($\omega$ +s and $m+1$ -s).
  \item  $\omega + \frac{1}{2} = \{n|\emptyset \} + \{0|1\} = \{n + \frac{1}{2}, \omega + 0 | \omega +1\} = \{\omega | \omega +1\}=(+++...+-)$ ($\omega$ +s.)
  \item  In general, if $r \in \mathbb{R}^{>0}, \omega + r = \{n|\emptyset \} + \{L|R\}$ (the canonical representation for $r$) = $\omega + L, n+r | \omega +R\} = \{\omega + L | \omega + R \}$. Inducting on the length of $r$, this equals $\omega \frown r$. This also works for negative, non-integer $r$ (we need $L$ to be nonempty for the argument to go through)--the full result holds for $r \in \mathbb{Z}^{<0}$ as well.
\end{enumerate}

\WikiLevelThree{Wednesday, October 29, 2014}
\begin{enumerate}
  \item  $\frac{1}{2} \omega = \{0|1\}*\{n|\emptyset \}=$ by the definition of multiplication, $\{\frac{1}{2} n + \omega * 0 - n*0 | \frac{1}{2} n + \omega * 1 - n*1\}=\{\frac{1}{2} n|\omega -\frac{1}{2} n\}=$ by cofinality $\{n|\omega -n\}= (+++...---...)$ ($\omega$ +s, $\omega$ -s).
  \item  $\frac{1}{\omega}= (+---...)$ ($\omega$ -s)? Call this number $\epsilon$. $0<\epsilon <r$ for all $r \in \mathbb{R}^{>0}$, i.e. $\epsilon$ is ''infinitesimal''. (It is the unique infinitesimal of length $\omega$.) Guess that $\epsilon = \frac{1}{\omega}$. Canonical representation of $\epsilon: \{0|\frac{1}{2^n}\}$.
\end{enumerate}

$\epsilon \omega = \{0|\frac{1}{2^n}\}*\{m|\emptyset \}=\{0|\frac{1}{n}\}*\{m|\emptyset \}$ by cofinality $=\{\epsilon *m +0*0-0*\omega|\epsilon *m+\frac{1}{n}\omega - \frac{1}{n}m\}= \{\epsilon *m|\epsilon *m+\frac{1}{n}\omega - \frac{1}{n}m\}.$

Now $\epsilon *m$ is still infinitesimal, $\epsilon m <1.   1< \epsilon *m+\frac{1}{n}\omega - \frac{1}{n}m$, as $\frac{1}{n}(\omega -n)$ is infinite. Since $0$ does not satisfy the cut, $\epsilon \omega =1$.
\begin{enumerate}
  \item $r + \epsilon (r \in \mathbb{R})$. First assume $r \in \mathbb{D}$, so $r=\{r_L|r_R\}, r_L, r_R \in \mathbb{D}$. $r + \epsilon = \{r_L + r_R\} + \{0|\frac{1}{n}\}=\{r+0, r_L + \epsilon|r+\frac{1}{n}, r_R + \epsilon\}= \{r|r+\frac{1}{n}\}$ by cofinality = $\{r|\mathbb{D}^{>r}\}=r\frown(+)$, of length $\omega +1$.
  \item What is $\lambda +r$, $\lambda$ a limit ordinal, $r \in \mathbb{R}$?
\end{enumerate}

\section{Combinatorics of Ordered Sets}
Let $S$ be a set. An \WikiItalic{ordering} $\leq$ on $S$ is an reflexive, transitive, antisymmetric, binary relation on $S$. Call $(S,\leq)$ an \WikiItalic{ordered set} (partial or total).

We say $\leq$ is \WikiItalic{total} if $x \leq y$ or $y \leq x$ for each $x, y \in S$.

Let $T$ be an ordered set, and $\phi: S \rightarrow T$ a map. Then $\phi$ is \WikiItalic{increasing} if $x \leq y \rightarrow \phi(x) \leq \phi(y)$, \WikiItalic{strictly increasing} if $x < y \rightarrow \phi(x) < \phi(y)$, a \WikiItalic{quasi-embedding} if $\phi(x) \leq \phi(y) \rightarrow x \leq y$.

Examples: let $(S, \leq_S), (T, \leq_T)$ be ordered sets.
\begin{enumerate}
  \item $S \coprod T$= disjoint union of $S$ and $T$ with the ordering $\leq_S \cup \leq_T$.
  \item $S \times T$ can be equipped with the ''product ordering'' $(x,y)\leq(x',y') \leftrightarrow x \leq_S x'$ and $y \leq_T y'$, or the ''lexicographic ordering'' $(x,y) \leq_{lex} (x',y') \leftrightarrow x <_S x'$ or $(x=x'$ and $y \leq_T y')$. The lexicographic ordering extends the product ordering.
  \item Let $S^*$ be the set of finite words on $S$. Define $x_{1}...x_{m} \leq^{*} y_{1}...y_{m}$ if there exists a strictly increasing $\phi: \{1...m\} \rightarrow \{1...n\}$ such that $x_i \leq_S y_{\phi(i)}$ for every $i=1...m$.
  \item Let $S^\diamond$ be the set of "commutative" finite words on $S=S*/ \sim$, where $x_{1}...x_{m} \sim y_{1}...y_{m}$ if $m=n$ and there exists a permutation of $\{1...m\}$ such that $x_i = y_{\phi(i)}$ for all $i$. Define  $x_{1}...x_{m} \leq^{\diamond} y_{1}...y_{m} \leftrightarrow$ there exists an injective $\phi: \{1...m\} \rightarrow \{1...n\}$ such that $x_i \leq_S y_{\phi(i)}$ for $i=1...m$.
  \item The natural surjective map $(S^*, \leq^*) \rightarrow (S^\diamond, \leq^\diamond)$ is increasing.
  \item $\mathbb{N}^m=\mathbb{N} * \mathbb{N} * ...\mathbb{N}$ with the product ordering. $X=\{x_1...x_m\}$ distinct indeterminates with trivial ordering. The map $\mathbb{N}^m= \rightarrow X^\diamond, \nu(v_1...v_n)=X_{1}^{v_1}...X_{m}^{v_m}$ is an isomorphism of ordered sets. $\leq^\diamond$ is divisibility of monomials.
\end{enumerate}

\WikiSigleStar Let $S$ be an ordered set. Call $F \subset S$ a ''final segment'' of $S$ if $x \leq y$ and $x \in F \rightarrow y \in F$ ($F$ is upward closed). Given $X \subset S$, define $(X) \subset F = \{y \in S|\exists x \in X, x \leq y\},$ the final segment of $S$ ''generated'' by $X$ (the notation corresponds to the ideal generated by monomials). Put $\mathcal{F}(S)=$ the set of all finite segments of $S$. Call $A \subset S$ an ''antichain'' if for $x, y \in A, x \neq y \rightarrow x \not\leq y$ and $y \not\leq x$. We say that $S$ is ''well-founded'' if there is no infinite sequence $x_{1}>x_{2}...$ in $S$.

\begin{definition} % ====Definition 4.1==== 
$(S, \leq)$ is \WikiItalic{noetherian} if it is well-founded and has no infinite antichains.
 \end{definition}


\WikiLevelThree{Friday, October 31, 2014}
Call an infinite sequence $(x_n)$ in $S$ \WikiItalic{good} if $x_i \leq x_j$ for some $i<j$.
\begin{proposition} % ====Proposition 4.2====  
The following are equivalent:
\begin{enumerate}
	\item  $S$ is Noetherian.
	\item  Every final segment of $S$ is finitely generated.
	\item  $\mathcal{F}(S)$ has the ascending chain condition.
	\item  Every infinite sequence in $S$ contains an increasing finite subsequence.
	\item  Every infinite sequence in $S$ is good.
\end{enumerate}
 \end{proposition}

\begin{proof} %\WikiBold{Proof:} 
$1 \rightarrow 2$: Let $F \in \mathcal{F}(S)$. Let $G$ be the set of minimal elements of $F$. Then $G$ is an antichain, hence finite. Suppose $x_i \in F \setminus (G)$, there is $x_2 \in F$ such that $x_1 > x_2, x_2 \ni (G)$. This yields an infinite sequence $x_1 > x_2...$, a contradiction.

$2 \leftrightarrow 3$ is a standard argument.

$3 \rightarrow 4$: Let $(x_i)$ be a sequence in $S$. We define a subsequence $x_{n_k}$ such that $x_{n_k} \leq x_{n_{k+1}} \forall k$, and there are infinitely many $n > n_k$ such that $x_n > x_{n_k}$. We let $n_1 = 1$. Inductively, suppose we have already defined $n_1...n_k$. Then the final segment $(x_n: n > n_k, x_n \geq x_{n_k})$ is finitely generated, so there is some $x_{n_{k+1}}$ such that for infinitely many $n>k$ with $x_n > x_{n_k}$ we have $x_n > x_{n_{k+1}}$.

$4 \rightarrow 5$ is obvious, as is $5 \rightarrow 1$.
 \end{proof}

\begin{corollary} % ====Corollary 4.3==== 
\begin{enumerate}
	\item If there exists an increasing surjection $S \twoheadrightarrow T$ and $S$ is noetherian, then $T$ is noetherian. (Use condition 5.)
	\item If there exists a quasi-embedding $S \rightarrow T$ and $T$ is noetherian, then $S$ is noetherian.
	\item If $S, T$ are noetherian, then so are $S \coprod T$ and $S \times T$. (For the product case, take an infinite sequence and apply condition 4 to each component.)
\end{enumerate}
 \end{corollary}

In particular, $\mathbb{N}^m$ is noetherian (``Dickson's Lemma'').
\begin{theorem} % ====Theorem 4.4 (Higman)==== 
If $S$ is noetherian, then $S^*$ is noetherian, (and then so is $S^\diamond$ by (4.3 (1))).
 \end{theorem}

\begin{proof}  %\WikiBold{Proof} (Nash-Williams)
 
 (Nash-Williams)

Suppose $w_1, w_2...$ is a bad sequence for $\leq^*$. We may assume that each $w_n$ is of minimal length under the condition $w_n \ni (w_1...w_{n-1})$. Then $w_n \neq \epsilon$ (the empty word) for any $n$. So $w_n=s_{n}*u_{n}, s_n \in S, u_n  \in S^*$ (splitting off the first letter). Since S is noetherian, we can take an infinite subsequence $s_{n_1} \leq s_{n_2}...$ of $s_n$. By minimality, $w_1...w_{n_{1}-1}, u_{n_1}, u_{n_2}...$ is good. So $\exists i<j$ such that $u_{n_i} \leq^* u_{n_j}$. Then $s_{n_i}u_{n_i} \leq s_{n_j}u_{n_j}$, that is, $w_{n_i} \leq w_{n_j}$, contradiction.
\end{proof}

Remark: suppose $\phi: S \rightarrow T$ is a \WikiItalic{strictly} increasing map of ordered sets, where $S$ is noetherian. Then $\phi$ has finite fibers.

Application: Let $\Gamma=(\Gamma, \leq, +)$ be a totally ordered abelian group.

\begin{corollary} % ====Corollary 4.5==== 
Let $A, B$ be subsets of $\Gamma$ which are well-ordered. Then $A \cup B$ is well-ordered, and the set $A+B=\{\alpha + \beta | \alpha \in A, \beta \in B\}$ is well-ordered, and for every $\gamma \in A+B$, there are only finitely many $(\alpha, \beta) \in A \times B$ with $\gamma = \alpha + \beta$.
 \end{corollary}

See (4.3) (3) and take increasing maps from $A \times B \rightarrow (A+B)$.

\begin{corollary} % ====Corollary 4.6==== 
Let $A \subset \Gamma^{>0}$ be well-ordered. Then $<A>=\{\alpha_1+...\alpha_n: \alpha_i \in A\}$ is also well-ordered, and for each $\gamma \in <A>$ there are only finitely many $(n, \alpha_1...\alpha_n)$ with $n \in \mathbb{N}, \alpha_1...\alpha_n \in A$ such that $\gamma=\alpha_1+...\alpha_n$.
 \end{corollary}

\begin{proof} %\WikiBold{Proof}:
We have the map $A^\diamond \rightarrow <A>, \alpha_1...\alpha_n \rightarrow \alpha_1+...\alpha_n$, onto and strictly increasing since all $\alpha_i >0$. Now use Higman's Theorem.
\end{proof}

\WikiLevelFour{Hahn Fields}

Let $k$ be a field, $\Gamma$ an ordered abelian group. Define $K=k((t^\Gamma))$ to be the set of all formal series $f(t)=\sum_{\gamma \in \Gamma} f_\gamma t^\gamma (f_\gamma \in k)$, whose \WikiItalic{support} supp$f=\{\gamma \in \Gamma: f_\gamma \neq 0\}$ is \WikiItalic{well-ordered}. Using (4.5), we can now define $f+g=\sum_{\gamma} (f_\gamma + g_\gamma)t^\gamma,$ and $f*g= \sum_{\gamma} (\sum_{\alpha+\beta=\gamma} f_\alpha * g_\beta)t^\gamma$. $K$ is an integral domain, and $k$ includes into $K$ by $a \rightarrow a*t^0$. We will show $K$ is actually a field.

\WikiLevelTwo{ Week 5 }

\WikiLevelThree{November 3, 2014 }
We will now show that the Hahn field $K$ defined last time is in fact a field. Define $v:K\setminus \{0\}\rightarrow \Gamma$ by $$vF:=\min \mathrm{supp}(f).$$ We will use a profound but simple remark about the structure of $K$ due (probably?) to Neumann: if $vF>0$ (i.e., $\mathrm{supp}(f)\subseteq \Gamma^{>0}$), then $\sum_{n=0}^\infty f^n$ makes sense as an element of $K$. Why? By (4.6), for any $\gamma\in \Gamma$ there are finitely many $n$ such that $\gamma\in\mathrm{supp}(f^n)$. By definition of multiplication in $K$, any $\gamma\in \mathrm{supp}(f^n)$ is an element of $\langle \mathrm{supp}(f)\rangle$. By the second part of (4.6), there are only finitely many such $n$.

Now write an arbitrary $g\in K\setminus \{0\}$ as $g=ct^\gamma(1-f)$ where $c\in K\setminus \{0\}$, $\gamma\in \Gamma$, and $vF=0$ (this is achieved by taking $\gamma$ to be $\min\mathrm{supp}(g)$). Then it can be checked that $g^{-1}=c^{-1}t^{-\gamma}\sum_{n=0}^\infty f^n$ works.

\section{ The $\omega^-$ map }
Let $(\Gamma, \le, +)$ be an ordered abelian group. Set $|\gamma|:=\max\{\gamma, -\gamma\}$ for $\gamma\in \Gamma$. We say $\alpha,\beta$ are \WikiItalic{archimedean equivalent} if there is some $n\ge 1$ such that $|\alpha|\le n|\beta|$ and $|\beta|\le n|\alpha|$. This is an equivalence relation on $\Gamma$, we write $\alpha\sim\beta$. Denote the equivalence classes $[\alpha]=\{\beta\in\Gamma:\beta\sim\alpha\}$. These equivalence classes $[\Gamma]=\{[\gamma]:\gamma\in\Gamma\}$ can be ordered by 
$$[\alpha]<[\beta]\quad \textrm{if and only if} \quad \textrm{for all }n\ge 1, n|\alpha|<|\beta|.$$
We also write $\alpha\ll\beta$ or $\alpha=o(\beta)$ instead of $[\alpha]<[\beta]$. This defines a total order on $[\Gamma]$ with smallest element $[0]$.

This equivalence relation gives a coarse view of the ordered abelian group $\Gamma$.

Some properties:
\begin{enumerate}
  \item  $[-\alpha]=[\alpha]$.
  \item  $[\alpha+\beta]\le \max\{[\alpha],[\beta]\}$ and equality holds if $[\alpha]\neq [\beta]$.
  \item  If $0\le \alpha\le \beta$, then $[\alpha]\le [\beta]$.
\end{enumerate}


We say that $\Gamma$ is \WikiItalic{archimedean} if $[\Gamma]\setminus \{[0]\}$ is a singleton.
\begin{lemma} % ====Lemma 5.1 (H�lder): ==== 
If $\Gamma$ is archimedean and $\epsilon\in \Gamma^{>0}$, then there is a unique embedding $(\Gamma,\le,+)\rightarrow (\mathbb{R},\le,+)$ with $\epsilon\mapsto 1$.
 \end{lemma}

The proof is easy, using Dedekind cuts.

\begin{lemma} % ====Lemma 5.2:==== 
Let $a\in \mathbf{No}$. Then there is a unique $x\in \mathbf{No}$ of minimal length with $a\sim x$.
 \end{lemma}

\begin{proof} %\WikiBold{Proof:} 
There is certainly an $x\in [a]$ of smallest length. Let $x\neq y$ with $x\sim a\sim y$. Put $z:=x\wedge y$. Then $z\sim x\sim y$ and $l(z)<l(x)$ or $l(z)<l(y)$.
 \end{proof}


For each $b\in\mathbf{No}$, define a "canonical element" $\omega^b$ of "$b$th order of magnitude":

\begin{definition} % ==== Definition 5.3: ==== 
Assuming that $\omega^c$ for $c<_s b$ has been defined already, set 
$$\omega^b:=\{0,r\omega^{b_L}\mid s\omega^{b_R}\}$$
where $r,s$ range over $\mathbb{R}^{>0}$ and $b=\{b_L\mid b_R\}$.
 \end{definition}


\begin{lemma} %==== Lemma 5.4: ====
Suppose $\gamma\in \mathbf{On}$ and $\omega^b$ has been defined for all $b\in \mathbf{No}$ with $\ell(b)<\gamma$ such that:
\begin{enumerate}
  \item  $0<\omega^b$,
  \item  $b<c \Rightarrow \omega^b\ll \omega^c.$
\end{enumerate}
Then (5.3) makes sense for $b\in \mathbf{No}$ with $l(b)\le \gamma$ and $(1),(2)$ above hold for all $b,c\in\mathbf{No}$ with $\ell(b),\ell(c)\le \gamma$. 
\end{lemma}

\begin{proof} %\WikiBold{Proof} 
Suppose $\ell(b)=\gamma$. Then $\ell(b_L),\ell(b_R)<\gamma$, so $\omega^{b_L} \ll \omega^{b_R}$. Therefore $r\omega^{b_L}<s\omega^{b_R}$ for any $r,s\in \mathbb{R}^{>0}$. So $0<s\omega^{b_R}$ for any $s\in \mathbb{R}^{>0}$. Hence $\omega^b$ is defined (i.e., this is actually a cut) and $\omega^b>0$.

Suppose $b<c$ with lengths $\le \gamma$. There are three cases: if $b<_s c$, then $b=c_L$, so $\omega^b\ll\omega^c$ by definition of $\omega^c$. If $c<_s b$, then $c=b_R$, so $\omega^b\ll\omega^c$ by definition of $\omega^b$. Otherwise, $b<b\wedge c<c$ and then by the two cases discussed before,
$\omega^b\ll\omega^{b\wedge c}\ll\omega^c$.
 \end{proof}


When it comes to surreal numbers, it seems that a good definition has invariance under different representations.
\begin{corollary} % ==== Corollary 5.5: ==== 
If $b=\{L\mid R\}$, then $\omega^b=\{0,\mathbb{R}^{>0}\cdot \omega^L\mid \mathbb{R}^{>0}\cdot \omega^R\}$.
 \end{corollary}

The proof is an exercise using cofinality and inverse cofinality theorems.

\begin{lemma} % ====Lemma 5.6: ==== 
Let $a\in \mathbf{No}$. Then $a=\omega^b$ for some $b\in\mathbf{No}$ if and only if $a$ is the element of minimal length of $[a]$.
 \end{lemma}

\begin{proof} %\WikiBold{Proof:} 
Suppose $a=\omega^b=\{0,r\omega^{b_L}\mid s\omega^{b_R}\}$. If $a\sim x$, then $r\omega^{b_L}<x<s\omega^{b_R}$ for some $r,s\in\mathbb{R}^{>0}$ so $a\le_s x$. This finishes one direction of the proof, the other will be done next time.
 \end{proof}

\WikiLevelThree{November 7, 2014}
No class on Wednesday. 

To finish the other direction of Lemma 5.6, we show that each $a\in\mathbf{No}$, $a>0$, is $\sim$ to some element of the form $\omega^b$ by induction on $\ell(a)$. Suppose $a=\{L\mid R\}$, with $0\in L$. By induction hypothesis, every element of $L\cup R$ is $\sim$ to some $\omega^b$. Put
$$F:=\{y\in\mathbf{No}:\exists a_L\in L:a_L\sim \omega^y\}$$
$$G:=\{y\in\mathbf{No}:\exists a_R\in R:a_R\sim \omega^y\}.$$

\WikiItalic{Case 1.} $F\cap G\neq \emptyset$. 

Let $y\in F\cap G$. Then there are $a_L\in L$, $a_R\in R$ with $a_L\sim\omega^y\sim a_R$. Since $0<a_L<a<a_R$, we get $a\sim \omega^y$.

\WikiItalic{Case 2.} $F\cap G=\emptyset$. 

It is easy to see that $F<G$. Now put $z:=\{F\mid G\}$. Note that $F$ and $G$ are sets because the different $\omega^y$ are in different archimedean classes. We distinguish three cases:

\WikiBold{Case 2a.} $[\omega^x]\ge [a]$ for some $x\in F$.

Take $a_L\in L$ with $\omega^X\sim a_L$. Then $[a]=[\omega^x]$.

\WikiBold{Case 2b.} $[\omega^x]\le [a]$ for some $x\in G$.

Then also $[a]=[\omega^x]$.

\WikiBold{Case 2c.} $[\omega^x]<[a]<[\omega^y]$ for all $x\in F$, $y\in G$.

We claim that $a=\omega^z$. To show this we prove that $(\{0\}\cup \mathbb{R}^> \omega^F, \mathbb{R}^>\omega^G)$ is cofinal in $(L,R)$. Let $a_L \in L\setminus \{0\}$. Then by induction hypothesis, there is $x\in F$ with $a_L\sim \omega^x$. So there is some $r\in \mathbb{R}^>$ such that $r\omega^x\ge a_L$. Similarly for each $a_R\in R$ there is some $y\in G$ and $r\in \mathbb{R}^>$ s.t. $r\omega^y\le a_R$. $\Box$

\begin{lemma} % ====Lemma 5.7:==== 
The map $a\mapsto \omega^a$ is an embedding of ordered groups $(\mathbf{No},+,\le)\hookrightarrow (\mathbf{No}^>,\cdot,\le)$.
 \end{lemma} 

\begin{proof} %\WikiBold{Proof:} 

By definition, $\omega^0=\{0\mid \emptyset\}=1$. By induction on $\ell(a)\oplus \ell(b)$ we show that $\omega^a\omega^b=\omega^{a+b}$.

Let $a=\{a_L\mid a_R\}$, $b=\{b_L\mid b_R\}$, so $\omega^a=\{0,r\omega^{a_L}\mid s\omega^{a_R}\}$ and $\omega^b=\{0,r'\omega^{b_L}\mid s'\omega^{b_R}\}$, where $r,s,r',s'$ range over $\mathbb{R}^>$. Then 
$$\omega^{a+b}=\{0,r\omega^{a_L+b},r'\omega^{a+b_L}\mid s\omega^{a_R+b}, s'\omega^{a+b_R}\}.$$
Using the definition of multiplication and induction hypothesis,
$\omega^a\cdot\omega^b$ has left part
$$\{0,r\omega^{a_L+b}, r'\omega^{b_L+a},r\omega^{a_L+b}+r'\omega^{b_L+a}-rr'\omega^{a_L+b_L},s\omega^{a_R+b}+s'\omega^{b_R+a}-ss'\omega^{a_R+b_R}\}$$
and right part
$$s\omega^{a_R+b},s'\omega^{b_R+a},r\omega^{a_L+b}+s'\omega^{b_R+a}-rs'\omega^{a_L+b_R},s\omega^{a_R+b}+r'\omega^{b_L+a}-r's\omega^{a_R+b_L}\}$$

We will show that these two cuts are mutually cofinal, proving the lemma. Note that the elements defining $\omega^{a+b}$ also appear in the cut for $\omega^a\omega^b$.

Also, 
\begin{enumerate}
  \item  $r\omega^{a_L+b}+r'\omega^{b_L+a}-rr'\omega^{a_L+b_L}\le (r+r')\omega^{\max(a_L+b,a+b_L)}$ (a term appearing in the cut for $\omega^{a+b}$).
  \item  $s\omega^{a_R+b}+s'\omega^{a+b_R}-ss'\omega^{a_R+b_R}<0$. (The third term  of the LHS has the highest archimedean class.)
  \item  $r\omega^{a_L+b}+s'\omega^{b_R+a}-rs'\omega^{a_L+b_R}>s''\omega^{b_R+a}$ for any $s''\in\mathbb{R}$ with $0<s''<s'$. (The second term of the LHS has the highest archimedean class.)
  \item  $s\omega^{a_R+b}+r'\omega^{b_L+a}-r's\omega^{a_R+b_L}>s''\omega^{a_R+b}$ for any $0<s''<s$. (The first term of the LHS has the highest archimedean class.)
\end{enumerate} 
\end{proof} 

That was messy, but nice because it all fell out of the definition of the $\omega^-$ map.

We now check that ordinal exponentiation agrees with the $\omega^-$ map.

\begin{lemma} % ====Lemma 5.8:==== 
For $a\in\mathbf{On}$, $\omega^a\in\mathbf{No}$ is the same as the ordinal $\omega^a$ (ordinal exponentiation).
 \end{lemma}

\begin{proof} %\WikiBold{Proof:} 
Write $\omega\uparrow a$ for ordinal exponentiation. By induction on $a\in \mathbf{On}$, we show that $\omega^a=\omega\uparrow a$. The base case was already done. Let $a=\{a_L\mid \emptyset\}$. Then using the induction hypothesis,
$$\omega^a=\{0,r\omega^{a_L}\mid \emptyset\}=\{0,r\omega\uparrow a_L\mid\emptyset\}$$
$$=\omega\uparrow a,$$
using the definition of ordinal exponentiation for the last equality.
 \end{proof}

\section{Section 6. The Normal Form}

Let $K=\mathbb{R}((t^\mathbb{No}))$ be the Hahn field and set $x:=\frac{1}{t}$. We think of the elements of $K$ as formal series in $x$:
$$f(x)=\sum_{i<\alpha}f_ix^{a_i}$$
where $\alpha\in \mathbf{On}$, $(a_i)$ is a strictly decreasing sequence in $\mathbf{No}$, and $f_i\in \mathbb{R}\setminus \{0\}$. So $\alpha$ is the order-type of $\mathrm{supp}(f)$, which we will denote by $\ell(f)$ (the agreement of the choice of this notation with the length of a surreal number is not a coincidence). We turn $K$ into an ordered field such that $f>0$ iff $f\neq 0$ and $f_0>0$. We will define an ordered field isomorphism between $K$ and $\mathbf{No}$, which will give a normal form for elements of $\mathbf{No}$ generalizing the Cantor normal form.

\WikiLevelTwo{ Week 6 }

Notes by Anton Bobkov

\WikiLevelThree{Monday, November 10, 2014}
We define a map which will eventually be proven to be an ordered field isomorphism.

\begin{align*}
  K = \R((t^\No)) \overset{\sim}{\longrightarrow} \No
\end{align*}

We have an element written as 
\begin{align*}
	&f = \sum_{\gamma \in \No} f_\gamma t^\gamma \\
	&\supp(f) = \{\gamma \colon f_\gamma \neq 0\}
\end{align*}
where $\supp(f)$ is a well-ordered sub''set''. Now let $x = t^{-1}$ and write
\begin{align*}
  f(x) = \sum_{i < \alpha} f_i x^{a_i}
\end{align*}
where $(a_i)_{i<\alpha}$ is strictly decreasing in $\No$, $\alpha$ ordinal and $f_i \in \R$ for $i < \alpha$. Also define $l(f(x))$ to be the order type of $\supp(f)$ (which may be smaller than $\alpha$ as we allow zero coefficients).

==== Question ====
	What is the relationship of what we are going to do with Kaplansky's results from valuation theory?

==== Definition/Theorem ====
For $f(x) = \sum_{i < \alpha} f_i x^{a_i}$ define $\sum_{i < \alpha} f_i \w^{a_i} = f(\omega)$ recursively on $\alpha$: \\ 
When $\alpha = \beta + 1$ is a successor:
\begin{align*}
	\sum_{i < \alpha} f_i \w^{a_i} = \paren{\sum_{i < \beta} f_i \omega^{a_i}} + f_\beta \w^{a_\beta}
\end{align*}
When $\alpha$ is a limit ordinal:
\begin{align*}
	\sum_{i < \alpha} f_i \w^{a_i} &= \curly{L \mid R} \\
  L_f &= \curly{\sum_{i < \beta} f_i \w^{a_i} + (f_\beta - \epsilon) \w^{a_\beta}
	\colon \beta < \alpha, \epsilon \in \R^{>0}} \\
  R_f &= \curly{\sum_{i < \beta} f_i \w^{a_i} + (f_\beta + \epsilon) \w^{a_\beta}
	\colon \beta < \alpha, \epsilon \in \R^{>0}}
\end{align*}
Simultaneously with this definition we prove the following statements by induction:

 '''Inequality:''' For \begin{align*}
	g(x) &= \sum_{i < \alpha} g_i x^{a_i}
\end{align*} we have $f(x) > g(x) \Rightarrow f(\w) > g(\w)$
 '''Inequality:''' For \begin{align*}
 '''Tail property:''' if $\gamma < \kappa < \alpha$
	\left| \sum_{i < \alpha} f_i \w^{a_i} - \sum_{i < \kappa} f_i \w^{a_i} \right| << \w^{a_\gamma}
\end{align*}

'''Proof of inequality'''

Suppose we have
\begin{align*}
  f(x) &= \sum_{i < \alpha} f_i x^{a_i} \\
  g(x) &= \sum_{i < \alpha} g_i x^{a_i}
\end{align*}

with $f(x) < g(x)$

Choose $\gamma$ smallest such that $f_\gamma \neq g_\gamma$.
It has to be that $f_\gamma > g_\gamma$. Also $f(x)\midr\gamma = g(x)\midr\gamma$

''Case 1'': $\alpha = \beta + 1$

\begin{align*}
  f(x) &= f(x)\midr\beta + f_\beta x^{a_\beta}\\
  g(x) &= g(x)\midr\beta + g_\beta x^{a_\beta}
\end{align*}

Suppose $\gamma = \beta$.
Then $\bar f(x) = \bar g(x)$, $\bar f(\w) = \bar g(\w)$, so compute
\begin{align*}
  f(\w) - g(\w) &= \\
	&= f(\w)\midr\beta + f_\beta \w^{a_\beta} - g(\w)\midr\beta - g_\beta \w^{a_\beta} \\
	&= f_\beta \w^{a_\beta} - g_\beta \w^{a_\beta} \\
	&= \paren{f_\beta - g_\beta} \w^{a_\beta} > 0
\end{align*}

Now suppose $\gamma < \beta$.

Group the terms
\begin{align*}
  f(\w) &= h(\w) + f_\gamma \w^{a_\gamma} + f^* + f_\beta \w^{a_\beta} \\
  g(\w) &= h(\w) + g_\gamma \w^{a_\gamma} + g^* + g_\beta \w^{a_\beta}
\end{align*}
where
\begin{align*}
	h(\w) &= f(\w)\midr\gamma = g(\w)\midr\gamma \\
	f^* &= f(\w)\midr\beta - f(\w)\midr{\gamma + 1} \\
	g^* &= g(\w)\midr\beta - g(\w)\midr{\gamma + 1}
\end{align*}

Then we have by tail property $f^* << x^{a_\gamma}$ and $g^* << x^{a_\gamma}$. Compute

\begin{align*}
  f(\w) - g(\w) &= (f_\gamma - g_\gamma) x^{a_\gamma} + (f* - g*) + (f_\beta - g_\beta) x^{a_\beta}
\end{align*}

We have $f_\gamma > g_\gamma$.
All $f*$, $g*$ and $(f_\beta - g_\beta) x^{a_\beta}$ are  $<< x^{a_\gamma}$.
Thus $f(\w) - g(\w) > 0$ as needed.

''Case 2'': $\alpha$ is a limit ordinal.

$f(\w)$ and $g(\w)$ are defined as 

\begin{align*}
  f(\w) &= \curly{L_f \mid R_f} \\
  g(\w) &= \curly{L_g \mid R_g}
\end{align*}

Recall that

\begin{align*}
  L_f &= \curly{\sum_{i < \beta} f_i \w^{a_i} + (f_\beta - \epsilon) \w^{a_\beta}
	\colon \beta < \alpha, \epsilon \in \R^{>0}} \\
  R_g &= \curly{\sum_{i < \beta} g_i \w^{a_i} + (g_\beta + \epsilon) \w^{a_\beta}
	\colon \beta < \alpha, \epsilon \in \R^{>0}}
\end{align*}

Pick any $\beta$ with $\gamma < \beta < \alpha$ and $\epsilon \in \R^{>0}$,
and pick limit elements $\bar f(\w) \in L_f$ and $\bar g(\w) \in R_g$ corresponding to $\beta, \epsilon$.

Then $\bar f(x) < \bar g(x)$ as first coefficient where they differ is $x^{a_\gamma}$ and $f_\gamma > g_\gamma$.
Thus by inductive hypothesis $\bar f(\w) < \bar g(\w)$.
As choice of those was arbitrary we have $L_f < R_g$ so $f(\w) > g(\w)$.

'''Proof of tail property'''

It is easy to see that statement holds for all $\gamma < \kappa < \alpha$ iff it holds for all $\gamma < \kappa \leq \alpha$.

''Case 1'': $\alpha = \beta + 1$.

Suppose we have $\gamma < \kappa < \alpha$, then $\gamma < \kappa \leq \beta$ and induction hypothesis applies.

\begin{align*}
	&\sum_{i < \alpha} f_i \w^{a_i} - \sum_{i < \kappa} f_i \w^{a_i} = \\
	&\brac{\sum_{i < \beta} f_i \w^{a_i} - \sum_{i < \kappa} f_i \w^{a_i}} + f_\alpha \w^{a_\alpha}
\end{align*}

Expression $\brac{\ldots}$ is $<< \w^{a_\gamma}$ by induction hypothesis. $f_\alpha \w^{a_\alpha} << \w^{a_\gamma}$ as $a_\alpha < a_\gamma$. Thus the entire sum is $<< \w^{a_\gamma}$ as needed.

''Case 2'': $\alpha$ is a limit ordinal.

Write definitions of $f(\w)$ using $\kappa$

\begin{align*}
  f(\w) &= \curly{L_f \mid R_f} \\
  F(\w) &= f(\w)\midr\kappa = \sum_{i < \kappa} f_i \w^{a_i}
\end{align*}

\begin{align*}
  L_f &= \curly{\sum_{i < \beta} f_i \w^{a_i} + (f_\beta - \epsilon) \w^{a_\beta}
	\colon \beta < \alpha, \epsilon \in \R^{>0}} \\
  R_f &= \curly{\sum_{i < \beta} f_i \w^{a_i} + (f_\beta + \epsilon) \w^{a_\beta}
	\colon \beta < \alpha, \epsilon \in \R^{>0}} \\
\end{align*}

Pick any $\beta$ with $\kappa < \beta < \alpha$ and $\epsilon \in \R^{>0}$,
and pick limit elements $\bar l(\w) \in L_f$ and $\bar r(\w) \in R_f$ corresponding to $\beta, \epsilon$.

By induction hypothesis we have 
\begin{align*}
	\bar l(\w) - F(\w) &= \bar l(\w) - \bar l(\w)\midr\kappa \ <<  \w^{a_\kappa} \\
	\bar r(\w) - F(\w) &= \bar r(\w) - \bar r(\w)\midr\kappa \ <<  \w^{a_\kappa}
\end{align*}

\begin{align*}
	l(\w) \leq f(\w) \leq r(\w) \\
\end{align*}
\begin{align*}
	l(\w) - F(\w) \leq f(\w) - F(\w) \leq r(\w) - F(\w)
\end{align*}
 
Thus $f(\w) - F(\w) << \w^{a_\kappa}$ as it is between two elements that are $<<  \w^{a_\kappa}$.

'''Proof of well-definiteness'''

We also need to check that the function is well-defined. For $f(x)$ define its reduced form, where we only keep non-zero coefficients.

\begin{align*}
	f(x) &= \sum_{i < \alpha} f_i \w^{a_i} \\
	\bar f(x) &= \sum_{j < \alpha'} f_j' \w^{a_j'}
\end{align*}

We need to check that $f(\w) = \bar f(\w)$

''Case 1'': $\alpha = \beta + 1$

\begin{align*}
	f(\w) &= g(\w) + f_\beta \w^{a_\beta} \\
	g(x) &= \sum_{i < \beta} f_i \w^{a_i} \\
	g(\w) &= \bar g(\w)
\end{align*}

If $f_\beta = 0$ then $\bar f(x) = \bar g(x)$ and $f(\w) = g(\w)$ so $f(\w) = g(\w) = \bar g(\w) = \bar f(\w)$ as needed.

Suppose $f_\beta \neq 0$. Then $\bar f(x) = \bar g(x) + f_\beta x^{a_\beta}$.
\begin{align*}
	f(\w) = g(\w) + f_\beta \w^{a_\beta} = \bar g(\w) + f_\beta \w^{a_\beta} = \bar f(\w)
\end{align*}

''Case 2'': $\alpha$ is a limit and non-zero coefficients are cofinal in $\alpha$.

In this case both $f(x)$ and $\bar f(x)$ have limit ordinals in their definitions. Moreover 
\begin{align*}
	L_{\bar f} &\subseteq L_f \\
	R_{\bar f} &\subseteq R_f
\end{align*}
and are cofinal. Thus $f(\w) = \bar f(\w)$.

''Case 3'': $\alpha$ is a limit and for some $\gamma < \alpha$
\begin{align*}
	\gamma \leq \beta < \alpha \Rightarrow f_\beta = 0
\end{align*}

\begin{align*}
	g(x) &= \sum_{i < \gamma} f_i x^{a_i} \\
  L_f^* &= \curly{\sum_{i < \beta} f_i \w^{a_i} + (f_\beta - \epsilon) \w^{a_\beta}
	\colon \gamma < \beta < \alpha, \epsilon \in \R^{>0}} \\
	&= \curly{g(\w) - \epsilon \w^{a_\beta}
	\colon \gamma < \beta < \alpha, \epsilon \in \R^{>0}} \\
  R_f^* &= \curly{\sum_{i < \beta} f_i \w^{a_i} + (f_\beta + \epsilon) \w^{a_\beta}
	\colon \gamma < \beta < \alpha, \epsilon \in \R^{>0}} \\
	&= \curly{g(\w) + \epsilon \w^{a_\beta}
	\colon \gamma < \beta < \alpha, \epsilon \in \R^{>0}}
\end{align*}

We have
\begin{align*}
	L_f^* &\subseteq L_f \\
	R_f^* &\subseteq R_f
\end{align*}
and are cofinal. $\curly{L_f^* - g(\w) \mid R_f^* - g(\w)} = 0$ as left side is negative and right side is positive.
Thus $f(\w) = \curly{L_f^* \mid R_f^*} = g(\w)$. As $\gamma < \alpha$ this case is covered by induction.

\WikiLevelThree{Wednesday, November 12, 2014}

==== Lemma 6.1 ====
$l(f(\w)) \geq l(f(x))$

'''Proof'''
Suppose $\beta < \alpha$. Then the elements used to define $\sum_{i < \w\cdot\beta} (\ldots)$ also appear in the cut for $\sum_{i < \w\cdot\alpha} (\ldots) = f(\w)$. Thus by uniqueness of the normal form.
\begin{align*}
	l\paren{\sum_{i < \w\cdot\beta} (\ldots)} < l(f(\w))
\end{align*}
So the map $\phi(\beta) = l(\sum_{i < \w\cdot\beta} (\ldots))$ is strictly increasing.
Any strictly increasing map $\phi$ on an initial segment of $\On$ satisfies $\phi(\beta) \geq \beta$.

==== Lemma 6.2 ====
The map

\begin{align*}
	K &\arr \No \\
	f(x) &\mapsto f(\w)
\end{align*}

is onto.

'''Proof'''

Let $a \in \No, a \neq 0$. By (5.6) there is a unique $b \in \No$ such that $\brac{a} = \brac{\w^b}$.
Put
\begin{align*}
	S = \curly{s \in \R \colon s\w^b \leq a}
\end{align*}
Then $S \neq \emptyset$ and bounded from above.
Put $r = \sup S \in \R$.
Then
\begin{align*}
	(r + \epsilon)\w^b > a > (r - \epsilon)\w^b
\end{align*}
for all $\epsilon \in \R^{>0}$
thus
\begin{align*}
	\abs{a - r\w^b} << \w^b \tag{*}
\end{align*}

Note $r \neq 0$; $r,b$ subject to $(*)$ are unique.

We set $\lt(a) = r\w^b$

Towards a contradiction assume that $a$ is not in the image of $f(x) \mapsto f(\w)$.
We shall inductively define a sequence $(a_i, f_i)_{i \in \On}$ where 

\begin{enumerate}
  \item $a_i \in \No$ is strictly decreasing; $f_i \in \R - \{0\}$
  \item $f_\alpha \w^{a_\alpha} = \lt\paren{a - \sum_{i < \alpha} f_i\w^{a_i}}$ for all $\alpha \in \On$
\end{enumerate}
''Case 1'': $\alpha = \beta + 1$

Take $(a_\alpha, f_\alpha)$ so that
\begin{align*}
	f_\alpha\w^{a_\alpha} = \lt\paren{a - \sum_{i < \alpha}(\ldots)}
\end{align*}

By inductive hypothesis, if $\beta < \alpha$

\begin{align*}
	f_\beta\w^{a_\beta} &= \lt\paren{a - \sum_{i < \beta}(\ldots)} \\
	\Rightarrow f_\alpha\w^{a_\alpha} &= \lt\paren{\paren{a - \sum_{i < \beta}(\ldots)} - f_\beta\w^{a_\beta}} \\
	&<< \w^{a_\beta} \text{ by (*)} \\
	\Rightarrow a_\alpha &< a_\beta
\end{align*}

''Case 2'': $\alpha$ limit

Take $(a_\alpha, f_\alpha)$ as above.
Let $\beta < \alpha$; to show $a_\alpha < a_\beta$.
We have

\begin{align*}
	a - \sum_{i \leq \beta} f_i \w^{a_i} = \paren{a - \sum_{i < \beta} f_i\w^{a_i}} - f_\beta\w^{a_\beta} << \w^{a_\beta}
\end{align*}

By the tail property
\begin{align*}
	&\sum_{i < \alpha} (\ldots) - \sum_{i \leq \beta} (\ldots) << \w^{a_\beta} \\
	\Rightarrow &a - \sum_{i < \alpha} (\ldots) << \w^{a_\beta} \\
	\Rightarrow &\lt\paren{a - \sum_{i < \alpha} (\ldots) } << \w^{a_\beta} \\
	\Rightarrow &a_\alpha < a_\beta
\end{align*}

This completes the induction.
Note that we showed that if $\alpha$ is a limit then $a - \sum_{i < \alpha} (\ldots) << \w^{a_\beta}$ for all $\beta < \alpha$.

So if $\curly{L \mid R}$ is the cut used to define $\sum_{i < \alpha} (\ldots)$ then $L < a < R$.
Let $\alpha = \w \cdot \alpha'$.
Hence 
\begin{align*}
	l(a) > l\paren{\sum_{i < \w\cdot\alpha'} (\ldots)} \geq \alpha'
\end{align*}
by (6.1).
So $l(a)$ is bigger than all limits - contradiction.

==== Lemma 6.4 ====
Let $r \in \R, a \in \No$. Then 
\begin{align*}
	r\w^a = \{(r - \epsilon) \w^a \mid (r+\epsilon)\w^a\}
\end{align*}
where $\epsilon$ ranges over $\R^{>0}$.

'''Proof'''
\begin{align*}
	r &= \{r - \epsilon \mid r + \epsilon\} \\
	\w^a &= \curly{0, s\w^{a_L} \mid t\w^{a_R}} \text{ where } s,t \in \R^{>0}
\end{align*}
\begin{align*}
	r\w^a = \{
	&(r - \epsilon) \w^a, (r - \epsilon)\w^a + \epsilon s \w^{a_L}, \\
	&(r + \epsilon) \w^a - \epsilon t \w^{a_R} \mid \\
	&(r + \epsilon) \w^a, (r + \epsilon)\w^a - \epsilon s \w^{a_R}, \\
	&(r - \epsilon) \w^a + \epsilon t \w^{a_L} \}
\end{align*}
Now use $\w^{a_L} << \w^a << \w^{a_R}$ and cofinality.

\WikiLevelThree{Friday, November 14, 2014}

==== Corollary 6.5 ====
\begin{align*}
	\sum_{i \leq \alpha} f_i\w^{a_i} =
	\curly{ \sum_{i < \alpha} f_i\w^{a_i} + (f_i - \epsilon) \w^{a_\alpha} \mid
	\sum_{i < \alpha} f_i\w^{a_i} + (f_i + \epsilon) \w^{a_\alpha} }
\end{align*}

'''Proof'''

''Case 1'': $\alpha$ is a limit
\begin{align*}
	&\sum_{i \leq \alpha} f_i\w^{a_i} = \sum_{i < \alpha} f_i\w^{a_i} + f_\alpha\w^{a_\alpha} = \\ \underset{(6.4)}{=}
	&\curly{\sum_{i < \beta} f_i \w^{a_i} + (f_\beta - \epsilon)\w^{a_\beta} \mid
	\sum_{i < \beta} f_i \w^{a_i} + (f_\beta + \epsilon)\w^{a_\beta}}_{\beta < \alpha, \epsilon \in \R^{>0}}
	+ \curly{(r - \epsilon) \w^\alpha \mid (r+\epsilon)\w^\alpha} = \\ =
	&\curly{\sum_{i \leq \beta} f_i \w^{a_i} - \epsilon\w^{a_\beta} + f_\alpha\w^{a_\alpha},
	\sum_{i < \alpha} f_i \w^{a_i} + (f_\alpha - \epsilon)\w^{a_\alpha} \mid \ldots} = \\ \underset{cofinality}{=}
	&\curly{ \sum_{i < \alpha} f_i\w^{a_i} + (f_i - \epsilon) \w^{a_\alpha} \mid \ldots }
\end{align*}

''Case 2'': $\alpha + 1$

\begin{align*}
	&\sum_{i \leq \alpha + 1} f_i \w^{a_i} = \sum_{i \leq \alpha} f_i \w^{a_i} + f_{\alpha + 1} \w^{a_{\alpha + 1}} = \\
	&\text{(by (6.4) and induction hypothesis)} \\
	= &\curly{\sum_{i \leq \alpha} f_i \w^{a_i} + (f_{\alpha} - \epsilon) \w^{a_\alpha} \mid \ldots} +
	\curly{(f_{\alpha + 1} - \epsilon) \w^{a_{\alpha + 1}} \mid \ldots} =  \\
	= &\curly{\sum_{i < \alpha} f_i \w^{a_i} + (f_{\alpha} - \epsilon) \w^{a_\alpha} + f_{\alpha + 1} \w^{a_{\alpha + 1}},
	\sum_{i \leq \alpha} f_i \w^{a_i} + (f_{\alpha + 1} - \epsilon) \w^{a_{\alpha + 1}} \mid \ldots}
\end{align*}

and again we are done by cofinality.

\NotesBy{Notes by Tyler Arant}
\Week{Week 7}
\Day{11/17, 11/19, 11/21}

\begin{lemma}[Associativity] \label{6.7} Let $\alpha, \beta \in \textbf{On}$, $(a_i)_{i<\alpha+\beta}$ be a strictly decreasing sequence in $\textbf{No}$ and $f_i\in \mathds{R}$ for $i<\alpha+\beta$.  Then,
$$\sum_{i<\alpha+\beta} f_i\omega^{a_i}=\sum_{i<\alpha} f_i\omega^{a_i} + \sum_{j<\beta}f_{\alpha+j}\omega^{a_{\alpha+j}}.$$
\end{lemma}

\begin{proof} We proceed by induction on $\beta$.  In the case that $\beta=\gamma+1$ is a successor ordinal, we have
\begin{align*} \sum_{i<\alpha+(\gamma+1)} f_i\omega^{a_i}&= \sum_{i<\alpha+\gamma} f_i\omega^{a_i} + f_{\alpha+\gamma}\omega^{a_{\alpha+\gamma}} \\
		&= \sum_{i<\alpha} f_i\omega^{a_i}+ \sum_{j<\gamma} f_{\alpha+j} \omega^{a_{\alpha+j}}+ f_{\alpha+\gamma}\omega^{a_{\alpha+\gamma}}\\
		& = \sum_{i<\alpha} f_i\omega^{a_i} + \sum_{j<\gamma+1}f_{\alpha+j}\omega^{a_{\alpha+j}}, \end{align*}
where the first and third equality use the definition of $\sum$ and the second equality uses the induction hypothesis.  

In the case where $\beta$ is a limit ordinal, we let 
$$\{L | R\} = \sum_{j<\beta}f_{\alpha+j}\omega^{a_{\alpha+j}}.$$
Using the definition of addition between surreal numbers and a simple cofinality argument, we obtain
$$\sum_{i<\alpha}f_i\omega^{a_i} + \sum_{j<\beta}f_{\alpha+j}\omega^{a_{\alpha+j}} = \left \{\sum_{i<\alpha}f_i\omega^{a_i} + L \biggl | \sum_{i<\alpha}f_i\omega^{a_i} + R \right \}.$$
A typical element of this cut is 
$$\sum_{i<\alpha}f_i\omega^{a_i} + \sum_{j\leq \gamma} f_{\alpha+j}\omega^{a_{\alpha+j}}-\varepsilon \omega^{a_{\alpha+\gamma}}  \qquad (\gamma<\beta, \varepsilon \in \mathds{R}^{>0}).$$
By inductive hypothesis, this equals
$$\sum_{i<\alpha+\gamma}f_i \omega^{a_i}- \varepsilon \omega^{a_{\alpha+\gamma}}.$$
But these elements are cofinal in the cut defining $\sum_{i<\alpha+\beta} f_i\omega^{a_i}$; hence, the claim follows by cofinality. 
\end{proof}

\begin{proposition}  Let $\alpha \in \textbf{On}$, $(a_i)_{i<\alpha}$ be a strictly decreasing sequence in $\textbf{No}$ and $f_i, g_i\in \mathds{R}$ for $i<\alpha$. Then, 
$$\sum_{i<\alpha}f_i\omega^{a_i} + \sum_{i<\alpha} g_i \omega^{a_i} = \sum_{i<\alpha}(f_i+g_i)\omega^{a_i}.$$
\end{proposition}

\begin{proof} We proceed by induction on $\alpha$.  If $\alpha=\beta+1$ is a successor, then
\begin{align*} \sum_{i<\beta+1}f_i\omega^{a_i} + \sum_{i<\beta+1} g_i \omega^{a_i} & = \left ( \sum_{i<\beta} f_i\omega^{a_i} + f_\beta \omega^{a_\beta} \right ) + \left (  \sum_{i<\beta} g_i \omega^{a_i} + g_\beta \omega^{a_\beta} \right ) \\
	& = \left ( \sum_{i<\beta} f_i\omega^{a_i} + \sum_{i<\beta} g_i \omega^{a_i} \right ) + ( f_\beta \omega^{a_\beta} + g_\beta \omega^{a_\beta}) \\
	& = \sum_{i<\beta} (f_i+g_i) \omega^{a_i} + (f_\beta+g_\beta)\omega^{a_\beta} \\
	& = \sum_{i<\beta+1}(f_i+g_i)\omega^{a_i}, \end{align*}
where the third equality uses the induction hypothesis.

Now suppose $\alpha$ is a limit.  One type of element from the lef-hand-side of the cut defining   $\sum_{i<\alpha}f_i\omega^{a_i} + \sum_{i<\alpha} g_i \omega^{a_i}$ is of the form
$$\sum_{i\leq \beta} f_i \omega^{a_i}-\varepsilon  \omega^{a_\beta} +\sum_{i<\alpha} g_i \omega^{a_i}$$
or of the form
$$\sum_{i<\alpha} f_i \omega^{a_i} +\sum_{i\leq \beta} g_i \omega^{a_i} -\varepsilon  \omega^{a_\beta}.$$
We have
\begin{align*} \sum_{i\leq \beta} f_i \omega^{a_i}-\varepsilon  \omega^{a_\beta} +\sum_{i<\alpha} g_i \omega^{a_i} 
 & = \sum_{i\leq \beta} f_i \omega^{a_i} + \sum_{i\leq \beta} g_i \omega^{a_i} + \sum_{\beta< i<\alpha} g_i \omega^{a_i} -\varepsilon  \omega^{a_\beta} \\
  & = \sum_{i\leq \beta}(f_i+g_i)\omega^{a_i} + \sum_{\beta< i<\alpha} g_i \omega^{a_i} -\varepsilon  \omega^{a_\beta}, \end{align*}
  where the first equality follows from $(\ref{6.7})$ and the second equality uses the inductive hypothesis.  But this is mutually cofinal with 
 $$\sum_{i\leq \beta}(f_i+g_i)\omega^{a_i} - \varepsilon \omega^{a_\beta}.$$
 Similarly if we star with $\sum_{i<\alpha} f_i \omega^{a_i} +\sum_{i\leq \beta} g_i \omega^{a_i} -\varepsilon  \omega^{a_\beta}.$
\end{proof}


\begin{lemma} \label{6.8} Let $\alpha \in \textbf{On}$, $(a_i)_{i<\alpha}$ be a strictly decreasing sequence in $\textbf{No}$, $b\in \textbf{No}$, and $f_i\in \mathds{R}$ for $i<\alpha$. Then,
$$\left ( \sum_{i<\alpha} f_i\omega^{a_i} \right ) \omega^b = \sum_{i<\alpha}f_i\omega^{a_i+b}.$$
Note that the sequence $(a_i+b)_i$ is also strictly decreasing.\end{lemma}

\begin{proof} We proceed by induction on $\alpha$.  If $\alpha=\beta +1$, then
\begin{align*}\left ( \sum_{i<\beta + 1} f_i\omega^{a_i} \right ) \omega^b 
	&= \left ( \sum_{i<\beta} f_i\omega^{a_i} + f_\beta \omega^{a_\beta} \right )\omega^b \\
	& = \left ( \sum_{i<\beta} f_i\omega^{a_i}\right ) \omega^b + f_\beta \omega^{a_\beta}\cdot \omega^b \\
	& = \sum_{i<\beta}f_i\omega^{a_i+b} + f_\beta\omega^{a_\beta+b} \\
	& = \sum_{i<\beta+1}f_i\omega^{a_i+b}, \end{align*}
where the third equality uses the inductive hypothesis.  

Now suppose $\alpha$ is a limit.  Recall that, by their respective definitions,
$$\omega^b=\{0, s\omega^{b'} \ | \ t\omega^{b''}\}$$
and
$$\sum_{i<\alpha}f_i\omega^{a_i} = \left \{ \sum_{i\leq \beta} f_i\omega^{a_i} - \varepsilon \omega^{a_\beta} \ : \ \beta<\alpha, \varepsilon\in \mathds{R}^{>0} \ \biggl | \  \sum_{i\leq \beta} f_i\omega^{a_i} + \varepsilon \omega^{a_\beta} \ : \ \beta<\alpha, \varepsilon\in \mathds{R}^{>0}\right \}.$$
Set $d:=\sum_{i<\alpha}f_i\omega^{a_i}$ and let $d', d''$ be elements from the left and right-hand sides, respectively, of the defining cut determined by the same choice of $\varepsilon$.  Note that
$$d-d' = \varepsilon \omega^{a_\beta} + c', \quad \text{where} \ c'\ll \omega^{a_\beta},$$
and
$$d''-d = \varepsilon \omega^{a_\beta} + c'', \quad \text{where} \ c''\ll \omega^{a_\beta}.$$	
It follows that
\begin{equation}\varepsilon_1\omega^{a_\beta}<d-d', d''-d<\varepsilon_2\omega^{a_\beta}, \quad \text{for all $\varepsilon_1<\varepsilon<\varepsilon_2$ in $\mathds{R}$}, \tag{$*$}\end{equation}
where $\varepsilon$ is given by the choice of $d', d''$.  Now,
\begin{align*} d\omega^b & = \{d' \ | \ d''\} \cdot \{0, s\omega^{b'} \ | \ t\omega^{b''}\} \\
			& = \{d'\omega^b, d' \omega^b+(d-d')s\omega^{b'}, \underline{d''\omega^b-(d''-d)t\omega^{b''}} \ | \\
			& \qquad  d''\omega^b, \underline{d'\omega^b+(d-d')t\omega^{b''}}, d''\omega^b-(d''-d)s\omega^{b'}\},\end{align*}
and we claim that the underlined terms are superfluous; in particular, 
\begin{enumerate}[(1)]
\item $d''\omega^\beta - (d''-d)t\omega^{b''} \leq d'\omega^b + (d-d')s\omega^{b'};$
\item  $d''\omega^b- (d''-d)s\omega^{b'}\leq d'\omega^b + (d-d')t\omega^{b''}$.
\end{enumerate}
To show (1), note that $\omega^{b''}\gg \omega^b\gg \omega^{b'}$ implies
$$(d''-d)t\omega^{b''}+(d-d')s\omega^{b'}\geq \varepsilon_1\omega^{a_\beta}t\omega^{b''} > 2\varepsilon_2 \omega^{a_\beta}\omega^b\geq (d''-d)\omega^b.$$
The verification for (2) is similar.  So, by (1), (2) and confinality, 
$$ d\omega^b =  \{d'\omega^b, d' \omega^b+(d-d')s\omega^{b'}  \ | \
		 d''\omega^b,  d''\omega^b-(d''-d)s\omega^{b'}\}.$$
We claim that we can further simplify this to 
$$d\omega^b=\{d'\omega^b \ | \ d''\omega^b\},$$
then we are done by inductive hypothesis.  Let now $\varepsilon_{1, 2}\in \mathds{R}^{>0}$ with $\varepsilon_1<\varepsilon<\varepsilon_2$ and
$$d_1'= \sum_{i\geq \beta}f_i\omega^{a_i}-\varepsilon_1\omega^{a_\beta}, \quad d_1''=\sum_{i\geq \beta}f_i\omega^{a_i}+\varepsilon_1\omega^{a_\beta}.$$
We claim that 
$$d_1'\omega^b>d'\omega^b+(d-d')s\omega^{b'}, \quad d_1''\omega^b<d''\omega^b-(d''-d)s\omega^{b'}.$$
Notice that the first claim holds if and only if $(d_1'-d')\omega^b>(d-d')s\omega^{b'}$.  But this inequality holds since
$$(d_1'-d)\omega^b= (\varepsilon-\varepsilon_2)\omega^{a_\beta}\omega^b> \varepsilon_2s\omega^{a_\beta}\omega^{b'} \geq (d-d')s\omega^{b'},$$
where the first inequality holds since $\omega^b\gg \omega^{b'}$ and the second inequality holds by $(*)$.  The second part of the claim is proved similarly.
\end{proof}

\begin{proposition} Let $\alpha, \beta \in \textbf{On}$, $(a_i)_{i<\alpha}$, $(b_j)_{j<\beta}$ be strictly decreasing sequences in $\textbf{No}$, and $f_i, g_i\in \mathds{R}$ for $i<\alpha$. Then,
$$\left (\sum_{i<\alpha}f_i\omega^{a_i} \right ) \left ( \sum_{j<\beta}g_j\omega^{b_j} \right ) = \sum_{i<\alpha, j<\beta} f_ig_j \omega^{a_i+b_j}.$$
\end{proposition}

\begin{proof} If either $\alpha$ or $\beta$ are successor ordinals, we verify the proposition by using the inductive hypothesis and lemma $(\ref{6.8})$.  Thus, we only need to consider the case where $\alpha$ and $\beta$ are both limits.  Put 
$$f=\sum_{i<\alpha}f_i X^{a_i}, \quad g=\sum_{j<\beta}g_jX^{a_j}\in K.$$
Recall that the typical element in the cut of $f(\omega)\cdot g(\omega)$ is 
\begin{equation} f(\omega)g(\omega)_{**} + f(\omega)_{*}g(\omega)-f(\omega)_*g(\omega)_{**}, \tag{$\dagger$} \end{equation}
where $*, **$ are either $L$ or $R$.  Moreover, this element is $<f(\omega)g(\omega)$ if and only if $(*, **)=(L, L)$ or $(R, R)$.  Take $f_*, g_{**}\in K$ such that $f_*(\omega)= f(\omega)_*$ and $g_{**}(\omega)=g(\omega)_{**}$.  Then, by inductive hypothesis, $\dagger$ equals
$$(f\cdot g)(\omega) -((f-f_*)(g-g_{**}))(\omega).$$
For example, 
$$f_*=\sum_{i<\gamma}f_iX^{a_i} + (f_\gamma\pm \varepsilon_1)X^{a_\gamma}, \quad \gamma<\alpha$$
implies $f-f_*= \pm \varepsilon_1X^{a_\gamma}+h_1$, where all the terms in $h_1$ have degree $>\gamma$. Similarly, $g-g_{**}= \pm \varepsilon_2X^{b_\delta} + h_2$, where $\delta <\beta$ and all the terms in $h_2$ have degree $>\delta$.  Thus,
$$(f-f_*)(g-g_{**})= \pm \varepsilon_1\varepsilon_2 X^{a_\gamma + b_\delta} + \text{higher order terms},$$
and
$$[(f-f_*)(g-g_{**})](\omega) = \pm \varepsilon_1\varepsilon_2\omega^{\alpha_\gamma+b_\delta} + h_3(\omega),$$
where $h_3(\omega)\ll\omega^{a_\gamma+b_\delta}$.  So by cofinality,
\begin{align*}f(\omega)g(\omega) &=\{ (f\cdot g)(\omega)-\varepsilon\omega^{a_\gamma+b_\delta} \ : \ \gamma<\alpha, \delta < \beta, \varepsilon\in \mathds{R}^{>0} \ | \\  
 &\qquad (f\cdot g)(\omega)+\varepsilon\omega^{a_\gamma+b_\delta} \ : \ \gamma<\alpha, \delta < \beta, \varepsilon\in \mathds{R}^{>0} \}. \end{align*}
 Now,
\begin{align*}(f\cdot g)(\omega) &=\{ (f\cdot g)(\omega)-\varepsilon\omega^{a_\gamma+b_\delta} \ : \ \gamma<\alpha, \delta < \beta \ \text{s.t} \ a_\alpha+b_\delta \in \text{supp}(f\cdot g), \varepsilon\in \mathds{R}^{>0} \ | \\  
 &\qquad f\cdot g(\omega)+\varepsilon\omega^{a_\gamma+b_\delta} \ : \ \gamma<\alpha, \delta < \beta \ \text{s.t} \ a_\alpha+b_\delta \in \text{supp}(f\cdot g),\varepsilon\in \mathds{R}^{>0} \}. \end{align*}
Thus, $(f\cdot g)(\omega)$ satisfies the cut for $f(\omega)\cdot g(\omega)$ and the claim follows by cofinality.  

\end{proof}

All together, this completes the proof of the following theorem.

\begin{theorem} The map
$$\mathds{R}((t^{\bf No})) \xrightarrow{\sim} {\bf No}, \quad \sum_{i<\alpha}f_iX^{a_i} \mapsto \sum_{i<\alpha} f_i\omega^{a_i},$$
is an ordered field isomorphism. \end{theorem}



\section{The Surreals as a Real Closed Field}

Let $K$ be a field. We call $K$ \textit{orderable} if some ordering on $K$ makes it an ordered field.  If $K$ is orderable, then $\text{char}(K)=0$ and $K$ is not algebraically closed. \footnote{To prove that $K$ is not algebraically closed: suppose $K$ is an algebraically closed ordered field and derive a contradiction using $i$, the square root of $-1$.}  We call $K$ \textit{euclidean} if $x^2+y^2\neq -1$ for all $x, y \in K$ and $K=\{\pm x^2 \ : \ x\in K\}$.  If $K$ is euclidean, then $K$ isa an ordered field for a unique ordering---namely, $a\geq 0 \iff \exists x\in K. x^2=a$.  

\begin{theorem}[Artin $\&$ Schreier, 1927] \label {7.1} For a field $K$, the following are equivalent. 
\begin{enumerate}[(1)]
\item $K$ is orderable, but has no proper orderable algebraic field extension. 
\item $K$ is euclidean and every polynomial $p\in K[X]$ of odd degree has a zero in $K$. 
\item $K$ is not algebraically closed, but $K(i)$, $i^2=-1$, is algebraically closed. 
\item $K$ is not algebraically closed, but has an algebraically closed field extension $L$ with $[L:K]<\infty$. 
\end{enumerate} 
We call $K$ \textit{real closed} if it satisfies one of these equivalent conditions.\footnote{See Lange's \textit{Algebra} for partial proof.}
\end{theorem}

\begin{corollary}\label{7.2} Let $K'$ be a subfield of a real closed field $K$.  Then $K'$ is real closed if and only if $K'$ is algebraically closed in $K$. \end{corollary}

\begin{proof} Suppose $K'$ is not algebraically closed in $K$.  Fix $a\in K\setminus K'$ that is algebraic over $K'$.  Then, $K'(a)$ is an proper orderable algebraic field extension of $K'$.  Thus, $K'$ is not real closed by $(1)$ of theorem $(\ref{7.1})$.

Conversely, suppose $K'$ is algebraically closed in $K$.  We verify that condition (2) of theorem $(\ref{7.1})$ holds for $K'$.  Since $K'$ is algebraically closed in $K$, any zero of a polynomial of the form $X^2-a$ or $-X^2-a$, where $a\in K'$, must be in $K'$.  This along with the fact that $K$ is euclidean implies that $K'$ is euclidean.  Moreover, if $p\in K'[X]$ has odd degree, then since $K$ satisfies (2), $p$ has a zero $a\in K$.  But, $a\in K'$ since $K'$ is algebraically closed in $K$.  Thus, $K$ is real closed.  
\end{proof}

The archetypical example of a real closed field is $\mathds{R}$.  By corollary $(\ref{7.2})$, the algebraic closure of $\mathds{Q}$ in $\mathds{R}$ is also real closed.  In fact, the algebraic closure of $\mathds{Q}$ in $\mathds{R}$ can be embedded into any real closed field.

\begin{proposition} Suppose $K$ is real closed and $p\in K[X]$.  Then,
\begin{enumerate}[(1)]
\item $p$ is monic and irreducible if and only if $p=X-a$ for some $a\in K$ or $p=(X-a)^2+b^2$ for some $a, b\in K$, $b\neq 0$. 
\item The map $x\mapsto p(x): K \rightarrow K$ has the intermediate value theorem.  \end{enumerate}\end{proposition}

\begin{theorem}[Tarksi] The theory of real closed ordered fields in the language $\mathcal{L}=\{0, 1, +, -,  \cdot, \leq\}$ of ordered rings admits quantifier elimination.  Hence, for any real closed field $K$, $\mathds{R}\equiv K$ and, if $\mathds{R}$ is a subfield of $K$, then $\mathds{R}\preceq K$.\end{theorem}

\begin{theorem} Let $\Gamma$ be a divisible ordered abelian group and let $k$ be a real closed field.  Then, $K=k((t^\Gamma))$ is real closed. \end{theorem}

We have $K[i]\cong k[i]((t^\Gamma))$, so it's enough to show the following theorem.

\begin{theorem} Let $\Gamma$ be a divisible ordered abelian group and let $k$ be an algebraically closed field of characteristic $0$.  Then, $K=k((t^\Gamma))$ is algebraically closed. \end{theorem}

\begin{remark} This theorem is still true if we drop the characteristic $0$ assumption, but it would require a different proof than the one given below. \end{remark}

\begin{proof} Let $P\in K[X]$ be monic and irreducible, and write
$$P=X^n+a_{n-1}X^{n-1}+\cdots +a_0 \quad (a_i\in K, n\geq n).$$
By replacing $P(X)$ by $P(X-a_{n-1})$, we get
$$P\left(X-\frac{a_{n-1}}{n}\right) = X^n + \text{terms of degree $<n-1$}.$$
Thus, we may assume $a_{n-1}=0$.  Put $\gamma_i:=va_i\in \Gamma\cup\{\infty\}$ (recall that $vf:= \min \text{supp} f$ for $f\in K$) and put
$$\gamma:= \min \left \{\frac{1}{n-i}\gamma_i \ : \ i=0, \dots, n-2 \right \} \in \Gamma.$$
Then,
$$t^{-n\gamma}P(t^\gamma X)= X^n + \sum_{i=0}^{n-2}a_it^{(i-n)\gamma}X^i,$$
where $v(a_it^{(i-n)\gamma}) = \gamma_i + (i-n)\gamma\geq 0$, with equality holding for some $i$.  Thus, we may assume $va_i\geq 0$ for all $i$, and $va_i=0$ for some $i$.  

Let $\mathcal{O}:= \{f\in K \ : \ vf\geq 0\}$.  It is readily verified that this is a subring of $K$ which contains $k$.  We have a ring morphism $\mathcal{O}\rightarrow k$ define by
$$f= \sum_{\gamma\geq 0} f_\gamma t^\gamma \mapsto f_0=: \overline{f}.$$

\begin{lemma} Let $P\in \mathcal{O}[X]$ be monic and $\overline{P}=Q_0R_0$, where $Q_0, R_0\in k[X]$ are monic and relatively prime.  Then there are monic $Q, R\in \mathcal{O}[X]$ with $P=QR$ and $\overline{Q}=Q_0$, $\overline{R}=R_0$.  \end{lemma}

The lemma applies to our $P$.  Since $P$ is assumed irreducible, the lemma implies $\overline{P}=(X-a)^n$ for some $a\in k$, i.e., 
$$\overline{P}= X^n-naX^{n-1}+ \text{lower degree terms}.$$
Since $a_{n-1}=0$, we have $na=0$; hence, $a=0$ since $k$ has characteristic $0$.  Thus, $\overline{P}=X^n$.  But, $va_i=0$ for some $i$, so we have a contradiction.  

We now prove the lemma.  Write $P=\sum_{i<\alpha}P_i(X)t^{a_i}\in k[X]((t^\Gamma))$, where $a_i$ is strictly increasing in $\Gamma$, $a_0=0$, $P_i(X)\in k[X]$ are of degree $<n$ for $i>0$, and $P_0=\overline{P}$.  Suppose we have a strictly increasing sequence $(b_i)_{i<\beta}$ in $\Gamma$ and sequences $(Q_i)_{i<\beta}$, $(R_i)_{i<\beta}$ of polynomials in $k[X]$ of degree $<\deg Q_0$ and $<\deg R_0$, respectively, such that for
$$Q_{<\beta}:= \sum_{i<\beta}Q_it^{b_i}, \quad R_{<\beta}:= \sum_{i<\beta}R_it^{b_i}$$
we have 
$$P\equiv Q_{<\beta}R_{<\beta} \mod{(t^b\mathcal{O})}$$
for all $b\in \Gamma$ with $b\leq b_i$ for some $i$.  Suppose $P\neq Q_{<\beta}R_{<\beta}$; we are going to find $b_\beta\in \Gamma$ and $Q_\beta, R_\beta\in k[X]$ of degrees $< \deg Q_0$ and $< \deg R_0$, respectively, such that
\begin{enumerate}%[$\bullet$]
\item $b_\beta >b_i$ for all $i<\beta$.
\item $P\equiv (Q_{<\beta}+Q_\beta t^{b_\beta})(R_{<\beta} + R_\beta t^{b_\beta}) \mod{(t^b\mathcal{O})}$ for all $b\leq b_\beta$.  
\end{enumerate}
To this end, let $\gamma:= v(P-R_{<\beta}Q_{<\beta})\in \Gamma$.  Then, $b_\beta:= \gamma>b_i$ for all $i<\beta$.  Consider any $G, H\in k[X]$; then
$$P\equiv (Q_{<\beta}+Q_\beta t^{b_\beta})(R_{<\beta} + R_\beta t^{b_\beta}) \mod{(t^b\mathcal{O})}$$
for all $b\leq b_\beta$.  To get this congruence to hold also for $b=b_\beta$, we need $G, H$ to satisfy an equation
$$S=Q_0H+R_0G,$$
where $S\in k[X]$ has degree $<0$.  But we can find such $G, H$ since $Q_0, R_0$ are relatively prime.  Then, take $Q_\beta=G$ and $R_\beta= G$ for such $G, H$.  

\end{proof}
\section{ Week 8 }
\subsection{November 24, 2014}
''Notes for today by Madeline Barnicle''

Write $x \in \mathbf{No}$ in normal form. Say all powers of $\omega$ are positive. Take an initial segment of $x$; the segment also has this property. The proof requires the ''sign sequence'' (chapter 5 of Gonshor), and we need this to delve into the exponential function. Instead, we will cover:
\subsubsection{Section 8: Analytic functions on $\mathbf{No}$}
Let $\Gamma$ be an ordered abelian group, $K = \mathbb{R}((t^{\Gamma})), x=t^{-1}$. Let $F: I \rightarrow \mathbb{R}$ ($I=(a, b), a<b \in \mathbb{R} \cup \{\pm \infty \})$. Suppose $F$ is analytic. Then $F$ extends to a function $F_K : I_K \rightarrow K$, where $I_K = \{f \in K: a<f<b\}$. To explain this, consider the $\mathbb{R}$-subspaces:

$K^{\downarrow}=\{f \in K:$ supp$(f) \subset \Gamma^{> 0} \}$ (infinitesimals of $K$)

$K^{\uparrow}=\{f \in K:$ supp$(f) \subset \Gamma^{< 0} \}$ (the "purely infinite" elements of $K$)

So $K=K^{\downarrow} \oplus \mathbb{R} \oplus K^{\uparrow}$. $O=K^{\downarrow} \oplus \mathbb{R}$. Then $I_K=\{c+\epsilon | c \in I \subset \mathbb{R}, \epsilon \in K^{\downarrow}\}$. Set $F_K (c+\epsilon)=\sum_{n=0}^{\infty} \frac{F^{(n)}(c)}{n!}\epsilon^{n} \in K$, which converges in $K$ by Neumann's lemma. For example, $c \rightarrow e^c: \mathbb{R} \rightarrow \mathbb{R}^{> 0}$ extends this way to $c+\epsilon \rightarrow e^{c+\epsilon}=e^c \sum_{n=0}^{\infty} \frac{\epsilon^n}{n!}, O \rightarrow K^{> 0}$.

Likewise, every analytic function $G: U \rightarrow \mathbb{R}$ where $U \subset \mathbb{R}^n$ is open extends to a $K$-valued function whose domain $U_k$ is the set of all points in $K^n$ of infinitesimal distance to a point in $U$.

'''Digression on exp for $\mathbf{No}$'''

An ''exponential function'' on $K$ is an isomorphism $(K, \leq, +) \rightarrow (K^{\geq 0}, \geq, \cdot).$

A negative result:
'''Theorem''' (F.-V. and S. Kuhlmann, Shelah): If the underlying class of $\Gamma$ is a ''set'', and $\Gamma \neq \{0\}$, then there is ''no'' exponential function on $\mathbb{R}((t^{\Gamma}))$.

Nevertheless, there is an exponential function on $\mathbf{No} \cong \mathbb{R}((t^{\mathbf{No}}))$ (Gonshor/Kruskal).

For the rest of the course we will focus on restricted analytic functions on $\mathbf{No}$.

Let $I=[-1,1] \subset \mathbb{R}$. A restricted analytic function is a function $F: \mathbb{R}^m \rightarrow \mathbb{R}$ such that $F(x)=0$ for $x \in \mathbb{R}^{m} \setminus I^m$, and $F \restriction I^m$ extends to an analytic function $U \rightarrow \mathbb{R}$ for some neighborhood $U$ of $I^m$.

Example: $F: \mathbb{R} \rightarrow \mathbb{R}$ given by $F(x)=0$ if $|x|>1, F(x)=e^x$ if $|x| \leq 1$. For each such $F$, we have a function $F_K: K^M \rightarrow K$ such that $F_K(x)=0$ if $x \in K^m \setminus {I_K}^M, F_K \restriction {I_K}^m$ extends to a function $G_K: U_K \rightarrow K$ where $G: U \rightarrow \mathbb{R}$ is an analytic extension of $F \restriction I^m$ to an open neighborhood $U$ of $I^m$.

Example: for $F$ as before, $F_K(x)=0$ if $|x|>1$. $F_K(c+\epsilon)=e^{c}\sum_{n} \frac{\epsilon^n}{n!}$, for $x=c+\epsilon, |x| \leq 1$.

Let $L_{an}$ be the language $\{0,+, -, \cdot, \leq\}$ of ordered rings, augmented by a function symbol for each restricted analytic $\mathbb{R}^m \rightarrow \mathbb{R}$ (as $m$ varies). Let $\mathbb{R}_{an} = \mathbb{R}$ with the natural $L_{an}$ structure, $K_{an}=\mathbb{R}((t^{\Gamma}))$ with the natural structure (extensions as above). $\mathbb{R}_{an} \leq K_{an}$.

'''Theorem''' (van den Dries, Macintyre, Marker, extending Denef-van den Dries): If $\Gamma$ is divisible, then $R_{an} \prec K_{an}$ (elementary substructure). In particular, $\mathbb{R} \prec \mathbf{No}$.

vd Dries and Ehrlich expanded this by adding the exponential function to both sides.

\subsubsection{Section 9: Power series and Weierstrass Preparation}
Let $A$ be a commutative ring with $1, X=(x_1...x_m)$ indeterminates. $A[|x|]=A[|x_1, ...x_m)|]=\{f=\sum_{i \in \mathbb{N}^m}f_i x^i, f_i \in A\}$, ''the ring of formal power series in $X$ over $A$.'' Here, $X^i= x_{1}^{i_1}...x_{m}^{i_m}$. These terms can be added or multiplied in the obvious way.

$A \subset A[x] \subset A[|x|]$. For $i=(i_1...i_m ) \in \mathbb{N}^m$, put $|i|=i_1+...i_m$. For $f \in A[|x|]$, order $(f)=$min$\{|i|: f_i \neq 0\}$ if $f \neq 0$, or $\infty$ if $f=0$.

order $(f+g) \geq$ min (ord($f$), ord($g$)). ord($fg$) $\geq$ ord ($f$) $+$ ord($g$), with equality if and only if $A$ is an integral domain. $A[|x|]$ is an integral domain if and only if $A$ is.

Let $(f_j)_{j \in J}$ be a family in $A[|x|]$. If for all $d \in \mathbb{N}$ there are only finitely many $j \in J$ with ord($f_j$) $\leq d$, then we can make sense of $\sum_{j \in J}f_j \in A[|x|]$. We often write $f \in A[|x|]$ as $f=\sum_{d \in \mathbb{N}} f_d$ where $f_d=\sum_{|i|=d}f_i x^i$ is the degree-$d$ homogeneous part of $f$.

\subsection{ November 26, 2014 }
Let $A$ be a commutative ring, $X$ a set of $m$ indeterminates $X_1,\ldots,X_m$. For $f=\sum f_i X^i \in A[[X]]$, the map $f: A[[X]]\rightarrow A$ given by $f\mapsto f_0$ (here $0$ is the multi-index $(0,0,\ldots,0)$) is a ring morphism sending $f$ to its ''constant term''. 

====Lemma 9.1====
Let $f\in A[[X]]$. Then $f$ is a unit in $A[[X]]$ if and only $f_0$ is a unit in $A$.

'''Proof:'''

The "if" direction is clear. For the converse, suppose $f_0g_0=1$, where $g_0\in A$. Then $fg_0=1-h$, where $\rm{ord}(h)\ge 1$. Now we can apply the usual geometric series trick: take $\sum_n h_n\in A[[X]]$, which is defined using the notion of infinite sum defined last time, and check that $g_0\cdot \sum_n h^n$ is an inverse for $f$. 


Define $\mathfrak{o}=\{f\in A[[X]]:\rm{ord}(f)\ge 1\}=\{f:f_0=0\}$. This is an ideal of $A[[X]]$, and $A[[X]]=A\oplus \mathfrak{o}$ as additive groups. 

Every $f\in \mathfrak{o}$ is of the form $f=x_1g_1+\cdots +x_mg_m$, where $g_i\in A[[X]]$ (of course, this representation is not unique). More generally, set $\mathfrak{o}^d:=$ the ideal of $A[[X]]$ generated by products $f_1,\ldots,f_d$ with $f_i\in \mathfrak{o}$. Equivalently, this is the ideal $\{f:\rm{ord}(f)\ge d\}$ or the ideal generated by monomials of the form $X^i$, where $|i|=d$. That these are indeed equivalent is a straightforward exercise.

We will also need to define ''substitution''. Let $Y=(y_1,\ldots y_n)$ be another tuple of distinct indeterminates, and let $g_1\ldots, g_m\in A[[Y]]$ with constant term $0$. Define a ring morphism $A[[X]]\rightarrow A[[Y]]$ by $f\mapsto f(g_1,\ldots, g_m)=\sum_i f_i g^i$. In the usual applications of this definition, we'll have $X=Y$.

Let's introduce some more basic definitions for working with several sets of indeterminates. Suppose that $X$ and $Y$ are sets of indeterminates $\{X_1,\ldots,X_m\}$ and $\{Y_1,\ldots,Y_n\}$ respectively, and that none of the indeterminates in $Y$ appears in $X$. Put $(X,Y):=(X_1,\ldots, X_m, Y_1,\ldots, Y_n)$. Then for $f\in A[[X,Y]]$, we can write 
$$f= \sum_{i,j} f_{ij} X^iY^j.$$
This can be rewritten as $\sum_j(\sum_i f_{ij}X^i)Y^j$ (the sums above are actually the infinite sums defined last time). This gives an identification of $A[[X,Y]]$ with $A[[X]][[Y]]$.

The previous result can be sharpened somewhat.
====Lemma 9.2==== Let $f\in A[[X,Y]]$. Then there are unique $g_1,\ldots, g_n\in A[[X,Y]]$ such that
$$f(X,Y)=f(X)+X_1 g_1(X, Y_1)+\ldots + Y_n g_n(X, Y_1,\ldots Y_n)$$
where $g_i\in A[[X,Y_1,\ldots, Y_i]]$. 

The proof is an exercise.

From now on, we will take $A$ to be a field $K$. Let $T$ be an indeterminate not among $X_1,\ldots, X_m$. We call $f(X,T)\in A[[X,T]]$ ''regular in $T$ of order $d$'' if 
$$f(0,T)=cT^d+\textrm{ terms of order larger than }d$$
where $c\in K-\{0\}$.

Writing $f=\sum_{i\in\mathbb{N}} f_i(X)T^i$, this is also equivalent to either of the following:
* $f_0(0)=\cdots =f_{d-1}(0)=0$ and $f_d(0)\neq 0$,
* $f_0,\ldots, f_{d-1}\in \mathfrak{o}$, $f_d\in \mathfrak{o}$, $f_d\in K[[X]]^\times$.

The reason for this definition is that there is a "Euclidean division" result which holds when dividing by regular power series.

====Theorem 9.3 (Division with remainder)====
Let $f\in K[[X,T]]$ be regular in $T$ of order $d$. Then for each $g\in K[[X,T]]$, there is a unique pair $(Q,R)$ where $Q\in K[[X,T]]$ and $R\in K[[X]][T]$ (so it is just a ''polynomial'' in $T$) such that $g=Qf+R$ and $\mathrm{deg}_TR<d$.

'''Proof:'''

We first reduce to the case $f=T^d+F$, where $F\in\mathfrak{o}[[T]]$. Here $\mathfrak{o}[[T]]$ is a notation for the set of power series in $T$ whose coefficients are in $\mathfrak{o}$. Write $f=\sum_i f_i(X) T^i$, where $f_0,\ldots, f_{d-1}\in\mathfrak{o}$, $f_d\in K[[X]]^\times$. Then set $u:=\sum_{i\ge d} f_i T^{i-d}$. Then $u^{-1}f= u^{-1}(\sum_{i<d} f_i(X)T^i)+T^d\in \mathfrak{o}[[T]]+T^d$.

Now we prove uniqueness: if $g=Q_1f+R_1=Q_2f+R_2$ are both as in the conclusion of the theorem, then $qf=r$ where $q:=Q_1-Q_2$ and $r:=R_2-R_1$, so and $\rm{deg}_TR<d$. Suppose $q=\sum_i q_i T^i$. For each $i$, the coefficient of $T^{d+i}$ in $qf$ is 0 (since $qf=r$ has degree $<d$), which can be expressed as $0=q_if_d+\sum_{j<i}q_jf_{d+i-j}+\sum_{i<j\le i+d}q_jf_{d+i-j}$. Thus $q_i\in \mathfrak{o}$ for each $i$. Repeating this inductively, we see that each $q_i$ must be in $\mathfrak{o}^n$ for each $n\in \mathbb{N}$, so they must all be $0$. Hence $q=0$, and then $r=0$.

Finally we show existence, the more interesting part of the proof. Define $K[[X]]$-linear maps $\tau,\lambda:K[[X,T]]\rightarrow K[[X,T]]$ and $\alpha:K[[X,T]]\rightarrow K[[X]][T]$ by 
* $\tau(g):=\sum_i g_{i+d} T^i$ (the higher order part of $g$, divided out by $T^d$),
* $\lambda(g):=-\tau(g)F$ (recalling that $F=f-T^d$ was the $\mathfrak{o}[[T]]$ part of $f$)
* $\alpha(g):=g_0+g_1T+\cdots+g_{d-1}T^d$ (the lower degree terms of $g$).

Notice that $g=\tau(g)T^d+\alpha(g)=\tau(g)f+\alpha(g)+\lambda(g)$. Informally, $\tau(g)$ is our first guess for the quotient part. Then $\alpha(g)$ collects the lower degree terms of the remainder, and $\lambda(g)$ is an extra part that we must work to refine. 

We will do this by iterating the process above. For each $n$, applying the process above to $\lambda^n(g)$ we have $\lambda^n(g)=\tau(\lambda^n(g))f+\alpha(\lambda^n(g))+\lambda^{n+1}(g)$. (Let us denote this equation by $(\ast)_n$). If $g\in \mathfrak{o}^n[[T]]$, then $\tau(g),\alpha(g)\in\mathfrak{o}^n[[T]]$, and $\lambda(g)\in\mathfrak{o}^{n+1}[[T]]$, so the order always increases after applying $\lambda$ (this is a sign that we make some progress after applying $\lambda$). 

Using $(\ast)_n$ to iterate the $\lambda$ away, we see that:
$$g=\tau(g)f+\alpha(g)+\lambda(g)$$
$$=\tau(g)f+\alpha(g)+\tau(\lambda(g))f+\alpha(\lambda(g))+\lambda^2(g)$$
$$=\sum_n\tau(\lambda^n(g))f+\sum_n\alpha(\lambda^n(g)).$$

We can take $Q:=\sum_n\tau(\lambda^n(g))$ and $R:=\sum_n\alpha(\lambda^n(g))$. These sums are defined since $\tau(\lambda^n(g))$ and $\alpha(\lambda^n(g))$ have order $n$.
h

\Week{ Week 9 }
\NotesBy{Notes for today by John Lensmire.}
\Day{ Monday 12-1-2014 }

\begin{corollary}\label{9.4}[Weierstrass Preparation] % ==== Corollary 9.4 (Weierstrauss Preparation) ==== 

Let $f\in K[[X,T]]$ be regular of order $d$ in $T$.
Then $f\in K[[X,T]]^\times$ and $W\in K[[X]][T]$ is monic of degree $d$ in $T$.
 \end{corollary}

\begin{proof} %\WikiBold{Proof:} 

Using Theorem \ref{9.3}, we can write $T^d = Qf+R$ where $Q\in K[[X,T]], R\in K[[X]][T], \mathrm{deg}_TR < d$.

Let $x=0$ to get
$$T^d = \left( \sum_i Q_i(0) T^i \right) (f_d(0) + \textrm{ terms of higher order} )
+ R_0(0) + R_1(0) T + \cdots + R_{d-1}(0) T^{d-1}.$$
Looking at the coefficient of $T^d$, we have $1 = Q_0(0) f_d(0)$.
This implies $Q_0 \in K[[X]]^\times$ and thus $Q\in K[[X,T]]^\times$.

Hence, we have $f = uW$ where $u = Q^{-1}$ and $W = T^d - R$.

Uniqueness follows from the uniqueness in Theorem \ref{9.3} (details are left as an exercise.)
 \end{proof}

\begin{remark}
The above proof shows that we can take $W$ to be a Weierstrauss polynomial, i.e. a monic polynomial
$W = T^d + W_{d-1} T^{d-1} + \cdots + W_0$ where $W_0,\ldots, W_{d-1}\in \mathfrak{o}[[X]]$.
\end{remark}

\begin{corollary} % ==== Corollary 9.5 ==== 
Suppose $K$ is infinite. Then the ring $K[[X]]$ is noetherian.
 \end{corollary}

\begin{proof} %\WikiBold{Proof:} 
We proceed by induction on $m$.

If $m=0$, $K$ is a field, hence noetherian.

From $m$ to $m+1$:
Let $\{0\} \neq I \subset K[[X,T]]$ be an ideal.
Take $f\in I\setminus \{0\}$, after replacing $I,f$ by images under a suitable automorphism of $K[[X,T]]$
(see last time) we can assume that $f$ is regular in $T$ of some order $d$.
Then each $g\in I$ can be written as $g = qf + r$, where $q\in K[[X,T]]$ and $r\in A[T]$ is of degree $<d$ ($A = K[[X]]$).
Hence, $r\in I \cap (A + AT + \cdots + AT^{d-1})$. By induction $J:= I\cap (A + AT + \cdots + AT^{d-1})$ is a finitely generated $A$-module.

Therefore, $I$ is generated by $f$ and the (finitely many) generators of $J$, as needed.
 \end{proof}

\section{Convergent Power Series }

A {\em polyradius is a vector} $r = (r_1,\ldots, r_m)\in (R^{\geq 0})^m$.
Given polyradii $r,s$ we write
\begin{enumerate}
  \item  $r\leq s \Leftrightarrow r_i \leq s_i$ for each $i$.
  \item  $r < s \Leftrightarrow r_i < s_i$ for each $i$.
  \item  $r^i = r_1^{i_1}\cdots r_m^{i_m}$ for $i = (i_1,\ldots, i_m)\in \mathbb{N}^m$.
\end{enumerate}

Given a polyradius $r$ and $a\in \mathbb{C}^m$, $D_r(a):= \{x\in \mathbb{C}^m |\ |x_i - a_i| < r_i
\textrm{ for } i = 1,\ldots,m$, called the open \textit{polydisk centered at $a$ with polyradius $r$}.
Its closure, \textit{the closed polydisk centered at $a$ with polyradius $r$}, is $\overline{D_r}(a) = \{x | \ |x_i - a_i| \leq r_i\}$.

For $f\in \mathbb{C}[[X]]$, define $\|f\|_r := \sum_i |f_i| r^i \in \mathbb{R}^{\leq 0} \cup \{+\infty\}$.
Writing $\|\cdot \| = \|\cdot \|_r$, it is easy to verify:
\begin{enumerate}
  \item  $\|f\| = 0 \Leftrightarrow f = 0$.
  \item  $\|c f\| = |c| \cdot \|f\|$ for $c\in \mathbb{C}$.
  \item  $\|f + g\| \leq \|f\| + \|g\|$.
  \item  $\|f\cdot g\| \leq \|f\| \cdot \|g\|$.
  \item  $\|X^i f\| = r^i \|f\|$.
  \item  $r\leq s \Rightarrow \|f\|_r \leq \|f\|_s$.
\end{enumerate}

\begin{definition} % ==== Definition 10.1 ==== 
$\mathbb{C}\{X\}_r := \{f\in \mathbb{C}[[X]] |\ \|f\|_r < +\infty \}$
 \end{definition}

\begin{lemma}\ % ==== Lemma 10.2 ==== 
\begin{enumerate}
  \item  $\mathbb{C} \{X\}_r$ is a subalgebra of $\mathbb{C}[[X]]$ containing $C[X]$.
  \item  $\mathbb{C} \{X\}_r$ is complete with respect to the norm $\|\cdot\|_r$.
\end{enumerate}
 \end{lemma}

\begin{proof} %\WikiBold{Proof:} 
1. is clear. 2. is routine using a "Cauchy Estimate": for every $i\in \mathbb{N}^m$, $|f_i|\leq \|f\|_r/r^i$,
and is left as an exercise.
 \end{proof}

Each $f\in\mathbb{C}\{X\}_r$ gives rise to a function $\overline{D_r}(0)\rightarrow\mathbb{C}$ as follows:
For $x\in\overline{D_r}(0)$, the series $\sum_i f_i x^i$ converges absolutely to a complex number $f(x)$.
This function $x\mapsto f(x)$ is continuous (as it is the limit of uniformly continuous functions).

For $f\in \mathbb{C}[[X,Y]]$, $f = \sum_{j\in \mathbb{N}^n} f_j(X) Y^j$, 
and $(r,s)\in (\mathbb{R}^{\geq 0})^{m+n}$ a polyradius, we have (from the definitions)
$\|f\|_{(r,s)} = \sum_j \|f_j\|_r s^j$.
Hence, $f\in \mathbb{C}\{X,Y\}_{(r,s)}$ and in particular $f_j\in \mathbb{C}\{X\}_r$ for all $j$.
Further, we have,
for $x\in \overline{D_r}(0)$, $f(x,y):= \sum_j f_j(X)Y^j\in \mathbb{C}\{Y\}_s$,
and for $(x,y)\in \overline{D_{(r,s)}}(0)$, $f(x,y) = \left( \sum_j f_j(X)Y^j\right)(y) = \sum_j f_j(x)y^j$.

\begin{lemma} % ==== Lemma 10.3 ==== 

The map that sends $f\in \mathbb{C}\{X\}_r$ to $f_r$ given by $x\mapsto f(x)$ is an injective ring morphism:
$$\mathbb{C}\{X\}_r \rightarrow \{\textrm{ ring of continuous functions } \overline{D_r}(0) \rightarrow \mathbb{C}\}$$
Further, $\|f_r\|_{\mathrm{sup}} \leq \|f\|_r$.
 \end{lemma}

\begin{proof} %\WikiBold{Proof:} 

All claims follow from definitions directly except injectivity. By induction on $m$, we show:
$f\in \mathbb{C}\{X\}_r\setminus \{0\}$ implies $f_r$ does not vanish identically on any open neighborhood of $0\in \mathbb{C}^m$.

If $m=1$, write $f = X^d (f_d + f_{d+1} X + \cdots )$, where $f_i\in \mathbb{C}, f_d\neq 0$.

For $|x|\leq r$, the series $f_d + f_{d+1}X + \cdots $ converges to a continuous function of $X$.
This function takes value $f_d\neq 0$ at $x=0$, hence is non-zero in a neighborhood around $0$.

For $m\geq 2$, write $f = \sum_i f_i(X') X_m^i$, where $X' = (X_1,\ldots, X_{m-1})$, $f_i \in \mathbb{C}\{X'\}_{r'}$
with $r' = (r_1,\ldots, r_{m-1})$.
Then $\|f\|_r = \sum_i \|f_i\|_{r'} r_m^i$ and $f(x) = \sum_i f_i(X')X_m^i$ for $X = (X',X_m)\in \overline{D_r}(0)$.
Fix $j$ such that $f_j(X') = 0$. By the induction hypothesis, there are $X'\in \mathbb{C}^{m-1}$ as close as we want to $0$
such that $f_j(X')\neq 0$. For such an $X'$ (as in the $m=1$ case), we have $f(X',X_m)\neq 0$ for all sufficiently small $X_m$.
 \end{proof}

\NotesBy{Notes for today by Assaf Shani}
\Day{ Wednesday 12-3-2014 }

\global\long\def\N{\omega^{\omega}}
\global\long\def\Z{\mathbb{Z}}
\global\long\def\Q{\mathbb{Q}}
\global\long\def\R{\mathbb{R}}
\global\long\def\lto{\longrightarrow}
\global\long\def\es{\emptyset}
\global\long\def\F{\mathcal{F}}
\global\long\def\force{\Vdash}
\global\long\def\dom{\textrm{dom}}
\global\long\def\em{\prec}
\global\long\def\cf{\textrm{cf}}
\newcommandx\cof[1][usedefault, addprefix=\global, 1=]{\mathrm{cof}\left(#1\right)}
\global\long\def\model{\vDash}
\global\long\def\crit{\mathrm{crit}}
\global\long\def\ult{\mathrm{Ult}}
\global\long\def\inj{\hookrightarrow}
\global\long\def\u{\mathcal{U}}
\global\long\def\dprime{\prime\prime}
\global\long\def\C{\mathbb{C}}
\global\long\def\v{\mathcal{V}}
\global\long\def\w{\mathcal{W}}
\global\long\def\i{\imath}
\global\long\def\P{\mathbb{P}}
\newcommandx\norm[1][usedefault, addprefix=\global, 1=]{\left\Vert #1\right\Vert }

\begin{defn*}
\label{10.4} $\C\left\{ X\right\} =\bigcup_{r}\C\left\{ X\right\} _{r}$.\\
$\C\left\{ X\right\} $ is a subring of $\C\left[\left[X\right]\right]$
containing $\C\left[X\right]$, called the ring of convergent power
series in $X$.\end{defn*}
\begin{xca*}
\label{10.5} Let $\left\langle f_{j}\right\rangle _{j\in J}$ be a family
in $\C\left\{ X\right\} _{r}$ such that $\mathrm{ord}\left(f_{j}\right)\lto\infty$,
so $\sum_{j\in J}\in\C\left[\left[X\right]\right]$ exists. Then $\left\Vert \sum_{j}f_{j}\right\Vert _{r}\leq\sum_{j}\left\Vert f_{j}\right\Vert _{r}$.
Assume $\sum_{j}\left\Vert f_{j}\right\Vert <\infty$, then $\sum_{j}f_{j}\in\C\left\{ X\right\} _{r}$
and\end{xca*}
\begin{enumerate}
\item $\forall\epsilon>0\ \exists\textrm{ finite }I_{\epsilon}\subset J$ s.t
$\forall\textrm{ finite }I_{\epsilon}\subset I\subset J$, $\left\Vert \sum_{j\in J}f_{j}-\sum_{j\in I}f_{j}\right\Vert _{r}<\epsilon$.
\item $\forall x\in D_{r}\left(0\right)$, $\left(\sum_{j\in J}f_{j}\right)\left(x\right)=\sum_{j\in J}f_{j}\left(x\right)$.\end{enumerate}
\begin{lem*}
\label{10.6} (Abel) Let $f\in\C\left[\left[X\right]\right]$, $s\in\left(\R^{>0}\right)^{m}$,
$L\in\R^{>0}$, s.t $\left|f_{i}\right|s^{i}\leq L$ for all $i$.\\
Then $f\in\C\left\{ X\right\} _{r}$ for all $r<s$.\end{lem*}
\begin{proof}
\[
\sum_{i}\left|f_{i}\right|r^{i}=\sum_{i}\left|f_{i}\right|s^{i}\prod_{k<m}\left(\nicefrac{r_{k}}{s_{k}}\right)^{i_{k}}\leq L\sum_{i}\prod_{k<m}\left(\nicefrac{r_{k}}{s_{k}}\right)^{i_{k}}=L\prod_{k<m}\left(\underbrace{1+\frac{r_{k}}{s_{k}}+\left(\frac{r_{k}}{s_{k}}\right)^{2}+...}_{=1-\frac{r_{k}}{s_{k}}}\right)<\infty.
\]
 \end{proof}
\begin{cor*}
\label{10.7} $f\in\C\left\{ X\right\} $ if and only if $\exists s,L$ s.t
$\forall i\left(\left|f_{i}\right|s^{i}\leq L\right)$.\end{cor*}
\begin{lem*}
\label{10.8} Let $f\in\C\left\{ X\right\} $, f$\left(0\right)=0$ . Then
$\left\Vert f\right\Vert _{r}\overset{r\lto0}{\lto}0$.\end{lem*}
\begin{proof}
Take $s$ s.t $\left\Vert f\right\Vert _{s}<\infty$ (exists since
$f\in\C\left\{ X\right\} $). For $r<s$, 
\[
\left\Vert f\right\Vert _{r}=\sum_{i\neq0}\left|f_{i}\right|r^{i}=\underbrace{\sum\left|f_{i}\right|s^{i}}_{=\left\Vert f\right\Vert _{s}}\underbrace{\left(\frac{r}{s}\right)^{i}}_{\underset{r\lto0}{\lto}0}\overset{r\lto0}{\lto}0.
\]
\end{proof}
\begin{cor*}
\label{10.9} $\C\left\{ X\right\} ^{\times}=\left\{ f\in\C\left\{ X\right\} ;\, f\left(0\right)\neq0\right\} $.\end{cor*}
\begin{proof}
$"\subset"$ is clear. For $"\supset"$, let $f\in\C\left\{ X\right\} $
with $f\left(0\right)\neq0$. Write $f=f\left(0\right)\left(1-g\right)$
where $g\in\C\left\{ X\right\} $, $g\left(0\right)=0$. Then $h=1+g+g^{2}+...$.
We have already seen that $h\in\C\left[\left[X\right]\right]$ and
$h=\left(1-g\right)^{-1}$. So we only need to show that $h\in\C\left\{ X\right\} $.
By \ref{10.8}, for small enough $r$, $\norm[g]_{r}<1$. So by \ref{10.5},
$\norm[h]_{r}\leq\sum_{k\neq0}^{\infty}\norm[g]_{r}^{k}<\infty$,
hence $h\in\C\left\{ X\right\} _{r}$.\end{proof}
\begin{lem*}
\label{10.10} Let $g_{1},...,g_{n}\in\C\left\{ Y\right\} $ have constant
term zero. Then for each $f\in\C\left\{ X\right\} $ we have $f\left(g_{1},...,g_{m}\right)\in\C\left\{ Y\right\} $.
Moreover, for $r,s$ small enough, $f\in\C\left\{ X\right\} _{r}$,
$g_{j}\in\C\left\{ Y\right\} _{s}$, for any $y\in\bar{D}_{s}\left(0\right)$,
$g_{j}\left(y\right)\in\bar{D}_{r}\left(0\right)$ and
\[
f\left(g_{1},...,g_{m}\right)\left(y\right)=f\left(g_{1}\left(y\right),...,g_{m}\left(y\right)\right).
\]
\end{lem*}
\begin{proof}
Exercise.\end{proof}
\begin{thm*}
\label{10.11} Let $f,g\in\C\left\{ X,T\right\} $ with $f$ regular of order
$d$. Then $g=Qf+R$ with $Q\in\C\left\{ X,T\right\} $ and $R\in\C\left\{ X\right\} \left[T\right]$
of degree $<d$.\end{thm*}
\begin{proof}
We repeat the Weierstrass division theorem, making sure that everything
converges.\\
Recall, $f=\sum_{i\geq0}f_{i}T^{i}$ where $f_{0}\left(0\right)=...=f_{d-1}\left(0\right)=0$
and $f_{d}\left(0\right)\neq0$. We defined $u=\sum_{i\geq d}f_{i}T^{i}$,
$F=u^{-1}\sum_{i<d}f_{i}T^{i}$ (note that by \ref{10.9} $u\in\C\left\{ X,T\right\} $).
So $u^{-1}f=T^{d}+F$, and we work with $T^{d}+F$ instead. i.e. assume
$f=T^{d}+F$.\\
Take $r'=\left(r_{1},...,r_{m}\right)$ and $r_{m+1}\in\R^{>0}$ small
enough such that for $r=\left(r_{1},...,r_{m+1}\right)$, $\norm[g]_{r}$,
$\norm[u^{-1}]_{r}$, \\
$\norm[f_{0}]_{r'}$,...,$\norm[f_{d-1}]_{r'}<\infty$. Thus 
\[
\norm[F]_{r}\leq\norm[u^{-1}]_{r}\sum_{i<d}\norm[f_{i}]_{r'}r_{m+1}^{i}.
\]
By \ref{10.8}, for each $i$, $\norm[f_{i}]_{r'}\lto0$ as $r'\lto0$.
Also, $\norm[u^{-1}]_{r}$ can only decrease by decreasing $r'$.
Thus by decreasing $r'$ we can assume $\norm[F]_{r}\leq r_{m+1}^{d}$.
Take $\epsilon=\frac{\norm[F]_{r}}{r_{m+1}^{d}}\in\left[0,1\right)$.
\\
Recall that for 
\[
\tau\left(g\right)=\sum_{i\geq d}g_{i}T^{i-d},\quad\alpha\left(g\right)=\sum_{i<d}g_{i}T^{i},\quad\lambda\left(g\right)=-\tau\left(g\right)F,
\]
we have $g=\tau\left(g\right)f+\alpha\left(g\right)+\lambda\left(g\right)$.
\begin{eqnarray*}
\norm[\tau\left(g\right)]_{r} & \leq & \sum_{i\geq d}\norm[g_{i}]_{r}r_{m+1}^{i-d}<r_{m+1}^{-d}\norm[g]_{r};\\
\norm[\alpha\left(g\right)]_{r} & \leq & \norm[g]_{r};\\
\norm[\lambda\left(g\right)]_{r} & \leq & \norm[\tau\left(g\right)]_{r}\norm[F]_{r}\leq r_{m+1}^{-d}\norm[g]_{r}\norm[F]_{r}\leq\epsilon\norm[g]_{r}.
\end{eqnarray*}
Inductively define $\lambda^{n}\left(g\right)$ according to 
\[
\left(\ast\right)\quad\lambda^{n}\left(g\right)=\tau\left(\lambda^{n}\left(g\right)\right)f+\alpha\left(\lambda^{n}\left(g\right)\right)+\lambda^{n+1}\left(g\right).
\]
Inductively, $\norm[\lambda^{n}\left(g\right)]_{r}\leq\epsilon^{n}\norm[g]_{r}$,
hence $\norm[\tau\left(\lambda^{n}\left(g\right)\right)]\leq\norm[g]\epsilon^{n}r_{m+1}^{-d}$,
$\norm[\alpha\left(\lambda^{n}\left(g\right)\right)]_{r}\leq\norm[g]\epsilon^{n}$.\\
Summing $\left(\ast\right)$ for $n\geq0$, we get $f=Qf+R$, for
$Q=\sum_{n}\tau\left(\lambda^{n}\left(g\right)\right)$, $R=\sum_{n}\alpha\left(\lambda^{n}\left(g\right)\right)$.
Furthermore $\norm[Q]_{r}\leq\sum_{n}\norm[\tau\left(\lambda^{n}\left(g\right)\right)]_{r}\leq\norm[g]_{r}r_{m+1}^{-d}\sum_{n}\epsilon^{n}<\infty$,
$\norm[R]\leq\sum_{n}\norm[\alpha\left(\lambda^{n}\left(g\right)\right)]\leq\norm[g]\sum\epsilon^{n}<\infty$.
\\
Thus $Q,R\in\C\left\{ X,T\right\} $. Note that by definition $R\in\C\left[\left[X\right]\right]\left[T\right]$
is of degree $d$, hence $R\in\C\left\{ X\right\} \left[T\right]$.\end{proof}
\begin{cor*}
\label{10.12} (Weierstrass preparation)

Let $f\in\C\left\{ X,T\right\} $ be regular in $T$. Then $f=uW$
where $u\in\C\left\{ X,T\right\} ^{\times}$ and $W\in\C\left\{ X\right\} \left[T\right]$
of degree $d$.\end{cor*}
\begin{proof}
Same as \ref{9.4}, using \ref{10.11} instead of \ref{9.3}, so that everything
converges.\end{proof}

\NotesBy{Notes for today by Tyler Arant}
\Day{ Friday 12-5-2014 }

\textbf{Correction for proof of Weierstrass Preparation.} We had $f, g\in \mathds{C}\{X, T\}$, $f$ regular of order $d$, and
$$F=u^{-1}\sum_{i<d}f_iT^i, \quad u=f_d + f_{d+1}T+ \cdots \in \mathds{C}\{X, T\}^\times.$$
Choose $(r', r_{m+1})\in (\mathds{R}^{>0})^{m+1}$ such that
$$\|g\|_r, \|u^{-1}\|_r, \|f_0\|_{r'}, \dots, \|f_{d+1}\|_{r'}<\infty.$$
Then, we can achieve
$$\|F\|_r\leq \|u^{-1}\|\cdot \sum_{i<d}\|f_i\|_{r'}r^i_{m+1}<r^d_{m+1},$$
since the $f_i$ vanish at $0$ we can make the norms as small as we want by choosing $r'$ small enough.\\

Let $R\subset S$ be an extension of commutative rings.

\begin{definition} $S$ is \textit{flat over} $R$ if each solution in $S$ to an equation
\begin{equation} r_1x_1+\cdots + r_n x_n=0 \qquad (r_i\in R) \tag{$*$}\end{equation}
is an $S$-linear combination of solutions in $R$. \end{definition}

\begin{lemma} If $S$ free as an $R$-module, then $S$ is flat over $R$. \end{lemma}

\begin{proof} Let $s=(s_1, \dots, s_n)\in S^n$ be a solution to $(*)$.  Take $R$-linearly independent $e_1, \dots, e_k\in S$ such that
$$s_i=\sum_j w_{ij} e_j \qquad (w_{ij}\in R).$$
Put $w_j=(w_{1j}, \dots, w_{nj})$.  Then $w_j$ is a solution to $(*)$ and $s=\sum_j e_jw_j.$
\end{proof}

We give some examples
\begin{enumerate}%[$\bullet$]
\item If $R$ is a field, then each $S$ is flat over $R$.  
\item $S=R[X_1, \dots, X_n]$ flat over $R$.
\end{enumerate}

\begin{lemma} Suppose $S$ is flat over $R$.  Then each solution in $S$ to a system
$$r_{i1}x_1 + \cdots + r_{in}x_n=0 \qquad (i=1, \dots, m ; r_{ij}\in R)$$
is an $S$-linear combination of solutions in $R$. \end{lemma}

\begin{proof} By induction on $m$.  \end{proof}

\begin{definition} We say that $S$ is \textit{faithfully flat} over $R$ if
\begin{enumerate}%[$\bullet$]
\item $S$ is flat over $R$.
\item Each equation 
$$r_1x_1+\cdots +r_nx_n=1 \qquad (r_i\in R)$$
that has a solution in $S$ has a solution in $R$. \end{enumerate}
\end{definition}

\begin{lemma} Suppose $S$ is flat over $R$.  The following are equivalent.
\begin{enumerate}[(1)]
\item $S$ is faithfully flat over $R$.
\item For each maximal ideal $\mathfrak{m}$ of $R$, we have $\mathfrak{m}S\neq S$.
\item Each system 
\begin{equation} \sum_{j=1}^n r_{ij}x_j = t_i \qquad (i=1, \dots, m ; r_{ij}, t_i\in R) \tag{$*$} \end{equation}
that has a solution in $S$ has a solution in $R$. 
\end{enumerate} \end{lemma}

$(1)\implies(2)$: Suppose, by means of contradiction, that $\mathfrak{m}S=S$.  Then, there are $r_i\in \mathfrak{m}$ and $s_i\in S$ such that
$$r_1s_1+\cdots + r_n s_n = 1,$$
i.e., $s=(s_1, \dots, s_n)$ is a solution to the equation
$$r_1x_1 + \cdots + r_n x_n = 1.$$
Since $S$ is faithfully flat over $R$, there is a solution $w=(w_1, \dots, w_n)$ so that
$$1= r_1w_1+\cdots + r_n w_n \in \mathfrak{m},$$
which is a contradiction. 

$(2)\implies (1)$:  Suppose $S$ is not faithfully flat over $R$; then, there are $r_i\in R$ such that
$$r_1x_1+\cdots + r_n x_n=1$$
has a solution $s$ in $S$ but not a solution in $R$.  Then the ideal $\mathfrak{a}=(r_1, \dots, r_n)$ in $R$ is proper.  Let $\mathfrak{m}$ be an ideal in $R$ that contains $\mathfrak{a}$. Then, $\mathfrak{m}S=S$ since
$$1=r_1s_1+\cdots r_n s_n \in \mathfrak{m}S.$$

$(3)\implies (1)$ trivially.

$(1)\implies (3)$: Suppose $(*)$ has a solution $s=(s_1, \dots, s_n)\in S^n$.  Then, $(1, s)= (1, s_1, \dots, s_n)$ is a solution to the homogenous system
\begin{align*} -t_1x_0+\sum_{j=1}^n r_{1j}x_j &=  0  \\
				&\vdots     \tag{$**$} \\
		-t_mx_0 + \sum_{j=1}^n r_{mj}x_j  & =  0 \end{align*}
Since $S$ is flat over $R$, $(1, s)$ is an $S$-linear combination of solutions $(u_1, v_1), \dots, (u_k, v_k)$ in $R^{1+n}$. So,
$$(1, s) = w_1(u_1, v_1) + \cdots + w_k(u_k, v_k) \qquad (w_i\in S),$$
hence
$$1= w_1 u_1 + \cdots + w_k u_k.$$
By $(1)$, there exists $\omega_1, \dots, \omega_k\in R$ such that
$$1= \omega_1u_1 + \cdots + \omega_k u_k.$$
Then, $\omega_1v_1+\cdots + \omega_k v_k\in R^n$ solves $(*)$. 



\begin{theorem} $\mathds{C}[[X]]$ is faithfully flat over $\mathds{C}\{X\}$. \end{theorem}

\begin{proof}[Proof sketch] We proceed by induction on the number of variable.  Consider
\begin{equation}f_1y_1+\cdots + f_n y_n = 0 \qquad (f_i\in \mathds{C}\{X, T\}. \tag{$*$} \end{equation}
We may assume that all $f_i$, if nonzero, are regular in $T$.  Then, apply Weierstass Preparation in $\mathds{C}\{X, T\}$ to the $f_i\neq 0$.  Then, we can assume that the $f_i\neq0$ are Weierstrass polynomials: $f_i\in \mathds{C}\{X\}[T]$ monic of some degree $d_i$.  Set
$$z_2 = \left [ \begin{array}{c} f_2 \\ -f_1 \\ 0 \\ \vdots \\ 0 \end{array} \right ], z_3= \left [ \begin{array}{c} f_3 \\ 0 \\ -f_1 \\ 0 \\ \vdots \\ 0 \end{array} \right ], \cdots, z_n= \left [ \begin{array}{c} f_n \\ 0 \\ \vdots \\ 0 \\ -f_1\end{array} \right ],$$
which together is a solution of $(*)$.  We have $y_i=q_if_1+r_i$, where $q_i\in \mathds{C}[[X, T]]$, and $r_i\in \mathds{C}[[X]][T]$ is of degree $<d_1$.  Then consider
$$y+q_2z_2= \left [ \begin{array}{c} * \\ r_2 \\ y_3 \\ \vdots \\ y_n \end{array} \right ], \dots , y+q_2z_2+\cdots f_ny_n= \left [ \begin{array}{c} * \\ r_2\\ r_3 \\ \vdots \\ r_n \end{array} \right ],$$
and conclude that we can assume $y_2, \dots, y_n\in \mathds{C}[[X]][T]$.   We have
$$g:= f_1y_1 = -(f_2y_2+\cdots + f_ny_n)\in \mathds{C}\{X\}[T].$$
We can find $h, r$ with $g=f_1h+r$ with $h, r\in \mathds{C}[[X]][T]$, $\deg r<d_1$.  So, $g=f_1y_1+0$ in $\mathds{C}[[X, T]]$, hence $r=0$ and $y_1=h\in \mathds{C}[[X]][T]$.  This reduces the proof to showing: $R\subset S$ flat $\implies R[T]\subset S[T]$ flat.
\end{proof}

\section{Restricted Analytic Functions}

\begin{lemma}\label{11.1}[Taylor expansion] Suppose $f\in \mathds{C}\{X\}_s$ and $b\in D_s(0)$, $j\in \mathds{N}^m$.  Then,
\begin{enumerate}[(1)]
\item $\partial^j f :=\left (\frac{\partial}{\partial X_1} \right )^{j_1} \cdots \left (\frac{\partial}{\partial X_m} \right )^{j_m}f\in \mathds{C}\{X\}_r$ for all $r<s$.
\item $(\partial^jf)(b):= \sum_{i\geq j} f_i \frac{i!}{(i-j)!}b^{i-j}$ converges absolutely.  
\item $\sum_j \frac{1}{j!} (\partial^jf)(b)X^j\in \mathds{C}\{X\}_{s(b)}$, where
$$s(b)=(s_1-|b_1|, \dots, s_m-|b_m|),$$
and
$$f(x+b) = \sum_j\frac{1}{j!}(\partial^jf)(b)x^j \qquad (x\in \overline{D_{s(b)}}(0)).$$
\end{enumerate}
\end{lemma}

\begin{proof} \begin{enumerate}[(1)]
\item By induction on $|j|$.  The case $\frac{\partial}{\partial x+k}$ follows from Abel's Lemma: a finite bound on $|f_i|s^i$ gives a finite bound on $i_k|f_i|r^{i-e_k}$ (for given $r<s$, with $e_k$ is the $k^th$ standard basis vector).  
\item Follows easily from $(1)$.
\item Left as an exercise using $(2)$ and multivariate binomial theorem. 
\end{enumerate}\end{proof}


\global\long\def\N{\omega^{\omega}}
\global\long\def\Z{\mathbb{Z}}
\global\long\def\Q{\mathbb{Q}}
\global\long\def\R{\mathbb{R}}
\global\long\def\lto{\longrightarrow}
\global\long\def\es{\emptyset}
\global\long\def\F{\mathcal{F}}
\global\long\def\force{\Vdash}
\global\long\def\dom{\textrm{dom}}
\global\long\def\em{\prec}
\global\long\def\cf{\textrm{cf}}
\newcommandx\cof[1][usedefault, addprefix=\global, 1=]{\mathrm{cof}\left(#1\right)}
\global\long\def\model{\vDash}
\global\long\def\crit{\mathrm{crit}}
\global\long\def\ult{\mathrm{Ult}}
\global\long\def\inj{\hookrightarrow}
\global\long\def\u{\mathcal{U}}
\global\long\def\dprime{\prime\prime}
\global\long\def\C{\mathbb{C}}
\global\long\def\v{\mathcal{V}}
\global\long\def\w{\mathcal{W}}
\global\long\def\i{\imath}
\global\long\def\P{\mathbb{P}}
\newcommandx\norm[1][usedefault, addprefix=\global, 1=]{\left\Vert #1\right\Vert }

\begin{defn*}
\label{10.4} $\C\left\{ X\right\} =\bigcup_{r}\C\left\{ X\right\} _{r}$.\\
$\C\left\{ X\right\} $ is a subring of $\C\left[\left[X\right]\right]$
containing $\C\left[X\right]$, called the ring of convergent power
series in $X$.\end{defn*}
\begin{xca*}
(10.5) Let $\left\langle f_{j}\right\rangle _{j\in J}$ be a family
in $\C\left\{ X\right\} _{r}$ such that $\mathrm{ord}\left(f_{j}\right)\lto\infty$,
so $\sum_{j\in J}\in\C\left[\left[X\right]\right]$ exists. Then $\left\Vert \sum_{j}f_{j}\right\Vert _{r}\leq\sum_{j}\left\Vert f_{j}\right\Vert _{r}$.
Assume $\sum_{j}\left\Vert f_{j}\right\Vert <\infty$, then $\sum_{j}f_{j}\in\C\left\{ X\right\} _{r}$
and\end{xca*}
\begin{enumerate}
\item $\forall\epsilon>0\exists\textrm{finite }I_{\epsilon}\subset J$ s.t
$\forall\textrm{finite }I_{\epsilon}\subset I\subset J$, $\left\Vert \sum_{j\in J}f_{j}-\sum_{j\in I}f_{j}\right\Vert _{r}<\epsilon$.
\item $\forall x\in D_{r}\left(0\right)$, $\left(\sum_{j\in J}f_{j}\right)\left(x\right)=\sum_{j\in J}f_{j}\left(x\right)$.\end{enumerate}
\begin{lem*}
(10.6) (Abel) Let $f\in\C\left[\left[X\right]\right]$, $s\in\left(\R^{>0}\right)^{m}$,
$L\in\R^{>0}$, s.t $\left|f_{i}\right|s^{i}\leq L$ for all $i$.\\
Then $f\in\C\left\{ X\right\} _{r}$ for all $r<s$.\end{lem*}
\begin{proof}
\[
\sum_{i}\left|f_{i}\right|r^{i}=\sum_{i}\left|f_{i}\right|s^{i}\prod_{k<m}\left(\nicefrac{r_{k}}{s_{k}}\right)^{i_{k}}\leq L\sum_{i}\prod_{k<m}\left(\nicefrac{r_{k}}{s_{k}}\right)^{i_{k}}=L\prod_{k<m}\left(\underbrace{1+\frac{r_{k}}{s_{k}}+\left(\frac{r_{k}}{s_{k}}\right)^{2}+...}_{=1-\frac{r_{k}}{s_{k}}}\right)<\infty.
\]
 \end{proof}
\begin{cor*}
(10.7) $f\in\C\left\{ X\right\} $ if and only if $\exists s,L$ s.t
$\forall i\left(\left|f_{i}\right|s^{i}\leq L\right)$.\end{cor*}
\begin{lem*}
(10.8) Let $f\in\C\left\{ X\right\} $, f$\left(0\right)=0$ . Then
$\left\Vert f\right\Vert _{r}\overset{r\lto0}{\lto}0$.\end{lem*}
\begin{proof}
Take $s$ s.t $\left\Vert f\right\Vert _{s}<\infty$ (exists since
$f\in\C\left\{ X\right\} $). For $r<s$, 
\[
\left\Vert f\right\Vert _{r}=\sum_{i\neq0}\left|f_{i}\right|r^{i}=\underbrace{\sum\left|f_{i}\right|s^{i}}_{=\left\Vert f\right\Vert _{s}}\underbrace{\left(\frac{r}{s}\right)^{i}}_{\underset{r\lto0}{\lto}0}\overset{r\lto0}{\lto}0.
\]
\end{proof}
\begin{cor*}
(10.9) $\C\left\{ X\right\} ^{\times}=\left\{ f\in\C\left\{ X\right\} ;\, f\left(0\right)\neq0\right\} $.\end{cor*}
\begin{proof}
$"\subset"$ is clear. For $"\supset"$, let $f\in\C\left\{ X\right\} $
with $f\left(0\right)\neq0$. Write $f=f\left(0\right)\left(1-g\right)$
where $g\in\C\left\{ X\right\} $, $g\left(0\right)=0$. Then $h=1+g+g^{2}+...$.
We have already seen that $h\in\C\left[\left[X\right]\right]$ and
$h=\left(1-g\right)^{-1}$. So we only need to show that $h\in\C\left\{ X\right\} $.
By (10.8), for small enough $r$, $\norm[g]_{r}<1$. So by (10.5),
$\norm[h]_{r}\leq\sum_{k\neq0}^{\infty}\norm[g]_{r}^{k}<\infty$,
hence $h\in\C\left\{ X\right\} _{r}$.\end{proof}
\begin{lem*}
(10.10) Let $g_{1},...,g_{n}\in\C\left\{ Y\right\} $ have constant
term zero. Then for each $f\in\C\left\{ X\right\} $ we have $f\left(g_{1},...,g_{m}\right)\in\C\left\{ Y\right\} $.
Moreover, for $r,s$ small enough, $f\in\C\left\{ X\right\} _{r}$,
$g_{j}\in\C\left\{ Y\right\} _{s}$, for any $y\in\bar{D}_{s}\left(0\right)$,
$g_{j}\left(y\right)\in\bar{D}_{r}\left(0\right)$ and
\[
f\left(g_{1},...,g_{m}\right)\left(y\right)=f\left(g_{1}\left(y\right),...,g_{m}\left(y\right)\right).
\]
\end{lem*}
\begin{proof}
Exercise.\end{proof}
\begin{thm*}
(10.11) Let $f,g\in\C\left\{ X,T\right\} $ with $f$ regular of order
$d$. Then $g=Qf+R$ with $Q\in\C\left\{ X,T\right\} $ and $R\in\C\left\{ X\right\} \left[T\right]$
of degree $<d$.\end{thm*}
\begin{proof}
We repeat the Weierstrass division theorem, making sure that everything
converges.\\
Recall, $f=\sum_{i\geq0}f_{i}T^{i}$ where $f_{0}\left(0\right)=...=f_{d-1}\left(0\right)=0$
and $f_{d}\left(0\right)\neq0$. We defined $u=\sum_{i\geq d}f_{i}T^{i}$,
$F=u^{-1}\sum_{i<d}f_{i}T^{i}$ (note that by (10.9) $u\in\C\left\{ X,T\right\} $).
So $u^{-1}f=T^{d}+F$, and we work with $T^{d}+F$ instead. i.e. assume
$f=T^{d}+F$.\\
Take $r'=\left(r_{1},...,r_{m}\right)$ and $r_{m+1}\in\R^{>0}$ small
enough such that for $r=\left(r_{1},...,r_{m+1}\right)$, $\norm[g]_{r}$,
$\norm[u^{-1}]_{r}$, \\
$\norm[f_{0}]_{r'}$,...,$\norm[f_{d-1}]_{r'}<\infty$. Thus 
\[
\norm[F]_{r}\leq\norm[u^{-1}]_{r}\sum_{i<d}\norm[f_{i}]_{r'}r_{m+1}^{i}.
\]
By (10.8), for each $i$, $\norm[f_{i}]_{r'}\lto0$ as $r'\lto0$.
Also, $\norm[u^{-1}]_{r}$ can only decrease by decreasing $r'$.
Thus by decreasing $r'$ we can assume $\norm[F]_{r}\leq r_{m+1}^{d}$.
Take $\epsilon=\frac{\norm[F]_{r}}{r_{m+1}^{d}}\in\left[0,1\right)$.
\\
Recall that for 
\[
\tau\left(g\right)=\sum_{i\geq d}g_{i}T^{i-d},\quad\alpha\left(g\right)=\sum_{i<d}g_{i}T^{i},\quad\lambda\left(g\right)=-\tau\left(g\right)F,
\]
we have $g=\tau\left(g\right)f+\alpha\left(g\right)+\lambda\left(g\right)$.
\begin{eqnarray*}
\norm[\tau\left(g\right)]_{r} & \leq & \sum_{i\geq d}\norm[g_{i}]_{r}r_{m+1}^{i-d}<r_{m+1}^{-d}\norm[g]_{r};\\
\norm[\alpha\left(g\right)]_{r} & \leq & \norm[g]_{r};\\
\norm[\lambda\left(g\right)]_{r} & \leq & \norm[\tau\left(g\right)]_{r}\norm[F]_{r}\leq r_{m+1}^{-d}\norm[g]_{r}\norm[F]_{r}\leq\epsilon\norm[g]_{r}.
\end{eqnarray*}
Inductively define $\lambda^{n}\left(g\right)$ according to 
\[
\left(\ast\right)\quad\lambda^{n}\left(g\right)=\tau\left(\lambda^{n}\left(g\right)\right)f+\alpha\left(\lambda^{n}\left(g\right)\right)+\lambda^{n+1}\left(g\right).
\]
Inductively, $\norm[\lambda^{n}\left(g\right)]_{r}\leq\epsilon^{n}\norm[g]_{r}$,
hence $\norm[\tau\left(\lambda^{n}\left(g\right)\right)]\leq\norm[g]\epsilon^{n}r_{m+1}^{-d}$,
$\norm[\alpha\left(\lambda^{n}\left(g\right)\right)]_{r}\leq\norm[g]\epsilon^{n}$.\\
Summing $\left(\ast\right)$ for $n\geq0$, we get $f=Qf+R$, for
$Q=\sum_{n}\tau\left(\lambda^{n}\left(g\right)\right)$, $R=\sum_{n}\alpha\left(\lambda^{n}\left(g\right)\right)$.
Furthermore $\norm[Q]_{r}\leq\sum_{n}\norm[\tau\left(\lambda^{n}\left(g\right)\right)]_{r}\leq\norm[g]_{r}r_{m+1}^{-d}\sum_{n}\epsilon^{n}<\infty$,
$\norm[R]\leq\sum_{n}\norm[\alpha\left(\lambda^{n}\left(g\right)\right)]\leq\norm[g]\sum\epsilon^{n}<\infty$.
\\
Thus $Q,R\in\C\left\{ X,T\right\} $. Note that by definition $R\in\C\left[\left[X\right]\right]\left[T\right]$
is of degree $d$, hence $R\in\C\left\{ X\right\} \left[T\right]$.\end{proof}
\begin{cor*}
(10.12) (Weierstrass preparation)

Let $f\in\C\left\{ X,T\right\} $ be regular in $T$. Then $f=uW$
where $u\in\C\left\{ X,T\right\} ^{\times}$ and $W\in\C\left\{ X\right\} \left[T\right]$
of degree $d$.\end{cor*}
\begin{proof}
Same as (9.4), using (10.11) instead of (9.3), so that everything
converges.\end{proof}
%% LyX 2.0.4 created this file.  For more info, see http://www.lyx.org/.
%% Do not edit unless you really know what you are doing.
\documentclass[12pt,oneside,english]{amsart}
\usepackage[T1]{fontenc}
\usepackage[latin9]{inputenc}
\usepackage{geometry}
\geometry{verbose,tmargin=3cm,bmargin=3cm,lmargin=2cm,rmargin=2cm}
\setlength{\parskip}{\medskipamount}
\setlength{\parindent}{0pt}
\usepackage{units}
\usepackage{amsthm}
\usepackage{amstext}
\usepackage{amssymb}
\usepackage{xargs}[2008/03/08]
\PassOptionsToPackage{normalem}{ulem}
\usepackage{ulem}

\makeatletter
%%%%%%%%%%%%%%%%%%%%%%%%%%%%%% Textclass specific LaTeX commands.
\numberwithin{equation}{section}
\numberwithin{figure}{section}
  \theoremstyle{plain}
  \newtheorem*{prop*}{\protect\propositionname}
  \theoremstyle{plain}
  \newtheorem*{cor*}{\protect\corollaryname}
 \theoremstyle{definition}
 \newtheorem*{defn*}{\protect\definitionname}
  \theoremstyle{remark}
  \newtheorem*{claim*}{\protect\claimname}
  \theoremstyle{plain}
  \newtheorem*{thm*}{\protect\theoremname}
  \theoremstyle{plain}
  \newtheorem*{lem*}{\protect\lemmaname}

%%%%%%%%%%%%%%%%%%%%%%%%%%%%%% User specified LaTeX commands.
 \usepackage{mathptmx}	 
 \usepackage[scaled=0.92]{helvet}
 \usepackage{courier}

\makeatother

\usepackage{babel}
  \providecommand{\claimname}{Claim}
  \providecommand{\corollaryname}{Corollary}
  \providecommand{\definitionname}{Definition}
  \providecommand{\lemmaname}{Lemma}
  \providecommand{\propositionname}{Proposition}
  \providecommand{\theoremname}{Theorem}

\begin{document}
\global\long\def\N{\mathbb{N}}
\global\long\def\Z{\mathbb{Z}}
\global\long\def\Q{\mathbb{Q}}
\global\long\def\R{\mathbb{R}}
\global\long\def\lto{\longrightarrow}
\global\long\def\es{\emptyset}
\global\long\def\F{\mathcal{F}}
\global\long\def\force{\Vdash}
\global\long\def\dom{\textrm{dom}}
\global\long\def\em{\prec}
\global\long\def\cf{\textrm{cf}}
\newcommandx\cof[1][usedefault, addprefix=\global, 1=]{\mathrm{cof}\left(#1\right)}
\global\long\def\model{\vDash}
\global\long\def\crit{\mathrm{crit}}
\global\long\def\ult{\mathrm{Ult}}
\global\long\def\inj{\hookrightarrow}
\global\long\def\u{\mathcal{U}}
\global\long\def\dprime{\prime\prime}
\global\long\def\C{\mathbb{C}}
\global\long\def\v{\mathcal{V}}
\global\long\def\w{\mathcal{W}}
\global\long\def\i{\imath}
\global\long\def\P{\mathbb{P}}
\newcommandx\norm[1][usedefault, addprefix=\global, 1=]{\left\Vert #1\right\Vert }
\global\long\def\del{\partial}
\global\long\def\an{\mathrm{An}}
\global\long\def\I{\mathrm{I}}
\global\long\def\L{\mathcal{L}}
\global\long\def\D{\mathcal{D}}


Proving, after all, that $\C\left[\left[X\right]\right]$ is faithfully
flat over $\C\left\{ X\right\} $.

We do this by induction on the number of variables in $X$. We only
show that $\C\left[\left[X,T\right]\right]$ is flat over $\C\left\{ X,T\right\} $.\\
Assume that $\C\left[\left[X\right]\right]$ is flat over $\C\left\{ X\right\} $,
and consider $\C\left[\left[X,T\right]\right]$ and $\C\left\{ X,T\right\} $.
\\
Consider the equation 
\[
\left(\ast\right)_{0}\qquad f_{1}y_{1}+...+f_{n}y_{n}=0,\qquad\textrm{where }f_{i}\in\C\left\{ X,T\right\} .
\]
We can assume that all $f_{i}$'s are non zero, and regular in $T$.
\\
Apply W.P to $f_{i}$ in $\C\left\{ X,T\right\} $, $f_{i}=u_{i}w_{i}$
where $u_{i}\in\C\left\{ X,T\right\} ^{\times}$ and $w_{i}\in\C\left\{ X\right\} \left[T\right]$
is monic of degree $d_{i}$. Note that if $\left(y_{1},...,y_{n}\right)$
is a solution to $\left(\ast\right)_{0}$, then $\left(u_{1}^{-1}y_{1},...,u_{n}^{-1}y_{n}\right)$
is a solution to 
\[
\left(\ast\right)\qquad w_{1}y_{1}+...+w_{n}y_{n}=0.
\]
So it is enough to show that each solution to $\left(\ast\right)$
in $\C\left[\left[X,T\right]\right]$ is a linear combination of solutions
from $\C\left\{ X,T\right\} $.\\
Consider $z_{2}=\left(w_{2},-w_{1},0,...,0\right)$, $z_{3}=\left(w_{3},0,-w_{1},0,...,0\right)$,
... $z_{n}=\left(w_{n},0,...,0,-w_{1}\right)$. Each $z_{i}$ is a
solution to $\left(\ast\right)$ (and are in $\C\left\{ X,T\right\} $).\\
Now suppose $y=\left(y_{1},...,y_{n}\right)$ is a solution to $\left(\ast\right)$
in $\C\left[\left[X,T\right]\right]$. \\
For each $i$, write $y_{i}=q_{i}w_{1}+r_{i}$ where $q_{i}\in\C\left[\left[X,T\right]\right]$
and $r_{i}\in\C\left[\left[X\right]\right]\left[T\right]$ of degree
$<d_{1}$. Note that
\[
y+q_{2}z_{2}=\left(\ast,r_{2},y_{3},...,y_{n}\right).
\]
Similarly,
\[
y+q_{2}z_{2}+...+q_{n}z_{n}=\left(r_{1},r_{2},r_{3},...,r_{n}\right),\textrm{ for some }r_{1}.
\]
Let $r=\left(r_{1},...,r_{n}\right)$. $r$ is a solution to $\left(\ast\right)$
and it is enough to show that $r$ is a linear combination of solutions
from $\C\left\{ X,T\right\} $.\\
First we claim that $r_{1}$ is also in $\C\left[\left[X\right]\right]\left[T\right]$.\\
By $\left(\ast\right)$, $w_{1}r_{1}=-\left(w_{2}r_{2}+...+w_{n}r_{n}\right)$,
so $w_{1}r_{1}\in\C\left[\left[X\right]\right]\left[T\right]$. By
division in the ring of polynomials $\C\left[\left[X\right]\right]\left[T\right]$,
there are some $h,r\in\C\left[\left[X\right]\right]\left[T\right]$
with $\deg r<d_{1}$ s.t 
\[
w_{1}r_{1}=w_{1}h+r.
\]
In $\C\left[\left[X,T\right]\right]$, we have
\[
w_{1}r_{1}=w_{1}r_{1}+0.
\]
Thus by uniqueness of Weierstrass division, we have $r=0$ and $r_{1}=h\in\C\left[\left[X\right]\right]\left[T\right]$. 

So $r\in\left(\C\left[\left[X\right]\right]\left[T\right]\right)^{n}$
is a solution to $\left(\ast\right)$, and we need to show that $r$
is a linear combination (over $\C\left[\left[X,T\right]\right]$)
of solutions from $\C\left\{ X,T\right\} $. We claim that in fact
$r$ is a linear combination in $\C\left[\left[X\right]\right]\left[T\right]$
of solutions from $\C\left\{ X\right\} \left[T\right]$. \\
Recall that, by the inductive hypothesis, $\C\left[\left[X\right]\right]$
is flat over $\C\left\{ X\right\} $. Thus we have reduced the theorem
to the following:
\begin{prop*}
Suppose $R$ is flat over $S$, then $R\left[T\right]$ is flat over
$S\left[T\right]$.\end{prop*}
\begin{proof}
Consider a homogenous linear equation $f_{1}x_{1}+...+f_{n}x_{n}=0$
with $f_{i}\in S\left[T\right]$, and suppose $\left(x_{1},...,x_{n}\right)\in R\left[T\right]$
is a solution. \\
Take some $d\in\N$ s.t $\deg f_{i},\deg x_{i}<d$ for all $i$. Write
$x_{i}=\sum_{j=0}^{d-1}x_{ij}T^{j}$ and $f_{i}=\sum_{j=0}^{d-1}f_{ij}T^{j}$.
By multiplying the polynomials, and equating the coefficients of each
degree of $T$ to $0$, we get a system of (at most $d^{2}-1$) homogenous
linear equations involving the $x_{ij}$'s and $f_{ij}$'s. Then $\left\{ x_{ij}\right\} $
is a solution to a system of homogenous linear equations over $S$.
Since $R$ is flat over $S$, $\left\{ x_{ij}\right\} $ is a linear
combinations of solutions in $S$. That is, for some $\left\{ y_{ij}^{s}\right\} $,
$\alpha_{s}\in R$, $s=1,...,k$, we have $x_{ij}=\sum_{s}\alpha_{s}y_{ij}^{s}$
and $y_{ij}^{s}\in S$.\\
Now take $y_{i}^{s}=\sum_{j=0}^{d-1}y_{ij}T^{j}$. Then $x_{i}=\sum_{s}\alpha_{s}y_{i}^{s}$
where $\left(y_{1}^{s},...,y_{n}^{s}\right)$ are solutions in $S\left[T\right]$
to the original equation.
\end{proof}
~

Back to $\mathsection$11.

As a consequence of (11.1) , 
\[
\frac{\del f}{\del x_{k}}\left(b\right)=\lim_{h\lto0}\frac{f\left(b+he_{k}\right)-f\left(b\right)}{h},\quad\textrm{where }e_{k}=\left(0,...,1,...,0\right)\in\C^{m}.
\]
Let $\u\subset\R^{m}$ be open. A function $f\colon\u\lto\R$ is analytic
if for each $a\in\u$, there is some $r$ and $f_{a}\in\R\left\{ X\right\} _{r}$
s.t 
\[
f\left(X\right)=f_{a}\left(X-a\right),
\]
for $X$ close to $a$. Note that then $f$ is $C^{\infty}$ on $\u$
and 
\[
f_{a}\left(X\right)=\sum_{i}\frac{1}{i!}\left(\frac{\del^{i}f}{\del X^{i}}\right)\left(a\right)X^{i}.
\]
Also, $f\left(a+X\right)=f_{a}\left(X\right)$ for small $x$.

Examples:
\begin{itemize}
\item Polynomials.
\item $\exp$, $\sin$, $\cos$, on $\R$.
\item $\log$, $x^{a}$ (for $a\in\R$), on $\R^{>0}$.
\end{itemize}
Let $\mathrm{An}\left(\u\right)$ be all analytic functions on $\u$.
$\an\left(\u\right)$ has an $\R$-algebra structure. If $f\in\an\left(\u\right)$,
then $\frac{\del f}{\del X_{i}}\in\an\left(\u\right).$

For each $a\in\u$ there is a map of $\R$-algebras:
\begin{eqnarray*}
\an\left(\u\right) & \lto & \R\left\{ X\right\} .\\
f & \mapsto & f_{a}
\end{eqnarray*}

\begin{prop*}
(11.2) (Analytic continuation) If $\u$ is connected, then the map
above is injective. In particular, $\an\left(\u\right)$ is an integral
domain.\end{prop*}
\begin{cor*}
(11.3) If $\u$ is connected, $f,g\in\an\left(\u\right)$ agree on
a non empty open subset of $\u$, then $f=g$.\end{cor*}
\begin{prop*}
(11.4) Let $f_{1},...,f_{n}\in\an\left(\u\right)$, $\v\subset\R^{n}$
open s.t $f\left(\u\right)\subset V$, where $f=\left(f_{1},...,f_{n}\right)\colon\u\lto\R^{n}$.

Then for any $g\in\an\left(\v\right)$ we have $g\circ f\in\an\left(\u\right)$
and 
\[
\left(g\circ f\right)_{a}=g_{f\left(a\right)}\left(f_{a}-f\left(a\right)\right).
\]

\end{prop*}
(11.5) \uline{Notation}: For $x=\left(x_{1},...,x_{m}\right)\in\R^{m}$,
$\left|x\right|=\max\left\{ \left|x\right|_{1},...,\left|x_{m}\right|\right\} $.
\\
For $\delta>0$, let $\delta=\left(\delta,...,\delta\right)$ and
$\R\left\{ X\right\} _{\delta^{+}}=\bigcup_{r>\delta}\R\left\{ X\right\} _{r}$.\\
Each $f\in\R\left\{ X\right\} _{\delta^{+}}$ gives rise to a function
$x\mapsto f\left(x\right)\colon\bar{B}_{\delta}\left(0\right)\lto\R$
which extendes to an analytic function on an open nbhd of $\bar{B}_{\delta}$.

Let $Y=\left(Y_{1},...,Y_{n}\right)$, $n\geq1$. Fix $f=f\left(X,Y\right)\in\R\left\{ X,Y\right\} $.
Then for small $x$ we have $f\left(x,Y\right)\in\R\left\{ Y\right\} $.\\
Consider the following question: How does W.P for $f\left(x,Y\right)$
depend on $x$, for small $x$?\\
We will show that for some $\epsilon>0$, $\bar{B}_{\epsilon}\subset\R^{m}$
can be covered by finitely many ``special sets'', on each of which
W.P is uniform in $x$.
\begin{defn*}
(11.6) A special subset of $\bar{B}_{\epsilon}$ is a finite union
of sets of the following form 
\[
\left\{ x\in\bar{B}_{\epsilon};\, f\left(x\right)=0,\, g_{1}\left(x\right)>0,\,...,\, g_{k}\left(x\right)>0\right\} ,
\]
where $f,g_{1},...,g_{k}\in\R\left\{ X\right\} _{\epsilon^{+}}$.
\end{defn*}
We first show that $\mathrm{ord}\left(f\left(x,Y\right)\right)$ takes
only finitely many values as $x$ ranges over a neighbourhood of $0\in\R^{m}$.
Write
\[
f\left(X,Y\right)=\sum_{j}f_{j}\left(X\right)Y^{j},
\]
where $f_{j}\left(X\right)\in\R\left\{ X\right\} $. Since $\R\left\{ X\right\} $
is neotherian, the ideal generated by $\left\{ f_{j}\left(X\right)\right\} _{j\in\N^{n}}$
is finitely generated, hence is generated by $\left\{ f_{j}\left(X\right)\right\} _{\left|j\right|<d}$
for some $d\in\N$. \\
Thus there are $g_{ij}\in\R\left\{ X\right\} $ such that for every
$\left|j\right|>d$,
\[
f_{j}\left(X\right)=\sum_{\left|i\right|\leq d}g_{ij}f_{i}.
\]
In the following, $i$ ranges over elements of $\N^{n}$ s.t $\left|i\right|\leq d$
and $j$ over elements of $\N^{n}$ s.t $\left|j\right|>d$. \\
Substituting the above, we get 
\[
\left(1\right)\qquad f\left(X,Y\right)=\sum_{i}f_{i}\left(X,Y\right)\left(Y^{i}+\sum_{j}g_{ij}\left(X\right)Y^{j}\right),
\]
in $\R\left[\left[X,Y\right]\right]$. Note that for each $i,j$,
$g_{ij}\in\R\left\{ X\right\} $, and so there is some $\delta$ s.t
$g_{ij}\in\R\left\{ X\right\} _{\delta}$. However, in order to have
equation $\left(1\right)$ as an equality in $\R\left\{ X\right\} $,
we need to find a uniform $\delta$ that works for all $i,j$.
\begin{claim*}
$\exists\delta\in\left(0,1\right]$ and there are $f_{i},\, g_{ij}\in\R\left\{ X\right\} _{\delta^{+}}$
(maybe different than above) such that $\sum_{j}g_{ij}\left(X\right)Y^{j}\in\R\left\{ X,Y\right\} _{\delta^{+}}$
and $\left(1\right)$ holds in $\R\left\{ X\right\} _{\delta^{+}}$
with $f_{i},g_{ij}$.\end{claim*}
\begin{proof}
Consider the linear equation 
\[
f=\sum_{i}f_{i}\left(Y^{i}+\sum_{\left|j\right|=d+1}Z_{ij}Y^{j}\right).
\]
By $\left(1\right)$ above, there is a solution $\left\{ Z_{ij};\,\left|i\right|\leq d,\,\left|j\right|=d+1\right\} $
in $\R\left[\left[X,Y\right]\right]$ (all the higher terms, $Y^{j}$
for $\left|j\right|>d+1$, are inside the $Z_{ij}$'s). Since $\R\left[\left[X,Y\right]\right]$
is f.f. over $\R\left\{ X,Y\right\} $, there is a solution $\left\{ Z_{ij}\right\} $
in $\R\left\{ X,Y\right\} $. Take $\delta$ for all $i,j$, $Z_{ij}\in\R\left\{ X,Y\right\} _{\delta^{+}}$.
Now write $Z_{ij}$ as power series in $Y$ with coefficients in $\R\left\{ X\right\} $
to get $\left(1\right)$ with $g_{ij}\left(X\right)\in\R\left\{ X\right\} _{\delta^{+}}$
for all $\left|i\right|\leq d$, $\left|j\right|>d$.
\end{proof}
We work with equation $\left(1\right)$ in $\R\left\{ X,Y\right\} _{\delta^{+}}$
as given by the claim above. For $\left|x\right|\leq\delta$,
\[
f\left(x,Y\right)=0\iff\forall i\left(f_{i}\left(x\right)=0\right).
\]
Also, if $f_{i}\left(x\right)\neq0$, then $\mathrm{ord}\left(f\left(x,Y\right)\right)\leq\left|i\right|$.
So
\[
f\left(x,Y\right)\neq0\implies\mathrm{ord}\left(f\left(x,Y\right)\right)\leq d.
\]
Define 
\begin{eqnarray*}
Z_{\delta} & = & \left\{ x\in\bar{B}_{\delta};\,\forall i\, f_{i}\left(x\right)=0\right\} .\\
S_{i} & = & \left\{ x\in\bar{B}_{\delta};\, f_{i}\left(x\right)\neq0\wedge\forall i'\neq i\left(\left|f_{i}\left(x\right)\right|\geq\left|f_{i'}\left(x\right)\right|\right)\right\} .
\end{eqnarray*}
Note that $\bar{B}_{\delta}=Z_{\delta}\cup\bigcup_{i}S_{i}$.\\
Fix some $i$, formally divide the expression for $f$ in $\left(1\right)$
by $f_{i}$, and introduce new variables $V_{i,i'}$ for the quotients
$\nicefrac{f_{i}}{f_{i'}}$, for $i\neq i'$. Let $V_{i}=\left(V_{ii'}\right)_{i'\neq i}$.
Define
\[
F_{i}=Y^{i}+\sum_{j}g_{ij}Y^{j}+\sum_{i'\neq i}V_{ii'}\left(Y^{i'}+\sum_{j}g_{i'j}Y^{j}\right)\in\R\left\{ X,V_{i},Y\right\} .
\]
For $x\in S_{i}$, let $v_{i}\left(x\right)=\left(\nicefrac{f_{i'}\left(x\right)}{f_{i}\left(x\right)}\right)_{i'\neq i}$.
Then for $x\in S_{i}$, $\left|v_{i}\left(x\right)\right|\leq1$,
and 
\[
\left(2\right)\qquad f\left(x,Y\right)=f_{i}\left(x\right)F_{i}\left(x,v_{i}\left(x\right),Y\right).
\]


Idea: apply W.P to $F_{i}$'s locally around every point $X=0$, $V_{i}=c$,
$Y=0$. \\
For $c=\left(c_{i'}\right)_{i'\neq i}$ with $\left|c\right|\leq1$,
put 
\[
F_{i,c}=F_{i}\left(X,c+V_{i},Y\right).
\]
Then for $x\in S_{i}$,
\[
f\left(x,Y\right)=f_{i}\left(x\right)F_{i,c}\left(x,v_{i}\left(x\right)-c,Y\right).
\]

\begin{claim*}
For each $c$ there is $\lambda=\lambda\left(c\right)\in\R^{n-1}$
with $\left|\lambda\right|\leq1$ such that $F_{i,c}\left(X,V_{i},\lambda\left(Y\right)\right)$
is regular in $Y_{n}$ of order$\leq\left|i\right|$, where for $Y=\left(Y_{1},...,Y_{n}\right)$,
$\lambda\left(Y\right)=\left(Y_{1}+\lambda Y_{n},...,Y_{n-1}+\lambda_{n-1}Y_{n},Y_{n}\right)$.\end{claim*}
\begin{proof}
Exercise.
\end{proof}
By W.P., for such $\lambda$,
\[
\left(3\right)\qquad F_{i,c}\left(X,V_{i},\lambda\left(Y\right)\right)=\u_{i,c}\w_{i,c},\quad\u_{i,c}\in\R\left\{ X,V_{i},Y\right\} ^{\ast},\:\w_{i,c}\in\R\left\{ X,V_{i},Y_{1},...,Y_{n-1}\right\} \left[Y_{n}\right].
\]
Take $\epsilon\left(i,c\right)\in\left(0,\delta\right]$ s.t 
\begin{itemize}
\item $\u_{i,c}\in\R\left\{ X,V_{i},Y\right\} _{\epsilon\left(i,c\right)^{+}}^{\ast}$.
\item $\w_{i,c}\in\R\left\{ X,V_{i},Y\right\} _{\epsilon\left(i,c\right)^{+}}.$
\end{itemize}
Let $\Gamma\left(i\right)=\left\{ i';\, i'\neq i\right\} $. Note
that our $c$'s vary over $I^{\Gamma\left(i\right)}$, where $\I=\left[-1,1\right]$.
By compactness, there is a finite set $C\left(i\right)$ s.t
\[
\I^{\Gamma\left(i\right)}\subset\bigcup_{c\in C\left(i\right)}B_{\epsilon\left(i,c\right)}\left(c\right).
\]
Now consider the finite set $\Gamma=\left\{ \left(i,c\right);\,\left|i\right|\leq d,\, d\in C\left(i\right)\right\} $.\\
Take $\epsilon>0$ s.t $\epsilon\leq\frac{\epsilon\left(i,c\right)}{4}$
for all $\left(i,c\right)\in\Gamma.$ For $\gamma=\left(i,c\right)\in\Gamma$,
let 
\[
S_{\gamma}=\left\{ x\in S_{i};\,\left|x\right|\leq\epsilon,\,\left|v_{i}\left(x\right)-c\right|<\epsilon\left(i,c\right)\right\} .
\]
Then $S_{i}\cap\bar{B}_{\epsilon}=\bigcup\left\{ S_{\gamma};\,\gamma=\left(i,c\right),\, c\in C\left(i\right)\right\} $.
\\
So $\bar{B}_{\epsilon}=\left(\underbrace{Z_{\delta}\cap\bar{B}_{\epsilon}}_{\equiv Z}\right)\cup\bigcup_{\gamma}S_{\gamma}$.\\
For $\gamma=\left(i,c\right)\in\Gamma$, by $\left(2\right)$ and
$\left(3\right)$,
\[
f\left(x,\lambda\left(Y\right)\right)=f_{i}\left(x\right)\u_{\gamma}\left(x,v_{i}\left(x\right),Y\right)W_{\gamma}\left(x,v_{i}\left(x\right),Y\right).
\]
Note: $\u_{\gamma}$ does not change sign on $\bar{B}_{\epsilon\left(\gamma\right)}$.
Therefore there is $\sigma\left(\gamma\right)\in\left\{ \pm1\right\} $
s.t
\[
\mathrm{sign}\left(f\left(x,\lambda\left(y\right)\right)\right)=\sigma\left(\gamma\right)\mathrm{sign}\left(f_{i}\left(x\right)\w_{\gamma}\left(x,v_{i}\left(x\right),y\right)\right)
\]
for $x\in S_{\gamma}$ and $\left|y\right|\leq2\epsilon$. Note that
$f\left(x,\lambda\left(y\right)\right)$ is defined since $\left|\lambda\right|\leq1$,
so $\left|\lambda\left(y\right)\right|\leq4\epsilon\leq\delta$.\\
For $\left|y\right|\leq\epsilon$, $\left|\left(-\lambda\right)\left(y\right)\right|\leq2\epsilon$.
Also, $\lambda\left(\left(-\lambda\right)\left(y\right)\right)=y$.
Thus for $\left|y\right|\leq\epsilon$
\[
\mathrm{sign}\left(f\left(x,y\right)\right)=\sigma\left(\gamma\right)\mathrm{sign}\left(f_{i}\left(x\right)\w_{\gamma}\left(x,v_{i}\left(x\right),\left(-\lambda\right)\left(y\right)\right)\right).
\]
Let $Z=\left(Z_{1},...,Z_{n}\right)$ be new variables and 
\[
\hat{\w}_{\gamma}\left(X,V_{i},Z\right)=f_{i}\left(\epsilon X\right)\w_{\gamma}\left(\epsilon X,\epsilon\left(\gamma\right)V_{\gamma},2\epsilon Z\right)\in\R\left\{ X,V_{i},Z\right\} _{1^{+}}.
\]
Then if $\left(x,y\right)\in I^{m+n}$ and $\epsilon x\in S_{\gamma}$,
then 
\[
\mathrm{sign}f\left(x,y\right)=\sigma\left(\gamma\right)\hat{\w}_{\gamma}\left(x,\nicefrac{v_{i}\left(\epsilon x\right)}{\epsilon\left(\gamma\right)},\nicefrac{\left(-\lambda\right)\left(y\right)}{2}\right).
\]



\subsection*{$\mathsection12$. Quantifier elimination for $\R_{an}$.}

Recall that $\L_{an}$ is the language $\left\{ 0,1,+,-,\cdot,\leq\right\} $
of ordered rings augmented by a function symbol for every restricted
analytic function $\R^{m}\lto\R$. \\
Let $\R_{an}$ be the ordered field of reals as an $\L_{an}$-structure
together with the map $^{-1}\colon\R\lto\R$ defined as $x^{-1}=\begin{cases}
\frac{1}{x} & x\neq0\\
x=0 & 0
\end{cases}$. Let $\L_{an}\left(^{-1}\right)=\L_{an}\cup\left\{ ^{-1}\right\} $.
\begin{thm*}
(12.1) (Denef-v. d. Dries 1988) $\left(\R_{an},\,^{-1}\right)$ has
quantifier elimination.
\end{thm*}
Call $f\colon\I^{m}\lto\R$ analytic if it extends to an analytic
function on an open nbhd of $\I^{m}$. Let $\L_{a}$ be the language
$\left\{ <\right\} $ expanded by $m$-ary function symbols for each
analytic $f\colon\I^{m}\lto\R$ with $f\left(\I^{m}\right)\subset\I$.\\
We consider $\I$ as an $\L_{a}$-structure.\\
Define $\D\colon\I^{2}\lto\I$ by
\[
D\left(x,y\right)=\begin{cases}
\nicefrac{x}{y} & \textrm{if }\left|x\right|\leq\left|y\right|\textrm{ and }y\neq0,\\
0 & \textrm{otherwise.}
\end{cases}
\]
Let $\L_{a,\D}=\L_{a}\cup\left\{ \D\right\} $.
\begin{thm*}
(12.2) The $\L_{a,\D}$-structure $\I$ has $q.e.$
\end{thm*}
Some general logical considerations:\\
Let $T$ be an $\L$-theory, and $T'$ be a definitional expansion
of $T$ to an $\L'$-theory. Assume further that $\L'\setminus\L$
consists only of function symbols, and for each $f\in\L'\setminus\L$
there is an existential $\L$-formula $\delta_{f}\left(x,y\right)$
s.t 
\[
T'\vdash f\left(x\right)=y\longleftrightarrow\delta_{f}\left(x,y\right).
\]
(e.g. $T=\mathrm{Th}_{\L_{a}}\left(\I\right)$ and $T'=\mathrm{Th}_{\L_{a,\D}}\left(\I\right)$.)
\begin{lem*}
(12.6) Every $\exists\L'$-formula is $T'$-equivalent to an $\exists\L$-formula.
\begin{lem*}
(12.7) Suppose that for each q.f. $\L$-formula $\varphi\left(x,y_{1},...,y_{n}\right)$,
$n\geq1$, there is a q.f. $\L'$-formula $\varphi'$$\left(x,z_{1},...,z_{n-1}\right)$
such that 
\begin{enumerate}
\item $T'\vdash\exists y\varphi\left(x,y\right)\longleftrightarrow\exists z\varphi'\left(x,z\right)$.
\item The function symbols in $\L'\setminus\L$ are only applied in $\varphi'$
to terms involving only $x$.
\end{enumerate}
\end{lem*}
Then $T'$ has $q.e$. (So by $\left(12.6\right)$, $T$ is model
complete).
\end{lem*}
We now verify that the conditions in lemma $\left(12.7\right)$ are
satisfied for $T=\mathrm{Th}_{\L_{a}}\left(\I\right)$ and $T'=\mathrm{Th}_{\L_{a,\D}}\left(\I\right)$.
\begin{lem*}
(12.8) Basic Lemma.

Let $\varphi\left(x,y\right)=\varphi\left(x_{1},...,x_{m},y_{1},...,y_{n}\right)$,
$n\geq1$ be a q.f. $\L_{a}$-formula. Then there is a q.f. $\L_{a,\D}$-formula
$\hat{\varphi}\left(x,z\right)$, $z=\left(z_{1},...,z_{n}\right)$
such that
\begin{enumerate}
\item $\I\model\exists y\varphi\left(x,y\right)\longleftrightarrow\exists z\hat{\varphi}\left(x,z\right)$.
\item In $\hat{\varphi}$, $\D$ is only applied to terms not involving
$z$, and $z_{n}$ only appears polynomially in $\hat{\varphi}$.
\end{enumerate}
\end{lem*}
Given the Basic Lemma, we can use Tarsky's quantifier elimination
for $\R$ (as an ordered ring) to eliminate $\exists z_{n}$ and produce
$\varphi'$ satisfying the hypothesis of $\left(12.7\right)$.

For an $\L_{a}$-formula and $\left(a,b\right)\in\I^{m+n}$, $\eta>0$,
consider the formula 
\[
\varphi_{a,b,\eta}=\varphi\wedge``\left|\left(x,y\right)-\left(a,b\right)\right|<\eta",
\]
where $``"$ means the formal sentence for that phrase. By compactness
of $I^{m+n}$, it is enough to show that for any $\left(a,b\right)$,
there is $\eta>0$ such that the Basic Lemma holds for the formula
$\varphi_{a,b,\eta}$. Using an affine transformation, it is enough
to consider only $\left(a,b\right)=\left(0,0\right)$. Let $\varphi_{\eta}=\varphi_{0,0,\eta}$.
Thus the lemma is reduced to the following:
\begin{lem*}
$\left(12.9\right)$ Local Basic Lemma. 

Let $\varphi\left(x,y\right)$ be a q.f. $\L_{a}$-formula . Then
there is $\epsilon\in\left(0,1\right)$ and a q.f. $\L_{a,\D}$ formula
$\varphi'\left(x,z\right)$, $z=\left(z_{1},...,z_{n}\right)$, s.t 
\begin{enumerate}
\item $\I\model\exists y\varphi_{\epsilon}\iff\exists z\left(\varphi_{\epsilon}^{\prime}\right)$.
\item In $\varphi'$, the function symbol $\D$ is applied only to terms
not involving $z$, and $z_{n}$ appears only polynomially.
\end{enumerate}
\end{lem*}
\begin{proof}
We may assume that all atomic subformulae of $\varphi$ are of the
form $f\left(x,y\right)>0$ or $f\left(x,y\right)=0$ for some analytic
$f\colon\I^{m}\lto\I$.\\
Apply uniform W.P. to the Taylor series at $0$ of all such $f$.
We get \end{proof}
\begin{itemize}
\item $\epsilon\in\left(0,1\right)$,
\item Finite cover $S_{\gamma}$ of $B_{\epsilon}\subset\R^{m}$ by special
sets.
\item For each $\gamma$, a $\lambda=\lambda\left(\gamma\right)\in\R^{n-1}$
with $\left|\lambda\right|\leq1$.
\item For each $f$ in $\varphi$, an $\L_{a,\D}$-term $t_{\gamma,f}\left(x,z\right)$;
\item $t_{f,\gamma}$ is polynomial in $z_{n}$,
\item $\D$ is only applied to terms involving only $x$ within $t_{f,\gamma}$.
\item $\forall x\in I^{m+n}$ with $\epsilon x\in S_{\gamma}$,
\[
f\left(\epsilon x,\epsilon y\right)\geq0\iff t_{f,\gamma}\left(x,\nicefrac{\left(-\lambda\right)\left(y\right)}{2}\right)\geq0.
\]

\end{itemize}
Replace $f\left(x,y\right)$ by $t_{f,\gamma}$ in $\varphi$ to get
$\varphi_{\gamma}$. Then 
\[
\I\model\exists y\varphi\left(\epsilon x,\epsilon y\right)\longleftrightarrow\exists z\left(\bigvee_{\gamma}x\in S_{\gamma}\wedge\varphi_{\gamma}\left(x,z\right)\wedge\left|\lambda\left(z\right)\right|\leq\frac{1}{2}\right).
\]
Finally we can convert this to a formula $\varphi'$ satisfying $\left(1\right)$
and $\left(2\right)$ of the Local Basic Lemma.
\end{document}

%\WikiLevelTwo{ Siddharth's extra lectures }
\section{ Siddharth's extra lectures }
\NotesBy{notes by Bill Chen}

A crucial concept for these lectures will be \WikiItalic{games}. These games where two players Left and Right alternate moves, and a player loses if she has no moves. The information of who goes first is not encoded into the game. Formally, a game $G$ consists of two sets of games, $G=\{G_L|G_R\}$, where the left side consists of the valid games which Left can move to, and similarly for the right side. (We use the "typical element" notation for sets, which carries over from the notation for surreal numbers.)

\begin{example} %====Example====
\begin{enumerate}
  \item  $\{\emptyset|\emptyset\}$ is called the zero game (abbreviated 0). There are no legal moves for either player, and the first player to move loses.
  \item  $\{\emptyset|0\}$ is the game 1. In this game, Left has no legal moves, and Right can move to the 0 game, so Right has a winning strategy no matter who moves first.
  \item  $\{0|\emptyset\}$. Here Left has a winning strategy.
  \item  $\{0|0\}$ First player to move wins. This is a valid game which is not a surreal number as we defined in Week 2.
\end{enumerate}
\end{example}

\begin{definition} % ====Definition 1==== 
\begin{enumerate}
  \item  $G>0$ if Left has a winning strategy.
  \item  $G<0$ if Right has a winning strategy.
  \item  $G\sim 0$ if the second player has a winning strategy. ($G$ is ''similar'' to $0$.)
  \item  $G\parallel 0$ if the first player has a winning strategy. ($G$ is ''fuzzy''.)
  \item  $G\ge 0$ means $G>0$ or $G\sim 0$.
\end{enumerate}
 \end{definition}

\begin{lemma} % ====Lemma 2 (Determinacy)==== 
For any game $G$, one of $G>0$, $G<0$, $G\sim 0$, or $G\parallel 0$ holds.
 \end{lemma}

\begin{proof} %\WikiBold{Proof:} 
Let $A$ be the assertion that there is a $G_L$ with $G_L\ge 0$, and $B$ be the assertion that there is a $G_R$ with $G_R\le 0$.

Then one can check that $G>0$ iff $A\& \neg B$, $G<0$ iff $\neg A \& B$, $G\sim 0$ iff $\neg A \& \neg B$, and $G\parallel 0$ iff $A\& B$. For example, if $A \& B$ holds, then the first player can move to a game that is positive or similar $0$. In the first case, the first player clearly wins. In the second case, the first player becomes the second player of the new game similar to $0$, and hence wins.
 \end{proof}

\begin{definition} % ====Definition 3==== 
 \end{definition}
If $G,H$ are games, the \WikiItalic{disjunctive sum} $G+H$ is the game in which $G$ and $H$ are "played in parallel." Formally,
$$G+H=\{G_L+H,G+H_L|G_R+H, G+H_R\}.$$

\begin{remark} %\WikiBold{Remark:}
By induction, can prove that $+$ is associative and commutative. 
\end{remark}

\begin{definition} % ====Definition 4==== 
If $G$ is a game, the \WikiItalic{negation} $-G$ is the game obtained by switching the roles of Left and Right. Formally,
$$-G=\{-G_R|-G_L\}.$$
 \end{definition}

Notice that these are the same definitions as for surreal numbers.

\begin{lemma} % ====Lemma 5 (Basic properties of $+$ and $-$)==== 
\begin{enumerate}
  \item  $-(G+H)=-G+-H$.
  \item  $--G=G$.
  \item  $G\sim 0$ iff $-G\sim 0$.
  \item  $G>0$ iff $-G<0$.
  \item  $G\parallel 0$ iff $-G\parallel 0$.
\end{enumerate}
 \end{lemma}

We won't prove this lemma, but it is not difficult.

\begin{lemma} % ====Lemma 6==== 
Let $H\sim 0$. Then:
\begin{enumerate}
  \item  If $G\sim 0$, then $G+H\sim 0$.
  \item  If $G>0$, then $G+H>0$.
  \item  If $G\parallel 0$, then $G+H\parallel 0$.
  \item  If $G+H\sim 0$, then $G\sim 0$.
  \item  If $G+H>0$, then $G>0$.
  \item  If $G+H\parallel 0$, then $G\parallel 0$.
\end{enumerate}
 \end{lemma}

\begin{proof} %\WikiBold{Proof:} 

Formally, this is proved by induction. We just describe the strategies in words.

For (1), if the second player has a winning strategy in $G$ and $H$, then the second player can use the winning strategy corresponding to the game in which the first player plays in.

For (2), the proof splits into cases. If Right moves first, either Right moves in $H$, so Left can play according to the second player's strategy in the $H$ game, or Right moves in $G$, so Left can play according to his strategy in $G$. If Left moves first, he plays according to his strategy in $G$ and then according to the previous sentence against the subsequent moves of Right.

(3), is a similar analysis.
 \end{proof}

The next three follow from the first three by using cases based on determinacy.

\begin{lemma} % ====Lemma 7==== 
\begin{enumerate}
  \item  $G+ -G\sim 0$.
  \item  If $G>0$ and $H>0$ then $G+H>0$.
\end{enumerate}
 \end{lemma}

\begin{proof} %\WikiBold{Proof:} 

The first assertion follows from the strategy of "playing Go on two boards against the same person."
 \end{proof}


\begin{definition} % ====Definition 8==== 
 \end{definition}


\end{document}