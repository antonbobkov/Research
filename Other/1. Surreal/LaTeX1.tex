\documentclass{article}

\usepackage{amsmath}
\usepackage{amssymb}
\usepackage{amsthm}
\usepackage{enumerate}
\usepackage{colonequals}
\usepackage{fullpage}

\newtheorem{theorem}{Theorem}
\newtheorem{defn}{Definition}
\newtheorem{cor}{Corollary}
\newtheorem{claim}{Claim}
\newtheorem{lem}{Lemma}
\newcommand{\R}{\mathbb{R}}
\newcommand{\concat}{\mathbin{\raisebox{1ex}{\scalebox{.7}{$\frown$}}}} %sequence concatenation
\newcommand{\dom}[1]{\operatorname{dom}\paren{#1}}
\newcommand{\ZFC}{\mathsf{ZFC}}
\newcommand{\NBG}{\mathsf{NBG}}
\newcommand{\coloneq}{\colonequals}
\newcommand{\N}{\mathbb{N}}

\newcommand{\No}{\mathbf{No}}
\newcommand{\On}{\mathbf{On}}
\newcommand{\paren}[1]{\left( #1 \right)}
\newcommand{\brac}[1]{\left[ #1 \right]}
\newcommand{\curly}[1]{\left\{ #1 \right\}}
\newcommand{\abs}[1]{\left| #1 \right|}
\newcommand{\rar}{\rightarrow}
\newcommand{\arr}{\rightarrow}

\DeclareMathOperator{\supp}{supp}
\DeclareMathOperator{\lt}{lt}

\newcommand{\w}{\omega}
\newcommand{\midr}[1]{\restriction_{#1}}


\title{Notes on Surreal Numbers \\ Math 285: Fall 2014}
\author{Class Taught by Prof. Aschenbrenner \\ Notes by John Susice}
\date{\today}
\begin{document}
\maketitle{}

Lemma Let $f \in K$ be such that $l(f) = \w \cdot \alpha$. Then $l(f(\w)) \geq \alpha$.

Proof

Suppose $\beta < \alpha$. Then the elements used to define $\sum_{i < \w\cdot\beta} (\ldots)$ also appear in the cut for $\sum_{i < \w\cdot\alpha} (\ldots) = f(\w)$. Thus by uniqueness of the normal form.
\begin{align*}
	l\paren{\sum_{i < \w\cdot\beta} (\ldots)} < l(f(\w))
\end{align*}
So the map $\phi(\beta) = l(\sum_{i < \w\cdot\beta} (\ldots))$ is strictly increasing.
Any strictly increasing map $\phi$ on an initial segment of $\On$ satisfies $\phi(\beta) \geq \beta$.


Lemma The map

\begin{align*}
	K &\arr \No \\
	f(x) &\mapsto f(\w)
\end{align*}

is onto.

Proof

Let $a \in \No, a \neq 0$. By (5.6) there is a unique $b \in \No$ such that $\brac{a} = \brac{\w^b}$.
Put
\begin{align*}
	S = \curly{s \in \R \colon s\w^b \leq a}
\end{align*}
Then $S \neq \emptyset$ and bounded from above.
Put $r = \sup S \in \R$.
Then
\begin{align*}
	(r + \epsilon)\w^b > a > (r - \epsilon)\w^b
\end{align*}
for all $\epsilon \in \R^{>0}$
thus
\begin{align*}
	\abs{a - r\w^b} << \w^b \tag{*}
\end{align*}

Note $r \neq 0$; $r,b$ subject to $(*)$ are unique.

We set $\lt(a) = r\w^b$

Towards a contradiction assume that $a$ is not in the image of $f(x) \mapsto f(\w)$.
We shall inductively define a sequence $(a_i, f_i)_{i \in \On}$ where 

\begin{itemize}
	\item $a_i \in \No$ is strictly decreasing; $f_i \in \R - \{0\}$
	\item $f_\alpha \w^{a_\alpha} = \lt\paren{a - \sum_{i < \alpha} f_i\w^{a_i}}$ for all $\alpha \in \On$
\end{itemize}

$\alpha = \beta + 1$

Take $(a_\alpha, f_\alpha)$ so that
\begin{align*}
	f_\alpha\w^{a_\alpha} = \lt\paren{a - \sum_{i < \alpha}(\ldots)}
\end{align*}

By inductive hypothesis, if $\beta < \alpha$

\begin{align*}
	f_\beta\w^{a_\beta} &= \lt\paren{a - \sum_{i < \beta}(\ldots)} \\
	\Rightarrow f_\alpha\w^{a_\alpha} &= \lt\paren{\paren{a - \sum_{i < \beta}(\ldots)} - f_\beta\w^{a_\beta}} \\
	&<< \w^{a_\beta} \text{by (*)} \\
	\Rightarrow a_\alpha &< a_\beta
\end{align*}

$\alpha$ limit

Take $(a_\alpha, f_\alpha)$ as above.
Let $\beta < \alpha$; to show $a_\alpha < a_\beta$.
We have

\begin{align*}
	a - \sum_{i leq \beta} f_i \w^{a_i} = \paren{a - \sum_{i < \beta} f_i\w^{a_i}} - f_\beta\w^{a_\beta} << \w^{a_\beta}
\end{align*}

By the tail property
\begin{align*}
	&\sum_{i < \alpha} (\ldots) - \sum_{i \leq \beta} (\ldots) << \w^{a_\beta} \\
	\Rightarrow &a - \sum_{i < \alpha} (\ldots) << \w^{a_\beta} \\
	\Rightarrow &\lt\paren{a - \sum_{i < \alpha} (\ldots) } << \w^{a_\beta} \\
	\Rightarrow &a_\alpha < a_\beta
\end{align*}

This completes the induction.
Note that we showed that if $\alpha$ is a limit then $a - \sum_{i < \alpha} (\ldots) << \w^{a_\beta}$ for all $\beta < \alpha$.

So if $\curly{L \mid R}$ is the cut used to define $\sum_{i < \alpha} (\ldots)$ then $L < a < R$.
Let $\alpha = \w \cdot \alpha'$.
Hence 
\begin{align*}
	l(a) > l\paren{\sum_{i < \w\cdot\alpha'} (\ldots)} \geq \alpha'
\end{align*}
by (6.1).
So $l(a)$ is bigger than all limits - contradiction.

Lemma Let $r \in \R, a \in \No$. Then 

\begin{align*}
	r\w^a = \{(r - \epsilon) \w^a \mid (r+\epsilon)\w^a\}
\end{align*}
where $\epsilon$ ranges over $\R^{>0}$.

Proof
\begin{align*}
	r &= \{r - \epsilon \mid r + \epsilon\} \\
	\w^a &= \curly{0, s\w^{a_L} \mid t\w^{a_R}} \text{ where } s,t \in \R^{>0}
\end{align*}
\begin{align*}
	r\w^a = \{
	&(r - \epsilon) \w^a, (r - \epsilon)\w^a + \epsilon s \w^{a_L}, \\
	&(r + \epsilon) \w^a - \epsilon t \w^{a_R} \mid \\
	&(r + \epsilon) \w^a, (r + \epsilon)\w^a - \epsilon s \w^{a_R}, \\
	&(r - \epsilon) \w^a + \epsilon t \w^{a_L} \}
\end{align*}
Now use $\w^{a_L} << \w^a << \w^{a_R}$ and cofinality.

Corollary 6.5
\begin{align*}
	\sum_{i \leq \alpha} f_i\w^{a_i} =
	\curly{ \sum_{i < \alpha} f_i\w^{a_i} + (f_i - \epsilon) \w^{a_\alpha} \mid
	\sum_{i < \alpha} f_i\w^{a_i} + (f_i + \epsilon) \w^{a_\alpha} }
\end{align*}

Proof

Suppose first that $\alpha$ is a limit
\begin{align*}
	&\sum_{i \leq \alpha} f_i\w^{a_i} = \sum_{i < \alpha} f_i\w^{a_i} + f_\alpha\w^{a_\alpha} \underset{(6.4)}{=} \\
	&\curly{\sum_{i < \beta} f_i \w^{a_i} + (f_\beta - \epsilon)\w^{a_\beta} \mid
	\sum_{i < \beta} f_i \w^{a_i} + (f_\beta + \epsilon)\w^{a_\beta}}_{\beta < \alpha, \epsilon \in \R^{>0}}
	+ \curly{(r - \epsilon) \w^\alpha \mid (r+\epsilon)\w^\alpha} = \\
	\curly{\sum_{i \leq \beta} f_i \w^{a_i} - \epsilon\w^{a_\beta} + f_\alpha\w^{a_\alpha}, \sum_{i < \alpha} f_i \w^{a_i}}
\end{align*}

\end{document}

