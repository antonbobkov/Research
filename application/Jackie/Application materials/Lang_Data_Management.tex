\documentclass[11pt]{amsart}

\usepackage{amsmath, amsthm, amssymb, amsopn, amsfonts, amscd, fullpage, url}
\usepackage{delarray}
\usepackage{enumerate}

\def\Aa{\mathfrak{a}}
\def\cc{\mathfrak{c}}
\def\F{\mathbb{F}}
\def\I{\mathbb{I}}
\def\aL{\mathcal{L}}
\def\m{\mathfrak{m}}
\def\OK{\mathcal{O}}
\def\p{\mathfrak{p}}
\def\Pp{\mathfrak{P}}
\def\Q{\mathbb{Q}}
\def\Z{\mathbb{Z}}

\usepackage{mdwlist}

\title{Data Management Plan}
\author{\textbf{Jaclyn Lang}}
\begin{document}
\maketitle

\section{Types of data}
With the exception of the second research objective, the proposed project is theoretical in nature.  As such, the most likely type of data to be produced is papers that will be published in peer-reviewed mathematical journals.

The second research objective will produce both programs in the open-source mathematical software Sage as well as a large set of data consisting of the Galois level of many classical modular forms.

\section{Access to data}
Preprints of the papers that are produced will be immediately made available on the website of the PI as well as posted to the arXiv e-print server.  This will give other mathematical researchers and the public the ability to access the new results while they go through an often lengthy peer-review process.  The papers will be promptly submitted to a peer-reviewed mathematical journal for publication.

Regarding the second research objective, the PI plans to make the Sage programs and the data they produce on the Galois level of classical modular forms publicly available on her webpage.  Furthermore, she plans to add the data to the $L$-functions and Modular Forms Database (\url{http://www.lmfdb.org/}), a website with an extensive and interactive database of mathematical objects arising in number theory.

\end{document}