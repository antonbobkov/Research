\documentclass[12pt]{amsart}

\usepackage{fullpage}
%\usepackage{a4wide}
\usepackage{amsmath}
\usepackage{amsfonts}
\usepackage{amssymb}
\usepackage[all]{xy}
\usepackage{mathrsfs}
\usepackage{stmaryrd}
\usepackage{oldgerm}
\usepackage[french]{babel}
\usepackage[T1]{fontenc}
\usepackage[latin1]{inputenc}
\usepackage{bbm}
%\usepackage{hyperref}
%\usepackage{pstricks}
%\usepackage{showkeys}
\usepackage{ulem}

%\headsep=0.05cm
%\oddsidemargin=0.26cm
%\textwidth 13cm
%\textheight 19cm
%\voffset -1cm
 
\def \ooverline #1#2#3%
{\mkern #1mu \overline{\mkern -#1mu #3 \mkern -#2mu }\mkern #2mu }

\def \Kbar {\ooverline40K{\mkern 1mu}{}}
\def \Rbar {\ooverline40R{\mkern 1mu}{}}
\def \Mbar {\ooverline40M{\mkern 1mu}{}}
\def \Nbar {\ooverline40N{\mkern 1mu}{}}
\def \Rhat {\widehat{\mkern -3mu \vrule height 1.9ex width 0pt \smash{R}}}
\def \Rbarhat {\widehat{\mkern -3mu \vrule height 1.9ex width 0pt \smash{\Rbar}}}
\def \Kbarhat {\widehat{\mkern -3mu \vrule height 1.9ex width 0pt \smash{\Kbar}}}

\DeclareMathOperator{\GG}{\mathbf{G}} \DeclareMathOperator{\ZZ}{\mathbf{Z}} \DeclareMathOperator{\QQ}{\mathbf{Q}} \DeclareMathOperator{\FF}{\mathbf{F}}
\DeclareMathOperator{\NN}{\mathbf{N}} \DeclareMathOperator{\RR}{\mathbf{R}} \DeclareMathOperator{\CC}{\mathbf{C}}

\DeclareMathOperator{\Frac}{Frac} \DeclareMathOperator{\Gal}{Gal} \DeclareMathOperator{\rg}{rg} \DeclareMathOperator{\et}{\textrm{\'et}} \DeclareMathOperator{\cris}{cris} \DeclareMathOperator{\HT}{HT} \DeclareMathOperator{\rig}{rig} \DeclareMathOperator{\lcris}{log-cris}
\DeclareMathOperator{\dR}{dR} \DeclareMathOperator{\ldR}{log-dR} \DeclareMathOperator{\f}{f} \DeclareMathOperator{\Isom}{Isom} \DeclareMathOperator{\T}{T}

\DeclareMathOperator{\End}{\mathsf{End}} \DeclareMathOperator{\Hom}{\mathsf{Hom}} \DeclareMathOperator{\Aut}{\mathsf{Aut}}
\DeclareMathOperator{\Ker}{Ker} \DeclareMathOperator{\im}{Im} \DeclareMathOperator{\Coker}{Coker} \DeclareMathOperator{\Id}{Id} \DeclareMathOperator{\I}{I}
\DeclareMathOperator{\isomto}{\overset{\sim}{\to}} \DeclareMathOperator{\from}{\leftarrow} \DeclareMathOperator{\dual}{\mathtt{D}}
\DeclareMathOperator{\Tor}{Tor} \DeclareMathOperator{\Ext}{Ext} \renewcommand{\H}{\mathrm{H}}

\DeclareMathOperator{\Spec}{Spec} \DeclareMathOperator{\Spf}{Spf} \DeclareMathOperator{\Spm}{Spm} \DeclareMathOperator{\D}{D} \DeclareMathOperator{\Fil}{Fil}

\DeclareMathOperator{\W}{W} \DeclareMathOperator{\A}{A} \DeclareMathOperator{\dd}{d} \DeclareMathOperator{\Hdg}{Hdg} \DeclareMathOperator{\Lie}{Lie}

\DeclareMathOperator{\bV}{\mathbf{V}} \DeclareMathOperator{\bL}{\mathbf{L}} \DeclareMathOperator{\bT}{\mathbf{T}}

\DeclareMathOperator{\Mat}{\mathsf{M}} \DeclareMathOperator{\GL}{\mathsf{GL}} \DeclareMathOperator{\GSp}{\mathsf{GSp}} \DeclareMathOperator{\Sp}{\mathsf{Sp}} 

\DeclareMathOperator{\Tr}{Tr} \DeclareMathOperator{\Int}{Int}

\newcommand{\loccit}{\textit{loc. cit.}}
\newcommand{\cf}{\textit{cf }}
\newcommand{\ie}{\textit{i.e. }}
\newcommand{\idem}{\textit{idem }}
\newcommand{\ops}{\textit{on peut supposer }}

\newcommand{\bC}{\mathbb{C}}
\newcommand{\bG}{\mathbb{G}}
\newcommand{\bR}{\mathbb{R}}
\newcommand{\bH}{\mathbb{H}}
\newcommand{\cM}{\mathcal{M}}
\newcommand{\cU}{\mathcal{U}}
\newcommand{\cP}{\mathcal{P}}
\newcommand{\cB}{\mathcal{B}}
\newcommand{\cG}{\mathcal{G}}
\newcommand{\cN}{\mathcal{N}}
\newcommand{\bQ}{\mathbb{Q}}
\newcommand{\cK}{\mathcal{K}}
\newcommand{\bh}{\mathbf{h}}
\newcommand{\bd}{\mathbf{d}}
\newcommand{\bK}{\mathbb{K}}
\newcommand{\bZ}{\mathbb{Z}}
\newcommand{\BbV}{\mathbb{V}}
\newcommand{\bw}{\mathbf{w}}
\newcommand{\bF}{\mathbf{F}}
\newcommand{\cF}{\mathcal{F}}
\newcommand{\cO}{\mathcal{O}}
\newcommand{\cR}{\mathcal{R}}
\newcommand{\cD}{\mathcal{D}}
\newcommand{\lU}{{}_U}
\newcommand{\bI}{\mathbf{I}}
\newcommand{\cW}{\mathcal{W}}
\newcommand{\cQ}{\mathcal{Q}}
\newcommand{\bW}{\mathbf{W}}
\newcommand{\bfV}{\mathbf{V}}
\newcommand{\bD}{\mathbf{D}}
\newcommand{\cV}{\mathcal{V}}
\newcommand{\bA}{\mathbf{A}}
\newcommand{\cX}{\mathcal{X}}
\newcommand{\cE}{\mathcal{E}}
\newcommand{\cI}{\mathcal{I}}
\newcommand{\cT}{\mathcal{T}}
\newcommand{\cH}{\mathcal{H}}
\newcommand{\cC}{\mathcal{C}}
\newcommand{\cL}{\mathcal{L}}
\newcommand{\cA}{\mathcal{A}}
%%%%%% gothic fonts

%\font\twelvefrak=eufm10 at 12pt
%\font\tenfrak=eufm10
%\font\sevenfrak=eufm7
%\font\fivefrak=eufm5
%\newfam\frakfam
%\textfont\frakfam=\twelvefrak
%\scriptfont\frakfam=\sevenfrak
%\scriptscriptfont\frakfam=\fivefrak
%\def\frak{\fam\frakfam\twelvefrak}


%%%%%% Bbb fonts

%\font\twelveBbb=msbm10 at 12pt
%\font\tenBbb=msbm10
%\font\sevenBbb=msbm7
%\font\fiveBbb=msbm5
%\newfam\Bbbfam
%\textfont\Bbbfam=\twelveBbb
%\scriptfont\Bbbfam=\sevenBbb
%\scriptscriptfont\Bbbfam=\fiveBbb
%\def\Bbb{\fam\Bbbfam\twelveBbb}


%%%%%% Theorem-like environments



\def\pf{{\it Proof. }}


%%%%%% Math symbols



\def\im{\mathop{\rm Im }}
\def\ker{\mathop{\rm Ker}}
\def\dist{{\rm Dist}}
\def\haut{{\rm ht}}

\def\al{\alpha}
\def\lam{\lambda}
\def\om{\omega}
\def\vep{\varepsilon}

\def\S{{\frak S}}
\def\M{{\frak M}}
%\def\T{{\frak T}}
\def\a{{\frak a}}
\def\p{{\frak p}}
\def\g{{\frak g}}
\def\t{{\frak t}}
\def\h{{\frak h}}
%\def\f{{\frak f}}
\def\q{{\frak q}}
\def\z{{\frak z}}
\def\b{{\frak b}}
\def\n{{\frak n}}
\def\u{{\frak u}}
\def\l{{\frak l}}
\def\U{{\cal U}}
\def\cZ{{\cal Z}}
\def\w{{\bf w}}
\def\C{{\Bbb C}}
\def\m{{\frak m}}
\def\s{{\frak s}}
\def\bW{\bf W}
\def\N{{\Bbb N}}
%\def\A{{\Bbb A}}
\def\Z{{\Bbb Z}}
\def\Zp{{{\Bbb Z}_p}}
\def\Q{{\Bbb Q}}
\def\e{{\bf e}}
\def\R{{\Bbb R}}
\def\C{{\Bbb C}}
\def\F{{{\Bbb F}}}
\def\P{{{\Bbb P}}}
\def\S{{{\Bbb S}}}
\def\barX{{\overline{X}}}
\def\X{{{\Bbb X}}}
\def\G{{{\Bbb G}}}
\def\barF{{\overline{F}}}
\def\sp{\mbox{\bf spin}} 
\def\adots{\mathinner{\mkern2mu\raise1pt\hbox{.}\mkern3mu\raise4pt\hbox{.}\mkern1mu\raise7pt\hbox{.}}}

 
\newtheorem{thm}{Th\'eor\`eme}
\newtheorem{cor}{Corollaire}
\newtheorem{pro}{Proposition}
\newtheorem{lem}{Lemme}
\newtheorem{sublem}{Sous-Lemme}
\newtheorem{de}{D\'efinition}

\newtheorem{theo}[subsection]{Th\'eor\`eme}
\newtheorem{lemm}[subsection]{Lemme}
\newtheorem{coro}[subsection]{Corollaire}
\newtheorem{prop}[subsection]{Proposition}
\theoremstyle{definition}
\newtheorem{rema}[subsection]{Remarque}
\theoremstyle{definition}
\newtheorem{defi}[subsection]{D\'efinition}
\newenvironment{nota}{\textbf{Notation. }}{}
\theoremstyle{definition}
\newtheorem{para}[subsection]{}
\theoremstyle{definition}
\newtheorem{exam}[subsection]{Example}
\theoremstyle{definition}
\newtheorem{rapp}[subsection]{Rappel}

\usepackage{color}
\newcommand{\change}[1]{\textcolor{red}{#1}}


%%%%%%%%%%%%%%%%%%%%%%%%%%%%%%%%%%%%%%%%%%%%%%%%%%%%%%%%%%%%%%%%%%%%%%%%%%%%%%%%%%%%%%%%%%%%%%%%%%%%%%%%%%%%%%%%%%%%%%%%%%%%%%%%%%%%%%%%%%%%%%%%%%%%%%%%%%%%%%%%%%%
\title{Letter in support of  J. Lang's visit at U. Paris 13}
\author{J. Tilouine, Professor of Mathematics, Universit\'e Paris 13\\ 99 av. J.-B. Cl\'ement, 93430 Villetaneuse, FRANCE\\ email : \MakeLowercase{jacques.tilouine@free.fr}, phone: +33-1-49 40 40 87}
\date{September 23, 2015}
\begin{document}
\maketitle

\vskip 5mm

I met Dr. Lang several times, either during visits at UCLA, or on the occasion of various conferences, at MSRI (in Fall 2014), 
and most recently in Montreal in March 2015. Her energy and her seriousness in research struck me from the beginning. 
So, when she talked to me about a possibility of spending a couple of years in Paris in our group, I was really delighted.

Her subject of research is the study of the image of Galois representations associated to $p$-adic families of automorphic forms,
 inside their Mumford-Tate group. This study, initiated by Hida, consists in two parts: a qualitative one, whose goal is to establish 
the existence of a maximal principal congruence subgroup of the Mumford-Tate group inside the Galois image (up to suitable conjugation), 
and a quantitative one, which consists in determining the level of this congruence sugroup 
-called the Galois level, in terms of a congruence ideal (or a $p$-adic $L$ function, when these two objects are related) 
between the given family and other, more degenerate, families. 
This intuition of Hida proved remarkably accurate. He found,  in a beautiful paper (Compos. Math. 2015), 
that these two ideals are indeed closely related in the $GL(2)/\QQ$ case . The hope is that after suitable reformulation, they might be equal. 
The contribution of Dr. Lang is a key step towards this reformulation. Indeed, in her thesis, she established
 a $\Lambda$-adic version of theorems by  Ribet and Momose. They showed that the image of Galois for  a given modular form
 is open (modulo center) in an inner form of $GL(2)$ 
over an explicit subring of the Hecke ring, called the conjugate self-twist Hecke subring (note that in the ordinary case, this inner form is 
$GL(2)$ itself). She treated the case of 
Hida families for $GL(2)$. Her main result concerns the definition and properties 
of a Galois representation defined over the $\Lambda$-adic analogue of the self-twist Hecke subring,
based on deformation theory. In a work in progress, Dr. Lang found another definition of this ring 
as the trace ring of the adjoint representation,
using an old paper by R. Pink (Compact subgroups of linear algebraic groups), as suggested  by Hida.  
This reformulation allows generalizations to Hida families over larger groups. 

In fact, in a joint paper
 (to appear in the Proc. of a Conf. on Serre's Conjecture, Hausdorff Institute, Bonn 2013)
Hida and myself found generalizations of Hida's results to several variable Hida families over larger groups, but
we needed to assume that the
$\Lambda$-adic  Hecke ring coincides with the (several variable) Iwasawa algebra, because 
we didn't tackle the problem of  conjugate self twist Hecke rings. 


Moreover, in a recent preprint ArXiv with my student A. Conti and A. Iovita, we proved, in the case of $GL(2)$, 
but in the non ordinary, 
finite slope case, qualitative and quantitative results similar to Hida's. 
In this paper, instead of using directly Hida's aforementioned paper, we chose to use Dr. Lang's thesis, because of its excellent lisibility and
the great degree of generality of the statements which make them easily applicable to our situation, 
despite our different setting.

It should also be mentioned that I have another PhD student, Huan Chen, at Paris 13 who
 is studying the image of Galois representations associated to  (ordinary) Hida families for higher rank unitary groups
(constructed in Geraghty's thesis). These families must be congruent to, but not equal to, families coming  from smaller unitary groups
by automorphic induction ( as studied by Arthur-Clozel, Harris-Labesse and Labesse). He studies also
 Hida families on symplectic groups containing subfamilies of automorphic symmetric power lifts,
 as constructed by Kim-Shahidi and Clozel-Thorne. I recommended him too 
to read J. Lang's thesis, because of the clarity of her formulations and proofs.

These studies do obviously fit in the new technology provided by Dr. Lang's thesis and project. I do hope that 
concrete interaction (and even joint works) with my students will result from Dr. Lang's stay at Paris 13.
I will be present at Paris 13 in 2016-2018, most of the time.

In our group, we usually conduct two activities weekly: an open seminar and semester-long workshops about  important recent results 
(more or less democratically chosen). On these days, people have lunch together and we then have informal discussions.
But besides these two meetings, we feel that a visitor does not owe daily presence at our university
(although we'll provide her of course with office, computer and access to electronic and physical library).
So during her stay, she will be encouraged, besides,  to attend whatever seminar she likes, in Paris, Orsay or IHES, 
and to interact with whomever she wishes. It is to be noted that in our group, besides my students and myself, 
interesting interlocutors for her include 

-P. Boyer, an expert in 
the cohomology of Shimura varieties, 

-A. Mokrane, an expert in $p$-adic Hodge theory,

-B. Stroh, an expert in arithmetic and rigid geometry of Shimura varieties 
especially for applications to $p$-adic automorphic forms, 

-F. Brumley,  an expert in automorphic forms and harmonic analysis
on locally symmetric domains

Our new project with Hida deals with applications of $R=T$ theorems to the relation between Galois level and congruence ideals in the
$Sym^3\colon GL_2\to GSp(4)$ case for instance. While it won't interfere directly with Dr. Lang's current projects, it may
also help to give them some impetus. 

In conclusion, I consider that  Dr. Lang's results and research program fit very well 
with the research themes of our group at Paris 13.
 I therefore expect a very fruitful stay for her, with great benefit for both parties. 
I totally support her application to visit us in 2016-2018.

\vskip 2cm

\hfill{Jacques Tilouine}

\hfill{Professor of Mathematics}

\hfill{Universit\'e de Paris 13}

\end{document}




















