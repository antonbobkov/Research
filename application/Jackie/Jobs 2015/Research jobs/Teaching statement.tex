\documentclass[12pt]{article}
\usepackage{times}
\usepackage{amsfonts,amsmath,amsthm,amssymb,url}

\textwidth 6.5in
\oddsidemargin 0.0in
\evensidemargin 0.0in
\usepackage{fancyhdr}
\textheight 9in
\topmargin -.5in
\pagestyle{fancy}
\lhead{Teaching Statement}
\chead{Jaclyn Lang}
\rhead{November 2015}

\usepackage[OT2,T1]{fontenc}
\DeclareSymbolFont{cyrletters}{OT2}{wncyr}{m}{n}
\DeclareMathSymbol{\Sha}{\mathalpha}{cyrletters}{"58}

\usepackage{titlesec}
\titleformat{\section}
	{\centering\normalfont\bfseries}{\thesection}{1em}{}
\titleformat{\subsection}
	{\normalfont\normalfont\itshape}{\thesubsection}{1em}{}

\usepackage{color}
\newcommand{\qtn}[1]{\textcolor{red}{#1}}

\newcommand{\Aa}{\mathfrak{a}}
\newcommand{\C}{\mathcal{C}}
\newcommand{\F}{\mathbb{F}}
\newcommand{\I}{\mathbb{I}}
\newcommand{\m}{\mathfrak{m}}
\newcommand{\OK}{\ensuremath{\mathcal{O}}}
\newcommand{\Pp}{\mathfrak{P}}
\newcommand{\Q}{\mathbb{Q}}
\newcommand{\Z}{\mathbb{Z}}

\newtheorem{theorem}{Theorem}
\newtheorem{definition}{Definition}

\DeclareMathOperator{\Frob}{Frob}
\DeclareMathOperator{\Gal}{Gal}
\DeclareMathOperator{\GL}{GL}
\DeclareMathOperator{\GSp}{GSp}
\DeclareMathOperator{\MT}{MT}
\DeclareMathOperator{\Res}{Res}
\DeclareMathOperator{\SL}{SL}
\DeclareMathOperator{\Spec}{Spec}
\DeclareMathOperator{\tr}{tr}

\begin{document}


\section*{Teaching at UCLA}
I have been a Teaching Assistant (TA) for lower and upper division courses as well as a Teaching Assistant Consultant (TAC).  The latter position is awarded to experienced TAs with a strong record of past instruction.  The TAC teaches and develops the curriculum for MATH 495 -- the course instructing first year TAs in effective teaching techniques -- and participates in a quarter-long campus-wide seminar on pedagogy through the Office of Instructional Development.  In 2014 I was awarded a Distinguished Teaching Award from the UCLA Mathematics Department based on evaluations from students and faculty.  In my classroom I focus on developing problem solving, technical communication, and programming skills, while mentoring my students, particularly those underrepresented in STEM fields.

\textbf{Problem Solving.}  In teaching problem solving skills, I employ Polya's techniques from his classic book \textit{How to Solve It}.  For example, visualization can be a powerful tool in Group Theory, where the material can seem abstract but is capturing something visual (symmetries of objects).  As a TA for Group Theory, I had the students make equilateral polygons and use them to understand multiplication in dihedral groups.  I also brought in toothpicks and gumdrops so that students could make their own dodecahedra and understand their symmetries -- something students remembered months later during evaluations.  Another technique that Polya stresses is the importance of reflecting on past work.  In Group Theory, I required students to self-evaluate their homework by ranking how well they believed they answered each question.  This did not affect their grades, but it forced them to reflect on their work and learning process while helping me write ``helpful comments on every homework assignment'' (student evaluation). 

\textbf{Technical Communication.}  A central goal of my classes is for students to learn technical communication skills, both written and oral, formal and informal.  As one of my evaluations noted, ``she encourages student participation and students interacting actively with each other.''  For example,  I developed a worksheet for the first discussion section of Integration and Infinite Series that immediately catalyzes mathematical student discussions.  The worksheet (available at \url{www.math.ucla.edu/~jaclynlang/}) asks students to sketch graphs of standard functions, evaluate limits, recall trigonometry, and remember rules for derivatives, exponents, and logarithms.  Students not only answer each question, but name a person with whom they solved it.  (The same name cannot appear more than four times.)  Thus, students discuss mathematics while reviewing the prerequisite material for the course.  This activity allows me to meet the students and gauge the background of the class.  The informal communication skills that students develop through such activities is useful in any career involving technical teamwork.  I know my students learn to value communication skills because their evaluations call me a ``great communicator.''    

\textbf{Programming.}  As computers become more powerful and ubiquitous, programming skills are becoming critical in many disciplines and careers.  Math courses are a natural place to introduce students to the basics of programming, and computational problems can lead to student research.  As a Calculus TA, I used the open source software Sage to supplement the numerical analysis material in the course.  Students wrote simple programs in Sage Math Cloud that implemented numerical integration techniques and error bounds.  Sage Math Cloud is well suited for assignments and classroom activities.  Students add me as a collaborator, and I can give feedback directly on their code.  Furthermore, some of my research projects involve computations in Sage.  By bringing Sage into the classroom, I can offer interested students opportunities to deepen their knowledge by conducting research with me or working on Sage development projects.  

\section*{Teaching beyond UCLA}
My teaching is influenced by my experiences outside of UCLA.  As a student at Bryn Mawr College, I ran problem sessions for math courses from Calculus to Abstract Algebra.  I met one-on-one with students, deepening conceptual understanding rather than just doing homework problems, which naturally hones the problem solving and communication skills mentioned above.  Many interactive activities I developed at UCLA were inspired by my experience at Bryn Mawr.

In summer 2010 I was a counselor for the Program in Mathematics for Young Scientists at Boston University, a six-week program for talented high school students.  I was responsible for guiding the learning of three students.  One had a strong background in competition math, one was severely unprepared for the program, and the last was talented but struggled with mathematical confidence.  I addressed this challenge by spending hours with each individual student, away from the pressures of a group setting, to create goals that were achievable and maximized her learning.  While such time is not always possible for large classes, I keep this example in mind in my classes and work with my students to articulate and achieve appropriate individual goals.

During summer 2012 I was a TA at the Summer Program for Women in Mathematics at George Washington University.  The program was for women majoring in math from around the country entering their final year of college.  Participants took math courses beyond the usual undergraduate curriculum and went on weekly field trips to companies and agencies that employ mathematicians with advanced degrees.  I lived with the students in the dorm to help them with their courses and provide guidance on career trajectories and opportunities.  Many students were curious about my experience in graduate school.  These informal conversations led one student to write in her evaluations that I was ``one of the most helpful parts of the program.''  I have incorporated mentoring into my classroom at UCLA by encouraging students to attend departmental activities and apply for math opportunities from a comprehensive list posted on my website.  My students have noticed these efforts, noting on evaluations that ``her concern for our education really showed.''

\section*{Supporting Diversity through Teaching}
I am committed to increasing participation and achievement of underrepresented minorities in mathematics.  Twice I have partnered with other graduate students at UCLA to bring underprivileged middle school girls to UCLA for EmpowHer STEM Day, where they participate in science  and math demonstrations.  My group explained basic probability using the Monty Hall Problem.  

My advocacy for diversity in STEM led to an interest in stereotype threat research and its implications for teaching mathematics.  I have read books on the subject including \textit{Whistling Vivaldi: How Stereotypes Affect Us and What We Can Do}, by Claude Steele, and I attended a related lecture series on Women in STEM curated by UCLA social psychologist Professor Jenessa Shapiro in 2011.  As a TAC, part of my contribution to the curriculum for MATH 495 was to introduce a day when Professor Shapiro gave a lecture on stereotype threat and tips for combating its effects in the classroom.  For example, research has shown that students who believe that math skills are malleable and grow with practice outperform those who believe math ability is a fixed genetic trait.  The difference is especially dramatic for students who are part of an underrepresented group.  

By teaching Polya's problem solving techniques, emphasizing communication and group work, and explicitly stressing the malleability of mathematical ability in my classes, my students internalize both the mathematics and the confidence to succeed in future technical work.  

%In small upper division classes I would follow the model that I experienced as an undergraduate at a liberal arts college.  Namely, homework and exams would be supplemented with projects such as in-class presentations and expository papers.  Such projects provide students with more freedom to explore topics in depth than they would on a problem set.  Furthermore, they develop formal written and oral technical communication skills that will be essential in nearly every career they may undertake.
\end{document}