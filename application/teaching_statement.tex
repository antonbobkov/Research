\documentclass[11pt]{article}
\usepackage[margin=.9in]{geometry}

\usepackage{fancyhdr}
\pagestyle{fancy}

%\usepackage{mathrsfs}

\usepackage{setspace}
% \setstretch{3}
% \doublespacing

\usepackage [english]{babel}
\usepackage [autostyle, english = american]{csquotes}
\MakeOuterQuote{"}

\rhead{Anton Bobkov}

\lhead{Teaching Statement}




\begin{document}
I've spent the past 5 years at UCLA teaching an array of undergraduate courses, including first-year calculus, advanced undergraduate classes, and several programming courses.  In my instruction I emphasize back and forth engagement with the material, student feedback, and use of visual aids, with an objective focused primarily on real world applications.  

I believe that visual aid is the most helpful thing for explaining mathematical ideas, so 
I make sure to include plenty of diagrams, pictures, and graphs in my teaching.
Even something small --- like using colors or smart management of the board space --- can have a tremendous effect in making the material clearer.
When possible, I try to have a demonstration related to the discussed mathematical topics. %, so the students can see real world applications of the material and actively participate/ make the material relevant to their personal lives etc.”.
For example, I would use a deck of cards to explain sorting algorithms
or a stack of blocks to demonstrate the divergence of the harmonic series.
In recent years a lot of interactive visualization tools have become available online, and I try to incorporate them into my lectures whenever possible.
For instance, when I introduced my students to the Taylor Polynomials I used a Mathematica online demo for approximating trigonometric functions that displayed a graph of a trigonometric function, overlaid with a graph of its approximation by a Taylor Polynomial.
What made this exercise particularly fun, was that you could drag a slide to change the center of the polynomial expansion or increase the degree of the polynomial, providing a clear and intuitive demonstration of the Taylor Polynomials; a concept that is hard to replicate via the usual means of drawing those graphs on a blackboard.


A demonstration can be interactive as well. In one of the programming classes I noticed that the students were struggling with the concept of backtracking. A typical example of backtracking is an algorithm for finding an exit out of a maze using a recursive function. To remedy this, I ran the following activity with the students. One student started simulating the algorithm by reading out loud the steps printed out on a piece of paper. Every time a recursive call was made, a new student got involved and started executing the new function. I tracked the progress through the maze on the board.
We've kept going until the maze was solved.  This has provided a very effective visualization of the call stack which was usually the hardest part to understand in backtracking algorithms.
The class where I did this demo had a much easier time with the backtracking homework compared to the previous classes, illustrating the effectiveness of this activity.  
%** It's a little unclear here whether this was a one time exercise, or something you did over a long stretch of time.  Maybe give a little bit of background as to how long this went on.


Another important component of my teaching is connecting the math with its real world applications. One of my favorite things about mathematics is its widespread use in all kinds of technical fields --- physics and natural sciences, engineering, computer science, medicine, and finance.
In my experience, even very abstract ideas have useful and often surprising applications.
I always make sure to mention, for example, an application of exponential functions to caffeine level in blood, how eigenvectors are used in facial recognition software,
or how geometric series arise when describing the multiplier effect in macroeconomics.
A student noted that "he explains topics thoroughly and easily shows anecdotes and applications to make concepts more understandable".

My primary goal is to have students interacting with the material at all times.
Rather than waiting for a student to first engage with a given topic by confronting their homework,
I instead, try give out worksheets in class shortly after a topic is covered. This gives the students a chance to familiarize themselves with the material, understand their knowledge gaps, and ask questions in class, instead of struggling alone.
While the students work in groups, I go back and forth to check on their progress and help when they get stuck.
This allows me to individually interact with the students while simultaneously gauging where the class is as a whole.
With this information, I can make a better decision on whether or not the class is ready to move on to a new topic.
If not, I'll revisit existing themes that the class may be struggling with and potentially adjust my presentation method so as to help the students see the material in a new, stickier, way.  
I typically aim to schedule at least 10 minutes of group work per session, as I find it emphasizes a social aspect of learning, and allows students to connect with their peers, explain ideas and guide one another through the more difficult parts of the material.
From a review: "I really appreciated the time he took in creating group work for us each class and making sure that we understood everything".

For a quicker measure of how well the class is doing I prepare multiple choice quizzes with one or two questions which I display on a projector and ask students to text their answers. %---Can you explain this more? Text to you? To eachother?
This allows the instantaneous feedback for me while keeping the students engaged and interacting with the material.
This is a much better way to measure students' level of understanding compared to soliciting answers via raised hands, as that tends to create a positive feedback loop, employing only those students who are actively participating.
Conversely, if a student finds themselves struggling with quizzes or worksheet questions, they take it as an indication that they should put more time into the class.
Consistent interaction with the material, and working on problems of an increasing difficulty by way of quizzes, worksheets, homework, and exams allows each student to build their confidence, while gaining a deeper understanding of the topics covered.

I make it a habit to reflect on my teaching practices and look for ways to improve.
This includes asking the students attending office hours what they think of the class so far and what they would like to see more of.
I also ask my classes to fill out an anonymous survey at several points throughout the quarter where they can voice criticisms and suggestions.
I do my best to respond and adapt my practices according to this feedback --- it has greatly helped me to form my teaching style.
A student noted: "I really appreciate how he asked for feedback and then responded and improved".



At UCLA, I have mentored a number of students doing independent study.
This has involved a wide variety of projects. Some only took a quarter and some were more serious year-long commitments.
The projects' topics included pure and applied math, data analysis, computer science, and finance. All the work had a strong programming component, so the students always had a functional program to show at the end.


My focus as a mentor was to balance students' independent exploration of the material, versus providing a structure with concrete goals and expectations. While I do let students learn different aspects of the material, experiment with them, and see what direction they want to take the project, the complexity and the range of the material can make it easy for the student to feel overwhelmed and discouraged. To counter this, I assign weekly goals. This way the student will feel accomplished having a concrete progress every week.  I also make sure to talk about the long term direction of the project every couple of weeks.


An example of this is a project that involved digit recognition using neural networks coded in Python. The student didn't have any prior experience in this area, so we have started slowly by following the tutorials online. For the first couple of weeks I would instruct to get a certain tutorial working, make minor changes to it, and research what other algorithms there are.
As we have progressed, the student became more familiar with the language and the code involved, gaining the confidence to work on more advanced topics. Originally we planned to adapt the algorithm to make it recognize Chinese characters. However, once we got through the basics, I spoke to the student and we determined that it would be more interesting to try to adapt the symbol recognition algorithm to analyzing sound wave data. That was the direction we took instead, focusing on classifying music from simple MIDI files. We were very flexible in changing end goals and it has allowed the project to evolve naturally in accord with the student's interest and comfort level.
This ensured the student's enthusiastic involvement and we have ended up with a more interesting project than the one we had planned in the beginning.


By emphasizing effective interaction with the material, I foster a connection to math that students carry beyond class. They come out as independent thinkers, confident with the knowledge acquired, and able to comfortably apply it in their future studies. A student wrote: "With you as my professor I felt like you really cared and showed interest in me which was huge.
Knowing I had someone who believed in me really made me want to dig deep and put my full effort into the class." I am very proud of a review like this and strive to keep discovering new ways of communicating the material and further refining my teaching techniques.


\end{document}

