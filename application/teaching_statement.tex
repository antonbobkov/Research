\documentclass[11pt]{article}
\usepackage[margin=1in]{geometry}

\usepackage{fancyhdr}
\pagestyle{fancy}

%\usepackage{mathrsfs}

\usepackage{setspace}
%\doublespacing


\rhead{Anton Bobkov}

\lhead{Teaching Statement}




\begin{document}

As UCLA student I had a wide variety of teaching experience, 
having taught in first-year calculus classes, advanced undergraduate classes, programming courses
as well as mentoring independent projects and reading courses for undergraduate students.
I was a recipient of UCLA's 2016 Departmental Teaching Award.
I have also engaged in private tutoring as well as volunteering to assist math instructors in local high schools.

When I teach, my primary goal is to have students interacting with the material as soon as possible.
Instead of waiting for them to first engage with the material only when they start their homework,
I give out worksheets as soon as a given topic as covered.
This way they get familiar with the basics faster which allows me to move on to harder topics and
give more abstract explanations sooner.
Having a class work on an assignment right away also allows me to gauge where all the students are at in their understanding of the material.
I encourage group work when possible.
For a quicker measure of students' level of understanding of the material I prepare one or two question multiple choice quizzes which I display on a projector and ask students to text message their answer.
This allows instanteneous feedback for me while keeping the students engaged and interacting with the material.
If most people get the question wrong, then it is a sign for me to slow down and go over the material again.
If majority of students answer correctly, then I can confidently move on to the next topic.
This is a much better way to measure students' level of understanding of the material compared to soliciting answers via raised hands as that usually only covers only those students who are actively participating.
This also provides constant feedback back to the students - if a student finds themselves struggling with quizzed or worksheet questions, they take it as an indication that they should put more time into the class.
Normally, the students only get such feedback through the homeworks, which could be a misleading indicator for the format and difficulty of the exams.
This empahsizes a social aspect of learning and allows students to interact with their peers to explain ideas and help each other to work trough harder parts of the material.

I believe that mathematics is best explained with the use of visual aid.
I make sure to include plenty of diagrams, pictures, and graphs in my teaching.
Even something small - like using colors or managing board space in a good way - can have a tremendous effect in making the material clearer.
When possible I try to demonstrate common mathematical or physical concepts using props.
For example, as a part of explanation of sorting algorithms I use a deck of cards,
or when explaining hyperbolic function I use a hanging rope as an illustration of catenoid.
This helps students to visualize explained concepts as well as creating a link to the real world application of the presented abstract concepts.
With the advance of the technology there are a lot of interactive visualization tools available online.
I try to incorporate them into my lectures when possible.
For example, when I was introducing students to Taylor Polynomilas I have used a Mathematica online demo for approximating trigonometric functions.
It would display a graph of a trigonometric function overlayed with a graph of its approximation using a Taylor Polynomial.
You can drag a slide to change the center of the approximation as well as change degree of the polynimial.
I felt like this provided clear and intuitive demonstration of Taylor polynomials, one that is hard to replicate via usual means of drawing those graphs on blackboard.

Another important component of my teaching is connecting the math explained with its real world applications.
A nice thing about mathematics is its widespread use in all kinds of technical fields - physics and natural sciences, engineering, computer science, medicine, finance, economics as well as statistics being very useful in social sciences.
In my experience even very abstract ideas often have useful and sometimes surprising applications.
It is my goal to always keep students conncected with the applications of the material they study.
I always make sure to mentinon, for example, application of exponential function to caffeine level in blood, connection of eigenvectors to facial recognition, the use of abstract groups in describing atom lattices, improper integrals for computation of terminal velocity needed to leave Earth's orbit, or how geometric series arise to describe multiplier effect in macroeconomics.
I believe that it provides a better motivation for studying the meaterial that sometimes can get a bit abstract.

When teaching I strive to get as much student feedback as possible.
For example, I always ask students attending office hours what they think of the direction the class is going and what they would like to see more.
I also ask them to fill out an anonymous survey at several points throughout the class where they can voice criticisms and suggestions.
I do my best to respond and adapt my teaching style according to this feedback - it has greatly helped me to form my teaching style.

When working with students one on one I try to gauge how comfortable with the material a student is and work from there.
Instead of just explaning a given concept or a problem, I try to engage the student right away.
This involves asking a lot of questions or ideally having the student work on a problem themselves
and guide them through the process.
When I was an instructor for a calculus course, I have worked with a student who was very uncomfortable with the material going into the class.
I have spent a lot of office hours with the student.
I would start with simple problems and once I saw that 

At UCLA, I worked as a mentor for students doing independent study.
This has encompassed a wide variety of projects.
I have worked with students both informally and formally for university credit,
over short periods like one quarter as well as on more serious year long projects.
The projects' topics included pure and applied math, data analysis, computer science, and finance.
All the projects had a strong programming component, so students always had a working program to show at the end.

My main focus as a mentor was to balance encouraging students to explore on their own versus providing a structure with concrete goals and expectations.
On one hand I would let students learn different aspects of the material, experiment with it, and see what direction they want to take their project.
Given complexity and range of the material, however, it is easy for the students to feel overwhelmed.
To counter that I would introduce concrete weekly goals and would often check in to talk about long term direction of the project.
This way even if the material is dense and frustrating to get through, the student would always feel accomplished having a concrete weekly progress as well as a feel for project's overall progress.

For example, one of the projects was doing image recognition of digits using neural networks.
The student didn't have any prior experience in this area, so we started really slowly by following tutorials online.
I would instruct to get a tutorial working, make minor changes to it, and research what other algorithms there are.
As we progressed, the student became more familiar with the language and the code involved, gaining confidence to work on more advanced topics.
Originally we have planned to write an algorithm recognizing chineese characters,
but as we got through the basics, I have talked to the student and we have determined that it would be more interesting to try to adapt symbol recognition algorithm to recognizing sounds
I value this flexibility a lot - it allows the project to evolve naturally in accord with the student's interest and their comfort with the material studied.

	
\end{document}

% Include edges between y as a chain minimal extension
