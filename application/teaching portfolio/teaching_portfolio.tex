%%%%%%%%%%%%%%%%%%%%%%%%%%%%%%%%%%%%%%%%%
% Medium Length Graduate Curriculum Vitae
% LaTeX Template
% Version 1.1 (9/12/12)
%
% This template has been downloaded from:
% http://www.LaTeXTemplates.com
% 
% Original author:
% Rensselaer Polytechnic Institute (http://www.rpi.edu/dept/arc/training/latex/resumes/)
%
% Important note:
% This template requires the res.cls file to be in the same directory as the
% .tex file. The res.cls file provides the resume style used for structuring the
% document.
%
%%%%%%%%%%%%%%%%%%%%%%%%%%%%%%%%%%%%%%%%%

%----------------------------------------------------------------------------------------
%	PACKAGES AND OTHER DOCUMENT CONFIGURATIONS
%----------------------------------------------------------------------------------------

\documentclass[margin, 10pt]{res} % Use the res.cls style, the font size can be changed to 11pt or 12pt here
\usepackage[T1]{fontenc}
\usepackage{hyperref}
\usepackage{verbatim}
\usepackage{enumerate}
\usepackage{enumitem}
\usepackage{amssymb}
\usepackage{helvet} % Default font is the helvetica postscript font
%\usepackage{newcent} % To change the default font to the new century schoolbook postscript font uncomment this line and comment the one above
\usepackage[final]{pdfpages}

%\usepackage{graphicx}

%\setlength{\textwidth}{5.1in} % Text width of the document
\setlength{\textwidth}{5.5in} % Text width of the document

\hypersetup{
     colorlinks   = true,
     urlcolor  = blue
}

\begin{document}

%----------------------------------------------------------------------------------------
%	NAME AND ADDRESS SECTION
%----------------------------------------------------------------------------------------

\moveleft.5\hoffset\centerline	{\Large \textsc{teaching portfolio}}

\phantom{hello}

\moveleft.5\hoffset\centerline{\large\bf Anton Bobkov}

\moveleft\hoffset\vbox{\hrule width\resumewidth height 1pt}\smallskip % Horizontal line after name; adjust line thickness by changing the '1pt'

\section{\textsc{Contact Information}}

\begin{tabular}{l|l}
Anton Bobkov, Graduate Student & {\it E-mail:} \href{mailto:bobkov@ucla.edu}{bobkov@ucla.edu}\\
UCLA Mathematics Department & {\it Phone:} (408) 813-6331 \\
Box 951555 \phantom{lots of text that takes up a bit of space} & \\
Los Angeles, CA 90095-1555 USA & 
\end{tabular}


%----------------------------------------------------------------------------------------

\begin{resume}


\section{\textsc{teaching philosophy}}
My priority as a teacher is to let the students develop a deep connection to the material that they can carry outside of the classroom.
I achieve this by emphasizing interaction and effective communication.
I make sure to incorporate demonstrations and interactive activities in my classroom when possible.
I also rely heavily on visual aid, online tools and use of technology for explaining mathematical ideas.
I relate the math to its real world applications to underline its utility and present it as a part of a bigger picture.
When working with students individually, I encourage independent work and exploration while providing concrete tasks and goals.

\section{\textsc{honors and awards}}
\textit{2016 Departmental Teaching Award} \\
Award given to Teaching Assistants in the mathematics department with excellent teaching records.

\section{\textsc{teaching experience}}
Instructor \hfill \textbf{2016 - 2017} 
  \begin{itemize}
  \item Math 31B: Integration and Infinite Series 
  \item Math 32BH: Calculus of Several Variables, Honors (Syllabus included this portfolio)
  \end{itemize}
Teacher Assistant \hfill \textbf{2012 - 2017} 
  \begin{itemize}
	  \item Math 31B: Integration and Infinite Series (Evaluations included in this porfolio)
  \item Math 33A: Linear Algebra and Applications (Evaluations included in this porfolio)
  \item Math 115B: Linear Algebra (upper division)
  \item Math 174E: Mathematics of Finance 
  \item PIC 10B: Intermediate Programming (Evaluations included in this porfolio)
  \item PIC 20A: Principles of Java Language with Applications
  \item PIC 40A: Introduction to Programming for Internet
  \end{itemize}

  I am also prepared to teach:
  \begin{itemize}
  \item Undergraduate math classes including: precalculus, single/multi-variable calculus, analysis, abstract algebra, topology, differential geometry, probability.
  \item Statistics, programming/computer science, finance.
  \end{itemize}

\section{\textsc{Undergraduate Projects}}
  \phantom{Mentor} \hfill \textbf{2016 - 2017}  
  \begin{itemize}
  \item Conway's game of life variations with C++/SDL graphics library %\hfill \textbf{Winter 2016} 
  \item App development platform with Typescript 2 %\hfill \textbf{Winter 2016} 
  % \item Internet browsing data/trends visualization with Python %\hfill \textbf{2016} 
  \item Optical character recognition via neural nets with Python %\hfill \textbf{2016} 
  \item Discrete signal processing with Matlab %\hfill \textbf{Fall 2017} 
  \end{itemize}

\section{\textsc{diversity statement}}
UCLA has a large and diverse undergraduate student body. As a teacher assistant and instructor I have interacted with many students, many from underrepresented groups. I make sure that all the students feel welcome in my class regardless of their background or their skill level. If a student is behind, I make sure to work more closely with them, spend more time in office hours, and encourage the student to continue with the course. This way I try to foster an atmosphere where all the students are motivated to realize their full potential and succeed.

Group work is another technique that I have found useful. Often a minority students feel isolated in their class, so working with peers helps them feel that they belong and are on an equal footing with everyone else. Semi-random group assignments also help to break up established social cliques and foster students' interaction outside their normal social comfort zone, ensuring better integration.

I have volunteered to work with math students in a local Los Angeles school from a predominantly Latina district where I was helping the students to catch up on their basic math skills. On the other end of the academic spectrum, I have attended the MSRI workshop "Connections for Women: Model Theory and Its Interactions with Number Theory and Arithmetic Geometry". The workshop was addressing the implicit bias and challenges female researchers face in STEM fields, as well as discussing ways to address that and increase representation of minority groups in academia.

As I go forward I hope to further work on outreach and increasing the diversity in STEM fields.


\section{\textsc{also included}}
\textit{Student Evaluations}
  \begin{itemize}
  \item Math 33A: Linear Algebra and Applications %\hfill \textbf{2012 - 2013} 
  \item PIC 10B: Intermediate Programming %\hfill \textbf{2015} 
  \item Math 31B: Integration and Infinite Series %\hfill \textbf{2012 - 2013} 
  \end{itemize}
\end{resume}

\textit{Practice Problems} \\
A set of practice problems for the first midterm in Math 31B: Integration and Infinite Series.

\textit{Course Syllabus} \\
A course syllabus for Math 32BH: Calculus of Several Variables (Honors)

\textit{Worksheet} \\
In-class assignment for PIC 10B: Intermediate Programming.

\textit{Research Project Summary} \\
Final report summarizing an independent project involving Conway's game of life variations.


\setboolean{@twoside}{false}
\includepdf[pages=-, offset=-80 0]{33a.pdf}
\includepdf[pages=-, offset=-80 0]{10b.pdf}
\includepdf[pages=-, offset=-80 0]{31b.pdf}
\includepdf[pages=-, offset=-80 0]{review.pdf}
\includepdf[pages=-, offset=-80 0]{syllabus.pdf}
\includepdf[pages=-, offset=-80 0]{programming.pdf}
\includepdf[pages=-, offset=-80 0]{independent.pdf}
\end{document}
