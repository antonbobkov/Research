UCLA has a large and diverse undergraduate student body. As a teacher assistant and instructor I have interacted with many students, many from underrepresented groups. I make sure that all the students feel welcome in my class regardless of their background or their skill level. If a student is behind, I make sure to work more closely with them, spend more time in office hours, and encourage the student to continue with the course. This way I try to foster an atmosphere where all the students are motivated to realize their full potential and succeed.

Group work is another technique that I have found useful. Often a minority students feel isolated in their class, so working with peers helps them feel that they belong and are on an equal footing with everyone else. Semi-random group assignments also help to break up established social cliques and foster students' interaction outside their normal social comfort zone, ensuring better integration.

I have volunteered to work with math students in a local Los Angeles school from a predominantly Latina district where I was helping the students to catch up on their basic math skills. On the other end of the academic spectrum, I have attended the MSRI workshop "Connections for Women: Model Theory and Its Interactions with Number Theory and Arithmetic Geometry". The workshop was addressing the implicit bias and challenges female researchers face in STEM fields, as well as discussing ways to address that and increase representation of minority groups in academia.

As I go forward I hope to further work on outreach and increasing the diversity in STEM fields.
